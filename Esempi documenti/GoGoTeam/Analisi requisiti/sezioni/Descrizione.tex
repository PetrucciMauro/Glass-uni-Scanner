\section{Descrizione generale}
	\subsection{Contesto d'uso del prodotto}
		\subsubsection{Processi produttivi e modalità d'uso}
		Il progetto \textbf{MyTalk} intende fornire un servizio di comunicazione real-time tra utenti.\\
		La comunicazione deve avvenire unicamente tramite browser$_{|g|}$ in modalità peer-to-peer$_{|g|}$, deve necessitare 
		dell'impiego di un server$_{|g|}$ esclusivamente per stabilire il canale di comunicazione tra gli utenti e l'intera applicazione deve essere contenuta in un'unica pagina web$_{|g|}$.\\
		Il servizio deve permettere ad utenti registrati di poterne chiamare altri conoscendone o l'indirizzo IP$_{|g|}$ o lo username di registrazione o scegliendo l'utente dalla lista degli utenti registrati presso il server$_{|g|}$.\\
		Il prodotto finito avrà lo scopo di abbassare il TCO$_{|g|}$ (Total Cost Ownership) per l'azienda che decidesse di adottarlo internamente, non necessitando di alcuna procedura di installazione se non quella del server$_{|g|}$ e del browser$_{|g|}$ Chrome$_{|g|}$ in ogni terminale.\\
		Il prodotto deve poter essere fruito anche da utenti non aziendali qualora il server$_{|g|}$ sia accessibile dall'esterno e l'azienda committente volesse rendere il servizio fruibile ad utenti esterni.

		\subsubsection{Ambiente di utilizzo}
			Per il corretto funzionamento del servizio è necessaria la corretta installazione delle librerie WebRTC$_{|g|}$ normalmente contenute di default nel browser$_{|g|}$ Chrome$_{|g|}$ versione 21 e successive.\\
			Si assicura inoltre che l'applicativo, essendo eseguito tramite browser$_{|g|}$, non dipenda in alcun modo dalla 
			piattaforma sottostante per quanto concerne il suo funzionamento di base. Tale uniformità non è però garantita per 
			quanto riguarda alcuni aspetti grafici (forma dei bottoni, bordi delle finestre, etc.) che dipendono dal sistema operativo 
			sottostante.\\
			Il prodotto non dovrà inoltre necessitare dell'installazione di alcun plugin$_{|g|}$ per svolgere le sue funzioni di base.\\
			Il prodotto necessiterà di una componente server$_{|g|}$ che avrà lo scopo di stabilire il canale di comunicazione tra gli utenti.
	\subsection{Funzioni del prodotto}
		Il prodotto sarà contenuto in un'unica pagina web$_{|g|}$ che permetterà l'instaurazione della comunicazione. Il sistema dovrà permettere di:
		\begin{itemize}
			\item poter effettuare chiamate audio/video conoscendo anche solo l'indirizzo IP$_{|g|}$ del ricevente;
			\item poter chiamare un altro utente conoscendone lo username con cui si è registrato al servizio;
			\item poter chiamare scegliendo il ricevente dalla lista di utenti registrati al server;
			\item poter ricevere chiamate da altri utenti registrati;
			\item potersi registrare presso un server$_{|g|}$ e contattare altri utenti registrati;
			\item poter modificare le proprie impostazioni di base (username, password, nome, cognome, azienda e numero di telefono);
			\item poter visualizzare le seguenti statistiche sulla chiamata una volta chiusa:
			\begin{itemize}
				\item durata della chiamata;
				\item latenza$_{|g|}$;
				\item velocità di comunicazione;
				\item quantità di byte.
			\end{itemize}
		\end{itemize}

		Il prodotto fornirà anche una interfaccia per l'amministratore che permetterà di monitorare le statistiche e le opinioni sul servizio inviate dai vari client$_{|g|}$.

	\subsection{Caratteristiche degli utenti}
		Il programma deve essere rivolto ad una tipologia di utenti eterogenea; non deve essere quindi necessaria alcuna conoscenza 
		preliminare in campo informatico se non quella derivante dalla capacità di chiamare 
		inserendo l'indirizzo IP$_{|g|}$ del 
		destinatario.\\
		L'applicativo si dovrà adattare sia ad esigenze aziendali che personali.\\
		Si prevedono due soli tipi di utenti che fruiranno del servizio:
		\begin{itemize}
		\item utente autenticato;
		\item amministratore autenticato.
		\end{itemize}
		L'utente autenticato è colui che si è registrato al servizio e che può quindi effettuare chiamate audio/video verso altri utenti registrati.\\
		L'amministratore autenticato è colui che può visualizzare le statistiche sulle chiamate effettuate dai vari utenti autenticati, al fine di monitorare l'applicazione una volta che è in esecuzione.
		\newpage
	\subsection{Vincoli generali}
		Dal capitolato \textbf{MyTalk}, \textit{software}$_{|g|}$ \textit{di comunicazione tra utenti senza requisiti  di installazione}, si possono 
		individuare i seguenti vincoli generali:
		\begin{itemize}
			\item connessione ad una rete web$_{|g|}$ o LAN$_{|g|}$ aziendale;
			\item utilizzo del linguaggio HTML5$_{|g|}$;
			\item utilizzo delle librerie WebRTC$_{|g|}$ per la gestione della comunicazione;
			\item utilizzo del linguaggio JavaScript$_{|g|}$ per la parte di programmazione lato client$_{|g|}$;
			\item utilizzo di una componente server$_{|g|}$ scritta in Java$_{|g|}$ per l'instaurazione del canale di comunicazione;
			\item utilizzo del protocollo di comunicazione WebSocket$_{|g|}$ per il passaggio da client$_{|g|}$ a server$_{|g|}$ 
			dei parametri necessari ad instaurare il canale di comunicazione;
			\item utilizzo del browser$_{|g|}$ per la gestione del servizio e dell'utilizzo dell'applicazione.
		\end{itemize}
\newpage
