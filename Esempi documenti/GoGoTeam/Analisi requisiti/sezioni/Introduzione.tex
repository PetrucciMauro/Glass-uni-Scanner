\section{Introduzione} {
	\subsection{Scopo del documento}
		Il presente documento si prefigge lo scopo di documentare i requisiti identificati dal proponente \emph{\Zucchetti} per il capitolato \textbf{MyTalk}.
	\subsection{Scopo del prodotto}
		Il prodotto denominato \textbf{MyTalk} si propone di fornire un sistema di comunicazione audio/video tra utenti tramite 
		l'utilizzo del browser$_{|g|}$ Chrome$_{|g|}$, senza l'installazione di componenti software$_{|g|}$ aggiuntive.\\
		Il sistema dovrà poter essere utilizzato da una tipologia eterogenea di utenti.
	\subsection{Glossario}
		Al fine di migliorare la comprensione del lettore ed evitare ambiguità rispetto ai termini tecnici utilizzati nel documento, viene allegato il file
		\emph{\Glossario}, nel quale vengono descritti i termini contrassegnati dal simbolo $_{|g|}$ alla fine della parola.
		Per i termini composti da pi\`u parole, oltre al simbolo $_{|g|}$, \`e presente anche la sottolineatura. 
	\subsection{Riferimenti}
	\subsubsection{Normativi}
		\begin{itemize}
			\item Capitolato d'appalto: \textbf{MyTalk}, \textit{software}$_{|g|}$ \textit{di comunicazione tra utenti senza requisiti  di installazione}, rilasciato da \textit{Zucchetti S.p.A.}, reperibile all'indirizzo \url{http://www.math.unipd.it/~tullio/IS-1/2012/Progetto/C1.pdf}.
			\item Norme di progetto (allegato $\NormeDiProgetto$).
		\end{itemize}
	\subsubsection{Informativi}
		\begin{itemize}
			\item Materiale del corso di Ingegneria del Software 2012-2013 - Dispense sui diagrammi dei casi d'uso - Prof. Tullio Vardanega e Riccardo Cardin
			      (\url {http://www.math.unipd.it/~tullio/IS-1/2012/Dispense/L06.pdf}).
			\item Regolamento: Documenti -- Prof. Tullio Vardanega (\url {http://www.math.unipd.it/~tullio/IS-1/2012/Progetto/PD01c.html}).
			\item UML distilled: guida rapida al linguaggio di modellazione standard - Capitolo 9 - Martin Fowler - 4$^{th}$ Edizione (2010).
			\item Dall'idea al codice con UML 2 - Capitolo 3, Paragrafo 1.2 - Luciano Baresi, Luigi Lavazza, Massimiliano Pianciamore - 1$^{th}$ Edizione.
		\end{itemize}
}