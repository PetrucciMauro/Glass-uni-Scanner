\appendix

\addappheadtotoc
\section{Glossario}\label{appendice}

\hfill\Huge{\textbf{B}}\\
\normalsize
	\begin{longtable}{p{0.25\textwidth} p{0.75\textwidth}} 
	    \toprule
	    \\
		    \textbf{Browser}:		&	programma che fornisce uno strumento per navigare in Internet e interagire con i contenuti presenti nel World Wide Web. Più tecnicamente, un browser è 
						un’applicazione client$_{|g|}$ che utilizza il protocollo HTTP$_{|g|}$ per inoltrare le richieste dell’utente ad un \underline{web server}$_{|g|}$.\\
	    \\
	    \textbf{Bug}:		&	errore o guasto di programmazione che porta al malfunzionamento di un programma. La causa della maggior parte del numero di bug è spesso il
						codice sorgente scritto dal programmatore, ma può anche accadere che sia il compilatore stesso a produrne. Un programma che contiene un gran numero
						di bug che interferiscono con la sua funzionalità è detto bacato.\\
	\end{longtable}

\hfill\Huge{\textbf{C}}\\	
\normalsize
	\begin{longtable}{p{0.25\textwidth} p{0.75\textwidth}} 
	    \toprule
	    \\
	  \textbf{Chrome}: &	browser$_{|g|}$ sviluppato da Google, basato su \underline{motore di rendering}\g~ß WebKit$_{|g|}$.\\
	
	\end{longtable}
	
\hfill\Huge{\textbf{F}}\\	
\normalsize
	\begin{longtable}{p{0.25\textwidth} p{0.75\textwidth}} 
	    \toprule
	    \\	
	\textbf{Firewall}: &	
	componente passivo di difesa perimetrale che può anche svolgere funzioni di collegamento tra due o più tronconi di rete.\\
	\\
	\textbf{Form}: &
	
	interfaccia di un'applicazione che consente all'utente di inviare uno o più dati liberamente inseriti da egli stesso.\\
	\end{longtable}
	
\hfill\Huge{\textbf{J}}\\	
\normalsize
	\begin{longtable}{p{0.25\textwidth} p{0.75\textwidth}} 
	    \toprule
	    \\	
  \textbf{JavaScript}:	&	linguaggio di scripting (cioè un linguaggio interpretato, dal browser$_{|g|}$ in questo caso, il cui scopo è l’interazione con altri programmi più complessi) orientato agli oggetti, usato comunemente nei siti web$_{|g|}$.\\	
\end{longtable}	
	
	\hfill\Huge{\textbf{L}}\\	
\normalsize
	\begin{longtable}{p{0.25\textwidth} p{0.75\textwidth}} 
	    \toprule
	    \\
		    \textbf{Layout}:		&	descrive la disposizione appropriata degli elementi che compongono una pagina web$_{|g|}$.\\
	\end{longtable}	
	
	\hfill\Huge{\textbf{S}}\\	
\normalsize
	\begin{longtable}{p{0.25\textwidth} p{0.75\textwidth}} 
	    \toprule
	    \\
	    \textbf{Server}:			&	componente che fornisce, a livello logico o fisico, un qualunque tipo di servizio ad altre componenti denominate client$_{|g|}$ attraverso una rete di computer o direttamente in locale. Rappresenta 
							dunque il nodo opposto al client$_{|g|}$ in una rete.\\
	\end{longtable}	
	\hfill\Huge{\textbf{W}}\\	
\normalsize
	\begin{longtable}{p{0.25\textwidth} p{0.75\textwidth}} 
	    \toprule
	    \\
	    	    \textbf{Web}:&	spesso è un’abbreviazione per World Wide Web (ragnatela mondiale). \`E un servizio Internet che permette di navigare ed usufruire dell’ampia gamma di contenuti presenti in rete.\\
	    \\
	\textbf{Web-based}: & 	
	un'applicazione web-based è un software accessibile via web per mezzo di un network.
	\end{longtable}

