\section{Installazione}{

\subsection{Fasi preliminari}{
\begin{enumerate}
	\item Aprire il file \path{/source/MyTalk/WEB-INF/classes/mytalk/model/server/dao/DBAccess.xml};
	\item Inserire le credenziali di accesso, nome utente e password, per accedere al DBMS MySQL;
	\item Aprire un terminale;
	\item Posizionarsi nella cartella \path{/source/MyTalk/DBMS};
	\item Accedere al DBMS MySQL con le proprie credenziali;
	\item Digitare source \texttt{make.sql} per la creazione del database e relative tabelle.
\end{enumerate}

}

\subsection{Installazione sistemi dove non presente un server Tomcat 7.x}{
La versione di Apache Tomcat fornita mantiene i parametri di default di rilascio. L'utente amministratore per la gestione del server è accessibile tramite le seguenti credenziali:
\begin{itemize}
	\item Utente amministratore: \texttt{tomcat}
	\item Password amministratore: \texttt{psw}
\end{itemize}

	\subsubsection{Installazione sistemi operativi GNU/Linux ed Apple OS X}{
		\begin{enumerate}
			\item Copiare la cartella \path{/source/MyTalk} nella cartella \path{/source/apache-tomcat/apache-tomcat-7.0.37-UNIX/webapps};
			\item Aprire il terminale;
			\item Posizionarsi nella cartella \path{/source/apache-tomcat/apache-tomcat-7.0.37-UNIX/bin};
			\item Digitare il comando \texttt{./startup.sh}.
		\end{enumerate}
	}
	
	\subsubsection{Installazione sistemi Microsoft Windows}{
		\begin{enumerate}
			\item Copiare la cartella \path{/source/MyTalk nella cartella /source/apache-tomcat/apache-tomcat-7.0.37-WIN{architettura}/webapps} dove \{architettura\} può essere 32 oppure 64 a seconda della versione del sistema operativo;
			\item Aprire il prompt dei programmi;
			\item Posizionarsi nella cartella\\ \path{/source/apache-tomcat/apache-tomcat-7.0.37-WIN{architettura}/bin};
			\item Digitare il comando \texttt{startup.bat}.
		\end{enumerate}
	}

}

\subsection{Installazione sistemi dove già presente un server Tomcat 7.x}{
	\begin{enumerate}
		\item Copiare la cartella \path{/source/MyTalk/} nella cartella \texttt{\{cartella di tomcat\}/webapps};
		\item {Se il server Tomcat è già attivo:
			\begin{enumerate}
				\item Aprire un browser web;
				\item accedere alla pagina\\
				\texttt{http://\{indirizzo server tomcat\}:\{porta di comunicazione\}/webadmin}
				\item Aggiornare la lista delle applicazioni;
				\item Avviare l'applicazione \path{/MyTalk}
			\end{enumerate}
			}
		\item Se il server non è attivo, avviare Tomcat mediante il comando di startup predefinito.
		\end{enumerate}
}

}