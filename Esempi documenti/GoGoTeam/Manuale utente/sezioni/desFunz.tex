\section{Descrizione funzionale}\label{desFunz}{
{
Nel presente documento sono specificate le seguenti funzionalità offerte dal sistema:
\begin{itemize}
	\item Registrazione, vedi cap. \ref{registrazione};
	\item Autenticazione, vedi cap. \ref{utNoAutent};
	\item Comunicazione, vedi cap. \ref{comunicazione};
	\item Gestione dei dati, vedi cap. \ref{gestDati}.
\end{itemize}
}

\subsection{Primo accesso all’applicazione}{
\begin{figure}[h!]
	\centering
		\includegraphics[scale=0.32]{\docsImg Autenticazione.png}
		\caption{Interfaccia dell'applicazione in fase di login.} 
		\label{fig:imgLogin}
	\end{figure}
Al primo accesso all'applicazione l'utente dovrà effettuare la registrazione, solo dopo essersi registrato potrà usufruire del servizio. Attenzione: se l'utente non effettuerà la registrazione non potrà utilizzare l'applicazione in alcun modo.\\
Dalla pagina iniziale dell’applicazione sarà quindi necessario accedere alla pagina di registrazione attraverso il link \texttt{Registrati}.	

%L’utente che desidera usufruire dell’applicazione dovrà essere registrato all’interno del database. Al primo accesso è obbligatorio effettuare la registrazione. \\
%Dalla pagina iniziale dell’applicazione sarà quindi necessario accedere alla pagina di registrazione attraverso il link \texttt{Registrati}.
	}
\newpage	
\subsection{Registrazione}\label{registrazione}{

\begin{figure}[h!]
	\centering
		\includegraphics[scale=0.55]{\docsImg Registrazione.png}
		\caption{Interfaccia dell'applicazione in fase di registrazione.}
		\label{fig:imgReg} 
	\end{figure}

L’utente che desidera registrarsi deve compilare almeno i campi obbligatori della form inserendo le proprie credenziali. Dopo aver effettuato l’inserimento dovrà confermare la registrazione premendo il bottone \texttt{Registrati}.\\ %Nel caso in cui le credenziali inserite contengano errori questi verranno opportunamente segnalati all’utente, altrimenti l’utente verrà automaticamente registrato e riportato alla pagina di Autenticazione.
Nel caso in cui l’utente sia già registrato è possibile tornare alla pagina di autenticazione premendo il link \texttt{Login}.
}

\subsection{Consentire la trasmissione}{
Prima di effettuare l'autenticazione l’utente dovrà consentire al browser l’accesso all’hardware del PC (webcam e microfono). Nella figura \ref{fig:imgLogin} è presente una barra grigia nella parte superiore, l'utente dovrà premere il bottone \texttt{Consenti}, altrimenti non sarà possibile usufruire del servizio di comunicazione.
}

\subsection{Utente non autenticato}\label{utNoAutent}{
Nella figura \ref{fig:imgLogin} viene mostrata l'interfaccia che l'utente non autenticato incontra accedendo al servizio \textbf{\mytalk}.
La pagina di Autenticazione si presenta come pagina iniziale del sito.\\
Le uniche operazioni che l'utente non autenticato può compiere sono la registrazione e il login al servizio, per poter usufruire delle altre funzionalità dovrà autenticarsi.
Se non si è ancora Registrati è possibile premere il link \texttt{Registrati}. Altrimenti, inserendo le proprie credenziali nella form e premendo il bottone \texttt{Accedi}, è possibile accedere al servizio.
}

\newpage

\subsection{Comunicazione}\label{comunicazione}{
\begin{figure}[h!]
	\centering
		\includegraphics[scale=0.32]{\docsImg ComunicazioneApertura.png}
		\caption{Interfaccia dell'applicazione in fase di prima apertura pagina Comunicazione.} 
		\label{fig:imgComAp}
	\end{figure}

La pagina di Comunicazione si presenta come pagina principale dell’applicazione. In tale pagina è possibile ricercare utenti ed effettuare videochiamate.
All’interno della pagina sono presenti due link che permettono di muoversi all’interno dell’applicazione:
\begin{itemize}
	\item Gestisci dati utente: permette all’utente di accedere alla pagina di gestione dei propri dati;
	\item Logout: permette all’utente di uscire dall’applicazione.
\end{itemize}
}

\subsubsection{Prima di ricevere o effettuare chiamate}{
Prima di poter ricevere o effettuare chiamate è necessario attivare il video locale attraverso il bottone \texttt{Attiva Video Locale}, situato sotto i due riquadri presenti nella parte sinistra della schermata (in cui verranno visualizzate la propria immagine e quella del destinatario), che si vede in figura \ref{fig:imgComAp}. In assenza di tale procedura l’utente autenticato non potrà accedere alle funzioni di videochiamata del sistema.
}

\subsubsection{Effettuare una chiamata}{
\begin{figure}[h!]
	\centering
		\includegraphics[scale=0.32]{\docsImg ChiamataInAttesa.png}
		\caption{Interfaccia dell'applicazione in fase di chiamata in attesa di risposta.} 
		\label{fig:imgChAt}
	\end{figure}
	
Per prima cosa è necessario individuare l’utente con il quale si desidera comunicare attraverso i pannelli di ricerca. Esistono tre tipi di pannelli di ricerca:
\begin{itemize}
	\item per Lista utenti;
	\item per Nome Utente;
	\item per Indirizzo IP.
\end{itemize}
Dopo aver selezionato l'utente si deve avviare la chiamata premendo il pulsante \texttt{Chiama}.
Se l’utente richiesto è online viene inviata la richiesta di comunicazione e verrà visualizzato il pannello di \texttt{Chiamata in Corso} come mostrato in figura \ref{fig:imgChAt}. Se l'utente chiamato non risponde gli verrà inviato automaticamente un messaggio per avvisarlo della chiamata persa.
Se l’utente selezionato invece è offline, ci sono due modi distinti di gestire la chiamata:
\begin{itemize}
	\item nella ricerca da lista utente il bottone \texttt{Chiama} non viene abilitato e quindi non si può effettuare la chiamata;
	\item nella ricerca per nome utente o per indirizzo IP invece si può far partire la chiamata, ma la richiesta non viene inviata e l'utente mittente viene opportunamente segnalato.
\end{itemize}
}
\newpage
\subsubsection{Ricerca tramite Lista Utenti}{
La lista utenti contiene le informazioni relative agli utenti registrati al servizio indicandone nome, cognome e indirizzo email. Inoltre ogni utente è contrassegnato da un indicatore verde se è online, rosso se è offline.
Dopo aver ricercato l’utente nel pannello Lista Utenti, è possibile mandare una richiesta di comunicazione  all’utente prescelto premendo il bottone \texttt{Chiama}.
\begin{figure}[h!]
	\centering
		\includegraphics[scale=0.32]{\docsImg RicercaLista.png}
		\caption{Interfaccia dell'applicazione in fase di ricerca tramite lista utenti.} 
		\label{fig:imgComLU}
	\end{figure}
	
\newpage

\subsubsection{Ricerca per Nome Utente}{
La ricerca per nome utente (email) si effettua inserendo, attraverso la form\g, l’indirizzo e-mail della persona desiderata.
\begin{figure}[h!]
	\centering
		\includegraphics[scale=0.32]{\docsImg RicercaEmail.png}
		\caption{Interfaccia dell'applicazione in fase di ricerca tramite nome utente.} 
		\label{fig:imgComNU}
	\end{figure}

}

\newpage
\subsubsection{Ricerca per IP}{
La ricerca per indirizzo IP si effettua inserendo, attraverso la form\g, l’indirizzo IP della persona desiderata. Nel riquadro blu viene invece mostrato il proprio indirizzo IP.
\begin{figure}[h!]
	\centering
		\includegraphics[scale=0.32]{\docsImg RicercaIP.png}
		\caption{Interfaccia dell'applicazione in fase di ricerca tramite indirizzo IP.} 
		\label{fig:imgComIP}
	\end{figure}
}

\newpage
\subsubsection{Ricezione di una chiamata}{
\begin{figure}[h!]
	\centering
		\includegraphics[scale=0.32]{\docsImg Ricezione.png}
		\caption{Interfaccia dell'applicazione in fase di ricezione di una chiamata.} 
		\label{fig:imgComRic}
	\end{figure}

Se un utente vi invia una richiesta di comunicazione apparirà il pannello di \texttt{Chiamata in arrivo}. Tale pannello permette di accettare la richiesta di comunicazione attraverso il bottone \texttt{Accetta} o di rifiutare attraverso il bottone \texttt{Rifiuta}.\\
Se la chiamata viene accettata apparirà il pannello \texttt{Informazioni Chiamata}, come si vede in figura \ref{fig:imgComInCorso}, se la chiamata viene rifiutata si ritorna alla pagina iniziale della comunicazione, mostrata in figura \ref{fig:imgComLU}.
Se invece non rispondete vi verrà inviato automaticamente un messaggio in segreteria per avvisarvi della chiamata persa.
}

\newpage

\subsubsection{Pannello Informazioni Chiamata}{


Quando lo scambio di richieste è andato a buon fine, a tutti gli utenti coinvolti nella conversazione apparirà il pannello \texttt{Informazioni Chiamata}, il quale visualizzerà i dati tecnici della chiamata:
	\begin{itemize}
	\item[-] Durata della conversazione;
	\item[-] Latenza;
	\item[-] Byte trasmessi;
	\item[-] Pacchetti trasmessi;
	\item[-] Pacchetti persi.
	\end{itemize}
Attraverso tale pannello è possibile esprimere un giudizio sulla comunicazione selezionando uno dei campi proposti.\\
Per terminare la conversazione si deve premere il bottone \texttt{Chiudi}.

\begin{figure}[h!]
	\centering
		\includegraphics[scale=0.32]{\docsImg Chiamata-in-corso.png}
		\caption{Interfaccia dell'applicazione in fase di chiamata in corso.} 
		\label{fig:imgComInCorso}
	\end{figure}
}

\subsubsection{Termine di una conversazione}{

Prima della conclusione di una conversazione sarebbe preferibile dare un giudizio riguardante la qualità dell’esperienza dell’utente. Questa valutazione è utile agli utenti amministratori per recuperare informazioni riguardanti il servizio.
Per terminare la comunicazione bisogna premere il pulsante \texttt{Chiudi}, presente in figura \ref{fig:imgComInCorso}, che riporta alla pagina iniziale della comunicazione (figura \ref{fig:imgComLU}).
}

\subsubsection{Segreteria}
Il pulsante per accedere alla segreteria è rappresentato da una busta e si trova subito dopo i bottoni predisposti per la ricerca utenti, come si vede in figura \ref{fig:imgComAp}.\\
Al primo accesso la segreteria sarà vuota, quindi si presenterà come rappresentato nella figura \ref{fig:imgComNesMes}.
La segreteria consente di:
\begin{itemize}
\item \textbf{Inviare messaggi ad altri utenti}:\\
Per inviare un messaggio ad un altro utente bisogna selezionare l'utente destinatario dalla lista utenti e premere il pulsante \texttt{Lascia Messaggio} che si vede in figura \ref{fig:imgComLU}. Al centro della schermata apparirà una finestra in cui digitare il messaggio, come mostrato in figura \ref{fig:imgComInvMes}, per inviarlo bisogna premere il pulsante \texttt{Invia}, altrimenti per annullare l'operazione bisogna premere il pulsante \texttt{Annulla}. I messaggi possono essere inviati anche se l'utente destinatario è offline, infatti il messaggio sarà memorizzato e poi verrà inoltrato solo quando il destinatario accederà al servizio.\\

\begin{figure}[h!]
	\centering
		\includegraphics[scale=0.32]{\docsImg InvioMessaggio.png}
		\caption{Interfaccia dell'applicazione in fase di scrittura e invio messaggi.} 
		\label{fig:imgComInvMes}
	\end{figure}

\item \textbf{Ricevere messaggi da altri utenti}:\\
Quando si riceve un messaggio l'icona della segreteria si modifica come si vede in figura \ref{fig:imgComRicMes}. Selezionando l'icona si potrà leggere il messaggio ricevuto, come si vede in figura \ref{fig:imgComLetMes} e, se ce ne sono, leggere i messaggi già ricevuti in precedenza.\\

\begin{figure}[h!]
	\centering
		\includegraphics[scale=0.32]{\docsImg RicezioneMessaggio.png}
		\caption{Interfaccia dell'applicazione in fase di ricezione messaggi.} 
		\label{fig:imgComRicMes}
	\end{figure}
	
\begin{figure}[h!]
	\centering
		\includegraphics[scale=0.32]{\docsImg LetturaMessaggio.png}
		\caption{Interfaccia dell'applicazione in fase di lettura messaggi.} 
		\label{fig:imgComLetMes}
	\end{figure}

\item \textbf{Eliminare messaggi ricevuti}:\\
Per eliminare i messaggi ricevuti bisogna premere il pulsante \texttt{Elimina tutti i messaggi} presente in figura \ref{fig:imgComLetMes}. Attenzione: non è possibile eliminare un messaggio alla volta, premendo il pulsante verranno quindi eliminati tutti i messaggi presenti. Dopo aver eliminato i messaggi l'interfaccia si presenterà come mostrato in figura \ref{fig:imgComNesMes}. Se in segreteria non è presente alcun messaggio il pulsante \texttt{Elimina tutti i messaggi} viene disabilitato.

\begin{figure}[h!]
	\centering
		\includegraphics[scale=0.32]{\docsImg NessunMessaggio.png}
		\caption{Interfaccia dell'applicazione senza messaggi in segreteria.} 
		\label{fig:imgComNesMes}
	\end{figure}

\end{itemize}

\newpage

\subsection{Gestione dei dati}{
\label{gestDati}
\begin{figure}[h!]
	\centering
		\includegraphics[scale=0.55]{\docsImg VisualizzazioneDati.png}
		\caption{Interfaccia dell'applicazione in fase di visualizzazione dati.} 
		\label{fig:imgVisDati}
	\end{figure}

La pagina di gestione dei dati è un’interfaccia che permette di visualizzare e, in caso, modificare le proprie credenziali. Essa presenta due pannelli:
\begin{itemize}
	\item Visualizza Dati: il pannello di apertura della pagina;
	\item Modifica Dati: vi si accede premendo il bottone \texttt{Modifica}.
\end{itemize}
All’interno della pagina sono presenti due link che permettono di muoversi all’interno dell’applicazione:
\begin{itemize}
	\item Chiama: permette all’utente di tornare alla pagina di Comunicazione;
	\item Logout: permette all’utente di uscire dall’applicazione. 
\end{itemize}
}

\subsubsection{Visualizza Dati}{
Questo pannello (figura \ref{fig:imgVisDati}) permette all’utente autenticato di visualizzare le proprie credenziali. Premendo il bottone \texttt{Modifica} è possibile modificare i dati aprendo il pannello \texttt{Modifica Dati}.

\subsubsection{Modifica Dati}{
\begin{figure}[h!]
	\centering
		\includegraphics[scale=0.55]{\docsImg ModificaDati.png}
		\caption{Interfaccia dell'applicazione in fase di modifica dati.} 
		\label{fig:imgModDati}
	\end{figure}

Questo pannello permette all’utente autenticato di modificare le proprie credenziali cambiando i dati presenti nella form\g~ e premendo il bottone \texttt{Conferma}. È possibile modificare tutti i dati o solo alcuni. Andata a buon fine la modifica, il sistema torna al pannello \texttt{Visualizza Dati}. Se l’utente autenticato desidera annullare le modifiche, o semplicemente tornare al pannello \texttt{Visualizza Dati} deve premere il bottone \texttt{Annulla}.
}
}

\subsection{Logout}{
	Per effettuare il logout dall'applicazione bisogna premere il link \texttt{Logout} che si vede in figura \ref{fig:imgComNU}, ma che è presente in ogni pagina accedibile dall'utente autenticato.\\
	
	\subsubsection{Aggiornamento della pagina}{
	È possibile effettuare il logout anche aggiornando la pagina, navigando verso un altro sito o chiudendo la finestra. In questi casi verrà richiesto all'utente se è sicuro di continuare l'operazione, come mostrato in figura \ref{fig:imgRic}.\\ Se viene premuto il pulsante \texttt{Ricarica questa pagina} sarà effettuato il logout e completata l'operazione richiesta. Se invece viene premuto il pulsante \texttt{Non ricaricare} l'operazione richiesta viene annullata e si rimane sulla stessa pagina.
	
	\begin{figure}[h!]
	\centering
		\includegraphics[scale=0.37]{\docsImg Ricaricamento.png}
		\caption{Interfaccia dell'applicazione in fase di ricaricamento pagina.} 
		\label{fig:imgRic}
	\end{figure}
	}
}
}