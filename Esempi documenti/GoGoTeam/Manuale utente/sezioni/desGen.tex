\section{Descrizione generale}\label{desGen}{

\subsection{Installazione del prodotto}{
Per l'installazione dell'applicazione web-based\g~ si veda il documento Manuale per l'installatore.
}

\subsection{Descrizione interfaccia grafica}{{
L'utente può interagire con l'applicazione \textbf{\mytalk} tramite un'interfaccia grafica semplice ed intuitiva, tale da facilitare le operazioni che si intendono effettuare.\\
Di seguito sono riportate le interfacce principali, che riguardano le funzionalità offerte dal servizio, e una rapida guida al loro uso.
}


\subsection{Layout\g~ dell'applicazione}{
In questa sezione descriveremo le aree di lavoro che compongono il layout\g~ dell'interfaccia attraverso la quale l'utente interagirà con \textbf{\mytalk}.
\begin{figure}[h!]
	\centering
	\includegraphics[scale=0.30]{\docsImg ComunicazioneAperturaColori.png}
	\caption{Layout\g~ dell'applicazione.}
	\label{fig:layoutGen} 
\end{figure}

L'area all'interno del riquadro nero contiene lo username dell'utente, il collegamento per richiamare la schermata di gestione dell'account e il collegamento per effettuare il logout da \textbf{\mytalk}.\\
L'area contenuta nel riquadro rosso contiene due zone, in quella superiore verrà visualizzata l'immagine dell'utente con cui si interagisce mentre in quella inferiore si vede la propria immagine.\\ 
L'area delimitata dal riquadro verde presenta il menù di navigazione dove sono elencate le operazioni che l'utente può compiere al fine di interagire con il sistema. Una volta scelta un'azione, nell'area delimitata dal riquadro blu viene visualizzato un insieme di informazioni relative alla scelta.

%L'area all'interno del riquadro nero contiene lo username dell'utente, il collegamento per richiamare la schermata di gestione dell'account e il collegamento per effettuare il logout da \textbf{\mytalk}.\\
%L'area delimitata dal riquadro verde presenta il menù di navigazione dove sono elencate le modalità con cui l'utente può scegliere il destinatario della videochiamata. Una volta scelta una modalità, nell'area delimitata dal riquadro blu viene visualizzato il modo per effettuare la scelta.
	}
}%Descrizione interfaccia grafica

}