\section{Errori e possibili cause} \label{errori} {

Vengono descritte le situazioni di errore che possono avvenire durante l'uso dell'applicazione:
\begin{itemize}
	\item \textbf{E1 - Pagina bianca all'accesso dell'applicazione: }{
		questa situazione si verifica perché nel browser\g~ che si sta attualmente usando non è abilitato l'utilizzo di Javascript\g .\\
		Nelle varie versioni del browser\g~ Google Chrome\g~ è possibile risolvere il problema in due modi:
		\begin{itemize}
			\item Globale: dal menù di personalizzazione è possibile abilitare Javascript\g~ per tutti i siti web visitati;
			\item Locale: nella barra degli indirizzi, abilitare l'esecuzione di Javascript\g~ unicamente nel sito del servizio (figura \ref{fig:JSErr}).
		\end{itemize}
	}
	\item \textbf{E2 - Credenziali non corrette: }{
		questa situazione si verifica perché nella scheda di autenticazione sono stati inseriti dei dati non validi. L'errore più comune è l'inserimento di caratteri maiuscoli al posto dei minuscoli (o viceversa).\\
		Riprovare ponendo attenzione durante la digitazione dello username e della password.
	}
	\item \textbf{E3 - Errore di connessione al server\g~: }{
		quando si sono verificati dei problemi nella connessione con il server\g~ viene visualizzato un messaggio d'errore.\\
		Verificare che la propria connessione internet sia attiva e che nessun firewall\g~ impedisca la trasmissione, dunque, ricaricare la pagina.
	}
	\item \textbf{E4 - Segnale video non presente: }{
		quando il segnale video non è presente significa che non è stato consentito l’accesso all’hardware video e audio.\\
		 Consentire l'accesso all'hardware tramite l'apposito pulsante presente nella parte superiore della pagina, se questo non è presente effettuare il logout e ricaricare la pagina.
	}
	\item \textbf{E5 - Parametro di ricerca errato: }{
		questa situazione si verifica perché si sono inseriti valori di ricerca errati (senza rispettare la formattazione richiesta) per la ricerca tramite nome utente o indirizzo IP.\\
		Riprovare ponendo attenzione durante la digitazione del valore (seguendo la linea guida presente per ciascun formato).
	}
	\item \textbf{E6 - Utente offline o inesistente: }{
		nella ricerca tramite nome utente o tramite indirizzo IP, quando l’utente indicato non è disponibile perché è offline oppure inesistente viene opportunamente segnalato.\\
		 Attendere che torni online o controllare il campo di ricerca.
	\item \textbf{E7 - Consentire trasmissione video: }{
		quando appare il messaggio che riporta la dicitura \texttt{consentire la trasmissione video}, come si vede in figura \ref{fig:ErroreVideo}, significa che un altro utente vi sta chiamando ma che il vostro video locale non è attivato. Per attivarlo bisogna premere il pulsante \texttt{Attiva Video Locale}.
		} 
	}
\end{itemize}

Tutti i form\g~ compilati dall'utente, che richiedono degli input da inserire, sono sottoposti a validazione in tempo reale. Se tutti i campi superano la validazione non compare nulla mentre se uno o più campi non la superano c'è la segnalazione dell'errore riscontrato come mostrato nelle figure \ref{fig:Errori} e \ref{fig:ErroreIP}.

\begin{center}
\begin{figure}[h]
	\centering
	\includegraphics[scale=0.55]{\docsImg Registrazione+errori.png}
	\caption{Immagine gestione errori.}
	\label{fig:Errori}
\end{figure}
\end{center}

\begin{center}
\begin{figure}[h]
	\centering
	\includegraphics[scale=0.37]{\docsImg ErroreIP.png}
	\caption{Immagine gestione errori nell'inserimento indirizzo IP.}
	\label{fig:ErroreIP}
\end{figure}
\end{center}

\begin{center}
\begin{figure}[h]
	\centering
	\includegraphics[scale=0.37]{\docsImg 10-mytalk-jserror.png}
	\caption{Abilitare l'esecuzione di Javascript\g~ solo per il sito corrente.}
	\label{fig:JSErr}
\end{figure}
\end{center}

\begin{center}
\begin{figure}[h]
	\centering
	\includegraphics[scale=0.37]{\docsImg ErroreVideo.png}
	\caption{Immagine gestione errore del video locale.}
	\label{fig:ErroreVideo}
\end{figure}
\end{center}
\clearpage
}