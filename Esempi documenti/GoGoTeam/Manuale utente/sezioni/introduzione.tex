\section{Introduzione}{

	\subsection{Scopo del documento}{
Lo scopo del presente documento è quello di fornire le linee guida per l’utilizzo del prodotto \textbf{\mytalk} da parte degli utenti, così da facilitarne l'uso e portare alla loro conoscenza tutte le funzionalità offerte dal software.
	}

%\subsection{Definizione del cliente del prodotto}
%Il software \textbf{\mytalk} è indirizzato a tutti coloro che desiderano comunicare con altre persone tramite videochiamate. \textbf{\mytalk} è stato progettato per le comunicazioni interne di una società.


\subsection{Definizione dell'utente finale del prodotto}{
Il prodotto è destinato a due differenti tipologie di utente:
\begin{itemize}
	\item[] \texttt{Utenti}: coloro che possono effettuare comunicazioni con altri utenti registrati;
	\item[] \texttt{Amministratori}: coloro che monitorano lo stato del sistema attraverso la raccolta delle informazioni di comunicazione effettuate dagli utenti.
\end{itemize}
	}

\subsection{Come leggere il manuale}{
Il presente manuale è un ausilio all'applicazione ed ha lo scopo di illustrare le funzionalità del prodotto in modo da spiegare il corretto uso per ottenere i risultati richiesti.\\
Il manuale è suddiviso in cinque sezioni:
\begin{itemize}
	\item {\textbf{Introduzione}: illustra come leggere questo manuale, come riportare malfunzionamenti ed i requisiti minimi di sistema.
	}
	\item {\textbf{Descrizione generale}: (vedi \ref{desGen}) contiene le informazioni per il primo accesso al prodotto ed una descrizione del layout\g~ dell'interfaccia grafica.
	}
	\item {\textbf{Descrizione funzionale}: (vedi \ref{desFunz})  contiene una descrizione generale dell'applicazione, le istruzioni per l'uso del prodotto ed un elenco delle funzionalità fornite dai vari elementi dell'interfaccia grafica.
	}
	\item {\textbf{Errori}: (vedi \ref{errori})  contiene un elenco dei messaggi d'errore che potrebbero comparire, una descrizione delle possibili cause ed un modo per risolvere la situazione.
	}
	\item {\textbf{Appendice}:(vedi \ref{appendice})   contiene termini che potrebbero risultare di difficile comprensione per l'utente. Nel documento tali termini saranno contrassegnati dal simbolo \g~ alla fine della parola; per i termini composti da più parole, oltre al simbolo \g , è presente anche la sottolineatura.

	}
\end{itemize}
\`E disponibile un video tutorial per l'utente finale all'indirizzo \url{http://www.youtube.com/watch?v=OGXz-vm7ooY}.
}

\subsection{Come riportare problemi e malfunzionamenti}{
Per qualsiasi bug\g~ o malfunzionamento inatteso di \textbf{\mytalk} che non è descritto nel presente documento si prega di segnalarlo al seguente indirizzo e-mail:
\begin{center}
	\url{\mailGruppo}
\end{center}
specificando le seguenti informazioni:
\begin{itemize}
	\item La dicitura ''Problemi'' nel campo relativo all'oggetto;
	\item Tipo di browser\g~ che si sta usando con relativa versione;
	\item Il problema che è sorto indicando l'operazione che si voleva eseguire;
	\item Una descrizione con la massima precisione degli eventuali messaggi di errore ricevuti.
\end{itemize}
}

\subsection{Requisiti di sistema}{
Per poter usufruire del prodotto, è necessario che la propria postazione di lavoro abbia una connessione a internet funzionante. Inoltre è necessario aver installato nel proprio sistema il browser\g~ Google Chrome\g~ versione 26 o successive.

}
	
}