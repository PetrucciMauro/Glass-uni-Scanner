\appendix
\appendixpage
\addappheadtotoc
\section{Specifica delle comunicazioni}
\label{par:XMLComm}

%
% Ricordarsi:
% <u t="123"> val </u>
% u = nodo elemento (in breve, elemento)
%	val = valore elemento
% t = nodo attributo (in breve, attributo)
%	123 = valore attributo
%

	{
In questo capitolo vengono specificate le regole adottate nei messaggi di scambio tra gli attori che utilizzano i servizi del programma. Nella comunicazione tra client e server è stato scelto di incapsulare le informazioni da inviare (oggetti JSON\g~ compresi) all’interno di messaggi XML\g~ .
Gli attori indicati sono:
\begin{itemize}
	\item[]{ Client: identifica la parte dell'applicativo utilizzata dall'utente finale. I componenti che lo identificano sono le classi definite nel package \path{mytalk.client.*}.\\
	Le comunicazioni inviate possono essere riferite ad altri Client, nella fase di instaurare una nuova comunicazione e nella sua gestione, e verso il Server, per ottenere informazioni.
	}
	\item[]{ Server: identifica la parte dell'applicativo che gestisce le informazioni e consente la tracciabilità dei client. I componenti che lo identificano sono le classi definite nel package \path{mytalk.server.*}.\\
	Le comunicazioni inviate possono essere riferite ai Client che si sono registrati nel server.
	}
\end{itemize}
	}

\subsection{Riferimento nomenclatura tag ed attributi}{
	Nei messaggi il nome dei nodi elemento e dei nodi attributo sono stati scelti per consentire una rapida interpretazione e per ridurre la dimensione finale del messaggio.\\
	Se ne specifica in seguito il nome e l'ambito di utilizzo.
	\begin{itemize}
		\item[] \texttt{c}:{ Communication.\\
		Dev'essere utilizzato come elemento radice per indicare tutti i messaggi dove si devono gestire informazioni relative alle comunicazioni.\\
		Contiene l'attributo obbligatorio:
		\begin{itemize}
			\item[] \texttt{op}:{ Operation.\\
			Definisce il tipo di operazione da effettuare:
				\begin{itemize}
					\item[-] \texttt{am}: Answering Message, aggiungi messaggio di segreteria (vedi \ref{opAM});
					\item[-] \texttt{amd}: Answering Message Delete, rimuovi messaggi in segreteria (vedi \ref{opAMD});
					\item[-] \texttt{add}: Add, aggiungi statistiche (vedi \ref{opCInsStat});
					\item[-] \texttt{callNegotation}: Call negotation, informazione per instaurare la comunicazione (vedi \ref{opCNeg});
					\item[-] \texttt{callExchange}: Call exchange, informazioni per avviare la comunicazione (vedi \ref{opCExg});
					\item[-] \texttt{find}: Find, effettua una ricerca secondo determinati parametri (vedi \ref{opCEus});
					\item[-] \texttt{stat}: Stat, richiedi informazioni statistiche secondo determinati parametri (vedi \ref{opCValStat});
					\item[-] \texttt{userList}: User list, lista utenti registrati (vedi \ref{opCUList}).
					
				\end{itemize}
		Può avere gli attributi opzionali:
				\begin{itemize}
					\item[-] \texttt{type}: Type, definisce il sottotipo di operazione da effettuare;
					\item[-] \texttt{h}: Hours, definisce il numero di ore.
				\end{itemize}
			}
		\end{itemize}
		}
		\item[] \texttt{d}:{ Date.\\
		Definisce la data generica nel formato \texttt{AAAAMMGGHHMMSS}.
		}
		\item[] \texttt{de}:{ Date end.\\
		Definisce la data di termine nel formato \texttt{AAAAMMGGHHMMSS}.
		}
		\item[] \texttt{ds}:{ Date start.\\
		Definisce la data di partenza nel formato \texttt{AAAAMMGGHHMMSS}.
		}		
		\item[] \texttt{dsc}:{ Description.\\
		Definisce le informazioni aggiuntive.
		}		
		\item[] \texttt{er}:{ Error.\\
		Definisce il commento di errore all'operazione.
		}
		\item[] \texttt{es}:{ Result.\\
		Definisce il valore risultato dell'operazione.
		}
		\item[] \texttt{g}:{ Grade.\\
		Definisce il valore di gradimento.
		}
		\item[] \texttt{ip}:{ IP.\\
		Definisce il valore IP.
		}
		\item[] \texttt{ol}:{ Online.\\
		Definisce il valore Online.
		}	
		\item[] \texttt{m}:{ E-mail.\\
		Definisce l'indirizzo e-mail.
		}
		\item[] \texttt{msg}:{ E-mail.\\
		Definisce il pacchetto di informazioni del messaggio di segreteria.
		}
		\item[] \texttt{nm}:{ Name.\\
		Definisce il nome.
		}
		\item[] \texttt{p}:{ Password.\\
		Definisce la password.
		}
		\item[] \texttt{pk}:{ Packages.\\
		Definisce il numero di pacchetti.
		}
		\item[] \texttt{pkL}:{ Packages lost.\\
		Definisce il numero di pacchetti persi.
		}
		\item[] \texttt{r}:{ Result.\\
		Elemento che raggruppa le informazioni.
		}
		\item[] \texttt{ref}:{ Reference.\\
		Definisce il valore di riferimento.
		}
		\item[] \texttt{sc}:{ Society.\\
		Definisce la società.
		}
		\item[] \texttt{sg}:{ Sender grade.\\
		Definisce il valore di gradimento del mittente.
		}		
		\item[] \texttt{sn}:{ Surname.\\
		Definisce il cognome.
		}
		\item[] \texttt{ss}:{ Stream size.\\
		Definisce la dimensione della comunicazione.
		}
		\item[] \texttt{st}:{ Status.\\
		Definisce lo stato dell'operazione.
		}
		\item[] \texttt{text}:{ Text.\\
		Definisce il messaggio testuale.
		}
		\item[] \texttt{tl}:{ Telephone.\\
		Definisce il numero di telefono.
		}
		\item[] \texttt{ud}:{ UserData.\\
		Dev'essere utilizzato come elemento radice per indicare tutti i messaggi dove si devono gestire informazioni relative all'utente client o amministratore.\\
		Contiene l'attributo obbligatorio:
		\begin{itemize}
			\item[] \texttt{op}:{ Operation.\\
			Specifica l'operazione da effettuare:
				\begin{itemize}
					\item[-] \texttt{log}: Login utente (vedi \ref{opLog});
					\item[-] \texttt{ulg}: Logout utente (vedi \ref{opUlg});
					\item[-] \texttt{add}: Add, aggiungi nuovo utente (vedi \ref{opUReg});
					\item[-] \texttt{udt}: User Data, richiedi dati utente (vedi \ref{opUDT});
					\item[-] \texttt{mod}: Modify, modifica i dati utente (vedi \ref{opUMod});
					\item[-] \texttt{dlt}: Delete, elimina utente (vedi \ref{opUDel}).
				\end{itemize}
			}
		\end{itemize}
		}
		\item[] \texttt{ur}:{ User receiver.\\
		Definisce il riferimento all'utente destinatario.
		}
		\item[] \texttt{us}:{ User sender.\\
		Definisce il riferimento all'utente mittente.
		}	
	\end{itemize}
}

\subsection{Comunicazioni Client-Server}{

	\subsubsection{Indirizzo IP}{
	\label{opIP}
	
		\begin{itemize}
			\item[] \textbf{Funzione:}{\\
				invia all'utente client il suo indirizzo IP.
				}
			
			\item[] \textbf{Messaggio Client -\textgreater~ Server:}{\\
				Nessuno.
				}
			\item[] \textbf{Messaggio Server -\textgreater~ Client:}{\\
			\textbf{Messaggio XML}\\
				\begin{lstlisting}
<ip>
	{Valore indirizzo IP}
</ip>
				\end{lstlisting}
			\textbf{Validazione XSD}\\
				\begin{lstlisting}
<xs:schema xmlns:xs="http://www.w3.org/2001/XMLSchema" 
 targetNamespace="http://www.mytalk.com" 
 xmlns="http://www.mytalk.com" 
 elementFormDefault="qualified">

<xs:element name="ip" type="xs:string" minOccurs="1" maxOccurs="1">
</xs:element>

</xs:schema>	
				\end{lstlisting}
				}
		\end{itemize}		
	}
	
	\subsubsection{Login utente}{
	\label{opLog}
		\begin{itemize}
			\item[] \textbf{Funzione:}{\\
				invia al Server le informazioni necessarie per autenticare l'utente.
				}
			
			\item[] \textbf{Messaggio Client -\textgreater~ Server:}{\\
			\textbf{Messaggio XML}\\
				\begin{lstlisting}
<ud op="log">
	<m>
		{E-mail utente}
	</m>
	<p>
		{Password utente}
	</p>
	<ip>
		{Valore indirizzo IP}
	</ip>
</ud>
				\end{lstlisting}
			\textbf{Validazione XSD}\\
				\begin{lstlisting}
<xs:schema xmlns:xs="http://www.w3.org/2001/XMLSchema"
targetNamespace="http://www.mytalk.com"
xmlns="http://www.mytalk.com"
elementFormDefault="qualified">

<xs:element name="ud">
  <xs:complexType>
    <xs:sequence>
      <xs:element name="m" type="emailAddress" minOccurs="1" maxOccurs="1"/>
      <xs:element name="p" type="xs:string" minOccurs="1" maxOccurs="1"/>
      <xs:element name="ip" type="xs:string" minOccurs="1" maxOccurs="1"/>
    </xs:sequence>
    <xs:attribute name="op" type="xs:string" use="required"/>
  </xs:complexType>
</xs:element>

<xs:simpleType name="emailAddress"> 
    <xs:restriction base="xs:string"> 
      <xs:pattern value="[^@]+@[^\.]+\..+"/> 
    </xs:restriction> 
</xs:simpleType> 

</xs:schema>	
				\end{lstlisting}
				}
				\item[] \textbf{Messaggio Server -\textgreater~ Client:}{\\
				\textbf{Messaggio XML}\\
				\begin{lstlisting}
<ud op="log">
	<es>
		{Esito operazione}
	</es>
	<m>
		{E-mail utente}
	</m>
	<p>
		{Password utente}
	</p>
</ud>
				\end{lstlisting}
				\textbf{Validazione XSD}\\
				\begin{lstlisting}
<xs:schema xmlns:xs="http://www.w3.org/2001/XMLSchema"
targetNamespace="http://www.mytalk.com"
xmlns="http://www.mytalk.com"
elementFormDefault="qualified">

<xs:element name="ud">
  <xs:complexType>
    <xs:sequence>
      <xs:element name="es" type="xs:string" minOccurs="1" maxOccurs="1"/>
      <xs:element name="m" type="emailAddress" minOccurs="1" maxOccurs="1"/>
      <xs:element name="p" type="xs:string" minOccurs="1" maxOccurs="1"/>
    </xs:sequence>
    <xs:attribute name="op" type="xs:string" use="required"/>
  </xs:complexType>
</xs:element>

<xs:simpleType name="emailAddress"> 
    <xs:restriction base="xs:string"> 
      <xs:pattern value="[^@]+@[^\.]+\..+"/> 
    </xs:restriction> 
</xs:simpleType> 

</xs:schema>	
				\end{lstlisting}
				Il valore contenuto nell'elemento \texttt{es} deve essere:
				\begin{itemize}
					\item \texttt{true}: login effettuato con successo;
					\item \texttt{false}: login non riuscito.
				\end{itemize}
			}
		\end{itemize}
	}%Login
	
		\subsubsection{Logout utente}{
	\label{opUlg}
		\begin{itemize}
			\item[] \textbf{Funzione:}{\\
				invia al Server le informazioni necessarie per indicare la disconnessione al servizio dell'utente.
				}
			
			\item[] \textbf{Messaggio Client -\textgreater~ Server:}{\\
				\textbf{Messaggio XML}\\
				\begin{lstlisting}
<ud op="ulg">
	<m>
		{E-mail utente}
	</m>
</ud>
				\end{lstlisting}
				\textbf{Validazione XSD}\\
				\begin{lstlisting}
<xs:schema xmlns:xs="http://www.w3.org/2001/XMLSchema"
targetNamespace="http://www.mytalk.com"
xmlns="http://www.mytalk.com"
elementFormDefault="qualified">

<xs:element name="ud">
  <xs:complexType>
    <xs:sequence>
      <xs:element name="m" type="emailAddress" minOccurs="1" maxOccurs="1"/>
    </xs:sequence>
    <xs:attribute name="op" type="xs:string" use="required"/>
  </xs:complexType>
</xs:element>

<xs:simpleType name="emailAddress"> 
    <xs:restriction base="xs:string"> 
      <xs:pattern value="[^@]+@[^\.]+\..+"/> 
    </xs:restriction> 
</xs:simpleType> 

</xs:schema>
				\end{lstlisting}
			}
				\item[] \textbf{Messaggio Server -\textgreater~ Client:}{\\
				Nessuno.
				}
		\end{itemize}
	}%Logout
	
		\subsubsection{Registrazione nuovo utente}{
	\label{opUReg}
		\begin{itemize}
			\item[] \textbf{Funzione:}{\\
				invia al Server le informazioni necessarie registrare un nuovo utente.
				}
			
			\item[] \textbf{Messaggio Client -\textgreater~ Server:}{\\
			\textbf{Messaggio XML}\\
				\begin{lstlisting}
<ud op="add">
	<m>
		{E-mail utente}
	</m>
	<p>
		{Password utente}
	</p>
	<nm>
		{Nome utente}
	</nm>
	<sn>
		{Cognome utente}
	</sn>
	<sc>
		{Societa' di riferimento per l'utente}
	</sc>
	<tl>
		{Telefono utente}
	</tl>
</ud>
				\end{lstlisting}
				\textbf{Validazione XSD}\\
				\begin{lstlisting}
<xs:schema xmlns:xs="http://www.w3.org/2001/XMLSchema"
targetNamespace="http://www.mytalk.com"
xmlns="http://www.mytalk.com"
elementFormDefault="qualified">

<xs:element name="ud">
  <xs:complexType>
    <xs:sequence>
      <xs:element name="m" type="emailAddress" minOccurs="1" maxOccurs="1"/>
      <xs:element name="p" type="xs:string" minOccurs="1" maxOccurs="1"/>
      <xs:element name="nm" type="xs:string" minOccurs="1" maxOccurs="1"/>
      <xs:element name="sn" type="xs:string" minOccurs="1" maxOccurs="1"/>
      <xs:element name="sc" type="xs:string" minOccurs="0" maxOccurs="1"/>
      <xs:element name="tl" type="xs:string" minOccurs="0" maxOccurs="1"/>
    </xs:sequence>
    <xs:attribute name="op" type="xs:string" use="required"/>
  </xs:complexType>
</xs:element>

<xs:simpleType name="emailAddress"> 
    <xs:restriction base="xs:string"> 
      <xs:pattern value="[^@]+@[^\.]+\..+"/> 
    </xs:restriction> 
</xs:simpleType> 

</xs:schema>
				\end{lstlisting}
				}
				
				\item[] \textbf{Messaggio Server -\textgreater~ Client:}{\\
				\textbf{Messaggio XML}\\
				\begin{lstlisting}
<ud op="add">
	<es>
		{Esito operazione}
	</es>
	<er>
		{Errore nell'operazione}
	</er>
</ud>
				\end{lstlisting}
				\textbf{Validazione XSD}\\
				\begin{lstlisting}
<xs:schema xmlns:xs="http://www.w3.org/2001/XMLSchema"
targetNamespace="http://www.mytalk.com"
xmlns="http://www.mytalk.com"
elementFormDefault="qualified">

<xs:element name="ud">
  <xs:complexType>
    <xs:sequence>
      <xs:element name="es" type="xs:string" minOccurs="1" maxOccurs="1"/>
      <xs:element name="er" type="xs:string" minOccurs="1" maxOccurs="1"/>
    </xs:sequence>
    <xs:attribute name="op" type="xs:string" use="required"/>
  </xs:complexType>
</xs:element>

</xs:schema>
				\end{lstlisting}
				Il valore contenuto nell'elemento \texttt{es} deve essere:
				\begin{itemize}
					\item \texttt{true}: login effettuato con successo;
					\item \texttt{false}: login non riuscito.
				\end{itemize}
				Nel caso di esito negativo dell'operazione, l'elemento \texttt{er} può contenere un messaggio informativo riguardanti le cause d'insuccesso.
				}
		\end{itemize}
	}%Registrazione
	
			\subsubsection{Richiesta dati utente}{
	\label{opUDT}
		\begin{itemize}
			\item[] \textbf{Funzione:}{\\
				invia al Server le informazioni per richiedere tutte le informazioni relative ad un specifico utente.
				}
			
			\item[] \textbf{Messaggio Client -\textgreater~ Server:}{\\
			\textbf{Messaggio XML}\\
				\begin{lstlisting}
<ud op="udt">
	<m>
		{E-mail utente}
	</m>
</ud>
				\end{lstlisting}
				\textbf{Validazione XSD}\\
				\begin{lstlisting}
<xs:schema xmlns:xs="http://www.w3.org/2001/XMLSchema"
targetNamespace="http://www.mytalk.com"
xmlns="http://www.mytalk.com"
elementFormDefault="qualified">

<xs:element name="ud">
  <xs:complexType>
    <xs:sequence>
      <xs:element name="m" type="emailAddress" minOccurs="1" maxOccurs="1"/>
    </xs:sequence>
    <xs:attribute name="op" type="xs:string" use="required"/>
  </xs:complexType>
</xs:element>

<xs:simpleType name="emailAddress"> 
    <xs:restriction base="xs:string"> 
      <xs:pattern value="[^@]+@[^\.]+\..+"/> 
    </xs:restriction> 
</xs:simpleType> 

</xs:schema>
				\end{lstlisting}
				}
				
				\item[] \textbf{Messaggio Server -\textgreater~ Client:}{\\
				\textbf{Messaggio XML}\\
				\begin{lstlisting}
<ud op="add">
	<m>
		{E-mail utente}
	</m>
	<p>
		{Password utente}
	</p>
	<nm>
		{Nome utente}
	</nm>
	<sn>
		{Cognome utente}
	</sn>
	<sc>
		{Societa' di riferimento per l'utente}
	</sc>
	<tl>
		{Telefono utente}
	</tl>
</ud>
				\end{lstlisting}
				\textbf{Validazione XSD}\\
				\begin{lstlisting}
<xs:schema xmlns:xs="http://www.w3.org/2001/XMLSchema"
targetNamespace="http://www.mytalk.com"
xmlns="http://www.mytalk.com"
elementFormDefault="qualified">

<xs:element name="ud">
  <xs:complexType>
    <xs:sequence>
      <xs:element name="m" type="emailAddress" minOccurs="1" maxOccurs="1"/>
      <xs:element name="p" type="xs:string" minOccurs="1" maxOccurs="1"/>
      <xs:element name="nm" type="xs:string" minOccurs="1" maxOccurs="1"/>
      <xs:element name="sn" type="xs:string" minOccurs="1" maxOccurs="1"/>
      <xs:element name="sc" type="xs:string" minOccurs="0" maxOccurs="1"/>
      <xs:element name="tl" type="xs:string" minOccurs="0" maxOccurs="1"/>
    </xs:sequence>
    <xs:attribute name="op" type="xs:string" use="required"/>
  </xs:complexType>
</xs:element>

<xs:simpleType name="emailAddress"> 
    <xs:restriction base="xs:string"> 
      <xs:pattern value="[^@]+@[^\.]+\..+"/> 
    </xs:restriction> 
</xs:simpleType> 

</xs:schema>
				\end{lstlisting}
				}
		\end{itemize}
	}%Dati utente
	
		\subsubsection{Modifica dati utente}{
	\label{opUMod}
		\begin{itemize}
			\item[] \textbf{Funzione:}{\\
				invia al Server le informazioni necessarie per modificare i dati di un utente specifico.
				}
			
			\item[] \textbf{Messaggio Client -\textgreater~ Server:}{\\
			\textbf{Messaggio XML}\\
				\begin{lstlisting}
<ud op="mod">
	<ref>
		{E-mail utente di riferimento}
	</ref>
	<m>
		{Nuova e-mail utente}
	</m>
	<p>
		{Nuova password utente}
	</p>
	<nm>
		{Nuovo nome utente}
	</nm>
	<sn>
		{Nuovo cognome utente}
	</sn>
	<sc>
		{Nuova societa' di riferimento per l'utente}
	</sc>
	<tl>
		{Nuovo telefono utente}
	</tl>
</ud>
				\end{lstlisting}
				\textbf{Validazione XSD}\\
				\begin{lstlisting}
<xs:schema xmlns:xs="http://www.w3.org/2001/XMLSchema"
targetNamespace="http://www.mytalk.com"
xmlns="http://www.mytalk.com"
elementFormDefault="qualified">

<xs:element name="ud">
  <xs:complexType>
    <xs:sequence>
      <xs:element name="ref" type="emailAddress" minOccurs="1" maxOccurs="1"/>
      <xs:element name="m" type="emailAddress" minOccurs="0" maxOccurs="1"/>
      <xs:element name="p" type="xs:string" minOccurs="0" maxOccurs="1"/>
      <xs:element name="nm" type="xs:string" minOccurs="0" maxOccurs="1"/>
      <xs:element name="sn" type="xs:string" minOccurs="0" maxOccurs="1"/>
      <xs:element name="sc" type="xs:string" minOccurs="0" maxOccurs="1"/>
      <xs:element name="tl" type="xs:string" minOccurs="0" maxOccurs="1"/>
    </xs:sequence>
    <xs:attribute name="op" type="xs:string" use="required"/>
  </xs:complexType>
</xs:element>

<xs:simpleType name="emailAddress"> 
    <xs:restriction base="xs:string"> 
      <xs:pattern value="[^@]+@[^\.]+\..+"/> 
    </xs:restriction> 
</xs:simpleType> 

</xs:schema>
				\end{lstlisting}
				}
				
				\item[] \textbf{Messaggio Server -\textgreater~ Client:}{\\
				\textbf{Messaggio XML}\\
				\begin{lstlisting}
<ud op="add">
	<es>
		{Esito operazione}
	</es>
	<er>
		{Errore nell'operazione}
	</er>
</ud>
				\end{lstlisting}
				\textbf{Validazione XSD}\\
				\begin{lstlisting}
<xs:schema xmlns:xs="http://www.w3.org/2001/XMLSchema"
targetNamespace="http://www.mytalk.com"
xmlns="http://www.mytalk.com"
elementFormDefault="qualified">

<xs:element name="ud">
  <xs:complexType>
    <xs:sequence>
      <xs:element name="es" type="xs:string" minOccurs="1" maxOccurs="1"/>
      <xs:element name="er" type="xs:string" minOccurs="1" maxOccurs="1"/>
    </xs:sequence>
    <xs:attribute name="op" type="xs:string" use="required"/>
  </xs:complexType>
</xs:element>

</xs:schema>
				\end{lstlisting}
				Il valore contenuto nell'elemento \texttt{es} deve essere:
				\begin{itemize}
					\item \texttt{true}: modifica effettuata con successo;
					\item \texttt{false}: modifica non riuscita.
				\end{itemize}
				Nel caso di esito negativo dell'operazione, l'elemento \texttt{er} può contenere un messaggio informativo riguardanti le cause d'insuccesso.
				}
		\end{itemize}
	}%modifica dati utente
	
	\subsubsection{Eliminazione utente}{
	\label{opUDel}
		\begin{itemize}
			\item[] \textbf{Funzione:}{\\
				invia al Server le informazioni per l'eliminazione di un specifico utente.
				}
			
			\item[] \textbf{Messaggio Client -\textgreater~ Server:}{\\
			\textbf{Messaggio XML}\\
				\begin{lstlisting}
<ud op="dlt">
	<m>
		{E-mail utente}
	</m>
</ud>
				\end{lstlisting}
				\textbf{Validazione XSD}\\
				\begin{lstlisting}
<xs:schema xmlns:xs="http://www.w3.org/2001/XMLSchema"
targetNamespace="http://www.mytalk.com"
xmlns="http://www.mytalk.com"
elementFormDefault="qualified">

<xs:element name="ud">
  <xs:complexType>
    <xs:sequence>
      <xs:element name="m" type="emailAddress" minOccurs="1" maxOccurs="1"/>
    </xs:sequence>
    <xs:attribute name="op" type="xs:string" use="required"/>
  </xs:complexType>
</xs:element>

<xs:simpleType name="emailAddress"> 
    <xs:restriction base="xs:string"> 
      <xs:pattern value="[^@]+@[^\.]+\..+"/> 
    </xs:restriction> 
</xs:simpleType> 

</xs:schema>
				\end{lstlisting}
				}
				
				\item[] \textbf{Messaggio Server -\textgreater~ Client:}{\\
				\textbf{Messaggio XML}\\
				\begin{lstlisting}
<ud op="add">
	<es>
		{Esito operazione}
	</es>
	<er>
		{Errore nell'operazione}
	</er>
</ud>
				\end{lstlisting}
				\textbf{Validazione XSD}\\
				\begin{lstlisting}
<xs:schema xmlns:xs="http://www.w3.org/2001/XMLSchema"
targetNamespace="http://www.mytalk.com"
xmlns="http://www.mytalk.com"
elementFormDefault="qualified">

<xs:element name="ud">
  <xs:complexType>
    <xs:sequence>
      <xs:element name="es" type="xs:string" minOccurs="1" maxOccurs="1"/>
      <xs:element name="er" type="xs:string" minOccurs="1" maxOccurs="1"/>
    </xs:sequence>
    <xs:attribute name="op" type="xs:string" use="required"/>
  </xs:complexType>
</xs:element>

</xs:schema>
				\end{lstlisting}
				Il valore contenuto nell'elemento \texttt{es} deve essere:
				\begin{itemize}
					\item \texttt{true}: eliminazione effettuata con successo;
					\item \texttt{false}: eliminazione non riuscita.
				\end{itemize}
				Nel caso di esito negativo dell'operazione, l'elemento \texttt{er} può contenere un messaggio informativo riguardanti le cause d'insuccesso.
				}
		\end{itemize}
	}%Eliminazione utente
	
	
	\subsubsection{Lista utenti}{
	\label{opCUList}
		\begin{itemize}
			\item[] \textbf{Funzione:}{\\
				invia al Server la richiesta per ottenere la lista di tutti gli utenti registrati.
				}
			
			\item[] \textbf{Messaggio Client -\textgreater~ Server:}{\\
			\textbf{Messaggio XML}\\
				\begin{lstlisting}
<c op="userList">
</c>
				\end{lstlisting}
				\textbf{Validazione XSD}\\
				\begin{lstlisting}
<xs:schema xmlns:xs="http://www.w3.org/2001/XMLSchema"
targetNamespace="http://www.mytalk.com"
xmlns="http://www.mytalk.com"
elementFormDefault="qualified">


<xs:element name="c">
  <xs:complexType>
    <xs:simpleContent>
      <xs:extension base="xs:string">
        <xs:attribute name="op" type="xs:string" use="required"/>
      </xs:extension>
    </xs:simpleContent>
  </xs:complexType>
</xs:element>


</xs:schema>
				\end{lstlisting}
				}
				
				\item[] \textbf{Messaggio Server -\textgreater~ Client:}{\\
				\textbf{Messaggio XML}\\
				\begin{lstlisting}
<c op="userList">
	<ud>
		<m>
			{E-mail utente}
		</m>
		<nm>
			{Nome utente}
		</nm>
		<sn>
			{Cognome utente}
		</sn>
		<ol>
			{Utente online}
		</ol>
	</ud>
</c>
				\end{lstlisting}
				\textbf{Validazione XSD}\\
				\begin{lstlisting}
<xs:schema xmlns:xs="http://www.w3.org/2001/XMLSchema"
targetNamespace="http://www.mytalk.com"
xmlns="http://www.mytalk.com"
elementFormDefault="qualified">

<xs:element name="c">
    <xs:complexType>
      <xs:sequence>
        <xs:element name="ud" minOccurs="0" maxOccurs="unbounded">
          <xs:complexType>
            <xs:sequence>
              <xs:element name="m" type="emailAddress" minOccurs="1" maxOccurs="1"/>
              <xs:element name="nm" type="xs:string" minOccurs="1" maxOccurs="1"/>
              <xs:element name="sn" type="xs:string" minOccurs="1" maxOccurs="1"/>
              <xs:element name="ol" type="xs:string" minOccurs="1" maxOccurs="1"/>
            </xs:sequence>
          </xs:complexType>
        </xs:element>
      </xs:sequence>
      <xs:attribute name="op" type="xs:string" use="required"/>
    </xs:complexType>
  </xs:element>

<xs:simpleType name="emailAddress"> 
    <xs:restriction base="xs:string"> 
      <xs:pattern value="[^@]+@[^\.]+\..+"/> 
    </xs:restriction> 
</xs:simpleType> 

</xs:schema>
				\end{lstlisting}
				}
		\end{itemize}
	}%Lista utenti
	
	\subsubsection{Negoziazione di chiamata}{
	\label{opCNeg}
		\begin{itemize}
			\item[] \textbf{Funzione:}{\\
				messaggio inviato al server da parte di un utente mittente con le informazioni necessarie per poter comunicare con un utente destinatario.
				}
			
			\item[] \textbf{Messaggio Client mittente -\textgreater~ Server:}{\\
			\textbf{Messaggio XML}\\
				\begin{lstlisting}
<c op="callNegotation" type="offer">
	<us>
		{E-mail utente mittente}
	</us>
	<ur>
		{E-mail utente destinatario}
	</ur>
</c>
				\end{lstlisting}
				\textbf{Validazione XSD}\\
				\begin{lstlisting}
<xs:schema xmlns:xs="http://www.w3.org/2001/XMLSchema"
targetNamespace="http://www.mytalk.com"
xmlns="http://www.mytalk.com"
elementFormDefault="qualified">

<xs:element name="c">
  <xs:complexType>
    <xs:sequence>
      <xs:element name="us" type="emailAddress" minOccurs="1" maxOccurs="1"/>
      <xs:element name="ur" type="emailAddress" minOccurs="1" maxOccurs="1"/>
    </xs:sequence>
    <xs:attribute name="op" type="xs:string" use="required"/>
    <xs:attribute name="type" type="xs:string" use="required"/>
  </xs:complexType>
</xs:element>

<xs:simpleType name="emailAddress"> 
    <xs:restriction base="xs:string"> 
      <xs:pattern value="[^@]+@[^\.]+\..+"/> 
    </xs:restriction> 
</xs:simpleType> 

</xs:schema>
				\end{lstlisting}
				}
				
			\item[] \textbf{Messaggio Client destinatario -\textgreater~ Server:}{\\
			\textbf{Messaggio XML}\\
				\begin{lstlisting}
<c op="callNegotation" type="offer">
	<us>
		{E-mail utente destinatario}
	</us>
	<ur>
		{E-mail utente mittente}
	</ur>
	<st>
		{Accettazione della chiamata}
	</st>
</c>
				\end{lstlisting}
				\textbf{Validazione XSD}\\
				\begin{lstlisting}
<xs:schema xmlns:xs="http://www.w3.org/2001/XMLSchema"
targetNamespace="http://www.mytalk.com"
xmlns="http://www.mytalk.com"
elementFormDefault="qualified">

<xs:element name="c">
  <xs:complexType>
    <xs:sequence>
      <xs:element name="us" type="emailAddress" minOccurs="1" maxOccurs="1"/>
      <xs:element name="ur" type="emailAddress" minOccurs="1" maxOccurs="1"/>
      <xs:element name="st" type="xs:string" minOccurs="1" maxOccurs="1"/>
    </xs:sequence>
    <xs:attribute name="op" type="xs:string" use="required"/>
    <xs:attribute name="type" type="xs:string" use="required"/>
  </xs:complexType>
</xs:element>

<xs:simpleType name="emailAddress"> 
    <xs:restriction base="xs:string"> 
      <xs:pattern value="[^@]+@[^\.]+\..+"/> 
    </xs:restriction> 
</xs:simpleType> 

</xs:schema>
				\end{lstlisting}
				Il valore contenuto nell'elemento \texttt{st} deve essere:
				\begin{itemize}
					\item \texttt{accept}: richiesta di comunicazione accettata da parte dell'utente destinatario;
					\item \texttt{refused}: richiesta di comunicazione rifiutata da parte dell'utente destinatario.
				\end{itemize}
				}
				
				\item[] \textbf{Messaggio Server -\textgreater~ Client mittente:}{\\
				\textbf{Messaggio XML}\\
				\begin{lstlisting}
<c op="callNegotation" type="answer">
	<us>
		{E-mail utente destinatario}
	</us>
	<ur>
		{E-mail utente mittente}
	</ur>
	<st>
		{Accettazione della chiamata}
	</st>
</c>
				\end{lstlisting}
				\textbf{Validazione XSD}\\
				\begin{lstlisting}
<xs:schema xmlns:xs="http://www.w3.org/2001/XMLSchema"
targetNamespace="http://www.mytalk.com"
xmlns="http://www.mytalk.com"
elementFormDefault="qualified">

<xs:element name="c">
  <xs:complexType>
    <xs:sequence>
      <xs:element name="us" type="emailAddress" minOccurs="1" maxOccurs="1"/>
      <xs:element name="ur" type="emailAddress" minOccurs="1" maxOccurs="1"/>
      <xs:element name="st" type="xs:string" minOccurs="1" maxOccurs="1"/>
    </xs:sequence>
    <xs:attribute name="op" type="xs:string" use="required"/>
    <xs:attribute name="type" type="xs:string" use="required"/>
  </xs:complexType>
</xs:element>

<xs:simpleType name="emailAddress"> 
    <xs:restriction base="xs:string"> 
      <xs:pattern value="[^@]+@[^\.]+\..+"/> 
    </xs:restriction> 
</xs:simpleType> 

</xs:schema>
				\end{lstlisting}
				Questo messaggio viene inviato dal Server solo nel caso in cui l'utente destinatario non sia attualmente online; in questo caso, il valore dell'elemento \texttt{st} deve essere \texttt{offline}.
				}
		\end{itemize}
	}%call Neg
	
	
	
	
	\subsubsection{Inserimento nuove statistiche di chiamata}{
	\label{opCInsStat}
		\begin{itemize}
			\item[] \textbf{Funzione:}{\\
				invia al Server le informazioni necessarie per inserire delle nuove statistiche di chiamata.
				}
			
			\item[] \textbf{Messaggio Client -\textgreater~ Server:}{\\
			\textbf{Messaggio XML}\\
				\begin{lstlisting}
<c op="add">
	<us>
		{E-mail utente mittente}
	</us>
	<ur>
		{E-mail utente destinatario}
	</ur>
	<pk>
		{Numero pacchetti trasmessi}
	</pk>
	<pkL>
		{Numero pacchetti persi}
	</pkL>
	<ss>
		{Numero di byte trasmessi}
	</ss>
	<ds>
		{Data d'inizio}
	</ds>
	<de>
		{Data di fine}
	</de>
	<sg>
		{Indice di gradimento dell'utente mittente}
	</sg>
	<rg>
		{Indice di gradimento dell'utente destinatario}
	</rg>
</c>
				\end{lstlisting}
				\textbf{Validazione XSD}\\
				\begin{lstlisting}
<xs:schema xmlns:xs="http://www.w3.org/2001/XMLSchema"
targetNamespace="http://www.mytalk.com"
xmlns="http://www.mytalk.com"
elementFormDefault="qualified">

<xs:element name="c">
  <xs:complexType>
    <xs:sequence>
      <xs:element name="us" type="emailAddress" minOccurs="1" maxOccurs="1"/>
      <xs:element name="ur" type="emailAddress" minOccurs="1" maxOccurs="1"/>
      <xs:element name="pk" type="xs:long" minOccurs="1" maxOccurs="1"/>
      <xs:element name="pkL" type="xs:long" minOccurs="1" maxOccurs="1"/>
      <xs:element name="ss" type="xs:long" minOccurs="1" maxOccurs="1"/>
      <xs:element name="ds" type="dateType" minOccurs="1" maxOccurs="1"/>
      <xs:element name="de" type="dateType" minOccurs="1" maxOccurs="1"/>
      <xs:element name="sg" type="grade" minOccurs="1" maxOccurs="1"/>
      <xs:element name="rg" type="grade" minOccurs="1" maxOccurs="1"/>
    </xs:sequence>
    <xs:attribute name="op" type="xs:string" use="required"/>
    <xs:attribute name="type" type="xs:string" use="required"/>
  </xs:complexType>
</xs:element>

<xs:simpleType name="emailAddress"> 
    <xs:restriction base="xs:string"> 
      <xs:pattern value="[^@]+@[^\.]+\..+"/> 
    </xs:restriction> 
</xs:simpleType>

<xs:simpleType name="dateType">
   <xs:restriction base="xs:string">
       <xs:pattern value="\d{14}"/>
   </xs:restriction>
</xs:simpleType>

<xs:simpleType name="grade">
   <xs:restriction base="xs:byte">
       <xs:minInclusive value="0"/>
       <xs:maxInclusive value="5"/>
   </xs:restriction>
</xs:simpleType>

</xs:schema>
				\end{lstlisting}
				}
				
				\item[] \textbf{Messaggio Server -\textgreater~ Client:}{\\
				\textbf{Messaggio XML}\\
				\begin{lstlisting}
<c op="add">
	<es>
		{Esito operazione}
	</es>
	<er>
		{Errore nell'operazione}
	</er>
</ud>
				\end{lstlisting}
				\textbf{Validazione XSD}\\
				\begin{lstlisting}
<xs:schema xmlns:xs="http://www.w3.org/2001/XMLSchema"
targetNamespace="http://www.mytalk.com"
xmlns="http://www.mytalk.com"
elementFormDefault="qualified">

<xs:element name="c">
  <xs:complexType>
    <xs:sequence>
      <xs:element name="es" type="xs:string" minOccurs="1" maxOccurs="1"/>
      <xs:element name="er" type="xs:string" minOccurs="1" maxOccurs="1"/>
    </xs:sequence>
    <xs:attribute name="op" type="xs:string" use="required"/>
  </xs:complexType>
</xs:element>

</xs:schema>
				\end{lstlisting}
				Il valore contenuto nell'elemento \texttt{es} deve essere:
				\begin{itemize}
					\item \texttt{true}: statistiche inserite con successo;
					\item \texttt{false}: inserimento nuove statistiche non riuscito.
				\end{itemize}
				Nel caso di esito negativo dell'operazione, l'elemento \texttt{er} può contenere un messaggio informativo riguardanti le cause d'insuccesso.
				}
		\end{itemize}
	}%Statistiche chiamata
	
	\subsubsection{Verifica esistenza utente}{
	\label{opCEus}
		\begin{itemize}
			\item[] \textbf{Funzione:}{\\
				invia al Server le informazioni necessarie per verificare se esiste un utente registrato che rispetti i vincoli fissati.
				}
			
			\item[] \textbf{Messaggio Client -\textgreater~ Server:}{\\
			\textbf{Messaggio XML}\\
				\begin{lstlisting}
<c op="find">
	<ip>
		{Valore indirizzo IP}
	</ip>
</c>
				\end{lstlisting}
				\textbf{Validazione XSD}\\
				\begin{lstlisting}
<xs:schema xmlns:xs="http://www.w3.org/2001/XMLSchema"
targetNamespace="http://www.mytalk.com"
xmlns="http://www.mytalk.com"
elementFormDefault="qualified">

<xs:element name="c">
  <xs:complexType>
    <xs:sequence>
      <xs:element name="ip" type="xs:string" minOccurs="1" maxOccurs="1"/>
    </xs:sequence>
    <xs:attribute name="op" type="xs:string" use="required"/>
  </xs:complexType>
</xs:element>

</xs:schema>
				\end{lstlisting}
				}
				
				\item[] \textbf{Messaggio Server -\textgreater~ Client:}{\\
				\textbf{Messaggio XML}\\
				\begin{lstlisting}
<c op="find">
	<es>
		{Esito operazione}
	</es>
	<m>
		{E-mail utente}
	</m>
</c>
				\end{lstlisting}
				\textbf{Validazione XSD}\\
				\begin{lstlisting}
<xs:schema xmlns:xs="http://www.w3.org/2001/XMLSchema"
targetNamespace="http://www.mytalk.com"
xmlns="http://www.mytalk.com"
elementFormDefault="qualified">

<xs:element name="c">
  <xs:complexType>
    <xs:sequence>
      <xs:element name="es" type="xs:string" minOccurs="1" maxOccurs="1"/>
      <xs:element name="m" type="emailAddress" minOccurs="0" maxOccurs="1"/>
    </xs:sequence>
    <xs:attribute name="op" type="xs:string" use="required"/>
  </xs:complexType>
</xs:element>

<xs:simpleType name="emailAddress"> 
    <xs:restriction base="xs:string"> 
      <xs:pattern value="[^@]+@[^\.]+\..+"/> 
    </xs:restriction> 
</xs:simpleType>

</xs:schema>
				\end{lstlisting}
				Il valore contenuto nell'elemento \texttt{es} deve essere:
				\begin{itemize}
					\item \texttt{true}: la ricerca ha prodotto qualche risultato;
					\item \texttt{false}: la ricerca non ha prodotto alcun risultato.
				\end{itemize}
				}
		\end{itemize}
	}%esistenza utente
	
	\subsubsection{Ricerca valori statistici}{
	\label{opCValStat}
		\begin{itemize}
			\item[] \textbf{Funzione:}{\\
				invia al Server le informazioni necessarie per ottenere delle particolari informazioni relative alle statistiche di comunicazione.
				}
			
			\item[] \textbf{Messaggio Client -\textgreater~ Server:}{\\
			\textbf{Messaggio XML}\\
				\begin{lstlisting}
<c op="stat" t="{Arco temporale in secondi}">
	<m>
		{E-mail utente}
	</m>
	<g>
		{Indice di gradimento}
	</g>
	<ds>
		{Data d'inizio}
	</ds>
</c>
				\end{lstlisting}
				\textbf{Validazione XSD}\\
				\begin{lstlisting}
<xs:schema xmlns:xs="http://www.w3.org/2001/XMLSchema"
targetNamespace="http://www.mytalk.com"
xmlns="http://www.mytalk.com"
elementFormDefault="qualified">

<xs:element name="c">
  <xs:complexType>
    <xs:sequence>
      <xs:element name="m" type="emailAddress" minOccurs="0" maxOccurs="1"/>
      <xs:element name="g" type="grade" minOccurs="0" maxOccurs="1"/>
      <xs:element name="ds" type="dateType" minOccurs="0" maxOccurs="1"/>
    </xs:sequence>
    <xs:attribute name="op" type="xs:string" use="required"/>
    <xs:attribute name="t" type="xs:nonNegativeInteger" use="required"/>
  </xs:complexType>
</xs:element>

<xs:simpleType name="emailAddress"> 
    <xs:restriction base="xs:string"> 
      <xs:pattern value="[^@]+@[^\.]+\..+"/> 
    </xs:restriction> 
</xs:simpleType>

<xs:simpleType name="dateType">
   <xs:restriction base="xs:string">
       <xs:pattern value="\d{14}"/>
   </xs:restriction>
</xs:simpleType>

<xs:simpleType name="grade">
   <xs:restriction base="xs:byte">
       <xs:minInclusive value="0"/>
       <xs:maxInclusive value="5"/>
   </xs:restriction>
</xs:simpleType>

</xs:schema>
				\end{lstlisting}
				Almeno uno degli elementi interni deve essere presente.
				}
				
				\item[] \textbf{Messaggio Server -\textgreater~ Client:}{\\
				\textbf{Messaggio XML}\\
				\begin{lstlisting}
<c op="stat" t="{Arco temporale in secondi}">
	<r>
		<us>
			{E-mail utente mittente}
		</us>
		<ur>
			{E-mail utente destinatario}
		</ur>
		<pk>
			{Numero pacchetti trasmessi}
		</pk>
		<pkL>
			{Numero pacchetti persi}
		</pkL>
		<ss>
			{Numero di byte trasmessi}
		</ss>
		<ds>
			{Data d'inizio}
		</ds>
		<de>
			{Data di fine}
		</de>
		<sg>
			{Indice di gradimento dell'utente mittente}
		</sg>
		<rg>
			{Indice di gradimento dell'utente destinatario}
		</rg>
	</r>
</c>
				\end{lstlisting}
				\textbf{Validazione XSD}\\
				\begin{lstlisting}
<xs:schema xmlns:xs="http://www.w3.org/2001/XMLSchema"
targetNamespace="http://www.mytalk.com"
xmlns="http://www.mytalk.com"
elementFormDefault="qualified">

<xs:element name="c">
  <xs:complexType>
    <xs:sequence>
      <xs:element name="r" minOccurs="0" maxOccurs="unbounded">
        <xs:complexType>
          <xs:sequence>
            <xs:element name="us" type="emailAddress" minOccurs="1" maxOccurs="1"/>
            <xs:element name="ur" type="emailAddress" minOccurs="1" maxOccurs="1"/>
            <xs:element name="pk" type="xs:long" minOccurs="1" maxOccurs="1"/>
            <xs:element name="pkL" type="xs:long" minOccurs="1" maxOccurs="1"/>
            <xs:element name="ss" type="xs:long" minOccurs="1" maxOccurs="1"/>
            <xs:element name="ds" type="dateType" minOccurs="1" maxOccurs="1"/>
            <xs:element name="de" type="dateType" minOccurs="1" maxOccurs="1"/>
            <xs:element name="sg" type="grade" minOccurs="1" maxOccurs="1"/>
            <xs:element name="rg" type="grade" minOccurs="1" maxOccurs="1"/>
          </xs:sequence>
        </xs:complexType>
      </xs:element>
    </xs:sequence>
    <xs:attribute name="op" type="xs:string" use="required"/>
    <xs:attribute name="t" type="xs:nonNegativeInteger" use="required"/>
  </xs:complexType>
</xs:element>

<xs:simpleType name="emailAddress"> 
  <xs:restriction base="xs:string"> 
    <xs:pattern value="[^@]+@[^\.]+\..+"/> 
  </xs:restriction> 
</xs:simpleType>

<xs:simpleType name="dateType">
  <xs:restriction base="xs:string">
    <xs:pattern value="\d{14}"/>
  </xs:restriction>
</xs:simpleType>

<xs:simpleType name="grade">
  <xs:restriction base="xs:byte">
    <xs:minInclusive value="0"/>
    <xs:maxInclusive value="5"/>
  </xs:restriction>
</xs:simpleType>

</xs:schema>
				\end{lstlisting}
				}
		\end{itemize}
	}%valori statistici
	
	\subsubsection{Aggiunta messaggio di segreteria}{
	\label{opAM}
		\begin{itemize}
			\item[] \textbf{Funzione:}{\\
				invia al Server le informazioni necessarie per lasciare un messaggio testuale in segreteria da far pervenire all'utente destinatario.
				}
			
			\item[] \textbf{Messaggio Client -\textgreater~ Server:}{\\
			\textbf{Messaggio XML}\\
				\begin{lstlisting}
<c op="am">
	<us>
		{E-mail utente mittente}	
	</us>
	<ur>
		{E-mail utente destinatario}
	</ur>
	<ds>
		{Data invio messaggio}
	</ds>
	<msg>
		{Messaggio testuale}
	</msg>
</c>
				\end{lstlisting}
				\textbf{Validazione XSD}\\
				\begin{lstlisting}
<xs:schema xmlns:xs="http://www.w3.org/2001/XMLSchema"
targetNamespace="http://www.mytalk.com"
xmlns="http://www.mytalk.com"
elementFormDefault="qualified">

<xs:element name="c">
  <xs:complexType>
    <xs:sequence>
      <xs:element name="us" type="emailAddress" minOccurs="1" maxOccurs="1"/>
      <xs:element name="ur" type="emailAddress" minOccurs="1" maxOccurs="1"/>
      <xs:element name="ds" type="dateType" minOccurs="1" maxOccurs="1"/>
      <xs:element name="msg" type="xs:string" minOccurs="1" maxOccurs="1"/>
    </xs:sequence>
    <xs:attribute name="op" type="xs:string" use="required"/>
  </xs:complexType>
</xs:element>

<xs:simpleType name="emailAddress"> 
    <xs:restriction base="xs:string"> 
      <xs:pattern value="[^@]+@[^\.]+\..+"/> 
    </xs:restriction> 
</xs:simpleType>

<xs:simpleType name="dateType">
   <xs:restriction base="xs:string">
       <xs:pattern value="\d{14}"/>
   </xs:restriction>
</xs:simpleType>

</xs:schema>
				\end{lstlisting}
				}
				
				\item[] \textbf{Messaggio Server -\textgreater~ Client:}{\\
				\textbf{Messaggio XML}\\
				\begin{lstlisting}
<c op="am">
	<ur>
		{E-mail utente destinatario}
	</ur>
	<msg>
		<us>
			{E-mail utente mittente}
		</us>
		<d>
			{Data d'inizio}
		</d>
		<text>
			{Data d'inizio}
		</text>
	</msg>
</c>
				\end{lstlisting}
				\textbf{Validazione XSD}\\
				\begin{lstlisting}
<xs:schema xmlns:xs="http://www.w3.org/2001/XMLSchema"
targetNamespace="http://www.mytalk.com"
xmlns="http://www.mytalk.com"
elementFormDefault="qualified">

<xs:element name="c">
    <xs:complexType>
      <xs:sequence>
        <xs:element type="emailAddress" name="ur"/>
        <xs:element name="msg" minOccurs="1" maxOccurs="unbounded">
          <xs:complexType>
            <xs:sequence>
              <xs:element type="emailAddress" name="us"/>
              <xs:element type="dateType" name="d"/>
              <xs:element type="xs:string" name="text"/>
            </xs:sequence>
          </xs:complexType>
        </xs:element>
      </xs:sequence>
      <xs:attribute type="xs:string" name="op"/>
    </xs:complexType>
  </xs:element>

<xs:simpleType name="emailAddress"> 
    <xs:restriction base="xs:string"> 
      <xs:pattern value="[^@]+@[^\.]+\..+"/> 
    </xs:restriction> 
</xs:simpleType>

<xs:simpleType name="dateType">
   <xs:restriction base="xs:string">
       <xs:pattern value="\d{14}"/>
   </xs:restriction>
</xs:simpleType>

</xs:schema>
				\end{lstlisting}
				}
		\end{itemize}
	}%Agg messaggio segreteria
	
		\subsubsection{Rimozione messaggi in segreteria}{
	\label{opAMD}
		\begin{itemize}
			\item[] \textbf{Funzione:}{\\
				invia al Server le informazioni necessarie eliminare i messaggi presenti in segreteria.
				}
			
			\item[] \textbf{Messaggio Client -\textgreater~ Server:}{\\
			\textbf{Messaggio XML}\\
				\begin{lstlisting}
<c op="amd">
	<m>
		{E-mail utente}
	</m>
</c>
				\end{lstlisting}
				\textbf{Validazione XSD}\\
				\begin{lstlisting}
<xs:schema xmlns:xs="http://www.w3.org/2001/XMLSchema"
targetNamespace="http://www.mytalk.com"
xmlns="http://www.mytalk.com"
elementFormDefault="qualified">

<xs:element name="c">
  <xs:complexType>
    <xs:sequence>
      <xs:element name="m" type="emailAddress" minOccurs="1" maxOccurs="1"/>
    </xs:sequence>
    <xs:attribute name="op" type="xs:string" use="required"/>
  </xs:complexType>
</xs:element>

<xs:simpleType name="emailAddress"> 
    <xs:restriction base="xs:string"> 
      <xs:pattern value="[^@]+@[^\.]+\..+"/> 
    </xs:restriction> 
</xs:simpleType>

</xs:schema>
				\end{lstlisting}
				}
				
				\item[] \textbf{Messaggio Server -\textgreater~ Client:}\\
				Nessuno.
		\end{itemize}
	}%Agg messaggio segreteria
}

\subsection{Comunicazioni Client-Client}{

	\subsubsection{Scambio informazioni di chiamata}{
	\label{opCExg}
		\begin{itemize}
			\item[] \textbf{Funzione:}{\\
				messaggio inviato da parte di un utente mittente con le informazioni necessarie per poter scambiare informazioni utili per poter instaurare una comunicazione con un utente destinatario.\\
				Il server ha la funzione di far rimbalzare il messaggio tra i due client.
				}
			
			\item[] \textbf{Messaggio Client Mittente -\textgreater~ Server:}{\\
			\textbf{Messaggio XML}\\
				\begin{lstlisting}
<c op="callExchange" type="offerDescription">
	<us>
		{E-mail utente mittente}
	</us>
	<ur>
		{E-mail utente destinatario}
	</ur>
	<dsc>
		{Contenuto informativo}
	</dsc>
</c>
				\end{lstlisting}
				\textbf{Validazione XSD}\\
				\begin{lstlisting}
<xs:schema xmlns:xs="http://www.w3.org/2001/XMLSchema"
targetNamespace="http://www.mytalk.com"
xmlns="http://www.mytalk.com"
elementFormDefault="qualified">

<xs:element name="c">
  <xs:complexType>
    <xs:sequence>
      <xs:element name="us" type="emailAddress" minOccurs="1" maxOccurs="1"/>
      <xs:element name="ur" type="emailAddress" minOccurs="1" maxOccurs="1"/>
      <xs:element name="dsc" type="xs:string" minOccurs="1" maxOccurs="1"/>
    </xs:sequence>
    <xs:attribute name="op" type="xs:string" use="required"/>
    <xs:attribute name="type" type="xs:string" use="required"/>
  </xs:complexType>
</xs:element>

<xs:simpleType name="emailAddress"> 
    <xs:restriction base="xs:string"> 
      <xs:pattern value="[^@]+@[^\.]+\..+"/> 
    </xs:restriction> 
</xs:simpleType> 

</xs:schema>
				\end{lstlisting}
				}
				
		\item[] \textbf{Messaggio Client Destinatario -\textgreater~ Server:}{\\
			\textbf{Messaggio XML}\\
				\begin{lstlisting}
<c op="callExchange" type="offerDescription">
	<us>
		{E-mail utente destinatario}
	</us>
	<ur>
		{E-mail utente mittente}
	</ur>
	<dsc>
		{Contenuto informativo}
	</dsc>
</c>
				\end{lstlisting}
				\textbf{Validazione XSD}\\
				\begin{lstlisting}
<xs:schema xmlns:xs="http://www.w3.org/2001/XMLSchema"
targetNamespace="http://www.mytalk.com"
xmlns="http://www.mytalk.com"
elementFormDefault="qualified">

<xs:element name="c">
  <xs:complexType>
    <xs:sequence>
      <xs:element name="us" type="emailAddress" minOccurs="1" maxOccurs="1"/>
      <xs:element name="ur" type="emailAddress" minOccurs="1" maxOccurs="1"/>
      <xs:element name="dsc" type="xs:string" minOccurs="1" maxOccurs="1"/>
    </xs:sequence>
    <xs:attribute name="op" type="xs:string" use="required"/>
    <xs:attribute name="type" type="xs:string" use="required"/>
  </xs:complexType>
</xs:element>

<xs:simpleType name="emailAddress"> 
    <xs:restriction base="xs:string"> 
      <xs:pattern value="[^@]+@[^\.]+\..+"/> 
    </xs:restriction> 
</xs:simpleType> 

</xs:schema>
				\end{lstlisting}
				}
		\end{itemize}
	}%scambio info chiamata

}
