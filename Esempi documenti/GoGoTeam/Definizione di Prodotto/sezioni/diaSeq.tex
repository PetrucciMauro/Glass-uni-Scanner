\section{Digrammi di sequenza}

\begin{sloppypar}

\subsection{Effettuazione chiamata}
\begin{center}
				\begin{figure}[h!tbp]
					\centering
					\includegraphics[scale=0.425,angle=90]{\docsImg chiamataUtente.pdf}
				\caption{DS 1 - La procedura di effettuazione di una chiamata.}	
				\end{figure}
			\end{center}
\noindent \textbf{Precondizione: }l'utente è autenticato e inserisce un nome utente esistente all'interno del sistema.\\

\textbf{Descrizione: }l'utente inserisce lo username dell'utente che vuole chiamare tramite un form o selezionandolo dalla lista degli utenti registrati al server\g. Vengono quindi eseguiti i seguenti passaggi:
\begin{enumerate}
	\item viene invocato il metodo della classe \texttt{CommunicationLogic call(String utente)};
	\item il metodo invoca a sua volta il metodo \texttt{inizialize()} della classe \texttt{PeerConnection} che invoca il metodo omonimo nella classe wrapper \texttt{WebRTCAdaptee};
	\item il metodo \texttt{inizialize()} inizializza l'oggetto \texttt{pc} con l'oggetto di tipo \texttt{webkitPeerConnection} appena creato;
	\item una volta ritornato dal metodo \texttt{inizialize()} la classe \texttt{CommunicationLogic} richiama il metodo \texttt{addStream(MediaStream stream)} dell'oggetto \texttt{connessione} di tipo \texttt{PeerConnection} con lo scopo di aggiungere lo stream video alla connessione, il metodo richiama l'omonimo metodo \texttt{addStream(MediaStream stream)} della classe \texttt{WebRTCAdaptee} che aggiunge lo stream locale alla connessione.
	\item Viene successivamente invocato il metodo \texttt{createOffer()} della classe \texttt{PeerConnection} che richiama a sua volta il metodo \texttt{offer()} della classe \texttt{WebRTCAdaptee}. L'invocazione del metodo fa si che venga creato un oggetto delle API WebRTC di tipo \texttt{RTCSessionDescription} e che questo oggetto venga impostato all'interno dell'oggetto \texttt{pc} di tipo \texttt{webkitRTCPeerConnection} come descrizione della sessione locale.
	\item Viene richiamato il metodo \texttt{getOffer()} della classe \texttt{CommunicationLogic} dal quale ritorna l'oggetto di tipo \texttt{RTCSessionDescription} creato in precedenza codificato in formato JSON.
	\item Viene invocato il metodo dell'oggetto \texttt{webSocketUser} di tipo \texttt{WebSocketUser} \texttt{call(String utente, String offer)} con lo scopo di inviare, tramite il server, l'oggetto \texttt{RTCSessionDescription} all'utente che si vuole chiamare.
\end{enumerate}
\textbf{Postcondizione: } l'utente indicato dall'utente autenticato è stato chiamato e gli è stato inviato l'oggetto di tipo \texttt{RTCsessionDescription} codificato in formato JSON.
\newpage


\subsection{Ricezione chiamata}
\begin{center}
				\begin{figure}[h!tbp]
					\centering
					\includegraphics[scale=0.375,angle=90]{\docsImg  rispostaChiamante.pdf}
				\caption{DS 2 - La procedura di notifica di una chiamata in entrata.}	
				\end{figure}
			\end{center}

\noindent \textbf{Precondizione: }l'utente è autenticato, non ci sono comunicazioni in atto.\\
\textbf{Descrizione: }l'oggetto di tipo \texttt{WebSocket} riceve un messaggio in entrata che indica che un altro utente vuole comunicare, vengono quindi eseguiti i seguenti passaggi:
\begin{enumerate}
	\item Il messaggio viene interpretato dal \texttt{WebSocket}, il quale rileva che si tratta di un messaggio che richiede di instaurare una chiamata, estrae quindi dal messaggio il nome dell'utente che ha chiamato e l'oggetto di tipo \texttt{RTCSessionDescription} in formato JSON\g .
	\item Vengono invocati i metodi \texttt{notifyCall(String utente)} della classe \texttt{UpdateViewLogic} passandogli come parametro il nome dell'utente che sta chiamando e \texttt{callEnter(String offer)} della classe \texttt{CommunicationLogic} passandogli come parametro l'oggetto \texttt{RTCSessionDescription} estratto dal messaggio ricevuto.
	\item Il metodo \texttt{notifyCall(String utente)} precedentemente invocato notifica all'interfaccia grafica la chiamata entrante.
	\item Il metodo \texttt{callEnter(String offer)} precedentemente invocato richiama il metodo \texttt{inizialize()} dell'oggetto di tipo \texttt{PeerConnection} che a sua volta richiama l'omonimo metodo della classe \texttt{WebRTCAdaptee} che inizializza il campo dati \texttt{pc} con un nuovo oggetto di tipo \texttt{webkitRTCPeerConnection}.
	\item Viene creato un oggetto di tipo \texttt{RTCSessionDescription} tramite l'invocazione del metodo \texttt{answer(String offer, MediaStream streamLocale)} che richiama, passando tramite opportuni metodi delle classi \texttt{PeerConnection e WebRTCAdaptee}, i metodi \texttt{addStream(MediaStream stream)}, \texttt{setRemoteDescription(RTCSessionDescription offer)} e \texttt{createAnswer()} delle WebRTC. La chiamata dei metodi delle WebRTC aggiunge la descrizione remota e lo stream locale alla connessione e crea un oggetto da inviare all'utente remoto di tipo \texttt{RTCSessionDescription}.
	\item Subito dopo la creazione dell'oggetto \texttt{RTCSessionDescription} vengono prodotti degli oggetti di tipo \texttt{RTCIceCandidate} che vengono salvati all'interno dell'oggetto in formato JSON.
\end{enumerate}
\textbf{Postcondizione: } all'utente autenticato è stato notificata la proposta di un altro utente di comunicare, è stata registrata la sessione remota ed è stata prodotta la sessione locale.
\newpage



\subsection{Accettazione chiamata}
\begin{center}
				\begin{figure}[h!tbp]
					\centering
					\includegraphics[scale=0.6,angle=90]{\docsImg  accettazioneChiamata.pdf}
				\caption{DS 3 - La procedura di accettazione di una chiamata in entrata.}	
				\end{figure}
			\end{center}
\noindent \textbf{Precondizione: }l'utente è autenticato, una chiamata gli è stata appena notificata e viene accettata.\\
\textbf{Descrizione: }l'utente autenticato accetta la chiamata in arrivo da un utente remoto, vengono eseguiti i seguenti passi:
\begin{enumerate}
	\item Viene invocato il metodo della classe \texttt{CommunicationLogic accept(String utente)} passandogli.
	\item Viene recuperata dalla classe \texttt{CommunicationLogic} tramite il metodo \texttt{getAnswer()} l'oggetto \texttt{RTCSessionDescription} che rappresenta poi la descrizione di sessione remota per l'utente remoto.
	\item Viene invocato il metodo \texttt{accept(String utente, String answer)} della classe \texttt{WebSocketUser} al fine di inviare all'utente che ha chiamato la descrizione della sessione.
	\end{enumerate}
\textbf{Postcondizione: } l'oggetto \texttt{RTCSessionDescription} è stato inviato all'utente che ha chiamato tramite \texttt{WebSocket}.
\newpage



\subsection{Chiamata accettata}
\begin{center}
				\begin{figure}[h!tbp]
					\centering
					\includegraphics[scale=0.575,angle=90]{\docsImg  chiamataAccettatav2.pdf}
				\caption{DS 4 - La procedura di scambio degli IceCandidate tra gli utenti durante l'attività di chiamata.}	
				\end{figure}
			\end{center}
\noindent \textbf{Precondizione: }l'utente è autenticato ed una chiamata precedentemente effettuata è stata accettata.\\
\textbf{Descrizione: }l'oggetto di tipo \texttt{WebSocket} riceve un messaggio in entrata che indica che l'utente precedentemente chiamato ha accettato la chiamata, viene estratto dal messaggio ricevuto il nome dell'utente chiamante e l'oggetto di tipo \texttt{RTCSessionDescription}, vengono quindi eseguiti i seguenti passaggi;
\begin{enumerate}
	\item Viene invocato il metodo della classe \texttt{CommunicationLogic acceptedCall(String utente, String offer)} passandogli come parametri i dati estratti dal messaggio.
	\item Viene impostata la descrizione remota rappresentata dall'oggetto \texttt{RTCSessionDescription} tramite una chiamata al metodo dell'oggetto \texttt{webkitPeerConnection} interno alla classe \texttt{WebRTCAdaptee}.
	\item Vengono estratti gli oggetti \texttt{RTCIceCandidate} in formato JSON creati in precedenza.
	\item Vengono inviati gli oggetti \texttt{RTCIceCandidate} tramite \texttt{WebSocket},richiamando il metodo \texttt{offer(String mittente, String candidate)}.
\end{enumerate}
\textbf{Postcondizione: } gli oggetti \texttt{RTCIceCandidate} sono stati inviati all'utente che ha accettato la chiamata.

\newpage

\subsection{Scambio candidati - utente chiamato}
\begin{center}
				\begin{figure}[h!tbp]
					\centering
					\includegraphics[scale=0.4,angle=90]{\docsImg  chiamataAccettatav3.pdf}
				\caption{DS 5 - La procedura per l'impostazione e l'invio degli IceCandidate tra gli utenti durante l'attività di chiamata.}	
				\end{figure}
			\end{center}
\noindent \textbf{Precondizione: }l'utente è autenticato ed una chiamata precedentemente ricevuta è stata accettata.\\
\textbf{Descrizione: }l'oggetto di tipo \texttt{WebSocket} riceve un messaggio in entrata. Il messaggio proviene dall'utente che ha precedentemente chiamato e la cui chiamata è già stata accettata, tale messaggio contiene in formato JSON\g~ gli oggetti di tipo \texttt{RTCIceCandidate} appartenenti all'utente che ha inviato il messaggio. Vengono quindi eseguiti i seguenti passaggi:
\begin{enumerate}
	\item Viene invocato il metodo della classe \texttt{CommunicationLogic receivedSession(String utente, Vector<String> candidate)} passandogli come parametri i dati estratti dal messaggio.
	\item Vengono aggiunti gli oggetti di tipo \texttt{RTCIceCandidate} uno alla volta, richiamando ripetutamente il metodo della classe \texttt{PeerConnection addCandidate(String candidate)}, che richiama a sua volta il metodo omonimo della classe \texttt{WebRTCAdaptee} il quale a sua volta richiama il metodo dell'oggetto \texttt{pc} di tipo \texttt{webkitRTCPeerConnection addCandidate(RTCIceCandidate)}.
	\item Vengono estratti gli oggetti \texttt{RTCIceCandidate} in formato JSON creati in precedenza.
	\item Vengono inviati gli oggetti \texttt{RTCIceCandidate} tramite \texttt{WebSocket}, richiamando il metodo \texttt{answer(String mittente, String candidate)}.
	\item Viene reperito l'url remoto creato all'impostazione della sessione remota.
	\item Viene impostato all'interno dell'interfaccia grafica l'url remoto invocando il metodo \texttt{callActive(String url)}.
\end{enumerate}
\textbf{Postcondizione: } gli oggetti \texttt{RTCIceCandidate} sono stati inviati all'utente che ha accettato la chiamata, l'url remoto è stato impostato.

\newpage

\subsection{Scambio candidati - utente chiamante}
\begin{center}
				\begin{figure}[h!tbp]
					\centering
					\includegraphics[scale=0.375,angle=90]{\docsImg  chiamataAccettatav4.pdf}
				\caption{DS 6 - La procedura per l'impostazione degli IceCandidate dell'utente che ha ricevuto la chiamata.}	
				\end{figure}
			\end{center}
\noindent \textbf{Precondizione: }l'utente è autenticato, una chiamata precedentemente effettuata è stata accettata.\\
\textbf{Descrizione: }l'oggetto di tipo \texttt{WebSocket} riceve un messaggio in entrata dall'utente che è stato precedentemente chiamato e che ha accettato la chiamata ed ha inviato in formato JSON\g~ i propri oggetti  \texttt{RTCIceCandidate}. Vengono quindi eseguiti i seguenti passaggi:
\begin{enumerate}
	\item Viene invocato il metodo della classe \texttt{CommunicationLogic receivedSessionPerformed(String utente, Vector<String> candidate)} passandogli come parametri i dati estratti dal messaggio.
	\item Vengono aggiunti gli oggetti di tipo \texttt{RTCIceCandidate} uno alla volta, richiamando ripetutamente il metodo della classe \texttt{PeerConnection addCandidate(String candidate)}, che richiama a sua volta il metodo omonimo della classe \texttt{WebRTCAdaptee} il quale a sua volta richiama il metodo dell'oggetto \texttt{pc} di tipo \texttt{webkitRTCPeerConnection addCandidate(RTCIceCandidate)}.
	\item Viene reperito l'url remoto creato all'impostazione della sessione remota.
	\item Viene impostato all'interno dell'interfaccia grafica l'url remoto invocando il metodo \texttt{callActive(String url)}.
\end{enumerate}
\textbf{Postcondizione: } l'url remoto è stato impostato, la chiamata ha avuto inizio.

\end{sloppypar}