\section{Standard di progetto}{
    \subsection{Standard di progettazione architetturale}	{
	Ciascun diagramma UML\g~ inserito nel presente documento è realizzato in conformità con lo standard UML 2.x. Per ulteriori informazioni sugli standard di progettazione architetturale adottati si rimanda ai documenti di Specifica Tecnica (allegato \textit{\SpecificaTecnica}) e Norme di Progetto (allegato \textit{\NormeDiProgetto}).\\
	Nella comunicazione tra client e server (vedi appendice \ref{par:XMLComm}) è stato scelto il formato XML\g~ per incapsulare i messaggi (anche quelli in formato JSON\g ). 
    }
    
	\subsection{Standard di documentazione del codice}{
	Per gli standard di documentazione del codice utilizzati si fa riferimento al documento Norme di Progetto (allegato \textit{\NormeDiProgetto}).
	}
	
	\subsection{Standard di denominazione di entità e relazioni}{
	Tutti gli elementi definiti, indipendentemente che siano package, classi, metodi o attributi, devo adottare una denominazione quanto più chiara ed esplicita anche a sacrificio della lunghezza del nome. Sono ammesse abbreviazioni purché siano non ambigue e di immediata comprensione.\\
	Per ulteriori informazioni si rimanda al documento Norme di Progetto (allegato \textit{\NormeDiProgetto}).
	}
	
	\subsection{Standard di programmazione}{
	Gli standard di programmazione sono definiti nel documento Norme di Progetto (allegato \textit{\NormeDiProgetto}) al quale si rimanda per ulteriori informazioni.
	}
	
	\subsection{Strumenti di lavoro}{
	Gli strumenti di lavoro adottati sono quelli individuati e descritti nel documento Norme di Progetto (allegato \textit{\NormeDiProgetto}).
	}
}
