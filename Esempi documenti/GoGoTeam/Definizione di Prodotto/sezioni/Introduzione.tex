\section{Introduzione}{
    \subsection{Scopo del documento}	{
Questo documento ha lo scopo di definire in modo approfondito la struttura e le relazioni dei vari componenti del prodotto \textbf{\mytalk}, riprendendo e specificando la struttura già definita nel documento di Specifica Tecnica (\emph{\SpecificaTecnica}).\\
Questo documento funge da guida e direttiva per i programmatori i quali dovranno implementare i componenti del sistema esattamente come indicato evitando aggiunte o modifiche alle API\g~ non previste.
    }
    
    \subsection{Scopo del prodotto}{
      Il prodotto \textbf{\mytalk} è volto ad offrire la possibilità agli utenti di comunicare tra loro, trasmettendo il segnale audio e video, attraverso 
      il browser\g~ mediante l'utilizzo di soli componenti standard, senza che sia necessario installare plugin\g~ o programmi aggiuntivi 
      (\textit{es.} Skype). Attualmente, infatti, la comunicazione istantanea tra utenti avviene solo tramite componenti non presenti di default nei
      browser\g . Il software\g~ dovrà risiedere in una singola pagina web\g~ e dovrà essere basato sulla tecnologia WebRTC\g~ 
      (\url {http://www.webrtc.org}).
    }

	\subsection{Norme sul documento corrente}{
		Valgono le seguenti norme relative alla struttura di questo documento:
		\begin{itemize}
			\item {Utilizzare esclusivamente i seguenti marcatori d'accesso ad attributi e metodi:
				\begin{itemize}
					\item + per i metodi ed attributi definiti \textbf{public};
					\item \# per i metodi ed attributi definiti \textbf{protected};
					\item - per i metodi ed attributi definiti \textbf{private}.
				\end{itemize}			
			}
			\item {Per definire i dettagli di ogni classe indicata in questo documento dev'essere utilizzata la seguente struttura:
				\begin{itemize}
					\item Descrizione della funzione svolta dalla classe;
					\item Relazioni con altre classi;
					\item Elenco attributi della classe, specificandone per ciascuno accessibilità, tipo, nome ed utilizzo;
					\item Elenco metodi della classe, specificandone per ciascuno accessibilità, tipo, nome ed utilizzo.
				\end{itemize}
			}
		\end{itemize}
		
	}
	
	
    \subsection{Glossario}{
	Al fine di migliorare la comprensione al lettore ed evitare ambiguità rispetto ai termini tecnici utilizzati nel documento, viene allegato il file
	\emph{\Glossario},  nel quale vengono descritti i termini contrassegnati dal simbolo \g~ alla fine della parola.
	Per i termini composti da più parole, oltre al simbolo \g, è presente anche la sottolineatura. 
    }

    \subsection{Riferimenti}{
	\subsubsection{Normativi}{
	    \begin{itemize}
		\item Capitolato d'appalto: \textbf{\mytalk}, \textit{software}\g~ \textit{di comunicazione tra utenti senza requisiti  di installazione}, rilasciato dal proponente \textit{Zucchetti S.p.A.}, reperibile all'indirizzo \url{http://www.math.unipd.it/~tullio/IS-1/2012/Progetto/C1.pdf}.
		\item Analisi dei requisiti (allegato \textit{\AnalisiDeiRequisiti}).
		\item Norme di progetto (allegato \textit{\NormeDiProgetto}).
	    \end{itemize}
	}
	\subsubsection{Informativi}{
	    \begin{itemize}
			\item Specifica Tecnica (allegato \textit{\SpecificaTecnica}).	
	    \end{itemize}
	}
    }
}
