\appendix
\appendixpage
\addappheadtotoc
\section{Obiettivi di qualità}
  \label{sec:Qualita}
  	In questa appendice vengono illustrati gli standard che \textit{GoGo Team} adotta come riferimenti: tali standard sono \texttt{ISO/IEC$_{|g|}$ 15504} per la qualità di processo e \texttt{ISO/IEC$_{|g|}$ 9126} per la qualità di prodotto.
    \subsection{Qualità dei processi}{
	La qualità del processo viene vista come un’esigenza, e la norma \texttt{ISO/IEC$_{|g|}$ 15504} (che ingloba SPICE, Software Process Improvement Capability dEtermination) 
	definisce un modello per la valutazione dei processi in un'organizzazione del settore IT (Information Technology), denominato SPY (SW Process Assessment \& Improvement, 
	vedi figura \ref{fig:ModelloSpy}).
	\begin{figure}[h!]
	    \centering
	    \includegraphics[scale=.6]{\docsImg ModelloSpy.pdf}
	    \caption{Modello SPY}
	    \label{fig:ModelloSpy}
	\end{figure} \\
	Tale standard definisce nove attributi di qualità:
	\begin{enumerate}
	    \item \textbf{Process performance}: un processo raggiunge i suoi obiettivi, trasformando input identificabili in output identificabili.
	    \item \textbf{Performance management}: l’attuazione di un processo è pianificata e controllata al fine di produrre risultati che rispondano agli obiettivi attesi.
	    \item \textbf{Work product management}: l’attuazione di un processo è pianificata e controllata al fine di produrre risultati che siano propriamente documentati, 
		  controllati e verificati.
	    \item \textbf{Process definition}: l’attuazione di un processo si basa su approcci standardizzati.
	    \item \textbf{Process resource}: il processo può contare sulle risorse adeguate (umane, infrastrutture, etc.) per essere attuato.
	    \item \textbf{Process measurement}: i risultati raggiunti e le misure rilevate durante l’attuazione di un processo sono stati usati per assicurarsi che l’attuazione di 
		  tale processo supporti efficacemente il raggiungimento di specifici obiettivi.
	    \item \textbf{Process control}: un processo è controllato attraverso la raccolta, analisi ed utilizzo delle misure di prodotto e di processo rilevate, al fine di 
		  correggere, se necessario, le sue modalità di attuazione.
	    \item \textbf{Process change}: le modifiche alla definizione, gestione, attuazione di un processo sono controllate.
	    \item \textbf{Continuous improvement}: le modifiche ad un processo sono identificate ed implementate al fine di assicurare il continuo miglioramento nel raggiungimento 
		  degli obiettivi rilevanti per l’organizzazione.
	\end{enumerate}
	
	La norma definisce poi quattro livelli di possesso di un attributo:
	\begin{itemize}
	    \item \textbf{N} -- non posseduto (0-15\% di possesso): non c’è evidenza, o ve ne è poca, del possesso di un attributo;
	    \item \textbf{P} -- parzialmente posseduto (16-50\% di possesso): vi è evidenza di approccio sistematico al raggiungimento del possesso di un attributo e del raggiungimento 
		  di tale possesso, ma alcuni aspetti del possesso possono essere non prevedibili;
	    \item \textbf{L} -- largamente posseduto (51-85\% di possesso): vi è evidenza di approccio sistematico al raggiungimento del possesso di un attributo e di un significativo livello 
		  di possesso di tale attributo, ma l’attuazione del processo può variare nelle diverse unità operative della organizzazione;
	    \item \textbf{F} -- (Fully) pienamente posseduto (86-100\% di possesso): vi è evidenza di un completo e sistematico approccio e di un pieno raggiungimento del possesso dell’attributo 
		  e non esistono significative differenze nel modo di attuare il processo tra le diverse unità operative.
	\end{itemize}

	Vi sono infine cinque livelli di maturità dei processi:
	\begin{itemize}
	    \item \textbf{Livello 1 – processo semplicemente attuato}: il processo viene messo in atto e raggiunge i suoi obiettivi. Non vi è evidenza di un approccio sistematico ad alcuno degli 
		  attributi definiti. Il raggiungimento di questo livello è dimostrato attraverso il possesso degli attributi di “process performance”.
	    \item \textbf{Livello 2 – processo gestito}: il processo è attuato ma anche pianificato, tracciato, verificato ed aggiustato se necessario, sulla base di obiettivi ben definiti. Il 
		  raggiungimento di questo livello è dimostrato attraverso il possesso degli attributi di “Performance management” e “Work product management”.
	    \item \textbf{Livello 3 – processo definito}: il processo è attuato, pianificato e controllato sulla base di procedure ben definite, basate sui principi del \underline{software engineering}$_{|g|}$. Il 
		  raggiungimento di questo livello è dimostrato attraverso il possesso degli attributi di  “Process definition” e “Process resource”.
	    \item \textbf{Livello  4 – processo predicibile}: il processo  è stabilizzato ed è attuato all’interno di definiti limiti riguardo i risultati attesi, le performance, le risorse impiegate, etc. 
		  Il raggiungimento di questo livello è dimostrato attraverso il possesso degli attributi di “Process measurement” e “Process control”.
	    \item \textbf{Livello 5 – processo ottimizzante}: il processo  è predicibile ed in grado di adattarsi per raggiungere obiettivi specifici e rilevanti per l'organizzazione. Il raggiungimento di questo 
		  livello è dimostrato attraverso il possesso degli attributi di  “Process change” e “Continuous improvement”.
	\end{itemize}
	I vantaggi derivanti dall’applicazione di questo standard sono molteplici: per l’azienda, permette di sfruttare conoscenza proveniente da \underline{best practice}$_{|g|}$, facilita il passaggio di consegne da una persona 
	all’altra, porta a riduzione dei costi, migliori tempi di consegna, maggiore qualità del prodotto, ritorno degli investimenti e soddisfazione del cliente. Per il cliente, invece, vi è una maggiore percezione 
	della qualità delle aziende sulla base del livello CMMI$_{|g|}$ raggiunto.
    }
    
    \subsection{Qualità del prodotto}{
    \label{sec:Qualita2}
	Si è scelto di seguire lo standard \texttt{ISO/IEC$_{|g|}$ 9126}, che definisce la qualità del prodotto software$_{|g|}$ come l’insieme delle 
	caratteristiche che incidono sulla capacità del prodotto di soddisfare requisiti espliciti o impliciti.\\
	Lo standard \texttt{ISO/IEC}$_{|g|}$ \texttt{9126} distingue poi tre tipi di qualità:
	\begin{itemize}
		\item qualità interna: rappresenta le qualità intrinseche del prodotto, cioè quelle misurabili direttamente a partire dal codice sorgente. 
		      Esse scaturiscono dai requisiti utente e dalla specifica tecnica;
		\item qualità esterna: è rappresentata dalle prestazioni del prodotto e dalle funzionalità che offre: riguarda quindi il comportamento 
		      dinamico del software$_{|g|}$. Essa deriva dai requisiti utente;
		\item qualità in uso: è rappresentata dal livello con cui il software$_{|g|}$ si rende utile all’utente, cioè la soddisfazione che 
		      deriva dal suo utilizzo e il grado di efficienza/efficacia che fornisce.
	\end{itemize}
	Vengono individuate sei caratteristiche indicatrici di qualità del prodotto software$_{|g|}$, ciascuna delle quali suddivisa 
	in sottocaratteristiche.

	\subsubsection{Funzionalità}{
	    \`E la capacità di fornire servizi in grado di soddisfare, in determinate condizioni, requisiti funzionali espliciti o impliciti. Le sue sottocaratteristiche sono:
	    \begin{itemize}
		\item \textbf{adeguatezza}: presenza di funzioni appropriate per compiti specifici che supportano gli obiettivi dell’utente;
		\item \textbf{accuratezza}: capacità di fornire risultati corretti, in accordo con i requisiti dati dall’utente;
		\item \textbf{interoperabilità}: capacità di interagire con altri sistemi;
		\item \textbf{sicurezza}: capacità di proteggere programmi e dati da accessi non autorizzati e consentire quelli autorizzati.\\
	    \end{itemize}
	    Nel Capitolato d'appalto non viene richiesto alcun livello di sicurezza, dato che l'obiettivo è quello di capire il funzionamento e le potenzialità delle librerie WebRTC$_{|g|}$.
	    La metrica, i valori attesi e gli strumenti adottati sono i seguenti:
	    \begin{itemize}
		\item \textbf{metrica}: percentuale di requisiti soddisfatti;
		\item \textbf{valore atteso}: l’obiettivo minimo è il soddisfacimento di tutti i requisiti obbligatori;
		\item \textbf{strumenti}: per garantire che il prodotto possieda tutte le funzionalità richieste sarà necessario il superamento di tutti i test effettuati con JUnit. Per quanto riguarda l'interfaccia, il suo funzionamento verrà garantito dalla verifica tramite normale utilizzo.
	    \end{itemize}
	}
	\subsubsection{Affidabilità}{
	    \`E la capacità di mantenere le prestazioni stabilite nelle condizioni e nei tempi fissati. Le sue sottocaratteristiche sono:
	    \begin{itemize}
		\item \textbf{Maturità (robustezza)}: capacità di evitare fallimenti dell'applicazione (\textit{failure}) a seguito di malfunzionamenti (\textit{fault});
		\item \textbf{Tolleranza errori}: capacità di mantenere determinati livelli di prestazione in caso di malfunzionamenti;
		\item \textbf{Recuperabilità}: capacità e velocità, in caso di malfunzionamento, di ripristinare i livelli di prestazione predeterminati e di recuperare dati.
	    \end{itemize}
	    La metrica, i valori attesi e gli strumenti adottati sono i seguenti:
	    \begin{itemize}
		\item \textbf{metrica}: percentuale di test di sistema soddisfatti;
		\item \textbf{valore atteso}: il superamento di tutti i test di sistema;
		\item \textbf{strumenti}: Chrome Driver / test manuali.
	    \end{itemize}
	}
	\subsubsection{Usabilità}{
	    \`E la capacità di essere compreso, appreso e usato con soddisfazione dall'utente in determinate condizioni d'uso. Le sue sottocaratteristiche sono:
	    \begin{itemize}
		\item \textbf{Comprensibilità}: capacità di ridurre l'impegno richiesto agli utenti per capirne il funzionamento e le modalità di utilizzo;
		\item \textbf{Apprendibilità}: capacità di ridurre l'impegno richiesto agli utenti per imparare ad usarlo;
		\item \textbf{Operabilità}: capacità di mettere in condizione gli utenti di farne uso per i propri scopi;
		\item \textbf{Attrattività/Piacevolezza}: capacità di essere piacevole per l'utente che ne fa uso.
	    \end{itemize}
	    La metrica, i valori attesi e gli strumenti adottati sono i seguenti:
	    \begin{itemize}
		\item \textbf{metrica}: non è possibile stabilire una metrica per questo aspetto;
		\item \textbf{valore atteso}: un utente con competenze di utilizzo del computer di base deve essere in grado di utilizzare il software$_{|g|}$;
		\item \textbf{strumenti}: prova da parte di un utente non informatico.
	    \end{itemize}
	}
	\subsubsection{Efficienza}{
	    \`E il rapporto fra prestazioni e quantità di risorse utilizzate, in condizioni definite di funzionamento. Le sue sottocaratteristiche sono:
	    \begin{itemize}
		\item \textbf{comportamento rispetto al tempo}: tempi di risposta, tempi di elaborazione e \underline{throughput rates}$_{|g|}$ adeguati per eseguire le 
		      funzioni richieste, sotto determinate condizioni;
		\item \textbf{uso di risorse}: utilizzo di una quantità e di una tipologia di risorse adeguate per eseguire le funzioni richieste, sotto determinate condizioni.\\
	     \end{itemize}
	     La metrica, i valori attesi e gli strumenti adottati sono i seguenti:
	     \begin{itemize}
		\item \textbf{metrica}: il tempo di risposta dell’applicazione, la latenza$_{|g|}$, il tempo necessario alla connessione;
		\item \textbf{valore atteso}: data la novità del tipo di applicazione da sviluppare, al momento non è possibile stabilire un valore atteso.
		      L'applicazione deve comunque rispondere in tempi ragionevoli;
		\item \textbf{strumenti}: il metodo \texttt{getStats}, che calcola le statistiche relative alla connessione delle librerie WebRTC$_{|g|}$.
	      \end{itemize}
	}
	\subsubsection{Manutenibilità}{
	    \`E la capacità di essere modificato con un impegno contenuto. Le sue sottocaratteristiche correlate sono:
	     \begin{itemize}
		\item \textbf{analizzabilità}: capacità di limitare l’impegno richiesto per diagnosticare carenze o cause di malfunzionamenti, o 
		      per identificare parti da modificare;
		\item \textbf{modificabilità}: capacità di limitare l’impegno richiesto per modificare il programma, rimuovere errori o sostituire componenti;
		\item \textbf{stabilità}: capacità di ridurre il rischio di comportamenti inaspettati a seguito dell’effettuazione di modifiche;
		\item \textbf{testabilità}: capacità di essere facilmente testato per validare le modifiche apportate.\\
	     \end{itemize}
	     La metrica, i valori attesi e gli strumenti adottati sono i seguenti:
	     \begin{itemize}
		\item \textbf{metrica}: i punti di riferimento sono quelli descritti nella sezione Metriche (cap. \ref{sec:Metriche});
		\item \textbf{valori attesi}: rispetto dei valori indicati nella sezione Metriche (cap. \ref{sec:Metriche}) per le singole metriche;
		\item \textbf{strumenti}: il plugin$_{|g|}$ Metrics per Eclipse.
	     \end{itemize}
	}

	\subsubsection{Portabilità}{
	     \`E la facilità con cui il software$_{|g|}$ può essere trasferito da un ambiente operativo ad un altro. Le sue sottocaratteristiche sono:
	     \begin{itemize}
		\item \textbf{adattabilità}: capacità di adattarsi a nuovi ambienti operativi limitando la necessità di apportare modifiche;
		\item \textbf{installabilità}: capacità di ridurre l’impegno richiesto per installarlo in un particolare ambiente operativo;
		\item \textbf{coesistenza}: capacità di coesistere con altri software$_{|g|}$ nel medesimo ambiente, condividendo risorse;
		\item \textbf{sostituibilità}: capacità di essere utilizzato al posto di un altro software$_{|g|}$ per svolgere gli stessi compiti nello stesso ambiente.\\
	     \end{itemize}
	     Il requisito obbligatorio da soddisfare per il prodotto \textbf{MyTalk} è il funzionamento sul browser$_{|g|}$ \underline{Google Chrome}$_{|g|}$.
	     La portabilità su piattaforme diverse dipende quindi dall'implementazione delle librerie WebRTC$_{|g|}$ e dalla capacità degli altri browser$_{|g|}$
	     di supportarle.
	}
}