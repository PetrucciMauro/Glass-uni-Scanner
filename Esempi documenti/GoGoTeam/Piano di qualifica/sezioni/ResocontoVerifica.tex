\section{Resoconto attività di verifica} {
  In questa sezione vengono descritte le procedure adottate durante il processo di verifica e i risultati ottenuti.
   
  \subsection{Dettaglio delle verifiche tramite analisi}{
	\subsubsection{Analisi dei requisiti}{
	 	Le tabelle presenti nel documento \emph{\AnalisiDeiRequisiti} evidenziano i seguenti requisiti:
	\begin{itemize}
		\item \textbf{totali}: 141
		\item \textbf{Funzionali Utente}: 90, di cui 55 obbligatori, 5 desiderabili, 30 opzionali;
		\item \textbf{Funzionali Amministratore}: 30, di cui 27 obbligatori, 3 opzionali;
		\item \textbf{di Vincolo}: 18, di cui 11 obbligatori, 2 desiderabili, 5 opzionali;
		\item \textbf{di Qualità}: 3, di cui 1 obbligatorio e 2 desiderabili.
	\end{itemize}
	Gli \underline{use case}$_{|g|}$ individuati sono 119, di cui 89 lato utente e 30 lato amministratore.
	\`E stato poi verificato che tutti i bisogni emersi siano stati coperti da requisiti, e che tutti i requisiti siano coperti
	dagli \underline{use case}$_{|g|}$.
		}
	}
	
	\subsubsection{Progettazione ad alto livello}{
	Nella fase di progettazione si verifica, tramite tracciamento, che ad ogni
	requisito funzionale espresso nell’analisi dei requisiti corrisponda una
	componente architetturale e viceversa, garantendo il pieno soddisfacimento dei
	requisiti stessi.\\
	Per consultare l’esito del tracciamento componenti - requisiti effettuato si
	rimanda al documento \textit{\SpecificaTecnica}.
	}
	
	\subsection{Dettaglio della verifiche tramite i test}{
	Si rimanda all'appendice \ref{sec:test} (Attività di test) per le misurazioni e i test svolti con i rispettivi risultati.
	}
	
	\subsection{Dettaglio dell’esito delle revisioni}{
	Le revisioni di progetto sono le scadenze alle quali il gruppo dovrà presentare, di volta in volta, la documentazione 
	necessaria prodotta al fine di coordinare tutte le risorse appartenenti al gruppo e procedere con lo sviluppo del
	progetto. Il numero totale di revisioni per l’anno accademico 2012/2013 è 4:
	\begin{itemize}
		\item[•] Revisione dei Requisiti
		\item[•] Revisione di Progetto
		\item[•] Revisione di Qualifica
		\item[•] Revisione di Accettazione
	\end{itemize}
	
	Ad ogni revisione il gruppo sottoporrà al \textit{Committente} la documentazione elaborata accompagnata da una breve presentazione. 
	\textit{GoGo Team} si propone di partecipare alla revisione di qualifica del 2013-06-11, come specificato nel documento.
	
	
	\subsubsection{Revisione dei requisiti}{
	In seguito alle osservazioni effettuate dal committente in sede di RR, il gruppo
	\textit{GoGo Team} si è riunito per analizzare gli esiti e correggere gli errori 
	rilevati con lo scopo di presentare una versione migliorativa dei vari
	documenti.\\
	Le correzioni svolte sono:
		\begin{itemize}
		
			\item[•] \textbf{Su tutti i documenti:}
				\begin{itemize}
					\item applicata una forma tabulare al registro delle modifiche;
 					\item aggiunto lo storico delle revisioni.
					\end{itemize}
	 
			\item[•] \textbf{Analisi dei requisiti:}
				\begin{itemize}
					\item revisione dei casi d'uso dell'ambito utente con conseguenti modifiche alla struttura 
					e con conseguente aggiunta di essi;
					\item revisione dei casi d'uso dell'ambito amministratore con conseguenti modifiche alla 
					struttura e con conseguente aggiunta di essi;
					\item aggiornamento dei requisiti e del relativo tracciamento in base alle modifiche e alle 
					aggiunte effettuate ai casi d'uso;
					\item inserito il tracciamento tra requisiti e casi d'uso.
				\end{itemize}
			
			\item[•] \textbf{Piano di qualifica:}
				\begin{itemize}
					\item riorganizzazione dell'indice;
					\item spostamento del capitolo 3 in appendice;
					\item spostamento del capitolo 2.4 in
					\emph{\NormeDiProgetto}.
				\end{itemize}
			
			\item[•] \textbf{Piano di progetto:}
				\begin{itemize}
					\item Il documento è stato integrato con i contenuti
				richiesti
					\item correzione di alcune terminologie utilizzate;
					\item è stata rivista l’assegnazione delle ore giornaliere di
				lavoro che ogni membro del gruppo dovrà dedicare al progetto;
					\item applicata una forma tabulare all'analisi dei rischi.
				\end{itemize}
			
			\item[•] \textbf{Norme di progetto:}
				\begin{itemize}
					\item il documento è stato integrato con i contenuti
					richiesti (norme relative alla progettazione, procedure per la gestione dei cambiamenti, garanzia dell'assenza di conflitto di interessi nell'attività di verifica);
					\item il capitolo riguardante il versionamento è stato
					modificato;
					\item sono state incorporate nel documento le specifiche degli
					strumenti e le procedure di utilizzo che erano state
					precedentemente inserite nella versione 1.0 del documento.
					\emph{\PianoDiQualifica}.		
				\end{itemize}
			
			\item[•] \textbf{Studio di fattibilità:} approfondita l'analisi della criticità;
			\item[•] \textbf{Glossario:} correzione di alcune terminologie utilizzate.
			
		\end{itemize}
	}
	
	\subsubsection{Revisione di progettazione ad alto livello}{

	In seguito alle osservazioni effettuate dal committente in sede di RP, il gruppo
	\textit{GoGo Team} si è riunito per analizzare gli esiti e correggere gli errori 
	rilevati con lo scopo di presentare una versione migliorativa dei vari
	documenti.\\
	Le correzioni svolte sono:
		\begin{itemize}
		
			\item[•] \textbf{Analisi dei requisiti:}
				\begin{itemize}
					\item modifica dell'immagine della gerarchia utente  (paragrafo 3.1);
					\item aggiunta di scenari alternativi ad alcuni casi d'uso;
					\item modifica dei requisiti FOB0 e QOB1.
				\end{itemize}
			
			\item[•] \textbf{Piano di qualifica:}
				\begin{itemize}
					\item rimozione del modello a V nel capitolo 2, non compatibile con il modello di ciclo di
					      vita adottato;
					\item estensione dei capitoli 2 (sottosezione ``Pianificazione Strategica e Generale'') e 4;
					\item spostamento nel documento \NormeDiProgetto~ delle risorse software del capitolo 3;
					\item estensione capitolo in appendice: Obiettivi di qualità, fissato metriche e valori attesi.
				\end{itemize}
			
			\item[•] \textbf{Piano di progetto:}
				\begin{itemize}
					\item ripartizione ore per ruolo;
					\item aggiunta preventivo a finire;
					\item motivazione spostamento RQ.
				\end{itemize}
			
			\item[•] \textbf{Norme di progetto:}
				\begin{itemize}
					\item definizione e stesura delle norme relativa ad ogni singola attività;
					\item riorganizzazione del documento;
					\item aggiornamento degli strumenti utilizzati.
				\end{itemize}
			
			\item[•] \textbf{Specifica tecnica:}
				\begin{itemize}
					\item Aggiunta dei diagrammi dei package;
					\item Aggiunta del DAO;
					\item Aggiunta e modifica delle classi interne;
					\item Spostamento di alcuni package.
				\end{itemize} 
			\item[•] \textbf{Glossario:} Nessuna correzione specifica da operare nel documento, che è stato comunque
				ampliato con qualche nuova definizione.
			
		\end{itemize}
	}
	
	\subsubsection{Revisione di qualifica}{

	In seguito alle osservazioni effettuate dal committente in sede di RQ, il gruppo
	\textit{GoGo Team} si è riunito per analizzare gli esiti e correggere gli errori 
	rilevati con lo scopo di presentare una versione migliorativa dei vari
	documenti.\\
	Le correzioni svolte sono:
		\begin{itemize}
		
			\item[•] \textbf{Analisi dei requisiti:}
				\begin{itemize}
					\item modifica dell'immagine della gerarchia utente (paragrafo 3.1);
					\item eliminazione del requisito FOB0.
				\end{itemize}
			
			\item[•] \textbf{Piano di qualifica:}  Nessuna correzione specifica da operare nel documento.
			
			\item[•] \textbf{Piano di progetto:}
				\begin{itemize}
					\item revisione e aggiornamento del preventivo a finire, ora consuntivo.
				\end{itemize}
			
			\item[•] \textbf{Norme di progetto:}
				\begin{itemize}
					\item modificata la struttura del documento
				\end{itemize}
			
			\item[•] \textbf{Specifica tecnica:}
				\begin{itemize}
					\item modifica dell'architettura per eliminare le dipendenze circolari;
					\item aggiunto il design pattern DAO e modificata la descrizione del design pattern MVP.
				\end{itemize}
			
			\item[•] \textbf{Definizione di prodotto:}
				\begin{itemize}	
					\item aggiornate le nuove classi e i relativi metodi;
					\item modifica dei diagrammi delle classi.
				\end{itemize}
				
			\item[•] \textbf{Manuale Amministratore:}
				\begin{itemize}	
					\item rivista la descrizione dello scopo del documento;
					\item aggiunto l'oggetto predefinito della mail;
					\item assegnato un codice di errore ad ogni errore.
				\end{itemize}		
			
			\item[•] \textbf{Manuale installazione:}
				\begin{itemize}
					\item riviste le istruzioni relative all'esecuzione di Tomcat;
					\item inserimento Javadoc.
				\end{itemize}
					
			\item[•] \textbf{Manuale utente:}
				\begin{itemize}	
					\item rivista la descrizione dello scopo del documento;
					\item aggiunto l'oggetto predefinito della mail;
					\item assegnato un codice di errore ad ogni errore;
					\item rivisti i riferimenti alle figure;
					\item riviste le descrizioni relative al primo accesso all'applicazione e alle funzionalità di chiamata e ricezione.
				\end{itemize}				
			 
			\item[•] \textbf{Glossario:} Nessuna correzione specifica da operare nel documento.
			
		\end{itemize}
	}	
	
  }
  \subsection{Revisione documentazione} {
	È stata effettuata, come prima verifica, un'attività di walkthrough su tutta la documentazione, che ha permesso di
	individuare errori grammaticali e di sintassi. Tali errori sono stati comunicati dal Verificatore al Redattore del documento,
	il quale ha poi provveduto alla loro correzione. Sono stati inoltre rivisti e corretti i riferimenti interni ed esterni ai documenti
	scritti in \LaTeX che non erano funzionanti e mostravano i doppi punti di domanda (??).\\
	\`E stato infine svolto un controllo ortografico su tutti i file \texttt{*.tex} prodotti tramite Aspell (eseguendo il comando
	riportato in \emph{\NormeDiProgetto}), che ha rilevato numerosi errori di battitura sfuggiti al Verificatore dopo una prima 
	lettura completa.
  }
}