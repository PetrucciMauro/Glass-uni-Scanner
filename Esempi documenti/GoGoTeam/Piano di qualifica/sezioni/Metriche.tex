\section{Interpretazione delle metriche}{
\label{sec:Metriche}
    Una metrica è una misura di una qualche proprietà relativa ad una porzione di software$_{|g|}$, allo scopo di fornire informazioni significative sulla qualità del codice prodotto.
    Non bisogna tuttavia basarsi solamente sulle metriche, che sono solamente indicatori a posteriori della bontà del lavoro svolto: un’importanza ancora maggiore la riveste il controllo sulla qualità del processo.\\
    Il plugin Metrics di Eclipse permette di monitorare, a tempo di compilazione, numerose metriche relative al codice Java$_{|g|}$ prodotto. Esso dà inoltre la possibilità di impostare dei limiti 
    per le metriche, segnalando eventuali violazioni. L’assenza di valori out of range (fuori limite), tuttavia, non è una condizione sufficiente ad indicare la bontà del codice scritto, che non può prescindere da una buona progettazione e che deve essere il più possibile corretto per costruzione.
    Le metriche che Metrics è in grado di calcolare sono le seguenti:

\paragraph{Dimensionali}\hbox{}
    \begin{itemize}
	\item NOF (Number Of Fields): numero di campi dati per classe;
	\item NSC (Number of Children): numero di sottoclassi di primo livello di una classe; è un indicatore del riuso del codice;
	\item NOC (Number of Classes): numero di classi in un particolare scope$_{|g|}$;
	\item NSF (Number of Static attributes): numero di attributi statici in un particolare scope$_{|g|}$;
	\item NOM (Number of Methods): numero di metodi in un particolare scope$_{|g|}$;
	\item NOI (Number Of Interfaces): numero di interfacce in un particolare scope$_{|g|}$;
	\item NSM (Number of Static Methods): numero di metodi statici$_{|g|}$ in un particolare scope$_{|g|}$;
	\item NOP (Number Of Packages): numero di package$_{|g|}$ in un particolare scope$_{|g|}$;
	\item MLOC (Method Lines Of Code): numero di linee di codice di un metodo, al netto delle linee vuote o commentate;
	\item TLOC (Total Lines Of Code): numero di linee di codice totali, al netto delle linee vuote o commentate;
	\item NBD (Nested Block Depth): profondità di annidamento dei blocchi;
	\item PAR (number of PARameters): numero di parametri in un particolare scope$_{|g|}$. L’obiettivo sarà quello di limitare questo numero a 5.
    \end{itemize}
    Anche se si ritiene ragionevole porsi l’obiettivo di mantenere la lunghezza dei moduli$_{|g|}$ al di sotto delle 400 righe per agevolarne la manutenibilità, le metriche riguardanti il numero 
    di linee di codice non devono essere viste troppo rigidamente. Piuttosto, la regola da seguire è quella della semplicità: ogni metodo dovrebbe fare una e una sola cosa. Quindi, invece di 
    porre un limite alla lunghezza dei metodi (magari \textit{42} righe?), è più logico chiedersi quanti compiti svolge un singolo metodo. Se la risposta alle domande “il mio codice è molto indentato?” e “ho inserito spazi vuoti per separare logicamente gruppi di operazioni?” è sì, allora è il caso di spezzare il metodo in altri più semplici. Seguendo queste linee guida, nella maggior parte dei casi la lunghezza non dovrebbe comunque superare le poche decine di righe.


\paragraph{Riguardanti i tipi}\hbox{}
  \begin{itemize}
    \item \textit{CA (Afferent Coupling)}:  numero di classi esterne al package$_{|g|}$ corrente che dipendono da classi interne al package$_{|g|}$; è equivalente a fan-in. Un valore superiore a 2 indica un  codice riusato. Più alto è il valore, maggiore è il riuso. Valori eccessivi indicano però un accoppiamento eccessivo: queste componenti vengono chiamate spesso perché svolgono più di un lavoro; 
    \begin{figure}[h!]
	    \centering
	    \includegraphics[scale=.7]{\docsImg CA.pdf}
	    \caption{Afferent Coupling}
	   % \label{fig:ModelloSpy}
	\end{figure} \\
    \newpage
    \item \textit{CE (Efferent Coupling)}: numero di classi interne al package$_{|g|}$ corrente che dipendono da classi esterne al package$_{|g|}$; è equivalente a fan-out. Alti valori 
	  indicano un’eccessiva dipendenza del modulo$_{|g|}$ e ne complicano il testing: questo valore va quindi mantenuto basso.
	   \begin{figure}[h!]
	    \centering
	    \includegraphics[scale=.7]{\docsImg CE.pdf}
	    \caption{Efferent Coupling}
	   % \label{fig:ModelloSpy}
	   \end{figure} \\
	  
  \end{itemize}
  L’accoppiamento fa riferimento ai legami esistenti tra unità (classi) separate in un programma. Se due classi dipendono strettamente l’una dall’altra, si dicono fortemente accoppiate 
  (strong coupling). L’obiettivo è quello di minimizzare l’accoppiamento (loose coupling), agevolando così la manutenibilità e la facilità di comprensione del software$_{|g|}$, dato che 
  non è necessario preoccuparsi di reperire le informazioni sulle altre classi associate. Seguendo questo principio, eventuali modifiche apportate ad una classe avranno poche o nessuna 
  ripercussione sulle altre classi che dipendono da lei.

\paragraph{Architetturali}\hbox{}
\begin{itemize}
       \item \textit{VG (McCabe Cyclomatic Complexity)}: complessità ciclomatica di McCabe. Misura il numero di cammini in un pezzo di codice. Ad ogni occorrenza di un branch, il numero 
	      incrementa di uno. I costrutti che incrementano la complessità ciclomatica sono: \textbf{if, else, while, do...while, switch} (+ 1 per ogni caso), \textbf{for, foreach, catch}.
	      La formula utilizzata per calcolare la complessità ciclomatica è mostrata nell’equazione (n) ed è definita in riferimento ad un grafo contenente i blocchi base di un 
	      programma, con un arco tra due blocchi se il controllo può passare da uno all’altro:
	      \begin{equation}
			      V(G) = e-n+2p
	      \end{equation}

	      \begin{itemize}
		    \item[] dove V(G) = complessità ciclomatica di G
		    \item[] e = numero of archi del grafo
		    \item[] n = numero di nodi del grafo
		    \item[] p = numero di componenti connesse
	      \end{itemize}
	      Verrà adottata la seguente tabella di riferimento per la complessità:\\
	      \begin{table}[h!]
		      \begin{center}
			      \begin{tabular}{l l c}				
				      \toprule
				      Complessità ciclomatica&	 Tipo di procedura &	 Rischio \\ 
				      \midrule
				      1-4 & Semplice &	Basso\\
				      5-10 & Ben strutturata e stabile &	Medio Basso\\
				      11-20 & Complessa &	Moderato\\
				      21-50 & Molto complessa, preoccupante &	Alto\\
				      \textgreater 50 & Soggetta a errori, problematiche, instabile  &	Molto alto\\
				      
				      \bottomrule
			      \end{tabular}
		      \end{center}	
		      \caption{Stima del costo e delle ore totali.}
	      \end{table}
	      L’obiettivo del gruppo sarà quello di mantenere una complessità ciclomatica $\leqslant$ 5;
	      \begin{itemize}
		      \item[$\bullet$] \textit{LCOM (Lack of Cohesion Of Methods)}: mancanza di coesione dei metodi. La coesione di una classe è influenzata da quanto i suoi metodi sono collegati alle 
				sue variabili d’istanza. In caso di perfetta coesione (tutti i metodi accedono a tutti gli attributi) il valore sarà 0, mentre in caso di totale assenza di 
				coesione (ogni metodo accede a una singola variabile) il valore sarà 1 e sarà opportuno dividere la classe.
	      
				\begin{equation}
				LCOM=\frac{ \frac{1}{n}(\sum_{i=0}^m m(A)_i)- m}{1-m}
				\end{equation}

				Sia m(A) il numero di metodi che accedono ad un particolare attributo. Si calcola la media per tutti gli n attributi, si sottrae il numero m di metodi e si 
				divide per 1 - m. Attraverso la coesione si è in grado di stabilire quali e quanti siano i compiti per i quali una classe (o metodo) è stata disegnata. Più una 
				classe ha una responsabilità ristretta a un solo compito, più si dice che è coesa. L’obiettivo da prefiggersi è quello di avere una coesione elevata: ciò 
				semplifica la comprensione di una classe (o metodo), ne semplifica la nomenclatura e ne favorisce il riutilizzo;

	      
	      
		      \item[$\bullet$] \textit{RMD (Normalized Distance)}: il suo numero dovrebbe essere prossimo allo zero per indicare un buon design dei package$_{|g|}$. E’ data da
			    \begin{equation}
				|RMA + RMI - 1|;
			    \end{equation}
       
		      \item[$\bullet$] \textit{RMA (Abstractness)}: numero di classi astratte e interfacce diviso il numero totale di classi in un particolare scope$_{|g|}$. 0 indica un package$_{|g|}$ 
			    interamente concreto, mentre 1 indica un package$_{|g|}$ interamente astratto. Classi che hanno elevati valori di CA e CE dovrebbero essere più astratte allo scopo 
			    di mascherare i dettagli implementativi e facilitare eventuali modifiche al codice;

		      \item[$\bullet$] \textit{RMI (Instability)}:
			    \begin{equation}
				\frac{CE}{CA+CE}  
			    \end{equation} 
			     0 indica un package$_{|g|}$ completamente stabile, mentre 1 indica un package$_{|g|}$ completamente instabile. Da questa metrica si evince come l’efferent coupling vada 
			      contro la stabilità del package$_{|g|}$: più un package$_{|g|}$ fa affidamento sugli altri, più è vulnerabile rispetto ai cambiamenti degli altri package$_{|g|}$;
       
		      \item[$\bullet$] \textit{NORM (Number of Overridden Methods)}: numero di metodi che sono overriding$_{|g|}$ di metodi presenti nelle classi superiori, in un particolare 
			      scope$_{|g|}$;
 
		      \item[$\bullet$] \textit{DIT (Depth of Inheritance Tree)}: profondità dell’albero di ereditarietà, intesa come la lunghezza massima dal nodo alla radice. Una profondità bassa 
			      può indicare una complessità inferiore, ma anche la possibilità che non ci sia abbastanza riuso del codice tramite l’ereditarietà. Dualmente, una profondità 
			      alta aumenta il potenziale riuso del codice, ma introduce maggiore complessità e possibili errori. Anche se non esistono valori standard per la profondità 
			      dell’albero di ereditarietà, assumiamo come limite superiore un valore pari a 6;
 
		      \item[$\bullet$] \textit{SIX (Specialization IndeX)}: indice di specializzazione, definito come 
			     \begin{equation}
				 \frac{NORM*DIT}{NOM}
			     \end{equation}

			      Si riferisce alla quantità di uso dell’ereditarietà; SIX fornisce un’indicazione approssimata del grado di specializzazione per ogni sottoclasse in un 
			      software$_{|g|}$ Object Oriented;



		       \item[$\bullet$] \textit{WMC (Weighted Methods per Class)}: somma della complessità ciclomatica per tutti i metodi di una classe.
		\end{itemize}

\end{itemize}

}
