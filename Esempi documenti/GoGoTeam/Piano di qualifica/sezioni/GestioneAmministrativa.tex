
\section{Gestione amministrativa della revisione}{
    \subsection{Comunicazione e risoluzione delle anomalie}{
	Un’anomalia è uno scostamento del comportamento del programma rispetto alle aspettative prefissate.\\
	Per la segnalazione e il trattamento delle anomalie si utilizza il servizio di ticketing offerto dalla piattaforma di sviluppo 
	SourceForge (\url{http://www.sourceforge.net}). Quando il Verificatore riscontra un’anomalia deve aprire un ticket\g su SourceForge. 
	Le modalità di risoluzione di quest’ultimo e la sua struttura vengono descritte in modo dettagliato nel documento \emph{\NormeDiProgetto}.
	Quando un documento/modulo$_{|g|}$ viene rilasciato in una nuova versione, il Verificatore controlla il registro delle modifiche e, in base ad esso, 
	effettua una verifica alla ricerca di anomalie da correggere. Se ne trova, apre un ticket$_{|g|}$~ su SourceForge e lo comunica all’Amministratore; 
	sarà poi compito della persona che ha modificato il documento/modulo$_{|g|}$ apportare le dovute modifiche, che devono essere approvate dall’Amministratore. 
    }

    \subsection{Trattamento della discrepanza}{
      Una discrepanza è una differenza tra il lavoro svolto e quanto era stato pianificato, sia per quanto riguarda i requisiti specificati nel capitolato d’appalto 
      che per quanto riguarda le norme di progetto.\\
      Risolvere una discrepanza comporta necessariamente dei costi, che dovranno venire quantificati dal Responsabile di Progetto sotto il punto di vista delle risorse umane, 
      economiche e temporali.
      Qualunque sia l’entità della discrepanza, essa dovrà essere registrata e tracciata tramite ticket$_{|g|}$.
    }

    \subsection{Procedure di controllo di qualità del processo}{
	L’organizzazione dei processi fa riferimento al ciclo di Deming, che ha come
	scopo il miglioramento continuo, secondo il principio del PDCA. Gli obiettivi
	sono promuovere una cultura volta al perseguimento della qualità e all’utilizzo
	ottimale delle risorse e dare una maggiore coesione tra le fasi di analisi,
	progettazione, realizzazione, verifica e collaudo. Tale metodo è suddiviso in
	quattro fasi:
	\begin{itemize}{
		\item[1.] \textbf{Plan - Pianificare}
			\begin{itemize}
			\item[(a)] Definire il problema/impostare il progetto;
			\item[(b)] Documentare la situazione di partenza;
			\item[(c)] Analizzare il problema;
			\item[(d)] Pianificare le azioni da realizzare.
			\end{itemize}
			
		\item[2.] \textbf{Do - Realizzare} 
			\begin{itemize}
				\item[(a)] Addestrare le persone incaricate della realizzazione;
				\item[(b)] Realizzare le azioni che sono state pianificate.
			\end{itemize}
		\item[3.] \textbf{Check -Verificare} 
			\begin{itemize}
			    \item[(a)] Verificare i risultati e confrontarli con gli obiettivi;
				\item[•] Se si è raggiunto l’obiettivo: passare alla lettera a della fase Act;
				\item[•] Se non si è raggiunto l’obiettivo: passare alla lettera b della fase Act.
			\end{itemize}
		\item[4.] \textbf{Act - Mantenere o Migliorare}
			\begin{itemize}
				\item[(a)] Obiettivo raggiunto:
				\begin{itemize}
					\item[i.] Standardizzare, consolidare e addestrare gli operatori;
					\item[ii.] Procedere a un nuovo PDCA per un ulteriore miglioramento sul tema.
					\end{itemize}
			\item[(b)] Obiettivo non raggiunto:
				\begin{itemize}
				\item[i.] Ripetere il ciclo PDCA sullo stesso problema, analizzando criticamente le varie fasi del ciclo precedente per individuare le cause del non raggiungimento dell’obiettivo.
				\end{itemize}
			\end{itemize}
	}\end{itemize}
	
	 
	\begin{figure}[h!]
	  \centering
	  \includegraphics[scale=.55]{\docsImg CicloDiDeming.pdf}
	  \caption{Ciclo di Deming}
	\end{figure}
	
	I parametri che permetteranno di valutare la qualità del processo saranno principalmente:
	\begin{itemize}
		\item il tempo impiegato per essere portato a termine;
		\item la quantità di risorse impiegate;
		\item l’aderenza del processo alla pianificazione iniziale;
		\item il numero di iterazioni che è stato fatto;
		\item la soddisfazione dei requisiti richiesti;
		\item il numero di difetti trovati durante la fase di testing;
		\item l'efficacia dell'attività di correzione dei difetti.
	\end{itemize}
	
	Tali parametri devono essere quantificati sia durante che al termine del processo, al fine di individuare eventuali problemi e capire, 
	attraverso un'analisi condivisa dai membri del gruppo, in quali aree c'è bisogno di un miglioramento. 
	Alla successiva istanziazione del processo, i dati raccolti la volta precedente vanno capiti e migliorati in modo da rendere più 
	efficiente ed efficace il processo stesso.\\

	Il gruppo \textit{GoGo Team} adotta come riferimento gli standard \texttt{ISO/IEC}$_{|g|}$ \texttt{15939} e 
	\texttt{ISO/IEC}$_{|g|}$ \texttt{14598} per la misurazione e valutazione del software$_{|g|}$ nel suo intero ciclo di vita. 
	In particolare, \texttt{ISO/IEC}$_{|g|}$ \texttt{15939} definisce il processo di misura un ``ciclo iterativo che prevede la misurazione, 
	la raccolta di feedback e l’utilizzo di questi ultimi per impostare azioni correttive per migliorare il processo di produzione PDCA''. 
	La norma suddivide il processo di verifica nelle seguenti quattro attività:
	\begin{enumerate}
		\item definire l’esigenza della misura in base ai requisiti da rispettare e alle risorse a disposizione;
		\item pianificare la misurazione definendo tecniche, metriche e procedure;
		\item rilevare le misure memorizzando i dati per creare uno storico;
		\item valutare le misure rilevate generando report e pianificando azioni correttive.
	\end{enumerate}
	Lo scopo delle misurazioni è quello di prevedere e stimare i costi per lo sviluppo e la qualità del prodotto. 
	Per tutta la durata del ciclo di vita esse verranno effettuate in diversi momenti:
	\begin{itemize}
		\item in fase di progettazione: hanno l’obiettivo di quantificare l’impegno richiesto dalla manutenzione futura e di prevenire 
		      problemi quando il software$_{|g|}$ sarà in uso; 
		\item in fase di collaudo e test: viene effettuato un confronto con le specifiche date e si cerca di individuare problemi non 
		      considerati precedentemente;
	\end{itemize}
	Vengono di seguito definiti i termini \textit{misurazione}, \textit{misura}, \textit{metrica} e \textit{indicatore}:
	\begin{itemize}
		\item \textit{misurazione}: l’uso di una metrica per assegnare un valore su una scala predefinita;
		\item \textit{misura}: risultato della misurazione;
		\item \textit{metrica}: insieme di regole per fissare le entità da misurare, gli attributi rilevanti, l’unità di misura, 
		      la procedura per assegnare e interpretare i valori;
		\item \textit{indicatore}: misura ottenuta indirettamente a partire da altre misure o tramite stime o predizioni; 
		      si utilizza un indicatore per quantificare aspetti difficilmente misurabili.
	\end{itemize}
	La norma \texttt{ISO/IEC}$_{|g|}$ \texttt{14598} descrive il processo di valutazione della qualità del software$_{|g|}$, 
	in accordo con la norma \texttt{ISO/IEC}$_{|g|}$ \texttt{9126} descritta in appendice \ref{sec:Qualita}. Essa verrà utilizzata per 
	valutare il software$_{|g|}$ durante tutto il suo ciclo di vita. Ogni valutazione deve possedere una descrizione, uno scopo, deve 
	identificare i prodotti da valutare e i risultati attesi per ogni caratteristica presa in esame. La figura seguente 
	schematizza il processo di valutazione.
 
	\begin{figure}[h!]
	  \centering
	  \includegraphics[scale=.55]{\docsImg ProcessoValutazione.pdf}
	  \caption{Il processo di valutazione}
	\end{figure}

	Poiché le caratteristiche definite nel modello \texttt{ISO/IEC}$_{|g|}$ \texttt{9126} non sono tutte richieste per il prodotto 
	che il gruppo deve sviluppare, verrà indicato nell'appendice \ref{sec:Qualita2} quali aspetti vengono ritenuti rilevanti e il profilo di qualità atteso per ognuno di loro. 
	In ogni caso, tutte le componenti del prodotto devono essere sottoposte a valutazione.
    }
}
