\section{Introduzione}{
    \subsection{Scopo del documento}{
	Questo documento ha lo scopo di illustrare le strategie adottate per implementare i processi di verifica e validazione del lavoro svolto da 
	\textit{GoGo Team} per assicurare la qualità del progetto \textbf{MyTalk} e dei processi volti alla sua produzione. In futuro il presente documento
	potrà essere aggiornato in seguito a scelte progettuali del gruppo e/o variazione dei requisiti da parte del \textit{Proponente}.
    }
    
    \subsection{Scopo del prodotto}{
      Il prodotto \textbf{MyTalk} è volto ad offrire la possibilità agli utenti di comunicare tra loro, trasmettendo il segnale audio e video, attraverso 
      il browser$_{|g|}$ mediante l'utilizzo di soli componenti standard, senza che sia necessario installare plugin$_{|g|}$ o programmi aggiuntivi 
      (\textit{es.} Skype). Attualmente, infatti, la comunicazione istantanea tra utenti avviene solo tramite componenti non presenti di default nei
      browser$_{|g|}$. Il software$_{|g|}$ dovrà risiedere in una singola pagina web$_{|g|}$ e dovrà essere basato sulla tecnologia WebRTC$_{|g|}$ 
      (\url {http://www.webrtc.org}).
    }

    \subsection{Glossario}{
	Al fine di migliorare la comprensione al lettore ed evitare ambiguità rispetto ai termini tecnici utilizzati nel documento, viene allegato il file
	\emph{\Glossario},  nel quale vengono descritti i termini contrassegnati dal simbolo $_{|g|}$ alla fine della parola.
	Per i termini composti da più parole, oltre al simbolo $_{|g|}$, è presente anche la sottolineatura. 
    }

    \subsection{Riferimenti}{
	\subsubsection{Normativi}{
	    \begin{itemize}
		\item Capitolato d'appalto: \textbf{MyTalk}, \textit{software}$_{|g|}$ \textit{di comunicazione tra utenti senza requisiti  di installazione}, rilasciato dal 
		      proponente \textit{Zucchetti S.p.A.}, reperibile all'indirizzo \url{http://www.math.unipd.it/~tullio/IS-1/2012/Progetto/C1.pdf}.
		\item Analisi dei requisiti (allegato \textit{\AnalisiDeiRequisiti}).
		\item Norme di progetto (allegato \textit{\NormeDiProgetto}).
	    \end{itemize}
	}
	\subsubsection{Informativi}{
	    \begin{itemize}
		\item Software Engeneering - Chapter 24: Quality Management, Chapter 26: Process Improvement -- Ian Sommerville - 9$^{th}$ Edition (2009).
		\item Materiale del corso di Ingegneria del Software 2012-2013 -- Prof. Tullio Vardanega e Riccardo Cardin
		      (\url {http://www.math.unipd.it/~tullio/IS-1/2012/}).
		\item \texttt{ISO/IEC$_{|g|}$ 9126:2001} (inglobato da \texttt{ISO/IEC$_{|g|}$ 25010:2011}): 
		      Systems and software engineering -- Systems and software Quality Requirements and Evaluation (SQuaRE) 
		      -- System and software quality models  (\url{http://www2.cnipa.gov.it/site/_contentfiles/01379900/1379951_ISO\%209126.pdf}).
		\item \texttt{ISO/IEC$_{|g|}$ 12207:2008}: Software Life Cycle Processes (\url{http://en.wikipedia.org/wiki/ISO/IEC_12207}).
		\item \texttt{ISO/IEC$_{|g|}$ 14598:2001} (inglobato da \texttt{ISO/IEC$_{|g|}$ 25040:2011}): Systems and software engineering -- 
		      Systems and software Quality Requirements and Evaluation (SQuaRE) -- Evaluation process
		      (\url{http://www2.cnipa.gov.it/site/_contentfiles/01379900/1379952_ISO\%2014598.pdf}).
		\item \texttt{ISO/IEC$_{|g|}$ 15504:1998}: Information Tecnology -- Process Assessment, conosciuto come SPICE (Software Process Improvement and 
		      Capability dEtermination) (\url{http://www2.cnipa.gov.it/site/_contentfiles/00310300/310320_15504.pdf}).
		\item \texttt{ISO/IEC$_{|g|}$ 15939:1998}: Software Engineering -- Software measurement process (2002) (\url{http://www2.cnipa.gov.it/site/_contentfiles/01379900/1379952_ISO\%2014598.pdf}).
	    \end{itemize}

	}
    }
}
