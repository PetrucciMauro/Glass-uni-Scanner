\section{Risorse}{
    \subsection{Risorse necessarie}{
	  L'attività di verifica richiederà l'utilizzo di risorse umane, hardware$_{|g|}$ e software$_{|g|}$.
	  \subsubsection{Risorse umane}{
	      I ruoli per assicurare la qualità del prodotto sono i seguenti:
	      \begin{itemize}
		  \item \textbf{Responsabile di Progetto}: supervisiona la qualità dei processi interni e coordina le attività di verifica. \`E
			responsabile nei confronti del committente della corretta realizzazione del prodotto e valuta le proposte di modifica avanzate
			dal Verificatore assegnando alla persona opportuna il compito di correzione.
		  \item \textbf{Amministratore di Progetto}: coordina le attività di verifica, definisce i piani di gestione di qualità, stabilisce
			le norme da seguire per la risoluzione di anomalie e discrepanze. 
		  \item \textbf{Verificatore}: esegue l'attività di verifica su ogni prodotto seguendo il metodo black-box utilizzando i test definiti dal Programmatore,
			riassume gli esiti di tali attività e, in caso di discrepanze, li presenta al Responsabile di Progetto e al Programmatore stesso.
		  \item \textbf{Programmatore}: si occupa della stesura del codice, della creazione dei test, della loro esecuzione e nel caso siano presenti errori, del loro debugging$_{|g|}$ con successiva correzione. Deve inoltre correggere gli errori riscontrati dai Verificatori nel lavoro da lui svolto.
	      \end{itemize}
	  }

	  \subsubsection{Risorse hardware}{
	      Saranno necessari:
	      \begin{itemize}
		  \item computer con installato software$_{|g|}$ necessario allo sviluppo del progetto in tutte le sue fasi;
		  \item luogo fisico in cui incontrarsi per lo sviluppo del progetto, possibilmente con una connessione ad Internet.
	      \end{itemize}
	  }

	  \subsubsection{Risorse software}{
	      Durante la fase di realizzazione del progetto saranno necessari:
	      \begin{itemize}
		  \item strumenti che consentano l'analisi statica del codice per poter misurare le metriche descritte in sezione \ref{sec:Metriche};
		  \item framework$_{|g|}$ per eseguire test di unità;
		  \item strumenti per automatizzare i test;
		  \item debugger$_{|g|}$ per i linguaggi di programmazione scelti;
		  \item browser$_{|g|}$ come piattaforma di testing dell'applicazione da sviluppare;
		  \item piattaforma di versionamento per la creazione e la gestione di ticket$_{|g|}$.
	      \end{itemize}
	  }
    }
}
