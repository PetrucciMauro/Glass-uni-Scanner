\section{Definizione del prodotto}

\subsection{Metodo e formalismo di specifica}
\noindent La specifica di questo documento è realizzata per fornire un'idea generale di quali dovranno essere
le componenti di alto livello da definire dettagliatamente nella prossima fase di progettazione. \\
I contenuti della specifica sono esposti seguendo l'approccio top-down$_{|g|}$ : dalla descrizione generale del sistema si scende sempre più in dettaglio passando alla descrizione delle singole componenti. \\Infine vengono descritti i design pattern utilizzati e come essi sono stati applicati.\\
Si è scelto di utilizzare i diagrammi dei package$_{|g|}$, delle classi, di attività e di sequenza relativi allo standard UML$_{|g|}$ 2.0 specificati alla sezione 6 di Norme di Progetto (\textit{\NormeDiProgetto}).

\subsection{Architettura generale del sistema}\label{archSys}
\noindent Il sistema si basa su un sistema client-server dove il servizio, situato in gran parte sul client, utilizza il server per comunicare con gli altri utenti e per accedere alla base di dati.\\
	\begin{figure}[h!tbp]
		\centering
		\includegraphics[scale=0.70]{\docsImg NetworkDiagram.pdf}
		\caption{Schema generale client-server.}
	\end{figure}
\noindent Studiando le architetture più comuni per questo tipo di sistemi abbiamo scelto di utilizzare il design pattern architetturale MVP (Model - View - Presenter) nella variante Passive View, per entrambe le tipologie di utenti che fruiranno del servizio.\\ \\ 
L'architettura ad alto livello dell'applicazione è rappresentata dalla figura \ref{fig:mvpAbstract}.\\
	\begin{figure}[h!tbp]
		\centering
		\includegraphics[scale=0.70]{\docsImg ModelViewPresenter.pdf}
		\caption{Schema generale del design pattern utilizzato e della sua ripartizione tra client\g~ e server\g .}		
		\label{fig:mvpAbstract}
	\end{figure}
Viene ora fatta un'analisi delle funzionalità che ciascuna parte dovrà adempiere, mentre la struttura di questo pattern viene analizzata nella sezione \ref{patternMVP}.

\subsubsection{Componente View}
\noindent La View del sistema è divisa in due parti:
\begin{itemize}
\item un'interfaccia utente: per la registrazione, l'autenticazione, la modifica dei dati dell'utente e la visualizzazione della comunicazione e delle relative statistiche;
\item un'interfaccia amministratore: per l'autenticazione e per la visualizzazione delle statistiche sulle chiamate effettuate dagli utenti.
\end{itemize}
\noindent Entrambe le parti sono interfacce web$_{|g|}$ e si è scelto di utilizzare HTML5$_{|g|}$ per le componenti statiche e JavaScript$_{|g|}$ per quelle dinamiche. Il layout$_{|g|}$ delle interfacce viene gestito tramite CSS3$_{|g|}$ .

\subsubsection{Componente Presenter}
\noindent Il Presenter, che rappresenta la Application Logic dell'applicazione, risiede per la maggior parte sul client\g~ e in parte minore sul server\g .\\
Tale componente svolge quattro ruoli fondamentali:
\begin{itemize}
\item comprendere ed elaborare gli input dell'utente;
\item instaurare e gestire la comunicazione;
\item permettere la comunicazione tra client\g~ e server\g ;
\item aggiornare la View con i dati ottenuti dal model.
\end{itemize}
\noindent Per poterlo fare è dotato di alcune caratteristiche:
\begin{itemize}
\item conosce le componenti Model e View. Fa da unico collegamento tra le parti permettendo il loro totale disaccoppiamento;
\item è in grado di elaborare gli input della View in strutture logiche che gli permettano di utilizzarli successivamente.
\end{itemize}
\noindent Il Presenter del sistema è diviso in due parti: amministratore e utente.\\
Le principali funzionalità offerte dal Presenter lato amministratore sono:
\begin{itemize}
\item gestire gli eventi di accesso e di uscita dal servizio provenienti dalla componente View;
\item gestire gli eventi generati dalla componente View, restituendo la modellazione dei dati reperiti dalla componente Model del server.
\end{itemize}
\noindent Le principali funzionalità offerte dal Presenter lato utente sono:
\begin{itemize}
\item gestire gli eventi di accesso e di uscita dal servizio provenienti dalla componente View;
\item controllare i dati inseriti dall'utente in fase di registrazione per verificarne la correttezza;
\item ricevere e gestire gli eventi riguardanti la visualizzazione e la modifica dei dati dell'utente;
\item gestire gli eventi riguardanti la comunicazione, ricevendo i dati dalla componente Model e aggiornando la View.
\item gestire la comunicazione peer-to-peer$_{|g|}$ grazie alle librerie WebRTC$_{|g|}$ ;
\item contenere e gestire le informazioni riguardanti le sessioni, sia quella locale che quella remota. Ciò avviene grazie alle librerie WebRTC\g;
\item gestire le informazioni riguardanti gli stream$_{|g|}$ audio e video che permettono agli utenti di comunicare; anche questo viene fatto grazie alle librerie WebRTC$_{|g|}$ .
\end{itemize}

\subsubsection{Componente Model}
\noindent Il Model, che rappresenta la Business Logic dell'applicazione,  risiede nel server$_{|g|}$ e in minima parte nel client$_{|g|}$.\\
La principale funzione svolta dal Model lato client$_{|g|}$ è:
\begin{itemize}
\item tener traccia dell'utente attualmente autenticato, se questo esiste;
\end{itemize}
\noindent Le principali funzioni svolte dal Model lato server$_{|g|}$ sono:
\begin{itemize}
\item ricevere, gestire e soddisfare le richieste che arrivano dal Presenter della componente amministratore e utente;
\item interfacciarsi con il database\g.
\end{itemize}