\section{Introduzione}{
    \subsection{Scopo del documento}{
	Questo documento ha lo scopo di definire l'architettura del prodotto \textbf{\mytalk}. Tale definizione inizia descrivendo il prodotto ad alto livello, dopo la quale segue un'analisi a basso livello, applicando quindi un approccio top-down\g.
    }
    
    \subsection{Scopo del prodotto}{
      Il prodotto \textbf{\mytalk} è volto ad offrire la possibilità agli utenti di comunicare tra loro, trasmettendo il segnale audio e video, attraverso 
      il browser\g~ mediante l'utilizzo di soli componenti standard, senza che sia necessario installare plugin\g~ o programmi aggiuntivi 
      (\textit{es.} Skype). Attualmente, infatti, la comunicazione istantanea tra utenti avviene solo tramite componenti non presenti di default nei
      browser\g . Il software\g~ dovrà risiedere in una singola pagina web\g~ e dovrà essere basato sulla tecnologia WebRTC\g~ 
      (\url {http://www.webrtc.org}).
    }

    \subsection{Glossario}{
	Al fine di migliorare la comprensione al lettore ed evitare ambiguità rispetto ai termini tecnici utilizzati nel documento, viene allegato il file
	\emph{\Glossario},  nel quale vengono descritti i termini contrassegnati dal simbolo \g~ alla fine della parola.
	Per i termini composti da più parole, oltre al simbolo \g, è presente anche la sottolineatura. 
    }

    \subsection{Riferimenti}{
	\subsubsection{Normativi}{
	    \begin{itemize}
		\item Capitolato d'appalto: \textbf{\mytalk}, \textit{software}\g~ \textit{di comunicazione tra utenti senza requisiti  di installazione}, rilasciato dal 
		      proponente \textit{Zucchetti S.p.A.}, reperibile all'indirizzo \url{http://www.math.unipd.it/~tullio/IS-1/2012/Progetto/C1.pdf}.
		\item Analisi dei requisiti (allegato \textit{\AnalisiDeiRequisiti}).
		\item Piano di qualifica (allegato \textit{\PianoDiQualifica}).
		\item Piano di progetto (allegato \textit{\PianoDiProgetto}).
		\item Norme di progetto (allegato \textit{\NormeDiProgetto}).
	    \end{itemize}
	}
	\subsubsection{Informativi}{
	    \begin{itemize}
		\item Ingegneria del software - Ian Sommerville - 8$^a$ 2007
		\item UML Distilled - Martin Fowler - 4$^a$ 2010
		\item Design Patterns - Erich Gamma, Richard Helm, Ralph Johnson, John Vlissides - 1$^a$ 2002 
		\item MVP: Model-View-Presenter - Mike Potel - 1996 reperibile al sito \url{http://www.wildcrest.com/Potel/Portfolio/mvp.pdf}
		\item WebRTC - Alan B. Johnston, Daniel C. Burnett - 1$^a$ 2012
		\item Slide del docente per l'anno accademico \uniAA~ reperibili al sito \url{http://www.math.unipd.it/~tullio/IS-1/2012/}
		
	    \end{itemize}
	}
    }
}
