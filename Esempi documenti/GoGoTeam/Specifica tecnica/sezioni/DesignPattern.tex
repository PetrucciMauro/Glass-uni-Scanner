\section{Design pattern}{
Nell'architettura del sistema \textbf{MyTalk} sono stati impiegati dei design pattern che forniscono delle soluzioni note, corrette e ripetute di alcuni problemi affrontati.

	\subsection{Adapter}{
\paragraph{Descrizione: }{il Design Pattern Adapter permette a classi diverse di operare anche quando utilizzano interfacce incompatibili. Il suo fine è di evitare la scrittura di parti del sistema già implementate in librerie esterne permettendone l'interoperabilità.\\
	\begin{figure}[h!tbp]
		\centering
		\includegraphics[scale=0.80]{\docsImg adapter.pdf}
		\caption{Diagramma delle classi del design pattern Adapter in formato UML\g .}
	\end{figure}
	}
	\paragraph{Giustificazione: }{la scelta di questo Design Pattern dipende dal fatto che le librerie WebRTC\g~ sono mutabili nel tempo in quanto non ancora standardizzate. L'adapter permetterà di modificare solo i metodi della classe senza invalidare l'interfaccia pubblica della classe adaptee.
	}
	\paragraph{Applicazione: }{il design pattern è utilizzato all'interno del \nolinkurl{package mytalk.client.presenter}.	
	}
	}

\subsection{Data Access Object}{
\paragraph{Descrizione: }{il DAO (Data Access Object) è un design pattern che riduce l'accoppiamento tra la business logic e la persistenza dei dati: la business logic necessita spesso di dati che sono memorizzati in un database. Il pattern permette all'applicazione di essere indipendente da eventuali cambiamenti nel database utilizzato.\\
\begin{figure}[h!tbp]
		\centering
		\includegraphics[scale=0.80]{\docsImg dao.pdf}
		\caption{Diagramma delle classi del design pattern DAO in formato UML\g .}
	\end{figure}
Ogni classe della business logic offre i metodi set e get per operare sui campi dati ai quali non è possibile avere l'accesso dall'esterno della classe; offre inoltre almeno i metodi:
\begin{itemize}
	\item Leggi dati da database;
	\item Inserisci dati da database;
	\item Aggiorna dati da database;
	\item Cancella dati da database;
\end{itemize}
eventualmente ridefiniti per consentire la realizzazione di diverse politiche di accesso al database.\\
La figura \ref{fig:daoSeq} illustra il diagramma di sequenza che mostra l'interazione tra i partecipanti di questo pattern.

\begin{figure}[h!tbp]
		\centering
		\includegraphics[scale=0.80]{\docsImg daoSeq.pdf}
		\caption{Diagramma di sequenza del design pattern DAO in formato UML\g .}
		\label{fig:daoSeq}
\end{figure}

\begin{itemize}
	\item Business Object: rappresenta la classe che richiede l'accesso ai dati del database;
	\item Data Access Object: fornisce un'astrazione per l'implementazione dell'accesso ai dati del database sottostante per fornire al business object un accesso trasparente;
	\item Data Source: rappresenta l'implementazione della sorgente dei dati che può essere un database, un file system, una repository ecc;
	\item Transfer Object: rappresenta l'oggetto utilizzato per il trasporto dei dati che può essere utilizzato sia per la scrittura sul database che per il recupero dei dati.
\end{itemize}
	}
	\paragraph{Giustificazione: }{
	\begin{itemize}
		\item[] {Vantaggi:
	\begin{itemize}
		\item Permette la trasparenza: il business object può accedere al database senza conoscerne nello specifico l'implementazione;
	\item Semplifica la migrazione verso un'implementazione diversa del database, che coinvolge solo lo strato del DAO;
	\item Riduce la complessità del codice del business object aumentandone quindi la comprensibilità e la manutenibilità;
	\item Centralizza tutto l'accesso ai dati in un livello separato.
	\end{itemize}
	}
		\item[] {Svantaggi:
	\begin{itemize}
	\item Aggiunge un ulteriore livello: il DAO crea un livello dizionale tra il client e il database che deve essere progettato ed implementato. I benefici giustificano però ampiamente questo costo.
	\end{itemize}
	}
	\end{itemize}
	}
	\paragraph{Applicazione: }{il design pattern è utilizzato completamente all’interno del package \nolinkurl{mytalk.server.model.dao}.	
	}
}


\subsection{MVP}\label{patternMVP} {
	\paragraph{Descrizione: }{il Model View Presenter (MVP) prevede di strutturare un'applicazione in tre componenti:
	\begin{itemize}
		\item Model: individua la rappresentazione dei dati dell'applicazione e delle politiche di accesso e modifica;
		\item View: si occupa di rappresentare l'interfaccia grafica con la quale l'utente interagirà con il Presenter;
		\item Presenter: identifica la parte logica del prodotto facendo da tramite tra le richieste inoltrate dalla View al Model ed aggiornando la View.
	\end{itemize}
	\begin{figure}[h!tbp]
		\centering
		\includegraphics[scale=0.80]{\docsImg mvp.pdf}
		\caption{Diagramma del design pattern MVP in formato UML\g .}
		\label{fig:mvpClassi}
\end{figure}
	Il prodotto MyTalk è sviluppato secondo la variante del pattern denominata "Passive View".
Essa prevede che sia il Presenter ad aggiornare la view, comunicando tramite la sua interfaccia, per riflettere i cambiamenti nel model; l'interazione con quest'ultimo è quindi gestita esclusivamente dal Presenter: la view non è consapevole dei cambiamenti del model.\\
La figura \ref{fig:mvpSeq} illustra il diagramma di sequenza per il design pattern MVP.
	\begin{figure}[h!tbp]
		\centering
		\includegraphics[scale=0.70]{\docsImg mvpSeq.pdf}
		\caption{Diagramma di sequenza del design pattern MVP in formato UML\g .}
		\label{fig:mvpSeq}
\end{figure}
}
	\paragraph{Giustificazione: }{
	durante lo sviluppo di un'architettura complessa che richiede l'implementazione di un'interfaccia utente elaborata, si corre il rischio che nella GUI venga inserita della logica che dovrebbe risiedere in altri strati dell'applicazione. Ciò comporta i seguenti svantaggi:
	\begin{itemize}
		\item Il codice nello strato dell'interfaccia grafica è molto difficile da testare senza eseguire manualmente il programma o mantenere degli script che automatizzino l'interazione con la GUI\g ;
		\item Duplicazione del codice che gestisce la logica di interfacce utente simili;
		\item Riduce la manutenibilità e la comprensibilità del codice.
	\end{itemize}
Per evitare questi problemi viene adottato il design pattern MVP che permette di separare l'interfaccia grafica (View), l'application logic (Presenter) e la business logic (Model).\\
Questa scelta implementativa comporta inoltre una più agevole progettazione in quanto le comunicazioni tra le parti sono incentrate nella componente Presenter. Questo approccio porta ai seguenti vantaggi:
\begin{itemize}
	\item Indipendenza tra le componenti Model e View;	
	\item Separazione dei ruoli e delle relative interfacce;
	\item Semplice supporto per nuove funzionalità alla tipologia di utente;
	\item Ruolo cardine della componente Presenter che semplifica la progettazione e la manutenzione.
\end{itemize}

\'E stata scelta la variante Passive View in quanto migliora la testabilità, poiché la logica della GUI\g~ può essere testata direttamente dal Presenter.
	
	}
	\paragraph{Applicazione: }{rappresenta la struttura portante dell'intera applicazione nonché la sua architettura generale come si può determinare alla sezione \ref{archSys}.
}
		}

}

