\section{Specifica della componente View} {
%solo client
\begin{sloppypar}
	\subsection{Package mytalk.client.view.user} {
		\begin{itemize}

		% ILogUser - inizio
		\item[•] \textbf{ILogUser}
		\begin{itemize}

			\item[] \textbf{Funzione:}\\
				Offre operazioni alle classi che compongono la GUI\g~ per l'autenticazione degli utenti e interagiscono con essa.

			\item[] \textbf{Relazioni con altre componenti:}\\
				L'interfaccia è implementata da:
				\begin{itemize}
					\item[] \path{mytalk.client.view.user.LogUser}.
				\end{itemize}
				L’interfaccia è utilizzata da:
				\begin{itemize}
					\item[] \path{mytalk.client.view.user.PageUserView}.
				\end{itemize}

			\item[] \textbf{Metodi:}\\
				\texttt{+ void loadViewLogUser();}\\
				Visualizza la GUI\g~ di autenticazione e seleziona le impostazioni di default\g~, i campi vuoti e l'errore (non visibile).\\

				\texttt{+ void removeViewLogUser();}\\
				Nasconde la GUI\g~ di autenticazione.\\

				\texttt{+ void errorView(String error);}\\
				Imposta e rende visibile il messaggio di errore.\\
		\end{itemize}
		% ILogUser - fine
		
		% IRegister - inizio
		\item[•] \textbf{IRegister}
		\begin{itemize}
			\item[] \textbf{Funzione:}\\
				Offre operazioni alle classi che compongono la GUI\g~ per la registrazione di nuovi utenti e interagiscono con essa.\\
				
			\item[] \textbf{Relazioni con altre componenti:}\\
				L'interfaccia è implementata da:
				\begin{itemize}
					\item[] \path{mytalk.client.view.user.Register}. 
				\end{itemize}
				L'interfaccia è utilizzata da:
				\begin{itemize}
					\item[] \path{mytalk.client.view.user.PageUserView}. 
				\end{itemize}
				
			\item[] \textbf{Metodi:}\\
				\texttt{+ void loadViewRegister();}\\
				Visualizza la GUI\g~ di registrazione.\\
				
				\texttt{+ void removeViewRegister();}\\
				Nasconde la GUI\g~ di registrazione.\\
				
				\texttt{+ void errorView(Vector<String> messages);}\\
				Imposta e rende visibili i messaggi di errore da segnalare all’utente.\\
		\end{itemize}
		% IRegister - fine
		
		% IUserDataView - inizio
		\item[•] \textbf{IUserDataView}
		\begin{itemize}
			\item[] \textbf{Funzione:}\\
				  Offre operazioni alle classi che compongono la GUI\g~ per la gestione dei dati di un utente e interagiscono con essa.\\
				
			\item[] \textbf{Relazioni con altre componenti:}\\
				L'interfaccia è implementata da:
				\begin{itemize}
					\item[] \path{mytalk.client.view.user.UserDataView}. 
				\end{itemize}
				L'interfaccia è utilizzata da:
				\begin{itemize}
					\item[] \path{mytalk.client.view.user.PageUserView}. 
				\end{itemize}
				
			\item[] \textbf{Metodi:}\\
				\texttt{+ void loadViewUserData();}\\
				Visualizza la GUI\g~ che permette all'utente di visualizzare i propri dati.\\
				
				\texttt{+ void removeViewUserData();}\\
				Nasconde la GUI\g~ di visualizzazione e modifica dati.\\
				
				\texttt{+ void errorView(Vector<String> messages);}\\
				Imposta e rende visibili i messaggi di errore da segnalare all’utente.\\
				
				\texttt{+ void setLabel(Vector<String> data);}\\
				Imposta i campi dei dati dell'utente estrapolati dal vettore \texttt{data}.\\
				
				\texttt{+ void windowClosing();}\\
				Gestisce l'evento di chiusura della finestra della GUI\g~ di gestione dei dati utente effettuando il logout.\\
		\end{itemize}
		% IUserDataView - fine
		
		% IShowDataUser - inizio
		\item[•] \textbf{IShowDataUser}
		\begin{itemize}
			\item[] \textbf{Funzione:}\\
				  Offre operazioni alle classi che compongono la GUI\g~ per la visualizzazione dei dati di un utente e interagiscono con essa.\\
				
			\item[] \textbf{Relazioni con altre componenti:}\\
				L'interfaccia è implementata da:
				\begin{itemize}
					\item[] \path{mytalk.client.view.user.ShowDataUser}. 
				\end{itemize}
				L'interfaccia è utilizzata da:
				\begin{itemize}
					\item[] \path{mytalk.client.view.user.UserDataView}. 
				\end{itemize}
				
			\item[] \textbf{Metodi:}\\
				\texttt{+ void dataRequest();}\\
				Invia una richiesta al Presenter per recuperare i dati relativi all’utente.\\
		\end{itemize}
		% IShowDataUser - fine

		% IModifyDataUser - inizio
		\item[•] \textbf{IModifyDataUser}
		\begin{itemize}
			\item[] \textbf{Funzione:}\\
				  Offre operazioni alle classi che compongono la GUI\g~ per la modifica dei dati di un utente e interagiscono con essa.\\
				
			\item[] \textbf{Relazioni con altre componenti:}\\
				L'interfaccia è implementata da:
				\begin{itemize}
					\item[] \path{mytalk.client.view.user.ModifyDataUser}. 
				\end{itemize}
				L'interfaccia è utilizzata da:
				\begin{itemize}
					\item[] \path{mytalk.client.view.user.UserDataView}. 
				\end{itemize}
				
			\item[] \textbf{Metodi:}\\
				\texttt{+ void sendRequest(Vector<String> data);}\\
				Invia una richiesta per sostituire i vecchi dati utente con quelli contenuti nel vettore \texttt{data}.\\
		\end{itemize}
		% IModifyDataUser - fine
		
		% LogUser - inizio
		\item[•] \textbf{LogUser}
		\begin{itemize}
			\item[] \textbf{Funzione:}\\
				  La classe ha il compito di:
				\begin{itemize}
					\item aggiornare opportunamente la GUI\g~ di autenticazione;
					\item gestire gli eventi che l’utente o il sistema possono innescare inoltrando le richieste all'interfaccia \texttt{ILogUserLogic};
					\item gestire i link presenti nella pagina.
				\end{itemize}
				
			\item[] \textbf{Relazioni con altre componenti:}\\
				Implementa l'interfaccia:
				\begin{itemize}
					\item[] \path{mytalk.client.view.user.ILogUser}. 
				\end{itemize}
				Usa le classi:
				\begin{itemize}
					\item[] \path{mytalk.client.presenter.user.logicUser.LogUserLogic};
					\item[] \path{mytalk.client.presenter.user.logicUser.UpdateViewLogic}.
				\end{itemize}
				Tramite le interfacce:
				\begin{itemize}
					\item[] \path{mytalk.client.presenter.user.logicUser.ILogUserLogic}; 
					\item[] \path{mytalk.client.presenter.user.logicUser.IUpdateViewLogic}.
				\end{itemize}

			\item[] \textbf{Attributi:}\\
				\texttt{- IUpdateViewLogic updateViewLogic;}\\
				Riferimento alla classe \texttt{UpdateViewLogic}.\\
				
				\texttt{- ILogUserLogic logUserLogic;}\\
				Riferimento alla classe \texttt{LogUserLogic}.\\

			\item[] \textbf{Oggetti:}\\
				\texttt{@UiField HTMLPanel MyDivLogin;}\\
				Pannello contente \texttt{FormLogUser} e \texttt{LinkRegistrati}.\\
				
				\texttt{@UiField FormPanel FormLogUser;}\\
				Form contenente \texttt{BoxUtente}, \texttt{BoxPassword}, \texttt{LabelError} e \texttt{LogUserSubmit}.\\
				
				\texttt{@UiField TextBox BoxUtente;}\\
				Campo per l'inserimento del nome utente.\\
				
				\texttt{@UiField TextBox BoxPassword;}\\
				Campo per l'inserimento della password.\\
				
				\texttt{@UiField InlineLabel LabelError;}\\
				Label che in presenza di errore segnala il tipo di errore.\\
				
				\texttt{@UiField PushButton LogUserSubmit;}\\
				Bottone per l'invio della richiesta controllo delle credenziali di autenticazione.\\
				
				\texttt{@UiField InlineHyperlink LinkRegistrati;}\\
				Link per passare alla grafica della registrazione.\\
				
			\item[] \textbf{Metodi:}\\
				\texttt{+ LogUser(IUpdateViewLogic updateViewLogic, IWebSocketUser webSocket);}\\
				Costruttore: inizializza \texttt{updateViewLogic} e \texttt{logUserLogic} con i valori ricevuti come parametri, imposta gli attributi degli oggetti che compongono la GUI\g~. Imposta come invisibile \texttt{LabelError}.\\
				
				\texttt{+ void loadViewLogUser();}\\
				Visualizza la GUI\g~ di autenticazione e seleziona le impostazioni di default\g~: \texttt{BoxUtente} e \texttt{BoxPassword} vuoti e \texttt{LabelError} invisibile.\\
				
				\texttt{+ void removeViewLogUser();}\\
				Nasconde la GUI\g~ di login.\\
				
				\texttt{+ void errorView(String error);}\\
				Imposta e rende visibile \texttt{LabelError} inserendo il contenuto della stringa \texttt{error}.\\
				
			\item[] \textbf{Eventi:}\\
				\texttt{@UiHandler void onLinkRegistratiClick(ClickEvent event);}\\
				All’evento \texttt{Click} dell’utente richiama il metodo \texttt{removeViewLogUser()} e, attraverso il riferimento \texttt{updateViewLogic}, richiama il metodo \texttt{loadViewRegister()}.\\
				
				\texttt{@UiHandler void onLogUserSubmitClick(ClickEvent event);}\\
				All'evento \texttt{Click} dell'oggetto \texttt{LogUserSubmit} i dati inseriti all'interno dei campi \texttt{BoxUtente} e \texttt{BoxPassword} vengono inseriti in un vettore e inviati attraverso il riferimento \texttt{logUserLogic} al metodo \texttt{validateData(Vector<String>)} per effettuare il controllo dei dati di autenticazione.\\
		\end{itemize}
		% LogUser - fine

		% Register - inizio
		\item[•] \textbf{Register}
		\begin{itemize}
			\item[] \textbf{Funzione:}\\
				  La classe ha il compito di:
				\begin{itemize}
					\item aggiornare opportunamente la GUI\g~ di registrazione;
					\item gestire gli eventi che l’utente o il sistema possono innescare inoltrando le richieste all'interfaccia \texttt{IRegisterLogic};
					\item gestire i link presenti nella pagina.
				\end{itemize}
				
			\item[] \textbf{Relazioni con altre componenti:}\\
				Implementa l'interfaccia:
				\begin{itemize}
					\item[] \path{mytalk.client.view.user.IRegister}. 
				\end{itemize}
				Usa le classi:
				\begin{itemize}
					\item[] \path{mytalk.client.presenter.user.logicUser.RegisterLogic}.;
					\item[] \path{mytalk.client.presenter.user.logicUser.UpdateViewLogic}.
				\end{itemize}
				Tramite le interfacce:
				\begin{itemize}
					\item[] \path{mytalk.client.presenter.user.logicUser.IRegisterLogic}; 
					\item[] \path{mytalk.client.presenter.user.logicUser.IUpdateViewLogic}.
				\end{itemize}

			\item[] \textbf{Attributi:}\\
				\texttt{- IUpdateViewLogic updateViewLogic;}\\
				Riferimento alla classe \texttt{UpdateViewLogic}.\\
				
				\texttt{- IRegisterLogic registerLogic;}\\
				Riferimento alla classe \texttt{RegisterLogic}.\\

			\item[] \textbf{Oggetti:}\\
				\texttt{@UiField HTMLPanel MyDiv;}\\
				Pannello contente \texttt{FormRegister} e \texttt{LinkLogin}.\\
				
				\texttt{@UiField FormPanel FormRegister;}\\
				Form contenente \texttt{GridRegistrati}.\\
				
				\texttt{@UiField Grid GridRegistrati;}\\
				Griglia contenente \texttt{BoxEMail}, \texttt{LabelErrorBoxEMail}, \texttt{BoxNome}, \texttt{LabelErrorBoxNome}, \texttt{BoxCognome}, \texttt{LabelErrorBoxCognome}, \texttt{BoxPassword}, \texttt{LabelErrorBoxPassword}, \texttt{BoxRipPass}, \texttt{LabelErrorBoxRipPass}, \texttt{BoxAzienda}, \texttt{LabelErrorBoxAzienda}, \texttt{BoxTelefono}, \texttt{LabelErrorBoxTelefono} e \texttt{ButtonRegistratiSubmit}.\\
				
				\texttt{@UiField TextBox BoxEMail;}\\
				Campo per l'inserimento dell'e-mail.\\
				
				\texttt{@UiField InlineLabel LabelErrorBoxEMail;}\\
				Label che contiene l’errore relativo all’e-mail errata.\\
				
				\texttt{@UiField TextBox BoxNome;}\\
				Campo per l'inserimento del nome.\\
				
				\texttt{@UiField InlineLabel LabelErrorBoxNome;}\\
				Label che contiene l’errore relativo al nome errato.\\

				\texttt{@UiField TextBox BoxCognome;}\\
				Campo per l'inserimento del cognome.\\
				
				\texttt{@UiField InlineLabel LabelErrorBoxCognome;}\\
				Label che contiene l’errore relativo al cognome errato.\\

				\texttt{@UiField TextBox BoxPassword;}\\
				Campo per l'inserimento della password.\\
				
				\texttt{@UiField InlineLabel LabelErrorBoxPassword;}\\
				Label che contiene l’errore relativo alla password errata.\\
				
				\texttt{@UiField TextBox BoxRipPass;}\\
				Campo per l'inserimento della password di conferma.\\
				
				\texttt{@UiField InlineLabel LabelErrorBoxRipPass;}\\
				Label che contiene l’errore relativo alla non uguaglianza delle due password inserite.\\
				
				\texttt{@UiField TextBox BoxAzienda;}\\
				Campo per l'inserimento dell'azienda.\\
				
				\texttt{@UiField InlineLabel LabelErrorBoxAzienda;}\\
				Label che contiene l’errore relativo al nome dell'azienda errato.\\
				
				\texttt{@UiField TextBox BoxTelefono;}\\
				Campo per l'inserimento del numero di telefono.\\
				
				\texttt{@UiField InlineLabel LabelErrorBoxTelefono;}\\
				Label che contiene l’errore relativo al numero di telefono errato.\\
				
				\texttt{@UiField InlineHyperlink LinkLogin;}\\
				Link per passare alla GUI\g~ di autenticazione.\\
				
				\texttt{@UiField PushButton ButtonRegistratiSubmit;}\\
				Bottone per l'invio della richiesta di controllo delle credenziali relative alla registrazione.\\
				
			\item[] \textbf{Metodi:}\\
				\texttt{+ Register(IUpdateViewLogic updateViewLogic, IWebSocketUser webSocket);}\\
				Costruttore: inizializza \texttt{updateViewLogic} e \texttt{registerLogic} con i valori ricevuti come parametri, imposta gli attributi degli oggetti che compongono la GUI\g~. Imposta  invisibili le \texttt{labelError}.\\
				
				\texttt{+ void loadViewRegister();}\\
				Visualizza la GUI\g~ di registrazione.\\
				
				\texttt{+ void removeViewRegister();}\\
				Nasconde la GUI\g~ di registrazione.\\
				
				\texttt{+ void errorView(Vector<String> messages);}\\
				Copia le stringhe di errore contenute nel vettore \texttt{messages} nei corrispondenti campi \texttt{LabelError}, che successivamente visualizza.\\
				
			\item[] \textbf{Eventi:}\\
				\texttt{@UiHandler void onLinkLoginClick(ClickEvent event);}\\
				All’evento \texttt{Click} dell’utente richiama il metodo \texttt{removeViewRegister()} e, attraverso il riferimento \texttt{updateViewLogic}, richiama il metodo \texttt{loadViewLogUser()}.\\
				
				\texttt{@UiHandler void onButtonRegistratiSubmitClick(ClickEvent event);}\\
				All'evento \texttt{Click} dell'oggetto \texttt{ButtonRegistratiSubmit} i dati inseriti all'interno dei campi \texttt{BoxEMail}, \texttt{BoxNome}, \texttt{BoxCognome}, \texttt{BoxPassword}, \texttt{BoxRipPass}, \texttt{BoxAzienda} e \texttt{BoxTelefono} vengono inseriti in un vettore e inviati attraverso il riferimento \texttt{registerLogic} al metodo \texttt{validateData(Vector<String>)}, il quale effettua il controllo dei dati di registrazione.\\
		\end{itemize}
		% Register - fine
		
		% UserDataView - inizio
		\item[•] \textbf{UserDataView}
		\begin{itemize}
			\item[] \textbf{Funzione:}\\
				  La classe ha il compito di:
				\begin{itemize}
					\item aggiornare opportunamente la GUI\g~ di gestione dei dati dell'utente autenticato;
					\item gestire gli eventi che l’utente o il sistema possono innescare inoltrando le richieste di visualizzazione all'interfaccia \texttt{IShowDataUser} e le richieste di modifica dei dati all'interfaccia \texttt{IModifyDataUser};
					\item gestire i link presenti nella pagina.
				\end{itemize}
				
			\item[] \textbf{Relazioni con altre componenti:}\\
				Implementa l'interfaccia:
				\begin{itemize}
					\item[] \path{mytalk.client.view.user.IUserDataView}. 
				\end{itemize}
				Usa le classi:
				\begin{itemize}
					\item[] \path{mytalk.client.view.user.ShowDataUser};
					\item[] \path{mytalk.client.view.user.ModifyDataUser};
					\item[] \path{mytalk.client.presenter.user.logicUser.DataUserLogic};
					\item[] \path{mytalk.client.presenter.user.logicUser.UpdateViewLogic}.
				\end{itemize}
				Tramite le interfacce:
				\begin{itemize}
					\item[] \path{mytalk.client.view.user.IShowDataUser};
					\item[] \path{mytalk.client.view.user.IModifyDataUser};
					\item[] \path{mytalk.client.presenter.user.logicUser.IDataUserLogic};
					\item[] \path{mytalk.client.presenter.logicUser.IUpdateViewLogic}.
				\end{itemize}

			\item[] \textbf{Attributi:}\\
				\texttt{- IUpdateViewLogic updateViewLogic;}\\
				Riferimento alla classe \texttt{UpdateViewLogic}.\\
				
				\texttt{- IDataUserLogic dataUserLogic;}\\
				Riferimento alla classe \texttt{DataUserLogic}.\\
				
				\texttt{- IShowDataUser showData;}\\
				Riferimento alla classe \texttt{ShowDataUser}.\\
				
				\texttt{- IModifyDataUser modifyData;}\\
				Riferimento alla classe \texttt{ModifyDataUser}.\\
				
				\texttt{- boolean checkModify;}\\
				Indica se l'utente desidera modificare la password.\\
				
				\texttt{- String attPassword;}\\
				Password attuale.\\

			\item[] \textbf{Oggetti:}\\
				\texttt{@UiField HTMLPanel MyDiv;}\\
				Pannello contente \texttt{FormShowData}, \texttt{FormModifyData}, \texttt{LabelUser}, \texttt{LinkCommunication} e \texttt{LinkLogout}.\\
				
				\texttt{@UiField InlineHyperlink LinkCommunication;}\\
				Link per passare alla GUI\g~ della comunicazione.\\
				
				\texttt{@UiField InlineHyperlink LinkLogout;}\\
				Link per passare alla GUI\g~ dell'autenticazione effettuando il logout.\\
				
				\texttt{@UiField InlineLabel LabelUser;}\\
				Label che visualizza lo username dell'utente autenticato.\\
				
				\texttt{@UiField CaptionPanel DatiUtente;}\\
				Pannello contenente la \texttt{FormShowData}.\\
				
				\texttt{@UiField FormPanel FormShowData;}\\
				Form contenente \texttt{LabelNome}, \texttt{LabelCognome}, \texttt{LabelEMail}, \texttt{LabelAzienda}, \texttt{LabelTelefono} e \texttt{ButtonModificaSubmit}.\\

				\texttt{@UiField InlineLabel LabelNome;}\\
				Label contenente il nome dell'utente.\\
				
				\texttt{@UiField InlineLabel LabelCognome;}\\
				Label contenente il cognome dell'utente.\\
				
				\texttt{@UiField InlineLabel LabelEmail;}\\
				Label contenente l'e-mail dell'utente.\\
				
				\texttt{@UiField InlineLabel LabelAzienda;}\\
				Label contenente il nome dell'azienda dell'utente.\\
				
				\texttt{@UiField InlineLabel LabelTelefono;}\\
				Label contenente il numero di telefono dell'utente.\\

				\texttt{@UiField PushButton ButtonModificaSubmit;}\\
				Bottone per passare al \texttt{CaptionPanel} \texttt{ModificaDati}.\\

				\texttt{@UiField CaptionPanel ModificaDati;}\\
				Pannello conetenente \texttt{FormModifyData}.\\

				\texttt{@UiField FormPanel FormModifyData;}\\
				Form contenente \texttt{BoxNome}, \texttt{BoxCognome}, \texttt{BoxEMail}, \texttt{BoxAzienda}, \texttt{BoxTelefono}, \texttt{BoxOldPass}, \texttt{BoxModifyCheck}, \texttt{BoxNewPass}, \texttt{BoxConfPass}, \texttt{LabelErrorNome}, \texttt{LabelErrorCognome}, \texttt{LabelEmailError}, \texttt{LabelAziendaError}, \texttt{LabelTelefonoError}, \texttt{LabelOldError}, \texttt{LabelNewError}, \texttt{LabelConfPassError}, \texttt{ButtonConfermaSubmit} e \texttt{ButtonIndietroSubmit}.\\

				\texttt{@UiField TextBox BoxNome;}\\
				Campo per l'inserimento del nome.\\

				\texttt{@UiField InlineLabel LabelErrorNome;}\\
				Label che contiene l’errore relativo al nome errato.\\
				
				\texttt{@UiField TextBox BoxCognome;}\\
				Campo per l'inserimento del cognome.\\

				\texttt{@UiField InlineLabel LabelErrorCognome;}\\
				Label che contiene l’errore relativo al cognome errato.\\
				
				\texttt{@UiField TextBox BoxEmail;}\\
				Campo per l'inserimento dell'e-mail.\\

				\texttt{@UiField InlineLabel LabelErrorEmail;}\\
				Label che contiene l’errore relativo all'e-mail errata.\\
				
				\texttt{@UiField TextBox BoxNome;}\\
				Campo per l'inserimento del nome.\\

				\texttt{@UiField InlineLabel LabelErrorNome;}\\
				Label che contiene l’errore relativo al nome errato.\\
				
				\texttt{@UiField TextBox BoxAzienda;}\\
				Campo per l'inserimento del nome dell'azienda.\\

				\texttt{@UiField InlineLabel LabelErrorAzienda;}\\
				Label che contiene l’errore relativo al nome dell'azienda.\\
				
				\texttt{@UiField TextBox BoxTelefono;}\\
				Campo per l'inserimento del numero di telefono.\\

				\texttt{@UiField InlineLabel LabelErrorTelefono;}\\
				Label che contiene l’errore relativo al numero di telefono errato.\\

				\texttt{@UiField TextBox BoxOldPass;}\\
				Campo per l'inserimento della password.\\

				\texttt{@UiField InlineLabel LabelOldError;}\\
				Label che contiene l’errore relativo alla password attuale errata.\\

				\texttt{@UiField SimpleCheckBox BoxModifyCheck;}\\
				Controllo per attivare la modifica della password.\\

				\texttt{@UiField TextBox BoxNewPass;}\\
				Campo per l'inserimento della nuova password.\\

				\texttt{@UiField InlineLabel LabelNewError;}\\
				Label che contiene l’errore relativo al formato della nuova password errato.\\

				\texttt{@UiField TextBox BoxConfPass;}\\
				Campo per l'inserimento della password di conferma.\\

				\texttt{@UiField InlineLabel LabelConfPassError;}\\
				Label che contiene l’errore relativo alla non uguaglianza delle due password inserite.\\

				\texttt{@UiField PushButton ButtonConfermaSubmit;}\\
				Bottone di invio dei dati al Presenter per il controllo del formato e della loro conformità.\\

				\texttt{@UiField PushButton ButtonIndietroSubmit;}\\
				Bottone per passare al \texttt{CaptionPanel} \texttt{DatiUtente}.\\

			\item[] \textbf{Metodi:}\\
				\texttt{+ UserDataView(IUpdateViewLogic updateViewLogic, IWebSocketUser webSocket);}\\
				Costruttore:
				\begin{itemize}
					\item inizializza \texttt{updateViewLogic} e \texttt{dataUserLogic} con i valori ricevuti come parametri;
					\item imposta \texttt{showData} e \texttt{modifyData} con il valore del riferimento \texttt{dataUserLogic};
					\item imposta gli attributi degli oggetti che compongono la GUI\g~ e imposta \texttt{checkModify} a \texttt{false};
					\item imposta gli attributi degli oggetti che compongono la GUI\g~ e imposta \texttt{checkModify} a \texttt{false}; 
					\item imposta invisibili tutte le \texttt{LabelError} e anche \texttt{DatiUtente} e \texttt{ModificaDati}.
				\end{itemize}
				
				\texttt{+ void loadViewUserData();}\\
				Visualizza la GUI\g~ di gestione dei dati utente e richiama il metodo \texttt{graphicInitialCharge()}.\\
				
				\texttt{+ void removeViewUserData();}\\
				Nasconde la GUI\g~ di gestione dei dati utente.\\
				
				\texttt{+ void setUsernameLabel(String username);}\\
				Imposta l'etichetta \texttt{LabelUser} con il nome dell'utente autenticato passato attraverso la stringa \texttt{username}.\\
				
				\texttt{+ void errorView(Vector<String> messages);}\\
				Copia le stringhe di errore contenute nel vettore \texttt{messages} nei corrispondenti campi \texttt{LabelError}, che successivamente visualizza.\\
				
				\texttt{+ void setLabel(Vector<String> data);}\\
				Imposta e rende visibili le \texttt{Label} con i dati attuali dell'utente contenuti nel vettore \texttt{data}.\\
				
				\texttt{- void graphicInitialCharge();}\\
				Imposta la GUI di gestione dei dati utente allo stato iniziale. Tale stato è il seguente:
				\begin{itemize}
					\item \texttt{DatiUtente}: visibile;
					\item \texttt{ModificaDati}: non visibile;
					\item tutte le \texttt{LabelError}: non visibili;
				\end{itemize}
				Inoltre, attraverso il riferimento \texttt{showData}, richiama il metodo \texttt{dataRequest()} che crea una richiesta per ottenere i dati dell'utente.\\

				\texttt{+ void windowClosing();}\\
				Gestisce l'evento di chiusura della finestra della GUI\g~ di gestione dei dati. Richiama il metodo \texttt{logoutUser()} attraverso il riferimento \texttt{dataUserLogic}.\\
				
			\item[] \textbf{Eventi:}\\
				\texttt{@UiHandler void onLinkCommunicationClick(ClickEvent event);}\\
				All'evento \texttt{Click} dell'oggetto \texttt{LinkCommunication} viene richiamato il metodo \texttt{removeViewUserData()} e attraverso il riferimento \texttt{udateViewLogic} richiama \texttt{loadViewCommunication()}.\\
				
				\texttt{@UiHandler void onLinkLogoutClick(ClickEvent event);}\\
				All'evento \texttt{Click} dell'oggetto \texttt{LinkLogout} viene richiamato il metodo \texttt{removeViewUserData()} il quale utilizza il riferimento \texttt{dataUserLogic} per effettuare il logout e richiama la grafica dell’interfaccia di autenticazione attraverso il metodo \texttt{logoutUser()}.\\

				\texttt{@UiHandler void onButtonModificaSubmitClick(ClickEvent event);}\\
				All'evento \texttt{Click} dell'oggetto \texttt{ButtonModificaSubmit} è reso visibile il caption \texttt{ModificaDati} e reso invisibile il caption \texttt{DatiUtente}.\\
				
				\texttt{@UiHandler void onButtonConfermaSubmitClick(ClickEvent event);}\\
				All'evento \texttt{Click} dell'oggetto \texttt{ButtonConfermaSubmit} i dati inseriti all'interno dei campi \texttt{BoxEMail}, \texttt{BoxNome}, \texttt{BoxCognome}, \texttt{BoxOldPass}, \texttt{BoxNewPass}, \texttt{BoxConfPass}, \texttt{BoxAzienda}, \texttt{BoxTelefono} e \texttt{attPassword} vengono inseriti in un vettore e inviati attraverso il riferimento \texttt{modifyData} al metodo \texttt{sendRequest(Vector<String>)} che permette di inviare la richiesta, controllare i dati inviati e, nel caso siano corretti, di modificarli.\\
				
				\texttt{@UiHandler void onButtonIndietroSubmitClick(ClickEvent event);}\\
				All'evento \texttt{Click} dell'oggetto \texttt{ButtonIndietroSubmit} è reso visibile il caption \texttt{DatiUtente} e reso invisibile il caption \texttt{ModificaDati}.\\
				
				\texttt{@UiHandler void onBoxModifyCheckClick(ClickEvent event);}\\
				All'evento \texttt{Click} dell'oggetto \texttt{BoxModifyCheck} il campo \texttt{checkModify} viene impostato con il valore opposto a quello corrente (es. se era \texttt{true} diventa \texttt{false}) e \texttt{BoxNewPass}, \texttt{BoxConfPass} vengono resi invisibili o visibili a seconda del valore di \texttt{checkModify}.\\
		\end{itemize}
		% UserDataView - fine
			
		% ShowDataUser - inizio
		\item[•] \textbf{ShowDataUser}
		\begin{itemize}
			\item[] \textbf{Funzione:}\\
				  La classe ha il compito di inoltrare al Presenter le richieste per la visualizzazione dei dati utente.\\
				
			\item[] \textbf{Relazioni con altre componenti:}\\
				Implementa l'interfaccia:
				\begin{itemize}
					\item[] \path{mytalk.client.view.user.IShowDataUser}.
				\end{itemize}
				Usa le classi:
				\begin{itemize}
					\item[] \path{mytalk.client.presenter.user.logicUser.DataUserLogic};
				\end{itemize}
				Tramite le interfacce:
				\begin{itemize}
					\item[] \path{mytalk.client.presenter.user.logicUser.IDataUserLogic}.
				\end{itemize}

			\item[] \textbf{Attributi:}\\
				\texttt{- IDataUserLogic dataUserLogic;}\\
				Riferimento alla classe \texttt{DataUserLogic}.\\
				
			\item[] \textbf{Metodi:}\\
				\texttt{+ ShowDataUser(IDataUserLogic dataUserLogic);}\\
				Costruttore: inizializza \texttt{dataUserLogic} con il valore ricevuto come parametro.\\
				
				\texttt{+ void dataRequest();}\\
				Invia una richiesta, attraverso il riferimento \texttt{dataUserLogic}, al metodo \texttt{getDataUser()} per reperire i dati dell'utente.\\
		\end{itemize}
		% ShowDataUser - fine
		
		% ModifyDataUser - inizio
		\item[•] \textbf{ModifyDataUser}
		\begin{itemize}
			\item[] \textbf{Funzione:}\\
				  La classe ha il compito di inoltrare al Presenter le richieste per la modifica dei dati utente.\\
				
			\item[] \textbf{Relazioni con altre componenti:}\\
				Implementa l'interfaccia:
				\begin{itemize}
					\item[] \path{mytalk.client.view.user.IModifyDataUser}.
				\end{itemize}
				Usa le classi:
				\begin{itemize}
					\item[] \path{mytalk.client.presenter.user.logicUser.DataUserLogic};
				\end{itemize}
				Tramite le interfacce:
				\begin{itemize}
					\item[] \path{mytalk.client.presenter.user.logicUser.IDataUserLogic}.
				\end{itemize}

			\item[] \textbf{Attributi:}\\
				\texttt{- IDataUserLogic dataUserLogic;}\\
				Riferimento alla classe \texttt{DataUserLogic}.\\
				
			\item[] \textbf{Metodi:}\\
				\texttt{+ ModifyDataUser(IDataUserLogic dataUserLogic);}\\
				Costruttore: inizializza \texttt{dataUserLogic} con il valore ricevuto come parametro.\\
				
				\texttt{+ void sendRequest(Vector <String> dati);}\\
				Invia una richiesta, attraverso il riferimento \texttt{dataUserLogic}, al metodo \texttt{checkNewData(Vector<String>)} che controlla che i nuovi dati dell'utente siano scritti in modo corretto e, quindi, li salva o ritorna un messaggio di errore.\\
		\end{itemize}
		% ModifyDataUser - fine

		% ICommunicationView - inizio
		\item[•] \textbf{ICommunicationView}
			\begin{itemize}
				\item \textbf{Funzione:} \\ 
					Interfaccia che offre operazioni alle classi che compongono la GUI\g~ della comunicazione e interagiscono con essa. \\

				\item  \textbf{Relazioni con altre componenti:} \\
					L’interfaccia è implementata da:
					\begin{itemize}
						\item \path{mytalk.client.view.user.CommunicationView}.
					\end{itemize}
					L’interfaccia è utilizzata da
					\begin{itemize}
						\item \path{mytalk.client.view.user.PageUserView}.
					\end{itemize}
				\item[] \textbf{Metodi:} \\
					\texttt{+ void loadViewCommunication();} \\
					Visualizza la GUI\g~ che permette all’utente di gestire le comunicazioni.\\
					\texttt{+ void removeViewCommunication();} \\
					Nasconde la GUI\g~ di comunicazione.\\
					\texttt{+ void setUsenameLabel(String username);}\\
					Imposta l'etichetta \texttt{LabelUser} con il nome dell'utente autenticato.\\
					\texttt{+ void errorView();}\\
					Imposta e rende visibili i messaggi di errore da segnalare all’utente.\\
					\texttt{+ void graphicInitialCharge();}\\
					Imposta i pannelli da visualizzare e le impostazioni di default. \\
					\texttt{+ void loadListBox(Vector <String> utenti);}\\
					Inserisce gli utenti registrati contenuti nel vettore utenti all’interno del pannello 
					\texttt{ListBox}.\\
					\texttt{+ void setVideoLocal(String url);}\\
					Imposta il video locale con l’indirizzo contenuto nella stringa url.\\
					\texttt{+ void waitView(String utente);}\\
					Visualizza il pannello, per il mittente, in attesa di risposta del destinatario.\\
					\texttt{+ void externalCall(String s);}\\
					Visualizza il pannello di chiamata in entrata.\\
					\texttt{+ void callActive(String remoteVideo, String nome);}\\
					Visualizza il pannello di informazioni di chiamata e imposta il video remoto con la stringa 
					\texttt{remoteVideo} e con il nome dell’utente con cui si è instaurata la comunicazione.\\
					\texttt{+ void updateFormInfoChiamata(Vector<Double> datiChiamata);}\\
					Visualizza le statistiche delle chiamate contenute nel vettore \texttt{datiChiamata}.\\
					\texttt{+ void windowClosing();}\\
					Gestisce l'evento di chiusura della finestra della GUI\g~ di comunicazione effettuando il logout e in caso di comunicazione attiva termina la chiamata.\\
			\end{itemize}
			% ICommunicationView - fine
		
		% CommunicationView - inizio
		\item[•] \textbf{CommunicationView}
			\begin{itemize}
				\item[]  \textbf{Funzione:} \\
				La classe ha il compito di:
				\begin{itemize}
					\item aggiornare opportunamente la GUI\g~ di comunicazione;
					\item gestire gli eventi che l'utente o il sistema possono innescare: inoltra 
					le richieste di comunicazione all’interfaccia \texttt{IMakeCall}, gestisce 
					gli eventi di comunicazione attraverso \texttt{ICallManage} e inoltra gli 
					eventi che gestriscono il video e l’audio a \texttt{IMediaManage};
					\item gestire i link presenti nella pagina.
				\end{itemize}
				\item[]  \textbf{Relazioni con altre componenti:} \\
				Implementa l’interfaccia:
				\begin{itemize}
					\item[] \path{mytalk.client.view.user.ICommunicationView}.
				\end{itemize}

				Usa le classi:
				\begin{itemize}
					\item[] \path{mytalk.client.view.user.MediaManage};
					\item[] \path{mytalk.client.view.user.CallManage};
					\item[] \path{mytalk.client.view.user.MakeCall};
					\item[] \path{mytalk.client.presenter.client.user.logicUser.CommunicationLogic};
					\item[] \path{mytalk.client.presenter.client.user.logicUser.UpdateViewLogic}.
				\end{itemize}
			
				Tramite le interfacce:
				\begin{itemize}
					\item[] \path{mytalk.client.view.user.IMediaManage};
					\item[] \path{mytalk.client.view.user.ICallManage};
					\item[] \path{mytalk.client.view.user.IMakeCall};
					\item[] \path{mytalk.client.presenter.client.user.logicUser.ICommunicationLogic};
					\item[] \path{mytalk.client.presenter.client.user.logicUser.IUpdateViewLogic}.
				\end{itemize}
				\item[]  \textbf{Attributi:} \\
					\texttt{- IUpdateViewLogic updateViewLogic}: riferimento alla classe \texttt{UpdateViewLogic}. 
						Permette di richiamare i metodi di aggiornamento della View.\\
					\texttt{- ICommunicationLogic communicationLogic}: riferimento alla classe 
						\texttt{CommunicationrLogic}. Viene utilizzato per richiamare il metodo che imposta il 
						nome dell’utente e per richiamare la GUI\g~ di \texttt{UserData}.\\
					\texttt{- ICallManage callManage}: riferimento alla classe \texttt{CallManager}. Utilizzato 
						per la gestione delle comunicazioni.\\
					\texttt{- IMakeCall makeCall}: riferimento alla classe \texttt{MakeCall}. Utilizzato per la 
						gestione della negoziazione di una comunicazione.\\
					\texttt{- IMediaManage mediaManage}: riferimento alla classe \texttt{MediaManager}. Utilizzato 
						per l’attivazione del video locale.\\
					\texttt{- boolean video}: valore booleano che indica lo stato di attivazione del video.	\\
					\texttt{- int count}: valore per contare il numero di richieste di conversazione.\\
					\texttt{- TimerCall t}: timer che effettua un conto alla rovescia per l’attesa della risposta 
						al momento dell’invio di una richiesta di comunicazione.\\
					\texttt{- RadioButton voto1}: radio button per assegnare un giudizio di 1.\\
					\texttt{- RadioButton voto2}: radio button per assegnare un giudizio di 2.\\
					\texttt{- RadioButton voto3}: radio button per assegnare un giudizio di 3.\\
					\texttt{- RadioButton voto4}: radio button per assegnare un giudizio di 4.\\
					\texttt{- RadioButton voto5}: radio button per assegnare un giudizio di 5.\\

				\item[] \textbf{Oggetti:}\\
				\begin{itemize}
					\item[] \texttt{@UiField HTMLPanel MyDiv}: pannello contenente \texttt{DivVideoCommunication} e 
						\texttt{DivPanelControl};
					\item[] \texttt{@UiField HTMLPanel DivVideoCommunication}: pannello contenente \texttt{DivVideo},
						  e \texttt{DivVideoLocal};
					\item[] \texttt{@UiField HTML DivVideo}: codice HTML\g~ per la grafica del video remoto, attivo 
						e inattivo.
					\item[] \texttt{@UiField HTML DivVideoLocal}: codice HTML\g~ per la grafica del video locale, 
						attivo e inattivo.
					\item[] \texttt{@UiField HTMLPanel DivPanelControl}: pannello contenente \texttt{LinkDatiUtente}, 
						\texttt{LinkLogout}, \texttt{LabelUser}, \texttt{TabBar}, \texttt{FormListaUtenti}, 
						\texttt{FormRicercaUtente}, \texttt{FormRicercaIP}, \texttt{FormRicezioneChiamata}, 
						\texttt{FormInfoChiamata}, \texttt{CaptionChiamataInCorso} e \texttt{FormActivateLocalVideo}.
					\item[] \texttt{@UiField InlineHyperlink LinkDatiUtente}: link per passare alla GUI\g~ di 
						gestione dati utente.
					\item[] \texttt{@UiField InlineHyperlink LinkLogout}: link per passare alla GUI\g~ 
						dell'autenticazione effettuando il logout.
					\item[] \texttt{@UiField InlineLabel LabelUser}: label che visualizza lo username dell'utente 
						autenticato.
					\item[] \texttt{@UiField Grid TabBar}: griglia contenente \texttt{LinkListaUtenti}, 
						\texttt{LinkRicercaUtente} e \texttt{LinkRicercaIP}.
					\item[] \texttt{@UiField InlineHyperlink LinkListaUtenti}: link per visualizzare la 
						\texttt{FormListaUtenti}.
					\item[] \texttt{@UiField FormPanel FormListaUtenti}: form contenente \texttt{GridBoxUtenti}.
					\item[] \texttt{@UiField Grid GridBoxUtenti}: griglia contenente \texttt{ListBoxUtenti}, 
						\texttt{LabelErroreSelezioneUtente} e \texttt{BoxChiamaLUSubmit}.
					\item[] \texttt{@UiField ListBox ListBoxUtenti}: oggetto contenente la lista di tutti gli 
						utenti registrati.
					\item[] \texttt{@UiField InlineLabel LabelErroreSelezioneUtente}: label che contiene l’errore 
						relativo alla selezione dell’utente.
					\item[] \texttt{@UiField PushButton BoxChiamaLUSubmit}: bottone per l’invio della richiesta di 
						comunicazione con relativo controllo di stato dell’utente
					\item[] \texttt{@UiField InlineHyperlink LinkRicercaUtente}: link per visualizzare la 
						\texttt{FormRicercaUtente}
					\item[] \texttt{@UiField FormPanel FormRicercaUtente}: form contenente \texttt{BoxEMail}, 
						\texttt{LabelErroreRicerca} e \texttt{ButtonRicercaChiamaSubmit}.
					\item[] \texttt{@UiField TextBox BoxEMail}: campo per l’inserimento dell’e-mail dell’utente 
						con il quale si desidera instaurare una comunicazione.
					\item[] \texttt{@UiField InlineLabel LabelErroreRicerca}: label che contiene l’errore relativo 
						alla ricerca dell’utente.
					\item[] \texttt{@UiField PushButton ButtonRicercaChiamaSubmit}: bottone per l’invio della 
						richiesta di comunicazione con relativo controllo di esistenza e di stato.
					\item[] \texttt{@UiField InlineHyperlink LinkRicercaIP}:  link per visualizzare la 
						\texttt{FormRicercaIP}.
					\item[] \texttt{@UiField FormPanel FormRicercaIP}: form contenente \texttt{BoxRicercaIP}, 
						\texttt{LabelErroreRicercaIP} e \texttt{ButtonRicercaIPChiamaSubmit}.
					\item[] \texttt{@UiField TextBox BoxRicercaIP}: campo per l’inserimento dell’indirizzo IP\g~ 
						dell’utente con il quale si desidera instaurare una comunicazione.
					\item[] \texttt{@UiField InlineLabel LabelErroreRicercaIP}: label che contiene l’errore 
						relativo alla ricerca dell’utente tramite indirizzo IP\g~.
					\item[] \texttt{@UiField PushButton ButtonRicercaIPChiamaSubmit}: bottone per l’invio della 
						richiesta di comunicazione con relativo controllo di esistenza e di stato tramite indirizzo IP\g~.
					\item[] \texttt{@UiField FormPanel FormRicezioneChiamata}: form contenente 
						\texttt{LabelNomeUtente}, \texttt{ButtonAccettaSubmit} e \texttt{ButtonRifiutaSubmit}.
					\item[] \texttt{@UiField InlineLabel LabelNomeUtente}: label che contiene il nome del mittente 
						della richiesta di chiamata.
					\item[] \texttt{@UiField PushButton ButtonAccettaSubmit}: bottone per accettare la richiesta 
						di chiamata.
					\item[] \texttt{@UiField PushButton ButtonRifiutaSubmit}: bottone per rifiutare la richiesta 
						di chiamata.
					\item[] \texttt{@UiField FormPanel FormInfoChiamata}: pannello contenente 
						\texttt{LabelInfoUtente}, \texttt{LabelDurata}, \texttt{LabelLatenza}, \texttt{LabelByte}, 
						\texttt{LabelPack}, \texttt{LabelLostPack}, \texttt{PanelVote} e \texttt{ButtonChiudiChiamataSubmit}.
					\item[] \texttt{@UiField InlineLabel LabelInfoUtente}: label che contiene il nome dell’utente 
						coinvolto nella comunicazione.
					\item[] \texttt{@UiField InlineLabel LabelDurata}: label che contiene la durata della 
						comunicazione.
					\item[] \texttt{@UiField InlineLabel LabelLatenza}: label che contiene la latenza della 
						comunicazione.
					\item[] \texttt{@UiField InlineLabel LabelByte}: label che contiene il numero di byte inviati 
						durante la comunicazione.
					\item[] \texttt{@UiField InlineLabel LabelPack}:  label che contiene il numero di pacchetti 
						inviati durante la comunicazione.
					\item[] \texttt{@UiField InlineLabel LabelLostPack}: label che contiene il numero di pacchetti 
						persi durante la comunicazione.
					\item[] \texttt{@UiField HorizontalPanel PanelVote}: pannello orizzontale contenente i 
						\texttt{RadioButton} per ricavare l’informazione del giudizio dell’utente riguardo alla 
						comunicazione effettuata.
					\item[] \texttt{@UiField PushButton ButtonChiudiChiamataSubmit}: bottone per l’invio della 
						richiesta di chiusura della comunicazione.
					\item[] \texttt{@UiField CaptionPanel CaptionChiamataInCorso}: pannello contenente \texttt{LabelTimer}.
					\item[] \texttt{@UiField InlineLabel LabelTimer}: label che indica lo stato di attesa del 
						mittente di una richiesta di chiamata attraverso un conto alla rovescia.
					\item[] \texttt{@UiField FormPanel FormActivateLocalVideo}: form contenente \texttt{AttivaVideoSubmit}.
					\item[] \texttt{@UiField PushButton AttivaVideoSubmit}: bottone per l’attivazione del video 
						locale.
				\end{itemize}

			\item[] \textbf{Metodi:}\\
				\texttt{+ CommunicationView(IUpdateViewLogic updateViewLogic, IWebSocketUser webSocket);}\\
				Costruttore: inizializza \texttt{updateViewLogic} e \texttt{communicationLogic} con i valori ricevuti come 
				parametri, imposta inoltre \texttt{webSocket}, \texttt{makeCall}, \texttt{callManage} e 
				\texttt{mediaManage} con il valore del riferimento \texttt{communicationLogic}. Imposta gli 
				attributi degli oggetti che compongono la GUI\g~ e imposta video a false e count a 0. Imposta 
				invisibili tutte le \texttt{LabelError} e i disabilita i bottoni Chiama.
				\texttt{+ void loadViewCommunication();}\\
				Visualizza la GUI\g~ di comunicazione e richiama il metodo \texttt{graphicInitialCharge()}.\\
				\texttt{+ void removeViewCommunication();}\\
				Nasconde la GUI\g~ di comunicazione.\\
				\texttt{+ void setUsernameLabel(String username);}\\
				Imposta l'etichetta LabelUser con il nome dell'utente autenticato passato attraverso la stringa username.\\
				\texttt{+ void errorView();}\\
				Rende visibili le \texttt{LabelError}\\
				\texttt{+ void graphicInitialCharge()};\\
				Nel caso in cui più richieste di comunicazione siano presenti  tale metodo non aggiorna la GUI\g~, in caso 
				contrario imposta la GUI\g~ di comunicazione allo stato iniziale. Tale stato è il seguente:
				\begin{itemize}
					\item[-] \texttt{count}: 0
					\item[-] \texttt{timer}: reset
					\item[-] \texttt{TabBar}: visibile
					\item[-] \texttt{LinkListaUtenti}: selezionato
					\item[-] \texttt{FormListaUtenti}: visibile
					\item[-] \texttt{tutte le LabelError}: valori iniziali e invisibili
					\item[-] \texttt{FormRicercaUtente}: invisibile
					\item[-] \texttt{FormRicercaIP}: invisibile
					\item[-] \texttt{CaptionChiamataInCorso}: invisibile
					\item[-] \texttt{FormRicezioneChiamata}: invisibile
					\item[-] \texttt{FormInfoChiamata}: invisibile
					\item[-] \texttt{DivVideo}: HTML immagine di attesa
				\end{itemize}
		
				Inoltre, attraverso il riferimento \texttt{communicationLogic}, richiama il metodo \texttt{setUserList()} che crea una richiesta per ottenere la lista degli utenti.\\
				\texttt{+ void loadListBox(Vector <String> users);}\\
				Imposta la ListBox con l’elenco degli utenti registrati contenuti nel vettore \texttt{users}\\
				\texttt{+ void setVideoLocal(String url); }\\
				Imposta il tag \texttt{DivVideoLocal} con il codice HTML\g~ per il video con l’indirizzo passato attraverso la stringa \texttt{url}.
				Inoltre rende invisibile \texttt{FormActivateLocalVideo}, attiva i bottoni \texttt{Chiama} e imposta video a \texttt{true}.\\
				\texttt{+ void waitView(String utente);}\\
				Rende invisibile \texttt{TabBar}, \texttt{FormListaUtenti}, \texttt{FormRicercaUtente} e \texttt{FormRicercaIP}. 
				Rende visibile \texttt{CaptionChiamataInCorso} e fa partire il conto alla rovescia per l’attesa della risposta. 
				Imposta inoltre il nome del destinatario con la stringa \texttt{utente} e incrementa \texttt{count}.\\
				\texttt{- void callRefuse(String utente);}\\
				Se il numero di chiamate in attesa è diverso da 1 allora viene solo decrementato \texttt{count}, altrimenti viene richiamato il metodo \texttt{graphicInitialCharge()} e richiamato attraverso il riferimento \texttt{callManage} il metodo \texttt{callStart(false, utente)} che dichiara che la negoziazione verso l’utente contenuto nel parametro \texttt{utente} è fallita.\\
				\texttt{+ void externalCall(String s);}\\
				Viene incrementato count (il numero di chiamate in attesa). Se il video è attivo e il count è 1 allora 
				\texttt{TabBar}, \texttt{FormListaUtenti}, \texttt{FormRicercaUtente} e \texttt{FormRicercaIP} vengono resi 
				invisibili, e viene visualizzato \texttt{FormRicezioneChiamata} impostando ad s il contenuto di \texttt{LabelNomeUtente}. 
				Nel caso in cui il video non sia attivo o il numero di chiamate in attesa sia diverso da 1 allora viene 
				visualizzato un messaggio di errore.\\
				\texttt{+ void callActive(String remoteVideo, String nome);}\\
				Blocca e reimposta il timer. Inoltre rende invisibili \texttt{FormRicezioneChiamata} e \texttt{CaptionChiamataInCorso}. 
				Rende visibile \texttt{FormInfoChiamata} impostando \texttt{LabelInfoUtente} con il parametro nome. Inoltre, 
				imposta il \texttt{DivVideo} con l’indirizzo del video remoto dell’utente remoto contenuto nel parametro \texttt{remoteVideo}.//
				\texttt{+ void updateFormInfoChiamata(Vector<Double> datiChiamata);}\\
				Imposta le Label di \texttt{FormInfoChiamata} con i parametri passati nel vettore \texttt{datiChiamata}.\\
				\texttt{+ void windowClosing();}\\
				Gestisce l'evento di chiusura della finestra della GUI\g~ della comunicazione. Richiama il metodo \texttt{close()} attraverso il riferimento \texttt{callManage} nel caso in cui ci sia una chiamata attiva e
				richiama il metodo \texttt{logoutUser()} attraverso il riferimento \texttt{dataUserLogic}.\\
	
			\item[] \textbf{Eventi:}
				\texttt{@UiHandler void onLinkCommunicationClick(ClickEvent event);}\\
				All'evento Click dell'oggetto \texttt{LinkDataUser} viene richiamato il metodo \texttt{removeViewCommunication() e attraverso il riferimento communicationLogic richiama loadViewDataUser()}. 
				Se sono presenti chiamate attive richiama, attraverso il riferimento \texttt{callManage}, il metodo \texttt{close()} e pone count a 0.\\
				\texttt{@UiHandler void onLinkLogoutClick(ClickEvent event);}\\
				All'evento Click dell'oggetto \texttt{LinkLogout} viene richiamato il metodo
				\texttt{removeViewCommunication()}. Inoltre utilizza il riferimento \texttt{communicationLogic} per 
				effettuare il logout e richiamare la GUI di autenticazione attraverso il metodo \texttt{logoutUser()}. 
				Se sono presenti chiamate attive richiama, attraverso il riferimento \texttt{callManage}, il metodo 
				\texttt{close()} e pone count a 0.\\
				\texttt{@UiHandler void onAttivaVideoSubmitClick(ClickEvent event);}\\
				All’evento Click dell’oggetto \texttt{AttivaVideoSubmit} viene richiamato, attraverso il riferimento 
				\texttt{mediaManage}, il metodo \texttt{activeVideoLocal()}.\\
				\texttt{@UiHandler void onLinkListaUtentiClick(ClickEvent event);}\\
				All'evento Click dell'oggetto \texttt{LinkListaUtenti} viene resa visibile \texttt{FormListaUtenti} e 
				vengono rese invisibili \texttt{FormRicercaUtente} e \texttt{FormRicercaIP}.\\
				\texttt{@UiHandler void onLinkRicercaUtenteClick(ClickEvent event);}\\
				All'evento Click dell'oggetto \texttt{LinkRicercaUtente} viene resa visibile la \texttt{FormRicercaUtente} 
				e vengono rese invisibili \texttt{FormListaUtenti} e \texttt{FormRicercaIP}.\\
				\texttt{@UiHandler void onLinkRicercaIPClick(ClickEvent event);}\\
				All'evento Click dell'oggetto \texttt{LinkRicercaIP} viene resa visibile la \texttt{FormRicercaIP} e 
				vengono rese invisibili \texttt{FormRicercaUtente} e \texttt{FormListaUtenti}.\\
				\texttt{@UiHandler void onBoxChiamaLUSubmitClick(ClickEvent event);}\\
				All'evento Click dell'oggetto \texttt{BoxChiamaLUSubmit}, se è stato selezionato un utente, viene inviata una richiesta di comunicazione all’utente selezionato attraverso il 
				riferimento \texttt{makeCall}, il quale richiama il metodo \texttt{searchUser(String tipo, String utente)}. 
				Nel caso non fosse stato selezionato alcun utente il metodo rende visibile \texttt{LabelErrorSelezioneUtente} 
				con il messaggio “Selezionare un utente.”\\
				\texttt{@UiHandler void onButtonRicercaChiamaSubmitClick(ClckEvent event);}\\
				All'evento Click dell'oggetto \texttt{BoxRicercaChiamaSubmit} viene inviata una 
				richiesta di comunicazione all’utente selezionato attraverso il riferimento \texttt{makeCall} che richiama 
				il metodo \texttt{searchUser(String tipo, String utente)}.\\
				\texttt{@UiHandler void onButtonRicercaIPChiamaSubmitClick(ClckEvent event);}\\
				All'evento Click dell'oggetto \texttt{BoxRicercaIPChiamaSubmit} viene inviata 
				una richiesta di comunicazione all’utente selezionato attraverso il riferimento \texttt{makeCall}, il quale 
				richiama il metodo \texttt{searchUser(String tipo, String utente)}.\\
				\texttt{@UiHandler void onButtonAccettaSubmitClick(ClickEvent event);}\\
				All'evento Click dell'oggetto \texttt{ButtonAccettaSubmit} vengono resi invisibili \texttt{FormRicezioneChiamata} 
				e \texttt{CaptionChiamataInCorso}. Inoltre, viene chiamato, attraverso il metodo \texttt{callManage}, il 
				metodo \texttt{callStart(true, utente)} che permette di instaurare la comunicazione.\\
				\texttt{@UiHandler void onButtonRifiutaSubmitClick(ClickEvent event);}\\
				All'evento Click dell'oggetto \texttt{ButtonRifiutaSubmit} chiama il metodo \texttt{graphicInitialCharge()}. 
				Inoltre, viene chiamato, attraverso il metodo \texttt{callManage}, il metodo \texttt{callStart(false, utente)} 
				che permette di rifiutare la comunicazione.\\
				\texttt{@UiHandler void onButtonChiudiChiamataSubmitClick(ClickEvent event);}\\
				All'evento Click dell'oggetto \texttt{ButtonChiudiChiamataSubmit}, count viene impostato a 0 e viene 
				richiamato, attraverso il riferimento \texttt{callManage}, il metodo \texttt{close(value)} che richiede la 
				terminazione della comunicazione inviando il voto dell’utente. Richiama, inoltre, il metodo 
				\texttt{graphicInitialCharge()}.\\
				\texttt{@UiHandler void onWindowClosing(Window.ClosingEvent event);}\\
				Alla chiusura della finestra vengono richiamati, attraverso i riferimenti \texttt{callManage} e 
				\texttt{communicationLogic}, ripettivamente i metodi \texttt{close(0)} e \texttt{logoutUser()}. 
				Il metodo \texttt{close(0)} viene richiamato solo nel caso in cui count sia maggiore di 0.
		\end{itemize}
		% CommunicationView - fine
	      
		% ICallManage - inizio
		\item[•] \textbf{ICallManage}
		\begin{itemize}
			\item[]  \textbf{Funzione:}\\
				Interfaccia che offre operazioni alle classi che compongono la GUI\g~ per la 
				gestione di una comunicazione attiva e interagiscono con essa.
		
			\item[]  \textbf{Relazion con altre componenti:} \\
				L’interfaccia è implementata da \path{mytalk.client.view.user.CallManage}.\\
				L’interfaccia è utilizzata da \path{mytalk.client.view.user.CommunicationView}.\\
					
			\item[]  \textbf{Metodi:}\\
				\texttt{+ void callStart(boolean b, String name);}\\
				Invia al Presenter, a seconda del valore \texttt{b}, la richiesta di accettazione o di rifiuto di una 
				comunicazione.\\
				\texttt{+ void close(int value);}\\
				Invia una richiesta al Presenter per chiudere la comunicazione attiva, inviando il 
				valore della valutazione dell’utente.\\
		\end{itemize}
		% ICallManage - fine
            
		% IMakeCall - inizio
		\item[•] \textbf{IMakeCall}
		\begin{itemize}
			\item[]  \textbf{Funzione:}
				Interfaccia che offre operazioni alle classi che compongono la 
				GUI\g~ per la gestione della negoziazione di una comunicazione e interagiscono con essa.
		
			\item[]  \textbf{Relazion con altre componenti:} \\
				L’interfaccia è implementata da \path{mytalk.client.view.user.MakeCall}.\\
				L’interfaccia è utilizzata da \path{mytalk.client.view.user.CommunicationView}. \\
					
			\item[]  \textbf{Metodi:}\\
				\texttt{+ void searchUser(String type, String user);}\\
				Invia una richiesta al Presenter per cercare e controllare la disponibilità dell’utente 
				desiderato.\\
		\end{itemize}
		% IMakeCall - fine
		
		% IMediaManage - inizio
		\item[•] \textbf{IMediaManage}
		\begin{itemize}
			\item[]  \textbf{Funzione:}
				Interfaccia che offre operazioni alle classi che compongono la GUI\g~ per la 
				gestione dei media per la comunicazione e interagiscono con essa.
		
			\item[]  \textbf{Relazion con altre componenti:} \\
				L’interfaccia è implementata da \path{mytalk.client.view.user.MediaManage}.\\
				L’interfaccia è utilizzata da \path{mytalk.client.view.user.CommunicationView}.\\
					
			\item[]  \textbf{Metodi:}\\
				\texttt{+ void activeVideoLocal();}\\
				Invia una richiesta al Presenter per recuperare l’indirizzo del video locale.\\
	
		\end{itemize}
		% IMediaManage - fine
		
		% CallManage - inizio
		\item[•] \textbf{CallManage}
		\begin{itemize}
			\item[]  \textbf{Funzione:} La classe ha il compito di inoltrare al Presenter le richieste per la gestione di una comunicazione attiva.
				
				\item[]  \textbf{Relazioni con altre componenti:} \\
					Implementa l’interfaccia:
					\begin{itemize}
						\item \path{mytalk.client.view.user.ICallManageUser}.
					\end{itemize}
					Usa le classi:
					\begin{itemize}
						\item \path{mytalk.client.presenter.client.user.logicUser.CommunicationLogic}.
					\end{itemize}
						
					Tramite le interfacce:
					\begin{itemize}
						\item \path{mytalk.client.presenter.client.user.logicUser.ICommunicationLogic}.
					\end{itemize}
						
				\item[] \textbf{Attributi:}\\
					\texttt{- ICommunicationLogic communicationLogic}: riferimento alla classe \texttt{CommunicationLogic}.\\
					
				\item[] \textbf{Metodi:}
					\texttt{+ CallManage(IComunicationLogic communicationLogic);}\\
					Costruttore: inizializza \texttt{communicationLogic} con il valore ricevuto come parametro.\\
					\texttt{+ void callStart(boolean b, String name);}\\
					Invia al Presenter a seconda del valore \texttt{b} la richiesta di accettazione o di rifiuto di una 
					comunicazione. Se \texttt{b} è \texttt{true} richiama, attraverso il riferimento \texttt{comunicationLogic}, il metodo 
					\texttt{accept(name)}. Nel caso contrario richiama \texttt{refuse(name)}.\\
					\texttt{+ void close(int value);}\\
					Invia una richiesta al Presenter per chiudere la comunicazione attiva, inviando il valore della 
					valutazione dell’utente.\\
		\end{itemize}
		% CallManage - fine

		% MakeCall - inizio
		\item[•] \textbf{MakeCall}
			\begin{itemize}
				\item[]  \textbf{Funzione:} \\
					La classe ha il compito di inoltrare al Presenter le richieste per la gestione della negoziazione 
					di una comunicazione.\\
					
					\item[]  \textbf{Relazioni con altre componenti:} \\
						Implementa l’interfaccia:
						\begin{itemize}
							\item \path{mytalk.client.view.user.IMakeCall}.
						\end{itemize}
						Usa le classi:
						\begin{itemize}
							\item \path{mytalk.client.presenter.client.user.logicUser.CommunicationLogic}.
						\end{itemize}
						Tramite le interfacce:
						\begin{itemize}
							\item \path{mytalk.client.presenter.client.user.logicUser.ICommunicationLogic}.
						\end{itemize}
							
					\item[]  \textbf{Attributi:}\\
						\texttt{- ICommunicationLogic communicationLogic}: riferimento alla classe \texttt{CommunicationLogic}.\\
						
					\item[] \textbf{Metodi:}\\
						\texttt{+ MakeCall(IComunicationLogic communicationLogic);}\\
						Costruttore: inizializza \texttt{communicationLogic} con il valore ricevuto come parametro.\\
						\texttt{+ void searchUser(String type, String user);}\\
						Invia una richiesta al Presenter per cercare l’utente desiderato e controllarne la disponibilità. 
						\texttt{type} indica in che chiave viene ricercato l’utente e user è la stringa per il confronto con il database.\\
			\end{itemize}
		% MakeCall - fine

		% MediaManage - inizio
		\item[•] \textbf{MediaManage}
			\begin{itemize}
				\item[]  \textbf{Funzione:} \\
				La classe ha il compito di inoltrare al Presenter le richieste per la gestione dei media per la 
				comunicazione.
				
				\item[]  \textbf{Relazioni con altre componenti:} \\
				Implementa l’interfaccia:
				\begin{itemize}
					\item \path{mytalk.client.view.user.IMediaManage}.
				\end{itemize}
				Usa le classi:
				\begin{itemize}
					\item \path{mytalk.client.presenter.client.user.logicUser.CommunicationLogic}.
				\end{itemize}					
				Tramite le interfacce:
				\begin{itemize}
					\item \path{mytalk.client.presenter.client.user.logicUser.ICommunicationLogic}.
				\end{itemize}
				
				\item[]  \textbf{Attributi:}\\
				\texttt{- ICommunicationLogic communicationLogic}: riferimento alla classe \texttt{CommunicationLogic}.\\
				
				\item[]  \textbf{Metodi:}\\
				\texttt{+ MediaManage(IComunicationLogic comunicationLogic);}\\
				Costruttore: inizializza \texttt{CommunicationLogic} con il valore ricevuto come parametro.\\

				\texttt{+ void activeVideoLocal();}\\
				Invia una richiesta al Presenter attraverso il metodo \texttt{getLocalUrl()} per recuperare l’indirizzo del 
				video locale.\\
			\end{itemize}
		% MediaManage - fine

		% IPageUserView - inizio
		\item[•] \textbf{IPageUserView}
		\begin{itemize}
			\item[] \textbf{Funzione:} \\
			Interfaccia che offre operazioni alle classi che compongono la GUI\g~ e interagiscono con essa.
			
			\item[] \textbf{Relazioni con altre componenti:} \\
			L'interfaccia è implementata da:
			\begin{itemize}
				\item \path{mytalk.client.view.user.PageUserView}.
			\end{itemize}
			L'interfaccia è utilizzata da:
			\begin{itemize}
				\item \path{mytalk.client.MyTalk};
				\item \path{mytalk.client.presenter.user.logicUser.UpdateUserView}.
			\end{itemize}
			
			\item[]  \textbf{Metodi:}\\
			\texttt{+ void updateViewLogUser(boolean update, String message);}\\
			Aggiorna la GUI\g~ a seconda del valore \texttt{update}:
			\begin{itemize}
				\item se \texttt{false}: aggiorna la GUI\g~ di autenticazione visualizzando l’errore opportuno contenuto in \texttt{message};
				\item se \texttt{true}: rende invisibile la GUI\g~ di autenticazione e visualizza la GUI\g~ di comunicazione.\\
			\end{itemize}
			
			\texttt{+ void notifyExternalCall(String s);}\\
			Notifica alla GUI\g~ di comunicazione l’arrivo di una richiesta di chiamata esterna.\\
			
			\texttt{+ void notifyRefuseCall(String s);}\\
			Notifica alla GUI\g~ di comunicazione il rifiuto di una richiesta di chiamata.\\
			
			\texttt{+ void setUserList(Vector<String> listaUtenti);}\\
			Imposta la lista dei possibili utenti con i quali instaurare una comunicazione nella GUI\g~ comunicazione.\\

			\texttt{+ void loadViewRegister();}\\
			Richiede la visualizzazione della GUI\g~ di registrazione.\\

			\texttt{+ void loadViewLogUser();}\\
			Richiede  la visualizzazione della GUI\g~ di autenticazione.\\

			\texttt{+ void updateViewRegister(boolean update, Vector<String> messages);}\\
			Aggiorna la GUI\g~ a seconda del valore \texttt{update}:
			\begin{itemize}
				\item se \texttt{false}: aggiorna la GUI\g~ di registrazione visualizzando gli errori opportuni contenuti in \texttt{messages}.
				\item se \texttt{true}: rende invisibile la GUI\g~ di registrazione e visualizza la GUI\g~ di autenticazione.\\
			\end{itemize}

			\texttt{+ void loadViewUserData();}\\
			Richiede la visualizzazione della GUI\g~ di gestione dei dati dell’utente.\\

			\texttt{+ void setLocalVideo(String url);}\\
			Invia l’indirizzo del video locale alla GUI\g~ di comunicazione.\\

			\texttt{+ void setUsernameLabel(String username);}\\
			Invia il nome utente alla GUI\g~ di comunicazione e alla GUI\g~ di gestione dei dati dell’utente.\\

			\texttt{+ void loadViewCommunication();}\\
			Richiede la visualizzazione della GUI\g~ di comunicazione.\\

			\texttt{+ void updateViewDataUser(boolean update, Vector<String> messages);}\\
			Aggiorna la GUI\g~ a seconda del valore \texttt{update}:
			\begin{itemize}
				\item se \texttt{false}: aggiorna la GUI\g~ di modifica dei dati dell’utente visualizzando gli errori opportuni contenuti in messages;
				\item se \texttt{true}: visualizza il pannello di visualizzazione dei dati dell’utente.\\
			\end{itemize}

			\texttt{+ void setUserDataLabel(Vector<String> userData);}\\
			Invia l’elenco dei dati dell’utente alla GUI\g~ di visualizzazione dei dati.\\

			\texttt{+ void callActive(String remoteVideo, String nome);}\\
			Notifica alla GUI\g~ di comunicazione l’inizio di una conversazione, inviandole l’indirizzo del video remoto e lo username dell’utente remoto.\\

			\texttt{+ void updateFormInfoChiamata(Vector<Double> datiChiamata);}\\
			Invia i dati delle statistiche alla GUI\g~ di comunicazione.\\

			\texttt{+ void updatePanelSearch(boolean b, String utente);}\\
			Aggiorna la GUI\g~ a seconda del valore \texttt{b}:
			\begin{itemize}
				\item se \texttt{false}: visualizza il pannello di attesa di risposta da parte dell’utente in fase di negoziazione;
				\item se \texttt{true}: aggiorna la grafica di comunicazione visualizzado l’errore opportuno.\\
			\end{itemize}

		\end{itemize}
		% IPageUserView - fine

		% PageUserView - inizio
		\item[•] \textbf{PageUserView}
		\begin{itemize}
			\item[] \textbf{Funzione:} \\
			La classe ha il compito di smistare le richieste di aggiornamento della View provenienti dal Presenter.
			
			\item[] \textbf{Relazioni con altre componenti:} \\
			Implementa l'interfaccia:
			\begin{itemize}
				\item \path{mytalk.client.view.user.IPageUserView}.
			\end{itemize}
			Usa le classi:
			\begin{itemize}
				\item \path{mytalk.client.view.user.LogUserView};
				\item \path{mytalk.client.view.user.RegisterView};
				\item \path{mytalk.client.view.user.UserDataView};
				\item \path{mytalk.client.view.user.CommunicationView};
				\item \path{mytalk.client.presenter.user.logicUser.UpdateViewLogic};
				\item \path{mytalk.client.presenter.user.logicUser.serverComUser.WebSocketUser}.
			\end{itemize}
			Tramite le interfacce:
			\begin{itemize}
				\item \path{mytalk.client.view.user.ILogUserView};
				\item \path{mytalk.client.view.user.IRegisterView};
				\item \path{mytalk.client.view.user.IUserDataView};
				\item \path{mytalk.client.view.user.ICommunicationView};
				\item \path{mytalk.client.presenter.user.logicUser.IUpdateViewLogic};
				\item \path{mytalk.client.presenter.user.serverComUser.IWebSocketUser}.
			\end{itemize}
			
			\item[] \textbf{Attributi:}\\
				\texttt{- ILogUser logUser}: riferimento alla classe LogUser.\\
				\texttt{- IRegister register}: riferimento alla classe Register.\\
				\texttt{- IUserDataView userDataView}: riferimento alla classe UserDataView.\\
				\texttt{- ICommunicationView communication}: riferimento alla classe CommunicationView.\\
			
			\item[] \textbf{Metodi:}\\
			\texttt{+ PageUserView()}\\
			Costruttore: inizializza, carica nel pannello di root e imposta la visibilità degli oggetti:
			\begin{itemize}
				\item logUser;
				\item register;
				\item userDataView;
				\item communication.
			\end{itemize}
			Imposta poi un listener che attende l’evento di chiusura della pagina e che in tal caso richiama i metodi di chiusura degli oggetti \texttt{userDataView} e \texttt{communication}.\\
			
			\texttt{+ void updateViewLogUser(boolean update, String message);}\\
			Aggiorna la GUI\g~ a seconda del valore \texttt{update}:
			\begin{itemize}
				\item se \texttt{false}: richiama il metodo \texttt{errorView(message)} attraverso il riferimento \texttt{logUser} per visualizzare gli errori di autenticazione;
				\item se \texttt{true}: richiama il metodo \texttt{removeViewLogUser()} attraverso il riferimento \texttt{logUser} e successivamente richiama il metodo \texttt{loadViewCommunication()} attraverso il riferimento \texttt{communication}.\\
			\end{itemize}
			
			\texttt{+ void notifyExternalCall(String s);}\\
			Richiama il metodo \texttt{externalCall(s)} attraverso il riferimento \texttt{communication} per notificare una richiesta di comunicazione in arrivo.\\
			
			\texttt{+ void notifyRefuseCall(String s);}\\
			Richiama il metodo \texttt{graphicInitialCharge()} attraverso il riferimento \texttt{communication} per notificare il rifiuto di una richiesta di comunicazione.\\
			
			\texttt{+ void setUserList(Vector<String> listaUtenti);}\\
			Richiama il metodo \texttt{loadListBox(listaUtenti)} attraverso il riferimento \texttt{communication} per impostare la lista utenti della comunicazione.\\

			\texttt{+ void loadViewRegister();}\\
			Richiama il metodo \texttt{loadViewRegister()} attraverso il riferimento \texttt{register} per visualizzare la registrazione.\\

			\texttt{+ void loadViewLogUser();}\\
			Richiama il metodo \texttt{loadViewLogUser()} attraverso il riferimento \texttt{logUser} per visualizzare l’autenticazione\\

			\texttt{+ void updateViewRegister(boolean update, Vector<String> messages);}\\
			Aggiorna la GUI\g~ a seconda del valore \texttt{update}:
			\begin{itemize}
				\item se \texttt{false}: richiama il metodo \texttt{errorView(messages)} attraverso il riferimento \texttt{register} per visualizzare gli errori di registrazione.
				\item se \texttt{true}: richiama il metodo \texttt{removeViewRegister()} attraverso il riferimento \texttt{register} e successivamente richiama il metodo \texttt{loadViewLogUser()} attraverso il riferimento \texttt{logUser}.\\
			\end{itemize}

			\texttt{+ void loadViewUserData();}\\
			Richiama il metodo \texttt{loadViewUserData()} attraverso il riferimento \texttt{userDataView} per visualizzare la gestione dai dell’utente.\\

			\texttt{+ void setLocalVideo(String url);}\\
			Richiama il metodo \texttt{setLocalVideo(url)} attraverso il riferimento \texttt{communication} per impostare il video locale.\\

			\texttt{+ void setUsernameLabel(String username);}\\
			Richiama il metodo \texttt{setUsenameLabel(username)} attraverso i riferimenti \texttt{userDataView} e \texttt{communication} per impostare il nome dell’utente autenticato.\\

			\texttt{+ void loadViewCommunication();}\\
			Richiama il metodo \texttt{loadViewCommunication()} attraverso il riferimento \texttt{communication} per visualizzare la GUI\g~ di comunicazione.\\

			\texttt{+ void updateViewDataUser(boolean update, Vector<String> messages);}\\
			Aggiorna la GUI\g~ a seconda del valore \texttt{update}:
			\begin{itemize}
				\item se \texttt{false}: richiama il metodo \texttt{errorView(messages)} attraverso il riferimento \texttt{updateUserView} per visualizzare gli errori di modifica dei dati;
				\item se \texttt{true}: richiama il metodo \texttt{loadViewUserData()} attraverso il riferimento \texttt{userDataView}.\\
			\end{itemize}

			\texttt{+ void setUserDataLabel(Vector<String> userData);}\\
			Richiama il metodo \texttt{setLabel(userData)} attraverso il riferimento \texttt{userDataView} per impostare le etichette che visualizzano i dati dell’utente.\\

			\texttt{+ void callActive(String remoteVideo, String nome);}\\
			Richiama il metodo \texttt{callActive(remoteVideo, nome)} attraverso il riferimento \texttt{communication} per notificare l’attivazione di una conversazione.\\

			\texttt{+ void updateFormInfoChiamata(Vector<Double> datiChiamata);}\\
			Richiama il metodo \texttt{updateFormInfoChiamata(datiChiamata)} attraverso il riferimento \texttt{communication} per impostare le etichette che visualizzano le statistiche della comunicazione.\\

			\texttt{+ void updatePanelSearch(boolean b, String utente);}\\
			Aggiorna la GUI\g~ a seconda del valore \texttt{b}:
			\begin{itemize}
				\item se \texttt{false}: richiama il metodo \texttt{errorView()} attraverso il riferimento \texttt{communication};
				\item se \texttt{true}: richiama il metodo \texttt{waitView()} che visualizza il pannello di attesa di risposta da parte dell’utente in fase di negoziazione.\\
			\end{itemize}

		\end{itemize}
		% PageUserView - fine
	\end{itemize}
	}

% ******************************************************************
% ********************* PARTE AMMINISTRATORE ***********************
% ******************************************************************

	\subsection{Package mytalk.client.view.administrator} {

	\begin{itemize}
	
		% ILogAdmin - inizio
		\item[•] \textbf{ILogAdmin}
			\begin{itemize}
				\item[]  \textbf{Funzione:} \\
				Interfaccia che offre operazioni alle classi che compongono la GUI\g~ per l'autenticazione degli amministratori e interagiscono con essa.
				
				\item[]  \textbf{Relazioni con altre componenti:} \\
				L'interfaccia è implementata da:
				\begin{itemize}
					\item \path{mytalk.client.view.administrator.LogAdmin}.
				\end{itemize}
				L’interfaccia è utilizzata da:
				\begin{itemize}
					\item \path{mytalk.client.view.adminstrator.PageAdminView}.
				\end{itemize}
				
				\item[]  \textbf{Metodi:}\\
				\texttt{+ void loadViewLogAdmin();}\\
				Visualizza la GUI\g~ di autenticazione e seleziona le impostazioni di default\g~, i campi vuoti e l'errore (non visibile).\\

				\texttt{+ void removeViewLogAdmin();}\\
				Nasconde la GUI\g~ di autenticazione.\\

				\texttt{+ void errorView(String error);}\\
				Imposta e rende visibile il messaggio di errore.\\
			\end{itemize}
		% ILogAdmin - fine
		
		% LogAdmin - inizio
		\item[•] \textbf{LogAdmin}
			\begin{itemize}
				\item[]  \textbf{Funzione:} \\
				La classe ha il compito di:
				\begin{itemize}
					\item aggiornare opportunamente la GUI\g~ di autenticazione dell’amministratore;
					\item gestire gli eventi che l’utente o il sistema possono innescare inoltrando le richieste all'interfaccia \texttt{ILogAdminLogic};
				\end{itemize}
				
				\item[]  \textbf{Relazioni con altre componenti:} \\
				Implementa l'interfaccia:
				\begin{itemize}
					\item \path{mytalk.client.view.administrator.ILogAdmin}.
				\end{itemize}
				Usa le classi:
				\begin{itemize}
					\item \path{mytalk.client.presenter.administrator.logicAdmin.LogAdminLogic}.
				\end{itemize}
				Tramite le interfacce:
				\begin{itemize}
					\item \path{mytalk.client.presenter.administrator.logicAdmin.ILogAdminLogic}.
				\end{itemize}

				\item[] \textbf{Attributi:}\\
					\texttt{- ILogAdminLogic logAdminLogic}: riferimento alla classe LogAdminLogic;\\
				
				\item[] \textbf{Oggetti:}\\
					\texttt[] \textbf{@UiField HTMLPanel MyDivLogin}: pannello contente \texttt{FormLogUser};\\
					\texttt[] \textbf{@UiField FormPanel FormLogUser}: form contenente \texttt{BoxUtente}, \texttt{BoxPassword}, \texttt{LabelError} e \texttt{LogUserSubmit};\\
					\texttt[] \textbf{@UiField TextBox BoxUtente}: campo per l'inserimento del nome utente;\\
					\texttt[] \textbf{@UiField TextBox BoxPassword}: campo per l'inserimento della password;\\
					\texttt[] \textbf{@UiField InlineLabel LabelError}: label che in presenza di errore ne segnala il tipo;\\
					\texttt[] \textbf{@UiField PushButton LogUserSubmit}: bottone per l'invio della richiesta di controllo delle credenziali di autenticazione dell’amministratore.\\
				
				\item[] \textbf{Metodi:}\\
					\texttt{+ LogAdmin(IUpdateViewLogic updateViewLogic, IWebSocketAdmin webSocket);}\\
					Costruttore: inizializza \texttt{logAdminLogic} con i valori ricevuti come parametri, imposta gli attributi degli oggetti che compongono la GUI\g~ e imposta come invisibile \texttt{LabelError}.\\

					\texttt{+ void loadViewLogAdmin();}\\
					Visualizza la GUI\g~ di autenticazione dell’amministratore e seleziona le impostazioni di default\g~: 
					\texttt{BoxUtente} e \texttt{BoxPassword} vuoti e \texttt{LabelError} invisibile.\\
					
					\texttt{+ void removeViewLogAdmin();}\\
					Nasconde la GUI\g~ di login.\\
					
					\texttt{+ void errorView(String error);}\\
					Imposta e rende visibile \texttt{LabelError} inserendo il contenuto della stringa \texttt{error}.\\
				
				\item[] \textbf{Eventi:}\\
					\texttt{@UiHandler void onLogUserSubmitClick(ClickEvent event);}\\
					All'evento \texttt{Click} dell'oggetto \texttt{LogUserSubmit} i dati inseriti all'interno dei campi \texttt{BoxUtente} e \texttt{BoxPassword} vengono inseriti in un vettore e inviati attraverso il riferimento \texttt{logAdminLogic} al metodo \texttt{validateData(Vector<String>)} per effettuare il controllo dei dati di autenticazione.\\
			\end{itemize}
		% LogAdmin - fine
		
		% IStatisticView - inizio
		\item[•] \textbf{IStatisticView}
			\begin{itemize}
				\item[]  \textbf{Funzione:} \\
				Offre operazioni alle classi che compongono la GUI\g~ per la visualizzazione delle statistiche da parte dell’amministratore e interagiscono con essa.
				
				\item[]  \textbf{Relazioni con altre componenti:} \\
				L'interfaccia è implementata da:
				\begin{itemize}
					\item \path{mytalk.client.view.administrator.StatisticView}.
				\end{itemize}
				L’interfaccia è utilizzata da:
				\begin{itemize}
					\item \path{mytalk.client.view.adminstrator.PageAdminView}.
				\end{itemize}

				\item[]  \textbf{Metodi:}\\
				\texttt{+ void removeViewStatistic();}\\
				Nasconde la GUI\g~ di visualizzazione statistiche.\\

				\texttt{+ void loadViewStatistic();}\\
				Visualizza la GUI\g~ che mostra le statistiche.\\
				
				\texttt{+ void setListData(Vector<Vector<String>> list);}\\
				Inserisce nella tabella contenente le statistiche i dati contenuti nel parametro \texttt{list}.\\
				
				\texttt{+ void errorView(String error);}\\
				Imposta e rende visibile il messaggio di errore.\\
				
				\texttt{+ void loadListBox(Vector<String> users);}\\
				Inserisce gli utenti registrati contenuti nel vettore \texttt{users} all’interno del pannello \texttt{ListBox}.\\
				
				\texttt{+ void setUsenameLabel(String username);}\\
				Imposta l'etichetta \texttt{LabelUser} con il nome dell'amministratore autenticato.\\
				
				\texttt{+ void windowClosing();}\\
				Gestisce l'evento di chiusura della finestra della GUI\g~ di  visualizzazione delle statistiche effettuando il logout.\\
			\end{itemize}
		% IStatisticView - fine
		
		% StatisticView - inizio
		\item[•] \textbf{StatisticView}
			\begin{itemize}
				\item[]  \textbf{Funzione:} \\
				La classe ha il compito di:
				\begin{itemize}
					\item aggiornare opportunamente la GUI\g~ di visualizzazione delle statistiche;
					\item gestire gli eventi che l’utente o il sistema possono innescare inoltrando le richieste all'interfaccia \texttt{IStatisticLogic};
				\end{itemize}
				
				\item[]  \textbf{Relazioni con altre componenti:} \\
				Implementa l'interfaccia:
				\begin{itemize}
					\item \path{mytalk.client.view.administrator.IStatisticView}.
				\end{itemize}
				Usa le classi:
				\begin{itemize}
					\item \path{mytalk.client.presenter.administrator.logicAdmin.StatisticLogic}.
				\end{itemize}
				Tramite le interfacce:
				\begin{itemize}
					\item \path{mytalk.client.presenter.administrator.logicAdmin.IStatisticLogic}.
				\end{itemize}
					
				\item[] \textbf{Attributi:}\\
					\texttt{- IUpdateViewLogic updateViewLogic}: riferimento alla classe UpdateViewLogic;\\
					\texttt{- IStatisticLogic statisticLogic}: riferimento alla classe StatisticLogic;\\
					\texttt{- int tipoFiltro}: identifica il tipo di indice usato:
					\begin{itemize}
						\item 1: giorno;
						\item 2: giudizio;
						\item 3: nome utente;
						\item 4: IP (non attivo al momento);
						\item 5: da lista. 
					\end{itemize}

				
				\item[] \textbf{Oggetti:}\\
					\texttt[] \textbf{@UiField HTMLPanel MyDiv}: pannello contente \texttt{DivStatisticVision} e \texttt{DivPanelControl};\\
					\texttt[] \textbf{@UiField HTMLPanel DivStatisticVision}: pannello contente \texttt{cellTable};\\
					\texttt[] \textbf{@UiField CellTable<Vector<Vector<String>>}: tabella per la visualizzazione dei dati delle statistiche;\\
					\texttt[] \textbf{@UiField HTMLPanel DivPanelControl}: pannello contenente \texttt{LinkLogout}, \texttt{LabelUser} e \texttt{FormFiltro}.\\
					\texttt[] \textbf{@UiField InlineHyperlink LinkLogout}: link per passare alla GUI\g~ dell'autenticazione dell’amministratore effettuando il logout;\\
					\texttt[] \textbf{@UiField InlineLabel LabelUser:}: label che visualizza lo username dell'utente autenticato;\\
					\texttt[] \textbf{@UiField FormPanel FormFiltro}: form contente \texttt{GridFiltro};\\
					\texttt[] \textbf{@UiField Grid GridFiltro}: griglia contenente \texttt{SelectDay}, \texttt{SelectGiudizio}, \texttt{SelectUser}, \texttt{SelectIP}, \texttt{SelectList}, \texttt{Aggiorna}, \texttt{ButtonRicercaInvio}, \texttt{LabelErrorRicerca}, \texttt{ListBoxUtenti};\\
					\texttt[] \textbf{@UiField PushButton SelectDay}: bottone per impostare il valore di \texttt{tipoFiltro} a 1 e rendere visibili \texttt{BoxRicerca} e \texttt{ButtonRicercaInvio};\\
					\texttt[] \textbf{@UiField PushButton SelectGiudizio}: bottone per impostare il valore di \texttt{tipoFiltro} a 2 e rendere visibili \texttt{BoxRicerca} e \texttt{ButtonRicercaInvio};\\
					\texttt[] \textbf{@UiField PushButton SelectUser}: bottone per impostare il valore di \texttt{tipoFiltro} a 3 e rendere visibili \texttt{BoxRicerca} e \texttt{ButtonRicercaInvio};\\
					\texttt[] \textbf{@UiField PushButton SelectIP}: bottone per impostare il valore di \texttt{tipoFiltro} a 4 e rendere visibili \texttt{BoxRicerca} e \texttt{ButtonRicercaInvio};\\
					\texttt[] \textbf{@UiField PushButton SelectList}: bottone per impostare il valore di \texttt{tipoFiltro} a 5 e rendere visibili \texttt{BoxRicerca} e \texttt{ButtonRicercaInvio};\\
					\texttt[] \textbf{@UiField PushButton Aggiorna}: bottone per inviare la richiesta di aggiornamento di \texttt{cellTable} e impostare invisibili \texttt{ListBoxUtenti}, \texttt{BoxRicerca} e \texttt{ButtonRicercaInvio};\\
					\texttt[] \textbf{@UiField PushButton ButtonRicercaInvio}: bottone per richiedere di filtrare i dati di \texttt{cellTable} a seconda del tipo di filtro e della chiave inserita in \texttt{BoxRicerca} o \texttt{ListBoxUtenti};\\
					\texttt[] \textbf{@UiField InlineLabel LabelErrorRicerca}: label che contiene l’errore relativo alla ricerca della chiave usata da \texttt{filtro};\\
					\texttt[] \textbf{@UiField ListBox ListBoxUtenti}: oggetto contenente la lista di tutti gli utenti registrati;\\
					\texttt[] \textbf{@UiField TextBox BoxRicerca}: campo per l’inserimento della chiave da usare come filtro;\\
				
				\item[] \textbf{Metodi:}\\
					\texttt{+ LogAdmin(IUpdateViewLogic updateViewLogic, IWebSocketAdmin webSocket);}\\
					Costruttore: inizializza \texttt{updateViewLogic} e \texttt{communicationLogic} con i valori ricevuti come parametri, imposta inoltre \texttt{webSocket} con il valore del riferimento \texttt{statisticLogic}. Imposta gli attributi degli oggetti che compongono la GUI\g~ e crea le colonne di \texttt{cellTable} attraverso il metodo \texttt{tabella()}.\\

					\texttt{+ void removeViewStatistic();}\\
					Nasconde la GUI\g~ di visualizzazione statistiche.\\
					
					\texttt{+ void loadViewStatistic();}\\
					\begin{itemize}
						\item visualizza la GUI\g~ di comunicazione e imposta come attivi tutti i filtri;
						\item richiama il metodo \texttt{setUsernameLabel()} attraverso il riferimento \texttt{updateViewLogic};
						\item imposta visibile la tabella;
						\item richiama il metodo \texttt{getListData()} attraverso il riferimento \texttt{statisticLogic} per richiedere di popolare la tabella.
					\end{itemize}
					
					\texttt{+ void setListData(Vector<Vector<String>> list);}\\
					Popola la tabella con le statistiche i dati contenuti nel parametro \texttt{list}.\\
					
					\texttt{+ void errorView(String error);}\\
					Rende visibile il messaggio di errore.\\
					
					\texttt{+ void loadListBox(Vector<String> users);}\\
					Inserisce gli utenti registrati contenuti nel vettore \texttt{users} all’interno del pannello \texttt{ListBox}.\\
					
					\texttt{+ void setUsenameLabel(String username);}\\
					Imposta l'etichetta \texttt{LabelUser} con il nome dell'amministratore autenticato.\\
					
					\texttt{+ void windowClosing();}\\
					Gestisce l'evento di chiusura della finestra della GUI\g~ di visualizzazione delle statistiche effettuando il logout.\\
					
					\texttt{- void tabella();}\\
					Metodo per la creazione delle colonne di \texttt{cellTable}.\\
				
				\item[] \textbf{Eventi:}\\
					\texttt{@UiHandler void onLinkLogoutClick(ClickEvent event);}\\
					All'evento \texttt{Click} dell'oggetto \texttt{LinkLogout} viene richiamato il metodo \texttt{removeViewStatistic()}. Inoltre utilizza il riferimento \texttt{communicationLogic} per effettuare il logout e richiamare la GUI\g~ di autenticazione attraverso il metodo \texttt{logoutUser()}.\\
					
					\texttt{@UiHandler void onSelectDayClick(ClickEvent event);}\\
					All'evento \texttt{Click} dell'oggetto \texttt{SelectDay} viene impostato \texttt{tipoFiltro} a 1. Viene disabilitato il bottone \texttt{SelectDay} e vengono resi visibili \texttt{BoxRicerca} e \texttt{ButtonRicercaInvio}.\\
					
					\texttt{@UiHandler void onSelectGiudizioClick(ClickEvent event);}\\
					All'evento \texttt{Click} dell'oggetto \texttt{SelectGiudizio} viene impostato \texttt{tipoFiltro} a 2. Viene disabilitato il bottone \texttt{SelectGiudizio} e vengono resi visibili \texttt{BoxRicerca} e \texttt{ButtonRicercaInvio}.\\
					
					\texttt{@UiHandler void onSelectUserClick(ClickEvent event);}\\
					All'evento \texttt{Click} dell'oggetto \texttt{SelectUser} viene impostato \texttt{tipoFiltro} a 3. Viene disabilitato il bottone \texttt{SelectUser} e vengono resi visibili \texttt{BoxRicerca} e \texttt{ButtonRicercaInvio}.\\
					
					\texttt{@UiHandler void onSelectIPClick(ClickEvent event);}\\
					All'evento \texttt{Click} dell'oggetto \texttt{SelectIP} viene impostato \texttt{tipoFiltro} a 4. Viene disabilitato il bottone \texttt{SelectIP} e vengono resi visibili \texttt{BoxRicerca} e \texttt{ButtonRicercaInvio}.\\
					
					\texttt{@UiHandler void onSelectListClick(ClickEvent event);}\\
					All'evento \texttt{Click} dell'oggetto \texttt{SelectList} viene impostato \texttt{tipoFiltro} a 5. Viene disabilitato il bottone \texttt{SelectList} e vengono resi visibili \texttt{BoxRicerca} e \texttt{ButtonRicercaInvio}.\\
					
					\texttt{@UiHandler void onAggiornaClick(ClickEvent event);}\\
					All'evento \texttt{Click} dell'oggetto \texttt{Aggiorna} vengono abilitati tutti i bottoni \texttt{Select} e vengono nascosti \texttt{ListBoxUtenti}, \texttt{BoxRicerca} e \texttt{ButtonRicercaInvio}. Inoltre viene richiamato il metodo \texttt{getListData()} attraverso il riferimento \texttt{statisticLogic}.\\
					
					\texttt{@UiHandler void onButtonRicercaInvioClick(ClickEvent event)}\\
					All’evento \texttt{Click} dell’oggetto \texttt{ButtonRicercaInvio} viene controllato il tipo di filtro e invocato il metodo \texttt{ricerca(chiave, tipoFiltro)} attraverso il riferimento \texttt{statisticLogic}.\\
			\end{itemize}
		% StatisticView - fine
		
		% IPageAdminView - inizio
		\item[•] \textbf{IPageAdminView}
			\begin{itemize}
				\item[]  \textbf{Funzione:} \\
				Interfaccia che offre operazioni alle classi che compongono la GUI\g~ e interagiscono con essa.
				
				\item[]  \textbf{Relazioni con altre componenti:} \\
				L'interfaccia è implementata da:
				\begin{itemize}
					\item \path{mytalk.client.view.administrator.PageAdminView}.
				\end{itemize}
				L’interfaccia è utilizzata da:
				\begin{itemize}
					\item \path{mytalk.client.MyTalkAdmin};
					\item \path{mytalk.client.presenter.administrator.logicAdmin.UpdateUserView}.
				\end{itemize}
				
				\item[]  \textbf{Metodi:}\\
					\texttt{+ void loadViewLogAdmin();}\\
					Richiede la visualizzazione della GUI\g~ di autenticazione dell’amministratore.\\
					
					\texttt{+ void updateViewLogAdmin(boolean loginOk, String errorMessage);}\\
					Aggiorna la grafica a seconda del valore di \texttt{loginOk}:
					\begin{itemize}
						\item se \texttt{false}: aggiorna la GUI\g~ di autenticazione visualizzando l’errore opportuno contenuto in \texttt{message};
						\item se \texttt{true}: rende invisibile la GUI\g~ di autenticazione e visualizza la GUI\g~ di visualizzazione delle statistiche.\\
					\end{itemize}
					
					\texttt{+ void setListData(Vector<Vector<String>> list);}\\
					Invia i dati delle statistiche alla tabella di visualizzazione.\\
					
					\texttt{+ void errorViewFilter(String error);}\\
					Notifica la presenza di errori nella chiave usata come filtro.\\
					
					\texttt{+ void setUserList(Vector<String> listaUtenti);}\\
					Imposta la lista degli utenti.\\
					
					\texttt{+ void setUsernameLabel(String username);}\\
					Invia il nome utente alla GUI\g~ di visualizzazione delle statistiche.\\
			\end{itemize}
		% IPageAdminView - fine
		
		% PageAdminView - inizio
		\item[•] \textbf{PageAdminView}
			\begin{itemize}
				\item[]  \textbf{Funzione:} \\
				La classe ha il compito di smistare le richieste di aggiornamento della View da parte del Presenter.
				
				\item[]  \textbf{Relazioni con altre componenti:} \\
				Implementa l'interfaccia:
				\begin{itemize}
					\item \path{mytalk.client.view.administrator.IPageAdminView}.
				\end{itemize}
				Usa le classi:
				\begin{itemize}
					\item \path{mytalk.client.view.administrator.LogAdminView};
					\item \path{mytalk.client.view.administrator.StatisticView};
					\item \path{mytalk.client.presenter.administrator.logicAdmin.UpdateViewLogic}.
				\end{itemize}
				Tramite le interfacce:
				\begin{itemize}
					\item \path{mytalk.client.view.administrator.ILogAdminView};
					\item \path{mytalk.client.view.administrator.IStatisticView};
					\item \path{mytalk.client.presenter.administrator.logicAdmin.IUpdateViewLogic}.
				\end{itemize}
					
				\item[] \textbf{Attributi:}\\
					\texttt{- ILogAdmin logAdmin}: riferimento alla classe LogAdmin;\\
					\texttt{- IStatisticView statisticView}: riferimento alla classe StatisticView;\\

				\item[] \textbf{Metodi:}\\
					\texttt{+ PageAdminView();}\\
					Costruttore:
					\begin{itemize}
						\item inizializza, carica nel pannello di root e imposta la visibilità degli oggetti \texttt{logAdmin} e \texttt{statisticView};
						\item imposta un listener che attende l’evento di chiusura della pagina e che in tal caso richiama i metodi di chiusura degli oggetti \texttt{statisticView}.\\
					\end{itemize}
					
					\texttt{+ void loadViewLogAdmin();}\\
					Richiama il metodo \texttt{loadViewLogAdmin()} attraverso il riferimento \texttt{logAdmin} per visualizzare l’autenticazione dell’amministratore.\\
					
					\texttt{+ void updateViewLogAdmin(boolean loginOk, String errorMessage);}\\
					Aggiorna la GUI\g~ a seconda del valore update:
					\begin{itemize}
						\item se \texttt{false}: richiama il metodo \texttt{errorView(message)} attraverso il riferimento \texttt{logAdmin} per visualizzare gli errori di autenticazione;
						\item se \texttt{true}: richiama il metodo \texttt{removeViewLogAdmin()} attraverso il riferimento \texttt{logAdmin} e successivamente richiama il metodo \texttt{loadViewStatistic()} attraverso il riferimento \texttt{statisticView}.\\
					\end{itemize}
					
					\texttt{+ void setListData(Vector<Vector<String>> list);}\\
					Invia i dati delle statistiche alla tabella di visualizzazione con il metodo \texttt{setListData(list)} attraverso il riferimento \texttt{statisticView}.\\
					
					\texttt{+ void errorViewFilter(String error);}\\
					Richiama il metodo \texttt{errorView(messages)} attraverso il riferimento \texttt{statisticView} per visualizzare gli errori di filtro di ricerca.\\
					
					\texttt{+ void setUserList(Vector<String> listaUtenti);}\\
					Richiama il metodo \texttt{loadListBox(listaUtenti)} attraverso il riferimento \texttt{statisticView} per impostare la lista utenti del filtro di ricerca per lista.\\
					
					\texttt{+ void setUsernameLabel(String username);}\\
					Richiama il metodo \texttt{setUsenameLabel(username)} attraverso il riferimento \texttt{statisticView}.\\
			\end{itemize}
		% PageAdminView - fine

	\end{itemize}
		
	}

	\subsection{Package mytalk.client.presenter.administrator.logicAdmin} {
		\begin{itemize}
		
		% IUpdateViewLogic - inizio
		\item[•] \textbf{IUpdateViewLogic}
			\begin{itemize}
				\item[]  \textbf{Funzione:} \\
				Interfaccia che offre operazioni alle classi che devono interagire con la View.
				
				\item[]  \textbf{Relazioni con altre componenti:} \\
				L'interfaccia è implementata da:
				\begin{itemize}
					\item \path{mytalk.client.view.administrator.UpdateViewLogic}.
				\end{itemize}
				L’interfaccia è utilizzata da:
				\begin{itemize}
					\item \path{mytalk.client.view.administrator.PageAdminView}.
				\end{itemize}
				
				\item[]  \textbf{Metodi:}\\
					\texttt{+ void checkLoggedAdmin(}\\
					Controlla l’esistenza di cookies\g~ di sessione.\\
					
					\texttt{+ void updateViewLogAdmin(boolean loginOk, String errorMessage);}\\
					Invia alla View il risultato dell’operazione di autenticazione.\\
					
					\texttt{+ void loadViewLogAdmin();}\\
					Richiede alla grafica il caricamento della GUI\g~ di autenticazione dell’amministratore.\\
					
					\texttt{+ void setListData(Vector<Vector<String>> list);}\\
					Invia i dati delle statistiche alla View.\\
					
					\texttt{+ void loginResult(boolean loginSuccess, Vector<String> loginInput);}\\
					A seconda del valore di \texttt{loginSuccess} crea i cookie\g~ o ritorna un’errore di autenticazione:
					\begin{itemize}
						\item se è \texttt{true}: crea il cookie\g~ e richiede l’aggiornamento della View;
						\item se è \texttt{false}: imposta l’errore da ritornare e richiede l’aggiornamento della View.\\
					\end{itemize}
					
					\texttt{+ void errorViewFilter(String string);}\\
					Richiede la visualizzazione dell’errore contenuto in \texttt{String}.\\
					
					\texttt{+ void setUserList(Vector<String> listaUtenti);}\\
					Invia alla View la lista degli utenti registrati.\\
					
					\texttt{+ void removeCookies();}\\
					Rimuove i cookie\g~.\\
					
					\texttt{+ void setUsernameLabel();}\\
					Invia alla View il nome utente salvato nel cookie\g~.\\
			\end{itemize}
		% IUpdateViewLogic - fine
		
		% UpdateViewLogic - inizio
		\item[•] \textbf{UpdateViewLogic}
			\begin{itemize}
				\item[]  \textbf{Funzione:} \\
				La classe ha il compito di inviare i risultati delle operazioni del Presenter alla View.
				
				\item[]  \textbf{Relazioni con altre componenti:} \\
				Implementa l'interfaccia:
				\begin{itemize}
					\item \path{mytalk.client.presenter.administrator.logicAdmin.IUpdateViewLogic}.
				\end{itemize}
				Usa le classi:
				\begin{itemize}
					\item \path{mytalk.client.view.administrator.PageAdminView}.
				\end{itemize}
				Tramite le interfacce:
				\begin{itemize}
					\item \path{mytalk.client.view.administrator.IPageAdminView}.
				\end{itemize}
					
				\item[] \textbf{Attributi:}\\
					\texttt{- IPageAdminView pageAdminView}: riferimento alla classe PageAdminView;\\

				\item[] \textbf{Metodi:}\\
					\texttt{+ UpdateViewLogic(IPageAdminView pageAdminView);}\\
					Costruttore: inizializza l'oggetto pageAdminView;\\
					
					\texttt{+ void checkLoggedAdmin();}\\
					Recupera il nome dell’utente contenuto nel cookie\g~ attraverso la chiamata \texttt{ManageCookies.getCookieUsername()}. Nel caso in cui il cookie\g~ sia vuoto il metodo chiama \texttt{loadViewLogAdmin()}, altrimenti invoca \texttt{updateViewLogAdmin(true, "``)}.\\
					
					\texttt{+ void updateViewLogAdmin(boolean loginOk, String errorMessage);}\\
					Richiama \texttt{updateViewLogAdmin(loginOk, errorMessage)} attraverso il riferimento \texttt{pageAdminView}.\\
					
					\texttt{+ void loadViewLogAdmin();}\\
					Richiama \texttt{loadViewLogAdmin()} attraverso il riferimento \texttt{pageAdminView}.\\
					
					\texttt{+ void setListData(Vector<Vector<String>> list);}\\
					Nel caso in cui \texttt{list} sia vuota richiama il metodo  \texttt{errorViewFilter(“Non ci sono riscontri.")} e, in ogni caso, chiama il metodo  \texttt{setListData(list)} attraverso il riferimento  \texttt{pageAdminView}.\\
					
					\texttt{+ void loginResult(boolean loginSuccess, Vector<String> loginInput);}\\
					A seconda del valore di \texttt{loginSuccess} crea i cookie\g~ o imposta un errore opportuno:
					\begin{itemize}
						\item se è \texttt{true}: crea il cookie\g~ e richiama \texttt{updateViewLogAdmin(true,"")};
						\item se è \texttt{false}: imposta l’errore e chiama \texttt{updateViewLogAdmin(false, errorMessage)}.\\
					\end{itemize}

					\texttt{+ void errorViewFilter(String string);}\\
					Richiama \texttt{errorViewFilter(error)} attraverso il riferimento \texttt{pageAdminView}.\\
					
					\texttt{+ void setUserList(Vector<String> listaUtenti);}\\
					Richiama \texttt{setUserList(listaUtenti)} attraverso il riferimento \texttt{pageAdminView}.\\
					
					\texttt{+ void removeCookies();}\\
					Richiede la rimozione dei cookie\g~ attraverso \texttt{ManageCookies.deleteCookies()}.\\
					
					\texttt{+ void setUsernameLabel();}\\
					Richiama \texttt{setUsernameLabel(ManageCookies.getCookieUsername())} attraverso il riferimento \texttt{pageAdminView}, dove \texttt{ManageCookies.getCookieUsername()} recupera il nome dell’utente.\\
					
			\end{itemize}
		% UpdateViewLogic - fine
		
		\end{itemize}
	}
\end{sloppypar}
}
