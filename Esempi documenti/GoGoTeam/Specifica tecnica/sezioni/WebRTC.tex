\appendix
\appendixpage
\addappheadtotoc
\section{WebRTC}{
	\subsection{Descrizione generale} {
		Le Web Real-Time Communications, d’ora in poi WebRTC$_{|g|}$, aggiungono nuove potenzialità al browser$_{|g|}$, 
		permettendogli di interagire direttamente con altri browser$_{|g|}$ stabilendo una comunicazione real-time. 
		Si può pensare di avere funzionalità simili a quelle di Skype$^{TM}$, ma senza installare software\g~ o plugin$_{|g|}$. 
		Queste funzionalità sono rese disponibili agli sviluppatori tramite tag standard HTML5$_{|g|}$ 
		(in realtà non ancora approvato come standard anche se supportato da molti browser\g , lo diventerà da fine 2014) 
		ed API$_{|g|}$ JavaScript$_{|g|}$.\\
		Lo scenario più comune è quello in cui entrambi i browser\g~ stanno eseguendo 
		la stessa applicazione WebRTC$_{|g|}$, scaricata dalla medesima pagina web\g . 
		Questo produce il cosiddetto “Triangolo WebRTC\g ”: il \underline{web server}$_{|g|}$ si occupa della connessione con i due browser$_{|g|}$, 
		detti anche \emph{peer}$_{|g|}$ (lati del triangolo), tra i quali è instaurata una connessione (base del triangolo), 
		detta \emph{peer connection}, attraverso la quale passa il flusso di dati.

		\begin{center}
			\begin{figure}[h]
				\centering
				\includegraphics[scale=0.6]{\docsImg Figura1.pdf}
			\caption{Il Triangolo WebRTC\g .}	
			\end{figure}
		\end{center}	


		Le tre azioni principali richieste per instaurare una sessione WebRTC\g~ sono:

		\begin{enumerate}
			\item ottenere i dati multimediali locali;
			\item stabilire una connessione tra il browser\g~ ed un altro peer\g~ (solitamente un altro browser\g );
			\item aggiungere i canali multimediale e di dati alla connessione.
		\end{enumerate}

		Tali azioni sono illustrate nella figura \ref{fig:figura2}:

		\begin{center}
			\begin{figure}[h]
				\centering
				\label{fig:figura2}
				\includegraphics[scale=0.6]{\docsImg Figura2.pdf}
			\caption{Instaurazione della sessione WebRTC\g .}	
			\end{figure}
		\end{center}

		Vengono ora descritti più in dettaglio i singoli passi, ai quali ne viene aggiunto un quarto che riguarda la chiusura della sessione.


		\subsubsection{Ottenere dati multimediali locali}{
			Il metodo più comune per ottenere dati locali è quello di utilizzare il metodo \texttt{getUserMedia()}, 
			che permette di ottenere un \texttt{LocalMediaStream}. I flussi multimediali possono poi essere messi insieme tramite 
			l’API$_{|g|}$ \texttt{MediaStream}. Per ragioni di privacy, è richiesto il consenso dell’utente per accedere al 
			suo microfono e alla sua webcam.
		}

		\subsubsection{Stabilire la Peer Connection}{
			Una Peer Connection è una connessione multimediale diretta tramite due web$_{|g|}$ browser$_{|g|}$ (peer$_{|g|}$), 
			per la cui instaurazione viene utilizzata l’API$_{|g|}$ \texttt{RTCPeerConnection}. Tale connessione permette 
			il flusso audio/video tra i due peer$_{|g|}$. È necessaria una connessione per ogni coppia di peer$_{|g|}$. 
			L’unico input che il costruttore di \texttt{RTCPeerConnection} richiede è un oggetto di configurazione che ICE$_{|g|}$ 
			(Interactive Connectivity Establishment) userà per “fare buchi” (punch holes) attraverso i NAT$_{|g|}$ 
			e i firewall$_{|g|}$ che separano i peer$_{|g|}$.
		}

		\subsubsection{Scambiare dati multimediali}{
			Ogni scambio di dati richiede una negoziazione (o rinegoziazione) tra i browser$_{|g|}$ su come i dati verranno 
			rappresentati nel canale. A tale scopo quando viene effettuata una richiesta, in locale o in remoto, 
			di aggiungere o rimuovere dati, il browser$_{|g|}$ genera un appropriato oggetto \texttt{SessionDescription} che rappresenta 
			il flusso di dati attraverso la peer connection. Una volta che i browser$_{|g|}$ si sono scambiati questo oggetto, 
			la sessione di scambio di dati può essere stabilita.
		}
		\subsubsection{Chiudere la connessione}{
			Uno qualsiasi dei browser$_{|g|}$ può chiudere la connessione. L’applicazione chiama il metodo \texttt{close()} 
			sulla \texttt{RTCPeerConnection} per indicare che ha finito di usare la connessione, solitamente in seguito 
			ad una richiesta dell’utente che ha chiuso il browser$_{|g|}$ o la scheda (tab) corrente.
		}
		
		\subsubsection{Esempio di connessione}{
			\begin{center}
				\begin{figure}[h]
					\centering
					\label{fig:figura3}
					\includegraphics[scale=0.6]{\docsImg Figura3.pdf}
				\caption{Instaurazione della sessione WebRTC\g .}	
				\end{figure}
			\end{center}
			Nella figura \ref{fig:figura3} è illustrato un esempio di instaurazione della sessione WebRTC$_{|g|}$.
			Essa è composta dai seguenti passi:
			\begin{enumerate}
				\item il browser$_{|g|}$ M richiede la pagina web$_{|g|}$ dal \underline{web server}$_{|g|}$;
				\item il \underline{web server}$_{|g|}$ fornisce la pagina web$_{|g|}$ a M con il codice 
				      JavaScript$_{|g|}$ relativo alle WebRTC$_{|g|}$;
				\item il browser$_{|g|}$ L richiede la pagina web$_{|g|}$ dal \underline{web server}$_{|g|}$;
				\item il \underline{web server}$_{|g|}$ fornisce la pagina web$_{|g|}$ a L con il codice 
				      JavaScript$_{|g|}$ relativo alle WebRTC$_{|g|}$;
				\item M decide di comunicare con L, lo script JavaScript$_{|g|}$ eseguito su M permette la spedizione 
				      al \underline{web server}$_{|g|}$ dell’oggetto \texttt{SessionDescription} di M;
				\item il \underline{web server}$_{|g|}$ invia l’oggetto \texttt{SessionDescription} di M al codice 
				      JavaScript$_{|g|}$ su L;
				\item lo script JavaScript$_{|g|}$ eseguito su L permette la spedizione dell’oggetto 
				      \texttt{SessionDescription} (risposta) al \underline{web server}$_{|g|}$;
				\item il \underline{web server}$_{|g|}$ invia l’oggetto \texttt{SessionDescription} di L ad M;
				\item M ed L iniziano il processo di \emph{hole punching} per determinare il modo migliore per comunicare;
				\item al completamento dell’hole punching, M ed L negoziano una chiave per la sicurezza della trasmissione;
				\item M ed L iniziano a scambiarsi dati, audio e video.
			\end{enumerate}
		}
	}
	\subsection{L’hole punching}{
		La natura del NAT$_{|g|}$ rende difficoltosa la creazione di sessioni peer-to-peer$_{|g|}$, ma usando una tecnica 
		chiamata “hole punching” essa può essere creata con successo in molti casi. L’hole punching è una tecnica 
		usata nelle reti di computer per stabilire comunicazioni tra due parti, situate in sedi differenti, 
		che sono entrambe dietro un NAT$_{|g|}$ e/o protette da firewall$_{|g|}$. Questa tecnica richiede quattro prerequisiti:
		\begin{enumerate}
			\item i due browser$_{|g|}$ che stanno provando a stabilire una connessione diretta devono, nello stesso tempo, 
			      inviare  i pacchetti "hole puching". Devono pertanto conoscere l’indirizzo al quale invieranno i pacchetti.
			      Il requisito 1 è soddisfatto utilizzando il \underline{web server}$_{|g|}$ per coordinare l'hole puching. 
			      Il \underline{web server}$_{|g|}$ sa che una dovrà essere stabilita una sessione tra i browser$_{|g|}$, 
			      e assicura che essi inizino l'hole punching approssimativamente nello stesso tempo.
			\item i due browser$_{|g|}$ hanno bisogno di conoscere più indirizzi IP$_{|g|}$ possibili che potranno essere 
			      usati per raggiungerli. Questi indirizzi sono spesso descritti come "indirizzi privati" se stanno 
			      all’interno di un NAT$_{|g|}$ o "indirizzi pubblici" se stanno al di fuori da un NAT$_{|g|}$. 
			      Il requisito 2 è soddisfatto utilizzando uno STUN$_{|g|}$ server$_{|g|}$. Ogni browser$_{|g|}$ chiede 
			      lo STUN$_{|g|}$ server\g~ inviando un pacchetto STUN$_{|g|}$. Lo STUN$_{|g|}$ server\g~ risponde indicando 
			      l'indirizzo IP$_{|g|}$ osservato nel pacchetto di test. Cioè, risponde con l'indirizzo mappato 
			      dal NAT$_{|g|}$. Questo indirizzo IP$_{|g|}$ ricavato dallo STUN$_{|g|}$ server$_{|g|}$ è condiviso 
			      con altri browser$_{|g|}$ e diventa un potenziale candidato di indirizzo. 
			      L’indirizzo può essere IPv4, IPv6 o una combinazione di entrambi.
			\item come ultima possibilità, deve esistere un server$_{|g|}$ con un indirizzo IP$_{|g|}$ pubblico e non 
			      nascosto da un NAT$_{|g|}$ e quindi raggiungibile da entrambi i browser$_{|g|}$. Il requisito 3 è 
			      soddisfatto utilizzando un TURN$_{|g|}$ server\g , che fornisce ai browser$_{|g|}$ un indirizzo di riserva. 
			      Questo indirizzo viene poi aggiunto alla lista dei candidati.
			\item devono essere usati flussi assimetrici. Ossia, UDP$_{|g|}$ deve operare in maniera simile a TCP$_{|g|}$. 
			      Il Requisito 4 è soddisfatto dal browser$_{|g|}$ inviando la trasmissione dalla stessa porta UDP$_{|g|}$ 
			      che il browser$_{|g|}$ sta usando per ascoltare la trasmissione in arrivo. 
			      Questo fa sì che le due sessioni monodirezionali RTP$_{|g|}$ inviate tramite UDP$_{|g|}$ appaiano al 
			      NAT$_{|g|}$ come una sessione RTP$_{|g|}$ bidirezionale.
		\end{enumerate}
	}
}