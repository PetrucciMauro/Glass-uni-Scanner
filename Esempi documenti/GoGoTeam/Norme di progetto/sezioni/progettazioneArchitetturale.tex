\section{Progettazione architetturale} {
\subsection{Attività}{
Lo scopo di questa attività è quello di realizzare una visione globale di ciò che dovrà essere il sistema a fronte dei requisiti ricavati dall'attività di analisi.\\
Terminata l'attività di progettazione architetturale si deve produrre un documento completo ed esplicativo: la Specifica Tecnica.\\
Le attività necessarie alla redazione del documento  sono:
\begin{itemize}
	\item definizione dell'architettura di prodotto a partire dall'Analisi dei Requisiti (vedi \textit{\AnalisiDeiRequisiti});
	\item individuazione e studio dei design pattern applicabili;
	\item individuazione della struttura dei package;
	\item individuazione delle classi che compongono il sistema e delle relazioni tra esse;
	\item analisi delle tecnologie da adottare;
	\item studio di fattibilità;
	\item tracciamento componenti-requisiti (vedi \ref{paTraccia}).
\end{itemize}
Si deve inoltre definire, in un documento specifico (\textit{\PianoDiQualifica}), vari test da eseguire sulle parti del sistema per verificarne la corretta interazione.
\begin{itemize}
	\item []{\textbf{Input:} Analisi dei Requisiti;}
	\item []{\textbf{Output:} Specifica Tecnica, test di integrazione;}
	\item []{\textbf{Risorse:} Progettisti, documentazione, strumentazione;}
	\item []{\textbf{Misurazioni:} avanzamento dell'elaborazione del documento Specifica Tecnica rispetto alla totalità dei requisiti definiti nel documento di Analisi dei Requisiti;}
	\item []{\textbf{Norme:} descritte in seguito.}
\end{itemize}

}

\subsection{Diagrammi UML}{
\subsubsection{Tipologia}{
Si deve utilizzare il linguaggio UML\g~ v2.x per definire:
\begin{itemize}
	\item [] \textbf{Diagrammi dei package\g~:} {per comprendere la struttura generale del sistema. \`E fondamentale che ciascun package\g~ sia definito in modo chiaro ed univoco in modo da facilitare, in fase di codifica, la comprensione delle interazioni tra i vari package\g~ .}
	\item [] \textbf{Diagrammi delle classi:} {per comprendere la complessità del sistema in relazione al numero delle classi.}
	\item [] \textbf{Diagrammi delle attività:} {per definire in modo non ambiguo i flussi delle varie attività, composte dalle rispettive componenti, ritenuti non banali.}
	\item [] \textbf{Diagrammi di sequenza:} {per definire in modo preciso le responsabilità dei vari package\g~ o classi nel portare a termine un determinato compito. Si deve limitare la rappresentazione di cicli o condizioni alternative ai soli casi in cui la loro omissione comporti un grave scompenso alla comprensione del diagramma.}
\end{itemize}
}%tipologia
\subsubsection{Classificazione}\label{classPCAS}{
Ciascuna tipologia di diagramma UML\g~ dev'essere identificata mediante la seguente nomenclatura:
\begin{center}
			\textit{\{Tipo\} \{A\}...\{Z\}}
		\end{center}dove \textit{Tipo} indica una tipologia dei seguenti diagrammi:
		\begin{itemize}
			\item [] \textbf{DP:} diagramma dei package\g ;
			\item [] \textbf{DC:} diagramma delle classi;
			\item [] \textbf{DA:} diagramma di attività;
			\item [] \textbf{DS:} digramma di sequenza.
		\end{itemize}e \textit{\{A\}...\{Z\}} indica una struttura gerarchica di insiemi. (\textit{es.} \texttt{A.B.C}: \texttt{A} indica il diagramma principale, \texttt{B} una specifica/estensione di \texttt{A}, \texttt{C} una specifica/estensione di \texttt{B} e cosi via). \\
}%class

}%diagrammi

\subsection{Design pattern}\label{NormeDesPat} {
Per descrivere i vari design pattern adottati si dovrà utilizzare la seguente specifica:
\begin{itemize}
			\item {\textbf{Descrizione generale:} breve descrizione della struttura del design pattern;}
			\item {\textbf{Motivazione:} descrivere il motivo della scelta di tale design pattern;}
			\item {\textbf{Contesto applicativo:} elencare i contesti dove il design pattern è stato applicato.}
		\end{itemize}
}%design pattern

\subsection{Immagini GUI}\label{IGUI} {
Per rendere più nitida l'idea di come l'utente finale deve interagire con il sistema, è necessario fornire un'anteprima dell'interfaccia grafica del prodotto per ogni attività di utilizzo.\\
Ogni immagine, rappresentante una GUI\g~ del prodotto, deve essere etichettata secondo la seguente convenzione
	\begin{center}
		\textit{GUI \{A\}...\{Z\}}
	\end{center}
	
	dove \textit{\{A\}...\{Z\}} indica una struttura gerarchica di insiemi. (\textit{es.} \texttt{A.B.C}: \texttt{A} indica la figura principale, \texttt{B} una specifica/estensione di \texttt{A}, \texttt{C} una specifica/estensione di \texttt{B} e cosi via). 
}%IGUI

\subsection{Norme di progettazione}{
Per rendere più semplice la comprensione degli schemi e la futura fase di verifica, è necessario prestare attenzione a:
\begin{itemize}
	\item {\textbf{Annidamento di chiamate:}} {non ci devono essere, all'interno dell'applicazione, chiamate annidate con una profondità massima maggiore di cinque.
	}
	\item {\textbf{Concorrenza:}} {nel caso in cui si renda necessario l'utilizzo di flussi concorrenti è necessario fornire il diagramma di flusso e la stima delle risorse necessarie all'implementazione. Nel caso in cui i benefici ottenuti dalla concorrenza non risultino essere equivalenti o inferiori alle risorse, questa dev'essere eliminata.	
	}
	\item {\textbf{Flussi condizionali:}} {nei casi in cui vengano utilizzati costruttori condizionali (\textit{es.} \textbf{if - else}) il loro annidamento non dovrà essere maggiore di cinque. Questo garantisce una maggiore chiarezza, migliore comprensione futura del codice ed una attività di verifica più semplice.
	}
	\item {\textbf{Numero di parametri:}} {non dovranno essere creati metodi aventi più di cinque parametri.
	}
	\item {\textbf{Ricorsione:}} {si deve evitare il più possibile l'utilizzo della ricorsione. Nel caso in cui l'utilizzo della ricorsione sia necessario, si deve fornire una opportuna dimostrazione di terminazione ed una stima dell'occupazione che questa provoca in memoria; qualora queste stime risultino essere eccessive, si deve procedere con l'eliminazione della ricorsione.
	}
\end{itemize}
Nel documento Piano di Qualifica (\textit{\PianoDiQualifica}) devono essere trattate ed approfondite le metriche  e la loro misurazione.
}%Norme di progettazione

\subsection{Tracciamento} {\label{paTraccia}
Nel documento deve essere riportato, in forma tabellare, il tracciamento di ogni requisito con la componente che ne assicura il soddisfacimento.\\
\`E richiesto di riportare, anche questo in forma tabellare, il tracciamento di ciascuna componente con il requisito che ne giustifica l'esistenza.
}%tracciamento

\subsection{Organizzazione dei contenuti}{
Il documento Specifica Tecnica deve contenere al suo interno:
\begin{itemize}
	\item analisi dei vari design pattern utilizzati nel modo descritto in \ref{NormeDesPat};
	\item definizione della struttura generale del sistema mediante diagrammi di package\g~ corredati da una sintetica spiegazione;
	\item {definizione generale delle sottoparti architetturali:
		\begin{enumerate}
			\item Diagramma delle classi;
			\item {Descrizione del diagramma delle classi composta dai seguenti punti:
				\begin{itemize}
					\item \textbf{Nome:} il nome della classe;
					\item \textbf{Tipo:} attributo che indica se si tratta di un'interfaccia o di una classe;
					\item \textbf{Package:} indica il package\g~ a cui appartiene l'interfaccia o la classe;
					\item \textbf{Descrizione:} una descrizione sintetica dell'interfaccia o della classe e delle sue funzionalità;
					\item \textbf{Relazioni con altre componenti:} indica le relazioni che la classe ha con altre interfacce o classi presenti all'interno della rappresentazione grafica della componente architetturale.
				\end{itemize}							
			}
			\item Diagramma di attività, corredato da una sintetica descrizione, per le sottoparti ritenute non banali che portino informazioni aggiuntive;
			\item Diagramma di sequenza;
			\item {Descrizione del diagramma di sequenza composta dai seguenti punti:
				\begin{itemize}
					\item Precondizione;
					\item Descrizione sintetica;
					\item Postcondizione.
				\end{itemize}							
			}
		\end{enumerate}
	}
\end{itemize}
}%Organizzazione dei contenuti

\subsection{Strumenti} {
Per l'esecuzione di questa attività si utilizzano i seguenti strumenti:
\begin{itemize}
	\item []{\textbf{\LaTeX :} per la stesura del documento Specifica Tecnica (dettaglio cap. \ref{TextLaTeX});}
	\item []{\textbf{Visual Paradigm:} per la stesura dei diagrammi UML\g~ (dettaglio cap. \ref{diaUML});}
	\item []{\textbf{git:} per il versionamento della documentazione (dettaglio cap. \ref{git});}
\end{itemize}

}%Strumenti

}%progettazione Archietturale