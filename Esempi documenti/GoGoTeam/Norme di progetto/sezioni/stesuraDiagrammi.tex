\section{Nomenclatura} {
	\subsection{Diagrammi UML} {
		Ogni diagramma UML\g~ inserito in un documento deve essere etichettato secondo la seguente convenzione
		\begin{center}
			\textit{\{Tipo\} \{A\}...\{Z\}}
		\end{center}dove \textit{Tipo} indica una tipologia dei seguenti diagrammi:
		\begin{itemize}
			\item [] \textbf{UC:} diagramma dei casi d’uso;
			\item [] \textbf{UCA:} diagramma dei casi d’uso per amministratore;
			\item [] \textbf{DC:} diagramma delle classi;
			\item [] \textbf{DO:} diagramma degli oggetti;
			\item [] \textbf{DS:} digramma di sequenza;
			\item [] \textbf{DA:} diagramma di attività;
			\item [] \textbf{DP:} diagramma dei package$_{|g|}$.
		\end{itemize}e \textit{\{A\}...\{Z\}} indica una struttura gerarchica di insiemi. (\textit{es.} \texttt{A.B.C}: \texttt{A} indica il diagramma principale, \texttt{B} una specifica/estensione di \texttt{A}, \texttt{C} una specifica/estensione di \texttt{B} e cosi via). \\
		
		Si consiglia la creazione di una tabella di tracciamento dei diagrammi nei documenti che li contengono.

		Per la stesura dei grafici UML\g \ viene utilizzato il programma Visual Paradigm for UML $\geqslant$ 10.0 (\url{http://www.visual-paradigm.com/}). Il programma viene utilizzato in licenza \emph{Community Edition} la quale ne permette l'uso per fini non commerciali.\\
		Vengono elencati i motivi che hanno portato la scelta del programma:
		\begin{itemize}
			\item pieno supporto allo standard UML\g \ v2.x (\url{http://www.visual-paradigm.com/support/documents/vpumluserguide.jsp});
			\item facilità di utilizzo e ottima qualità dei diagrammi prodotti (con possibilità di apportare modifiche successive);
			\item esportazione dei grafici nei formati PDF\g, PNG\g \ e JPG\g;
			\item programma cross-platform\g \ disponibile per tutti gli OS\g \ adottati.
		\end{itemize} 
	}
	
	\subsection{Descrizione delle singole classi}{
La descrizione testuale delle interfacce e delle classi dovrà contenere, in questo ordine:
\begin{itemize}
\item \textbf{Nome:} il nome della classe;
\item \textbf{Tipo:} attributo che indica se si tratta di un'interfaccia o di una classe;
\item \textbf{Package:} indica il package a cui appartiene l'interfaccia o la classe;
\item \textbf{Descrizione:} una descrizione sintetica dell'interfaccia o della classe e delle sue funzionalità;
\item \textbf{Relazioni con altre componenti:} indica le relazioni che la classe ha con altre interfacce o classi presenti all'interno della rappresentazione grafica della componente architetturale.
\end{itemize}

}
	\subsection{Diagrammi Design Pattern} {
		Per descrivere i vari design pattern adottati si dovrà utilizzare la seguente specifica:
		\begin{itemize}
			\item[•] {\textbf{Descrizione generale:} breve descrizione della struttura del design pattern;}
			\item[•] {\textbf{Motivazione:} descrivere il motivo della scelta di tale design pattern;}
			\item[•] {\textbf{Contesto applicativo:} elencare i contesti dove il design pattern è stato applicato.}
		\end{itemize}
	}

	\subsection{Norme requisiti di progetto} {
		Ogni requisito deve rispettare la seguente forma 
		\begin{center}
			\textit{\{Tipologia\}\{Importanza\}\{X\}} 
		\end{center}
		
		\textbf{Tipologia utente:}
		\begin{itemize}
			\item []\textbf {\{F\}:} indica un requisito funzionale;
			\item []\textbf {\{Q\}:} indica un requisito di qualità;
			\item []\textbf {\{T\}:} indica un requisito tecnologico;
			\item []\textbf {\{V\}:} indica un requisito di vincolo.
		\end{itemize}
		
		\textbf{Tipologia amministratore:}
		\begin{itemize}
			\item []\textbf {\{FA\}:} indica un requisito funzionale.
		\end{itemize}
		
		\textbf{Importanza:}	
		\begin{itemize}
			\item []\textbf{\{OB\}:} indica un requisito obbligatorio; 
			\item []\textbf{\{DE\}:} indica un requisito desiderabile;
			\item []\textbf{\{OP\}:} indica un requisito opzionale.
		\end{itemize}
		
		La X indica il numero di requisito.\\
		Se un requisito è un sotto-requisito di un altro, dev'essere denominato nel seguente modo
		\begin{center}
			\textit{\{Tipologia\}\{Importanza\}\{X\}.\{Y\}}
		\end{center}
		dove X è il numero del requisito principale, mentre Y quello del sotto - requisito. \`E possibile indicare anche la fonte dei requisiti utilizzando i seguenti acronimi o abbreviazioni:
		\begin{itemize}
			\item []\textbf{CA:} capitolato;
			\item []\textbf{IP:}	incontro proponente;
			\item []\textbf{IF:} interno fornitore.
		\end{itemize}

		Si consiglia nell'Analisi dei Requisiti la creazione di una tabella di tracciamento dei requisiti. Si lascia agli Analisti la definizione della struttura della tabella.
	}
	
	\subsection{Immagini GUI}{Ogni immagine rappresentante una GUI$_{|g|}$ del prodotto deve essere etichettata secondo la seguente convenzione
	\begin{center}
		\textit{GUI \{A\}...\{Z\}}
	\end{center}
	
	dove \textit{\{A\}...\{Z\}} indica una struttura gerarchica di insiemi. (\textit{es.} \texttt{A.B.C}: \texttt{A} indica la figura principale, \texttt{B} una specifica/estensione di \texttt{A}, \texttt{C} una specifica/estensione di \texttt{B} e cosi via). 
}
	
	\subsection{Riferimento a componenti} {Ogni componente deve essere etichettata secondo la seguente convenzione
	\begin{center}
		\textit{\{Tipo\} \{Numero\}}
	\end{center}
 dove \textit{Tipo} indica una tipologia dei seguenti componenti:
 \begin{itemize}
 	\item[•] {\textbf{M:} model}
 	\item[•] {\textbf{V:} view}
 	\item[•] {\textbf{C:} controller}
 \end{itemize}
e \textit{Numero} è un intero positivo.\\

(\textit{es.} \texttt{M1} indica una componente di model; \texttt{V1} indica una componente di view; \texttt{C1} indica una componente di controller). \\

}
