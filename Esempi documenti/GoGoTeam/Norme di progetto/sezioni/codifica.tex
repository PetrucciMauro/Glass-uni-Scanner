\section{Codifica} {
\subsection{Attività}{
Lo scopo di questa attività è quello di produrre il codice che soddisfi, il più fedelmente possibile, le specifiche definite nel documento Definizione di Prodotto.\\
L'attività termina con la presentazione del codice sorgente del sistema.
\begin{itemize}
	\item []{\textbf{Input:} Definizione di Prodotto;}
	\item []{\textbf{Output:} codice sorgente;}
	\item []{\textbf{Risorse:} Programmatori, strumentazione;}
	\item []{\textbf{Misurazioni:} avanzamento dell'elaborazione del codice sorgente rispetto alla totalità della Definizione di Prodotto;}
	\item []{\textbf{Norme:} descritte in seguito.}
\end{itemize}
}

	\subsection{Intestazione file}{
		Ogni file di codice sorgente deve iniziare con un'intestazione che deve rispettare il seguente schema:
		
		{ \footnotesize
		\begin{lstlisting}[basicstyle=\ttfamily]
	/**
	 * Name: {Nome del file}
	 * Package: {Package di appartenenza}
	 * Author: {Autore del file}
	 * Date: {Data di approvazione del file}
	 *
	 * Changes:
	 * Version		Date				Changes				Reason 
	 * {X}.{Y}		AAAA-MM-GG	{Cambiamento}	{Motivazione}
	 *
	 *
	 * ------------------------------------------------------------
	 * Copyright (C) 2013 GoGoTeam
	 * 
	 * This file is part of MyTalk.
	 * 
	 * MyTalk is free software: you can redistribute it and/or modify
	 * it under the terms of the GNU General Public License
	 * as published by the Free Software Foundation, version 2 of the
	 * License.
	 * 
	 * MyTalk is distributed in the hope that it will be useful,
	 * but WITHOUT ANY WARRANTY; without even the implied warranty of
	 * MERCHANTABILITY or FITNESS FOR A PARTICULAR PURPOSE.  See the
	 * GNU General Public License for more details.
	 * 
	 * You should have received a copy of the GNU General Public 
	 * License along with MyTalk.  If not, see 
	 * <http://www.gnu.org/licenses/>.
	 * ------------------------------------------------------------
	 *
	 */
		\end{lstlisting}
}
		\begin{itemize}
			\item \textbf{Name}: indica il nome del file comprensivo di estensione;
			\item \textbf{Package}$_{|g|}$: deve essere comprensivo della gerarchia del package$_{|g|}$;
			\item \textbf{Author}: indica l'autore del file e non necessariamente il programmatore che sta modificando il file attuale;
			\item \textbf{Date}: indica la data di creazione del file (come indicato nel cap. \ref{formatiRiferimento});
			\item \textbf{Changes}: indica la lista di avanzamento delle  modifiche del file apportate. In dettaglio:
			\begin{itemize}
				\item \textbf{Version}: ultima modifica apportata al file (come indicato nel cap. \ref{versCod}).
				\item \textbf{Date}: rappresenta la data dell'avvenuta modifica (come indicato nel cap. \ref{formatiRiferimento}).
				\item \textbf{Changes}: rappresenta la lista dei cambiamenti effettuati. Questa lista deve fornire le informazioni inerenti i metodi modificati preceduti da una delle seguenti etichette:
				\begin{itemize}
					\item []\texttt{[+]} indica che il metodo è stato creato; 
					\item []\texttt{[-]} indica che il metodo è stato eliminato;
					\item []\texttt{[x]} indica che il metodo ha subito un cambiamento.
				\end{itemize}
				\item \textbf{Reason}: motivazione di modifica.
			\end{itemize}
			L'inserimento dell'ultima versione è aggiunto a capo della lista  in modo da avere uno storico a ritroso delle modifiche fatte al file.
		\end{itemize} 
	}

	\subsection{Convenzione codifica codice} {
		I programmatori devono seguire la Java Code Convention (\url{http://www.oracle.com/technetwork/java/codeconvtoc-136057.html}) in modo tale da rendere il codice più leggibile e mantenibile.
		Tale convenzione può essere integrata o modificata solo su autorizzazione del Responsabile di Progetto.
		
		\subsubsection{Struttura interna delle classi}{
			All’interno di una classe dovranno comparire nel seguente ordine:
			\begin{itemize}
				\item Variabili statiche
				\item Variabili di istanza
				\item Costruttori
				\item Metodi
			\end{itemize}
		}
		
		\subsubsection{Struttura dei codici}{
		
			\begin{itemize}
			
		        \item[•]	 { \textbf{Struttura linea di codice:} La lunghezza della
		        linea di codice deve facilitare la lettura e fornire una buona
		        presentazione visiva. Per questo scopo  si è scelto l’utilizzo di
		        spazi prima e dopo la maggior parte degli operatori, cercando di non
		        alterare la funzione del codice, ed evitare per ciascuna riga
		        l’inserimento di più istruzioni.
		        }
			
			    \item[•] { \textbf{Rientri e uso delle parentesi graffe} La
			    parentesi graffa di apertura di un blocco di codice dovrà essere
			    posta sulla stessa riga della dichiarazione del blocco di codice
			    interessato, mentre la parentesi graffa di chiusura va posizionata
			    in una nuova riga, allineata verticalmente rispetto al primo
			    carattere della dichiarazione del blocco di codice interessato.\\
				Ogni livello di indentazione deve rientrare utilizzando il tasto di
				tabulazione. Di seguito si riporta un esempio d’uso:
					\begin{lstlisting}[basicstyle=\ttfamily]
				public void stampa(String) {
				    .........
				}
				\end{lstlisting}
				}
				
				\item[•] { \textbf{Blocchi funzionali:} Il codice viene diviso in
				paragrafi utilizzando le righe vuote per separarli. Questa
				convenzione è stata pensata per favorire la leggibilità e
				semplificare la comprensione della logica del codice stesso.
				}				
				
				\item[•] { \textbf{Commenti:} I commenti vanno sempre scritti in una
				nuova riga lasciando una riga di spazio prima di essi, anche se
				inseriti all’interno di un blocco di codice. Il carattere che indica
				l’inizio del commento deve essere // (doppia barra) oppure nella
				forma /* */ se il commento necessita di più righe .
				}
			\end{itemize}
			
		}
		
		
		\subsubsection{Convenzione sui nomi}{
			\begin{itemize}
				\item[•] {\textbf{Classi:} l'identificatore deve avere la prima
				lettera maiuscola e tutte le altre minuscole. Se è composto da più
				parole, la prima lettera di ogni parola deve essere in maiuscolo e
				non ci devono essere dei caratteri speciali.
				}
				\item[•] {\textbf{Interfacce:} l'identificatore sarà composto dalla 
				lettera \textit{i} maiuscola seguita dal nome della classe che
				implementerà tale interfaccia. Esempio: classe \textit{Pippo},
				interfaccia \textit{IPippo}.
				}
				\item[•] {\textbf{Metodi:} l'identificatore dovrebbe dare
				un'immediata idea del suo comportamento contenendo almeno un verbo
				scritto in minuscolo. Se è composto, la seconda parola andrà scritta
				solo con la prima lettera in maiuscolo. Esempio \textit{pagareTassa}.
				}
				\item[•] {\textbf{Variabili:} scritte con nomi brevi ed evocativi con la prima lettera in minuscola. Nel caso siano composte da più parole, l'iniziale di ogni parola dovrebbe essere maiuscola.}
				\item[•] {\textbf{Costanti:} dovrebbero essere scritte in maiuscolo.
				Nel caso siano composte da più parole si dovrebbe utilizzare il
				simbolo di underscore.}
			\end{itemize}
			}	
		}
		
\subsection{Strumenti} {
Per l'esecuzione di questa attività si utilizzano i seguenti strumenti:
\begin{itemize}
	\item []{\textbf{Eclipse:} per la realizzazione del codice (dettaglio cap. \ref{IDEEclipse});}
	\item []{\textbf{GWT:} per la codifica dei linguaggi web (dettaglio cap. \ref{FWgwt});}
	\item []{\textbf{git:} per il versionamento della documentazione (dettaglio cap. \ref{git});}
\end{itemize}

}%Strumenti

}