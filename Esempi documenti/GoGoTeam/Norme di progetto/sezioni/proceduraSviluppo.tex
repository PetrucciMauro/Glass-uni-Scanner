\section{Procedura per lo sviluppo delle applicazioni}{

\subsection{Gestione incarichi}{
	Il servizio SourceForge\g \ usato permette l'utilizzo di servizi di tracking\g \ e milestone\g .\\
	Questi strumenti dovono essere utilizzati nella fase di sviluppo seguendo una rigida normativa.

	\begin{enumerate}
		\item \textbf{Creazione milestone\g:} \label{milestone} {
			il Responsabile ha il compito di creare una milestone\g \ (figura \ref{CreaMS}) per le successive revisioni a cui il gruppo deciderà di partecipare. Sarà possibile tenere sotto controllo il suo stato di avanzamento analizzando il numero di ticket\g~ completati ed i rimanenti ancora aperti.
		}
		
		\begin{center}
			\begin{figure}[h]
				\centering
				\label{CreaMS}
				\includegraphics[scale=0.7]{\docsImg 1.pdf}
				\caption{Modello per la creazione di una milestone\g.}	
			\end{figure}
		\end{center}

		\item \textbf{Creazione ticket\g~:} \label{creazioneTicket} {
			il Responsabile deve creare una serie di ticket\g \  corrispondenti ai compiti da svolgere. Il compito deve essere assegnato al componente del gruppo ritenuto più idoneo alla sua soddisfazione (figura \ref{CreaTk}).\\
			La compilazione del ticket\g~ deve seguire le seguenti direttive:
			\begin{itemize}
				\item \textbf{Title:} specifica, sinteticamente, l'oggetto del compito;
				\item \textbf{Status:} da impostare come \texttt{open};
				\item \textbf{Owner:} imposta il riferimento al membro del gruppo che avrà incarico di soddisfare il compito indicato;
				\item \textbf{Labels:} indica il campo del compito in oggetto. L'etichetta deve essere composta da un identificativo quale:
					\begin{itemize}
						\item [] \textbf{\texttt{D}:} per i documenti;
						\item [] \textbf{\texttt{C}:} per il codice applicativo;
					\end{itemize}
				\item \textbf{Private:} identifica il compito come privato, la sua abilitazione viene lasciata a giudizio del Responsabile;
				\item \textbf{Summary:} descrizione del compito da svolgere;
			\end{itemize}
		}

		\begin{center}
			\begin{figure}[h]
				\centering
				\label{CreaTk}
				\includegraphics[scale=0.7]{\docsImg 2.pdf}
				\caption{Modello per la creazione di un nuovo ticket\g.}	
			\end{figure}
		\end{center}
		
		\item \textbf{Esecuzione compito:} \label{esecuzioneTicket} {
			ciascun membro del gruppo deve visionare i ticket\g~ a lui assegnati e modificare il loro stato ad \texttt{accepted}.\\
			Una volta terminato il compito richiesto, l'incaricato deve segnare il ticket\g~ corrispondente come \texttt{closed} (figura \ref{AssTk}).	
		}

		\begin{center}
			\begin{figure}[h]
				\centering
				\label{AssTk}
				\includegraphics[scale=0.7]{\docsImg 3.pdf}
				\caption{Modello per l'assegnazione di un ticket\g.}	
			\end{figure}
		\end{center}

		\item \textbf{Creazione ticket\g~ di verifica:} \label{creazioneTicketVer} {
			il Responsabile, ad ogni ticket\g~ segnalato \texttt{closed}, ha il compito di creare un nuovo ticket\g~ associato di verifica (figura \ref{VerTk}).
			\begin{itemize}
				\item \textbf{Title:} deve essere specificato l'oggetto del compito anteposto dall'identificativo \texttt{VERIFICA}:
				\begin{center}
					\texttt{VERIFICA: \{Oggetto del compito\}}
				\end{center}
				\item \textbf{Status:} da impostare come \texttt{open};
				\item \textbf{Owner:} imposta il riferimento all'utente Verificatore;
				\item \textbf{Labels:} si deve aggiungere al label del titcket originario il suffisso \texttt{v};
				\item \textbf{Private:} identifica il compito come privato, la sua abilitazione viene lasciata a giudizio del Responsabile;
				\item \textbf{Summary:} descrizione, lasciata a giudizio del Responsabile;
			\end{itemize}			
		}
		
		\newpage

		\begin{center}
			\begin{figure}[h!]
				\centering
				\label{VerTk}
				\includegraphics[scale=0.57]{\docsImg 4.pdf}
				\caption{Modello per la creazione e gestione delle verifiche.}	
			\end{figure}
		\end{center}	

		\item \textbf{Fase di verifica:} \label{faseVerificaTicket} {
			il Verificatore può rifiutare o accettare il ticket\g~ assegnato (figura \ref{VerTk}).
			Nel primo caso, il Verificatore deve modificare lo stato del ticket\g~ a \texttt{wont-fix} specificandone il motivo di rifiuto nel campo \texttt{Summary}.\\
			Il Verificatore, una volta accettato il compito, deve impostare lo stato del ticket\g~ ad \texttt{accepted} e procedere con la fase di verifica.
			Se la verifica ha esito positivo, lo stato del ticket\g~ passa a \texttt{closed} mentre, nel caso di riscontro di errori gravi, il Verificatore deve creare un nuovo ticket\g :
			\begin{itemize}
				\item \textbf{Title:} specifica, sinteticamente, l'oggetto del compito;
				\item \textbf{Status:} da impostare come \texttt{open};
				\item \textbf{Owner:} imposta il riferimento al Responsabile di Progetto;
				\item \textbf{Label:} label del titcket\g~ originario;
				\item \textbf{Private:} identifica il compito come privato, la sua abilitazione viene lasciata a giudizio del Verificatore;
				\item \textbf{Summary:} descrizione dell'errore;
			\end{itemize}
			Il Responsabile deve valutare se accettare o rifiutare il ticket\g~ di verifica. Nel caso di accettazione, il Responsabile deve modificare il campo \texttt{Owner} indicando il riferimento all'utente del gruppo designato per la soddisfazione del compito.\\
			Una volta che il ticket\g~ viene segnalato come \texttt{closed} il Responsabile deve seguire una nuova procedura di verifica (punto \ref{faseVerificaTicket}).
		}

		\item \textbf{Chiusura milestone\g :} \label{closeMS} {
			Una volta che tutti i ticket\g~ appartenenti alla stessa milestone\g~ sono segnati come \texttt{closed}, il Responsabile può rendere la milestone\g~ \texttt{closed} e crearne una nuova ripartendo dal punto \ref{milestone}.
		}

	\end{enumerate}

}%attivita'

	\subsection{Norme di versionamento}{
		\subsubsection{Documentazione}\label{versDocs}{
		Al fine di uniformare lo stile di versionamento degli elementi dei documenti interni si è scelta la seguente norma di versionamento:
		\begin{itemize}
			\item \textbf{Notazione:} per indicare la versione attuale del documento si usa la seguente notazione:
			\begin{center}
				\texttt{\{X\}.\{Y\}}
			\end{center}
			dove:
			\begin{itemize}
				\item \textbf{\texttt{\{X\}}:} indica un'evoluzione significativa del documento. Si verifica l’aumento di questo numero quando il documento è in uno stato formale;
				\item \textbf{\texttt{\{Y\}}:} indica l'apporto di modifiche al documento che possono venire apportate dal Redattore stesso al fine di integrarne o ampliarne il contenuto, su sua iniziativa o su richiesta del Verificatore oppure del \textit{Proponente}/\textit{Committente};
			\end{itemize}			
			\item \textbf{Utilizzo:} la notazione di versionamento è impiegata nei seguenti casi:
			\begin{itemize}
				\item \textbf{Nomenclatura documenti:} ogni documento in formato PDF\g~ deve seguire la seguente convenzione:
				\begin{center}
					\texttt{NomeFile\_v\{X\}.\{Y\}.pdf}
				\end{center}
				
Ogni modifica ed ogni nuova versione devono venire apportate nel registro modifiche del documento inserendo i campi richiesti: Versione, Autore, Data e Descrizione.\\
Quando i redattori eseguono delle correzioni al documento a seguito di una revisione da parte dei verificatori, devono riportare nel registro delle modifiche la modifica  fatta ed il suo relativo titolo (\textit{es.} Revisione).
							
				\item \textbf{Riferimenti interni:} ogniqualvolta  in un documento sia presente un riferimento ad un altro documento è necessario inserirne il  nome e la sua versione: 
				\begin{center}
					\texttt{Documento di Riferimento - v\{X\}.\{Y\}}
				\end{center}
				\item \textbf{Stato del documento:}
					\begin{itemize}
						\item {\textbf{Bozza:} indica un documento in prima stesura;}
						\item {\textbf{Preliminare:} indica un documento in redazione, al quale si applicano ciclicamente revisioni e correzioni;}
						\item {\textbf{Formale:} indica un documento ufficiale e approvato dal responsabile.}
			Inoltre, deve essere specificato il tipo di utilizzo;
						\item {\textbf{Interno:} indica un documento che deve rimanere all'interno di \textit{GoGo Team};} 
						\item {\textbf{Esterno:} indica un documento che può venire distribuito a terzi.}
					\end{itemize}
				\item \textbf{Intestazione:} deve essere specificata la versione attuale del documento;
				\item \textbf{Pagina:} deve essere specificata la versione attuale del documento nella nota a piè di pagina seguendo il formato indicato in cap. \ref{formatiRiferimento}.
			\end{itemize}
		\end{itemize}		
	}
	}%documentazione


\subsubsection{Codifica}\label{versCod}{
	Al fine di uniformare lo stile di versionamento degli elementi di codifica si è scelta la seguente norma di versionamento:
	\begin{center}
		\texttt{\{X\}.\{Y\}}
	\end{center}
dove:
\begin{itemize}
				\item \textbf{\texttt{\{X\}}:} indica un'evoluzione significativa del file di codifica. Si verifica l'aumento di questo numero quando il file è in una versione stabile e completa;
				\item \textbf{\texttt{\{Y\}}:} indica l'apporto di modifiche al file che possono venire apportate dal Programmatore stesso al fine di integrarne o ampliarne le funzionalità richieste.
\end{itemize}
}%cod


}
