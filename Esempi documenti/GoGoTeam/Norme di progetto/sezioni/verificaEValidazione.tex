\section{Verifica e validazione}
\subsection{Attività}{
L'obiettivo principale di questa attività è quello di dimostrare che il prodotto risultante dall'attività di codifica funzioni correttamente (verifica) e che sia il prodotto atteso (validazione).
Al termine dell'attività ci si aspetta di avere dei dati che confermano la correttezza del prodotto.
\begin{itemize}
\item [] \textbf{Input:} codice sorgente, stime dei test;
\item [] \textbf{Output:} esiti dei test;
\item [] \textbf{Risorse:} verificatori, strumentazione;
\item [] \textbf{Misurazioni:} stato di avanzamento rispetto alle stime dei test;
\item [] \textbf{Norme:} descritte in seguito.
\end{itemize}
L'esito delle attività di verifica e validazione dovrà produrre i risultati che stimano:
\begin{itemize}
\item \textbf{Copertura dei requisiti:} non inferiore ai requisiti definiti come obbligatori;
\item \textbf{Copertura del codice:} non inferiore al 60\%;
\item \textbf{Codice morto:} non presente.
\end{itemize}
Tali dati dovranno essere creati grazie all'ausilio di Metrics (cap. \ref{plugMETRICS}) e EclEmma (cap. \ref{plugECLE}).
}

\subsection{Test di unità e integrazione}
La verifica dovrà essere realizzata basandosi sulle definizioni dei test di unità e integrazione forniti dalle attività di progettazione architetturale e di progettazione di dettaglio. Per le varie strategie di verifica si dovrà fare riferimento al Piano di Qualifica (\textit{\PianoDiQualifica}).
\newline
Nel documento riassuntivo dell'esito dei test si dovrà presentare per ogni tipologia di test effettuato:
\begin{itemize}
\item specifica dell'ambiente di testing;
\item classi testate o componenti testate dal particolare test;
\item descrizione del comportamento del test;
\item errori individuati a seguito dell'esecuzione dei test.
\end{itemize}
Si dovrà riportare l'esito dei test in formato tabellare.
\newline
La codifica dei test dovrà essere effettuata su \textit{Eclipse} con l'ausilio del framework \textit{JUnit}(cap. \ref{junit}).



\subsection{Test di sistema}
La validazione dovrà essere realizzata basandosi sulle definizioni dei test di sistema forniti dall'attività di analisi dei requisiti. Per le varie strategie di validazione si dovrà fare riferimento al Piano di Qualifica (\textit{\PianoDiQualifica}).\\
Si dovranno riportare in formato tabellare gli esiti dei test effettuati in corrispondenza ai requisiti testati.