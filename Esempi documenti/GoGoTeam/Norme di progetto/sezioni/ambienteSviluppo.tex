\section{Ambiente di sviluppo}{
	In questa sezione si descrivono gli strumenti scelti dal gruppo \textit{\ggt} per un ottimale sviluppo di progetto.

	\subsection{Ambiente generale} {
		\subsubsection{Sistemi Operativi}{
			L'intero sviluppo del progetto viene svolto in ambienti Unix-Like e Windows, nello specifico, Ubuntu $\geqslant$  11.10, Mac OS X $\geqslant$  10.7 e Windows $\geqslant$ 7.\\
			Tale scelta è maturata dopo che i componenti del gruppo hanno analizzato l'ambito di sviluppo del progetto, che è composto da una parte client\g , sviluppata tramite applicativo web\g~ , e una parte server\g , sviluppata in ambiente JVM\g . Tali tecnologie sono indipendenti dall'ambiente di sviluppo e di impiego.
		}
		\subsubsection{Browser}{
			Sotto esplicita richiesta da parte del \textit{Proponente}, lo sviluppo dell'applicativo, nella sua parte \underline{web client}\g , dev'essere svolto in riferimento all'ambiente Google Chrome\g~ $\geqslant$ 26.0 (\url{https://www.google.com/intl/it/chrome/browser/}).
		}

		\subsubsection{Server}{
			Sotto consiglio da parte del \textit{Proponente}, lo sviluppo dell'applicativo, nella sua parte server\g , dev'essere svolto in ambiente JVM$_{|g|}$. Si è scelto di utilizzare il server$_{|g|}$ \underline{Apache Tomcat}\g~ $\geqslant$ 7.0.27 (\url{http://tomcat.apache.org/}) e linguaggio JVM\g \ Java\g~ $\geqslant$ 1.6.0 (\url{http://www.oracle.com/technetwork/java/index.html}).
		}
	}

	\subsection{Ambiente di documentazione} {
		\subsubsection{Editor testi}\label{TextLaTeX}{
			La scrittura dei documenti deve essere svolta tramite linguaggio di markup\g~ \LaTeX \  (\url{http://www.latex-project.org/}) ed utilizzando, per tutti gli ambienti OS\g~ , l'editor Texmaker $\geqslant$ 1.9.9 (\url{http://www.xm1math.net/texmaker/}) oppure l'editor Kile $\geqslant$ 2.1.3 (\url{http://www.kile.sourceforge.net}).\\
			La compilazione deve produrre un file in formato PDF\g \ che sarà prodotto dal comando, integrato anche nell'editor, 
			\begin{center}
				\texttt{pdflatex NomeFile.tex}
			\end{center}
		}
		\subsubsection{Verifica ortografica}{
			La verifica ortografica per i documenti scritti in linguaggio \LaTeX \ deve essere effettuata tramite lo strumento Aspell $\geqslant$ 0.60.6 (\url{http://aspell.net/}).\\
			Il programma Aspell deve essere usato via terminale eseguendo il comando
			\begin{center}
				\texttt{aspell --mode=tex --lang=it check NomeFile.tex}
			\end{center}
		}
		
		\subsubsection{Diagrammi UML}\label{diaUML}{		
		Per la stesura dei grafici UML\g \ viene utilizzato il programma Visual Paradigm for UML $\geqslant$ 10.0 (\url{http://www.visual-paradigm.com/}). Il programma viene utilizzato in licenza \emph{Community Edition} la quale ne permette l'uso per fini non commerciali.\\
		Vengono elencati i motivi che hanno portato la scelta del programma:
		\begin{itemize}
			\item pieno supporto allo standard UML\g \ v2.x (\url{http://www.visual-paradigm.com/support/documents/vpumluserguide.jsp});
			\item facilità di utilizzo e ottima qualità dei diagrammi prodotti (con possibilità di apportare modifiche successive);
			\item esportazione dei grafici nei formati PDF\g, PNG\g \ e JPG\g;
			\item programma cross-platform\g \ disponibile per tutti gli OS\g \ adottati.
		\end{itemize} 		
		}
		
		\subsubsection{Gestione della pianificazione}{
			Per la gestione della pianificazione delle attività e dei ruoli durante le diverse fasi di progetto si è deciso di usare Microsoft Project 2010 (\url{http://www.microsoft.com/project/en-us/project-management.aspx}).
			La scelta è stata vincolata dal miglior grado di qualità che il programma esprime nella gestione dei grafici di pianificazione rispetto alla concorrenza.\\
			Il programma viene utilizzato in licenza accademica per mezzo della MSDN-AA (\url{https://msdnaa.studenti.math.unipd.it/}) la quale è disponibile per tutti gli studenti iscritti ai corsi di laurea in matematica ed informatica (facenti parte del Dipartimento di Matematica dell'Università degli studi di Padova), per un uso privato non commerciale.
		}
	}

	\subsection{Ambiente di sviluppo delle applicazioni} {
		\subsubsection{Versionamento}{
		\label{git}
		Lo strumento di versionamento scelto è git (\url{http://git-scm.com/}).
		git si presenta come uno strumento diffuso nell'ambito del sviluppo software\g , fatto che comporta la presenza di una grande quantità di documentazione a supporto del prodotto, la quale è liberamente accessibile dai componenti del gruppo. 
		Il programma, inoltre, si integra con estrema semplicità negli strumenti di sviluppo scelti dal gruppo.\\
		Il servizio di repository\g~ git viene fornito di default da  SourceForge (\url{https://sourceforge.net/}).
		}
		
		\subsubsection{IDE di base}\label{IDEEclipse}{
			Per la scrittura del codice sarà utilizzato l'IDE\g~ Eclipse $\geqslant$ 3.5 (\url{http://www.eclipse.org/}).\\
			Questo strumento è stato scelto per la sua elevata possibilità di personalizzazione ed estensione con componenti necessari ad uno sviluppo più agevole delle applicazioni.
		}
		\subsubsection{Framework di codifica linguaggi web}\label{FWgwt}{
			Per la codifica delle parti web-side viene impiegato il framework\g~ GWT (Google Web Toolkit) $\geqslant$ 2.5.0 (\url{https://developers.google.com/web-toolkit/}).\\
			GWT è stato scelto per i seguenti motivi:
			\begin{itemize}
				\item sviluppo in linguaggio Java;
				\item riutilizzo del codice;
				\item creazione di pagine web dinamiche mediante tecnologia AJAX\g ;
				\item indipendenza dall'ambiente di sviluppo.
			\end{itemize}
			Il framework è disponibile come componente aggiuntivo integrabile nell'ambiente Eclipse.
		}
	}
}