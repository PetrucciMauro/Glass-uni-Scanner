\section{Documenti} {
	\label{commitDoc}
	\subsection{Struttura documentazione}{ 
		Tutti i documenti devono essere realizzati utilizzando il template \LaTeX \ presente nel repository\g
		all'indirizzo
		\begin{center}
			\url{https://sourceforge.net/p/ggt-mytalk/code/docs/template/}
		\end{center}
		L'utilizzo del template deve limitarsi all'invocazione delle funzionalità che consentono la formattazione dei documenti: ciò permette di effettuare modifiche stilistiche localizzate al codice del template per farle ricadere in maniera automatica in tutti i documenti, evitando quindi la modifica a mano di ogni documento.\\
		Nello specifico, il template regola:
		\begin{itemize}
			\item pacchetti \LaTeX\ usati;
			\item comandi personalizzati;
			\item riferimenti testuali interni;
			\item formattazione copertina documento;
			\item formattazione pagine interne;
			\item stile tabelle;
			\item casi particolari di sillabazione di alcune parole;
			\item stili di tipografia.
		\end{itemize}
		Le modifiche al template possono essere avanzate, tramite i metodi di comunicazione interna (cap. \ref{commint}), da tutti i membri del gruppo al Responsabile di Progetto il quale avrà compito di valutare e, nel caso affermativo, applicare le modifiche richieste.
	}

	\subsection{Norme tipografiche}{
	\label{NorTip}
		Al fine di evitare incongruenze tra i vari documenti, si specificano in questa sezione le informazioni riguardanti 
		l'ortografia, la tipografia e l'assunzione di uno stile uniforme in tutti i documenti.
		\subsubsection{Stile del testo} {
			\begin{itemize}
				\item \textbf{Grassetto:} viene utilizzato per evidenziare dei passaggi estremamente importanti all'interno del testo e per evidenziare l'elemento trattato negli elenchi.
				\item \textbf{Corsivo:} deve essere utilizzato per porre particolare enfasi a termini significativi e per riportare citazioni da fonti esterne.
				\item \textbf{Sottolineato:} deve essere utilizzato per specificare termini composti da più parole, seguiti anche dall'identificativo \g, la cui definizione compare nel Glossario.
				\item \textbf{Maiuscolo:} deve essere limitato all'indicazione di acronimi e nei casi specificati nei Formati di Riferimento (cap. \ref{formatiRiferimento});
				\item \textbf{Monospace:} deve essere utilizzato per indicare una normativa adottata, riferimenti (cap. \ref{formatiRiferimento}) e per inserire porzioni di codice all'interno dei documenti.
			\end{itemize}
		}
		\subsubsection{Punteggiatura}{
			\begin{itemize}
				\item \textbf{Punteggiatura:} qualsiasi segno di punteggiatura non deve seguire un carattere di spaziatura, ma deve essere seguito da un carattere di spazio.
				\item \textbf{Lettera maiuscola:} deve seguire esclusivamente un punto, un punto esclamativo o un punto interrogativo.
				\item \textbf{Parentesi:} una qualsiasi frase racchiusa fra parentesi non deve iniziare con un carattere di spaziatura e non deve chiudersi con un carattere di punteggiatura e/o di spaziatura.
			\end{itemize}
		}
		\subsubsection{Composizione testo}{
			\begin{itemize}
				\item \textbf{Elenchi:}
				\begin{itemize}
					\item se la lista contiene termini identificativi, questi devono iniziare con la maiuscola e non terminare con un carattere di punteggiatura;
					\item se i termini sono frasi considerate parte integrante di quella che introduce la lista allora devono iniziare con la minuscola e terminare con il punto e virgola, tranne l'ultimo termine che deve finire con il punto;
					\item se i termini sono complessi e costituiti da frasi distinte rispetto al periodo introduttivo, si usa la maiuscola e il punto alla fine di ogni termine.
				\end{itemize}
				\item \textbf{Note a piè di pagina:} ne è vietato l'uso per non interrompere il flusso di lettura del documento.
				\item \textbf{Glossario:} le parole accompagnate da \g~ sono esclusivamente quelle che presentano una corrispondenza nel Glossario.
			\end{itemize}	
		}
	}

	\subsection{Formati di riferimento}{
		\label{formatiRiferimento}
		\begin{itemize}
			\item \textbf{Riferimenti:}
			\begin{itemize}
				\item \textbf{Percorsi:} per gli indirizzi web\g~ completi e indirizzi e-mail deve essere utilizzato il comando appositamente fornito da \LaTeX
				\begin{center}
					\texttt{$\backslash$url\{Percorso\}}
				\end{center}
				nel formato tipografico monospace.
				\item \textbf{Link a file PDF\g~ :} devono essere contrassegnati in corsivo mediante l'uso del comando personalizzato:
				\begin{center}
					\texttt{\$$\backslash$nomePdf\$}
				\end{center}
				\item \textbf{Ancore:} i riferimenti alle sezioni interne del medesimo documento devono essere scritte nel formato
				\begin{center}
					(cap. \texttt{$\backslash$ref\{label di riferimento della sezione\}})
				\end{center}
			\end{itemize}
			\item \textbf{Data:} se non specificato diversamente, deve essere espressa seguendo il formalismo dello standard \texttt{ISO\g~ 8601:2004} nel formato:
			\begin{center}
				\texttt{AAAA-MM-GG}
			\end{center}
			dove:
			\begin{itemize}
				\item [] \textbf{AAAA:} rappresenta l'anno in quattro cifre;
				\item [] \textbf{MM:} rappresenta il mese in due cifre;
				\item [] \textbf{GG:} rappresenta il giorno in due cifre.
			\end{itemize}
			\item \textbf{Abbreviazioni:} sono ammesse le sigle per i nomi dei documenti
			\begin{itemize}
				\item [] \textbf{AR:} Analisi dei Requisiti
				\item [] \textbf{GL:} Glossario
				\item [] \textbf{NP:} Norme di Progetto
				\item [] \textbf{PQ:} Piano di Qualifica
				\item [] \textbf{PP:} Piano di Progetto
				\item [] \textbf{SF:} Studio di Fattibilità
			\end{itemize}
			 e per i nomi delle revisioni
			\begin{itemize}
				\item [] \textbf{RR:} Revisione dei Requisiti
				\item [] \textbf{RP:} Revisione di Progettazione
				\item [] \textbf{RQ:} Revisione di Qualifica
				\item [] \textbf{RA:} Revisione di Accettazione
			\end{itemize}
			\item \textbf{Nomi ricorrenti:}
			\begin{itemize}
				\item \textbf{Ruoli di progetto:} devono essere formattati  utilizzando la prima lettera maiuscola di ogni parola che non sia una preposizione (\textit{es.} Responsabile di Progetto).
				\item \textbf{Nomi dei documenti:} devono essere formattati utilizzando la prima lettera maiuscola di ogni parola che non sia una preposizione (\textit{es.} Norme di Progetto).
				\item \textbf{Nomi dei file:} il riferimento deve essere comprensivo dell'estensione del file ed essere formattato con carattere monospace.
				\item \textbf{Nomi propri:} l'utilizzo dei nomi propri deve seguire il formalismo: 
				\begin{center}
					\begin{itemize}
						\item[] \textit{Cognome Nome} per elenchi;
						\item[] \textit{Nome Cognome} per tutti gli altri riferimenti.
					\end{itemize}	
				\end{center}
				\item \textbf{Nome del gruppo:} deve essere identificato nel seguente stile: 
				\begin{center}
					\textit{\ggt}
				\end{center}
				\item \textbf{Nome del proponente:} deve essere identificato nel seguente stile:
				\begin{center}
					\textit{Proponente} o \textit{\Zucchetti}
				\end{center}
				\item \textbf{Nome del committente:} deve essere identificato nel seguente stile:
				\begin{center}
				\textit{Committente} o \textit{\Vardanega} e \textit{\Cardin}
				\end{center}
				\item \textbf{Nome del progetto:} deve essere identificato nel seguente stile:
				\begin{center}
					\textbf{\mytalk}
				\end{center}
			\end{itemize}
			\item \textbf{Informazioni di pagina:} devono essere espresse nel formato
			\begin{center}
				\texttt{\{n\} di \{totale pagine\} }
			\end{center}
		\end{itemize}
	}

	\subsection{Immagini}{
		  Tutte le immagini devono essere in formato JPG\g~ , PNG\g~ o PDF\g .
		  Ogni figura inserita deve avere una breve didascalia composta da un identificativo numerico univoco seguito da, ove sia ritenuto necessario, una breve descrizione.
	}
	
	\subsection{Tabelle}{
		Ogni tabella deve essere stilata in formato \LaTeX . Ciascuna tabella deve avere una didascalia composta da un identificativo numerico univoco seguito da, ove sia ritenuto necessario, una breve descrizione.
	}
	
	\subsection{Struttura tipi documenti} {
		\subsubsection{Verbali incontri} {
			\label{VerbInc}
			Per \emph{Verbali degli incontri} si intendono quei documenti redatti dal Responsabile di Progetto in occasione di 
			incontri esterni ed interni. Per tali documenti è prevista  una sola stesura in quanto promemoria 
			dell'incontro avvenuto.
			Non è perciò previsto il versionamento. I Verbali degli incontri devono essere denominati secondo il 
			seguente criterio:
			\begin{center}
				\texttt{Verbale\{tipo incontro\}\_\{data incontro\}}
			\end{center}
			dove:
			\begin{itemize}
				\item \textbf{tipo incontro:} indica il tipo di incontro effettuato e deve essere identificato con I 
				(interno) o E (esterno);
				\item \textbf{data incontro:} indica la data in cui è stato tenuto l'incontro seguendo il formato data indicato in cap. \ref{formatiRiferimento}.
			\end{itemize}
			La prima pagina di ogni verbale deve obbligatoriamente contenere i seguenti campi, nell'ordine indicato:
			\begin{itemize}
				\item \textbf{data};
				\item \textbf{luogo:} espresso nel formato
				\begin{center}
					\{città\}(\{sigla\_provincia\}) \{sede\}
				\end{center}
				\item \textbf{ora di ritrovo:} espressa nel formato
				\begin{center}
					Ora dell'incontro: \{hh\}:\{mm\}
				\end{center}
				in cui il campo \texttt{hh} indicante le ore e il campo \texttt{mm} indicante i minuti vanno espressi nel formato 24h specificato nella normativa \texttt{ISO\g~ 8601:2004};
				\item \textbf{durata dell'incontro:} espressa nel formato
				\begin{center}
					Durata dell'incontro: \{x\} min
				\end{center}
				dove il valore \texttt{x} sta ad indicare l'effettiva durata dell'incontro;
				\item \textbf{partecipanti interni:} lista degli appartenenti al gruppo \textit{\ggt} presenti all'incontro
				\begin{center}
					Partecipanti del gruppo: \{nome\} \{cognome\}
				\end{center}
				\item \textbf{partecipanti esterni:} rappresentanti della ditta/e
				\begin{center}
					Partecipanti esterni: \{nome\} \{cognome\} \{ruolo\}
				\end{center}
				dove il campo \texttt{ruolo} rappresenta il ruolo assunto all'interno dell'azienda a cui fanno capo; nel caso il partecipante sia il \emph{Committente}, il campo viene compilato con \texttt{Committente}.
				
				\item \textbf{contenuto:} la decisione del formato è lasciata al Responsabile di Progetto, il quale adotta lo stile più consono in base al tipo di incontro svolto;
				\item \textbf{firme:} devono essere comprese quelle di tutti i partecipanti del gruppo \textit{\ggt} a conferma della presa visione del documento. 
			\end{itemize}
		 }
		 \subsubsection{Lettera di presentazione}{
			La lettera di presentazione deve contenere:
			\begin{itemize}
				\item Logo del gruppo
				\item Intestazione nel seguente formato:
				\begin{center}
					\Vardanega \\
					Università degli Studi di Padova\\
					Via Trieste 63\\
					35121 Padova (PD)
				\end{center}
				\item Breve introduzione (facoltativa)
				\item Elenco di tutti i documenti in consegna
				\item Varie ed eventuali, osservazioni (facoltative)
				\item Firma del responsabile nel seguente formato
				\begin{center}
					\textit{\{Nome\} \{Cognome\}}\\
					\textit{il Responsabile del gruppo \ggt}\\
					\textit{\{Firma del responsabile\}}
				\end{center}
			\end{itemize}
		 }
	}

	\subsection{Contenuto dei documenti}{
		Ogni documento ufficiale deve essere composto dalle seguenti sezioni:
		\begin{itemize}
			\item Pagina iniziale con titolo, logo ed informazioni del documento
			\item Sommario
			\item Registro delle modifiche
			\item Indice del documento
			\item Indice delle tabelle (se presenti)
			\item Indice delle figure (se presenti)
			\item Introduzione
		\end{itemize}
		Le pagine contengono le seguenti informazioni di base:
		\begin{itemize}
			\item[] \textbf{Intestazione}
				\begin{itemize}
					\item Logo
					\item Nome documento
				\end{itemize}
			\item[] \textbf{Piè di pagina}
				\begin{itemize}
					\item Versione
					\item Ateneo - Anno Accademico
					\item Contatto mail ufficiale di gruppo
					\item Licenza
					\item Pagina
				\end{itemize}
		\end{itemize}
	}
}