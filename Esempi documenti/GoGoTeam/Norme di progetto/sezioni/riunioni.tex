\section{Riunioni}{

	\subsection{Interne}{
		\label{commitRiu}
		\begin{enumerate}
		
		\item Ciascun componente del gruppo può avanzare una richiesta di riunione interna. Tale richiesta deve pervenire al Responsabile via e-mail ufficiale del gruppo.
			
		\item Una volta valutata la motivazione, allegata alla richiesta della nuova riunione, il Responsabile verifica la disponibilità nel calendario di tutti i membri del gruppo. 
		
		\item Il Responsabile, entro due giorni lavorativi, pubblica un nuova discussione sulla pagina dedicata al gruppo, sulla piattaforma Google Groups (cap. \ref{commint}), indicando la data e il luogo dell'incontro con prefissato il tag \texttt{[Riunione Interna]}.
		
		\item La riunione può essere svolta con l'accettazione da parte di almeno altri tre membri del gruppo (Responsabile a parte). Qualora non si raggiunga il numero minimo di componenti si ritorna al punto due.
		\end{enumerate}
		
		\subsubsection{Casi particolari}{
			La richiesta di riunione interna è stata indetta nell'avvicinamento (cinque giorni lavorativi) di una milestone\g . Se approvata dal Responsabile, la riunione verrà indetta il giorno stesso o il seguente.
		}
	}

	\subsection{Esterne}{
		Con riunioni esterne si intende qualsiasi incontro fra \textit{Proponente} / \textit{Committente}  e un gruppo di rappresentanza (composto da almeno la maggioranza assoluta) del gruppo di progetto. 
		\subsubsection{Richiesta}{
			La richiesta di indire una riunione esterna  può essere avanzata da qualsiasi componente del gruppo; è compito del Responsabile contattare ed organizzare l'evento con l'interessato. Una volta pianificato l'incontro il Responsabile deve comunicare con il resto del gruppo secondo i passi indicati alla sezione \ref{commitRiu} con tag \texttt{[Riunione Esterna]}.
		}
		\subsubsection{Esito}{
			Ad ogni incontro il Responsabile  ha il compito di stilare un verbale (cap. \ref{VerbInc}) che evidenzia i chiarimenti emersi durante l'incontro.
		}
	}
}