\section{Progettazione di dettaglio} {
\subsection{Attività}{
Lo scopo di questa attività è di realizzare una visione dettagliata di quello che dovrà essere il sistema estendendo in modo consono ed appropriato l'architettura specificata nel documento Specifica Tecnica.\\
Terminata l'attività di organizzazione in dettaglio dell'intero sistema si deve produrre un documento completo ed esplicativo: la Definizione di Prodotto.\\
Le attività necessarie alla redazione del documento  sono:
\begin{itemize}
	\item descrizione dettagliata delle singole unità (interfacce, classi, metodi e campi dati) che compongono il programma;
	\item rappresentazione grafica, mediante diagrammi di sequenza, del comportamento del programma;
	\item tracciamento componenti-requisiti.
\end{itemize}
Si deve inoltre definire, in un documento specifico (\textit{\PianoDiQualifica}), vari test da eseguire sulle classi in modo da provare il loro corretto comportamento.
\begin{itemize}
	\item []{\textbf{Input:} Specifica Tecnica;}
	\item []{\textbf{Output:} Definizione di Prodotto, test di unità;}
	\item []{\textbf{Risorse:} Progettisti, strumentazione;}
	\item []{\textbf{Misurazioni:} avanzamento dell'elaborazione del documento Definizione di Prodotto rispetto alla totalità della Specifica Tecnica;}
	\item []{\textbf{Norme:} descritte in seguito.}
\end{itemize}
}

\subsection{Diagrammi UML}{
\subsubsection{Tipologia}{
Si deve utilizzare il linguaggio UML\g~ v2.x per definire:
\begin{itemize}
	\item [] \textbf{Diagrammi delle classi:} {per rendere chiare le relazioni tra le classi all'interno delle singole componenti.}
	\item [] \textbf{Diagrammi delle attività:} {per definire in modo non ambiguo i flussi delle varie attività, composte dalle rispettive componenti, ritenuti non banali e non descritti nel documento di Specifica Tecnica.}
	\item [] \textbf{Diagrammi di sequenza:} {per definire in modo preciso le responsabilità dei vari package\g~ o classi nel portare a termine un determinato compito, nel caso in cui  ciò non fosse stato specificato adeguatamente nel documento Specifica Tecnica. Si deve limitare la rappresentazione di cicli o condizioni alternative ai soli casi in cui la loro omissione comporti un grave scompenso alla comprensione del diagramma.}
\end{itemize}
}%tipologia
\subsubsection{Classificazione}\label{classCAS}{
Ciascuna tipologia di diagramma UML\g~ dev'essere identificata mediante la seguente nomenclatura:
\begin{center}
			\textit{\{Tipo\} \{A\}...\{Z\}}
		\end{center}dove \textit{Tipo} indica una tipologia dei seguenti diagrammi:
		\begin{itemize}
			\item [] \textbf{DC:} diagramma delle classi;
			\item [] \textbf{DA:} diagramma di attività;
			\item [] \textbf{DS:} digramma di sequenza.
		\end{itemize}e \textit{\{A\}...\{Z\}} indica una struttura gerarchica di insiemi. (\textit{es.} \texttt{A.B.C}: \texttt{A} indica il diagramma principale, \texttt{B} una specifica/estensione di \texttt{A}, \texttt{C} una specifica/estensione di \texttt{B} e cosi via). \\
}%class

}%diagrammi

\subsection{Specifica delle classi}{
Nel documento è importante specificare ciascuna classe mediante il seguente schema:
\begin{itemize}
	\item {\textbf{Funzione:}} specifica la funzione svolta dalla classe;
	\item {{\textbf{Relazione con altre componenti:}} specifica la tipologia di relazione della classe con le altre componenti;
	}
	\item {\textbf{Attributi:}} definiti all'interno della classe corredati da una breve descrizione;
	\item {\textbf{Metodi:}} definiti nella classe corredati da una descrizione che ne descriva, nel caso si presentasse, eccezioni lanciate o gestite oppure tipi di oggetti di ritorno.
\end{itemize}
}%specifica classe

\subsection{Tracciamento} {
Nel documento deve essere riportato, in forma tabellare, il tracciamento di ogni requisito con le varie classi che ne assicurano il soddisfacimento.
\`E richiesto di riportare, anche questo in forma tabellare, il tracciamento di ciascuna classe con il requisito che ne giustifica l'esistenza.
}%tracciamento

\subsection{Organizzazione dei contenuti}{
Il documento Definizione di Prodotto deve contenere al suo interno:
\begin{itemize}
	\item diagrammi delle classi per ogni suo componente;
	\item specifiche e descrizioni per ogni classe specificata.
\end{itemize}
}%organizzazione dei contenuti

\subsection{Strumenti} {
Per l'esecuzione di questa attività si utilizzano i seguenti strumenti:
\begin{itemize}
	\item []{\textbf{\LaTeX :} per la stesura del documento Specifica Tecnica (dettaglio cap. \ref{TextLaTeX});}
	\item []{\textbf{Visual Paradigm:} per la stesura dei diagrammi UML\g~ (dettaglio cap. \ref{diaUML});}
	\item []{\textbf{git:} per il versionamento della documentazione (dettaglio cap. \ref{git});}
\end{itemize}

}%Strumenti

}%progettazione dettaglio