\section{Errori e possibili cause} \label{errori} {

Vengono descritte le situazioni di errore che possono avvenire durante l'uso dell'applicazione:
\begin{itemize}
	\item \textbf{E1 - Pagina bianca all'accesso dell'applicazione: }{
		questa situazione si verifica perché nel browser\g~ che si sta attualmente usando non è abilitato l'utilizzo di Javascript\g .\\
		Nelle varie versioni del browser\g~ Google Chrome\g~ è possibile risolvere il problema in due modi:
		\begin{itemize}
			\item Globale: dal menù di personalizzazione è possibile abilitare Javascript\g~ per tutti i siti web visitati;
			\item Locale: Nella barra degli indirizzi, abilitare l'esecuzione di Javascript\g~ unicamente nel sito del servizio (figura \ref{fig:JSErr}).
		\end{itemize}
	}

	\begin{center}
		\begin{figure}[h!]
		\centering
		\includegraphics[scale=0.6]{\docsImg ErroreJs.png}
		\caption{Abilitare l'esecuzione di Javascript\g~ solo per il sito corrente.}
		\label{fig:JSErr}
		\end{figure}
	\end{center}

	\item \textbf{E2 - Credenziali non corrette: }{
		questa situazione si verifica perché nella scheda di autenticazione sono stati inseriti dei dati non validi. L'errore più comune è l'inserimento di caratteri maiuscoli al posto dei minuscoli (o viceversa).\\
		Riprovare ponendo attenzione durante la digitazione dello username e della password (figure \ref{fig:LoginErr1} e \ref{fig:LoginErr2}).
	}

\begin{center}
\begin{figure}[h!]
	\centering
	\includegraphics[scale=0.4]{\docsImg ErroreLogin1.png}
	\caption{Errore di login dovuto ad errori nell'inserimento di username e password.}
	\label{fig:LoginErr1}
\end{figure}
\end{center}

\begin{center}
\begin{figure}[h!]
	\centering
	\includegraphics[scale=0.4]{\docsImg ErroreLogin2.png}
	\caption{Errore di login dovuto a username o password inesistenti.}
	\label{fig:LoginErr2}
\end{figure}
\end{center}

	\item \textbf{E3 - Errore di connessione al server\g~: }{
		quando si sono verificati dei problemi nella connessione con il server\g~ viene visualizzato un messaggio d'errore:\\
		\textit{Verificare che la propria connessione internet sia attiva e che nessun firewall\g~ impedisca la trasmissione, successivamente ricaricare la pagina.}\\
		Verificare la propria connessione ad internet, altrimenti contattare il gestore del servizio.
	}

	\item \textbf{E4 - Parametro di filtro errato: }{
		questa situazione si verifica perché si sono inseriti valori di filtro errati senza rispettare la formattazione richiesta.\\
		Riprovare ponendo attenzione durante la digitazione del valore seguendo la linea guida presente per ciascun formato (figura \ref{fig:GiudizioErr}).
	}

\begin{center}
\begin{figure}[h!]
	\centering
	\includegraphics[scale=0.3]{\docsImg ErroreGiudizio.png}
	\caption{Errore dovuto ad una ricerca in base ad un giudizio non valido.}
	\label{fig:GiudizioErr}
\end{figure}
\end{center}

\end{itemize}

Tutti i form\g~ compilati dall'utente, che richiedono degli input da inserire, sono sottoposti a validazione in tempo reale. Come si vede dalla figura (figura \ref{fig:DataErr}) a fianco dei campi che superano la validazione non compare nulla mentre i campi che non la superano vengono affiancati dalla segnalazione dell'errore riscontrato.


\begin{center}
\begin{figure}[h!]
	\centering
	\includegraphics[scale=0.3]{\docsImg ErroreData.png}
	\caption{Errore dovuto ad un formato non corretto della data.}
	\label{fig:DataErr}
\end{figure}
\end{center}



}
