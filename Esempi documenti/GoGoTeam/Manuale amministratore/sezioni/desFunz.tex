\section{Descrizione funzionale}\label{desFunz}{
{
Nel presente documento sono specificate le seguenti funzionalità offerte dal sistema:
\begin{itemize}
	\item Autenticazione, vedi cap. \ref{utNoAutent};
	\item Filtro informazioni, vedi cap. \ref{infGest}.
\end{itemize}
}

\subsection{Utente non autenticato}\label{utNoAutent}{

\begin{figure}[h!]
	\centering
		\includegraphics[scale=0.5]{\docsImg Login.png}
		\caption{Interfaccia dell'applicazione in fase di login.}
		\label{fig:imgLogin} 
	\end{figure} 
	
Nella figura \ref{fig:imgLogin} viene mostrata l'interfaccia che l'utente non autenticato incontra accedendo al servizio \textbf{\mytalk}. Da questa è possibile accedere al servizio inserendo le proprie credenziali di accesso:
\begin{itemize}
	\item email;
	\item password.
\end{itemize}
	
Di default viene fornito un utente amministratore con le seguenti credenziali:
\begin{itemize}
	\item Utente: admin@mytalk.com
	\item Password: adminmytalk
\end{itemize}
Per la gestione degli utenti amministratori, si rimanda al \textit{\ManualeInstallatore}.
}


\subsection{Filtro informazioni}\label{infGest}{
\subsubsection{Interfaccia generale}{
Una volta effettuata l'autenticazione, l'utente può interagire con le funzionalità offerte dal servizio \textbf{\mytalk}.
L'utente amministratore ha la possibilità di monitorare tutte le informazioni relative alle comunicazioni tra utenti.
In figura \ref{fig:layoutGen} viene mostrata l'interfaccia che l'utente amministratore trova al suo accesso.


La tabella a sinistra dispone in ordine temporale tutte le informazioni relative alle comunicazioni avvenute negli ultimi 7 giorni.
}

\subsubsection{Filtro delle informazioni}{
Attraverso i pulsanti presenti nell'area a destra, figura \ref{fig:layoutGen} , è possibile applicare un filtro di ricerca alle informazioni presenti. Il risultato viene mostrato nella tabella di sinistra il cui contenuto viene aggiornato di conseguenza.\\
Le operazioni possibili sono:
\begin{itemize}
	\item \textbf{Giorno: }{Inserendo una specifica data in formato \texttt{GG-MM-AAAA} (dove \texttt{GG} indica il giorno, \texttt{MM} indica il mese e \texttt{AAAA} indica l'anno) è possibile avere le informazioni delle comunicazioni relative dal giorno selezionato (figura \ref{fig:imgGiorno}).
	}
	
	\item \textbf{Utente: }{Inserendo uno specifico riferimento e-mail di un utente è possibile avere informazioni di comunicazione relative alle comunicazioni effettuate (figura \ref{fig:imgLogin}).

	}
	\item \textbf{Lista: }{Selezionando un specifico riferimento e-mail di un utente presente nella lista è possibile avere informazioni di comunicazione relative alle comunicazioni effettuate (figura \ref{fig:imgList}).
}

	\item \textbf{Giudizio: }{Inserendo un valore di giudizio compreso tra 1 e 5 è possibile avere informazioni di comunicazione relative alle comunicazioni che hanno ricevuto quel giudizio (figura \ref{fig:imgGiudizio}).

}
\end{itemize}

}
\begin{figure}[h!]
	\centering
		\includegraphics[scale=0.4]{\docsImg FiltroGiorno.png}
		\caption{Interfaccia dell'applicazione in fase di ricerca tramite giorno.}
		\label{fig:imgGiorno} 
\end{figure}

\begin{figure}[h!]
	\centering
		\includegraphics[scale=0.4]{\docsImg FiltroUtente.png}
		\caption{Interfaccia dell'applicazione in fase di ricerca tramite inserimento riferimento utente.} 
		\label{fig:imgLogin}
\end{figure}

	\begin{figure}[h!]
	\centering
		\includegraphics[scale=0.4]{\docsImg FiltroLista.png}
		\caption{Interfaccia dell'applicazione in fase di ricerca utente tramite lista.}
		\label{fig:imgList} 
\end{figure}

\begin{figure}[h!]
	\centering
		\includegraphics[scale=0.4]{\docsImg FiltroGiudizio.png}
		\caption{Interfaccia dell'applicazione in fase di ricerca tramite giudizio espresso.}
		\label{fig:imgGiudizio}
\end{figure}
}
\newpage