\section{Descrizione generale}\label{desGen}{

\subsection{Installazione del prodotto}{
Per l'installazione dell'applicazione web-based\g~ si veda il documento \textit{\ManualeInstallatore}.
}

\subsection{Descrizione interfaccia grafica}{
{
L'utente può interagire con l'applicazione \textbf{\mytalk} tramite un'interfaccia grafica semplice ed intuitiva, tale da facilitare le operazioni che si intendono effettuare.\\
In questa sezione si dà una descrizione delle funzionalità offerte e una rapida guida all'uso.
}

\subsection{Layout\g~ dell'applicazione}{
In questa sezione vengono descritte le aree di lavoro che compongono il layout\g~ dell'interfaccia attraverso la quale l'utente amministratore può interagire con \textbf{\mytalk}.
\begin{figure}[h!]
	\centering
	\includegraphics[scale=0.4]{\docsImg Statistic.png}
	\caption{Layout\g~ dell'applicazione.}
	\label{fig:layoutGen} 
\end{figure}

L'area all'interno del riquadro blu è informativa: contiene il titolo e il collegamento per richiamare la schermata di autenticazione/deautenticazione a \textbf{\mytalk}.\\
L'area delimitata dal riquadro verde presenta il menu di navigazione dove sono elencate le operazioni che l'utente amministratore può compiere al fine di interagire con il sistema. Una volta scelta un'azione, nell'area delimitata dal riquadro magenta viene visualizzato un insieme di informazioni relative alla scelta.
	}
}%Descrizione interfaccia grafica

}