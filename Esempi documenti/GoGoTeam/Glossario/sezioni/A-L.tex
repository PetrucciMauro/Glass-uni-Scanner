\hfill\Huge{\textbf{0-9}}\\
\normalsize
	\begin{longtable}{p{0.25\textwidth} p{0.75\textwidth}} 
	    \toprule
	    \\
	    \textbf{3DS}:	&	è uno dei formati utilizzati da 3D Studio Max per la modellazione 3D, animazione e rendering$_{|g|}$ software$_{|g|}$.\\
	\end{longtable}
\newpage

\hfill\Huge{\textbf{A}}\\
\normalsize
	\begin{longtable}{p{0.25\textwidth} p{0.75\textwidth}} 
	    \toprule
	    \\
	    \textbf{Android}:		&	\underline{sistema operativo}$_{|g|}$ \underline{open source}$_{|g|}$ per dispositivi mobili sviluppato da Google Inc.\\
	    \\
	    \textbf{AJAX}: 		&	acronimo di Asynchronous JavaScript$_{|g|}$ and XML$_{|g|}$, è una tecnica per la realizzazione di applicazioni web$_{|g|}$ interattive. Lo sviluppo di applicazioni HTML$_{|g|}$ con AJAX 
						si basa su uno scambio di dati in background fra web$_{|g|}$ browser$_{|g|}$ e server$_{|g|}$, che consente l’aggiornamento dinamico di una pagina web$_{|g|}$ senza esplicito ricaricamento 
						da parte dell’utente. AJAX è asincrono nel senso che i dati extra sono richiesti al server$_{|g|}$ e caricati in  background senza interferire con il comportamento della pagina 
						esistente.\\
	    \\
	    \textbf{Apache}:		&	detto anche \textit{Apache HTTP Server}, è un \underline{web server}$_{|g|}$ \underline{open source}$_{|g|}$.\\
	    \\
	    \textbf{Apache Tomcat}: 	&	contenitore servlet$_{|g|}$ \underline{open source}$_{|g|}$ che fornisce una piattaforma software$_{|g|}$ per l’esecuzione di applicazioni web$_{|g|}$ sviluppate in linguaggio Java$_{|g|}$. 
						La sua distribuzione standard include anche le funzionalità di \underline{web server}$_{|g|}$ tradizionale, che corrispondono al prodotto Apache$_{|g|}$.\\
	    \\
	    \textbf{API}:		&	acronimo che significa Application Programming Interface (Interfaccia di Programmazione di un’Applicazione) e indica un insieme di procedure disponibili al 
						programmatore che forniscono un insieme di strumenti specifici per compiere un determinato compito all’interno di un certo programma. Le API permettono di 
						evitare ai programmatori di riscrivere ogni volta tutte le funzioni necessarie al programma da zero, favorendo così il riuso del codice. \\
	    \\
	    \textbf{Applet}:		&	piccole applicazioni scritte in linguaggio Java$_{|g|}$ che vengono eseguite sul client$_{|g|}$ all’interno di un programma contenitore, tipicamente un browser$_{|g|}$, al fine di potenziare i servizi offerti 
						dalla pagina web$_{|g|}$ che non potrebbero essere forniti tramite codice HTML$_{|g|}$. Tipicamente un’applet contiene un’interfaccia utente, al contrario degli script$_{|g|}$.\\
	    \\
	    \textbf{Assessment}:	&	momento finale del ciclo di vita di un sistema software$_{|g|}$, nel quale si effettua una diagnosi dello stato del sistema, a fronte dei bisogni aziendali,
						per capire se è necessario attuare interventi correttivi.\\
	    \\
	    \textbf{aXAPI}:		&	API$_{|g|}$ basata sull'architettura REST$_{|g|}$ che consente l'interazione remota da applicazioni di terze parti per il controllo del bilanciamento del carico del server$_{|g|}$.\\
	\end{longtable}
\newpage


\hfill\Huge{\textbf{B}}\\
\normalsize
	\begin{longtable}{p{0.25\textwidth} p{0.75\textwidth}} 
	    \toprule
	    \\
	    \textbf{Back-end}:		&	parte di un sistema software$_{|g|}$ che elabora i dati generati dal front-end$_{|g|}$.\\
	    \\
	    \textbf{Best practice}:	&	in italiano migliore prassi, indica le esperienze più significative, o comunque quelle che hanno permesso di ottenere i migliori risultati 
						in un determinato contesto. \`E un'idea che asserisce l'esistenza di una tecnica, un metodo, un processo o un'attività, che sono più efficaci
						nel raggiungere un particolare risultato, di qualunque altra tecnica, metodo, processo, o attività. \\
	    \\
	    \textbf{Booleano}:  	&	l'algebra di Boole, anche detta algebra booleana o reticolo booleano, è un'algebra astratta che opera essenzialmente con i soli valori di verità 0 e 1.\\
	    \\
	    \textbf{Bottom-up}:		&	tipo di strategia di elaborazione dell'informazione nel quale vengono specificate le parti individuali nel dettaglio, 
						per poi connetterle tra di loro per formare componenti più grandi, fino a formare il sistema completo.\\
	    \\
	    \textbf{Break}:		&	istruzione che forza l’uscita da un ciclo \texttt{while}, \texttt{for}, \texttt{do...while} prima della fine di quest’ultimo.
						Viene utilizzata anche nel costrutto \texttt{switch}$_{|g|}$ per evitare l’esecuzione di tutti i casi dello \texttt{switch}$_{|g|}$.\\
	    \\
	    \textbf{Brainstorming}:	&	tecnica di gruppo volta a far emergere delle tecniche di soluzione di un problema, in cui ogni componente propone soluzioni di qualsiasi tipo,
						senza che nessuna di esse venga censurata. La scrematura delle soluzioni proposte avviene in un secondo momento.\\
	    \\
	    \textbf{Breakpoint}:	&	segnale che indica al debugger$_{|g|}$ di sospendere temporaneamente l’esecuzione di un programma in qualsiasi punto, permettendo comunque di riprenderla in 
						qualsiasi momento. Tutti gli elementi del programma (funzioni, variabili, oggetti, etc.) rimangono attivi, ma le loro attività sono sospese, permettendo 
						l’individuazione di violazioni e bug$_{|g|}$.\\
	    \\
	    \textbf{Bridge JDBC-ODBC}: 	&	un'implementazione di driver per i database che utilizza ODBC$_{|g|}$. Il driver converte le chiamate ai metodi JDBC in chiamate a funzioni ODBC$_{|g|}$. Un bridge di 
						solito si usa quando non è disponibile un driver in Java$_{|g|}$ per un certo database; in realtà questa situazione era più frequente tempo fa, quando JDBC non era 
						ancora abbastanza diffuso.\\
	    \\
	    \textbf{Browser}:		&	programma che fornisce uno strumento per navigare in Internet e interagire con i contenuti presenti nel World Wide Web. Più tecnicamente, un browser è 
						un’applicazione client$_{|g|}$ che utilizza il protocollo HTTP$_{|g|}$ per inoltrare le richieste dell’utente ad un \underline{web server}$_{|g|}$.\\
	    \\
	    \textbf{Bug}:		&	errore o guasto di programmazione che porta al malfunzionamento di un programma. La causa della maggior parte del numero di bug è spesso il
						codice sorgente scritto dal programmatore, ma può anche accadere che sia il compilatore stesso a produrne. Un programma che contiene un gran numero
						di bug che interferiscono con la sua funzionalità è detto bacato.\\
	    \\
	    \textbf{Bytecode}: 		&	linguaggio intermedio più astratto tra il linguaggio macchina e il linguaggio di programmazione, usato per descrivere le operazioni che costituiscono un programma.\\
	\end{longtable}
\newpage


\hfill\Huge{\textbf{C}}\\
\normalsize
	\begin{longtable}{p{0.25\textwidth} p{0.75\textwidth}} 
	    \toprule
	    \\
	    \textbf{C/C++}:			&	linguaggi di programmazione ad alto livello. La differenza tra C e C++ sta nel supporto agli oggetti, assente nel C e presente nel C++.\\
	    \\
	    \textbf{Chrome}:			&	browser$_{|g|}$ sviluppato da Google, basato su \underline{motore di rendering}$_{|g|}$ WebKit$_{|g|}$.\\
	    \\
	    \textbf{Client}:			&	computer o programma che inoltra le richieste dell’utente ad un programma server$_{|g|}$. L’esempio più comune è un browser$_{|g|}$ che invia richieste ad un \underline{web server}$_{|g|}$ per poter visionare le pagine che interessano.\\
	    \\
	    \textbf{Client-server}:		&	architettura di rete nella quale un computer client$_{|g|}$ si connette ad un computer server$_{|g|}$ utilizzandone i servizi offerti.\\
	    \\
	    \textbf{Cloud computing}: 		&	in italiano nuvola informatica, indica un insieme di tecnologie che permettono di memorizzare ed elaborare dati, grazie all’utilizzo di risorse hardware$_{|g|}$ e software$_{|g|}$ distribuite in rete.\\
	    \\
	    \textbf{Cluster}:			&	si indica un agglomerato di di oggetti dello stesso tipo. Un cluster è un grupo di server$_{|g|}$ o di altre risorse che agisce come un unico sistema e garantisce quindi performances migliori.\\
	    \\
	    \textbf{CMMI}:			&	acronimo che sta per Capability Maturity Model Integration, è un approccio al miglioramento dei processi il cui obiettivo è di aiutare le organizzazioni a migliorare 
							le loro prestazioni, fornendo loro gli elementi essenziali per un efficace miglioramento dei processi.\\
	    \\
	    \textbf{Compilatore}:		&	programma che traduce una serie di istruzioni (codice sorgente), scritte in un particolare linguaggio di programmazione, in istruzioni di un altro linguaggio (codice oggetto) interpretabile direttamente dal computer.\\
	    \\
	    \textbf{Cookie}:			&	piccoli file contenenti stringhe di testo inviati da un \underline{web server}$_{|g|}$ a un \underline{web client}$_{|g|}$. Sono utilizzati per eseguire autenticazioni automatiche, tracciare sessioni e memorizzare informazioni 
							specifiche degli utenti che accedono al server$_{|g|}$.\\
	    \\
	    \textbf{Copertura}:			&	misura utilizzata nel test del software$_{|g|}$ per determinare quale percentuale del codice del progetto viene effettivamente testata dai test codificati come unit test (test d'unità).\\
	    \\
	    \textbf{Core}:			&	si intende tipicamente il "nucleo elaborativo" di un microprocessore. Questo infatti è costituito in realtà da 2 componenti principali: il core appunto, e il package$_{|g|}$ che lo contiene.\\
	    \\
	    \textbf{Cross-platform:}		&	in italiano multipiattaforma, indica un’applicazione in grado di funzionare su più di un sistema o, appunto, piattaforma (\textit{p.es.} Windows, Unix e Macintosh). Può essere riferito ad un linguaggio di programmazione, 
							ad un'applicazione software$_{|g|}$ o ad un dispositivo hardware$_{|g|}$. Alcuni esempi di linguaggi multipiattaforma sono: C$_{|g|}$, C++$_{|g|}$, Java$_{|g|}$, JavaScript$_{|g|}$.\\
	    \\
	    \textbf{CSS/CSS3}:			&	acronimo che sta per Cascading Style Sheet (fogli di stile a cascata), è un linguaggio utilizzato per dare uno stile agli elementi HTML$_{|g|}$ e definirne il layout all’interno 
							di una pagina web$_{|g|}$. Per esempio, i CSS si occupano dei font, dei colori, dei margini, delle linee, delle altezze, delle larghezze, delle immagini di sfondo, del posizionamento 
							e di molte altre cose.\newline
							L’ultima versione di CSS è denominata CSS3, non è ancora uno standard del W3C$_{|g|}$ ma è pienamente compatibile all’indietro e inizia ad essere supportata da tutti i principali browser$_{|g|}$.\\
	\end{longtable}
\newpage


\hfill\Huge{\textbf{D}}\\
\normalsize
	\begin{longtable}{p{0.25\textwidth} p{0.75\textwidth}} 
	    \toprule
	    \\
	    \textbf{Database}:		&	archivio dati, o un insieme di archivi, in cui le informazioni in esso contenute sono strutturate e collegate tra loro secondo un particolare modello logico e in modo tale da consentire la gestione/organizzazione efficiente dei 
						dati stessi grazie a particolari applicazioni software$_{|g|}$ dedicate (DBMS$_{|g|}$), basate su un'architettura di tipo client-server$_{|g|}$, e ai cosiddetti \underline{query language}$_{|g|}$ per l'interfacciamento con le richieste dell'utente 
						(query$_{|g|}$ di ricerca o interrogazione, inserimento, cancellazione ed aggiornamento).\\
	    \\
	    \textbf{DBMS}:		&	software$_{|g|}$ che consente di creare e gestire un database$_{|g|}$. Se compatibile SQL$_{|g|}$ permette di essere interrogato e modificato da parte di altre applicazioni.\\
	    \\
	    \textbf{Default}: 		&	ci si riferisce allo stato o alla risposta di un sistema qualunque in assenza (per difetto, cioè in mancanza) di interventi espliciti (ad esempio input o configurazioni dell'utente), ovverosia predefinito.\\
	    \\
	    \textbf{Debugger}:		&	strumento per il debugging$_{|g|}$. \`E un programma/software$_{|g|}$ specificatamente progettato per l'analisi e l'eliminazione dei bug$_{|g|}$. 
						Assieme al compilatore è fra i più importanti strumenti di sviluppo a disposizione di un programmatore, spesso compreso all'interno di un ambiente integrato di sviluppo (IDE$_{|g|}$).\\
	    \\
	    \textbf{Debugging}:		&	attività che consiste nell'individuazione della porzione di software$_{|g|}$ affetta da errore (bug$_{|g|}$) rilevato a seguito dell'utilizzo del programma. 
						L'errore può essere rilevato sia in fase di collaudo del programma, quando cioè questo è ancora in fase di sviluppo e non è stato ancora dichiarato pronto per essere utilizzato dall'utente finale, 
						sia in fase di utilizzo del programma da parte dell'utente finale. Alla rilevazione dell'errore segue la fase di debugging.\\
	    \\
	    \textbf{Deliverable}:	&	nell’ambito del project management, indica un oggetto materiale o immateriale prodotto come risultato di un’attività di progetto. 
						Un deliverable può essere costituito da un insieme di deliverable più piccoli. In altre parole si tratta di un risultato verificabile prodotto da un task (attività).\\
	    \\
	    \textbf{Driver}:		&	componente attiva fittizia utilizzata in fase di testing per simulare una parte di software$_{|g|}$ non ancora implementata. In particolare,
						questa componente viene utilizzata nel test d'integrazione bottom-up per simulare un'unità chiamante.\\
	    \\
	    \textbf{Duplex}:		&	modalità di trasmissione e ricezione di informazioni bidirezionale attraverso un canale di comunicazione. Si contrappone al simplex, che è invece monodirezionale.
						Si distingue inoltre tra \underline{full-duplex}$_{|g|}$ e \underline{half-duplex}$_{|g|}$.\\
	\end{longtable}
\newpage


\hfill\Huge{\textbf{F}}\\
\normalsize
\label{tabVers}
	\begin{longtable}{p{0.25\textwidth} p{0.75\textwidth}} 
	    \toprule
	    \\
	    \textbf{Flash}:		&	software$_{|g|}$ grafico che permette di creare animazioni vettoriali, principalmente utilizzato in ambito web$_{|g|}$ per applicazioni come giochi, contenuti audio/video 
						o anche interi siti web$_{|g|}$.\\
	    \\
	    \textbf{Firefox}:		&	browser$_{|g|}$ web$_{|g|}$ \underline{open source}$_{|g|}$ creato da Mozilla Foundation e basato su \underline{motore di rendering}$_{|g|}$ Gecko$_{|g|}$.\\
	    \\
	    
	    \textbf{Framework}:		&	piattaforma riusabile utilizzata per per sviluppare prodotti software$_{|g|}$. Esempi di framework sono compilatori, librerie di codice, API$_{|g|}$.
						Un framework ha un comportamento di default, che non può essere modificato ma può essere esteso dall’utente (programmatore in questo caso) in base alle sue necessità.\\
	    \\
	    \textbf{Freeware}:		&	software$_{|g|}$ che viene distribuito in modo gratuito, con o senza codice sorgente (a discrezione dell'autore). 
						La licenza ne permette la redistribuzione gratuita.\\
	    \\
	    \textbf{Front-end}:		&	parte di un sistema software$_{|g|}$ che gestisce l’interazione con l’utente o con sistemi esterni che producono dati in ingresso. Si distingue dal back-end$_{|g|}$.\\
	    \\
	    \textbf{FTP}:		& 	acronimo di File Transfer Protocol è un protocollo per la trasmissione di dati tra host$_{|g|}$ basato su TCP$_{|g|}$.\\
	    \\
	    \textbf{Full-duplex}:	&	è un sistema che permette la comunicazione in entrambe le direzioni ed essa, diversamente dall'half-duplex$_{|g|}$, può avvenire simultaneamente. Le reti telefoniche terrestri sono 
						full-duplex, permettono cioè di parlare e ascoltare nello stesso momento.\\
	\end{longtable}
\newpage


\hfill\Huge{\textbf{G}}\\
\normalsize
\label{tabVers}
	\begin{longtable}{p{0.25\textwidth} p{0.75\textwidth}} 
	    \toprule
	    \\
	    \textbf{Gecko}:			&	scritto in linguaggio C++$_{|g|}$, è progettato per supportare gli standard aperti usati su Internet. Gecko offre una ricca API$_{|g|}$ progettata per le applicazioni 
							legate al web$_{|g|}$, come i browser$_{|g|}$, la presentazione dei contenuti e i programmi \underline{client-server}$_{|g|}$.\\
	    \\
	    \textbf{Google App }\newline
	    \textbf{Engine}:			& 	piattaforma di \underline{cloud computing}$_{|g|}$ gestita dai Google Data Center che permette l’hosting$_{|g|}$ e sviluppo applicazioni di Google. App Engine offre scalabilità automatica per le 
							applicazioni a seconda del numero di richieste per quell’applicazione.\\
	    \\
	    \textbf{Google Chrome }\newline
	    \textbf{Frame}:			& 	plugin$_{|g|}$ gratuito per \underline{Internet Explorer}$_{|g|}$ che integra nel browser$_{|g|}$ Microsoft alcune delle funzionalità più avanzate offerte da 
							Google Chrome$_{|g|}$, come il supporto alle più recenti tecnologie HTML5$_{|g|}$, un motore JavaScript$_{|g|}$ più efficiente e il supporto alle WebRTC$_{|g|}$.\\
	    \\
	    \textbf{GUI}:			&	acronimo che sta per Graphical User Interface (Interfaccia Utente Grafica). Si tratta di un’interfaccia che permette all’utente di interagire con il programma basandosi su oggetti 
							grafici e testo. Un esempio di GUI è l’ambiente desktop a finestre dei sistemi operativi.\\
	\end{longtable}
\newpage


\hfill\Huge{\textbf{H}}\\
\normalsize
\label{tabVers}
	\begin{longtable}{p{0.25\textwidth} p{0.75\textwidth}} 
	    \toprule
	    \\
	    \textbf{Half-duplex}:		&	è un sistema che fornisce una comunicazione in entrambe le direzioni, ma con la possibilità di usare soltanto una direzione alla volta (non simultaneamente). Quando una parte comincia 
							a ricevere un segnale deve poi aspettare che il trasmettitore interrompa la trasmissione prima di poter rispondere (ad esempio: walkie-talkie).\\
	    \\
	    \textbf{HAProxy}:			&	soluzione \underline{open source}$_{|g|}$ veloce ed affidabile che offre alta disponibilità, bilanciamento del carico e proxy$_{|g|}$ per i protocolli TCP$_{|g|}$ e HTTP$_{|g|}$-based.\\
	    \\
	    \textbf{Hardware}:			&	la parte fisica di un computer. Include le parti elettriche, elettroniche, meccaniche, ottiche e le periferiche.\\
	    \\
	    \textbf{Hash}:			&	tabella hash, struttura dati utilizzata per mettere in corrispondenza una data chiave con un dato valore. Se ben dimensionata diventa particolarmente efficiente, dato che il costo 
							dell’operazione di ricerca è indipendente dal numero di elementi che la tabella contiene.\\
	    \\
	    \textbf{Hole punching}: 		&	tecnica sviluppata nel mondo del gaming per stabilire una connessione \underline{peer-to-peer}$_{|g|}$ tra due host che stanno dietro ad un NAT$_{|g|}$ o ad un firewall$_{|g|}$.\\
	    \\
	    \textbf{Host}:			&	ogni terminale collegato ad una rete o più in particolare ad Internet. Un host può essere di diverso tipo, ad esempio computer, palmari, dispositivi mobili e così via, fino a includere web TV, dispositivi domestici.\\
	    \\
	    \textbf{Hosting}:			&	disponibilità, a pagamento o gratuita, di uno spazio su un \underline{web server}$_{|g|}$ per la registrazione di file personali o per lo sviluppo di un sito.\\
	    \\
	    \textbf{HTML/HTML5}:		&	acronimo che sta per HyperText Markup Language. \`E un linguaggio di markup$_{|g|}$ utilizzato per descrivere la struttura delle pagine web$_{|g|}$.\newline
							La nuova versione di HTML è denominata HTML5, non è ancora uno standard del W3C$_{|g|}$ ma è in una fase di definizione avanzata ed è già supportata dai principali browser$_{|g|}$.\\
	    \\
	    \textbf{HTTP}:	 		&	acronimo che sta per HyperText Transfer Protocol (protocollo di trasferimento di un ipertesto). \`E un protocollo di tipo richiesta/risposta, ed è quello principalmente usato 
							per la trasmissione di dati sul web$_{|g|}$.\\
	\end{longtable}
\newpage


\hfill\Huge{\textbf{I}}\\
\normalsize
\label{tabVers}
	\begin{longtable}{p{0.25\textwidth} p{0.75\textwidth}} 
	    \toprule
	    \\
	    \textbf{ICE}: 			&	acronimo che sta per Interactive Connectivity Establishment. \`E un protocollo standardizzato per l’\underline{hole punching}$_{|g|}$. Ogni host$_{|g|}$ ha una serie di potenziali indirizzi candidati, che vengono scambiati attraverso un 
							server$_{|g|}$ (detto rendezvous). Entrambi gli host$_{|g|}$ inviano pacchetti, denominati “pacchetti di \underline{hole punching}$_{|g|}$”, più o meno allo stesso tempo. Questo invio di pacchetti permette di mappare le regole del NAT$_{|g|}$ e di filtrarle, 
							fino a stabilire una connessione.\\
	    \\
	    \textbf{IDE}: 			&	acronimo che sta per Integrated Developement Environment (Ambiente di Sviluppo Integrato). Software$_{|g|}$ che, in fase di programmazione, aiuta i programmatori nello sviluppo del codice sorgente di un programma. 
							In genere consiste in un editor di testo, un compilatore, un debugger$_{|g|}$ e un’utility per la creazione di GUI$_{|g|}$.\\
	    \\
	    \textbf{IEC}:			&	acronimo che sta per International Electronic Commission. \`E un’organizzazione internazionale per la definizione di standard in materia di elettronica, elettricità e tecnologie correlate. 
							Molti dei suoi standard sono definiti in collaborazione con ISO$_{|g|}$.\\
	    \\
	    \textbf{Internet Explorer}:		&	oggi noto anche come Windows Internet Explorer, è un browser$_{|g|}$ web$_{|g|}$ proprietario sviluppato da Microsoft.\\
	    \\
	    \textbf{IP}:			&	acronimo che sta per Internet Protocol. \`E il protocollo attraverso il quale i dati vengono inviati da un computer all’altro in Internet.
							Ogni computer collegato ad Internet ha un indirizzo IP che lo identifica univocamente.\\
	    \\
	    \textbf{ISO}:			&	acronimo che sta per International Organization for Standardization. \`E la più importante organizzazione internazionale per la definizione di norme tecniche.\\
	    \\
	    \textbf{ISO/OSI}: 			&	(Open Systems Interconnection) è uno standard per reti di calcolatori, stabilito nel 1978 dall'ISO$_{|g|}$, che stabilisce per l'architettura logica di rete un'architettura a strati composta da una pila di protocolli suddivisa in 7 livelli, 
							i quali insieme espletano in maniera logico-gerarchica tutte le funzionalità della rete.
	\end{longtable}
\newpage

\hfill\Huge{\textbf{J}}\\
\normalsize
\label{tabVers}
	\begin{longtable}{p{0.25\textwidth} p{0.75\textwidth}} 
	    \toprule
	    \\
	    \textbf{Java}:		&	linguaggio di programmazione ad oggetti creato dalla Sun Microsystems. Può essere utilizzato per creare applicazioni che possono essere eseguite su una singola macchina oppure essere distribuite 
						su server$_{|g|}$ e client$_{|g|}$ di una rete. Java è un linguaggio portabile, nel senso che può essere eseguito su ogni computer che disponga di una \underline{Java Virtual Machine (JVM)}$_{|g|}$. Java è un marchio registrato di Oracle.\\   
	    \\
	    \textbf{JavaScript}:	&	linguaggio di scripting (cioè un linguaggio interpretato, dal browser$_{|g|}$ in questo caso, il cui scopo è l’interazione con altri programmi più complessi) orientato agli oggetti, usato comunemente nei siti web$_{|g|}$.\\
	    \\
	    \textbf{JDK}:		&	(Java Development Kit) dall'introduzione di Java$_{|g|}$ è sempre stato l'ambiente di sviluppo più utilizzato dai programmatori Java$_{|g|}$.\\
	    \\
	    \textbf{JPG}:		&	detto anche JPEG, è un acronimo che sta per Joint Photographic Experts Group. \`E lo standard internazionale più diffuso per la compressione di immagini. 
						La sua codifica è lossy (a perdita di informazione) e il suo formato è aperto (cioè la sua specifica è pubblica).\\
	    \\
	    \textbf{JQuery}:		&	libreria di funzioni JavaScript$_{|g|}$, \underline{cross-platform}$_{|g|}$ per le applicazioni web$_{|g|}$, che si propone come obiettivo quello di semplificare la programmazione lato client$_{|g|}$ delle pagine HTML$_{|g|}$.\\
	    \\
	    \textbf{JSON}:		&	acronimo che sta per JavaScript Object Notation. \`E un formato adatto per lo scambio dei dati in applicazioni \underline{client-server}$_{|g|}$.\\
	    \\
	    \textbf{JVM}:		&	acronimo che sta per Java Virtual Machine. \`E il componente della piattaforma Java$_{|g|}$ che interpreta i programmi compilati.\\
	\end{longtable}
\newpage

\hfill\Huge{\textbf{L}}\\
\normalsize
\label{tabVers}
	\begin{longtable}{p{0.25\textwidth} p{0.75\textwidth}} 
	    \toprule
	    \\
	    \textbf{LAN}:		&	acronimo che sta per Local Area Network (rete in area locale). \`E una tipologia di rete informatica contraddistinta da un’estensione territoriale limitata, di solito inferiore a qualche chilometro. 
						Un esempio di LAN è una rete installata in un’abitazione o in un’azienda all’interno di un edificio, o al massimo più edifici adiacenti tra loro.\\
	    \\
	    \textbf{Latenza}:		&	intervallo di tempo che intercorre tra il momento in cui arriva l’input al sistema ed il momento in cui è disponibile il suo output.
						In altre parole, la latenza è la velocità di risposta di un sistema.\\
	    \\
	    \textbf{Layout}:		&	descrive la disposizione appropriata degli elementi che compongono una pagina web$_{|g|}$.\\
	    \\
	    \textbf{Lazy evaluation}:	&	in italiano valutazione pigra, è una tecnica che consiste nel ritardare una computazione finché il risultato non è richiesto effettivamente. Ad esempio, nel caso della valutazione 
						della condizione d’ingresso dei cicli o degli \texttt{if}, se questa è composta da due espressioni booleane$_{|g|}$ che sono legate tra loro da un \texttt{and} e la prima è falsa, la seconda non viene valutata.\\
	    \\
	    \textbf{Livello datalink}: 	&	è il secondo livello dei protocolli del modello \underline{ISO/OSI}$_{|g|}$ per l'interconnessione di sistemi aperti. Questo livello in trasmissione riceve pacchetti dal livello di rete e forma i frame che 
						vengono passati al sottostante livello fisico con l'obiettivo di permettere il trasferimento affidabile dei dati attraverso il sottostante canale.\\
	    \\
	    \textbf{Load balancing}:	&	in italiano bilanciamento del carico, tecnica che consiste nel distribuire il carico di uno specifico servizio, ad esempio la fornitura di un sito web$_{|g|}$, tra più server$_{|g|}$. Si aumentano così la 
						scalabilità e l'affidabilità di tutta l’architettura nel suo complesso. La scalabilità deriva dal fatto che, nel caso sia necessario, si possono aggiungere nuovi server al cluster$_{|g|}$, mentre la 
						maggiore affidabilità deriva dal fatto che la rottura di uno dei server non compromette la fornitura del servizio; non a caso i sistemi di load balancing in genere integrano dei sistemi di monitoraggio 
						che escludono automaticamente dal cluster$_{|g|}$ i server$_{|g|}$ non raggiungibili ed evitano in questo modo di far fallire una porzione delle richieste degli utenti.\\
	\end{longtable}
\newpage