\section{Consuntivo e preventivo a finire}{

	La seguente sezione mette in evidenza come le ore preventivate per ogni ruolo e persone, durante tutto lo sviluppo del progetto, abbiano avuto dei cambiamenti.\\
	Il cambiamento del numero di ore può aumentare o diminuire. La causa di tale mutamento può essere dovuta ad una scorretta pianificazione rispetto a un certo periodo o al verificarsi di alcuni rischi non previsti o mal gestiti.
	
	\subsection{Fase dalla RR alla RP}{
	\subsubsection{Consuntivo}{
	\begin{table}[h!]
		\begin{center}
			\begin{tabular}{l c c c c c c}				
				\toprule
				&	RE& AM& AN& PRT& PRM& VE \\ 
				\midrule
				\BM	& 0	& 9	& 0  	& 28 	& 0 	& 0\\ 
				\BA	& 0	& 0	& 12 	& 4  	& 0 	& 16 \\
				\CD	& 0	& 0	& 11 	& 28 	&0 	& 0\\ 
				\LS	& 6	& 5	& 12 	& 0  	&0 	& 6\\
				\PV 	& 0	& 0	& 11 	& 27 	&0 	& 0\\
				\ZF 	& 10	& 0	& 11 	& 0  	&0 	& 16\\
				\ZE 	& 0	& 5	& 11 	& 0  	&0	& 14\\ \hline
				\textbf{Totale}&	 16 &	19 &		68	&	87 &		0 & 52\\
				\bottomrule
			\end{tabular}	
		\caption{Consuntivo di distribuzione ruoli tra i componenti del gruppo.}
		\end{center}
	\end{table}

	\begin{figure}[h!]
	\centering
		\includegraphics[scale=0.55]{\docsImg ConPre.png}
		\caption{Consuntivo per ruolo.} 
	\end{figure}

	\newpage
	\begin{table}[h!]
		\begin{center}
			\renewcommand{\arraystretch}{1.5}% Wider
			\begin{tabular}{c| c c c c c c c |c }
				\toprule
				\multicolumn{2}{c}{}&	RE& AM& AN& PRT& PRM& VE& Totale \\ 
				\midrule
				\multirow{2}{*}{Ore}	& Preventivo	& 16	& 21	& 74  	& 84 	& 0 	& 50	& 245\\ 
							& Consuntivo	& 16 	& 19 	& 68	& 87 	& 0 	& 52	& 242\\ \hline
				\multirow{2}{*}{Costo}	& Preventivo	& 480	& 420	& 1850 	& 1848 	& 0 	& 750	& 5348\\ 
							& Consuntivo	& 480	& 380 	& 1700  & 1914 	& 0	& 780	& 5254\\
				\bottomrule
			\end{tabular}
		\end{center}
		\caption{Confronto tra consuntivo e preventivo su ore e costi.}
		\label{rc_RR-RP}
	\end{table}

\textbf{Note sul consuntivo}{
Dal confronto tra consuntivo e preventivo, vedi tabella \ref{rc_RR-RP}, si può notare che il budget fissato per il completamento dell'attuale fase non è stato superato, anzi lo sforzo economico è stato al di sotto delle previsioni.\\
Si può notare che le ore svolte per il ruolo di Analista a consuntivo, sono state inferiori a quelle preventivate, in quanto l'attività di tracciamento componenti-requisiti si è rivelata più agevole di quanto previsto, vista la semplicità con la quale l'applicazione può essere suddivisa. In questo modo le ore in avanzo sono state sfruttate dal Progettista.\\
Il risparmio per questa fase è di 3 ore, equivalenti in termini economici a 94 euro, e sarà sfruttato per risolvere, ove necessario, problemi che potranno presentarsi in futuro.
	  }
	 }
	 \subsubsection{Preventivo a finire}{
	 Per questa fase è stato speso il \textbf{38\%} della spesa totale preventivata.
		 	\begin{figure}[h!]
	\centering
		\includegraphics[scale=0.85]{\docsImg RR-RP.png}
		\caption{Spesa rimanente per fase RP}  	
	\end{figure}
	\newpage
	\begin{figure}[h!]
	\centering
		\includegraphics[scale=0.85]{\docsImg RR_RP1.png}
		\caption{Spesa rimanente per ruolo RP} 
	
	\end{figure}
	\begin{figure}[h!]
	\centering
		\includegraphics[scale=0.85]{\docsImg faseRP.png}
		\caption{Confronto tra la spesa sostenuta in RP e quella totale} 
	
	\end{figure}
	 }
	
	}
\newpage
	\subsection{Fase dalla RP alla RQ}{
	\subsubsection{Consuntivo}
	\begin{table}[h!]
		\begin{center}
			\begin{tabular}{l c c c c c c}				
				\toprule
				&	RE& AM& AN& PRT& PRM& VE \\ 
				\midrule
				\BM	&	0	&	0	&	5	&	0	&	0	&	45\\ 
				\BA	&	0	&	0	&	0	&	15	&	0	&	33\\
				\CD	&	5	&	4	&	0	&	14	&	29	&	0\\ 
				\LS	&	0	&	0	&	0	&	24	&	8	&	15\\
				\PV &	2	&	8	&	0	&	0	&	25	&	26\\
				\ZF 	&	0	&	6	&	0	&	17	&	14	&	18\\
				\ZE &	2	&	0	&	0	&	13	&	10	&	17\\ \hline
				\textbf{Totale}&	 9 &	18 &	5	&	83 	&	86 	& 154\\
				\bottomrule
			\end{tabular}	
		\caption{Consuntivo di distribuzione ruoli tra i componenti del gruppo.}
		\end{center}
	\end{table}

	\begin{figure}[h!]
	\centering
		\includegraphics[scale=0.85]{\docsImg ConRQ.png}
		\caption{Consuntivo per ruolo.} 
	\end{figure}


	\begin{table}[h!]
		\begin{center}
			\renewcommand{\arraystretch}{1.5}% Wider
			\begin{tabular}{c| c c c c c c c |c }
				
				\toprule
				\multicolumn{2}{c}{}&	RE& AM& AN& PRT& PRM& VE& Totale \\ 
				\midrule
				\multirow{2}{*}{Ore}	& Preventivo	& 13& 15	& 6& 80 	& 75 & 170	& 359\\ 
							& Consuntivo	& 9 	& 18 	& 5	& 83 & 86 & 154	& 355\\ \hline
				\multirow{2}{*}{Costo}	& Preventivo	& 390	& 300	& 150 	& 1760 	& 1125 	& 2550	& 6275\\ 
							& Consuntivo	& 270	& 360 	& 125  & 1826 	& 1290	& 2310	& 6181\\
				\bottomrule
			\end{tabular}
		\end{center}
		\caption{Confronto tra consuntivo e preventivo su ore e costi.}
		\label{rc_RP-RQ}
	\end{table}

\textbf{Note sul consuntivo}{
 In questa fase il gruppo ha riscontrato diverse difficoltà identificabili in alcuni rischi elencati nella sezione \ref{cap:AnalisiRischi} (Analisi e procedure di mitigazione dei rischi).\\
Il Responsabile è intervenuto per attuare la soluzione più appropriata con lo scopo di evitare un ritardo insanabile e un aumento dei costi eccessivo del progetto.\\
L'assenza di alcuni componenti e la scarsa conoscenza delle tecnologie utilizzate ha causato notevoli problemi. Inoltre, ha comportato l'avverarsi di un altro rischio, ovvero lo slittamento della consegna per la prima Revisione di Qualifica, in quanto il gruppo non avrebbe avuto una preparazione adeguata.\\
I rischi sopra elencati erano stati identificati come rischi ad elevata probabilità di verificarsi.
\begin{itemize}
\item[•] Per sanare le lacune tecnologiche sono state dedicate molte ore, non figurabili nel contratto, dedicate allo studio delle tecnologie causando un ulteriore ritardo sulla fase di codifica.
\item[•] Per quanto riguarda l'assenza di alcuni componenti, i loro compiti sono stati svolti dalle persone che erano presenti in quel momento a titolo gratuito. Il Responsabile ha dovuto riorganizzare i ruoli tenendo conto che non si presentasse un conflitto tra essi.
\end{itemize}
Dal confronto tra consuntivo e preventivo, vedi tabella \ref{rc_RP-RQ}, si può notare che il budget fissato per il completamento dell'attuale fase è stato inferiore a quanto preventivato. Le ore mancanti causate dell'assenza dei componenti non potranno essere recuperate in fase RA . Il gruppo ha ritenuto necessario aggiungere ore in più, rispetto a quanto preventivato, all'amministratore, progettista, programmatore e verificatore.

	  }
	  
	\subsubsection{Preventivo a finire}{
	
	 Per questa fase è stato speso il \textbf{45\%} della spesa totale preventivata.
	\begin{figure}[h!]
	\centering
		\includegraphics[scale=0.85]{\docsImg RP-RQ.png}
		\caption{Spesa rimanente per fase RQ}  	
	\end{figure}
	\begin{figure}[h!]
	\centering
		\includegraphics[scale=0.85]{\docsImg RP-RQ1.png}
		\caption{Spesa rimanente per ruolo RQ} 
	
	\end{figure}
	
	\begin{figure}[h!]
	\centering
		\includegraphics[scale=0.85]{\docsImg faseRQ.png}
		\caption{Confronto tra la spesa sostenuta in RP e quella totale} 
	
	\end{figure}
	
\newpage
In questa fase è stato speso circa l'83\% della spesa totale.
La nuova pianificazione delle attività prevede una consegna della Revisione di accettazione con conseguente
conclusione del progetto alla seconda consegna del periodo estivo (luglio 2013) e quindi alla quarta convocazione. Visto il consumo quasi totale delle ore fino a qui previste e la prolungata assenza di alcuni componenti, sarà necessaria l’erogazione di ore aggiuntive da parte di alcuni membri del gruppo a titolo gratuito. Si prevede infatti che parte delle ore ancora disponibili nel periodo tra RQ ed RA saranno impiegate per la correzione delle problematiche evidenziate dal committente. Per minimizzare le ore di volontariato si prevede che si potranno soddisfare solo i requisiti obbligatori e saranno trascurati quelli desiderabili ed opzionali.
	}
	
\newpage	
	\subsection{Fase dalla RQ alla RA}{
	
	Le ore disponibili per questa fase saranno utilizzate per la correzione delle problematiche evidenziate dal \textit{Committente.}\\
	
	
	\subsubsection{Consuntivo}
	\begin{table}[h!]
		\begin{center}
			\begin{tabular}{l c c c c c c}				
				\toprule
				&	RE& AM& AN& PRT& PRM& VE \\ 
				\midrule
				\BM	&	5	&	0	&	0	&	0	&	13	&	0\\ 
				\BA	&	2	&	3	&	0	&	0	&	0	&	0\\
				\CD	&	0	&	0	&	0	&	0	&	0	&	14\\ 
				\LS	&	0	&	0	&	0	&	0	&	5	&	24\\
				\PV &	1	&	3	&	0	&	0	&	2	&	0\\
				\ZF 	&	0	&	2	&	0	&	2	&	0	&	9\\
				\ZE &	0	&	0	&	0	&	0	&	10	&	23\\ \hline
				\textbf{Totale}&	 8 &	8 &	0	&	2 &		30 & 70\\
				\bottomrule
			\end{tabular}	
		\caption{Consuntivo delle ore per ruolo per persona.}
		\end{center}
	\end{table}
	
		\begin{table}[h!]
		\begin{center}
			\renewcommand{\arraystretch}{1.5}% Wider
			\begin{tabular}{c| c c c c c c c |c }
				
				\toprule
				\multicolumn{2}{c}{}&	RE& AM& AN& PRT& PRM& VE& Totale \\ 
				\midrule
				\multirow{2}{*}{Ore}	& Preventivo & 13& 14& 0&  8& 31 & 65	& 131\\ 
							& Consuntivo & 8 	& 8 	& 0	& 2 & 30 & 70	& 118\\ \hline
				\multirow{2}{*}{Costo}	& Preventivo & 390	& 280	& 0 	& 176 	& 465 	& 975 & 2286\\ 
							& Consuntivo & 240	& 160 & 0 &  44 	& 450	& 1050	& 1944\\
				\bottomrule
			\end{tabular}
		\end{center}
		\caption{Preventivo a finire delle ore e dei costi.}
		%\label{rc_RR-RP}
	\end{table}
	
		\textbf{Note sul consuntivo}{
	A causa dei problemi riscontrati nella fase precedente e la continua assenza di un componente, in questa fase si è deciso di correggere solo gli errori individuati dal \textit{Committente} e aggiungere qualche requisito desiderabili. Quelli opzionali saranno trascurati. 
	Rispetto a quanto era stato preventivato, il gruppo ha risparmiato 530 euro.
		\begin{table}[h!]
		\begin{center}
			\begin{tabular}{l c c}				
				\toprule
					&Ore totali& Costo totale \\ 
				\midrule
				preventivo	&	735	& 13909\\ 
				consuntivo	&	715	& 13379\\
				\bottomrule
			\end{tabular}	
		\caption{Confronto tra consuntivo e preventivo.}
		\end{center}
	\end{table}
	}
	}
	
	\newpage
	\subsection{Ruoli aziendali}{
	Nella tabella \ref{fig: ruoli} viene riportato la ripartizione oraria dei differenti ruoli impegnati nel progetto.\\
	Le ore impiegate per la terminazione del prodotto sono concentrate nei ruoli dei progettista e del verificatore. Questo perché il gruppo ha ritenuto opportuno e necessario concentrarsi nell'attività di verifica e validazione.
	
	\begin{table}[h!]
		\begin{center}
			\begin{tabular}{l c c c c c c c}				
				\toprule
				&	RE& AM& AN& PRT& PRM& VE& Totale \\ 
				\midrule
				\BM	&	5	&	9	&	5	&	28	&	13	&	45	&	105\\ 
				\BA	&	2	&	3	&	12	&	19	&	0	&	49 	&	85\\
				\CD	&	5	&	4	&	11	&	42	&	29	&	14 	&	105\\ 
				\LS	&	6	&	5	&	12	&	24	&	13	&	45  &	105\\
				\PV &	3	&	11	&	11	&	27	&	27	&	26  &	105\\
				\ZF 	&	10	&	8	&	11	&	19	&	14	&	43  &	105\\	
				\ZE &	2	&	5	&	11	&	13	&	20	&	54	&	105\\ \hline
				\textbf{Totale per ruolo}&	 33 &	45 &	73	&	172 &	116 & 276 & 715\\ \hline
				Percentuale	&	4,71\% & 6,42\% & 10,42\% & 24,57\% & 16,22\% &  39,43\% & 100\%\\ 
				Preventivo & 42 & 50 & 80  & 172 & 106 & 285 & 735\\
				\bottomrule
			\end{tabular}	
		\caption{Le ore che il singolo dipendente ha dedicato ai ruoli aziendali.} \label{fig: ruoli}
		\end{center}
	\end{table}
	
	
	\begin{figure}[h!]
	\centering
		\includegraphics[scale=0.85]{\docsImg IstograConsuntivo.png}
		\caption{Grafico ore consuntivate per componente}  	
	\end{figure}
	Si può infine notare dalla figura \ref{fig: torta} che nonostante le 20 ore in meno i due ruoli a progetto che assumono maggiore importanza sono quelli del Verificatore e del Progettista. Entrambi sottolineano la qualità e la funzionalità del prodotto; i Progettisti pongono le basi a priori mentre i Verificatori la accertano a posteriori.
	 Si rimanda alla figura \ref{fig: prev}, corrispondente al preventivo progetto MyTalk, per osservare le precise differenze percentuali.
	\newpage
	\begin{figure}[h!]
	\centering
		\includegraphics[scale=0.85]{\docsImg TortaConsuntivo.png}
		\caption{Grafico ore consuntivate per ruolo} \label{fig: torta}
	\end{figure}
	}
}
