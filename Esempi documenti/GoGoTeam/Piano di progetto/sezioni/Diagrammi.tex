\section{Diagrammi temporali}{
	Per ogni revisione è possibile visualizzare la pianificazione delle attività.
Per accelerare le fasi, più persone ricoprono nello stesso lasso di tempo il medesimo ruolo, suddividendo tra loro le attività previste dalla fase in questione.\\
Prendendo come esempio l'attività di analisi nella fase di sviluppo dalla RR alla RP (figura \ref{pp}), si può notare come sia garantito il lavoro di tre analisti in contemporanea impiegando le risorse umane a disposizione per l'attività (figura \ref{rrrpRuoli}).

%\newpage
	\subsection{Pianificazione temporale dalla fase RR alla fase RP}{
	\begin{figure}[h!]
		\centering
		\includegraphics[scale=0.7]{\docsImg RR_RP_Sottoattivita.png}
		\caption{Diagramma di Gantt per la fase RP.}
		\label{pp}
	\end{figure}
	}
\newpage
	\subsection{Pianificazione temporale dalla fase RP alla fase RQ}{
	\begin{figure}[h!]
		\centering
		\includegraphics[scale=0.7]{\docsImg RP_RQ_Sottoattivita.png}
		\caption{Diagramma di Gantt per la fase RQ.}
		
	\end{figure} 
	
	}
\newpage	
	\subsection{Pianificazione temporale dalla fase RQ alla fase RA }{
	\begin{figure}[h!]
		\centering
		\includegraphics[scale=0.7]{\docsImg RQ_RA_Sottoattivita.png}
		\caption{Diagramma di Gantt per la fase RA.}
	\end{figure} 
	}

}
