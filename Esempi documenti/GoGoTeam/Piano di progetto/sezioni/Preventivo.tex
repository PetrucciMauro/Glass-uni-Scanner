\section{Preventivo}{
	\subsection{Prospetto orario}{
		Il gruppo conta sette studenti, considerando la durata del progetto che è di circa dieci settimane, ognuno di essi lavorerà al suo sviluppo per una media di due ore giornaliere (cinque giorni a settimana), per un totale di centocinque ore a persona.
Considerando lo scopo didattico del progetto e dall'esplicita richiesta dal docente, ogni componente può ricoprire più ruoli, sia contemporaneamente che in fasi distinte del progetto, in ogni caso garantendo sempre assenza di conflitto di interessi tra i ruoli assunti. Nella distribuzione dei ruoli e l'assegnazione delle attività ai componenti sarà garantita un'equa ripartizione del carico di lavoro individuale.
In tabella vengono evidenziate le ore coperte dal singolo componente per il singolo ruolo e la totalità delle ore che ogni membro del gruppo ha dedicato al progetto.
\\Aggiungere grafici.
}
	\subsection{Prospetto economico}{
		Nella tabella \ref{CostiRuolo} viene riportato il riepilogo delle ore associate a ogni singolo ruolo presente in tabella \ref{tabellaCostiUnitari} e il costo a carico del \textit{Committente}. Viene inoltre calcolato il costo totale. \\
	\begin{table}[h!]
	 \centering
		\begin{tabular}{l c c}		
				\toprule
				Ruolo&	Costo totale ruolo(\euro)&	Ore totali di lavoro \\ 
				\midrule
				Responsabile&	1260	&	42\\
				Amministratore&	1000	& 50\\
				Analista&	2000	&	80\\
				Progettista&		3784&	172\\
				Programmatore&	1590	&	10\\
				Verificatore&	4275&	285\\ \hline
				\textbf{Totale}&	 13909&	735\\
				\bottomrule
			\end{tabular}
			\caption{Preventivo costi per ruolo}	
			 \label{CostiRuolo}
	\end{table}	
	\begin{figure}[h!]
	\centering
		\includegraphics[scale=0.55]{\docsImg CostiPerRuolo.pdf}
		\caption{Costi per ruolo} 
	\end{figure} 
	%Anche per i costi il ruolo predominante riguarda il Progettista.
	}
}
