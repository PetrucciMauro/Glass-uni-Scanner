
\section{Gestione dei rischi}{
	
	\subsection{Identificazione del rischio}{
	L'analisi dei rischi, alla quale il progetto è sottoposto, è una delle attività più importanti e più difficile da programmare. Per l'identificazione di essi il gruppo ha deciso di utilizzare  un approccio brainstorming\g, che permette di ottenere una visione più ampia possibile dell'argomento trattato, anche in ambiti nei quali le persone non hanno esperienza. 
	}
	
	\subsection{Quantificazione del rischio}{
		In questa sezione si descrivono alcuni rischi che possono manifestarsi durante la realizzazione del prodotto.\\ Il rischio è definito come ``valore di un effetto indesiderabile che rappresenta una minaccia per il raggiungimento di un dato obiettivo. Un rischio di processo minaccia la tempistica o il costo del processo; un rischio del prodotto può determinare il mancato soddisfacimento di alcuni requisiti del sistema'' (cit. Sommerville).\\
	Di seguito, nella tabella, si stabiliscono le probabilità di rischio:	
	\begin{table}[h]
	 %\label{}
		\begin{center}
		%\rowcolors{2}{light}{}
			%\begin{tabular}{p{0.30\textwidth} p{0.25\textwidth} p{0.35\textwidth}}	
			\begin{tabular}{l l }				
				\toprule
				Livello di probabilità	&	Criterio	 \\ \hline
				Basso& \`E improbabile che  si verifichi \\
				Medio& Uguale probabilità che si verifichi o meno \\
				Alto&  Altamente probabile che si verifichi\\
				\bottomrule
			\end{tabular}
		\end{center}

	\caption{Probabilità di rischio}
		%\label{t1}
	\end{table}
	}
	
	\newpage
	\subsection{Analisi e procedure di mitigazione dei rischi}{
	\label{cap:AnalisiRischi}
	Di seguito, per ogni rischio identificato, viene delineata un'analisi generale e una procedura di mitigazione.
		
	\begin{table}[h!]
	\scriptsize
		\begin{center}
			%\rotatebox{90}{
				\begin{minipage}{1\linewidth}
					\begin{tabular}{l l c l}				
					\toprule
					Rischio&	 Analisi	& Probabilità & Procedura di mitigazione \\ 
					\midrule
					\begin{minipage}{0.2\linewidth}
					Assenza di un \newline componente.
					\end{minipage}
					&
					\begin{minipage}{0.3\linewidth}
					Il valore del rischio è proporzionale al ruolo che quel dipendente ha assunto in quel periodo o in quella fase specifica.
					\end{minipage}				
					 & 
					 alta
					 &
					 \begin{minipage}{0.3\linewidth}
					 Assegnare ad un altro membro del gruppo il ruolo mancante, evitando un conflitto tra i ruoli. Se l'assenza è permanente, sarà necessaria una riorganizzazione dei ruoli tra i membri restanti.
					\end{minipage}
					\\
					\\
					\bottomrule
					\\
					\begin{minipage}{0.2\linewidth}
					Scarsa conoscenza \newline delle \newline tecnologie utilizzate.
					\end{minipage}										
					 &
					\begin{minipage}{0.3\linewidth}
					Alcune tecnologie che saranno utilizzate durante lo sviluppo del progetto sono sconosciute ai membri del gruppo.
					\end{minipage}
					& 
					alta
					&
					\begin{minipage}{0.3\linewidth}
					Studio della documentazione online fornita dagli enti che sviluppano le tecnologie adottate.
					\end{minipage}	
					\\
					\\
					\bottomrule
					\\
					\begin{minipage}{0.2\linewidth}
					Difficoltà \newline nell'applicazione.
					\end{minipage}
					&
					\begin{minipage}{0.3\linewidth}
					Differenze sostanziali  tra quanto pianificato e quanto si rivelerà a consuntivo. I ritardi potrebbero essere causati da un'insufficiente organizzazione o da una scorretta  assegnazione dei ruoli, per i quali è stata richiesta la rotazione obbligatoria da parte del docente.
					\end{minipage}
					&
					media
					&
					 \begin{minipage}{0.3\linewidth}
					Riassegnazione dei ruoli attraverso una veloce ripianificazione, cercando di evitare ulteriori sprechi di tempo.
					\end{minipage}
					\\
					\\
					\bottomrule
					\\
					\begin{minipage}{0.2\linewidth}				
					Uso di una \newline piattaforma \newline di recente sviluppo.
					\end{minipage}
					&
					\begin{minipage}{0.3\linewidth}
					 WebRTC\g~e WebSocket\g, non ancora divenute standard, possano mutare anche significativamente durante la fase di sviluppo.
					\end{minipage}					 
					& 
					media
					&
					\begin{minipage}{0.3\linewidth}
					Verranno pianificati opportuni workaround\g~per garantire lo sviluppo e il funzionamento del prodotto.\newline
					Se il cambiamento non potrà essere risolto con questo approccio, verrà richiesta una riunione col \textit{Committente/Proponente} per discutere di un cambio di piattaforma o una ridefinizione del prodotto stesso o di alcuni requisiti.
					\end{minipage}
					\\  
					\\
					\bottomrule
					\\
					\begin{minipage}{0.2\linewidth}	
					Variazione \newline dei requisiti. 
					\end{minipage}
					&
					\begin{minipage}{0.3\linewidth}
					 Il \textit{Committente} non ha escluso l'avanzamento di variazioni ai requisiti nel corso di qualsiasi fase di sviluppo del progetto.
					\end{minipage}	
					& 
					media
					 &
					\begin{minipage}{0.3\linewidth}
					Il Responsabile di Progetto e il team discuteranno delle modifiche da apportare all’analisi dei requisiti e, di conseguenza, all’intero progetto, in modo da calcolare il prima possibile una nuova stima dei relativi costi e ridurre il danno.
					\end{minipage}	
					 \\
					 \\
					 \bottomrule
					\\
					\begin{minipage}{0.2\linewidth}
					Preparazione \newline poco adeguata \newline per una revisione.
					\end{minipage}
					& 
					\begin{minipage}{0.3\linewidth}
					L’esito negativo di una revisione tenuta con il \textit{Committente}  è un rischio presente in ogni fase di sviluppo. Il verificarsi di questo evento crea inevitabilmente ritardi dovuti all’obbligatoria correzione e verifica della revisione.
					\end{minipage}
					& 
					media
					&
					\begin{minipage}{0.3\linewidth}
					Miglioramento nel metodo di lavoro. Diventano indispensabili l’attività di controllo, il flusso di informazioni costante sulle specifiche riguardanti gli ambiti nei quali si sta lavorando e gli incontri con il \textit{Committente}, al fine di comprendere le esigenze dello stesso e le specifiche del progetto.
					\end{minipage}	
					\\
					\\
					\bottomrule
					\\
					\begin{minipage}{0.2\linewidth}
					Linguaggi di \newline programmazione.
					\end{minipage}
					&
					\begin{minipage}{0.3\linewidth}
					Non si ritiene che l’uso del linguaggio di programmazione imposto dal \textit{Committente} possa comportare particolari problemi.
					\end{minipage}	 
					&
					bassa
					\\  
					\bottomrule
					\end{tabular}
				\end{minipage}
				%}
			
		\end{center}	
		\caption{Analisi e procedure di mitigazione dei rischi}
	\end{table}
	
	
	}
}

