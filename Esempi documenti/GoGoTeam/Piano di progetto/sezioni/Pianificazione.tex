\section{Preventivo e pianificazione delle singole fasi}{
	\subsection{Prospetto orario}{
		\label{RRRRR}
		Il gruppo conta sette studenti. Considerando la durata del progetto che è di circa dieci settimane, ognuno di essi lavorerà al suo sviluppo per un totale di centocinque ore a persona. Ogni componente potrà gestirsi queste ore in base ai propri impegni personali e professionali, pur garantendo il raggiungimento degli obiettivi di progetto nei tempi prefissati.\\
Considerando lo scopo didattico del progetto e l'esplicita richiesta dal docente, ogni componente deve ricoprire più ruoli, sia contemporaneamente che in fasi distinte del progetto, in ogni caso garantendo sempre assenza di conflitto di interessi tra i ruoli assunti. Nella distribuzione dei ruoli e l'assegnazione delle attività ai componenti sarà garantita un'equa ripartizione del carico di lavoro individuale.
In tabella vengono evidenziate le ore coperte dal singolo componente per il singolo ruolo e la totalità delle ore che ogni membro del gruppo dedicherà al progetto.

\begin{table}[h!]
		\scriptsize
		\begin{center}
			\begin{tabular}{l c c c c c c c}				
				\toprule
				&	Responsabile & Amministratore & Analista & Progettista & Programmatore & Verificatore & \textbf{Totale}\\ 
				\midrule
				\BM	& 6	& 11& 6  & 28 & 11 & 43 & 105\\ 
				\BA	& 3	& 3	& 13 & 18  & 10 & 58  & 105\\
				\CD	& 8	& 5	& 12 & 38 &28 & 14  & 105\\ 
				\LS	& 6	& 5	& 13 & 30  &10 & 41  & 105\\
				\PV & 3	& 12	& 12 & 26 &26 & 26  & 105\\
				\ZF & 10	& 9	& 12 & 16  &16 & 42  & 105\\
				\ZE & 6	& 5	& 12 & 16  &5	 & 61  & 105\\
				\bottomrule
			\end{tabular}
		\end{center}	
		\caption{Preventivo orario per singolo componente.}
\end{table}	


}
	\subsection{Ruoli e costi}{
	I costi e i ruoli sono consultabili alla tabella \ref{tabellaCostiUnitari}.
	\begin{table}[h!]
	 \centering
		\begin{tabular}{p{0.25\textwidth} p{0.15\textwidth}}	
				\toprule
				Ruolo			&	Costo (\euro) \\ 
				\midrule
				Amministratore &	 20\\
				Analista 	&	25\\
				Progettista	& 	22\\ 
				Programmatore & 15\\
				Responsabile & 30  \\
				Verificatore &	15  \\
				\bottomrule
			\end{tabular}
			\caption{Costi unitari per ruolo}	
			 \label{tabellaCostiUnitari}
	\end{table}	
	}
 

\newpage
	\subsection{Preventivo progetto \mytalk}{
	Di seguito viene presentato il conseguente preventivo delle ore e dei costi per l'intero progetto.
		\begin{table}[h!]
	% \label{tabellaFirma}
		\begin{center}
		%\rowcolors{2}{light}{}
			%\begin{tabular}{p{0.30\textwidth} p{0.25\textwidth} p{0.35\textwidth}}	
			\begin{tabular}{l c c c}				
				\toprule
				Ruolo&	Costo totale ruolo(\euro) 	&	Ore totali di lavoro &	Peso del ruolo \%  \\ 
				\midrule
				Responsabile&	1260	&	42&	5,71\\
				Amministratore&	1000	& 50&	6,80\\
				Analista&	2000	&	80&	10,88\\
				Progettista&		3784&	172&	 23,40\\
				Programmatore&	1590	&	106&		14,42\\
				Verificatore&	4275&	285&		38,78\\ \hline
				\textbf{Totale}&		13909&	735&\\
				\bottomrule
			\end{tabular}
		\end{center}	
		\caption{Stima del costo e delle ore totali.}
	\end{table}
	\begin{figure}[h!]
	\centering
		\includegraphics[scale=0.55]{\docsImg TortaTotale.pdf}
		\caption{Stima delle percentuali dei ruoli attivi nello sviluppo dell'intero progetto.} 
		\label{fig: prev}
		\includegraphics[scale=0.55]{\docsImg IstogrammaTotale.pdf}

	\caption{Stima delle ore totali previste per ogni ruolo.}
	
	\end{figure} 
	}
\newpage	
\subsection{Pianificazione per la revisione di progettazione ad alto livello (RP)}{
\begin{table}[h!]
	% \label{tabellaFirma}
		\begin{center}
		%\rowcolors{2}{light}{}
			%\begin{tabular}{p{0.30\textwidth} p{0.25\textwidth} p{0.35\textwidth}}	
			\begin{tabular}{l c c c}				
				\toprule
				Ruolo&	 Costo totale ruolo(\euro) 	&	Ore totali lavoro &	Peso del ruolo \% \\ 
				\midrule
				Responsabile&	480&		16&	6,53\\
				Amministratore&	420&		21&	8,57\\
				Analista&	1850&	74&	30,20\\
				Progettista&		1848	&	84&	34,29\\
				Programmatore&	0&	0&	0	\\
				Verificatore&	750&		50&	20,41\\ \hline
				\textbf{Totale}&		5348&	245	&\\
				\bottomrule
			\end{tabular}
		\end{center}	
		\caption{Stima del costo e delle ore per la fase di sviluppo dalla RR alla RP.}
	\end{table}
	\begin{figure}[h!]
	\centering
		\includegraphics[scale=0.55]{\docsImg TortaRP.pdf}
		\caption{Stima delle percentuali dei ruoli attivi per la fase di sviluppo dalla RR alla RP.}  		
		\includegraphics[scale=0.55]{\docsImg IstogrammaRP.pdf}

	\caption{Stima delle ore previste per la fase di sviluppo dalla RR alla RP.}
	
	\end{figure} 
	
		\textbf{Periodo:} dal 2013-01-14 al 2013-01-29\\
	Si stende la Specifica Tecnica, si integrano il Piano di Qualifica, il Piano di Progetto, le Norme di Progetto e si aggiorna l'Analisi dei Requisiti.
		 Si prevede di consegnare per la Revisione  in data 2013-01-30 i seguenti documenti:
		 \begin{itemize}
		 	\item AnalisiDeiRequisiti\textunderscore v2.0.pdf
		 	\item Glossario\textunderscore v2.0.pdf
		 	\item NormeDiProgetto\textunderscore v2.0.pdf
		 	\item PianoDiProgetto\textunderscore v2.0.pdf
		 	\item PianoDiQualifica\textunderscore v2.0.pdf
		 	\item SpecificaTecnica\textunderscore v1.0.pdf
		 	\item StudioDiFattibilita\textunderscore v2.0.pdf
		 \end{itemize}
		
	}
	\newpage
	\subsection{Pianificazione per la revisione di qualifica (RQ)}{
	\begin{table}[h!]
	% \label{tabellaFirma}
		\begin{center}
		%\rowcolors{2}{light}{}
			%\begin{tabular}{p{0.30\textwidth} p{0.25\textwidth} p{0.35\textwidth}}	
			\begin{tabular}{l c c c}				
				\toprule
				Ruolo&	 Costo totale ruolo(\euro) 	&	Ore totali di lavoro &	Peso del ruolo \% \\ 
				\midrule
			Responsabile	&	390&		13&	3,62 \\
			Amministratore&	300&		15&	4,18 \\
			Analista	&	150&		6&	1,67\\
			Progettista&		1760	&	80&	22,28\\
			Programmatore&	1125	&	75&	20,89\\
			Verificatore	&	2550&	170&		47,35\\ \hline
				\textbf{Totale}&		6275&	359	&\\
				\bottomrule
			\end{tabular}
		\end{center}	
		\caption{Stima del costo e delle ore per la fase di sviluppo dalla RP alla RQ.}
	\end{table}
	\begin{figure}[h!]
	\centering
		\includegraphics[scale=0.55]{\docsImg TortaRQ.pdf}
		\caption{Stima delle percentuali dei ruoli attivi per la fase di sviluppo dalla RP alla RQ.}  		
		\includegraphics[scale=0.55]{\docsImg IstogrammaRQ.pdf}

	\caption{Stima delle ore previste per la fase di sviluppo dalla RP alla RQ.}
	
	\end{figure} 	
	Si vuole mettere in evidenza che durante la fase di RQ il ruolo dell'analista è ancora attivo per 6 ore. In seguito al rilascio del primo prototipo c'è la necessità di effettuare l'analisi dei requisiti desiderabili e opzionali che si è scelto di sviluppare in comune accordo con il \textit{Committente}.
	
	\textbf{Periodo:} dal 2013-02-04 al 2013-02-26\\
		\begin{itemize}
			\item Progettazione di dettaglio
			\item Stesura documento definizione di prodotto
			\item Inizio codifica: verrà creato un prototipo da proporre al \textit{Proponente}. Tale prototipo conterrà delle funzionalità minimali
			\item Test
		\end{itemize}
		 Si completa la Specifica Tecnica, si estendono la Definizione di Prodotto e i Manuali, si integra il Piano di Progetto e si aggiornano il Piano di Qualifica, le Norme di Progetto, l'Analisi dei Requisiti.
		 Si prevede di consegnare per la Revisione  in data 2013-02-27 i seguenti documenti:
		 \begin{itemize}
		 	\item AnalisiDeiRequisiti\textunderscore v3.0.pdf
		 	\item DefinizioneDiProdotto\textunderscore v1.0.pdf
		 	\item Glossario\textunderscore v3.0.pdf
		 	\item ManualeAmministratore\textunderscore v1.0.pdf
		 	\item ManualeUtente\textunderscore v1.0.pdf
		 	\item NormeDiProgetto\textunderscore v3.0.pdf
		 	\item PianoDiProgetto\textunderscore v3.0.pdf
		 	\item PianoDiQualifica\textunderscore v3.0.pdf
		 	\item SpecificaTecnica\textunderscore v2.0.pdf
		 \end{itemize}
		
	}
	\newpage
	\subsection{Pianificazione per la revisione di accettazione (RA)}{
		\begin{table}[h!]
	% \label{tabellaFirma}
		\begin{center}
		%\rowcolors{2}{light}{}
			%\begin{tabular}{p{0.30\textwidth} p{0.25\textwidth} p{0.35\textwidth}}	
			\begin{tabular}{l c c c}				
				\toprule
				Ruolo&	 Costo totale ruolo(\euro) 	&	Ore totali di lavoro &	Peso del ruolo \% \\ 
				\midrule
				Responsabile&	390&		13&	9,92\\
				Amministratore&	280&		14&	10,69\\
				Analista	&	0&	0& 0\\
				Progettista&		176&	8&	6,11\\
				Programmatore&	465&	31&	23,66\\
				Verificatore	&	975&		65&	49,62\\ \hline
				\textbf{Totale}&		2286&	131	&\\
				\bottomrule
			\end{tabular}
		\end{center}	
		\caption{Stima del costo e delle ore per la fase di sviluppo dalla RQ alla RA.}
	\end{table}
	\begin{figure}[h!]
	\centering
		\includegraphics[scale=0.55]{\docsImg TortaRA.pdf}
		\caption{Stima delle percentuali dei ruoli attivi per la fase di sviluppo dalla RQ alla RA.}  		
	\includegraphics[scale=0.55]{\docsImg IstogrammaRA.pdf}

	\caption{Stima delle ore previste per la fase di sviluppo dalla RQ alla RA.}
	
	\end{figure} 	
		\textbf{Periodo:} dal 2013-03-04 a fine marzo\\
		\begin{itemize}
			\item Test e correzione del prodotto
			\item Implementazione di alcuni requisiti desiderabili
			\item Incremento Manuale Utente
		 
		\end{itemize}
		 Si completano tutti i documenti.
		 Si prevede di consegnare per la Revisione i seguenti documenti:
		 \begin{itemize}
		 	\item AnalisiDeiRequisiti\textunderscore v3.0.pdf
		 	\item DefinizioneDiProdotto\textunderscore v2.0.pdf
		 	\item Glossario\textunderscore v4.0.pdf
		 	\item ManualeAmministratore\textunderscore v2.0.pdf
		 	\item ManualeUtente\textunderscore v2.0.pdf
		 	\item NormeDiProgetto\textunderscore v4.0.pdf
		 	\item PianoDiProgetto\textunderscore v4.0.pdf
		 	\item PianoDiQualifica\textunderscore v4.0.pdf
		 	\item SpecificaTecnica\textunderscore v3.0.pdf
		 \end{itemize}
	}
	\newpage
	
	\subsection{Distribuzione dei ruoli tra i singoli membri}{
	 Di seguito verranno adottate le seguenti abbreviazioni:
	\begin{itemize}
		\item []\textbf{RE:}	 Responsabile
		\item []\textbf{AM:} Amministratore
		\item []\textbf{AN:}	Analista
		\item []\textbf{PRT:} Progettista
		\item []\textbf{PRM:} Programmatore
		\item []\textbf{VE:} Verificatore
	\end{itemize}		 
	Ogni singolo componente del gruppo, come indicato in nella sezione \ref{RRRRR}, ricoprirà tutti i ruoli aziendali garantendo assenza di conflitti d'interesse come indicato nella sezione 5.8 di \emph{\NormeDiProgetto}.
	\subsubsection{Preventivo rotazione ruoli dalla RR alla RP}{
	\begin{table}[h!]
		\begin{center}
			\begin{tabular}{l c c c c c c}				
				\toprule
				&	RE& AM& AN& PRT& PRM& VE \\ 
				\midrule
				\BM	& 0	& 11& 0  & 28 & 0 & 0\\ 
				\BA	& 0	& 0	& 13 & 2  & 0 & 16 \\
				\CD	& 0	& 0	& 12 & 28 &0 & 0\\ 
				\LS	& 6	& 5	& 13 & 0  &0 & 4\\
				\PV & 0	& 0	& 12 & 26 &0 & 0\\
				\ZF & 10	& 0	& 12 & 0  &0 & 16\\
				\ZE & 0	& 5	& 12 & 0  &0	 & 14\\ \hline
				\textbf{Totale}&	 16 &	21 &		74	&	84 &		0 & 50\\
				\bottomrule
			\end{tabular}
		\end{center}	
		\caption{Copertura dei ruoli dalla RR alla RP.}
	\end{table}	
	
	\begin{figure}[h!]
	\centering
		\includegraphics[scale=0.55]{\docsImg RR_RP_risorse.png}
		
		\caption{Calendario dettagliato di rotazione dei ruoli.}	
		\label{rrrpRuoli}	
	\end{figure} 
	}
	\newpage
	\subsubsection{Preventivo rotazione ruoli dalla RP alla RQ}{
	\begin{table}[h!]
		\begin{center}
			\begin{tabular}{l c c c c c c}				
				\toprule
				&	RE& AM& AN& PRT& PRM& VE \\ 
				\midrule
				\BM	&	0	&	0	&	6	&	0	&	0	&	43\\ 
				\BA	&	0	&	0	&	0	&	16	&	0	&	42\\
				\CD	&	8	&	5	&	0	&	10	&	28	&	0\\ 
				\LS	&	0	&	0	&	0	&	28	&	10	&	16\\
				\PV &	2	&	6	&	0	&	0	&	16	&	26\\
				\ZF 	&	0	&	4	&	0	&	10	&	16	&	17\\
				\ZE &	3	&	0	&	0	&	16	&	5	&	26\\ \hline
				\textbf{Totale}&	 13 &	15 &	6	&	80 &		75 & 170\\
				\bottomrule
			\end{tabular}
		\end{center}	
		\caption{Copertura dei ruoli dalla RP alla RQ.}
	\end{table}	
	
	\begin{figure}[h!]
	\centering
		\includegraphics[scale=0.6]{\docsImg RP_RQ_risorse.png}
		\caption{Calendario dettagliato di rotazione dei ruoli.}		
	\end{figure} 	
	}
	\newpage
	\subsubsection{Preventivo rotazione ruoli dalla RQ alla RA}{
		\begin{table}[h!]
		\begin{center}
			\begin{tabular}{l c c c c c c}				
				\toprule
				&	RE& AM& AN& PRT& PRM& VE \\ 
				\midrule
				\BM	&	6	&	0	&	0	&	0	&	11	&	0\\ 
				\BA	&	3	&	3	&	0	&	0	&	10	&	0\\
				\CD	&	0	&	0	&	0	&	0	&	0	&	14\\ 
				\LS	&	0	&	0	&	0	&	2	&	0	&	21\\
				\PV &	1	&	6	&	0	&	0	&	10	&	0\\
				\ZF 	&	0	&	5	&	0	&	6	&	0	&	9\\
				\ZE &	3	&	0	&	0	&	0	&	0	&	21\\ \hline
				\textbf{Totale}&	 13 &	14 &		0	&	8 &		31 & 65\\
				\bottomrule
			\end{tabular}
		\end{center}	
		\caption{Copertura dei ruoli dalla RQ alla RA.}
	\end{table}	
	
	\begin{figure}[h!]
	\centering
	\includegraphics[scale=0.7]{\docsImg RQ_RA_risorse.png}
		\caption{Calendario dettagliato di rotazione dei ruoli.}		
	\end{figure} 
		}
	}
%	\newpage\begin{flushright}
%	\end{flushright}
}

\subsubsection{Preventivo di distribuzione in percentuale dei carichi di lavoro}{
		\begin{table}[h!]
		\begin{center}
			\begin{tabular}{l c c c c c c}				
				\toprule
				&	RE& AM& AN& PRT& PRM& VE \\ 
				\midrule
				\BM	&	14,29	&	22,00	&	7,50 	&	16,28	&	10,38	&	15,09\\ 
				\BA	&	7,14 	&	6,00 	&	16,25	&	10,47	&	9,43 	&	20,35\\
				\CD	&	19,05	&	10,00	&	15,00	&	22,09	&	26,42	&	4,91\\ 
				\LS	&	14,29	&	10,00	&	16,25	&	17,44	&	9,43 	&	14,39\\
				\PV &	7,14 	&	24,00	&	15,00	&	15,12	&	24,53	&	9,12\\
				\ZF 	&	23,81	&	18,00	&	15,00	&	9,30 	&	15,09	&	14,74\\
				\ZE &	14,29	&	10,00	&	15,00	&	9,30 	&	4,72 	&	21,40\\
				\bottomrule
			\end{tabular}
		\end{center}	
		\caption{Preventivo di distribuzione dei ruoli tra i componenti dell’azienda.}
	\end{table}	
	
}
