\section{Scelta modello ciclo di vita}{
Per lo sviluppo del prodotto \textbf{\mytalk}, il gruppo \textit{\ggt~} ha escluso immediatamente il modello di ciclo di vita sequenziale in quanto richiede un elevato livello di esperienza, di cui gli studenti non sono in possesso, e non prevede il rilascio di prototipi durante il ciclo di vita,  cosa che aumenta di molto il rischio di fallimento.
Un’altra possibile scelta era il modello evolutivo, ma è stato scartato sia perché implica un elevato costo per il continuo passaggio attraverso le fasi del ciclo di vita, sia perché non riduce la distanza temporale per il completamento del ciclo di sviluppo.
L’unico modello che si presta alle nostre esigenze è il modello incrementale.
Non avendo esperienza riguardo allo sviluppo di prodotti di queste dimensioni, il gruppo ha adottato questo modello di ciclo di vita che porta alla realizzazione del prodotto per passi pianificati, assicurando convergenza entro tempi e costi previsti.

\begin{center}
\begin{figure}[h]
\centering
\includegraphics[scale=0.2]{\docsImg cicloInc.png}
\label{fig: cicloIncrementale}
\caption{Consegna incrementale}
\end{figure}
\end{center}

Le attività principali di sviluppo, analisi e progettazione architetturale ad alto livello sono svolte una sola volta: essendo stati identificati e fissati completamente i requisiti principali è possibile definire l’architettura di sistema. 
Le iterazioni garantiscono un'attività continuativa di controllo e di valutazione nella fase di realizzazione.
Questo modello stabilisce una base che soddisfa i requisiti più importanti sulla quale è possibile apportare delle aggiunte.
In dettaglio si è scelto il modello incrementale il quale prevede il rilascio di un prototipo che non soddisfi tutte le aspettative richieste dall'utente, ma solo successivamente gli verranno aggiunte le funzionalità che non include. 
Secondo quanto pianificato, conclusa la prima iterazione del ciclo, il gruppo rilascia un prototipo che soddisfa i requisiti obbligatori richiesti dal \textit{Committente}  (rilascio previsto in data 2013-02-11), mentre nella seconda iterazione andrà a implementare le funzionalità opzionali, in comune accordo con il \textit{Committente}.
}

