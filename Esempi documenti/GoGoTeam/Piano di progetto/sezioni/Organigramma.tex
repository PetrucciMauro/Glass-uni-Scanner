\section{Organigramma}{

	\subsection{Accettazione dei componenti}{
	\begin{table}[h]
	 \label{tabellaFirma}
		\begin{center}
		%\rowcolors{2}{light}{}
			%\begin{tabular}{p{0.30\textwidth} p{0.25\textwidth} p{0.35\textwidth}}	
			\begin{tabular}{l c l }				
				\toprule
				Nominativo			&	Data di accettazione	&	Firma \\ 
				\midrule
				\BM	&	2012-11-30	& \includegraphics[scale=0.040]{\docsImg mb.jpg} \\ 
				\BA	&	2012-11-30	& \includegraphics[scale=0.040]{\docsImg ab.jpg}\\
				\CD	& 	2012-11-30 	& \includegraphics[scale=0.040]{\docsImg dc.jpg}\\ 
				\LS	& 	2012-11-30 	& \includegraphics[scale=0.040]{\docsImg sl.jpg}\\
				\PV & 	2012-11-30 	& \includegraphics[scale=0.040]{\docsImg vp.jpg}\\
				\ZF &	2012-11-30 	& \includegraphics[scale=0.040]{\docsImg fz.jpg}\\
				\ZE &	2012-11-30 	& \includegraphics[scale=0.040]{\docsImg ez.jpg}\\
				\bottomrule
			\end{tabular}
		\end{center}

	\caption{Lista di accettazione}
		%\label{t1}
	\end{table}
	}
	\subsection{Componenti}{
	\begin{table}[h]
	 \label{tabellaComp}
		\begin{center}
		%\rowcolors{2}{light}{}
		%\begin{tabular}{p{0.30\textwidth} p{0.15\textwidth} p{0.45\textwidth}}	
		\begin{tabular}{l c l }
						
		\toprule
		Nominativo		& Matricola	& Indirizzo email\\ 
		\midrule
		\BM	& 614511	 & matteo.belletti@studenti.unipd.it\\
		\BA	& 580088	 & alessandro.bonaldo.1@studenti.unipd.it\\
		\CD	& 560717 & davide.ceccon@studenti.unipd.it\\ 
		\LS	& 596308 & sara.lazzaretto@studenti.unipd.it\\
		\PV & 561159 & valentina.pasqualotto@studenti.unipd.it\\
		\ZF	& 563399 & francesco.zattarin@studenti.unipd.it\\
		\ZE	& 593668 & elena.zerbato@studenti.unipd.it\\
		\bottomrule
		\end{tabular}
		\end{center}

	\caption{Descrizione componenti del gruppo}
		\end{table}
	}
	
%	\subsection{Attribuzione dei ruoli}{
%	Il gruppo conta sette studenti, considerando la durata del progetto che è di circa dieci settimane, ognuno di essi lavorerà al suo sviluppo per una media di due ore giornaliere (cinque giorni a settimana), per un totale di centocinque ore a persona.
%Considerando lo scopo didattico del progetto e dall'esplicita richiesta dal docente, ogni componente può ricoprire più ruoli, sia contemporaneamente che in fasi distinte del progetto, in ogni caso garantendo sempre assenza di conflitto di interessi tra i ruoli assunti. Nella distribuzione dei ruoli e l'assegnazione delle attività ai componenti sarà garantita un'equa ripartizione del carico di lavoro individuale.}
}

