\section{Processi di ingegneria dei requisiti}{
	\subsection{Fattibilità}{
		A partire da informazioni preliminari sul capitolato lo studio di fattibilità dovrà generare un rapporto che indichi la convenienza o meno del gruppo nello sviluppo del sistema. In particolare si dovrà considerare:
		\begin{enumerate}
			\item sufficienza di risorse umane
			\item rapporto tra i costi ed i benefici
			\item rischi individuati
		\end{enumerate}
		Nello stimare i benefici dovrà essere data molta importanza alle competenze che i membri del gruppo acquisirebbero nello sviluppo del sistema
	 }
	\subsection{Deduzione e analisi dei requisiti}{ 
		\subsubsection{Scoperta dei requisiti}{
			\textbf{problemi e fonti}\\\\
			L’analisi dei requisiti dovrà iniziare dall’identificazione dei bisogni. Questi dovranno essere ottenuti da:
			\begin{enumerate}
				\item \textbf{CAP}: capitolato d'appalto proposto
				\item \textbf{PRO}: minute degli incontri con il proponente
				\item \textbf{INT}: standard di qualità del software (ISO/IEC 9126:2001)
			\end{enumerate}
			I bisogni dovranno essere enumerati così da essere tracciabili con i requisiti specificati.
			La enumerazione dovrà considerare la provenienza usando i codici sopra descritti. La numerazione dei bisogni non sarà sequenziale per permettere maggiore flessibilità nella loro gestione.\\\\
			\textbf{interviste}\\\\
			Si dovranno evitare quando possibile interviste puramente aperte in cui non c’è nulla di predefinito. Si dovrà preparare un insieme di domande da porre al proponente in modo da guidare l’intervista o almeno cominciarla. Nelle interviste potrebbe essere utile discutere
			con il proponente dei casi d’uso analizzati durante la fase di analisi internamente al gruppo.
			Le richieste di interviste al proponente avverrà con le modalità descritte in ”comunicazioni esterne”. Durante ogni intervista dovrà essere scritta una minuta che sarà confermata dal proponente eventualmente con le opportune modifiche. La minuta sarà confermata al termine dell’incontro. Quando non fosse un problema per il proponente l’audio dell’intervista dovrà essere registrato per favorire la futura fase di analisi.\\\\
			\textbf{riunioni interne e casi d'uso}\\\\
			Individualmente e durante le riunioni interne gli analisti dovranno analizzare le informazioni raccolte dalle interviste con il proponente per individuare problemi e fonti da cui attingere i requisiti.\\
			L’individuazione dei requisiti funzionali sarà guidata dai casi d’uso. I casi d’uso potranno avere rappresentazione a diagrammi ma ogni caso d’uso dovrà avere anche la rappresentazione testuale. In particolare nella rappresentazione testuale si definirà:
			\begin{enumerate}
				\item identificativo
				\item attore primario
				\item precondizioni
				\item postcondizioni
				\item scenario principale
				\item estensioni
			\end{enumerate}
			Per la sintassi si rimanda a ”Dall’idea al codice con UML2.0, Luciano Baresi, Luigi Lavazza, Massimiliano Pianciamore”.
			}
			\subsubsection{Classificazone e priorità}{
				I requisiti dovranno essere classificati in:
				\begin{enumerate}
					\item requisiti di processo
					\item requisiti di prodotto
				\end{enumerate}
				I requisiti di prodotto saranno classificati in base a:
				\begin{enumerate}
					\item importanza
					\item provenienza
					\item tipologia
				\end{enumerate}
				Dove i gradi di importanza saranno:
				\begin{itemize}
						\item \textbf{OB}: requisito obbligatorio
						\item \textbf{DE}: requisito desiderabile
						\item \textbf{OP}: requisito opzionale
				\end{itemize}
				La provenienza può essere:
				\begin{itemize}
					\item \textbf{CAP}: da capitolato
					\item \textbf{INT}: da analisi interna
					\item \textbf{PRO}: da incontro con proponente
				\end{itemize}
				Mentre le tipologie saranno:
				\begin{itemize}
					\item \textbf{F}: requisito funzionale
					\item \textbf{P}: requisito prestazionale
					\item \textbf{Q}: requisito di qualità
					\item \textbf{V}: requisito di vincolo
				\end{itemize}
			}
			\subsubsection{Specifica}{
				Nella specifica dei requisiti dovrà essere considerato come riferimento lo standard IEEE 830-1998. In particolare saranno da perseguire le seguenti caratteristiche dei requisiti:
				\begin{enumerate}
					\item non ambigui
					\item corretti
					\item completi
					\item verificabili
					\item consistenti
					\item modificabili
					\item tracciabili
					\item ordinati per rilevanza
				\end{enumerate}
				I requisiti dovranno essere specificati in un documento ”Analisi dei requisiti” secondo la struttura definita nello standard IEEE 830-1998. La specifica dei requisiti dovrà essere documentata in forma tabellare per evitare ambiguità. Per ogni requisito dovranno essere definiti un codice, una descrizione, un riferimento alla fonte e un riferimento alla verifica. Al fine di rendere meno ambigui i requisiti sara redatto un ”glossario” contenente la definizione di tutti i termini non ovvi usati in fase di analisi.
			}
			\subsubsection{Verifica dei requisiti}{
				Per ogni requisito di processo specificato dovrà essere presente in ”Piano di qualifica” un riferimento alle sezioni di ”Norme di progetto” in cui viene assicurato il soddisfacimento del requisito. Per ogni requisito di prodotto specificato dovrà essere descritto brevemente il metodo che verrà usato per verificarne il soddisfacimento.\\Per favorire la tracciabilit`a tra requisiti e metodi di verifica dovr`a essere presente in ”Piano di qualifica” una tabella in cui si definiscono: codice di requisito, codice di verifica e modalità di verifica. Se il requisito è di processo, modalità di verifica conterrà i riferimenti alle sezioni corrispondenti in ”Norme di progetto”.
			}
		}
		\subsection{Validazione dei requisiti}{
			\subsubsection{Interna}{
				Dovranno essere verificate la correttezza e la completezza dei requisiti rispetto ai bisogni. Questo dovrà essere fatto tramite tracciamento tra specifica dei requisiti e bisogni individuati.\\Dovranno essere verificate la correttezza e la completezza dei metodi di verifica dei requisiti
				rispetto ai requisiti. Questo dovrà essere fatto tramite tracciamento tra specifica dei requisiti e metodi di verifica.
			}
				\subsubsection{Esterna}{
					Terminata la validazione interna verranno presentati al proponente i documenti ”Analisi dei requisiti” e ”Piano di qualifica”, se accettati costituiranno una baseline per la fase successiva del progetto altrimenti altrimenti verranno gestite le richieste di modifica secondo i metodi descritti in ”Gestione dei cambiamenti”.
				}
		}
		\subsection{Gestione delle modifiche ai requisiti}{
			A tutte le proposte di modifica dei requisiti dovrà essere applicata la seguente procedura:
			\begin{enumerate}
				\item deduzione, analisi e specifica dei cambiamenti
				\item stima dei costi del cambiamento considerando quante modifiche dovranno essere fatte ai requisiti e al progetto del sistema
				\item decisione ed eventuale implementazione del cambiamento nei requisiti e nel progetto di sistema
			\end{enumerate}
			Per gestire i cambiamenti e per facilitare il tracciamento dei requisiti verr`a usato il software iConcur Axiom 4.1. L’amministratore avr`a il compito di gestire il server e amministrare i diritti di accesso degli utenti alle funzionalità fornite. In particolare gli analisti dovranno usare i modelli definiti all’inizio della fase di analisi. Per evitare problemi dovuti a modifiche concorrenti alla base dati l’amministratore dovrà garantire che ad ogni istante solo un analista possa modificare un certo sottoalbero della foresta dei requisiti e dei test.
			}