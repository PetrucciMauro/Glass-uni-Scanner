\subsection{Classe Command}{
	\subsubsection{Classe Invoker}{
	\textbf{Funzione}\\
		\indent Classe, componente invoker del Design Pattern Command. Implementato tramite design pattern singleton.\\
	\textbf{Scope}\\
		\indent Model::SlideShow::SlideShowActions::Command::Invoker.\\
	\textbf{Utilizzo}\\
		\indent Viene utilizzata per eseguire i comandi e memorizzarli all’interno di stack dedicate all’implementazione delle funzionalità di annulla e ripristina.\\
	\textbf{Attributi}
	\begin{itemize}
		\item \textbf{undoStack}
		\begin{itemize}
			\item \textbf{Accesso}: Private;
			\item \textbf{Tipo}: Array;
			\item \textbf{Descrizione}: contiene l’elenco dei comandi eseguiti e annullabili.
		\end{itemize}
		\item \textbf{redoStack}
		\begin{itemize}
			\item \textbf{Accesso}: Private;
			\item \textbf{Tipo}: Array;
			\item \textbf{Descrizione}: contiene l’elenco dei comandi annullati e ripristinabili.
		\end{itemize}
	\end{itemize}
	\noindent{\textbf{Metodi}}
	\begin{itemize}
		\item \textbf{execute}(AbstractCommand)
		\begin{itemize}
			\item \textbf{Accesso}: Public;
			\item \textbf{Tipo di ritorno}: Void;
			\item \textbf{Descrizione}: invoca il metodo doAction() del comando ricevuto e invoca
			undoStack.push(AbstractCommand) e redoStack.clear().
		\end{itemize}
		\item \textbf{undo}()
		\begin{itemize}
			\item \textbf{Accesso}: Public;
			\item \textbf{Tipo di ritorno}: Void;
			\item \textbf{Descrizione}: invoca il metodo undoAction() dell’ultimo comando in undoStack() e invoca command=undoStack.pop() e redoStack.push(command).
		\end{itemize}
		\item \textbf{redo}()
		\begin{itemize}
			\item \textbf{Accesso}: Public;
			\item \textbf{Tipo di ritorno}: Void;
			\item \textbf{Descrizione}: invoca il metodo doAction() dell’ultimo comando in redoStack() e invoca command=redoStack.pop() e undoStack.push(command).
		\end{itemize}
		\item \textbf{getUndoStack}()
		\begin{itemize}
			\item \textbf{Accesso}: Public;
			\item \textbf{Tipo di ritorno}: Boolean;
			\item \textbf{Descrizione}: ritorna true se undoStack non è vuoto, false altrimenti.
		\end{itemize}
		\item \textbf{getRedoStack}()
		\begin{itemize}
			\item \textbf{Accesso}: Public;
			\item \textbf{Tipo di ritorno}: Boolean;
			\item \textbf{Descrizione}: ritorna true se redoStack non è vuoto, false altrimenti.
		\end{itemize}
	\end{itemize}
	
	\subsubsection{Classe AbstractCommand}{
		\textbf{Funzione}\\
			\indent Classe concreta, i suoi elementi rappresentano un oggetto di tipo testo.\\
	   	\textbf{Scope}\\
			\indent Model::SlideShow::SlideShowActions::Command::AbstractCommand.\\
		\textbf{Utilizzo}\\
			\indent È classe base per i comandi di modifica, inserimento ed eliminazione degli elementi della presentazione.\\
		\textbf{Attributi}
		\begin{itemize}
			\item \textbf{id}
			\begin{itemize}
				\item \textbf{Accesso}: Private;
				\item \textbf{Tipo}: Integer;
				\item \textbf{Descrizione}: indica il codice identificativo dell’oggetto su cui viene eseguito il comando.
			\end{itemize}
			\item \textbf{executed}
			\begin{itemize}
				\item \textbf{Accesso}: Private;
				\item \textbf{Tipo}: Boolean;
				\item \textbf{Descrizione}: è settata a false di default, indica se il comando è stato eseguito.
			\end{itemize}
			\item \textbf{enabler}
			\begin{itemize}
				\item \textbf{Accesso}: Private;
				\item \textbf{Tipo}: Object;
				\item \textbf{Descrizione}: corrisponde all'oggetto InsertEditRemove.
			\end{itemize}
			\item \textbf{obj}
			\begin{itemize}
				\item \textbf{Accesso}: Private;
				\item \textbf{Tipo}: Object;
				\item \textbf{Descrizione}: mantiene salvati id e tipo dell'elemento su cui abstractCommand è stato richiamato.
			\end{itemize}
		\end{itemize}
		
		\noindent{\textbf{Metodi}}
		\begin{itemize}
			\item \textbf{AbstractCommand}(spec:object)
			\begin{itemize}
				\item \textbf{Accesso}: Public;
				\item \textbf{Tipo di ritorno}: Void;
				\item \textbf{Descrizione}: costruisce l’oggetto abstractCommand e setta:\\
				\begin{itemize}
				\item id=spec.id;
				\item obj.type=spec.type;
				\item obj.id=spec.id;
				\item executed=0;
				\item enabler=insertEditRemove();
				\end{itemize}
			\end{itemize}
			\item \textbf{getObj}()
			\begin{itemize}
				\item \textbf{Accesso}: Public;
				\item \textbf{Tipo di ritorno}: Object;
				\item \textbf{Descrizione}: metodo che ritorna obj.
			\end{itemize}
			\item \textbf{setId}(id)
			\begin{itemize}
				\item \textbf{Accesso}: Public;
				\item \textbf{Tipo di ritorno}: Void;
				\item \textbf{Descrizione}: metodo che cambia l'id con il nuovo id passato per parametro.
			\end{itemize}
			\item \textbf{setExecuted}()
			\begin{itemize}
				\item \textbf{Accesso}: Public;
				\item \textbf{Tipo di ritorno}: Void;
				\item \textbf{Descrizione}: metodo che setta executed a true.
			\end{itemize}
			\item \textbf{getExecuted}()
			\begin{itemize}
				\item \textbf{Accesso}: Public;
				\item \textbf{Tipo di ritorno}: Void;
				\item \textbf{Descrizione}: metodo che ritorna executed.
			\end{itemize}
			\item \textbf{getId}()
			\begin{itemize}
				\item \textbf{Accesso}: Public;
				\item \textbf{Tipo di ritorno}: Integer;
				\item \textbf{Descrizione}: metodo che ritorna l'id dell'elemento.
			\end{itemize}
			\item \textbf{getPosition}()
			\begin{itemize}
				\item \textbf{Accesso}: Public;
				\item \textbf{Tipo di ritorno}: Object;
				\item \textbf{Descrizione}: metodo che ritorna le posizioni xIndex e yIndex dell'elemento.
			\end{itemize}
			\item \textbf{getRotation}()
			\begin{itemize}
				\item \textbf{Accesso}: Public;
				\item \textbf{Tipo di ritorno}: Object;
				\item \textbf{Descrizione}: metodo che ritorna la rotazione dell'elemento.
			\end{itemize}
			\item \textbf{getSize}()
			\begin{itemize}
				\item \textbf{Accesso}: Public;
				\item \textbf{Tipo di ritorno}: Object;
				\item \textbf{Descrizione}: metodo che ritorna le dimensioni dell'elemento.
			\end{itemize}
			\item \textbf{getEnabler}()
			\begin{itemize}
				\item \textbf{Accesso}: Public;
				\item \textbf{Tipo di ritorno}: Object;
				\item \textbf{Descrizione}: metodo che ritorna enabler.
			\end{itemize}
		\end{itemize}
		\noindent{\textbf{Ereditata da}:}
		\begin{itemize}
			\item ConcreteTextInsertCommand (\S\ref{TextInsertCommand});
			\item ConcreteFrameInsertCommand (\S\ref{FrameInsertCommand});
			\item ConcreteImageInsertCommand (\S\ref{ImageInsertCommand});
			\item ConcreteSVGInsertCommand (\S\ref{SVGInsertCommand});
			\item ConcreteAudioInsertCommand (\S\ref{AudioInsertCommand});
			\item ConcreteVideoInsertCommand (\S\ref{VideoInsertCommand});
			\item ConcreteBackgroundInsertCommand (\S\ref{BackgroundInsertCommand});
			\item ConcreteTextRemoveCommand (\S\ref{TextRemoveCommand});
			\item ConcreteFrameRemoveCommand (\S\ref{FrameRemoveCommand});
			\item ConcreteImageRemoveCommand (\S\ref{ImageRemoveCommand});
			\item ConcreteSVGRemoveCommand (\S\ref{SVGRemoveCommand});
			\item ConcreteAudioRemoveCommand (\S\ref{AudioRemoveCommand});
			\item ConcreteVideoRemoveCommand (\S\ref{VideoRemoveCommand});
			\item ConcreteEditSizeCommand (\S\ref{EditSizeCommand});
			\item ConcreteEditPositionCommand (\S\ref{EditPositionCommand});
			\item ConcreteEditRotationCommand (\S\ref{EditRotationCommand});
			\item ConcreteEditColorCommand (\S\ref{EditColorCommand});
			\item ConcreteEditBackgroundCommand (\S\ref{EditBackgroundCommand});
			\item ConcreteEditFontCommand (\S\ref{EditFontCommand});
			\item ConcreteEditBookmarkCommand (\S\ref{EditBookmarkCommand});
			\item ConcretePortaAvantiCommand (\S\ref{PortaAvantiCommand});
			\item ConcretePortaDietroCommand (\S\ref{PortaDietroCommand});
			\item ConcreteAddToMainPathCommand (\S\ref{AddToMainPathCommand});
			\item ConcreteRemoveFromMainPathCommand (\S\ref{RemoveFromMainPathCommand});
			\item ConcreteNewChoicePathCommand (\S\ref{NewChoicePathCommand});
			\item ConcreteDeleteChoicePathCommand (\S\ref{DeleteChoicePathCommand});
			\item ConcreteAddToChoicePathCommand (\S\ref{AddToChoicePathCommand});
			\item ConcreteRemoveFromChoicePathCommand (\S\ref{RemoveFromChoicePathCommand}).
		\end{itemize}
	}
\subsubsubsection{Classe ConcreteTextInsertCommand}{
	\label{TextInsertCommand}
	\textbf{Funzione}\\
		\indent Classe concreta, è interfaccia del Design Pattern Command.\\
   	\textbf{Scope}\\
		\indent Model::SlideShow::SlideShowActions::Command::AbstractCommand::\-ConcreteTextInsertCommand.\\
	\textbf{Utilizzo}\\
		\indent Viene costruito da Premi::Controller::EditController, riceve le coordinate di inserimento di un testo nella presentazione e invoca il metodo di Inserter insertText() passandogliele.\\
	\noindent{\textbf{Metodi}}
	\begin{itemize}
		\item \textbf{ConcreteTextInsertCommand}(spec: object)
		\begin{itemize}
			\item \textbf{Accesso}: Public;
			\item \textbf{Tipo di ritorno}: Void;
			\item \textbf{Descrizione}: costruisce l’oggetto ConcreteTextInsertCommand.
		\end{itemize}
		\item \textbf{doAction}()
		\begin{itemize}
			\item \textbf{Accesso}: Public;
			\item \textbf{Tipo}: Void;
			\item \textbf{Descrizione}: invoca il metodo di Inserter insertText(spec). Se executed è settato a false lo setta come true, altrimenti invoca il metodo inserisciTesto(spec) di EditController.
		\end{itemize}
		\item \textbf{undoAction}()
		\begin{itemize}
			\item \textbf{Accesso}: Public;
			\item \textbf{Tipo}: Void;
			\item \textbf{Descrizione}: invoca il metodo di Editor removeText(id) passando come parametro il campo id. Invoca il metodo rimuoviElemento(spec) di EditController.
		\end{itemize}
	\end{itemize}
	}
	
\subsubsubsection{Classe ConcreteFrameInsertCommand}{
	\label{FrameInsertCommand}
	\textbf{Funzione}\\
		\indent Classe concreta, è interfaccia del Design Pattern Command.\\
   	\textbf{Scope}\\
		\indent Model::SlideShow::SlideShowActions::Command::AbstractCommand::\-ConcreteFrameInsertCommand.\\
	\textbf{Utilizzo}\\
		\indent Viene costruito da Premi::Controller::EditController, riceve le coordinate di inserimento di un testo nella presentazione e invoca il metodo di Inserter insertFrame() passandogliele.\\
	
	\noindent{\textbf{Metodi}}
	\begin{itemize}
		\item \textbf{ConcreteFrameInsertCommand}(spec: object)
		\begin{itemize}
			\item \textbf{Accesso}: Public;
			\item \textbf{Tipo di ritorno}: Void;
			\item \textbf{Descrizione}: costruisce l’oggetto ConcreteFrameInsertCommand.
		\end{itemize}
		\item \textbf{doAction}()
		\begin{itemize}
			\item \textbf{Accesso}: Public;
			\item \textbf{Tipo di ritorno}: Void;
			\item \textbf{Descrizione}: invoca il metodo di Inserter insertFrame(spec). Se executed è settato a false lo setta come true, altrimenti invoca il metodo inserisciFrame(spec) di EditController.
		\end{itemize}
		\item \textbf{undoAction}()
		\begin{itemize}
			\item \textbf{Accesso}: Public;
			\item \textbf{Tipo di ritorno}: Void;
			\item \textbf{Descrizione}: invoca il metodo di Editor removeFrame(id) passando come parametro il campo id. Invoca il metodo rimuoviElemento(spec) di EditController.
		\end{itemize}
	\end{itemize}
	}

\subsubsubsection{Classe ConcreteImageInsertCommand}{
	\label{ImageInsertCommand}
	\textbf{Funzione}\\
		\indent Classe concreta, è interfaccia del Design Pattern Command.\\
   	\textbf{Scope}\\
		\indent Model::SlideShow::SlideShowActions::Command::AbstractCommand::\-ConcreteImageInsertCommand.\\
	\textbf{Utilizzo}\\
		\indent Viene costruito da Premi::Controller::EditController, riceve le coordinate di inserimento di un’immagine nella presentazione e invoca il metodo di Inserter insertImage() passandogliele.\\
	
	\noindent{\textbf{Metodi}}
	\begin{itemize}
		\item \textbf{ConcreteImageInsertCommand}(spec: object)
		\begin{itemize}
			\item \textbf{Accesso}: Public;
			\item \textbf{Tipo di ritorno}: Void;
			\item \textbf{Descrizione}: costruisce l’oggetto ConcreteImageInsertCommand.
		\end{itemize}
		\item \textbf{doAction}()
		\begin{itemize}
			\item \textbf{Accesso}: Public;
			\item \textbf{Tipo di ritorno}: Void;
			\item \textbf{Descrizione}: invoca il metodo di Inserter insertImage(spec). Se executed è settato a false lo setta come true, altrimenti invoca il metodo inserisciImmagine() di EditController passandogli spec.
		\end{itemize}
		\item \textbf{undoAction}()
		\begin{itemize}
			\item \textbf{Accesso}: Public;
			\item \textbf{Tipo di ritorno}: Void;
			\item \textbf{Descrizione}: invoca il metodo di Editor removeImage(id) passando come parametro il campo id. Invoca il metodo rimuoviElemento(spec) di EditController.
		\end{itemize}
	\end{itemize}
	}

\subsubsubsection{Classe ConcreteSVGInsertCommand}{
	\label{SVGInsertCommand}
	\textbf{Funzione}\\
		\indent Classe concreta, è interfaccia del Design Pattern Command.\\
   	\textbf{Scope}\\
		\indent Model::SlideShow::SlideShowActions::Command::AbstractCommand::\-ConcreteSVGInsertCommand.\\
	\textbf{Utilizzo}\\
		\indent Viene costruito da Premi::Controller::EditController, riceve le coordinate di inserimento di un SVG nella presentazione e invoca il metodo di Inserter insertSVG() passandogliele.\\
	
	\noindent{\textbf{Metodi}}
	\begin{itemize}
		\item \textbf{ConcreteSVGInsertCommand}(spec: object)
		\begin{itemize}
			\item \textbf{Accesso}: Public;
			\item \textbf{Tipo di ritorno}: Void;
			\item \textbf{Descrizione}: costruisce l’oggetto ConcreteSVGInsertCommand.
		\end{itemize}
		\item \textbf{doAction}()
		\begin{itemize}
			\item \textbf{Accesso}: Public;
			\item \textbf{Tipo di ritorno}: Void;
			\item \textbf{Descrizione}: invoca il metodo di Inserter insertSVG(spec). Se executed è settato a false lo setta come true, altrimenti invoca il metodo update(id) di EditController.
		\end{itemize}
		\item \textbf{undoAction}()
		\begin{itemize}
			\item \textbf{Accesso}: Public;
			\item \textbf{Tipo di ritorno}: Void;
			\item \textbf{Descrizione}: invoca il metodo di Editor removeSVG(id) passando come parametro il campo id. Invoca il metodo remove(id) di EditController.
		\end{itemize}
	\end{itemize}
	}

\subsubsubsection{Classe ConcreteAudioInsertCommand}{
	\label{AudioInsertCommand}
	\textbf{Funzione}\\
		\indent Classe concreta, è interfaccia del Design Pattern Command.\\
   	\textbf{Scope}\\
		\indent Model::SlideShow::SlideShowActions::Command::AbstractCommand::\-ConcreteAudioInsertCommand.\\
	\textbf{Utilizzo}\\
		\indent Viene costruito da Premi::Controller::EditController, riceve le coordinate di inserimento di un audio nella presentazione e invoca il metodo di Inserter insertAudio() passandogliele.\\	
	\noindent{\textbf{Metodi}}
	\begin{itemize}
		\item \textbf{ConcreteAudioInsertCommand}(spec: object)
		\begin{itemize}
			\item \textbf{Accesso}: Public;
			\item \textbf{Tipo di ritorno}: Void;
			\item \textbf{Descrizione}: costruisce l’oggetto ConcreteAudioInsertCommand.
		\end{itemize}
		\item \textbf{doAction}()
		\begin{itemize}
			\item \textbf{Accesso}: Public;
			\item \textbf{Tipo di ritorno}: Void;
			\item \textbf{Descrizione}: invoca il metodo di Inserter insertAudio(spec). Se executed è settato a false lo setta come true, altrimenti invoca il metodo inserisciAudio() di EditController passandogli spec.
		\end{itemize}
		\item \textbf{undoAction}()
		\begin{itemize}
			\item \textbf{Accesso}: Public;
			\item \textbf{Tipo di ritorno}: Void;
			\item \textbf{Descrizione}: invoca il metodo di Editor removeAudio(id) passando come parametro il campo id. Invoca il metodo rimuoviElemento(spec) di EditController.
		\end{itemize}
	\end{itemize}
	}

\subsubsubsection{Classe ConcreteVideoInsertCommand}{
	\label{VideoInsertCommand}
	\textbf{Funzione}\\
		\indent Classe concreta, è interfaccia del Design Pattern Command.\\
   	\textbf{Scope}\\
		\indent Model::SlideShow::SlideShowActions::Command::AbstractCommand::\-ConcreteVideoInsertCommand.\\
	\textbf{Utilizzo}\\
		\indent Viene costruito da Premi::Controller::EditController, riceve le coordinate di inserimento di un video nella presentazione e invoca il metodo di Inserter insertVideo() passandogliele..\\
	
	\noindent{\textbf{Metodi}}
	\begin{itemize}
		\item \textbf{ConcreteVideoInsertCommand}(spec:object)
		\begin{itemize}
			\item \textbf{Accesso}: Public;
			\item \textbf{Tipo di ritorno}: Void;
			\item \textbf{Descrizione}: costruisce l’oggetto ConcreteVideoInsertCommand e setta xIndex, yIndex, rotation, ref.
		\end{itemize}
		\item \textbf{doAction}()
		\begin{itemize}
			\item \textbf{Accesso}: Public;
			\item \textbf{Tipo di ritorno}: Void;
			\item \textbf{Descrizione}: invoca il metodo di Inserter insertVideo(spec). Se executed è settato a false lo setta come true, altrimenti invoca il metodo inserisciVideo() di EditController passandogli spec.
		\end{itemize}
		\item \textbf{undoAction}()
		\begin{itemize}
			\item \textbf{Accesso}: Public;
			\item \textbf{Tipo di ritorno}: Void;
			\item \textbf{Descrizione}: invoca il metodo di Editor removeVideo(id) passando come parametro il campo id. Invoca il metodo rimuoviElemento(spec) di EditController.
		\end{itemize}
	\end{itemize}
	}

\subsubsubsection{Classe ConcreteBackgroundInsertCommand}{
	\label{BackgroundInsertCommand}
	\textbf{Funzione}\\
		\indent Classe concreta, è interfaccia del Design Pattern Command.\\
   	\textbf{Scope}\\
		\indent Model::SlideShow::SlideShowActions::Command::AbstractCommand::\-ConcreteBackgroundInsertCommand.\\
	\textbf{Utilizzo}\\
		\indent Viene costruito da Premi::Controller::EditController, riceve i parametri di inserimento di uno sfondo nella presentazione e invoca il metodo di Inserter insertBackground() passandogliele.\\
	\textbf{Attributi}
	\begin{itemize}
		\item \textbf{ref}
		\begin{itemize}
			\item \textbf{Accesso}: Private;
			\item \textbf{Tipo}: String;
			\item \textbf{Descrizione}: rappresenta l’url dell’oggetto inserito.
		\end{itemize}
		\item \textbf{color}
		\begin{itemize}
			\item \textbf{Accesso}: Private;
			\item \textbf{Tipo}: String;
			\item \textbf{Descrizione}: rappresenta il colore dello sfondo inserito.
		\end{itemize}
		\item \textbf{oldBackground}
		\begin{itemize}
			\item \textbf{Accesso}: Private;
			\item \textbf{Tipo}: SlideShowElements::Background;
			\item \textbf{Descrizione}: rappresenta il vecchio sfondo della presentazione.
		\end{itemize}
	\end{itemize}
	
	\noindent{\textbf{Metodi}}
	\begin{itemize}
		\item \textbf{ConcreteBackgroundInsertCommand}(ref: string, color:string)
		\begin{itemize}
			\item \textbf{Accesso}: Public;
			\item \textbf{Tipo di ritorno}: Void;
			\item \textbf{Descrizione}: costruisce l’oggetto ConcreteVideoInsertCommand e setta ref e color.
		\end{itemize}
		\item \textbf{doAction}()
		\begin{itemize}
			\item \textbf{Accesso}: Public;
			\item \textbf{Tipo di ritorno}: Void;
			\item \textbf{Descrizione}: invoca il metodo di Inserter insertBackground (ref, color) passando come parametri i  campi dati ref e color. insertBackground restituisce un int che rappresenta l’id dell’oggetto o eventualmente un oggetto di tipo Background a cui ConcreteBackgroundInsertCommand istanzia oldBackground e al cui id inizializza il campo id. Se executed è settato a false lo setta come true, altrimenti invoca il metodo updateSfondo(spec) di EditController.
		\end{itemize}
		\item \textbf{undoAction}()
		\begin{itemize}
			\item \textbf{Accesso}: Public;
			\item \textbf{Tipo di ritorno}: Void;
			\item \textbf{Descrizione}: viene invocato il metodo InsertEditRemove::insertBackground(SlideShowElement::Background) e il metodo updateSfondo(oldBackground) di EditController.
		\end{itemize}
	\end{itemize}
	}

\subsubsubsection{Classe ConcreteTextRemoveCommand}{
	\label{TextRemoveCommand}
	\textbf{Funzione}\\
		\indent Classe concreta, è interfaccia del Design Pattern Command.\\
   	\textbf{Scope}\\
		\indent Model::SlideShow::SlideShowActions::Command::AbstractCommand::\-ConcreteTextRemoveCommand.\\
	\textbf{Utilizzo}\\
		\indent Viene costruito da Premi::Controller::EditController, riceve l’id dell’elemento testo da rimuovere dalla presentazione e invoca il metodo di Remover removeText(id).\\
	\textbf{Attributi}
	\begin{itemize}
		\item \textbf{oldText}
		\begin{itemize}
			\item \textbf{Accesso}: Private;
			\item \textbf{Tipo}: SlideShowElements::Text;
			\item \textbf{Descrizione}: è una copia dell’elemento testo rimosso. 
		\end{itemize}
	\end{itemize}
	
	\noindent{\textbf{Metodi}}
	\begin{itemize}
		\item \textbf{ConcreteTextRemoveCommand}(spec: object)
		\begin{itemize}
			\item \textbf{Accesso}: Public;
			\item \textbf{Tipo di ritorno}: Void;
			\item \textbf{Descrizione}: costruisce l’oggetto ConcreteTextRemoveCommand e setta l’attributo id.
		\end{itemize}
		\item \textbf{doAction}()
		\begin{itemize}
			\item \textbf{Accesso}: Public;
			\item \textbf{Tipo di ritorno}: Void;
			\item \textbf{Descrizione}: invoca il metodo InsertEditRemove::removeText(id) passando come parametro l’id dell’elemento. removeText restituisce un oggetto di tipo SlideShowElements::Text a cui viene inizializzato oldText. Se executed è settato a false lo setta come true, altrimenti invoca il metodo rimuoviElemento(spec) di EditController.
		\end{itemize}
		\item \textbf{undoAction}()
		\begin{itemize}
			\item \textbf{Accesso}: Public;
			\item \textbf{Tipo di ritorno}: Void;
			\item \textbf{Descrizione}: invoca il metodo InsertEditRemove::insertText(Text) passando come parametro text. Invoca il metodo inserisciTesto(oldText) di EditController.
		\end{itemize}
	\end{itemize}
	}

\subsubsubsection{Classe ConcreteFrameRemoveCommand}{
	\label{FrameRemoveCommand}
	\textbf{Funzione}\\
		\indent Classe concreta, è interfaccia del Design Pattern Command.\\
   	\textbf{Scope}\\
		\indent Model::SlideShow::SlideShowActions::Command::AbstractCommand::\-ConcreteFrameRemoveCommand.\\
	\textbf{Utilizzo}\\
		\indent Viene costruito da Premi::Controller::EditController, riceve l’id del frame da rimuovere dalla presentazione e invoca il metodo InsertEditRemove::removeFrame(id).\\
	\textbf{Attributi}
	\begin{itemize}
		\item \textbf{oldFrame}
		\begin{itemize}
			\item \textbf{Accesso}: Private;
			\item \textbf{Tipo}: SlideShowElements::Frame;
			\item \textbf{Descrizione}: è una copia dell’oggetto rimosso.
		\end{itemize}
	\end{itemize}
	
	\noindent{\textbf{Metodi}}
	\begin{itemize}
		\item \textbf{ConcreteFrameRemoveCommand}(spec: object)
		\begin{itemize}
			\item \textbf{Accesso}: Public;
			\item \textbf{Tipo di ritorno}: Void;
			\item \textbf{Descrizione}: costruisce l’oggetto ConcreteFrameRemoveCommand e setta il campo dati id.
		\end{itemize}
		\item \textbf{doAction}()
		\begin{itemize}
			\item \textbf{Accesso}: Public;
			\item \textbf{Tipo di ritorno}: Void;
			\item \textbf{Descrizione}: invoca il metodo InsertEditRemove::removeFrame(id) passando come parametro il  campo dati id. removeFrame restituisce una copia dell’oggetto rimosso che verrà settata come campo dati oldFrame. Se executed è settato a false lo setta come true, altrimenti invoca il metodo rimuoviElemento(spec) di EditController.
		\end{itemize}
		\item \textbf{undoAction}()
		\begin{itemize}
			\item \textbf{Accesso}: Public;
			\item \textbf{Tipo di ritorno}: Void;
			\item \textbf{Descrizione}: invoca il metodo InsertEditRemove::insertFrame(Frame) passando come parametro il campo oldFrame. Invoca il metodo inserisciFrame(oldFrame) di EditController.
		\end{itemize}
	\end{itemize}
	}

\subsubsubsection{Classe ConcreteImageRemoveCommand}{
	\label{ImageRemoveCommand}
	\textbf{Funzione}\\
		\indent Classe concreta, è interfaccia del Design Pattern Command\\
   	\textbf{Scope}\\
		\indent Model::SlideShow::SlideShowActions::Command::AbstractCommand::\-ConcreteImageRemoveCommand.\\
	\textbf{Utilizzo}\\
		\indent Viene costruito da Premi::Controller::EditController, riceve l’id dell’immagine da rimuovere dalla presentazione e invoca il metodo InsertEditRemove::removeImage(id).\\
	\textbf{Attributi}
	\begin{itemize}
		\item \textbf{oldImage}
		\begin{itemize}
			\item \textbf{Accesso}: Private;
			\item \textbf{Tipo}: SlideShowElements::Image;
			\item \textbf{Descrizione}: è una copia dell’oggetto rimosso
		\end{itemize}
	\end{itemize}
	
	\noindent{\textbf{Metodi}}
	\begin{itemize}
		\item \textbf{ConcreteImageRemoveCommand}(spec: object)
		\begin{itemize}
			\item \textbf{Accesso}: Public;
			\item \textbf{Tipo di ritorno}: Void;
			\item \textbf{Descrizione}: costruisce l’oggetto ConcreteImageRemoveCommand e setta il campo id.
		\end{itemize}
		\item \textbf{doAction}()
		\begin{itemize}
			\item \textbf{Accesso}: Public;
			\item \textbf{Tipo di ritorno}: Void;
			\item \textbf{Descrizione}: invoca il metodo InsertEditRemove::removeImage(id) passando come parametro l’id dell’oggetto da rimuovere. removeImage restituisce un oggetto di tipo SlideShowElements::Image cui sarà settato il campo oldImage. Se executed è settato a false lo setta come true, altrimenti invoca il metodo rimuoviElemento(spec) di EditController.
		\end{itemize}
		\item \textbf{undoAction}()
		\begin{itemize}
			\item \textbf{Accesso}: Public;
			\item \textbf{Tipo di ritorno}: Void;
			\item \textbf{Descrizione}: invoca il metodo InsertEditRemove::insertImage(Image) passando come parametro il campo oldImage. Invoca il metodo inserisciImmagine() di EditController passandogli oldImage.
		\end{itemize}
	\end{itemize}
	}

\subsubsubsection{Classe ConcreteSVGRemoveCommand}{
	\label{SVGRemoveCommand}
	\textbf{Funzione}\\
		\indent Classe concreta, è interfaccia del Design Pattern Command.\\
   	\textbf{Scope}\\
		\indent Model::SlideShow::SlideShowActions::Command::AbstractCommand::\-ConcreteSVGRemoveCommand.\\
	\textbf{Utilizzo}\\
		\indent Viene costruito da Premi::Controller::EditController, riceve l’id dell’elemento SVG da rimuovere dalla presentazione e invoca il metodo InsertEditRemove::removeSVG().\\
	\textbf{Attributi}
	\begin{itemize}
		\item \textbf{oldSVG}
		\begin{itemize}
			\item \textbf{Accesso}: Private;
			\item \textbf{Tipo}: SlideShowElements::SVG;
			\item \textbf{Descrizione}: è una copia dell’oggetto che rappresenta l’elemento rimosso.
		\end{itemize}
	\end{itemize}
	
	\noindent{\textbf{Metodi}}
	\begin{itemize}
		\item \textbf{ConcreteSVGRemoveCommand}(spec:object)
		\begin{itemize}
			\item \textbf{Accesso}: Public;
			\item \textbf{Tipo di ritorno}: Void;
			\item \textbf{Descrizione}: costruisce l’oggetto ConcreteSVGRemoveCommand e setta il campo dati id.
		\end{itemize}
		\item \textbf{doAction}()
		\begin{itemize}
			\item \textbf{Accesso}: Public;
			\item \textbf{Tipo di ritorno}: Void;
			\item \textbf{Descrizione}: invoca il metodo InsertEditRemove::removeSVG(id) passando come parametro il campo dati id.	removeSVG restituisce un elemento di tipo SlideShowElements::SVG che rappresenta l’elemento rimosso e a cui viene inizializzato il campo oldSVG. Se executed è settato a false lo setta come true, altrimenti invoca il metodo update(id) di EditController.
		\end{itemize}
		\item \textbf{undoAction}()
		\begin{itemize}
			\item \textbf{Accesso}: Public;
			\item \textbf{Tipo di ritorno}: Void;
			\item \textbf{Descrizione}: invoca il metodo di InsertEditRemove::insertSVG(SVG) passando come parametro il campo oldSVG. Invoca il metodo update(id) di EditController.
		\end{itemize}
	\end{itemize}
	}

\subsubsubsection{Classe ConcreteAudioRemoveCommand}{
	\label{AudioRemoveCommand}
	\textbf{Funzione}\\
		\indent Classe concreta, è interfaccia del Design Pattern Command.\\
   	\textbf{Scope}\\
		\indent Model::SlideShow::SlideShowActions::Command::AbstractCommand::\-ConcreteAudioRemoveCommand.\\
	\textbf{Utilizzo}\\
		\indent Viene costruito da Premi::Controller::EditController, riceve l’id dell’elemento audio da rimuovere dalla presentazione e invoca il metodo InsertEditRemove::removeAudio(id).\\
	\textbf{Attributi}
	\begin{itemize}
		\item \textbf{oldAudio}
		\begin{itemize}
			\item \textbf{Accesso}: Private;
			\item \textbf{Tipo}: SlideShowElements::Audio;
			\item \textbf{Descrizione}: è una copia dell’oggetto che rappresenta l’elemento rimosso.
		\end{itemize}
	\end{itemize}
	
	\noindent{\textbf{Metodi}}
	\begin{itemize}
		\item \textbf{ConcreteAudioRemoveCommand}(id:integer)
		\begin{itemize}
			\item \textbf{Accesso}: Public;
			\item \textbf{Tipo di ritorno}: Void;
			\item \textbf{Descrizione}: costruisce l’oggetto ConcreteAudioRemoveCommand e setta il campo dati id.
		\end{itemize}
		\item \textbf{doAction}()
		\begin{itemize}
			\item \textbf{Accesso}: Public;
			\item \textbf{Tipo di ritorno}: Void;
			\item \textbf{Descrizione}: invoca il metodo InsertEditRemove::removeAudio(id) passando come parametro il campo dati id. removeAudio restituisce una copia dell’oggetto Audio rimosso. Se executed è settato a false lo setta come true, altrimenti invoca il metodo rimuoviElemento(spec) di EditController.
		\end{itemize}
		\item \textbf{undoAction}()
		\begin{itemize}
			\item \textbf{Accesso}: Public;
			\item \textbf{Tipo di ritorno}: Void;
			\item \textbf{Descrizione}: invoca il metodo InsertEditRemove::insertAudio(Audio) passando come parametro il campo oldAudio. Invoca il metodo inserisciAudio() di EditController passandogli oldAudio.
		\end{itemize}
	\end{itemize}
	}

\subsubsubsection{Classe ConcreteVideoRemoveCommand}{
	\label{VideoRemoveCommand}
	\textbf{Funzione}\\
		\indent Classe concreta, è interfaccia del Design Pattern Command.\\
   	\textbf{Scope}\\
		\indent Model::SlideShow::SlideShowActions::Command::AbstractCommand::\-ConcreteVideoRemoveCommand.\\
	\textbf{Utilizzo}\\
		\indent Viene costruito da Premi::Controller::EditController, riceve l’id dell’elemento video da rimuovere dalla presentazione e invoca il metodo InsertEditRemove::removeVideo(id).\\
	\textbf{Attributi}
	\begin{itemize}
		\item \textbf{oldVideo}
		\begin{itemize}
			\item \textbf{Accesso}: Private;
			\item \textbf{Tipo}: SlideShowElements::Video;
			\item \textbf{Descrizione}: è una copia dell’oggetto che rappresenta l’elemento rimosso.
		\end{itemize}
	\end{itemize}
	
	\noindent{\textbf{Metodi}}
	\begin{itemize}
		\item \textbf{ConcreteVideoRemoveCommand}(id:integer)
		\begin{itemize}
			\item \textbf{Accesso}: Public;
			\item \textbf{Tipo di ritorno}: Void;
			\item \textbf{Descrizione}: costruisce l’oggetto ConcreteVideoRemoveCommand e setta il campo dati id.
		\end{itemize}
		\item \textbf{doAction}()
		\begin{itemize}
			\item \textbf{Accesso}: Public;
			\item \textbf{Tipo di ritorno}: Void;
			\item \textbf{Descrizione}: invoca il metodo InsertEditRemove::removeVideo(id) passando come parametro il campo dati id. removeVideo restituisce una copia dell’oggetto Video rimosso. Se executed è settato a false lo setta come true, altrimenti invoca il metodo rimuoviElemento(spec) di EditController.
		\end{itemize}
		\item \textbf{undoAction}()
		\begin{itemize}
			\item \textbf{Accesso}: Public;
			\item \textbf{Tipo di ritorno}: Void;
			\item \textbf{Descrizione}: invoca il metodo InsertEditRemove::insertVideo(Video) passando come parametro il campo oldVideo. Invoca il metodo inserisciVideo() di EditController passandogli oldVideo.
		\end{itemize}
	\end{itemize}
	}

\subsubsubsection{Classe ConcreteEditPositionCommand}{
		\label{EditPositionCommand}
		\textbf{Funzione}\\
			\indent Classe concreta, è interfaccia del Design Pattern Command.\\
	   	\textbf{Scope}\\
			\indent Model::SlideShow::SlideShowActions::Command::AbstractCommand::\-ConcreteEditPositionCommand.\\
		\textbf{Utilizzo}\\
			\indent Viene costruito da Premi::Controller::EditController, riceve l’id dell’elemento dalla presentazione e le coordinate x e y in cui deve essere spostato l’elemento, invoca il metodo InsertEditRemove::editPosition(spec).\\
		\textbf{Attributi}
		\begin{itemize}
			\item \textbf{oldPosition}
			\begin{itemize}
			\item \textbf{Accesso}: Private;
			\item \textbf{Tipo}: Oggetto;
			\item \textbf{Descrizione}: oggetto che contiene i parametri di posizione dell'elemento modificato, contiene al suo interno i campi:
			\begin{itemize}
			\item \textbf{id}
			\begin{itemize}
				\item \textbf{Accesso}: Private;
				\item \textbf{Tipo}: Int;
				\item \textbf{Descrizione}: indica l'id dell’elemento da modificare.
			\end{itemize}
			\item \textbf{tipo}
			\begin{itemize}
				\item \textbf{Accesso}: Private;
				\item \textbf{Tipo}: String;
				\item \textbf{Descrizione}: indica il tipo dell’elemento da modificare.
			\end{itemize}
			\item \textbf{xIndex}
			\begin{itemize}
				\item \textbf{Accesso}: Private;
				\item \textbf{Tipo}: Double;
				\item \textbf{Descrizione}: indica la vecchia posizione sull’asse x dell’elemento.
			\end{itemize}
			\item \textbf{yIndex}
			\begin{itemize}
				\item \textbf{Accesso}: Private;
				\item \textbf{Tipo}: Double;
				\item \textbf{Descrizione}: indica la vecchia posizione sull’asse y dell’elemento.
			\end{itemize}
		\end{itemize}
			\end{itemize}
			\end{itemize}
		\noindent{\textbf{Metodi}}
		\begin{itemize}
			\item \textbf{ConcreteEditPositionCommand}(spec:object)
			\begin{itemize}
				\item \textbf{Accesso}: Public;
				\item \textbf{Tipo di ritorno}: Void;
				\item \textbf{Descrizione}: costruisce l’oggetto ConcreteEditPositionCommand.
			\end{itemize}
			\item \textbf{doAction}()
			\begin{itemize}
				\item \textbf{Accesso}: Public;
				\item \textbf{Tipo di ritorno}: Void;
				\item \textbf{Descrizione}: invoca il metodo InsertEditRemove::editPosition(spec) passando come parametro l'oggetto spec. editPosition ritorna un oggetto a cui ConcreteEditPositionCommand inizializza oldPosition. Se executed è settato a false lo setta come true, altrimenti invoca il metodo muoviElemento(spec) di EditController.
			\end{itemize}
			\item \textbf{undoAction}()
			\begin{itemize}
				\item \textbf{Accesso}: Public;
				\item \textbf{Tipo di ritorno}: Void;
				\item \textbf{Descrizione}: invoca il metodo InsertEditRemove::editPosition(spec) passando come parametro il campo dati oldPosition. Invoca il metodo muoviElemento(oldPosition) di EditController.
			\end{itemize}
		\end{itemize}
		}
	
\subsubsubsection{Classe ConcreteEditRotationCommand}{
		\label{EditRotationCommand}
		\textbf{Funzione}\\
			\indent Classe concreta, è interfaccia del Design Pattern Command.\\
	   	\textbf{Scope}\\
			\indent Model::SlideShow::SlideShowActions::Command::AbstractCommand::\-ConcreteEditRotationCommand.\\
		\textbf{Utilizzo}\\
			\indent Viene costruito da Premi::Controller::EditController, riceve l’id dell’elemento dalla presentazione e il grado a cui deve essere ruotato l’elemento, invoca il metodo InsertEditRemove::editRotation(spec).\\
		\textbf{Attributi}
		\begin{itemize}
			\item \textbf{oldRotation}
			\begin{itemize}
			\item \textbf{Accesso}: Private;
			\item \textbf{Tipo}: Oggetto;
			\item \textbf{Descrizione}: oggetto che contiene i parametri di rotazione dell'elemento modificato, contiene al suo interno i campi:
			\begin{itemize}
			\item \textbf{id}
			\begin{itemize}
				\item \textbf{Accesso}: Private;
				\item \textbf{Tipo}: Int;
				\item \textbf{Descrizione}: indica l'id dell’elemento da modificare.
			\end{itemize}
			\item \textbf{tipo}
			\begin{itemize}
				\item \textbf{Accesso}: Private;
				\item \textbf{Tipo}: String;
				\item \textbf{Descrizione}: indica il tipo dell’elemento da modificare.
			\end{itemize}
			\item \textbf{rotation}
			\begin{itemize}
				\item \textbf{Accesso}: Private;
				\item \textbf{Tipo}: Double;
				\item \textbf{Descrizione}: indica il vecchio grado di rotazione dell’elemento.
			\end{itemize}
		\end{itemize}
			\end{itemize}
			\end{itemize}
		\noindent{\textbf{Metodi}}
		\begin{itemize}
			\item \textbf{ConcreteEditRotationCommand}(spec:object)
			\begin{itemize}
				\item \textbf{Accesso}: Public;
				\item \textbf{Tipo di ritorno}: Void;
				\item \textbf{Descrizione}: costruisce l’oggetto ConcreteEditRotationCommand.
			\end{itemize}
			\item \textbf{doAction}()
			\begin{itemize}
				\item \textbf{Accesso}: Public;
				\item \textbf{Tipo di ritorno}: Void;
				\item \textbf{Descrizione}: invoca il metodo InsertEditRemove::editRotation(spec) passando come parametro l'oggetto spec. editRotation ritorna un oggetto a cui ConcreteEditRotationCommand inizializza oldRotation. Se executed è settato a false lo setta come true, altrimenti invoca il metodo ruotaElemento(spec) di EditController.
			\end{itemize}
			\item \textbf{undoAction}()
			\begin{itemize}
				\item \textbf{Accesso}: Public;
				\item \textbf{Tipo di ritorno}: Void;
				\item \textbf{Descrizione}: invoca il metodo InsertEditRemove::editRotation(spec) passando come parametro il campo dati oldRotation. Invoca il metodo ruotaElemento(oldRotation) di EditController.
			\end{itemize}
		\end{itemize}
		}

\subsubsubsection{Classe ConcreteEditSizeCommand}{
	\label{EditSizeCommand}
	\textbf{Funzione}\\
		\indent Classe concreta, è interfaccia del Design Pattern Command.\\
   	\textbf{Scope}\\
		\indent Model::SlideShow::SlideShowActions::Command::AbstractCommand::\-ConcreteEditSizeCommand.\\
	\textbf{Utilizzo}\\
		\indent Viene costruito da Premi::Controller::EditController, riceve l’id dell’elemento dalla presentazione e le dimensioni a cui deve essere ridimensionato l’elemento, invoca il metodo InsertEditRemove::editSize(spec).\\
			\textbf{Attributi}
			\begin{itemize}
				\item \textbf{oldSize}
				\begin{itemize}
				\item \textbf{Accesso}: Private;
				\item \textbf{Tipo}: Oggetto;
				\item \textbf{Descrizione}: oggetto che contiene i parametri di dimensione dell'elemento modificato, contiene al suo interno i campi:
				\begin{itemize}
				\item \textbf{id}
				\begin{itemize}
					\item \textbf{Accesso}: Private;
					\item \textbf{Tipo}: Int;
					\item \textbf{Descrizione}: indica l'id dell’elemento da modificare.
				\end{itemize}
				\item \textbf{tipo}
				\begin{itemize}
					\item \textbf{Accesso}: Private;
					\item \textbf{Tipo}: String;
					\item \textbf{Descrizione}: indica il tipo dell’elemento da modificare.
				\end{itemize}
				\item \textbf{oldHeight}
				\begin{itemize}
					\item \textbf{Accesso}: Private;
					\item \textbf{Tipo}: Double;
					\item \textbf{Descrizione}: indica la vecchia altezza dell’elemento.
				\end{itemize}
				\item \textbf{oldWidth}
				\begin{itemize}
					\item \textbf{Accesso}: Private;
					\item \textbf{Tipo}: Double;
					\item \textbf{Descrizione}: indica la vecchia larghezza dell’elemento.
				\end{itemize}
			\end{itemize}
				\end{itemize}
				\end{itemize}
			\noindent{\textbf{Metodi}}
			\begin{itemize}
				\item \textbf{ConcreteEditSizeCommand}(spec:object)
				\begin{itemize}
					\item \textbf{Accesso}: Public;
					\item \textbf{Tipo di ritorno}: Void;
					\item \textbf{Descrizione}: costruisce l’oggetto ConcreteEditSizeCommand.
				\end{itemize}
				\item \textbf{doAction}()
				\begin{itemize}
					\item \textbf{Accesso}: Public;
					\item \textbf{Tipo di ritorno}: Void;
					\item \textbf{Descrizione}: invoca il metodo InsertEditRemove::editSize(spec) passando come parametro l'oggetto spec. editSize ritorna un oggetto a cui ConcreteEditSizeCommand inizializza oldSize. Se executed è settato a false lo setta come true, altrimenti invoca il metodo ridimensionaElemento(spec) di EditController.
				\end{itemize}
				\item \textbf{undoAction}()
				\begin{itemize}
					\item \textbf{Accesso}: Public;
					\item \textbf{Tipo di ritorno}: Void;
					\item \textbf{Descrizione}: invoca il metodo InsertEditRemove::editSize(spec) passando come parametro il campo dati oldSize. Invoca il metodo ridimensionaElemento(oldSize) di EditController.
				\end{itemize}
			\end{itemize}
			}

\subsubsubsection{Classe ConcreteEditContentCommand}{
	\label{EditContentCommand}
	\textbf{Funzione}\\
		\indent Classe concreta, è interfaccia del Design Pattern Command.\\
   	\textbf{Scope}\\
		\indent Model::SlideShow::SlideShowActions::Command::AbstractCommand::\-ConcreteEditContentCommand.\\
	\textbf{Utilizzo}\\
		\indent Viene costruito da Premi::Controller::EditController, riceve l’id dell’elemento dalla presentazione e il contenuto di testo che deve essere assegnato all'elemento, invoca il metodo InsertEditRemove::editContent(spec).\\
			\textbf{Attributi}
			\begin{itemize}
				\item \textbf{oldContent}
				\begin{itemize}
				\item \textbf{Accesso}: Private;
				\item \textbf{Tipo}: Oggetto;
				\item \textbf{Descrizione}: oggetto che contiene i parametri di dimensione dell'elemento modificato, contiene al suo interno i campi:
				\begin{itemize}
				\item \textbf{id}
				\begin{itemize}
					\item \textbf{Accesso}: Private;
					\item \textbf{Tipo}: Int;
					\item \textbf{Descrizione}: indica l'id dell’elemento da modificare.
				\end{itemize}
				\item \textbf{tipo}
				\begin{itemize}
					\item \textbf{Accesso}: Private;
					\item \textbf{Tipo}: String;
					\item \textbf{Descrizione}: indica il tipo dell’elemento da modificare.
				\end{itemize}
				\item \textbf{oldContent}
				\begin{itemize}
					\item \textbf{Accesso}: Private;
					\item \textbf{Tipo}: Double;
					\item \textbf{Descrizione}: indica il vecchio contenuto dell’elemento.
				\end{itemize}
			\end{itemize}
				\end{itemize}
				\end{itemize}
			\noindent{\textbf{Metodi}}
			\begin{itemize}
				\item \textbf{ConcreteEditContentCommand}(spec:object)
				\begin{itemize}
					\item \textbf{Accesso}: Public;
					\item \textbf{Tipo di ritorno}: Void;
					\item \textbf{Descrizione}: costruisce l’oggetto ConcreteEditContentCommand.
				\end{itemize}
				\item \textbf{doAction}()
				\begin{itemize}
					\item \textbf{Accesso}: Public;
					\item \textbf{Tipo di ritorno}: Void;
					\item \textbf{Descrizione}: invoca il metodo InsertEditRemove::editContent(spec) passando come parametro l'oggetto spec. editContent ritorna un oggetto a cui ConcreteEditContentCommand inizializza oldContent. Se executed è settato a false lo setta come true, altrimenti invoca il metodo aggiornaTesto() di EditController passandogli spec.
				\end{itemize}
				\item \textbf{undoAction}()
				\begin{itemize}
					\item \textbf{Accesso}: Public;
					\item \textbf{Tipo di ritorno}: Void;
					\item \textbf{Descrizione}: invoca il metodo InsertEditRemove::editContent(spec) passando come parametro il campo dati oldContent. Invoca il metodo aggiornaTesto() di EditController passandogli oldContent.
				\end{itemize}
			\end{itemize}
			}

\subsubsubsection{Classe ConcreteEditBackgroundCommand}{
	\label{EditBackgroundCommand}
	\textbf{Funzione}\\
		\indent Classe concreta, è interfaccia del Design Pattern Command.\\
   	\textbf{Scope}\\
		\indent Model::SlideShow::SlideShowActions::Command::AbstractCommand::\-ConcreteEditBackgroundCommand.\\
	\textbf{Utilizzo}\\
		\indent Viene costruito da Premi::Controller::EditController, riceve l’id dell’elemento dalla presentazione e lo sfondo da applicargli, invoca il metodo InsertEditRemove::editBackground(spec).\\
			\textbf{Attributi}
			\begin{itemize}
				\item \textbf{oldBackground}
				\begin{itemize}
				\item \textbf{Accesso}: Private;
				\item \textbf{Tipo}: Oggetto;
				\item \textbf{Descrizione}: oggetto che contiene i vecchi parametri dello sfondo dell'elemento modificato, contiene al suo interno i campi:
				\begin{itemize}
				\item \textbf{id}
				\begin{itemize}
					\item \textbf{Accesso}: Private;
					\item \textbf{Tipo}: Int;
					\item \textbf{Descrizione}: indica l'id dell’elemento da modificare.
				\end{itemize}
				\item \textbf{tipo}
				\begin{itemize}
					\item \textbf{Accesso}: Private;
					\item \textbf{Tipo}: String;
					\item \textbf{Descrizione}: indica il tipo dell’elemento da modificare.
				\end{itemize}
				\item \textbf{backgroundImage}
				\begin{itemize}
					\item \textbf{Accesso}: Private;
					\item \textbf{Tipo}: String;
					\item \textbf{Descrizione}: indica l'url dell'immagine di sfondo dell’elemento.
				\end{itemize}
				\item \textbf{backgroundColor}
				\begin{itemize}
					\item \textbf{Accesso}: Private;
					\item \textbf{Tipo}: string;
					\item \textbf{Descrizione}: indica il colore dello sfondo dell’elemento.
				\end{itemize}
			\end{itemize}
				\end{itemize}
				\end{itemize}
			\noindent{\textbf{Metodi}}
			\begin{itemize}
				\item \textbf{ConcreteEditBackgroundCommand}(spec:object)
				\begin{itemize}
					\item \textbf{Accesso}: Public;
					\item \textbf{Tipo di ritorno}: Void;
					\item \textbf{Descrizione}: costruisce l’oggetto ConcreteEditBackgroundCommand.
				\end{itemize}
				\item \textbf{doAction}()
				\begin{itemize}
					\item \textbf{Accesso}: Public;
					\item \textbf{Tipo di ritorno}: Void;
					\item \textbf{Descrizione}: invoca il metodo InsertEditRemove::editBackground(spec) passando come parametro l'oggetto spec. editBackground ritorna un oggetto a cui ConcreteEditBackgroundCommand inizializza oldBackground. ConcreteEditBackgroundCommand assegna quindi i campi dati id e tipo dell'oggetto oldBackground. Se executed è settato a false lo setta come true, altrimenti invoca il metodo updateSfondoFrame(spec) di EditController.
				\end{itemize}
				\item \textbf{undoAction}()
				\begin{itemize}
					\item \textbf{Accesso}: Public;
					\item \textbf{Tipo di ritorno}: Void;
					\item \textbf{Descrizione}: invoca il metodo InsertEditRemove::editBackground(spec) passando come parametro il campo dati oldBackground. Invoca il metodo updateSfondoFrame(oldBackground) di EditController.
				\end{itemize}
			\end{itemize}
			}

\subsubsubsection{Classe ConcreteEditColorCommand}{
	\label{EditColorCommand}
	\textbf{Funzione}\\
		\indent Classe concreta, è interfaccia del Design Pattern Command.\\
   	\textbf{Scope}\\
		\indent Model::SlideShow::SlideShowActions::Command::AbstractCommand::\-ConcreteEditColorCommand.\\
	\textbf{Utilizzo}\\
		\indent Viene costruito da Premi::Controller::EditController, riceve l’id dell’elemento dalla presentazione e il colore da applicargli, invoca il metodo InsertEditRemove::editColor(spec).\\
			\textbf{Attributi}
			\begin{itemize}
				\item \textbf{oldColor}
				\begin{itemize}
				\item \textbf{Accesso}: Private;
				\item \textbf{Tipo}: Oggetto;
				\item \textbf{Descrizione}: oggetto che contiene i vecchi parametri dello sfondo dell'elemento modificato, contiene al suo interno i campi:
				\begin{itemize}
				\item \textbf{id}
				\begin{itemize}
					\item \textbf{Accesso}: Private;
					\item \textbf{Tipo}: Int;
					\item \textbf{Descrizione}: indica l'id dell’elemento da modificare.
				\end{itemize}
				\item \textbf{tipo}
				\begin{itemize}
					\item \textbf{Accesso}: Private;
					\item \textbf{Tipo}: String;
					\item \textbf{Descrizione}: indica il tipo dell’elemento da modificare.
				\end{itemize}
				\item \textbf{color}
				\begin{itemize}
					\item \textbf{Accesso}: Private;
					\item \textbf{Tipo}: String;
					\item \textbf{Descrizione}: indica il colore dell’elemento.
				\end{itemize}
			\end{itemize}
				\end{itemize}
				\end{itemize}
			\noindent{\textbf{Metodi}}
			\begin{itemize}
				\item \textbf{ConcreteEditColorCommand}(spec:object)
				\begin{itemize}
					\item \textbf{Accesso}: Public;
					\item \textbf{Tipo di ritorno}: Void;
					\item \textbf{Descrizione}: costruisce l’oggetto ConcreteEditColorCommand.
				\end{itemize}
				\item \textbf{doAction}()
				\begin{itemize}
					\item \textbf{Accesso}: Public;
					\item \textbf{Tipo di ritorno}: Void;
					\item \textbf{Descrizione}: invoca il metodo InsertEditRemove::editColor(spec) passando come parametro l'oggetto spec. editColor ritorna un oggetto a cui ConcreteEditColorCommand inizializza oldColor. ConcreteEditColorCommand assegna quindi i campi dati id e tipo dell'oggetto oldColor. Se executed è settato a false lo setta come true, altrimenti invoca il metodo cambiaColoreTesto() di EditController passandogli spec.
				\end{itemize}
				\item \textbf{undoAction}()
				\begin{itemize}
					\item \textbf{Accesso}: Public;
					\item \textbf{Tipo di ritorno}: Void;
					\item \textbf{Descrizione}: invoca il metodo InsertEditRemove::editColor(spec) passando come parametro il campo dati oldColor. Invoca il metodo cambiaColoreTesto() di EditController passandogli oldColor.
				\end{itemize}
			\end{itemize}
			}
			
\subsubsubsection{Classe ConcreteEditFontCommand}{
	\label{EditFontCommand}
	\textbf{Funzione}\\
		\indent Classe concreta, è interfaccia del Design Pattern Command.\\
	  	\textbf{Scope}\\
		\indent Model::SlideShow::SlideShowActions::Command::AbstractCommand::\-ConcreteEditFontCommand.\\
	\textbf{Utilizzo}\\
		\indent Viene costruito da Premi::Controller::EditController, riceve l’id dell’elemento dalla presentazione e lo sfondo da applicargli, invoca il metodo InsertEditRemove::editFont(spec).\\
			\textbf{Attributi}
			\begin{itemize}
				\item \textbf{oldFont}
				\begin{itemize}
				\item \textbf{Accesso}: Private;
				\item \textbf{Tipo}: Oggetto;
				\item \textbf{Descrizione}: oggetto che contiene i vecchi parametri dello sfondo dell'elemento modificato, contiene al suo interno i campi:
				\begin{itemize}
				\item \textbf{id}
				\begin{itemize}
					\item \textbf{Accesso}: Private;
					\item \textbf{Tipo}: Int;
					\item \textbf{Descrizione}: indica l'id dell’elemento da modificare.
				\end{itemize}
				\item \textbf{tipo}
				\begin{itemize}
					\item \textbf{Accesso}: Private;
					\item \textbf{Tipo}: String;
					\item \textbf{Descrizione}: indica il tipo dell’elemento da modificare.
				\end{itemize}
				\item \textbf{font}
				\begin{itemize}
					\item \textbf{Accesso}: Private;
					\item \textbf{Tipo}: String;
					\item \textbf{Descrizione}: indica il font dell’elemento.
				\end{itemize}
			\end{itemize}
				\end{itemize}
				\end{itemize}
			\noindent{\textbf{Metodi}}
			\begin{itemize}
				\item \textbf{ConcreteEditFontCommand}(spec:object)
				\begin{itemize}
					\item \textbf{Accesso}: Public;
					\item \textbf{Tipo di ritorno}: Void;
					\item \textbf{Descrizione}: costruisce l’oggetto ConcreteEditFontCommand.
				\end{itemize}
				\item \textbf{doAction}()
				\begin{itemize}
					\item \textbf{Accesso}: Public;
					\item \textbf{Tipo di ritorno}: Void;
					\item \textbf{Descrizione}: invoca il metodo InsertEditRemove::editFont(spec) passando come parametro l'oggetto spec. editFont ritorna un oggetto a cui ConcreteEditFontCommand inizializza oldFont. ConcreteEditFontCommand assegna quindi i campi dati id e tipo dell'oggetto oldFont. Se executed è settato a false lo setta come true, altrimenti invoca il metodo cambiaFontTesto() di EditController passandogli spec.
				\end{itemize}
				\item \textbf{undoAction}()
				\begin{itemize}
					\item \textbf{Accesso}: Public;
					\item \textbf{Tipo di ritorno}: Void;
					\item \textbf{Descrizione}: invoca il metodo InsertEditRemove::editFont(spec) passando come parametro il campo dati oldFont. Invoca il metodo cambiaFontTesto() di EditController passandogli oldFont.
				\end{itemize}
			\end{itemize}
	}

\subsubsubsection{Classe ConcreteEditBookmarkCommand}{
	\label{EditBookmarkCommand}
	\textbf{Funzione}\\
		\indent Classe concreta, è interfaccia del Design Pattern Command.\\
	  	\textbf{Scope}\\
		\indent Model::SlideShow::SlideShowActions::Command::AbstractCommand::\-ConcreteEditFontCommand.\\
	\textbf{Utilizzo}\\
		\indent Viene costruito da Premi::Controller::EditController, riceve l’id dell’elemento dalla presentazione e lo sfondo da applicargli, invoca il metodo InsertEditRemove::editFont(spec).\\
			\textbf{Attributi}
			\begin{itemize}
				\item \textbf{oldFont}
				\begin{itemize}
				\item \textbf{Accesso}: Private;
				\item \textbf{Tipo}: Oggetto;
				\item \textbf{Descrizione}: oggetto che contiene i vecchi parametri dello sfondo dell'elemento modificato, contiene al suo interno i campi:
				\begin{itemize}
				\item \textbf{id}
				\begin{itemize}
					\item \textbf{Accesso}: Private;
					\item \textbf{Tipo}: Int;
					\item \textbf{Descrizione}: indica l'id dell’elemento da modificare.
				\end{itemize}
				\item \textbf{tipo}
				\begin{itemize}
					\item \textbf{Accesso}: Private;
					\item \textbf{Tipo}: String;
					\item \textbf{Descrizione}: indica il tipo dell’elemento da modificare.
				\end{itemize}
				\item \textbf{font}
				\begin{itemize}
					\item \textbf{Accesso}: Private;
					\item \textbf{Tipo}: String;
					\item \textbf{Descrizione}: indica il font dell’elemento.
				\end{itemize}
			\end{itemize}
				\end{itemize}
				\end{itemize}
			\noindent{\textbf{Metodi}}
			\begin{itemize}
				\item \textbf{ConcreteEditFontCommand}(spec:object)
				\begin{itemize}
					\item \textbf{Accesso}: Public;
					\item \textbf{Tipo di ritorno}: Void;
					\item \textbf{Descrizione}: costruisce l’oggetto ConcreteEditFontCommand.
				\end{itemize}
				\item \textbf{doAction}()
				\begin{itemize}
					\item \textbf{Accesso}: Public;
					\item \textbf{Tipo di ritorno}: Void;
					\item \textbf{Descrizione}: invoca il metodo InsertEditRemove::editFont(spec) passando come parametro l'oggetto spec. editFont ritorna un oggetto a cui ConcreteEditFontCommand inizializza oldFont. ConcreteEditFontCommand assegna quindi i campi dati id e tipo dell'oggetto oldFont. Se executed è settato a false lo setta come true, altrimenti invoca il metodo updateBookmark() di EditController passandogli spec.
				\end{itemize}
				\item \textbf{undoAction}()
				\begin{itemize}
					\item \textbf{Accesso}: Public;
					\item \textbf{Tipo di ritorno}: Void;
					\item \textbf{Descrizione}: invoca il metodo updateBookmark() di EditController passandogli spec.
				\end{itemize}
			\end{itemize}
	}

\subsubsubsection{Classe ConcretePortaAvantiCommand}{
	\label{PortaAvantiCommand}
	\textbf{Funzione}\\
		\indent Classe concreta, è interfaccia del Design Pattern Command.\\
	  	\textbf{Scope}\\
		\indent Model::SlideShow::SlideShowActions::Command::AbstractCommand::\-ConcretePortaAvantiCommand.\\
	\textbf{Utilizzo}\\
		\indent Viene costruito da Premi::Controller::EditController, riceve l’id dell’elemento dalla presentazione, invoca il metodo InsertEditRemove::portaAvanti(spec).\\
			\textbf{Attributi}
			\begin{itemize}
				\item \textbf{oldZIndex}
				\begin{itemize}
					\item \textbf{Accesso}: Private;
					\item \textbf{Tipo}: Oggetto;
					\item \textbf{Descrizione}: oggetto che contiene i vecchi parametri dello z-index dell'elemento modificato. Tale oggetto contiene i seguenti campi:
					\begin{itemize}
						\item \textbf{id}
						\begin{itemize}
							\item \textbf{Accesso}: Private;
							\item \textbf{Tipo}: Int;
							\item \textbf{Descrizione}: indica l'id dell’elemento da modificare.
						\end{itemize}
						\item \textbf{tipo}
						\begin{itemize}
							\item \textbf{Accesso}: Private;
							\item \textbf{Tipo}: String;
							\item \textbf{Descrizione}: indica il tipo dell’elemento da modificare.
						\end{itemize}
					\end{itemize}
				\end{itemize}
			\end{itemize}
				
			\noindent{\textbf{Metodi}}
			\begin{itemize}
				\item \textbf{ConcretePortaAvantiCommand}(spec:object)
				\begin{itemize}
					\item \textbf{Accesso}: Public;
					\item \textbf{Tipo di ritorno}: Void;
					\item \textbf{Descrizione}: costruisce l’oggetto ConcretePortaAvantiCommand.
				\end{itemize}
				\item \textbf{doAction}()
				\begin{itemize}
					\item \textbf{Accesso}: Public;
					\item \textbf{Tipo di ritorno}: Void;
					\item \textbf{Descrizione}: invoca il metodo InsertEditRemove::portaAvanti(spec) passando come parametro l'oggetto spec. Se executed è settato a false lo setta come true, altrimenti invoca il metodo portaAvanti(spec) di EditController.
				\end{itemize}
				\item \textbf{undoAction}()
				\begin{itemize}
					\item \textbf{Accesso}: Public;
					\item \textbf{Tipo di ritorno}: Void;
					\item \textbf{Descrizione}: invoca il metodo InsertEditRemove::portaDietro(spec) passando come parametro il campo dati oldZIndex. Invoca il metodo portaDietro(oldZIndex) di EditController.
				\end{itemize}
			\end{itemize}
	}

\subsubsubsection{Classe ConcretePortaDietroCommand}{
	\label{PortaDietroCommand}
	\textbf{Funzione}\\
		\indent Classe concreta, è interfaccia del Design Pattern Command.\\
	  	\textbf{Scope}\\
		\indent Model::SlideShow::SlideShowActions::Command::AbstractCommand::\-ConcretePortaDietroCommand.\\
	\textbf{Utilizzo}\\
		\indent Viene costruito da Premi::Controller::EditController, riceve l’id dell’elemento dalla presentazione, invoca il metodo InsertEditRemove::portaDietro(spec).\\
			\textbf{Attributi}
			\begin{itemize}
				\item \textbf{oldZIndex}
				\begin{itemize}
					\item \textbf{Accesso}: Private;
					\item \textbf{Tipo}: Oggetto;
					\item \textbf{Descrizione}: oggetto che contiene i vecchi parametri dello z-index dell'elemento modificato. Tale oggetto contiene i seguenti campi:
					\begin{itemize}
						\item \textbf{id}
						\begin{itemize}
							\item \textbf{Accesso}: Private;
							\item \textbf{Tipo}: Int;
							\item \textbf{Descrizione}: indica l'id dell’elemento da modificare.
						\end{itemize}
						\item \textbf{tipo}
						\begin{itemize}
							\item \textbf{Accesso}: Private;
							\item \textbf{Tipo}: String;
							\item \textbf{Descrizione}: indica il tipo dell’elemento da modificare.
						\end{itemize}
					\end{itemize}
				\end{itemize}
			\end{itemize}
				
			\noindent{\textbf{Metodi}}
			\begin{itemize}
				\item \textbf{ConcretePortaDietroCommand}(spec:object)
				\begin{itemize}
					\item \textbf{Accesso}: Public;
					\item \textbf{Tipo di ritorno}: Void;
					\item \textbf{Descrizione}: costruisce l’oggetto ConcretePortaDietroCommand.
				\end{itemize}
				\item \textbf{doAction}()
				\begin{itemize}
					\item \textbf{Accesso}: Public;
					\item \textbf{Tipo di ritorno}: Void;
					\item \textbf{Descrizione}: invoca il metodo InsertEditRemove::portaDietro(spec) passando come parametro l'oggetto spec. Se executed è settato a false lo setta come true, altrimenti invoca il metodo portaDietro(spec) di EditController.
				\end{itemize}
				\item \textbf{undoAction}()
				\begin{itemize}
					\item \textbf{Accesso}: Public;
					\item \textbf{Tipo di ritorno}: Void;
					\item \textbf{Descrizione}: invoca il metodo InsertEditRemove::portaAvanti(spec) passando come parametro il campo dati oldZIndex. Invoca il metodo portaAvanti(oldZIndex) di EditController.
				\end{itemize}
			\end{itemize}
	}

	\subsubsubsection{Classe ConcreteAddToMainPathCommand}{
		\label{addFrameToMainPathCommand}
		\textbf{Funzione}\\
			\indent Classe concreta, è interfaccia del Design Pattern Command.\\
	   	\textbf{Scope}\\
			\indent Model::SlideShow::SlideShowActions::Command::AbstractCommand::\-concreteAddToMainPathCommand.\\
		\textbf{Utilizzo}\\
			\indent Viene costruito da Premi::Controller::EditController, riceve l’id del frame da inserire nella presentazione e la sua posizione all'interno del percorso principale.\\
				
				\noindent{\textbf{Metodi}}
				\begin{itemize}
					\item \textbf{concreteAddToMainPathCommand}(spec:object)
					\begin{itemize}
						\item \textbf{Accesso}: Public;
						\item \textbf{Tipo di ritorno}: Void;
						\item \textbf{Descrizione}: costruisce l’oggetto concreteAddToMainPathCommand.
					\end{itemize}
					\item \textbf{doAction}()
					\begin{itemize}
						\item \textbf{Accesso}: Public;
						\item \textbf{Tipo di ritorno}: Void;
						\item \textbf{Descrizione}: invoca il metodo InsertEditRemove::addFrameToMainPath(spec) passando come parametro l'oggetto spec. Se executed è posto a zero, lo pone a uno, altrimenti invoca il metodo aggiungiMainPath(spec) di EditController.
					\end{itemize}
					\item \textbf{undoAction}()
					\begin{itemize}
						\item \textbf{Accesso}: Public;
						\item \textbf{Tipo di ritorno}: Void;
						\item \textbf{Descrizione}: invoca il metodo InsertEditRemove::removeFrameToMainPath(id) passando come parametro il campo id del parametro spec. Invoca il metodo rimuoviMainPath(spec) di EditController.
					\end{itemize}
				\end{itemize}
				}
				
	\subsubsubsection{Classe ConcreteRemoveFromMainPathCommand}{
		\label{RemoveFrameFromMainPathCommand}
		\textbf{Funzione}\\
			\indent Classe concreta, è interfaccia del Design Pattern Command.\\
	   	\textbf{Scope}\\
			\indent Model::SlideShow::SlideShowActions::Command::AbstractCommand::\-concreteRemoveFromMainPathCommand.\\
		\textbf{Utilizzo}\\
			\indent Viene costruito da Premi::Controller::EditController, riceve l’id del frame da rimuovere dal percorso principale della presentazione.\\
				\textbf{Attributi}
			\begin{itemize}
				\item \textbf{oldFrame}
				\begin{itemize}
				\item \textbf{Accesso}: Private;
				\item \textbf{Tipo}: Oggetto;
				\item \textbf{Descrizione}: oggetto che contiene i vecchi parametri che indicano la posizione del frame all'interno del percorso originale prima della rimozione:
				\begin{itemize}
				\item \textbf{id}
				\begin{itemize}
					\item \textbf{Accesso}: Private;
					\item \textbf{Tipo}: Int;
					\item \textbf{Descrizione}: indica l'id del frame da rimuovere all'interno del percorso principale.
				\end{itemize}
				\item \textbf{pos}
				\begin{itemize}
					\item \textbf{Accesso}: Private;
					\item \textbf{Tipo}: Integer;
					\item \textbf{Descrizione}: indica la posizione del frame nel percorso principale, prima dell'avvenuta rimozione.
				\end{itemize}
			\end{itemize}
				\end{itemize}
				\end{itemize}
				\noindent{\textbf{Metodi}}
				\begin{itemize}
					\item \textbf{concreteRemoveFromMainPathCommand}(spec:object)
					\begin{itemize}
						\item \textbf{Accesso}: Public;
						\item \textbf{Tipo di ritorno}: Void;
						\item \textbf{Descrizione}: costruisce l’oggetto concreteRemoveFromMainPathCommand.
					\end{itemize}
					\item \textbf{doAction}()
					\begin{itemize}
						\item \textbf{Accesso}: Public;
						\item \textbf{Tipo di ritorno}: Void;
						\item \textbf{Descrizione}: invoca il metodo InsertEditRemove::removeFrameFromMainPath(spec) passando come parametro l'oggetto spec.  removeFrameFromMainPath ritorna un oggetto a cui concreteRemoveFromMainPathCommand inizializza oldFrame. Se executed è posto a zero, lo pone a uno, altrimenti invoca il metodo rimuoviMainPath(null, spec) di EditController.
					\end{itemize}
					\item \textbf{undoAction}()
					\begin{itemize}
						\item \textbf{Accesso}: Public;
						\item \textbf{Tipo di ritorno}: Void;
						\item \textbf{Descrizione}: invoca il metodo InsertEditRemove::addFrameToMainPath(id) passando come parametro l'oggetto oldFrame. Invoca il metodo aggiungiMainPath(oldFrame) di EditController.
					\end{itemize}
				\end{itemize}
				}

	\subsubsubsection{Classe ConcreteNewChoicePathCommand}{
		\label{NewChoicePathCommand}
		\textbf{Funzione}\\
			\indent Classe concreta, è interfaccia del Design Pattern Command.\\
	   	\textbf{Scope}\\
			\indent Model::SlideShow::SlideShowActions::Command::AbstractCommand::\-ConcreteNewChoicePathCommand.\\
		\textbf{Utilizzo}\\
			\indent Viene costruito da Premi::Controller::EditController, riceve l'id del frame da cui parte il nuovo percorso scelta.\\
				\textbf{Attributi}
			\begin{itemize}
				\item \textbf{PathId}
				\begin{itemize}
					\item \textbf{Accesso}: Private;
					\item \textbf{Tipo}: Int;
					\item \textbf{Descrizione}: indica l'id del percorso.
				\end{itemize}
				\noindent{\textbf{Metodi}}
				\begin{itemize}
					\item \textbf{concreteNewChoicePathCommand}(id: integer)
					\begin{itemize}
						\item \textbf{Accesso}: Public;
						\item \textbf{Tipo di ritorno}: Void;
						\item \textbf{Descrizione}: costruisce l’oggetto concreteNewChoicePathCommand.
					\end{itemize}
					\item \textbf{doAction}()
					\begin{itemize}
						\item \textbf{Accesso}: Public;
						\item \textbf{Tipo di ritorno}: Void;
						\item \textbf{Descrizione}: invoca il metodo InsertEditRemove::addChoicePath() passando come parametro il valore spec.id ricevuto come parametro.  addChoicePath ritorna il valore a cui viene inizializzato pathId. Se executed è posto a zero, lo pone a uno, altrimenti invoca il metodo appropriato di EditController.
					\end{itemize}
					\item \textbf{undoAction}()
					\begin{itemize}
						\item \textbf{Accesso}: Public;
						\item \textbf{Tipo di ritorno}: Void;
						\item \textbf{Descrizione}: invoca il metodo InsertEditRemove::deleteChoicePath() passando come parametro il valore pathId. Invoca il metodo appropriato di EditController.
					\end{itemize}
				\end{itemize}
				\end{itemize}

\subsubsubsection{Classe ConcreteNewChoicePathCommand}{
	\label{NewChoicePathCommand}
	\textbf{Funzione}\\
		\indent Classe concreta, è interfaccia del Design Pattern Command.\\
   	\textbf{Scope}\\
		\indent Model::SlideShow::SlideShowActions::Command::AbstractCommand::\-ConcreteNewChoicePathCommand.\\
	\textbf{Utilizzo}\\
		\indent Viene costruito da Premi::Controller::EditController, riceve l'id da cui parte il nuovo percorso scelta.\\
			\textbf{Attributi}
		\begin{itemize}
			\item \textbf{pathId}
			\begin{itemize}
			\item \textbf{Accesso}: Private;
			\item \textbf{Tipo}: Integer;
			\item \textbf{Descrizione}: copia dell'id del percorso eliminato.
			\end{itemize}
			\end{itemize}
			\noindent{\textbf{Metodi}}
			\begin{itemize}
				\item \textbf{concreteNewChoicePathCommand}(spec:object)
				\begin{itemize}
					\item \textbf{Accesso}: Public;
					\item \textbf{Tipo di ritorno}: Void;
					\item \textbf{Descrizione}: costruisce l’oggetto concreteDeleteChoicePathCommand.
				\end{itemize}
				\item \textbf{doAction}()
				\begin{itemize}
					\item \textbf{Accesso}: Public;
					\item \textbf{Tipo di ritorno}: Void;
					\item \textbf{Descrizione}: invoca il metodo InsertEditRemove::addChoicePath() passando come parametro il valore spec.id. deleteChoicePath ritorna la copia del percorso rimosso, a cui viene inizializzato oldPath. Se executed è posto a zero, lo pone a uno, altrimenti invoca il metodo appropriato EditController.
				\end{itemize}
				\item \textbf{undoAction}()
				\begin{itemize}
					\item \textbf{Accesso}: Public;
					\item \textbf{Tipo di ritorno}: Void;
					\item \textbf{Descrizione}: invoca il metodo InsertEditRemove::deleteChoicePath() passando come parametro l'oggetto pathId. Invoca il metodo appropriato di EditController.
				\end{itemize}
			\end{itemize}
			
			}

\subsubsubsection{Classe ConcreteDeleteChoicePathCommand}{
	\label{DeleteChoicePathCommand}
	\textbf{Funzione}\\
		\indent Classe concreta, è interfaccia del Design Pattern Command.\\
   	\textbf{Scope}\\
		\indent Model::SlideShow::SlideShowActions::Command::AbstractCommand::\-ConcreteDeleteChoicePathCommand.\\
	\textbf{Utilizzo}\\
		\indent Viene costruito da Premi::Controller::EditController, riceve l'id del  percorso scelta da rimuovere.\\
			\textbf{Attributi}
		\begin{itemize}
			\item \textbf{oldPath}
			\begin{itemize}
			\item \textbf{Accesso}: Private;
			\item \textbf{Tipo}: Oggetto;
			\item \textbf{Descrizione}: copia del percorso eliminato.
			\end{itemize}
			\end{itemize}
			\noindent{\textbf{Metodi}}
			\begin{itemize}
				\item \textbf{concreteDeleteChoicePathCommand}(spec: object)
				\begin{itemize}
					\item \textbf{Accesso}: Public;
					\item \textbf{Tipo di ritorno}: Void;
					\item \textbf{Descrizione}: costruisce l’oggetto concreteDeleteChoicePathCommand.
				\end{itemize}
				\item \textbf{doAction}()
				\begin{itemize}
					\item \textbf{Accesso}: Public;
					\item \textbf{Tipo di ritorno}: Void;
					\item \textbf{Descrizione}: invoca il metodo InsertEditRemove::deleteChoicePath() passando come parametro il valore spec.id. deleteChoicePath ritorna la copia del percorso rimosso, a cui viene inizializzato oldPath. Se executed è posto a zero, lo pone a uno, altrimenti invoca il metodo appropriato di EditController.
				\end{itemize}
				\item \textbf{undoAction}()
				\begin{itemize}
					\item \textbf{Accesso}: Public;
					\item \textbf{Tipo di ritorno}: Void;
					\item \textbf{Descrizione}: invoca il metodo InsertEditRemove::addChoicePath() passando come parametro l'oggetto oldPath. Invoca il metodo appropriato di EditController.
				\end{itemize}
			\end{itemize}
			
			}

\subsubsubsection{Classe ConcreteRemoveFromChoicePathCommand}{
	\label{RemoveFromChoicePathCommand}
	\textbf{Funzione}\\
		\indent Classe concreta, è interfaccia del Design Pattern Command.\\
   	\textbf{Scope}\\
		\indent Model::SlideShow::SlideShowActions::Command::AbstractCommand::\-ConcreteRemoveFromChoicePathCommand.\\
	\textbf{Utilizzo}\\
		\indent Viene costruito da Premi::Controller::EditController, riceve l'id del frame da rimuovere da un percorso a scelta.\\
			\textbf{Attributi}
		\begin{itemize}
			\item \textbf{oldFrame}
			\begin{itemize}
			\item \textbf{Accesso}: Private;
			\item \textbf{Tipo}: Oggetto;
			\item \textbf{Descrizione}: copia del frame eliminato.
			\end{itemize}
			\end{itemize}
			\noindent{\textbf{Metodi}}
			\begin{itemize}
				\item \textbf{concreteRemoveFromChoicePathCommand}(spec:object)
				\begin{itemize}
					\item \textbf{Accesso}: Public;
					\item \textbf{Tipo di ritorno}: Void;
					\item \textbf{Descrizione}: costruisce l’oggetto concreteRemoveFromChoicePathCommand.
				\end{itemize}
				\item \textbf{doAction}()
				\begin{itemize}
					\item \textbf{Accesso}: Public;
					\item \textbf{Tipo di ritorno}: Void;
					\item \textbf{Descrizione}: invoca il metodo InsertEditRemove::removeFrameFromChoicePath(spec). removeFrameFromChoicePath ritorna la copia del frame rimosso, a cui viene inizializzato oldFrame. Se executed è posto a zero, lo pone a uno, altrimenti invoca il metodo appropriato di EditController.
				\end{itemize}
				\item \textbf{undoAction}()
				\begin{itemize}
					\item \textbf{Accesso}: Public;
					\item \textbf{Tipo di ritorno}: Void;
					\item \textbf{Descrizione}: invoca il metodo InsertEditRemove::addFrameToChoicePath() passando come parametro l'oggetto oldFrame. Invoca il metodo appropriato di EditController.
				\end{itemize}
			\end{itemize}
			
			}

\subsubsubsection{Classe ConcreteAddToChoicePathCommand}{
	\label{AddToChoicePathCommand}
	\textbf{Funzione}\\
		\indent Classe concreta, è interfaccia del Design Pattern Command.\\
   	\textbf{Scope}\\
		\indent Model::SlideShow::SlideShowActions::Command::AbstractCommand::\-ConcreteAddToChoicePathCommand.\\
	\textbf{Utilizzo}\\
		\indent Viene costruito da Premi::Controller::EditController, riceve l'id del frame da aggiungere ad un percorso a scelta.\\
			
		\noindent{\textbf{Metodi}}
		\begin{itemize}
			\item \textbf{concreteAddToChoicePathCommand}(spec:object)
			\begin{itemize}
				\item \textbf{Accesso}: Public;
				\item \textbf{Tipo di ritorno}: Void;
				\item \textbf{Descrizione}: costruisce l’oggetto concreteAddToChoicePathCommand.
			\end{itemize}
			\item \textbf{doAction}()
			\begin{itemize}
				\item \textbf{Accesso}: Public;
				\item \textbf{Tipo di ritorno}: Void;
				\item \textbf{Descrizione}: invoca il metodo InsertEditRemove::addFrameToChoicePath(spec).  Se executed è posto a zero, lo pone a uno, altrimenti invoca il metodo appropriato di EditController.
			\end{itemize}
			\item \textbf{undoAction}()
			\begin{itemize}
				\item \textbf{Accesso}: Public;
				\item \textbf{Tipo di ritorno}: Void;
				\item \textbf{Descrizione}: invoca il metodo InsertEditRemove::removeFrameFromChoicePath(spec) e invoca il metodo appropriato di EditController.
			\end{itemize}
		\end{itemize}
			
			}

\subsubsubsection{Classe ConcreteAddToMainPathCommand}{
	\label{AddToMainPathCommand}
	\textbf{Funzione}\\
		\indent Classe concreta, è interfaccia del Design Pattern Command.\\
   	\textbf{Scope}\\
		\indent Model::SlideShow::SlideShowActions::Command::AbstractCommand::\-ConcreteAddToMainPathCommand.\\
	\textbf{Utilizzo}\\
		\indent Viene costruito da Premi::Controller::EditController, riceve l'id del frame da aggiungere al percorso principale.\\
			
		\noindent{\textbf{Metodi}}
		\begin{itemize}
			\item \textbf{concreteAddToMainPathCommand}(spec:object)
			\begin{itemize}
				\item \textbf{Accesso}: Public;
				\item \textbf{Tipo di ritorno}: Void;
				\item \textbf{Descrizione}: costruisce l’oggetto concreteAddToMainPathCommand.
			\end{itemize}
			\item \textbf{doAction}()
			\begin{itemize}
				\item \textbf{Accesso}: Public;
				\item \textbf{Tipo di ritorno}: Void;
				\item \textbf{Descrizione}: invoca il metodo InsertEditRemove::addFrameToMainPath(spec).  Se executed è posto a zero, lo pone a uno, altrimenti invoca il metodo aggiungiMainPath(spec) di EditController.
			\end{itemize}
			\item \textbf{undoAction}()
			\begin{itemize}
				\item \textbf{Accesso}: Public;
				\item \textbf{Tipo di ritorno}: Void;
				\item \textbf{Descrizione}: invoca il metodo InsertEditRemove::removeFrameFromMainPath(spec) e invoca il metodo rimuoviMainPath() di EditController passandogli spec.
			\end{itemize}
		\end{itemize}
			
	}

\subsubsubsection{Classe ConcreteRemoveFromMainPathCommand}{
	\label{RemoveFromMainPathCommand}
	\textbf{Funzione}\\
		\indent Classe concreta, è interfaccia del Design Pattern Command.\\
   	\textbf{Scope}\\
		\indent Model::SlideShow::SlideShowActions::Command::AbstractCommand::\-ConcreteRemoveFromMainPathCommand.\\
	\textbf{Utilizzo}\\
		\indent Viene costruito da Premi::Controller::EditController, riceve l'id del frame da togliere dal percorso principale.\\
			
		\noindent{\textbf{Metodi}}
		\begin{itemize}
			\item \textbf{concreteRemoveFromMainPathCommand}(spec:object)
			\begin{itemize}
				\item \textbf{Accesso}: Public;
				\item \textbf{Tipo di ritorno}: Void;
				\item \textbf{Descrizione}: costruisce l’oggetto concreteRemoveFromMainPathCommand.
			\end{itemize}
			\item \textbf{doAction}()
			\begin{itemize}
				\item \textbf{Accesso}: Public;
				\item \textbf{Tipo di ritorno}: Void;
				\item \textbf{Descrizione}: invoca il metodo InsertEditRemove::removeFrameFromMainPath(spec).  Se executed è posto a zero, lo pone a uno, altrimenti invoca il metodo rimuoviMainPath() di EditController passandogli spec.
			\end{itemize}
			\item \textbf{undoAction}()
			\begin{itemize}
				\item \textbf{Accesso}: Public;
				\item \textbf{Tipo di ritorno}: Void;
				\item \textbf{Descrizione}: invoca il metodo InsertEditRemove::addFrameToMainPath(spec) e invoca il metodo aggiungiMainPath(spec) di EditController.
			\end{itemize}
		\end{itemize}
			
	}