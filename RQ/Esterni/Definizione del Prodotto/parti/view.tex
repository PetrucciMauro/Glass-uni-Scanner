\section{Package View}{
	\textbf{\tipo}: contiene le classi che istanzieranno gli oggetti per l'interfaccia grafica del software.\\
	\textbf{\relaz}: utilizza le classi contenute nel package Controller per le comunicazioni con il Model.\\
	\textbf{\attivita}: rappresenta l'intera GUI del nostro sistema.
\subsection{View::Pages}{
\textbf{\tipo}: contiene le pagine in Html, rappresentano l'interfaccia grafica vera e propria.\\
\textbf{\relaz}: utilizza le classi contenute nel package Controller.\\
\textbf{\attivita}: rappresenta le pagine fisiche del software.
}
\subsection{View::Pages::Index}{
	\textbf{Funzione}\\
		\indent Questa pagina si occuperà di mostrare all'utente header e footer validi per ogni pagina html e permettendogli di accedere alle pagine Login, Registrazione, Home e Profile e di poter effettuare il logout dal sistema .\\
	\textbf{Relazioni d'uso con altri moduli}\\
		\indent Questa pagina utilizzerà le seguenti classi:
	\begin{itemize}
		\item Controller::HeaderController.
	\end{itemize}
	\textbf{Input utente}
		\begin{itemize}
		\item bottoneAccedi(): richiama Controller::HeaderController::goLogin() che reindirizza alla pagina View::Pages::Login;
		\item bottoneRegistrati(): richiama Controller::HeaderController::goRegistrazione() che reindirizza alla pagina View::Pages::Registrazione;
		\item bottoneHome(): richiama Controller::HeaderController::goHome() che reindirizza alla pagina View::Pages::Home;
		\item bottoneProfilo(): richiama Controller::HeaderController::goProfile() che reindirizza alla pagina View::Pages::Profilo;
	\end{itemize}
	}
\subsection{View::Pages::Login}{
	\textbf{Funzione}\\
		\indent Questa pagina si occuperà di mostrare all'utente la possibilità di effettuare il login.\\
	\textbf{Relazioni d'uso con altri moduli}\\
		\indent Questa pagina utilizzerà le seguenti classi:
	\begin{itemize}
		\item Controller::AccessController.
	\end{itemize}
	\textbf{Input utente}
		\begin{itemize}
		\item bottoneLogin(): attiva il metodo Controller::AccessController::login() che controlla se i campi della form sono stati compilati correttamente. Se l'operazione ha successo, viene effettuato il reindirizzamento alla pagina View::Pages::Home;
	\end{itemize}
	}
	\subsection{View::Pages::Registrazione}{
	\textbf{Funzione}\\
		\indent Questa pagina si occuperà di mostrare all'utente la possibilità di effettuare la registrazione al sistema.\\
	\textbf{Relazioni d'uso con altri moduli}\\
		\indent Questa pagina utilizzerà le seguenti classi:
	\begin{itemize}
		\item Controller::AccessController.
	\end{itemize}
	\textbf{Input utente}
		\begin{itemize}
		\item bottoneRegistrati(): attiva il metodo Controller::AccessController::registration() che controlla se i campi della form sono stati compilati correttamente. Se l'operazione ha successo, viene effettuato il reindirizzamento alla pagina View::Pages::Home;
	\end{itemize}
	}
\subsection{View::Pages::Home}{
	\textbf{Funzione}\\
	\indent Questa pagina si occuperà di mostrare all'utente le presentazioni presenti sul proprio database dando la possibilità di eliminarle, eseguirle, scaricarle in locale, rinominarle, modificarle o crearne di nuove.\\
	\textbf{Relazioni d'uso con altri moduli}\\
	\indent Questa pagina utilizzerà le seguenti classi:
	\begin{itemize}
		\item Controller::HomeController
	\end{itemize}
	\textbf{Input utente}
	\begin{itemize}
		\item bottoneNuovaPresentazione(): bottone che richiama il metodo di Controller::HomeController::createSlideShow();
		\item bottoneElimina(): bottone che richiama il metodo di Controller::HomeController::deleteSlideShow() passandogli il nome della presentazione da eliminare;
		\item bottoneRinomina(): bottone che richiama il metodo di Controller::HomeController::renameSlideShow() passandogli il nome della presentazione da rinominare;
		\item bottoneEsegui(): bottone che richiama il metodo di Controller::HomeController::goExecute() passandogli il nome della presentazione da eseguire;
		\item bottoneEdit(): bottone che richiama il metodo di Controller::HomeController::goEdit() passandogli il nome della presentazione da modificare;
		\item bottoneSalva(): bottone che richiama il metodo di Controller::HomeController::salvaManifest() passandogli il nome della presentazione da salvare in locale.
	\end{itemize}
	}
\subsection{View::Pages::Profile}{
	\textbf{Funzione}\\
	\indent Questa pagina si occuperà di mostrare all'utente la possibilità di cambiare i propri dati personali.\\
	\textbf{Relazioni d'uso con altri moduli}\\
	\indent Questa pagina utilizzerà le seguenti classi:
	\begin{itemize}
		\item Controller::ProfileController
	\end{itemize}
	\textbf{Input utente}
	\begin{itemize}
		\item bottoneCambiaPassword(): bottone che richiama il metodo di Controller::ProfileController::changePassword(). Il risultato dell'operazione viene in ogni caso comunicato all'utente.
	\end{itemize}
}
\subsection{View::Pages::Execution}{
	\textbf{Funzione}\\
	\indent Questa pagina si occuperà di gestire l'esecuzione di una presentazione utilizzando il framework Impress.js associato alla pagina.\\\\
	\textbf{Relazioni d'uso con altri moduli}\\
	\indent Questa pagina utilizzerà le seguenti classi:
	\begin{itemize}
		\item Controller::ExecutionController
	\end{itemize}
	\textbf{Input utente}
	\begin{itemize}
		\item +next(): metodo che viene invocato premendo il tasto "freccia destra" e che invoca a sua volta il metodo impress().next() implementato all'interno del framework Impress.js e che visualizzerà il frame successivo della presentazione;
		\item +prev(): metodo che viene invocato premendo il tasto "freccia sinistra" e che invoca a sua volta il metodo impress().prev() implementato all'interno del framework Impress.js e che visualizzerà il frame precedente della presentazione;
		\item +bookmark(): metodo che viene invocato premendo il tasto "barra spaziatrice" e che invoca a sua volta il metodo impress().bookmark() implementato all'interno del framework Impress.js e che visualizzerà il frame con bookmark successivo.
	\end{itemize}
}
\subsection{View::Pages::Edit}{
	\textbf{Funzione}\\
	\indent Questa pagina si occuperà di mostrare all'utente la possibilità di apportare modifiche ad una presentazione.\\
	\textbf{Relazioni d'uso con altri moduli}\\
	\indent Questa pagina utilizzerà le seguenti classi:
	\begin{itemize}
		\item Controller::EditController
	\end{itemize}
	\textbf{Attributi}\\
		\begin{itemize}
			\item active: oggetto che rappresenta l'elemento attualmente selezionato;
			\item mainPath: oggetto che rappresenta il percorso principale della presentazione.
		\end{itemize}
	\textbf{Input utente}
	\begin{itemize}
		\item bottoneEseguiPresentazione(): bottone che richiama Controller::EditController::goExecute() per eseguire la presentazione;
		\item bottoneAnnulla: bottone che richiama Controller::EditController::annullaModifica() per annullare l'ultima modifica eseguita;
		\item bottoneRipristina: bottone che richiama Controller::EditController::ripristinaModifica() per ripristina l'ultima modifica annullata;
		\item bottoneInserisciFrame: bottone che richiama Controller::EditController::inserisciFrame() per inserire un frame nel piano della presentazione;
		\item bottoneInserisciTesto: bottone che richiama Controller::EditController::inserisciTesto() per inserire un elemento testo nel piano della presentazione;
		\item bottoneInserisciImmagine: bottone che richiama Controller::EditController::inserisciImmagine() passandogli le immagini da inserire;
		\item bottoneInserisciAudio: bottone che richiama Controller::EditController::inserisciAudio() passandogli gli audio da inserire;
		\item bottoneInserisciVideo: bottone che richiama Controller::EditController::inserisciVideo() passandogli i video da inserire;
		\item bottoneRuota: bottone che richiama Controller::EditController::ruotaElemento() passandogli il valore della rotazione da applicare all'elemento selezionato;
		\item bottoneCambiaColoreSfondo: bottone che richiama Controller::EditController::cambiaColoreSfondo() passandogli il valore del colore da applicare al background della presentazione;
		\item bottoneCambiaImmagineSfondo: bottone che richiama Controller::EditController::cambiaImmagineSfondo() passandogli l'immagine da applicare al background della presentazione;
		\item eliminaSfondoPresentazine: bottone che richiama Controller::EditController::rimuoviSfondo() per resettare lo sfondo della presentazione;
		\item bottoneCambiaColoreSfondoFrame: bottone che richiama Controller::EditController::cambiaColoreSfondoFrame() passandogli il valore del colore da applicare al frame selezionato;
		\item bottoneCambiaImmagineSfondoFrame: bottone che richiama Controller::EditController::cambiaImmagineSfondoFrame() passandogli l'immagine da applicare al background del frame selezionato;
		\item eliminaSfondoFrame: bottone che richiama Controller::EditController::rimuoviSfondo() per resettare lo sfondo del frame selezionato;
		\item bottoneAggiungiPercorsoPrincipale: bottone che richiama Controller::EditControlelr::aggiungiMainPath() per aggiungere il frame corrente al percorso principale di presentazione;

		\item inserisciFrame(spec): funzione javascript che permette l'inserimento di un nuovo frame. Se spec è definito, le proprietà contenute in esso vengono assegnate al frame appena inserito;
		\item inserisciTesto(spec): funzione javascript che permette l'inserimento di un nuovo elemento testo. Se spec è definito, le proprietà contenute in esso vengono assegnate al testo appena inserito;
		\item inserisciImmagine(x, spec): funzione javascript che permette l'inserimento di un nuovo elemento immagine con path passato tramite il parametro x. Se spec è definito, le proprietà contenute in esso vengono assegnate all'immagine appena inserita;
		\item inserisciAudio(x, spec): funzione javascript che permette l'inserimento di un nuovo elemento audio con path passato tramite il parametro x. Se spec è definito, le proprietà contenute in esso vengono assegnate all'audio appena inserito;
		\item inserisciVideo(x, spec): funzione javascript che permette l'inserimento di un nuovo elemento video con path passato tramite il parametro x. Se spec è definito, le proprietà contenute in esso vengono assegnate al video appena inserito;
		\item elimina(id): funzione javascript che permette l'eliminazione dell'elemento id dal piano della presentazione;
		\item rotate(el, value): funzione javascript che permette di ruotare l'elemento el in base al valore definito dal parametro value;
		\item portaAvanti(id): funzione javascript che incrementa la proprietà zIndex dell'elemento id;
		\item mandaDietro(id): funzione javascript che decrementa la proprietà zIndex dell'elemento id;
		\item Drag&Drop: evento javascript che permette di poter spostare un elemento all'interno del piano della presentazione;
		\item Resizable: evento javascript che permette di poter ridimensionare un elemento all'interno del piano della presentazione;


		\item bottoneTextEdit(): bottone che modifica il testo selezionato e attiva il metodo in Controller::EditController che si occupa dell'aggiornamento dell'elemento testuale modificato; 
		
		\item bottoneInsertChoice(): bottone che fa inserire all'utente il testo della scelta e fa scegliere il frame al quale farà riferimento, attiva il metodo di Controller::EditController che si occuperà dell'aggiornamento della presentazione;
		\item bottoneBookmark(): bottone che assegna o rimuove il bookmark al frame, attiva il metodo di Controller::EditController che si occuperà dell'aggiornamento della presentazione;

		\item bottoneInsertSvg(): bottone che fa comparire nel piano della presentazione il nuovo elemento svg selezionato e attiva il metodo in Controller::EditController che si occuperà di aggiornare le informazioni della presentazione;
	\end{itemize}
}

