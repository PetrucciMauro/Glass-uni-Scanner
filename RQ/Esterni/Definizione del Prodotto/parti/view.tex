\section{Package Premi::View}{
	Tutti i package seguenti appartengono al package Premi, quindi per ognuno di essi lo scope sarà: Premi::[nome package]. \\\\
	\textbf{\tipo}: contiene le classi che istanzieranno gli oggetti per l'interfaccia grafica del Software\ped{g}.\\
	\textbf{\relaz}: utilizza le classi contenute nel package Controller per le comunicazioni con il Model.\\
	\textbf{\attivita}: rappresenta l'intera GUI del nostro sistema.
\subsection{View::Pages}{
\textbf{\tipo}: contiene le pagine in Html, rappresentano l'interfaccia grafica vera e propria.\\
\textbf{\relaz}: utilizza le classi contenute nel package Controller.\\
\textbf{\attivita}: rappresenta le pagine fisiche del Software\ped{g}.
}
\subsection{View::Pages::Index}{
	\textbf{Funzione}\\
		\indent Questa pagina si occuperà di mostrare all'utente header e footer validi per ogni pagina html e permettendogli di accedere alle pagine Login\ped{g}, Registrazione, Home e Profile e di poter effettuare il Logout\ped{g} dal sistema .\\
	\textbf{Relazioni d'uso con altri moduli}\\
		\indent Questa pagina utilizzerà le seguenti classi:
	\begin{itemize}
		\item Controller::HeaderController.
	\end{itemize}
	\textbf{Input utente}
		\begin{itemize}
		\item bottoneAccedi(): richiama Controller::HeaderController::goLogin() che reindirizza alla pagina View::Pages::Login;
		\item bottoneRegistrati(): richiama Controller::HeaderController::goRegistrazione() che reindirizza alla pagina View::Pages::Registrazione;
		\item bottoneHome(): richiama Controller::HeaderController::goHome() che reindirizza alla pagina View::Pages::Home;
		\item bottoneProfilo(): richiama Controller::HeaderController::goProfile() che reindirizza alla pagina View::Pages::Profilo;
	\end{itemize}
	}
\subsection{View::Pages::Login}{
	\textbf{Funzione}\\
		\indent Questa pagina si occuperà di mostrare all'utente la possibilità di effettuare il Login\ped{g}.\\
	\textbf{Relazioni d'uso con altri moduli}\\
		\indent Questa pagina utilizzerà le seguenti classi:
	\begin{itemize}
		\item Controller::AccessController.
	\end{itemize}
	\textbf{Input utente}
		\begin{itemize}
		\item bottoneLogin(): attiva il metodo Controller::AccessController::login() che controlla se i campi della form sono stati compilati correttamente. Se l'operazione ha successo, viene effettuato il reindirizzamento alla pagina View::Pages::Home;
	\end{itemize}
	}
	\subsection{View::Pages::Registrazione}{
	\textbf{Funzione}\\
		\indent Questa pagina si occuperà di mostrare all'utente la possibilità di effettuare la registrazione al sistema.\\
	\textbf{Relazioni d'uso con altri moduli}\\
		\indent Questa pagina utilizzerà le seguenti classi:
	\begin{itemize}
		\item Controller::AccessController.
	\end{itemize}
	\textbf{Input utente}
		\begin{itemize}
		\item bottoneRegistrati(): attiva il metodo Controller::AccessController::registration() che controlla se i campi della form sono stati compilati correttamente. Se l'operazione ha successo, viene effettuato il reindirizzamento alla pagina View::Pages::Home;
	\end{itemize}
	}
\subsection{View::Pages::Home}{
	\textbf{Funzione}\\
	\indent Questa pagina si occuperà di mostrare all'utente le presentazioni presenti sul proprio database dando la possibilità di eliminarle, eseguirle, scaricarle in locale, rinominarle, modificarle o crearne di nuove.\\
	\textbf{Relazioni d'uso con altri moduli}\\
	\indent Questa pagina utilizzerà le seguenti classi:
	\begin{itemize}
		\item Controller::HomeController
	\end{itemize}
	\textbf{Input utente}
	\begin{itemize}
		\item bottoneNuovaPresentazione(): bottone che richiama il metodo di Controller::HomeController::createSlideShow();
		\item bottoneElimina(): bottone che richiama il metodo di Controller::HomeController::deleteSlideShow() passandogli il nome della presentazione da eliminare;
		\item bottoneRinomina(): bottone che richiama il metodo di Controller::HomeController::renameSlideShow() passandogli il nome della presentazione da rinominare;
		\item bottoneEsegui(): bottone che richiama il metodo di Controller::HomeController::goExecute() passandogli il nome della presentazione da eseguire;
		\item bottoneEdit(): bottone che richiama il metodo di Controller::HomeController::goEdit() passandogli il nome della presentazione da modificare;
	\end{itemize}
	}
\subsection{View::Pages::Profile}{
	\textbf{Funzione}\\
	\indent Questa pagina si occuperà di mostrare all'utente la possibilità di cambiare i propri dati personali.\\
	\textbf{Relazioni d'uso con altri moduli}\\
	\indent Questa pagina utilizzerà le seguenti classi:
	\begin{itemize}
		\item Controller::ProfileController
	\end{itemize}
	\textbf{Input utente}
	\begin{itemize}
		\item bottoneCambiaPassword(): bottone che richiama il metodo di Controller::ProfileController::changePassword(). Il risultato dell'operazione viene in ogni caso comunicato all'utente.
	\end{itemize}
}
\subsection{View::Pages::Execution}{
	\textbf{Funzione}\\
	\indent Questa pagina si occuperà di gestire l'esecuzione di una presentazione utilizzando il Framework\ped{g} Impress.js associato alla pagina.\\\\
	\textbf{Relazioni d'uso con altri moduli}\\
	\indent Questa pagina utilizzerà le seguenti classi:
	\begin{itemize}
		\item Controller::ExecutionController
	\end{itemize}
	\textbf{Input utente}
	\begin{itemize}
		\item +next(): metodo che viene invocato premendo il tasto "freccia destra" e che invoca a sua volta il metodo impress().next() implementato all'interno del Framework\ped{g} Impress.js e che visualizzerà il Frame\ped{g} successivo della presentazione;
		\item +prev(): metodo che viene invocato premendo il tasto "freccia sinistra" e che invoca a sua volta il metodo impress().prev() implementato all'interno del Framework\ped{g} Impress.js e che visualizzerà il Frame\ped{g} precedente della presentazione;
		\item +bookmark(): metodo che viene invocato premendo il tasto "barra spaziatrice" e che invoca a sua volta il metodo impress().bookmark() implementato all'interno del Framework\ped{g} Impress.js e che visualizzerà il Frame\ped{g} con Bookmark\ped{g} successivo.
	\end{itemize}
}
\subsection{View::Pages::Edit}{
	\textbf{Funzione}\\
	\indent Questa pagina si occuperà di mostrare all'utente la possibilità di apportare modifiche ad una presentazione.\\
	\textbf{Relazioni d'uso con altri moduli}\\
	\indent Questa pagina utilizzerà le seguenti classi:
	\begin{itemize}
		\item Controller::EditController
	\end{itemize}
	\textbf{Attributi}\\
		\begin{itemize}
			\item active: oggetto che rappresenta l'Elemento\ped{g} attualmente selezionato. Metodi:
			\begin{itemize}
				\item getId(): ritorna l'Elemento\ped{g} selezionato;
				\item getTipo(): ritorna il tipo dell'Elemento\ped{g} selezionato;
				\item select(id): seleziona l'Elemento\ped{g} id nel piano delle presentazione;
				\item deselect(): deseleziona l'Elemento\ped{g} attivo.
			\end{itemize}
			\item mainPath: oggetto che rappresenta il Percorso\ped{g} principale della presentazione. Metodi:
			\begin{itemize}
				\item addToMainPath(id, position): metodo che aggiunge il Frame\ped{g} id nella posizione position del Percorso\ped{g} principale;
				\item removeFromMainPath(id): metodo che elimina il Frame\ped{g} id dal Percorso\ped{g} principale;
			\end{itemize}
		\end{itemize}
	\textbf{Input utente}
	\begin{itemize}
		\item bottoneEseguiPresentazione(): bottone che richiama Controller::EditController::goExecute() per eseguire la presentazione;
		\item bottoneAnnulla: bottone che richiama annullaModifica() di EditController per annullare l'ultima modifica eseguita;
		\item bottoneRipristina: bottone che richiama ripristinaModifica() di EditController per ripristina l'ultima modifica annullata;
		\item bottoneInserisciFrame: bottone che richiama inserisciFrame() di EditController per inserire un Frame\ped{g} nel piano della presentazione\ped{g};
		\item bottoneInserisciTesto: bottone che richiama inserisciTesto() di EditController per inserire un Elemento\ped{g} testo nel piano della presentazione\ped{g};
		\item bottoneInserisciImmagine: bottone che richiama inserisciImmagine() di EditController passandogli le immagini da inserire;
		\item bottoneInserisciAudio: bottone che richiama inserisciAudio() di EditController passandogli gli audio da inserire;
		\item bottoneInserisciVideo: bottone che richiama inserisciVideo() di EditController passandogli i video da inserire;
		\item bottoneRuota: bottone che richiama ruotaElemento() di EditController passandogli il valore della rotazione da applicare all'Elemento\ped{g} selezionato;
		\item bottoneCambiaColoreSfondo: bottone che richiama cambiaColoreSfondo() di EditController passandogli il valore del colore da applicare al background della presentazione;
		\item bottoneCambiaImmagineSfondo: bottone che richiama cambiaImmagineSfondo() di EditController passandogli l'immagine da applicare al background della presentazione;
		\item eliminaSfondoPresentazine: bottone che richiama rimuoviSfondo() di EditController per resettare lo sfondo della presentazione;
		\item bottoneCambiaColoreSfondoFrame: bottone che richiama cambiaColoreSfondoFrame() di EditController passandogli il valore del colore da applicare al Frame\ped{g} selezionato;
		\item bottoneCambiaImmagineSfondoFrame: bottone che richiama cambiaImmagineSfondoFrame() di EditController passandogli l'immagine da applicare al background del Frame\ped{g} selezionato;
		\item eliminaSfondoFrame: bottone che richiama rimuoviSfondoFrame() di EditController per resettare lo sfondo del Frame\ped{g} selezionato;
		\item bottoneAggiungiPercorsoPrincipale: bottone che richiama aggiungiMainPath() di EditController per aggiungere il Frame\ped{g} corrente al Percorso\ped{g} principale di presentazione;
		\item bottoneRimuoviPercorsoPrincipale: bottone che richiama rimuoviMainPath() di EditController per rimuovere il Frame\ped{g} corrente dal Percorso\ped{g} principale di presentazione;
		\item bottonePortaAvanti(): bottone che richiama portaAvanti() di EditController per cambiare il parametro z-index dell'Elemento\ped{g} selezionato;
		\item bottonePortaDietro(): bottone che richiama portaDietro() di EditController per cambiare il parametro z-index dell'Elemento\ped{g} selezionato;
		\item bottoneBookmark(): bottone che assegna o rimuove il Bookmark\ped{g} al Frame\ped{g}, attiva il metodo di updateBookmark() di EditController che si occuperà dell'aggiornamento della presentazione;

		\item inserisciFrame(spec): Funzione\ped{g} JavaScript\ped{g} che permette l'inserimento di un nuovo Frame\ped{g}. Se spec è definito, le proprietà contenute in esso vengono assegnate al Frame\ped{g} appena inserito;
		\item inserisciTesto(spec): Funzione\ped{g} JavaScript\ped{g} che permette l'inserimento di un nuovo Elemento\ped{g} testo. Se spec è definito, le proprietà contenute in esso vengono assegnate al testo appena inserito;
		\item inserisciImmagine(x, spec): Funzione\ped{g} JavaScript\ped{g} che permette l'inserimento di un nuovo Elemento\ped{g} immagine con path passato tramite il parametro x. Se spec è definito, le proprietà contenute in esso vengono assegnate all'immagine appena inserita;
		\item inserisciAudio(x, spec): Funzione\ped{g} JavaScript\ped{g} che permette l'inserimento di un nuovo Elemento\ped{g} audio con path passato tramite il parametro x. Se spec è definito, le proprietà contenute in esso vengono assegnate all'audio appena inserito;
		\item inserisciVideo(x, spec): Funzione\ped{g} JavaScript\ped{g} che permette l'inserimento di un nuovo Elemento\ped{g} video con path passato tramite il parametro x. Se spec è definito, le proprietà contenute in esso vengono assegnate al video appena inserito;
		\item elimina(id): Funzione\ped{g} JavaScript\ped{g} che permette l'eliminazione dell'Elemento\ped{g} id dal piano della presentazione\ped{g};
		\item rotate(el, value): Funzione\ped{g} JavaScript\ped{g} che permette di ruotare l'Elemento\ped{g} el in base al valore definito dal parametro value;
		\item portaAvanti(id): Funzione\ped{g} JavaScript\ped{g} che incrementa la proprietà zIndex dell'Elemento\ped{g} id;
		\item mandaDietro(id): Funzione\ped{g} JavaScript\ped{g} che decrementa la proprietà zIndex dell'Elemento\ped{g} id;
		\item aggiornaTesto(id, element): metodo che richiama aggiornaTesto(id, element.value) di EditController per aggiornare il json con il nuovo testo inserito;
		\item Drag\&Drop: evento JavaScript\ped{g} che permette di poter spostare un Elemento\ped{g} all'interno del piano della presentazione\ped{g};
		\item Resizable: evento JavaScript\ped{g} che permette di poter ridimensionare un Elemento\ped{g} all'interno del piano della presentazione\ped{g};
		\item mediaControl(): metodo che permette di riprodurre un File\ped{g} media o di fermarne la riproduzione;
	\end{itemize}
}

