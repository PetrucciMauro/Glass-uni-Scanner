\section{Analisi dei rischi}{
Sono di seguito elencati i rischi evidenziati nella parte di analisi del Progetto\ped{g}. 
I rischi saranno caratterizzati dalla pericolosità che potrà essere: bassa, media o alta.
Un'altra caratterizzazione è data dalla probabilità che l'evento associato al rischio si verifichi, la probabilità potrà essere: bassa, media o alta.\\
Per ogni rischio sarà definito un metodo con cui il rischio stesso sarà valutato nell'evolversi del Progetto\ped{g}.\\
Per ogni rischio saranno definiti i metodi da usare come contromisure per diminuire la probabilità che il rischio si verifichi oppure per limitare il danno che il rischio creerebbe nel momento in cui si verificasse l'evento associato.\\
All'inizio di ogni nuova fase di Progetto\ped{g} il responsabile dovrà aggiornare l'analisi dei rischi con nuovi rischi evidenziati dall'avanzare dello stato del Progetto\ped{g} e aggiornando pericolosità e probabilità di avverarsi dei rischi precedentemente inseriti.

	\renewcommand*{\arraystretch}{1.7}
	\begin{longtable} [c]{|>{\centering\arraybackslash}m{3cm} | >{\centering\arraybackslash}m{6cm} | >{\centering\arraybackslash}m{6cm} |}
			\caption{Analisi dei Rischi \label{tab:analisirischi}}\\
			 \hline
			 \textbf{Rischio} & \textbf{Pericolosità} & \textbf{Probabilità} \\
			 \hline \endfirsthead
			 \hline
			 \textbf{Rischio} & \textbf{Pericolosità} & \textbf{Probabilità} \\
			 \hline \endhead
			 \hline \endfoot
			 \hline \endlastfoot
			  \multirow{4}{3cm}{Conoscenza delle tecnologie adottate} & Media & Media\\
			  \cline{2-3}
			  & \multicolumn{2}{p{12cm}|}{\textbf{Controllo}: il responsabile dovrà verificare la conoscenza da parte dei membri del gruppo dei linguaggi di programmazione e delle tecnologie che saranno adottate per lo sviluppo del sistema prima che il Progetto\ped{g} entri in fase di Codifica\ped{g}.} \\
			  \cline{2-3}
			  & \multicolumn{2}{p{12cm}|}{\textbf{Contromisure}: il responsabile fornirà o consiglierà i documenti contenenti la base teorica e pratica per un utilizzo efficace dei linguaggi e delle tecnologie adottate per lo sviluppo del sistema.} \\
			  \cline{2-3}
			  & \multicolumn{2}{p{12cm}|}{\textbf{Riscontro}: questo è il rischio che maggiormente ha rallentato il proseguo del Progetto\ped{g}. Essendo buona parte delle tecnologie completamente sconosciute al gruppo, ogni componente ha dovuto impiegare uno sforzo maggiore nel loro studio ma questo ha permesso di rispettare il piano di Progetto\ped{g} e non accumulare ritardi. Si ritiene che per le prossime fasi questo rischio non avrà alcun peso.} \\
			  \hline
			  \multirow{4}{3cm}{Conoscenza degli strumenti di progetto} & Media & Media\\
			  \cline{2-3}
			  & \multicolumn{2}{p{12cm}|}{\textbf{Controllo}: l'amministratore prima che il Progetto\ped{g} entri nella fase di progettazione dovrà verificare che tutti i componenti abbiano le conoscenze necessarie per utilizzare efficacemente gli strumenti per lo sviluppo e l'amministrazione del Progetto\ped{g}.} \\
			  \cline{2-3}
			  & \multicolumn{2}{p{12cm}|}{\textbf{Contromisure}: l'amministratore fornirà i documenti contenenti la base teorica e pratica per un utilizzo efficace degli strumenti scelti per lo sviluppo e l'amministrazione del Progetto\ped{g}.} \\
  			 % \cline{2-3}
  			  & \multicolumn{2}{p{12cm}|}{\textbf{Riscontro}: questo rischio ha avuto finora una rilevanza minima. Con poche ore di lavoro il gruppo è riuscito a padroneggiare senza grossi problemi quasi tutti gli strumenti che deve utilizzare.} \\
			  \hline
			  \multirow{4}{3cm}{Inesperienza di pianificazione} & Alta & Media\\
			  \cline{2-3}
			  & \multicolumn{2}{p{12cm}|}{\textbf{Controllo}: il responsabile di Progetto\ped{g} dovrà monitorare il completamento delle attività assegnate ai componenti e confrontare lo stato del Progetto\ped{g} con lo stato atteso dalla pianificazione.} \\
			  \cline{2-3}
			  & \multicolumn{2}{p{12cm}|}{\textbf{Contromisure}: per ridurre la pericolosità del rischio è stato deciso di adottare un ciclo di vita\ped{g} incrementale ovvero in una prima Iterazione\ped{g} si provvederà alla progettazione in dettaglio e Codifica\ped{g} di una base di prodotto che comprenderà i Requisiti\ped{g} obbligatori mentre ad una seconda Iterazione\ped{g} verrà effettuata progettazione in dettaglio e Codifica\ped{g} dei Requisiti\ped{g} considerati desiderabili o opzionali. In questo modo anche se fosse stato sottostimato lo sforzo per lo sviluppo dei Requisiti\ped{g} obbligatori del sistema si verrebbe comunque ad avere una base di prodotto con le funzionalità fondamentali, nel caso in cui il Proponente\ped{g} considerasse di grande valore i Requisiti\ped{g} desiderabili o opzionali che non potrebbero essere garantiti dall'attuale piano di Progetto\ped{g} si provvederà ad aggiornare la pianificazione con inevitabili conseguenze sul prospetto economico.} \\
			  \cline{2-3}
			  & \multicolumn{2}{p{12cm}|}{\textbf{Riscontro}: nonostante i tempi ristretti, grazie al ciclo di vita\ped{g} incrementale le scadenze sono state rispettate, anche se con dei leggeri ritardi che il gruppo ha cercato di coprire. Si presume che per le prossime revisioni questo rischio non darà grossi problemi.} \\			  
			  \hline 
			  \multirow{4}{3cm}{Problemi hardware del server} & Alta & Bassa\\
			  \cline{2-3}
			  & \multicolumn{2}{p{12cm}|}{\textbf{Controllo}: ogni componente del gruppo usando i servizi offerti dal Server\ped{g} controlleranno che esso funzioni\ped{g} correttamente, in caso contrario contatteranno l'amministratore.} \\
			  \cline{2-3}
			  & \multicolumn{2}{p{12cm}|}{\textbf{Contromisure}: ogni due giorni dovrà esser fatto in automatico il backup dei dati presenti sul Server\ped{g} in un'apposita cartella su Google Drive.} \\
			  \cline{2-3}
			  & \multicolumn{2}{p{12cm}|}{\textbf{Riscontro}: nonostante i diversi momenti in cui il Server\ped{g} era offline per problemi tecnici, il gruppo è comunque riuscito a continuare col proprio lavoro; questo anche perché la Repository\ped{g} si trova in un Server\ped{g} esterno a quello usato internamente dal gruppo.} \\
			  %\hline
			  \pagebreak
			  \multirow{4}{3cm}{Problemi personali dei componenti} & Media & Alta\\
			  \cline{2-3}
			  & \multicolumn{2}{p{12cm}|}{\textbf{Controllo}: quando un componente non potrà essere in grado di ricoprire i ruoli a lui assegnati dovrà segnalare il fatto attraverso il calendario condiviso di gruppo e comunicare il problema al responsabile di Progetto\ped{g}.} \\
			  \cline{2-3}
			  & \multicolumn{2}{p{12cm}|}{\textbf{Contromisure}: il responsabile di Progetto\ped{g} ottenuta la segnalazione di impossibilità da parte di un componente di rivestire il ruolo a lui assegnato provvederà a modificare la pianificazione in base a quanto riportato nel calendario di gruppo.} \\
 			  \cline{2-3}
 			  & \multicolumn{2}{p{12cm}|}{\textbf{Riscontro}: gli impegni personali dei componenti non hanno finora causato difficoltà, anche perché sono state adottate le contromisure di cui sopra.} \\
			  \hline
			  \multirow{4}{3cm}{Problemi di relazione tra i componenti} & Alta & Media\\
			  \cline{2-3}
			  & \multicolumn{2}{p{12cm}|}{\textbf{Controllo}: il responsabile dovrà monitorare la nascita di problemi relazionali tra i componenti del gruppo.} \\
			  \cline{2-3}
			  & \multicolumn{2}{p{12cm}|}{\textbf{Contromisure}: una volta osservato un problema relazionale tra due componenti del gruppo il responsabile dovrà intervenire favorendo la convergenza di opinione oppure provvedere a modificare la pianificazione per ridurre la necessità di relazione tra i due componenti.} \\
			  \cline{2-3}
			  & \multicolumn{2}{p{12cm}|}{\textbf{Riscontro}: all'interno del gruppo si è instaurato un clima abbastanza sereno il che permette ad ogni componente di lavorare senza grandi problemi. Senza dubbio, ci sono stati alcuni momenti di discussione, ma questo non ha influito particolarmente nel Progetto\ped{g}.} \\
			  \hline
			  \multirow{4}{3cm}{Ambiguità dei requisiti} & Alta & Media\\
			  \cline{2-3}
			  & \multicolumn{2}{p{12cm}|}{\textbf{Controllo}: al fine di ridurre l'ambiguità dei Requisiti\ped{g} si dovrà verificare che ogni termine non ovvio presente in analisi sia compreso nel glossario di Progetto\ped{g}.} \\
			  \cline{2-3}
			  & \multicolumn{2}{p{12cm}|}{\textbf{Contromisure}: stesura di un glossario di Progetto\ped{g}, inoltre gli incontri con il Proponente\ped{g} dovranno produrre un verbale secondo quanto descritto nelle Norme di Progetto\ped{g}. La specifica dei Requisiti\ped{g} dovrà essere accettata dal Proponente\ped{g} prima di passare alla fase di Progettazione.}\\
			  \cline{2-3}
			  & \multicolumn{2}{p{12cm}|}{\textbf{Riscontro}: grazie al glossario, non ci sono stati problemi nella comprensione dei Requisiti\ped{g}. Ci si impegna nel mantenere tale documento aggiornato e arricchito qualora se ne presentasse il bisogno.} \\
	\end{longtable}
}


				   