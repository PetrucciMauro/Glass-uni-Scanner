\section{Consuntivo}
Questa sezione contiene il prospetto economico che riporta le spese effettivamente sostenute. Vengono riportate le ore impiegate per svolgere i compiti preventivati, sia per
ruolo che per persona. In base alle differenza di ore tra il preventivo e il consuntivo, detta conguaglio, avremmo un bilancio:
\begin{itemize}
\item \textbf{Positivo:} se il preventivo ha superato il consuntivo;
\item \textbf{Negativo:} se il consuntivo ha superato il preventivo;
\item \textbf{In pari:} se consuntivo e preventivo coincidono.
\end{itemize}

\subsection{Analisi}

Si riporta di seguito il consuntivo della fase di \textbf{Analisi.}\\
La tabella sottostante riporta le ore effettivamente impiegate e tra parentesi la differenza di ore tra preventivo e consuntivo, divise per ruolo. Come si può notare dal valore riportato nella riga \textbf{Differenza dei totali}, intesa come differenza tra preventivo e consuntivo, il segno negativo indica che è stata impiegata un'ora in più per svolgere le attività programmate con un bilancio in passivo di 10\euro.

	\begin{table}[H]
		\centering
	  \begin{tabular}{p{\dimexpr 0.3\linewidth-2\tabcolsep}p{\dimexpr 0.2\linewidth-2\tabcolsep}
		    							p{\dimexpr 0.2\linewidth-2\tabcolsep}}
		   \toprule Ruolo & Ore & Costi \\
		   \midrule
		   Responsabile & 55(-2) & 1650(-60) \\
		   Amministratore & 16(-1) & 320(-20) \\
		   Analista & 75(+3) & 1875(+75) \\
		   Programmatore & 0 & 0 \\
		   Progettista & 0 & 0 \\
		   Verificatore & 32(+1) & 480(+15) \\
		   \hline
		   \textbf{Totale consuntivo} & 180 & 4335 \\
		   \textbf{Totale preventivo} & 179 & 4325 \\
		   \textbf{Differenza dei totali} & -1 & -10 \\
		   \bottomrule
	 \end{tabular}
	 	\label{tab:costuntivoRequisiti}
	 	\caption{Differenza preventivo-consuntivo per ruolo, fase di Analisi}
	\end{table}

\subsubsection{Conclusioni}
L'attuazione delle attività pianificate e riportate nella tabella \ref{tab:pianorequisiti} si è discostata leggermente da quanto pianificato nella \textbf{Progettazione}.
Il gruppo ha impiegato, in totale , un'ora in più per completare la fase di \textbf{Analisi} provocando così un deficit nel bilancio di 35\euro.\\
Tale passivo non andrà ad influenzare il costo totale del Progetto\ped{g} in quanto le ore impiegate in questa fase non vengono poste a carico del Proponente\ped{g}.

\subsection{Progettazione}

Si riporta di seguito il consuntivo della fase di \textbf{Progettazione} che andrà ad incidere sulle fasi successive.\\
La tabella sottostante riporta le ore effettivamente impiegate e tra parentesi la differenza di ore tra preventivo e consuntivo, divise per ruolo. Come si può notare dal valore riportato nella riga \textbf{Differenza dei totali}, intesa come differenza tra preventivo e consuntivo, il segno positivo indica che sono state impiegate 2 ore in meno risparmiando 50\euro.

	\begin{table}[H]
		\centering
	  \begin{tabular}{p{\dimexpr 0.3\linewidth-2\tabcolsep}p{\dimexpr 0.2\linewidth-2\tabcolsep}
		    							p{\dimexpr 0.2\linewidth-2\tabcolsep}}
		   \toprule Ruolo & Ore & Costi \\
		   \midrule
		   Responsabile & 14(-1) & 390(-30) \\
		   Amministratore & 10(-1) & 200(-20) \\
		   Analista & 6(0) & 150(0) \\
		   Programmatore & 0 & 0 \\
		   Progettista & 76(0) & 1672(0) \\
		   Verificatore & 39(0) & 585(0) \\
		   \hline
		   \textbf{Totale consuntivo} & 143 & 2977 \\
		   \textbf{Totale preventivo} & 145 & 3027 \\
		   \textbf{Differenza dei totali} & 2 & 50 \\
		   \bottomrule
	 \end{tabular}
	 	\label{tab:costuntivoProgettazione}
	 	\caption{Differenza preventivo-consuntivo per ruolo, fase di Progettazione}
	\end{table}

\subsubsection{Conclusioni}

Dalla tabella riportata si può osservare che le ore effettive per svolgere le attività pianificate, sono state inferiori rispetto alla pianificazione, questo ci fa capire che la pianificazione era pessimistica sui tempi di Progetto\ped{g} per questo periodo.
Nonostante questo abbiamo accumulato un sensibile ritardo dovuto ad una ottimistica pianificazione nei tempi di calendario; questo ha portato ad una modifica della pianificazione per le successive attività, in particolare le date per la revisione di qualifica e la revisione di accettazione sono state spostate al 24/08/2015




