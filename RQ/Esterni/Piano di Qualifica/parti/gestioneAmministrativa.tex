\section{Gestione amministrativa della revisione}{
\subsection{Comunicazione e risoluzione di anomalie}{
Un'\textit{anomalia} corrisponde a:
\begin{itemize}
	\item Violazione delle norme tipografiche in un documento;
	\item Uscita dal range d'accettazione degli indici di misurazione;
	\item Incongruenza del prodotto con funzionalità presenti nell'analisi dei Requisiti\ped{g};
	\item Incongruenza del Codice\ped{g} con il design del prodotto.
\end{itemize}

In caso un \emph{Verificatore} riscontri un'anomalia, aprirà un Ticket\ped{g} nel sistema di Ticketing\ped{g} con le modalità specificate nelle Norme di Progetto\ped{g}. \\
Le modalità di risoluzione di quest'ultimo e la sua struttura vengono descritte in modo dettagliato all'interno del documento \href{run:../../Interni/\fNormeDiProgetto}{\fEscapeNormeDiProgetto}. \\
Quando viene rilasciata una nuova versione di un documento od un modulo, il \emph{Verificatore} controlla il registro delle modifiche ed in base ad esso effettua una verifica alla ricerca di anomalie da correggere. Se ne trova, apre un Ticket\ped{g} e lo comunica all'\emph{Amministratore}; s'occuperà della correzione la persona che ha apportato la modifica al documento o modulo. Le nuove modifiche dovranno essere approvate dall'\emph{Amministratore}.
}
}