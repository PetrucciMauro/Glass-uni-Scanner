\section{Pianificazione dei test}{
Si descrivono di seguito tutti i test di Validazione\ped{g}, sistema ed integrazione previsti,prevedendo un aggiornamento futuro per i test di unità. Per le tempistiche di esecuzione
dei test si faccia riferimento al  \href{run:../../Esterni/\fPianoDiProgetto}{\fEscapePianoDiProgetto}.
\subsection{Test di sistema}{
In questa sezione vengono descritti i test di sistema che permettono di verificare il
comportamento dinamico del sistema completo rispetto ai Requisiti\ped{g} descritti nell’ \href{run:../../Esterni/\fAnalisiDeiRequisiti}{\fEscapeAnalisiDeiRequisiti}.
	I test di sistema riportati sono quelli relativi ai Requisiti\ped{g} Software\ped{g} individuati e pertanto
	meritevoli di un test.
	
	
		\renewcommand*{\arraystretch}{1.4}
		\begin{longtable} [c]{| p{3cm} | p{6cm} |p{3cm}|}
			\caption{Descrizione dei test di sistema per i Requisiti\ped{g} Funzionali
			 \label{tab:verReqFunz}}\\
			 \hline
			 \textbf{Test} & \textbf{Descrizione} & \textbf{Requisito} \\
			 \hline
			 \endfirsthead
			 \hline
			 \textbf{Test} & \textbf{Descrizione} & \textbf{Requisito} \\
			 \hline
			\endhead
			 \hline
			 \endfoot
			 \hline
			 \endlastfoot
			TS 1 & Viene verificato che ci si possa registrare al sistema inserendo username e password & RF 1\\
			\hline
			TS 1.1 & Viene verificato che si possa immettere uno username univoco & RF 1.1\\
			\hline
			TS 1.2 & Viene verificato che si possa immettere una password valida & RF 1.2\\
			\hline
			TS 3 & Viene verificato che ci si possa autenticare con username e password & RF 3\\
			\hline
			TS 3.1 & Viene verificato che si possa immettere uno username & RF 3.1\\
			\hline
			TS 3.2 & Viene verificato che si possa immettere una password & RF 3.2\\
			\hline
			TS 4 & Viene verificato che si possa creare una nuova presentazione vuota & RF 4\\
			\hline
			TS 7 & Viene verificato che si possa passare in modalità modifica di una presentazione da Desktop\ped{g} & RF 7\\
			\hline
			TS 7.1 & Viene verificato che si possa inserire un nuovo Frame\ped{g} nel piano della presentazione\ped{g} & RF 7.1\\
			\hline
			TS 7.1.1 & Viene verificato che si possa scegliere il tipo di Frame\ped{g} da inserire & RF 7.1.1\\
			\hline
			TS 7.4 & Viene verificato che si possa spostare un Frame\ped{g} in modalità modifica & RF 7.4\\
			\hline
			TS 7.7 & Viene verificato che si possa passare in modalità modifica di un Frame\ped{g} & RF 7.7\\
			\hline
			TS 7.7.1 & Viene verificato che si possa inserire del testo all'interno di un Frame\ped{g} & RF 7.7.1\\
			\hline
			TS 7.7.4 & Viene verificato che si possa modificare del testo già presente all'interno di un Frame\ped{g} & RF 7.7.4\\
			\hline
			TS 7.7.7 & Viene verificato che si possa inserire un'immagine all'interno del Frame\ped{g} & RF 7.7.7\\
			\hline
			TS 7.7.10 & Viene verificato che si possa modificare la dimensione di un'immagine & RF 7.7.10\\
			\hline
			TS 7.7.13 & Viene verificato che si possa inserire un video o un audio all'interno di un Frame\ped{g} & RF 7.7.13\\
			\hline
			TS 7.7.16 & Viene verificato che si possa modificare la dimensione di un video all'interno di un Frame\ped{g} & RF 7.7.16\\
			\hline
			TS 7.7.19 & Viene verificato che si possa spostare un Elemento\ped{g} all'interno del Frame\ped{g} & RF 7.7.19\\
			\hline
			TS 7.7.25 & Verificare che si possa inserire un Elemento scelta\ped{g} in un Frame\ped{g} & RF 7.7.25\\
			\hline
			TS 7.7.28 & Viene verificato che si possa modificare un Elemento scelta\ped{g} & RF 7.7.28\\
			\hline
			TS 7.7.31 & Viene verificato che si possa modificare la forma di un Frame\ped{g} & RF 7.7.31\\
			\hline
			TS 7.7.34 & Viene verificato che si possa modificare la dimensione di un Frame\ped{g} & RF 7.7.34\\
			\hline
			TS 7.7.37 & Viene verificato che si possa modificare  lo spessore del bordo di un Frame\ped{g} & RF 7.7.37\\
			\hline
			TS 7.7.40 & Viene verificato che si possa modificare il colore del bordo di un Frame\ped{g} & RF 7.7.40\\
			\hline
			TS 7.7.43 & Viene verificato che si possa modificare lo sfondo di un Frame\ped{g} & RF 7.7.43\\
			\hline
			TS 7.7.46 & Viene verificato che si possa ruotare un Frame\ped{g} & RF 7.7.46 \\
			\hline
			TS 7.10 & Viene verificato che si possa eliminare un Frame\ped{g} dal piano di una presentazione & RF 7.10\\
			\hline
			TS 7.13 & Viene verificato che si possa inserire un'immagine di sfondo in un'area della presentazione & RF 7.13 \\
			\hline
			TS 7.16 & Viene verificato che si possa inserire un colore di sfondo in un'area della presentazione & RF 7.16\\
			\hline
			TS 7.19 & Viene verificato che si possa definire un Percorso\ped{g} di visualizzazione & RF 7.19\\
			\hline				
			TS 7.19.1 & Viene verificato che si possa impostare un Frame\ped{g} iniziale per il Percorso\ped{g} di presentazione & RF7.19.1\\
			\hline
			TS 7.19.4 & Viene verificato che si possa definire una transizione tra due Frame\ped{g} & RF 7.19.4\\
			\hline
			TS 7.19.10 & Viene verificato che si possa eliminare una transizione tra due Frame\ped{g} & RF 7.19.10\\
			\hline
			TS 7.19.13 & Viene verificato che si possa togliere un Frame\ped{g} dal Percorso\ped{g} di presentazione & RF 7.19.13\\
			\hline
			TS 7.22 & Viene verificato che si possa assegnare un Bookmark\ped{g} ad un Frame\ped{g} & RF 7.22\\
			\hline
			TS 7.25 & Viene verificato che si possa rimuovere un Bookmark\ped{g} da un Frame\ped{g} & RF 7.25\\
			\hline
			TS 7.28 & Viene verificato che si possa modificare la velocità di transizione tra due Frame\ped{g} consecutivi & RF 7.28\\
			\hline
			TS 7.31 & Viene verificato che si possa impostare un effetto di transizione tra due Frame\ped{g} consecutivi & RF 7.31\\
			\hline
			TS 7.34 & Viene verificato che si possa impostare il tempo di attesa tra due Frame\ped{g} consecutivi durante la "riproduzione automatica" & RF 7.34\\
			\hline
			TS 7.37 & Viene verificato che si possa inserire un Elemento\ped{g} SVG in un Frame\ped{g} o nel piano della presentazione\ped{g} & RF 7.37\\
			\hline
			TS 7.40.1 & Viene verificato che si possa modificare le dimensioni di un Elemento\ped{g} SVG  & RF 7.40.1\\
			\hline
			TS 7.40.2 & Viene verificato che si possa modificare il colore di un Elemento\ped{g} SVG  & RF 7.40.2\\
			\hline
			TS 7.43 & Viene verificato che si possa eliminare un Elemento\ped{g} & RF 7.43 \\
			\hline
			TS 7.46 & Viene verificato che si possa ruotare un Elemento\ped{g} & RF 7.46 \\
			\hline			 
			TS 10 & Viene verificato il poter passare in modalità modifica di una presentazione da mobile & RF 10\\
			\hline
			TS 10.1 & Viene verificato che si possa editare da mobile il testo all'interno di un Frame\ped{g} & RF 10.1\\
			\hline
			TS 10.4 & Viene verificato che si possa modificare da mobile il testo presente all'interno di un Frame\ped{g} & RF 10.4\\
			\hline
			TS 10.5 & Viene verificato che si possa assegnare un Bookmark\ped{g} ad un Frame\ped{g} da mobile & RF 10.5\\
			\hline
			TS 10.8 & Viene verificato che si possa rimuovere un Bookmark\ped{g} ad un Frame\ped{g} da mobile & RF 10.8\\
			\hline
			TS 13 & Viene verificato che si possa caricare un File\ped{g} media presente in locale nel Server\ped{g} & RF 13\\
			\hline
			TS 16 & Viene verificato che si possano eliminare dal Server\ped{g} i File\ped{g} media caricati & RF 16\\
			\hline
			TS 17 & Viene verificato che si possano rinominare i File\ped{g} media presenti sul Server\ped{g} & RF 17\\
			\hline
			TS 19 & Viene verificato che si possano rinominare le presentazioni salvate nel Server\ped{g} & RF 19\\
			\hline
			TS 25 & Viene verificato che si possano rinominare le Infografiche\ped{g} salvate nel Server\ped{g} & RF 25\\
			\hline
			TS 31 & Viene verificato che si possa eliminare dal Server\ped{g} un'Infografica\ped{g} creata & RF 31\\
			\hline
			TS 34 & Viene verificato che si possa eliminare dal Server\ped{g} una presentazione creata & RF 34\\
			\hline
			TS 35 & Viene verificato che si possano visualizzare le presentazioni salvate sul Server\ped{g} & RF 35\\
			\hline
			TS 36 & Viene verificato che si possano visualizzare le Infografiche\ped{g} salvate sul Server\ped{g} & RF 36\\
			\hline
			TS 37 & Viene verificato che si possano visualizzare i File\ped{g} media salvati sul Server\ped{g} & RF 37\\						
			\hline
			TS 43 & Viene verificato che si possa modificare la propria password di accesso al sistema & RF 43\\
			\hline
			TS 46 & Viene verificato che si possa scaricare in locale un'Infografica\ped{g} creata sul Server\ped{g} & RF 46\\
			\hline
			TS 49 & Viene verificato che si possa salvare in locale una presentazione creata sul Server\ped{g} & RF 49\\
			\hline
			TS 52 & Viene verificato che si possa rimuovere una presentazione salvata in locale & RF 52\\
			\hline
			TS 55 & Viene verificato che si possa annullare una modifica non voluta & RF 55\\
			\hline
			TS 58 & Viene verificato che si possa ripristinare una modifica annullata precedentemente  & RF 58\\
			\hline
			TS 61.1 & Viene verificato che si possa eseguire una presentazione in modalità manuale & RF 61.1\\
			\hline
			TS 61.1.1 & Viene verificato che durante la presentazione si possa passare al Frame\ped{g} successivo o al precedente & RF 61.1.1\\
			\hline
			TS 61.1.4 & Viene verificato che si possa selezionare un Elemento scelta\ped{g} se presente nel Frame\ped{g} & RF 61.1.4\\
			\hline
			TS 61.1.7 & Viene verificato che si possa passare al Frame\ped{g} con Bookmark\ped{g} successivo o precedente & RF 61.1.7\\
			\hline
			TS 61.1.10 & Viene verificato che si possa passare da un Frame\ped{g} visualizzato al suo Frame\ped{g} contenitore & RF 61.1.10\\
			\hline
			TS 61.1.13 & Viene verificato che si possa zoomare in una parte qualsiasi del Frame\ped{g} & RF 61.1.13\\
			\hline
			TS 61.1.16.1 & Viene verificato che si possa iniziare l’esecuzione di un video/audio all'interno di un Frame\ped{g} & RF 61.1.16.1\\
			\hline
			TS 61.1.16.4 & Viene verificato che si possa Sospendere\ped{g} e poi riprendere l'esecuzione di un video/audio all'interno di un Frame\ped{g} & RF 61.1.16.4\\
			\hline
			TS 61.1.16.7 & Viene verificato che si possa eseguire un video/audio da un punto qualsiasi dello stesso & RF 61.1.16.7\\
			\hline
			TS 61.1.16.10 & Viene verificato che si possa interrompere l'esecuzione di un video/audio & RF 61.1.16.10\\
			\hline
			TS 61.4 & Viene verificato che si possa eseguire una presentazione in modalità automatica & RF 61.4\\
			\hline
			TS 61.4.1 & Viene verificato che si possa chiudere una presentazione in esecuzione automatica & RF 61.4.1\\
			\hline
			TS 61.4.4 & Viene verificato che si possa Sospendere\ped{g} e riavviare una presentazione in esecuzione automatica & RF 61.4.4\\
			\hline
			TS 61.4.7 & Viene verificato che si possa impostare la velocità di riproduzione della presentazione & RF 61.4.7\\
			\hline
			TS 61.4.10.1 & Viene verificato che si possa Sospendere\ped{g} la riproduzione di un video/audio e riprenderla successivamente & RF 61.4.10.1\\
			\hline
			TS 61.4.10.4 & Viene verificato che si possa riprodurre un File\ped{g} video/audio da un punto qualsiasi dello stesso & RF 61.4.10.4\\
			\hline
			TS 61.4.10.7 & Viene verificato che si possa saltare la riproduzione di un video/audio nel Frame\ped{g} visualizzato & RF 61.4.10.7\\
			\hline
			TS 61.4.10.10 & Viene verificato che si possa visualizzare la riproduzione del video a schermo intero & RF 61.4.10.10\\
			\hline
			TS 61.7 & Viene verificato che si possa passare da presentazione automatica a presentazione manuale e viceversa & RF 61.7, RF 61.10\\
			\hline
			TS 64 & Viene verificato che si possa effettuare il Logout\ped{g} dal Server\ped{g} & RF 64\\
			\hline
			TS 67.1 & Viene verificato che l'amministratore possa inserire dei Template\ped{g} di presentazioni & RF 67.1\\
			\hline
			TS 67.4 & Viene verificato che l'amministratore possa inserire Template\ped{g} di Infografiche\ped{g} & RF 67.4\\
			\hline
			TS 67.7 & Viene verificato che l'amministratore possa inserire elementi\ped{g} grafici & RF 67.7\\
			\hline
			TS 67.10 & Viene verificato che l'amministratore possa eliminare un Template\ped{g} & RF 67.10\\
			\hline
			TS 67.13 & Viene verificato che l'amministratore possa ripristinare un Template\ped{g} eliminato & RF 67.13\\
			\hline
			TS 70.1 & Viene verificato che si possa selezionare una presentazione da cui produrre l'Infografica\ped{g}  & RF 70.1\\
			\hline
			TS 70.4 & Viene verificato che si possa selezionare Template\ped{g} di Infografica\ped{g} & RF 70.4\\
			\hline
			TS 70.5 & Viene verificato che si possano selezionare gli elementi\ped{g} dell'Infografica\ped{g}  & RF 70.5\\
			\hline
			TS 70.10 & Viene verificato che si possa passare in modalità modifica di un'Infografica\ped{g} & RF 70.10\\
			\hline
			TS 70.10.1 & Viene verificato che si possa modificare un Elemento\ped{g} di un'Infografica\ped{g} & RF 70.10.1\\
			\hline
			TS 70.10.1.1 & Viene verificato che si possano modificare le dimensioni di un Elemento\ped{g} grafico & RF 70.10.1.1\\
			\hline
			TS 70.10.1.4 & Viene verificato che si possa modificare un Elemento\ped{g} testuale  & RF 70.10.1.4\\
			\hline
			TS 70.10.1.4.1 & Viene verificato che si possa modificare il Font\ped{g} di un Elemento\ped{g} testuale & RF 70.10.1.4.1\\
			\hline
			TS 70.10.1.4.4 & Viene verificato che si possa modificare la dimensione del carattere di un Elemento\ped{g} testuale & RF 70.10.1.4.4\\
			\hline
			TS 70.10.1.4.7 & Viene verificato che si possa modificare lo stile del testo & RF 70.10.1.4.7\\
			\hline
			TS 70.10.1.4.10 & Viene verificato che si possa modificare il colore della scritta del testo & RF 70.10.1.4.10\\
			\hline
			TS 70.10.1.4.13 & Viene verificato che si possa modificare il colore di sfondo del testo & RF 70.10.1.4.13\\
			\hline
			TS 70.10.1.7 & Viene verificato che si possa cambiare la posizione di un Elemento\ped{g} & RF 70.10.1.7\\
			\hline
			TS 70.10.4 & Viene verificato che si possa rimuovere lo sfondo dell'Infografica\ped{g} & RF 70.4.4\\
			\hline
			TS 70.10.7 & Viene verificato che si possa inserire uno sfondo nell'Infografica\ped{g} & RF 70.10.7\\
			\hline
			TS 70.10.10 & Viene verificato che si possa inserire un Elemento\ped{g} immagine nell'Infografica\ped{g} & RF 70.10.10\\
			\hline
			TS 70.10.13 & Viene verificato che si possa inserire del testo nell'Infografica\ped{g} & RF 70.10.13\\
			\hline
			TS 70.10.16 & Viene verificato che si possa inserire un Frame\ped{g} nella sua interezza presente nella presentazione & RF 70.10.16\\
			\hline
			TS 70.10.19 & Viene verificato che si possano eliminare elementi\ped{g} grafici o testuali & RF 70.10.19\\
			\hline
			TS 70.13 & Viene verificato che si possa salvare l'Infografica\ped{g} nel suo spazio & RF 70.13\\
			\hline
			TS 70.16 & Viene verificato che si possa annullare e ripristinare una modifica appena effettuata & RF 70.16\\
			\hline
			TS 70.19 & Viene verificato che si possa esportare un'Infografica\ped{g} in formato stampabile & RF 70.19\\
			\hline
			TS 73 & Viene verificato che si possa creare un'Infografica\ped{g} & RF 73\\
\end{longtable}

		\begin{longtable} [c]{| p{7cm} |p{4cm}|}
			\caption{
			Descrizione dei test di sistema
			per i Requisiti\ped{g} di Qualità e Vincoli \label{tab:verReqQualVinc}}\\
			 \hline
			 \textbf{Descrizione} & \textbf{Requisito} \\
			 \hline
			 \endfirsthead
			 \hline
			 \textbf{Descrizione} & \textbf{Requisito} \\
			 \hline
			\endhead
			 \hline
			 \endfoot
			 \hline
			 \endlastfoot
			Viene verificato che ogni funzionalità dell'applicazione sia documentata & RQ\ped{g} 1, 7\\
			\hline
			Viene verificato che sia disponibile un tutorial per la creazione delle presentazioni & RQ\ped{g} 4\\
			\hline
			Viene verificato che sia disponibile la documentazione sui test eseguiti & RQ\ped{g} 10\\
			\hline
			Viene verificato che il sistema offra la possibilità di eseguire offline le presentazioni & RQ\ped{g} 13\\
\end{longtable}
}
\newpage
\subsection{Test d'integrazione}{
	Vengono descritti i test di integrazione che permettono di verificare la corretta integrazione e comunicazione tra parti distinte di sistema.	\\
	Verrà utilizzata la strategia di integrazione incrementale per una più semplice individuazione dei difetti, per assemblare il sistema in passi di integrazione reversibili e per avere corretti flussi di controllo da parti “consumatore” verso parti “produttore”. In particolare per limitare il numero di stub richiesti verrà adottato un approccio Bottom-up integrando prima le parti con minore dipendenza funzionale. \\
	Di seguito viene riportato il diagramma per chiarire l'albero dei test d'integrazione.	\\
	
	\begin{figure}[H]
	  \centering
	    \includegraphics[scale=0.35]{\imgs {IntegrazioneDiagramma}.pdf}
	  \caption{Sequenza d’integrazione delle componenti}
	  \label{fig:SeqIntComp}
	\end{figure}
	
	\begin{longtable} [c]{| p{2cm} | p{6cm} |p{3cm} | p{2cm} |}
			\caption{Descrizione dei test di Integrazione \label{tab:verReqInteg}}\\
		 \hline
		 \textbf{Test} & \textbf{Descrizione} & \textbf{Componenti aggiuntive}& \textbf{Stato} \\
		 \hline
		 \endfirsthead
		 \hline
		 \textbf{Test} & \textbf{Descrizione} & \textbf{Componenti aggiuntive}& \textbf{Stato} \\
		 \hline
			\endhead
		 \hline
		 \endfoot
		 \hline
		 \endlastfoot
			TIA1 & Si verifica che NodeApi si integri correttamente con le classi che compongono AccessControll & NodeApi, AccessControll & N.E.\\
			\hline
			TIA2 & Si verifica che il Loader interagisca correttamente con NodeApi e AccessControll  & Loader & N.E.\\
			\hline
			TIA3 & Si verifica che Registration interagisca correttamente con le NodeApi, Loader e AccessControll & Registration & N.E.\\
			\hline
			TIA4 & Si verifica che Authentication interagisca correttamente con le NodeApi, Loader e AccessControll & Authentication & N.E.\\
			\hline
			TIC2 & Si verifica che i componenti di SlideShowElements dialoghino correttamente con serverRelations &  & N.E.\\
			\hline
			TIC3 & Si verifica che le classi che compongono InseritEditRemove interagiscano correttamente con SlideShowElements e serverRelations & SlideshowActions & N.E.\\
			\hline
			TIC4 & Si verifica che le classi che compongono Command si integrino con InsertEditRemove &  & N.E.\\
			\hline
			TID1 & Si verifica che il Controller  ed il Model dialoghino correttamente & Controller & N.E.\\
			\hline
			TID2 & Si verifica che la View si integri correttamente con il Controller & View & N.E.\\
		\end{longtable}
}
}
\newpage
\subsection{Test d'unità}
		
	\begin{longtable} [c]{| p{2cm} | p{6cm} |p{3cm} | p{2cm} |}
				\caption{Descrizione dei test di Unità per il Server \label{tab:verTestUnit}}\\
		 \hline
		 \textbf{Test} & \textbf{Descrizione} & \ \textbf{Metodi} & \ \textbf{Stato} \\
		 \hline
		 \endfirsthead
		 \hline
		 \textbf{Test} & \textbf{Descrizione} & \ \textbf{Metodi} & \textbf{Stato} \\
		 \hline
				\endhead
		 \hline
		 \endfoot
		 \hline
		 \endlastfoot
		 TU1 & viene verificato l'inserimento di un utente nel database in seguito ad una chiamata al metodo del server con un utente non ancora registrato & POST account/register & success \\
		 TU2 & viene verificato che il server non inserisce in database un utente gia registrato chiamando il metodo opportuno del server con un utente inserito & POST account/register & success \\
		 TU3 & viene verificato il metodo di autenticazione del server attraverso una chiamata ad esso con un utente inserito e verificandone il token ritornato  & GET account/authenticate & success \\
		 TU4 & viene verificato il metodo di autenticazione del server attraverso una chiamata ad esso con un utente non inserito e verificando la risposta del server & GET account/authenticate & success \\
		 TU5 & viene verificato il metodo di autenticazione del server attraverso una chiamata ad esso con un utente inserito nel database ma con il parametro password errato e verificando la risposta del server & GET account/authenticate & success \\
		 TU6 & viene verificato che il server modifica la password salvata in database di un utente attraverso una chiamata con parametri corretti rispetto ad un utente inserito & POST account/changepassword & success \\
		 TU7 & viene verificato che il server non modifica la password salvata in database di un utente attraverso una chiamata con parametro un utente non inserito & POST account/changepassword & success \\
		 TU8 & viene verificato che il server non modifica la password salvata in database di un utente attraverso una chiamata con parametro password errato rispetto all'utente indicato  & POST account/changepassword & success \\
		 TU9 & viene inserita una presentazione in database e viene verificato che una chiamata al metodo del server ritorni un array il cui primo elemento è corretto rispetto alla presentazione inserita & GET private/api/presentations & success \\
		 TU10 & viene chiamato il metodo del server per l'inserimento di una nuova presentazione, si verifica che la presentazione sia stata inserita e che abbia il nome corretto & POST private/api/presentations/new/[newpres] & success \\
		 TU11 & viene chiamato il metodo del server per l'inserimento di una nuova presentazione come copia di una gia esistente e se ne verifica il corretti inserimento in database & POST private/api/presentations/new/[oldpres]/[newpres] & success \\
		 TU12 & viene inserita una presentazione in database e viene chiamato il metodo del server per il recupero della presentazione, si verifica il corretto recupero nella risposta  & GET private/api/presentations/[presentation] & success \\
		 TU13 & viene inserita una presentazione in database, poi viene chiamato il metodo del server per la eliminazione della presentazione, si verifica la cancellazione della presentazione in database & DELETE private/api/presentations/[presentation] & success \\
		 TU14 & viene inserita una presentazione in database e viene chiamato il metodo del server per la rinomina della presentazione verificando che il nome sia cambiato con quello indicato  & RENAME private/api/presentations/[oldname]/rename/[new] & success \\
		 TU15 & viene inserito un elemento in una presentazione in database, viene chiamato il metodo del server con i parametri corretti e si osserva la cancellazione dell'elemento dalla presentazione & DELETE private/api/presentations/[presname]/delete/[type]/[id] & success \\
		 TU16 & viene chiamato il metodo del server per l'inserimento di un elemento in una presentazione, viene verificato l'inserimento dell'elemento in database & POST private/api/presentations/[presname]/element & success \\
		 TU17 & viene verificata la modifica dell'elemento paths di una presentazione in database attraverso la chiamata al server con dei parametri di esempio & PUT private/api/presentations/[presname]/paths & success \\
		 TU18 & viene inserito un elemento in una presentazione in database, viene chiamato il metodo del server per la modifica di questo elemento con dei valori di esempio, si verifica la modifica dei valori in database & PUT private/api/presentations/[presname]/element & success \\
		 TU19 & viene inserita una immagine e viene chiamato il metodo del server verificando che ritorni un'array con il primo elemento avente il nome dell'immagine inserita & GET private/api/files/image & success \\
		 TU20 & viene inserito un file immagine, viene chiamato il metodo del server per la cancellazione di quel file e se ne verifica la cancellazione dalla opportuna cartella del server & DELETE private/api/files/[image] & success \\
		 TU21 & viene chiamato il metodo del server per l'inserimento di un file immagine e si verifica l'inserimento del file nella cartella destinazione sul server & POST private/api/files/[image] & success \\
		 TU22 & viene inserito un file immagine, viene chiamato il metodo del server per la rinomina del file, si verifica che il file nella cartella del server sia stato rinominato & POST private/api/files/[image]/[newname] & success \\
		 TU23 & viene inserita una immagine con valori degli attributi height e width noti, viene chiamato il metodo del server per questa immagine e si verificano i valori ritornati con quelli noti  & GET private/api/files/sizeImage/[image] & success \\
		 TU24 & viene inserito un file audio e viene chiamato il metodo del server verificando che ritorni un'array con il primo elemento avente il nome del file inserito & GET private/api/files/audio & success \\
		 TU25 & viene inserito un file audio, viene chiamato il metodo del server per la cancellazione di quel file e se ne verifica la cancellazione dalla opportuna cartella del server & DELETE private/api/files/[audio] & success \\
		 TU26 & viene chiamato il metodo del server per l'inserimento di un file audio e si verifica l'inserimento del file nella cartella destinazione sul server & POST private/api/files/[audio] & success \\
		 TU27 & viene inserito un file audio, viene chiamato il metodo del server per la rinomina del file, si verifica che il file nella cartella del server sia stato rinominato & POST private/api/files/[audio]/[newname] & success \\
		 TU28 & viene inserito un file video e viene chiamato il metodo del server verificando che ritorni un'array con il primo elemento avente il nome del file inserito & GET private/api/files/video & success \\
		 TU29 & viene inserito un file video, viene chiamato il metodo del server per la cancellazione di quel file e se ne verifica la cancellazione dalla opportuna cartella del server & DELETE private/api/files/[video] & success \\
		 TU30 & viene chiamato il metodo del server per l'inserimento di un file video e si verifica l'inserimento del file nella cartella destinazione sul server & POST private/api/files/[video] & success \\
		 TU31 & viene inserito un file video, viene chiamato il metodo del server per la rinomina del file, si verifica che il file nella cartella del server sia stato rinominato & POST private/api/files/[video]/[newname] & success \\
 		\end{longtable}	
 		
		\begin{longtable} [c]{| p{2cm} | p{6cm} |p{3cm} | p{2cm} |}
						\caption{Descrizione dei test di Unità per ServerRelation; classe: fileServerRelation \label{tab:verTestUnit}}\\
		 \hline
		 \textbf{Test} & \textbf{Descrizione} & \ \textbf{Metodi} & \ \textbf{Stato} \\
		 \hline
		 \endfirsthead
		 \hline
		 \textbf{Test} & \textbf{Descrizione} & \ \textbf{Metodi} & \textbf{Stato} \\
		 \hline
						\endhead
		 \hline
		 \endfoot
		 \hline
		 \endlastfoot
		 TU32 & viene verificato che il ritorno della funzione sia corrispondente ai file immagini dell'utente nel server & getMetasImages() & success \\
		 TU33 & viene verificato che il ritorno della funzione sia corrispondente ai file audio dell'utente nel server & getMetasAudios() & success \\
		 TU34 & viene verificato che il ritorno della funzione sia corrispondente ai file video dell'utente nel server & getMetasVideos() & success \\
		 TU35 & viene verificato che il ritorno della funzione sia corretto rispetto ai valori di width e height del file immagine inserito nel server & getImageDimension(img:string) & success \\
		 TU36 & viene verificata l'eliminazione di un file immagine con la chiamata della funzione su un file immagine inserito nel server & deleteImage(img:string) & success \\
		 TU37 & viene verificata l'eliminazione di un file audio con la chiamata della funzione su un file audio inserito nel server & deleteAudio(img:string) & success \\
		 TU38 & viene verificata l'eliminazione di un file video con la chiamata della funzione su un file video inserito nel server & deleteVideo(img:string) & success \\
		 TU39 & viene verificata la rinominazione di un file immagine inserito nel server attraverso la chiamata alla funzione con i parametri corretti & renameImage(old:string,new:string) & success \\
		 TU40 & viene verificata la rinominazione di un file audio inserito nel server attraverso la chiamata alla funzione con i parametri corretti & renameAudio(old:string,new:string) & success \\
		 TU41 & viene verificata la rinominazione di un file video inserito nel server attraverso la chiamata alla funzione con i parametri corretti & renameVideo(old:string,new:string) & success \\
		 \end{longtable}
 \begin{longtable} [c]{| p{2cm} | p{6cm} |p{3cm} | p{2cm} |}
	 						\caption{Descrizione dei test di Unità per ServerRelation; classe: serverRelation \label{tab:verTestUnit}}\\
 		 \hline
 		 \textbf{Test} & \textbf{Descrizione} & \ \textbf{Metodi} & \ \textbf{Stato} \\
 		 \hline
 		 \endfirsthead
 		 \hline
 		 \textbf{Test} & \textbf{Descrizione} & \ \textbf{Metodi} & \textbf{Stato} \\
 		 \hline
	 						\endhead
 		 \hline
 		 \endfoot
 		 \hline
 		 \endlastfoot
 		 TU42 & viene inserita una presentazione in database e viene verificato che il ritorno della funzione sia consistente che le presentazioni presenti in database & getPresentationsMeta() & success \\
 		 TU43 & viene chiamata la funzione per l'inserimento di una nuova presentazione, viene verificato l'inserimento della presentazione in database & newPresentation(name:string) & success \\
 		 TU44 & viene chiamata la funzione per l'inserimento di una nuova presentazione come copia di una gia esistente, viene verificato l'inserimento della presentazione in database come copia della esistente & newCopyPresentation(name:string) & success \\
 		 TU45 & viene inserita una presentazione in database e viene chiamato il metodo del server per il recupero della presentazione, si verifica il corretto ritorno della funzione & getPresentation(name:string) & success \\
 		 TU46 & viene inserita una presentazione in database e viene chiamato il metodo di eliminazione della presentazione dell'utente, si verifica la cancellazione della presentazione & deletePresentation(name:string) & success \\
 		 TU47 & viene inserita una presentazione in database e viene chiamato il metodo di rinominazione della presentazione dell'utente, si verifica la rinominazione della presentazione & renamePresentation(old:string,new:string) & success \\
 		 TU48 & viene chiamato il metodo del server per l'inserimento di un elemento in una presentazione, viene verificato l'inserimento dell'elemento in database & newElement(presentation:string,onj:object,callback:function) & success \\
 		 TU49 & viene inserito un elemento in una presentazione in database, viene chiamato il metodo con i parametri corretti e si osserva la modifica dell'elemento dalla presentazione & updateElement(presentation:string,onj:object,callback:function) & success \\
 		 TU50 & viene inserito un elemento in una presentazione in database, viene chiamato il metodo con i parametri corretti e si osserva la cancellazione dell'elemento dalla presentazione & deleteElement(presentation:strin,type:string,id:string) & success \\
 		 TU51 & viene chiamato il metodo per la modifica del campo paths di una presentazione e si osserva la modifica del campo dati della presentazioni in database & updatePaths(presentation:string,obj:object) & success \\
 		 \end{longtable}
  \begin{longtable} [c]{| p{2cm} | p{6cm} |p{3cm} | p{2cm} |}
		 	 						\caption{Descrizione dei test di Unità per ServerRelation; classe: Loader \label{tab:verTestUnit}}\\
	  		 \hline
	  		 \textbf{Test} & \textbf{Descrizione} & \ \textbf{Metodi} & \ \textbf{Stato} \\
	  		 \hline
	  		 \endfirsthead
	  		 \hline
	  		 \textbf{Test} & \textbf{Descrizione} & \ \textbf{Metodi} & \textbf{Stato} \\
	  		 \hline
		 	 						\endhead
	  		 \hline
	  		 \endfoot
	  		 \hline
	  		 \endlastfoot
 		 TU52 & vengono chiamati i metodi addInsert(), addUpdate e addDelete() su un oggetto mook per la simulazione di altre classi del model per il recupero degli elementi della presentazione associati, si verifica lo stato della presentazione alle diverse chiamate del metodo  & update() & success \\
 		 \end{longtable}
 		 
	 \begin{longtable} [c]{| p{2cm} | p{6cm} |p{3cm} | p{2cm} |}
		 			\caption{Descrizione dei test di Unità \label{tab:verTestUnit}}\\
	 		 \hline
	 		 \textbf{Test} & \textbf{Descrizione} & \ \textbf{Stato} \\
	 		 \hline
	 		 \endfirsthead
	 		 \hline
	 		 \textbf{Test} & \textbf{Descrizione} & \textbf{Stato} \\
	 		 \hline
		 			\endhead
	 		 \hline
	 		 \endfoot
	 		 \hline
	 		 \endlastfoot
          TU53 & Si verifica che le classi Services vengano istanziate correttamente & Success \\
          \hline
          TU54 & Si verifica che HeaderController venga istanziato correttamente & Success \\
          \hline 
          TU55 & Si verifica che AuthenticationController venga istanziato correttamente & Success\\
          \hline
          TU56 & Si verifica che HomeController venga istanziato correttamente & Success \\
          \hline
          TU57 & Si verifica che ProfileController venga istanziato correttamente & Success \\
          \hline
          TU58 & Si verifica che ExecutionController venga istanziato correttamente & Success \\
          \hline
          TU59 & Si verifica che EditController venga istanziato correttamente & Success \\
		 		\end{longtable}
		 		
		   \begin{longtable} [c]{| p{2cm} | p{6cm} |p{3cm} | p{2cm} |}
		 		 	 						\caption{Descrizione dei test di Unità per SlideShowElements \label{tab:verTestUnit}}\\
		 	  		 \hline
		 	  		 \textbf{Test} & \textbf{Descrizione} & \ \textbf{Metodi} & \ \textbf{Stato} \\
		 	  		 \hline
		 	  		 \endfirsthead
		 	  		 \hline
		 	  		 \textbf{Test} & \textbf{Descrizione} & \ \textbf{Metodi} & \textbf{Stato} \\
		 	  		 \hline
		 		 	 						\endhead
		 	  		 \hline
		 	  		 \endfoot
		 	  		 \hline
		 	  		 \endlastfoot
		  		 TU60 & Richiama il costruttore di Text(...) e verifica che sia stato creato un oggetto di tipo text  & slideShowElements::Text() & success \\
		  		 TU61 & Richiama il costruttore di Frame(...) e verifica che sia stato creato un oggetto di tipo frame & slideShowElements::Frame() & success \\
		  		 TU62 & Richiama il costruttore di Image(...) e verifica che sia stato creato un oggetto di tipo image & slideShowElements::Image() & success \\
		  		 TU63 & Richiama il costruttore di Audio(...) e verifica che sia stato creato un oggetto di tipo audio  & slideShowElements::Audio() & success \\
		  		 TU64 & Richiama il costruttore di Video(...) e verifica che sia stato creato un oggetto di tipo video  & slideShowElements::Video() & success \\
		  		 TU65 & Richiama il costruttore di Background(...) e verifica che sia stato creato un oggetto di tipo background  & slideShowElements::Background() & success \\
		  		 
		  		 \end{longtable}
		  		   \begin{longtable} [c]{| p{2cm} | p{6cm} |p{3cm} | p{2cm} |}
		 		 	 						\caption{Descrizione dei test di Unità per InsertEditRemove \label{tab:verTestUnit}}\\
		 	  		 \hline
		 	  		 \textbf{Test} & \textbf{Descrizione} & \ \textbf{Metodi} & \ \textbf{Stato} \\
		 	  		 \hline
		 	  		 \endfirsthead
		 	  		 \hline
		 	  		 \textbf{Test} & \textbf{Descrizione} & \ \textbf{Metodi} & \textbf{Stato} \\
		 	  		 \hline
		 		 	 						\endhead
		 	  		 \hline
		 	  		 \endfoot
		 	  		 \hline
		 	  		 \endlastfoot
		  		 TU66 & Richiama il costruttore di InsertEditRemove() e verifica che venga istanziato un oggetto di tale classe  & slideShowActions::InsertEditRemove() & success \\
		  		 TU67 & Richiama il costruttore di InsertEditRemove() e verifica che sia stata generata una sola istanza di tale classe  & slideShowActions::InsertEditRemove() & success \\
		 		 TU68 & Richiama il InsertEditRemove()::insertText e verifica che venga inserito nella presentazione un oggetto Text & slideShowActions::InsertEditRemove()::insertText(...) & success \\
		 		 TU69 & Richiama il InsertEditRemove()::insertFrame e verifica che venga inserito nella presentazione un oggetto Frame  & slideShowActions::InsertEditRemove()::insertFrame(...) & success \\
		 		 TU70 & Richiama il InsertEditRemove()::insertImage e verifica che venga inserito nella presentazione un oggetto Image  & slideShowActions::InsertEditRemove()::insertImage(...) & success \\
		 		 TU71 & Richiama il InsertEditRemove()::insertSVG e verifica che venga inserito nella presentazione un oggetto SVG  & slideShowActions::InsertEditRemove()::insertSVG(...) & success \\
		 		 TU72 & Richiama il InsertEditRemove()::insertAudio e verifica che venga inserito nella presentazione un oggetto Audio  & slideShowActions::InsertEditRemove()::insertAudio(...) & success \\
		 		 TU73 & Richiama il InsertEditRemove()::insertVideo e verifica che venga inserito nella presentazione un oggetto Video  & slideShowActions::InsertEditRemove()::insertVideo(...) & success \\
		 		 TU74 & Richiama il InsertEditRemove()::editPosition(...) su ogni elemento della presentazione e verifica che ogni elemento venga modificato appropriatamente, controlla che venga ritornato l'oggetto precedente alla modifica   & slideShowActions::InsertEditRemove()::editPosition(...) & success \\
		 		 TU75 & Richiama il InsertEditRemove()::editSize(...) su ogni elemento della presentazione e verifica che ogni elemento venga modificato appropriatamente, controlla che venga ritornato l'oggetto precedente alla modifica   & slideShowActions::InsertEditRemove()::editSize(...) & success \\
		 		 TU76 & Richiama il InsertEditRemove()::editRotation(...) su ogni elemento della presentazione e verifica che ogni elemento venga modificato appropriatamente, controlla che venga ritornato l'oggetto precedente alla modifica   & slideShowActions::InsertEditRemove()::editRotation(...) & success \\
		 		 TU77 & Richiama il InsertEditRemove()::editBackground(...) su ogni elemento della presentazione e verifica che ogni elemento venga modificato appropriatamente, controlla che venga ritornato l'oggetto precedente alla modifica   & slideShowActions::InsertEditRemove()::editBackground(...) & success \\
		 		 TU78 & Richiama il InsertEditRemove()::editColor(...) su ogni elemento della presentazione e verifica che ogni elemento venga modificato appropriatamente, controlla che venga ritornato l'oggetto precedente alla modifica   & slideShowActions::InsertEditRemove()::editColor(...) & success \\
		 		 TU79 & Richiama il InsertEditRemove()::editShape(...) su ogni elemento della presentazione e verifica che ogni elemento venga modificato appropriatamente, controlla che venga ritornato l'oggetto precedente alla modifica   & slideShowActions::InsertEditRemove()::editPosition(...) & success \\
		 		 TU80 & Richiama il InsertEditRemove()::removeText(...), controlla che venga ritornato l'oggetto precedente alla modifica & slideShowActions::InsertEditRemove()::removeText(...) & success \\
		 		 TU81 & Richiama il InsertEditRemove()::removeFrame(...), controlla che venga ritornato l'oggetto precedente alla modifica & slideShowActions::InsertEditRemove()::removeFrame(...) & success \\
		 		 TU82 & Richiama il InsertEditRemove()::removeImage(...), controlla che venga ritornato l'oggetto precedente alla modifica & slideShowActions::InsertEditRemove()::removeImage(...) & success \\
		 		 TU83 & Richiama il InsertEditRemove()::removeAudio(...), controlla che venga ritornato l'oggetto precedente alla modifica & slideShowActions::InsertEditRemove()::removeAudio(...) & success \\
		 		 TU84 & Richiama il InsertEditRemove()::removeVideo(...), controlla che venga ritornato l'oggetto precedente alla modifica & slideShowActions::InsertEditRemove()::removeVideo(...) & success \\
		 		 TU85 & Richiama il InsertEditRemove()::removeBackground(...), controlla che venga ritornato l'oggetto precedente alla modifica & slideShowActions::InsertEditRemove()::removeBackground(...) & success \\
		 		 TU86 & Richiama il InsertEditRemove()::removeSVG(...), controlla che venga ritornato l'oggetto precedente alla modifica & slideShowActions::InsertEditRemove()::removeSVG(...) & success \\
		 		 
		  		 \end{longtable}
 				 