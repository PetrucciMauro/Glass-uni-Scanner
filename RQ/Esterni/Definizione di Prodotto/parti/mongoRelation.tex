\subsection{serverRelation}{

	\subsubsection{MongoRelation}{
	
	\begin{figure}[H]
		\includegraphics[scale=0.8]{\imgs {mongoRelation}.pdf}
		\label{fig:nodeAPI}
		\caption{Diagramma classe Model::serverRelation::mongoRelation::MongoRelation}
	\end{figure}
	
		\textbf{Metodi:}
		\begin{itemize}
		\item \textbf{getPresentationsMeta() : object}
			\begin{itemize}
			\item \textbf{Accessibilit\`{a}:} public
			\item \textbf{Descrizione:} effettua una chiamata asincrona verso il Server nodeJs al servizio "/private/api/presentations GET" ritornando l'oggetto con le informazioni sulle presentazioni create dall'utente
			\item \textbf{Tipo di ritorno:} object
			\end{itemize}
		
		\item \textbf{newPresentation(name : string) : void}
			\begin{itemize}
			\item \textbf{Accessibilit\`{a}:} public
			\item \textbf{Descrizione:} effettua una chiamata asincrona verso il Server nodeJs al servizio "/private/api/presentations/new/[name] POST" per creare una nuova presentazione sul database MongoDB
			\item \textbf{Tipo di ritorno:} object
			\end{itemize}
			
		\item \textbf{newCopyPresentation(nameOldPresentation : string, nameNewPresentation : string) : void}
			\begin{itemize}
			\item \textbf{Accessibilit\`{a}:} public
			\item \textbf{Descrizione:} effettua una chiamata asincrona verso il Server nodeJs al servizio "/private/api/presentations/new/[nameOldPresentation]/[nameNewPresentation] POST" per creare una nuova presentazione sul database MongoDB
			\item \textbf{Tipo di ritorno:} object
			\end{itemize}
			
		\item \textbf{getPresentation(namePresentation : string) : object}
			\begin{itemize}
			\item \textbf{Accessibilit\`{a}:} public
			\item \textbf{Descrizione:} effettua una chiamata asincrona verso il Server nodeJs al servizio "/private/api/presentations/[namePresentation] GET" per ricevere la presentazione [namePresentation] dell'utente
			\item \textbf{Tipo di ritorno:} object
			\end{itemize}
			
		\item \textbf{deletePresentation(namePresentation : string) : bool}
			\begin{itemize}
			\item \textbf{Accessibilit\`{a}:} public
			\item \textbf{Descrizione:} effettua una chiamata asincrona verso il Server nodeJs al servizio "/private/api/presentations/[namePresentation] DELETE" per eliminare la presentazione [namePresentation] creata dall'utente dal database MongoDB
			\item \textbf{Tipo di ritorno:} bool
			\end{itemize}
			
		\item \textbf{renamePresentation(name : string, newName : string) : bool}
			\begin{itemize}
			\item \textbf{Accessibilit\`{a}:} public
			\item \textbf{Descrizione:} effettua una chiamata asincrona verso il Server nodeJs al servizio "/private/api/presentations/[name]/rename/[newName] POST" per rinominare la presentazione [name] dell'utente in [newName] 
			\item \textbf{Tipo di ritorno:} bool
			\end{itemize}
			
		\item \textbf{updateElement(namePresentation: string, element : object, callback) }
			\begin{itemize}
			\item \textbf{Accessibilit\`{a}:} public
			\item \textbf{Descrizione:} effettua una chiamata asincrona verso il Server nodeJs al servizio "/private/api/presentations/[namePresentation]/element PUT" per aggiornare l'elemento passato come parametro, esegue la funzione callback
			\item \textbf{Tipo di ritorno:} bool
			\end{itemize}
			
		\item \textbf{deleteElement(namePresentation: string, typeObj : string, idElement : string, callback) }
			\begin{itemize}
			\item \textbf{Accessibilit\`{a}:} public
			\item \textbf{Descrizione:} effettua una chiamata asincrona verso il Server nodeJs al servizio "/private/api/presentations/[namePresentation]/[typeObj]/[idElement] DELETE" per eliminare l'elemento con identificativo [idElement] dalla presentazione nel database MongoDB, esegue la funzione callback
			\item \textbf{Tipo di ritorno:} bool
			\end{itemize}
			
		\item \textbf{newElement(namePresentation: string, element : object, callback)}
			\begin{itemize}
			\item \textbf{Accessibilit\`{a}:} public
			\item \textbf{Descrizione:} effettua una chiamata asincrona verso il Server nodeJs al servizio "/private/api/presentations/[namePresentation]/element POST" per inserire un nuovo elemento nella presentazione dell'utente(element) nella base dati MongoDB, esegue la funzione callback
 			\item \textbf{Tipo di ritorno:} bool
			\end{itemize}

		\end{itemize}
	}
}
	
	
	
	
	
	
	
	
	
	
	
	
	
	
	