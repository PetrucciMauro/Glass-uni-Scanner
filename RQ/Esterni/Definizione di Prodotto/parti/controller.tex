\section{Package Premi::Controller}{
		\label{sec:controller}
		Tutti i package seguenti appartengono al package Premi, quindi per ognuno di essi lo scope sarà: Premi::[nome package].\\\\
		\textbf{\tipo}: contiene le classi che gestiscono i segnali e le chiamate effettuati dalla View.\\
		\textbf{\relaz}: comunica con il Model per la gestione del profilo e delle presentazioni.\\
		\subsection{Controller::IndexController}{
				\label{sub:indexcontroller}
				\textbf{Funzione}\\
					\indent Questa classe si occuperà di controllare che le credenziali di accesso siano corrette nel caso dell'autenticazione oppure di registrare un nuovo utente.\\
				\textbf{Relazioni d'uso con altri moduli}\\
					\indent Questa classe utilizzerà le seguenti classi:
				\begin{itemize}
					\item View::Pages::Index;
					\item Model::MongoRelations::AccessControl::Authentication;
					\item Model::MongoRelations::AccessControl::Registration.
				\end{itemize}
				\textbf{Attributi}\\
					\indent Al momento non sono stati previsti degli attributi.\\
				\textbf{Metodi}
					\begin{itemize}
					\item +index(): costruttore che dovrà inizializzare gli eventuali attributi ed inizializzare la pagina View::Pages::Index;
					\item +login(): metodo che richiama Model::MongoRelations::AccessControl::Authentication per effettuare l'autenticazione al sistema: se l'operazione va a buon fine, login() reindirizza alla pagina View::Pages::Home tramite il metodo goHome() altrimetti invia un segnale di errore che verrà visualizzato in View::Pages::Index;
					\item +subscription(): metodo che richiama Model::MongoRelations::AccessControl::Registration per registrare un nuovo utente nel sistema: se l'operazione va a buon fine, Subscription() reindirizza alla pagina View::Pages::Home tramite il metodo goHome() altrimetti segnala un errore che verrà visualizzato in View::Pages::Index;
					\item +goHome(): metodo che reindirizza alla pagina View::Pages::Home.
				\end{itemize}
			}
			\subsection{Controller::HomeController}{
					\label{sub:homecontroller}
					\textbf{Funzione}\\
					\indent Questa classe si occuperà di gestire i segnali e le chiamate provenienti dalla pagina View::Pages::Home.\\
					\textbf{Relazioni d'uso con altri moduli}\\
					\indent Questa classe utilizzerà le seguenti classi:
					\begin{itemize}
						\item View::Pages::Home;
						\item Model::MongoRelations::AccessControl::LoaderClass;
						\item Model::MongoRelations::AccessControl::Authentication;
						\item Model::ApacheRelations::ResourceGetter. [[[[[[[[[[[[[[NOOOOOO]]]]]]]]]]]]]]
					\end{itemize}
					\textbf{Attributi}\\
					\indent Al momento non sono stati previsti degli attributi.\\\\
					\textbf{Metodi}
					\begin{itemize}
						\item +home(): costruttore che dovrà inizializzare gli eventuali attributi e che dovrà inizializzare la pagina View::Pages::Home in base all'id del token dell'utente attivo e riportando l'elenco delle presentazioni dell'utente richiamando Model::MongoRelations::AccessControl::LoaderClass. Nel caso in cui il token risultasse non valido o scaduto, verrà invocato il metodo goIndex() che riporterà alla pagina di autenticazione;
						\item +goExecute(string id): metodo che reindirizza alla pagina View::Pages::Execute passando l'id della presentazione da eseguire;
						\item +goIndex(): metodo che reindirizza alla pagina View::Pages::Index;
						\item +goProfile(): metodo che reindirizza alla pagina View::Pages::Profile per la gestione del profilo utente;
						\item +goEdit(string id): metodo che reindirizza alla pagina View::Pages::Edit passando l'id della presentazione da modificare;
						\item +deleteSlideShow(string id): metodo per l'eliminazione di una presentazione. Esso riceve l'id della presentazione dalla pagina View::Pages::Home e chiama il metodo [[[[[[[[[[[METODO ELIMINAZIONE PRESENTAZIONE]]]]]]]]]]] che rimuoverà la presentazione dal database e comunicherà alla pagina View::Pages::Home l'avvenuta eliminazione con conseguente rimozione dalla pagina della miniatura della presentazione in oggetto. In caso di errore esso verrà visulizzato da View::Pages::Home; 
						\item +download(string id): metodo che riceve l'id della presentazione da View::Pages::Home e permette di poterla scaricare in locale. Il metodo richiama [[[[[[[[[[METODO DOWNLOAD PRESENTAZIONE]]]]]]]]]];
						\item +renameSlideShow(string id): metodo che riceve da View::Pages::Home l'id della presentazione e una stringa con cui rinominarlo. Il metodo richiamerà [[[[[[[[[[[METODO RINOMINA PRESENTAZIONE]]]]]]]]]]] che si preoccuperà di rinominarla nel database e, se l'operazione è andata a buon fine, renameSlideShow() lo comunicherà a View::Pages::Home, altrimenti segnalerà un errore;
						\item +execute(string id): metodo che richiama la funzione goExecute();
						\item +editSlideShow(): metodo che richiama la funzione goEdit(id); l'id passato sarà quello della presentazione da modificare;
						\item +logout(): metodo che richiama deAuthenticate() fornito da Model::MongoRelations::AccessControl::Authentication permettendo di effettuare il logout dal sistema; dopo il logout HomeController reindirizzerà alla pagina View::Pages::Index tramite il metodo goIndex();
						\item +newSlideShow(): metodo che richiama [[[[[[[[[[METODO MODEL CREAZIONE PRESENTAZIONE]]]]]]]]]] per la creazione di una nuova presentazione del database; se la creazione è andata a buon fine richiamerà il metodo goEdit(string id), passando l'id della nuova presentazione, in modo da entrare in modalità modifica.
					\end{itemize}
				}
				\subsection{Controller::ProfileController}{
					\textbf{Funzione}\\
					\indent Questa classe consentirà la modifica dei dati personali dell'utente.\\\\
					\textbf{Relazioni d'uso con altri moduli}\\
					\indent Questa classe utilizzerà le seguenti classi:
					\begin{itemize}
						\item Model::ApacheRelations::FileManager; [[[[[[[[[[NOOOOOOO]]]]]]]]]]
						\item Model::MongoRelations::AccessControl::Authentication.
					\end{itemize}
					\textbf{Attributi}\\
					\indent Al momento non sono stati previsti degli attributi.\\\\
					\textbf{Metodi}
					\begin{itemize}
						\item +Profile(): costruttore che dovrà inizializzare gli eventuali attibuti e la pagina View::Pages::Profile in base all'id del token dell'utente attivo e nel caso in cui nessun token risultasse attivo richiamerà il metodo goIndex(). Profile() dialogherà con Model::ApacheRelations::FileManager [[[[[[[[[NOOOOO]]]]]]]]] per recuperare tutti i file media dell'utente caricati sul server e con Model::MongoRelations::AccessControl::Authentication per ottenere i dati personali dell'utente [[[[[[[[[I DATI PERSONALI ARRIVANO DA AUTHENTICATION??]]]]]]]]];
						\item +changePassword(string password): metodo che dialoga con Model::MongoRelations::AccessControl::Authentication inviandogli la stringa della nuova password da sostituire con quella vecchia. In caso di errore, esso verrà mostrato tramite la pagina View::Pages::Profile;
						\item +uploadMedia(string percorso): metodo che invia a Model::ApacheRelations::FileManager il percorso del file media da caricare sul proprio spazio sul server, restituisce la schermata aggiornata con la nuova miniatura del file media se il caricamento è andato a buon fine, altrimenti restituisce un errore;
						\item deleteMedia(string id): metodo che invia l'id del file media a Model::ApacheRelations::FileManager per rimuoverlo dal database;
						\item +renameSlideShow(string id, string name): metodo che invia l'id del file media e una stringa a Model::ApacheRelations::FileManager per rinominarlo;
						\item +goHome(): metodo che reindirizza alla pagina View::Pages::Home;
						\item +goIndex(): metodo che reindirizza alla pagina View::Pages::Index.
					\end{itemize}
				}
				\subsection{View::Pages::Execution}{
					\textbf{Funzione}\\
					\indent Questa classe si occuperà di gestire i segnali che arrivano da View::Pages::Execution.\\\\
					\textbf{Relazioni d'uso con altri moduli}\\
					\indent Questa classe utilizzerà le seguenti classi:
					\begin{itemize}
						\item View::Pages::Execution;
						\item Model::MongoRelations::Loader::LoaderClass;
					\end{itemize}
					\textbf{Attributi}\\
					\indent Al momento non sono stati previsti degli attributi.\\\\
					\textbf{Metodi}
					\begin{itemize}
						\item Execution(string id): costruttore che dovrà inizializzare gli eventuali attibuti e la pagina View::Pages::Execution in base all'id della presentazione. Execution(string id) controllerà la presenza del token di sessione, usando il metodo goIndex() in caso negativo, e dialogherà con Model::MongoRelations::Loader::LoaderClass per caricare la presentazione dal database;
						\item +goHome(): metodo che reindirizza alla pagina View::Pages::Home;
						\item +goIndex(): metodo che reindirizza alla pagina View::Pages::Index;
						\item +goEdit(): metodo che reindirizza alla pagina View::Pages::Edit.
					\end{itemize}
				}
				\subsection{Controller::EditController}{
					\textbf{Funzione}\\
					\indent Questa classe si occuperà di mostrare all'utente la possibilità di apportare modifiche ad una presentazione.\\\\
					\textbf{Relazioni d'uso con altri moduli}\\
					\indent Questa classe utilizzerà le seguenti classi:
					\begin{itemize}
						\item View::Pages::Edit;
						\item Model::SlideShow::SlideShowActions::Command:
						\begin{itemize}
							\item ConcreteTextInsertCommand;
							\item ConcreteFrameInsertCommand;
							\item ConcreteImageInsertCommand;
							\item ConcreteSVGInsertCommand;
							\item ConcreteAudioInsertCommand;
							\item ConcreteVideoInsertCommand;
							\item ConcreteBackgroundInsertCommand;
							\item ConcreteTextRemoveCommand;
							\item ConcreteFrameRemoveCommand;
							\item ConcreteImageRemoveCommand;
							\item ConcreteSVGRemoveCommand;
							\item ConcreteAudioRemoveCommand;
							\item ConcreteVideoRemoveCommand;
							\item ConcreteBackgroundRemoveCommand;
							\item ConcreteEditSizeCommand;
							\item ConcreteEditPositionCommand;
							\item ConcreteEditRotationCommand;
							\item ConcreteEditColorCommand;
							\item ConcreteEditBackgroundCommand;
							\item ConcreteEditFontCommand;
							\item ConcreteEditContentCommand; [[[[[[[[[[[[[[[[ESISTE????]]]]]]]]]]]]]]]]
							\item Invoker;
						\end{itemize}
						\item Model::ApacheManager::FileManager;
						\item Model::ApacheRelations::ResourceGetter;
						\item Model::MongoRelations::Loader::Autenticazione;
						\item Model::MongoRelations::Loader::Loaderclass;
					\end{itemize}
					\textbf{Attributi}\\
					\indent Al momento non sono stati previsti degli attributi.\\\\
					\textbf{Metodi}
					\begin{itemize}
						\item +Edit(string id):  costruttore che dovrà inizializzare gli eventuali attibuti e la pagina View::Pages::Edit in base all'id della presentazione. Edit(string id) controllerà la presenza del token di sessione, usando il metodo goIndex() in caso negativo, e dialogherà con Model::MongoRelations::Loader::LoaderClass per caricare la presentazione dal database;
						\item +insertFrame(string shape, int posX, int posY): metodo che dialoga con Model::SlideShow::SlideShowActions::Command::ConcreteFrameInsertCommand passandogli le informazioni per l'inserimento di un nuovo frame e le sue coordinate. Se l'inserimento avviene correttamente, View::Pages::Edit verrà aggiornata col nuovo frame inserito, altrimenti verrà segnalato un errore;
						\item +insertMedia(string path, int posX, int posY): metodo che dialoga con ConcreteImageInsertCommand, ConcreteVideoInsertCommand e ConcreteAudioInsertCommand (facenti parte di Model::SlideShow::SlideShowActions::Command) comunicando il percorso del file media in locale in modo da caricarlo sul server e le sue coordinate di posizione sul piano della presentazione. Se l'inserimento avviene correttamente, View::Pages::Edit verrà aggiornata col nuovo file media inserito, altrimenti verrà segnalato un errore;
						\item +moveElement(string id, int posX, int posY): metodo che dialoga con Model::SlideShow::SlideShowActions::Command::ConcreteEditPositionCommand passando l'id dell'elemento spostato e le sue nuove coordinate di posizione. Se la modifica avviene correttamente, View::Pages::Edit verrà aggiornata con l'elemento nella nuova posizione, altrimenti verrà segnalato un errore;
						\item +insertText(string text, int posX, int posY): metodo che dialoga con Model::SlideShow::SlideShowActions::Command::ConcreteTextInsertCommand comunicando la stringa del testo e le coordinate di posizione sul piano della presentazione. Se l'inserimento avviene correttamente, View::Pages::Edit verrà aggiornata col nuovo elemento testo inserito, altrimenti verrà segnalato un errore;
						\item +textEdit(string id, string text): metodo che dialoga con Model::SlideShow::SlideShowActions::Command::ConcreteEditContentCommand passando l'id dell'elemento di testo e la nuova stringa associata. Se la modifica avviene correttamente, View::Pages::Edit verrà aggiornata con l'elemento di testo aggiornato con la nuova stringa, altrimenti verrà segnalato un errore;
						\item +deleteElement(string id): metodo che dialoga con ConcreteTextRemoveCommand, ConcreteFrameRemoveCommand, ConcreteImageRemoveCommand, ConcreteSVGRemoveCommand, ConcreteAudioRemoveCommand e ConcreteVideoRemoveCommand (facenti parte di Model::SlideShow::SlideShowActions::Command) comunicando l'id dell'elemento da rimuovere dal piano della presentazione. Se la rimozione avviene correttamente, View::Pages::Edit verrà aggiornata col nuovo file media inserito, altrimenti verrà segnalato un errore;
						\item +insertChoice(string id, string choiceId): [[[[[[[[[[[[[[[[[[[[VEDEREEEEEEE]]]]]]]]]]]]]]]]]]]];
						\item +bookmark(string id): [[[[[[[[[[[[[[[[[[[VEDEREEEEEE]]]]]]]]]]]]]]]]]]];
						\item +changeSize(string id, int zoom): metodo che dialoga con Model::SlideShow::SlideShowActions::Command::ConcreteEditSizeCommand passando l'id dell'elemento da ingrandire e il nuovo valore di zoom da applicare. Se la modifica avviene correttamente, View::Pages::Edit verrà aggiornata con l'elemento zoomato al valore indicato, altrimenti verrà segnalato un errore;
						\item +changeRotation(string id, int rotation): metodo che dialoga con Model::SlideShow::SlideShowActions::Command::ConcreteEditRotationCommand passando l'id dell'elemento da ruotare e il nuovo valore di rotazione da applicare. Se la modifica avviene correttamente, View::Pages::Edit verrà aggiornata con l'elemento ruotato al valore indicato, altrimenti verrà segnalato un errore;
						\item +changePath(array id): [[[[[[[[[[[[[[[VEDEREEEEEE]]]]]]]]]]]]]]];
						\item +frameBackground(string id, string path): metodo che dialoga con Model::SlideShow::SlideShowActions::Command::ConcreteBackgroundInsertCommand passando l'id del frame e il percorso dell'immagine da impostare come nuovo sfondo del frame. Se la modifica avviene correttamente, View::Pages::Edit verrà aggiornata con lo sfondo nuovo, altrimenti verrà segnalato un errore;
						\item +slideshowBackground(string path): metodo che dialoga con Model::SlideShow::SlideShowActions::Command::ConcreteEditBackgroundCommand passando il percorso dell'immagine da impostare come nuovo sfondo della presentazione. Se la modifica avviene correttamente, View::Pages::Edit verrà aggiornata con lo sfondo nuovo, altrimenti verrà segnalato un errore;
						\item +insertSvg(string shape, string color, int posX, int posY): metodo che dialoga con Model::SlideShow::SlideShowActions::Command::ConcreteSVGInsertCommand comunicando il tipo di SVG da inserire, le sue coordinate di posizione sul piano della presentazione e il suo colore. Se l'inserimento avviene correttamente, View::Pages::Edit verrà aggiornata col nuovo SVG, altrimenti verrà segnalato un errore;
						\item +esegui(string id): metodo che richiama la funzione goExecute(id);
						\item +goExecute(string id): reindirizza alla pagina View::Pages::Execute passando l'id della presentazione da eseguire;
						\item +goHome(): metodo che reindirizza alla pagina View::Pages::Home;
						\item +goIndex(): metodo che reindirizza alla pagina View::Pages::Index.
					\end{itemize}
				}
				}