\section {Specifica classi del front-end}
\subsection{Premi::App}
	\label{sec:premiapp}
	\textbf{Funzione}\\
	\indent Questa classe si occuperà di fare il boostrap dell'applicazione instanziando la rootscope  e iniettando tutti i moduli necessari.\\
	\textbf{Relazioni d'uso con altri moduli}\\
	\indent Questa classe utilizzerà le seguenti classi:
	\begin{itemize}
		\item Services::Utils;
		\item Services::Main;
		\item Services::toPages.
	\end{itemize}
	\textbf{Attributi}\\
	\begin{itemize}
		\item scope:Object\\
		Questo campo dati rappresenta l’oggetto che permette la comunicazione tra la view ed il controller, rendendo possibile l’accesso al model mantenendolo sincronizzato, implementando in questo modo il 2-way data binding;
		\item rootScope:Object\\
		Questo campo dati rappresenta lo scope radice dell’applicazione. Tutti gli altri scope	discendono da questo;
		\item \textdollar routeProvider::Object\\
		Questo campo dati rappresenta il servizio che collega tra loro controller, view e l'URL corrente nel browser;
		\item \textdollar mdIconProvider\\
		\item \textdollar mdThemingProvider\\
		\item \textdollar httpProvider\\
		\item \textdollar provide\\
		\item \textdollar locationProvider\\
	\end{itemize}
	\textbf{Metodi}
	\begin{itemize}
		\item run;
		\item config;
		\item interceptors;
		\item decorator.
	\end{itemize}

\subsection {Package Premi::Services}
\label{sec:services}
\textbf{\tipo}: i servizi sono degli oggetti che incapsulano del codice che si occupa di eseguire uno specifico compito, il quale sarà poi utilizzato all’interno di una o più parti dell’applicazione.\\
\textbf{\relaz}: comunica con il controller per la gestione delle funzioni da eseguire e con il model per la gestione delle componenti necessarie.\\
\subsubsection{Services::Main}{
		\label{sub:servicesMain}
		\textbf{Funzione}\\
		\indent Questa classe si occuperà di eseguire le funzioni base dell'applicazione, in particolare autenticazione e registrazione al server degli utenti.\\
		\textbf{Relazioni d'uso con altri moduli}\\
		\indent Questa classe utilizzerà le seguenti classi:
		\begin{itemize}
			\item Services::Utils;
			\item localStorage:Object\\
				\indent Servizio angular che permette il salvataggio in locale di oggetti necessari al garantimento delle funzioni dell'applicazione.
		\end{itemize}
		\textbf{Attributi}
		\begin{itemize}
			\item \textbf{login}
			\begin{itemize}
				\item \textbf{Accesso}: Private;
				\item \textbf{Tipo}: Oggetto;
				\item \textbf{Descrizione}: oggetto che mantiene la sessione corrente.
			\end{itemize}
		\end{itemize}
		\textbf{Metodi}
		\begin{itemize}
			\item \textbf{value}()
			\begin{itemize}
				\item \textbf{Accesso}: Private;
				\item \textbf{Tipo di ritorno}: Void;
				\item \textbf{Descrizione}: metodo che controlla se è stato effettuato un refresh della pagina, in tal caso ripristina login.
			\end{itemize}
			\item \textbf{register}(formData, success, error)
			\begin{itemize}
				\item \textbf{Accesso}: Public;
				\item \textbf{Tipo di ritorno}: Void;
				\item \textbf{Descrizione}: metodo che, attraverso l'oggetto formData contenente le credenziali di accesso, effettua la registrazione al server di un nuovo utente. Se l'operazione ha successo viene invocato success altrimenti error.
			\end{itemize}
			\item \textbf{login}(formData, success, error)
			\begin{itemize}
				\item \textbf{Accesso}: Public;
				\item \textbf{Tipo di ritorno}: Void;
				\item \textbf{Descrizione}: metodo che, attraverso l'oggetto formData contenente le credenziali di accesso, effettua l'autenticazione al server di un utente. Se l'operazione ha successo viene invocato success altrimenti error.
			\end{itemize}
			\item \textbf{logout}(success, error)
			\begin{itemize}
				\item \textbf{Accesso}: Public;
				\item \textbf{Tipo di ritorno}: Void;
				\item \textbf{Descrizione}: metodo che effettua il logout dal server. Se l'operazione ha successo viene invocato success altrimenti error.
			\end{itemize}
			\item \textbf{changepassword}(formData, success, error)
			\begin{itemize}
				\item \textbf{Accesso}: Public;
				\item \textbf{Tipo di ritorno}: Void;
				\item \textbf{Descrizione}: metodo che, attraverso l'oggetto formData contenente le credenziali di accesso e la nuova password, effettua il cambio della password di un utente. Se l'operazione ha successo viene invocato success altrimenti error.
			\end{itemize}
			\item \textbf{getToken}()
			\begin{itemize}
				\item \textbf{Accesso}: Public;
				\item \textbf{Tipo di ritorno}: JSONWebToken;
				\item \textbf{Descrizione}: metodo che ritorna il token di sessione.
			\end{itemize}
			\item \textbf{getUser}()
			\begin{itemize}
				\item \textbf{Accesso}: Public;
				\item \textbf{Tipo di ritorno}: Object;
				\item \textbf{Descrizione}: metodo che ritorna l'utente attualmente autenticato con il server.
			\end{itemize}
		\end{itemize} 
}
\subsubsection{Services::Upload}{
		\label{sub:servicesUpload}
		\textbf{Funzione}\\
		\indent Questa classe si occuperà di eseguire l'upload di file media nel database.\\
		\textbf{Relazioni d'uso con altri moduli}\\
		\indent Questa classe utilizzerà le seguenti classi:
		\begin{itemize}
			\item Services::Main;
			\item Services::Utils;
		\end{itemize}
		\textbf{Attributi}
		\begin{itemize}
			\item \textbf{image}
			\begin{itemize}
				\item \textbf{Accesso}: Private;
				\item \textbf{Tipo}: Array;
				\item \textbf{Descrizione}: array contenente i formati immagini accettati per l'upload.
			\end{itemize}
			\item \textbf{audio}
			\begin{itemize}
				\item \textbf{Accesso}: Private;
				\item \textbf{Tipo}: Array;
				\item \textbf{Descrizione}: array contenente i formati audio accettati per l'upload.
			\end{itemize}
			\item \textbf{video}
			\begin{itemize}
				\item \textbf{Accesso}: Private;
				\item \textbf{Tipo}: Array;
				\item \textbf{Descrizione}: array contenente i formati video accettati per l'upload.
			\end{itemize}
		\end{itemize}
		\textbf{Metodi}
		\begin{itemize}
			\item \textbf{uploadmedia}(files, success, error)
			\begin{itemize}
				\item \textbf{Accesso}: Public;
				\item \textbf{Tipo di ritorno}: Void;
				\item \textbf{Descrizione}: metodo che richiama Model::\-serverRelation::\-?????????????????????????????????????????????????????????????????????????????????? effettuando l'upload dei file contenuti nell'array files passato come parametro. Se l'operazione ha successo viene invocato success altrimenti error.
			\end{itemize}
			\item \textbf{isImage}(files)
			\begin{itemize}
				\item \textbf{Accesso}: Public;
				\item \textbf{Tipo di ritorno}: Bool;
				\item \textbf{Descrizione}: metodo che ritorna true se i file contenuti nell'array files, passato come parametro, rispettano almeno uno tra i formati contenuti nell'array image, altrimenti ritorna false.
			\end{itemize}
			\item \textbf{isAudio}(files)
			\begin{itemize}
				\item \textbf{Accesso}: Public;
				\item \textbf{Tipo di ritorno}: Bool;
				\item \textbf{Descrizione}: metodo che ritorna true se i file contenuti nell'array files, passato come parametro, rispettano almeno uno tra i formati contenuti nell'array audio, altrimenti ritorna false.
			\end{itemize}
			\item \textbf{isVideo}(files)
			\begin{itemize}
				\item \textbf{Accesso}: Public;
				\item \textbf{Tipo di ritorno}: Bool;
				\item \textbf{Descrizione}: metodo che ritorna true se i file contenuti nell'array files, passato come parametro, rispettano almeno uno tra i formati contenuti nell'array video, altrimenti ritorna false.
			\end{itemize}
			\item \textbf{getFileUrl}()
			\begin{itemize}
				\item \textbf{Accesso}: Public;
				\item \textbf{Tipo di ritorno}: Stringa;
				\item \textbf{Descrizione}: metodo che ritorna il percorso di salvataggio dei file dell'utente corrente.
			\end{itemize}
		\end{itemize} 
}

\subsubsection{Services::Utils}{
		\label{sub:servicesUtils}
		\textbf{Funzione}\\
		\indent Questa classe si occuperà di eseguire piccole funzionalità utili ad ogni parte dell'applicazione.\\
		\textbf{Metodi}
		\begin{itemize}
			\item \textbf{decodeToken}(token)
			\begin{itemize}
				\item \textbf{Accesso}: Public;
				\item \textbf{Tipo di ritorno}: Object;
				\item \textbf{Descrizione}: metodo che decodifica token, passato come parametro, e ritorna l'oggetto utente corrispondente.
			\end{itemize}
			\item \textbf{grade}(password)
			\begin{itemize}
				\item \textbf{Accesso}: Public;
				\item \textbf{Tipo di ritorno}: String;
				\item \textbf{Descrizione}: metodo che determina la robustezza del parametro password. La lunghezza minima di una password è stata impostata a sei caratteri.
			\end{itemize}
			\item \textbf{hostname}()
			\begin{itemize}
				\item \textbf{Accesso}: Public;
				\item \textbf{Tipo di ritorno}: String;
				\item \textbf{Descrizione}: metodo che ritorna il dominio dell'applicazione.
			\end{itemize}
			\item \textbf{isUndefined}(object)
			\begin{itemize}
				\item \textbf{Accesso}: Public;
				\item \textbf{Tipo di ritorno}: Bool;
				\item \textbf{Descrizione}: metodo che ritorna true se il parametro object risulta indefinito, altrimento ritorna false.
			\end{itemize}
			\item \textbf{isObject}(object)
			\begin{itemize}
				\item \textbf{Accesso}: Public;
				\item \textbf{Tipo di ritorno}: Void;
				\item \textbf{Descrizione}: metodo che ritorna true se il parametro object risulta definito, altrimento ritorna false.
			\end{itemize}
			\item \textbf{encrypt}(string)
			\begin{itemize}
				\item \textbf{Accesso}: Public;
				\item \textbf{Tipo di ritorno}: String;
				\item \textbf{Descrizione}: metodo che ritorna il parametro string criptato. Il metodo di criptaggio scelto è lo SHA-1.
			\end{itemize}
		\end{itemize}
}

\subsubsection{Services::SharedData}{
		\label{sub:servicesSharedData}
		\textbf{Funzione}\\
		\indent Questa classe mantiene in memoria la presentazione sulla quale l'utente sta lavorando.\\
		\textbf{Relazioni d'uso con altri moduli}\\
		\indent Questa classe utilizzerà le seguenti classi:
		\begin{itemize}
			\item Services::Utils;
			\item Services::Main;
			\item localStorage:Object\\
				\indent Servizio angular che permette il salvataggio in locale di oggetti necessari al garantimento delle funzioni dell'applicazione.
		\end{itemize}
		\textbf{Attributi}\\
		\begin{itemize}
			\item \textbf{myPresentation}
			\begin{itemize}
				\item \textbf{Accesso}: Private;
				\item \textbf{Tipo}: Object;
				\item \textbf{Descrizione}: oggetto che rappresenta l'attuale presentazione aperta.
			\end{itemize}
		\end{itemize}
		\textbf{Metodi}
		\begin{itemize}
			\item \textbf{getPresentazione}(idSlideShow)
			\begin{itemize}
				\item \textbf{Accesso}: Public;
				\item \textbf{Tipo di ritorno}: Object;
				\item \textbf{Descrizione}: metodo che, nel caso in cui il parametro idSlideShow sia definito, richiama il metodo Model::\-ServerRelation::\-MongoRelation::\-getPresentation(idSlideShow) assegnando il risultato a myPresentation. In ogni caso myPresentation viene ritornato.
			\end{itemize}
		\end{itemize}
}
\subsubsection{Services::toPages}{
		\label{sub:servicestoPages}
		\textbf{Funzione}\\
		\indent Questa classe si occuperà di eseguire i reindirizzamenti alle pagine corrette.\\
		\textbf{Relazioni d'uso con altri moduli}\\
		\indent Questa classe utilizzerà le seguenti classi:
		\begin{itemize}
			\item Services::Utils;
			\item Services::Main;
			\item Services::SharedData;
			\item \$ http:Object\\
				\indent Servizio Angular che permette la comunicazione in remoto con un server.
			\item \$ location:Object\\
				\indent Servizio Angular che gestisce gli indirizzi URL.
		\end{itemize}
		\textbf{Metodi}
		\begin{itemize}
			\item \textbf{sendRequest}(dest, success, error)
			\begin{itemize}
				\item \textbf{Accesso}: Private;
				\item \textbf{Tipo di ritorno}: Object;
				\item \textbf{Descrizione}: metodo che ritorna una richiesta http all'indirizzo definito dal parametro dest. Se l'operazione ha successo viene invocato success altrimenti error.
			\end{itemize}
			\item \textbf{loginpage}()
			\begin{itemize}
				\item \textbf{Accesso}: Public;
				\item \textbf{Tipo di ritorno}: Object;
				\item \textbf{Descrizione}: metodo che permette di accedere alla pagina di Login. Esso richiama il metodo sendRequest() il quale, se ha successo, reindirizza alla pagina richiesta.
			\end{itemize}
			\item \textbf{registrazionepage}()
			\begin{itemize}
				\item \textbf{Accesso}: Public;
				\item \textbf{Tipo di ritorno}: Object;
				\item \textbf{Descrizione}: metodo che permette di accedere alla pagina di Registrazione. Esso richiama il metodo sendRequest() il quale, se ha successo, reindirizza alla pagina richiesta.
			\end{itemize}
			\item \textbf{homepage}()
			\begin{itemize}
				\item \textbf{Accesso}: Public;
				\item \textbf{Tipo di ritorno}: Object;
				\item \textbf{Descrizione}: metodo che permette di accedere alla pagina Home. Esso richiama il metodo sendRequest() il quale, se ha successo, reindirizza alla pagina richiesta.
			\end{itemize}
			\item \textbf{profilepage}()
			\begin{itemize}
				\item \textbf{Accesso}: Public;
				\item \textbf{Tipo di ritorno}: Object;
				\item \textbf{Descrizione}: metodo che permette di accedere alla pagina Profile. Esso richiama il metodo sendRequest() il quale, se ha successo, reindirizza alla pagina richiesta.
			\end{itemize}
			\item \textbf{editpage}(slideId)
			\begin{itemize}
				\item \textbf{Accesso}: Public;
				\item \textbf{Tipo di ritorno}: Object;
				\item \textbf{Descrizione}: metodo che permette di accedere alla pagina di Edit. Esso richiama il metodo sendRequest() il quale, se ha successo, reindirizza alla pagina richiesta e richiama il metodo SharedData.forEdit() passandogli il parametro slideId.
			\end{itemize}
			\item \textbf{executionpage}(slideId)
			\begin{itemize}
				\item \textbf{Accesso}: Public;
				\item \textbf{Tipo di ritorno}: Object;
				\item \textbf{Descrizione}: metodo che permette di accedere alla pagina di Execution. Esso richiama il metodo sendRequest() il quale, se ha successo, reindirizza alla pagina richiesta e richiama il metodo SharedData.forEdit() passandogli il parametro slideId.
			\end{itemize}
		\end{itemize} 
}

\subsection{Package Premi::Controller}{
\label{sec:controller}
Tutti i package seguenti appartengono al package Premi, quindi per ognuno di essi lo scope sarà: Premi::[nome package].\\\\
\textbf{\tipo}: contiene le classi che gestiscono i segnali e le chiamate effettuati dalla View.\\
\textbf{\relaz}: comunica con il Model per la gestione del profilo e delle presentazioni.\\

\subsubsection{Controller::HeaderController}
	\label{sub:HeaderController}
	\textbf{Funzione}\\
	\indent Questa classe si occuperà di controllare l'Header dell'applicazione.\\
	\textbf{Relazioni d'uso con altri moduli}\\
	\indent Questa classe utilizzerà le seguenti classi:
	\begin{itemize}
		\item View::Pages::Index;
		\item Services::Utils;
		\item Services::Main;
		\item Services::toPages.
		\item \$ scope:Object\\
			\indent Questo campo dati rappresenta l’oggetto che permette la comunicazione tra la view ed il controller, rendendo possibile l’accesso al model mantenendolo sincronizzato, implementando in questo modo il 2-way data binding.
		\item \$ rootScope:Object\\
			\indent Questo campo dati rappresenta lo scope radice dell’applicazione. Tutti gli altri scope discendono da questo.
	\end{itemize}

	\textbf{Metodi}
	\begin{itemize}
		\item \textbf{goLogin}()
		\begin{itemize}
			\item \textbf{Accesso}: Public;
			\item \textbf{Tipo di ritorno}: Void;
			\item \textbf{Descrizione}: metodo che richiama Services::\-toPages::\-loginpage() per effettuare il reindirizzamento alla pagina di login.
		\end{itemize}
		\item \textbf{goRegistrazione}()
		\begin{itemize}
			\item \textbf{Accesso}: Public;
			\item \textbf{Tipo di ritorno}: Void;
			\item \textbf{Descrizione}: metodo che richiama Services::\-toPages::\-registrazionepage() per effettuare il reindirizzamento alla pagina di registrazione.
		\end{itemize}
		\item \textbf{goHome}()
		\begin{itemize}
			\item \textbf{Accesso}: Public;
			\item \textbf{Tipo di ritorno}: Void;
			\item \textbf{Descrizione}: metodo che richiama Services::\-toPages::\-homepage() per effettuare il reindirizzamento alla pagina home.
		\end{itemize}
		\item \textbf{goProfile}()
		\begin{itemize}
			\item \textbf{Accesso}: Public;
			\item \textbf{Tipo di ritorno}: Void;
			\item \textbf{Descrizione}: metodo che richiama Services::\-toPages::\-profilepage() per effettuare il reindirizzamento alla pagina profile.
		\end{itemize}
		\item \textbf{who}()
		\begin{itemize}
			\item \textbf{Accesso}: Public;
			\item \textbf{Tipo di ritorno}: String;
			\item \textbf{Descrizione}: metodo che ritorna lo username dell'utente attualmente autenticato.
		\end{itemize}
		\item \textbf{isToken}()
		\begin{itemize}
			\item \textbf{Accesso}: Public;
			\item \textbf{Tipo di ritorno}: Boolean;
			\item \textbf{Descrizione}: metodo che verifica l'effettiva autenticazione dell'utente.
		\end{itemize}
		\item \textbf{logout}()
		\begin{itemize}
			\item \textbf{Accesso}: Public;
			\item \textbf{Tipo di ritorno}: Void;
			\item \textbf{Descrizione}: metodo che richiama Services::\-Main::\-logout() per effettuare il logout dal server. Se l'operazione va a buon fine, viene effettuato il reindirizzamento alla pagina di login richiamando Services::\-toPages::\-loginpage().
		\end{itemize}
		\item \textbf{error}()
		\begin{itemize}
			\item \textbf{Accesso}: Public;
			\item \textbf{Tipo di ritorno}: String;
			\item \textbf{Descrizione}: metodo che ritorna l'errore individuato; esso deve essere posto all'interno di \$ rootScope.error.
		\end{itemize}
	\end{itemize}

%CONTROLLER
\subsubsection{Controller::AccessController}{
	\label{sub:AccessController}
	\textbf{Funzione}\\
		\indent Questa classe si occuperà di controllare che le credenziali di accesso siano corrette nel caso dell'autenticazione oppure di registrare un nuovo utente.\\
	\textbf{Relazioni d'uso con altri moduli}\\
		\indent Questa classe utilizzerà le seguenti classi:
	\begin{itemize}
		\item View::Pages::Login;
		\item View::Pages::Registrazione;
		\item Model::\-serverRelation::\-Authentication;
		\item Model::\-serverRelation::\-Registration.
		\item Services::Utils;
		\item Services::Main;
		\item Services::toPages;
		\item \$ scope:Object\\
    		\indent Questo campo dati rappresenta l’oggetto che permette la comunicazione tra la view ed il controller, rendendo possibile l’accesso al model mantenendolo sincronizzato, implementando in questo modo il 2-way data binding.
	\end{itemize}
	\textbf{Attributi}\\
    \begin{itemize}
    	\item \textbf{user}()
		\begin{itemize}
			\item \textbf{Accesso}: Private;
			\item \textbf{Tipo}: Object;
			\item \textbf{Descrizione}: oggetto contenente username e password derivanti dal form della pagina html.
		\end{itemize}
    	\item \textbf{getData}()
		\begin{itemize}
			\item \textbf{Accesso}: Private;
			\item \textbf{Tipo}: Object;
			\item \textbf{Descrizione}: metodo che ritorna un oggetto contenente i campi dati username e password ricavati dallo \$ scope. La password viene criptata grazie a Services::\-Utils::\-encrypt(password).
		\end{itemize}
    \end{itemize}
	\textbf{Metodi}
	\begin{itemize}
		\item \textbf{reset}()
		\begin{itemize}
			\item \textbf{Accesso}: Public;
			\item \textbf{Tipo di ritorno}: Void;
			\item \textbf{Descrizione}: metodo che cancella i valori delle variabili all'interno di \$ scope.
		\end{itemize}
		\item \textbf{login}()
		\begin{itemize}
			\item \textbf{Accesso}: Public;
			\item \textbf{Tipo di ritorno}: Void;
			\item \textbf{Descrizione}: metodo che controlla se user è definito e, in caso affermativo, richiama Services::\-Main::\-login(data) per effettuare il login al server. Se l'operazione ha successo viene effettuato il reindirizzamento alla pagina home richiamando Services::\-toPages::\-homepage().
		\end{itemize}
        \item \textbf{registration}()
		\begin{itemize}
			\item \textbf{Accesso}: Public;
			\item \textbf{Tipo di ritorno}: Void;
			\item \textbf{Descrizione}: metodo che controlla se user è definito e, in caso affermativo, richiama Services::\-Main::\-register(data) per effettuare la registrazione al server. Se l'operazione ha successo viene effettuato il reindirizzamento alla pagina home richiamando Services::\-toPages::\-homepage().
		\end{itemize}
	\end{itemize}
}
\subsubsection{Controller::HomeController}{
	\label{sub:homecontroller}
	\textbf{Funzione}\\
	\indent Questa classe si occuperà di gestire i segnali e le chiamate provenienti dalla pagina Home.\\
	\textbf{Relazioni d'uso con altri moduli}\\
	\indent Questa classe utilizzerà le seguenti classi:
	\begin{itemize}
		\item View::Pages::Home;
		\item Model::\-serverRelation::\-MongoRelation;
		\item Services::Utils;
		\item Services::Main;
		\item Services::toPages;
		\item \$ scope:Object\\
			\indent Questo campo dati rappresenta l’oggetto che permette la comunicazione tra la view ed il controller, rendendo possibile l’accesso al model mantenendolo sincronizzato, implementando in questo modo il 2-way data binding.
	\end{itemize}
	\textbf{Attributi}\\
	\begin{itemize}
		\item \textbf{mongo}()
		\begin{itemize}
			\item \textbf{Accesso}: Private;
			\item \textbf{Tipo}: Object;
			\item \textbf{Descrizione}: oggetto che mantiene una istanza di Model::\-serverRelation::\-MongoRelation.
		\end{itemize}
    \end{itemize}
	\textbf{Metodi}
	\begin{itemize}
		\item \textbf{update}()
		\begin{itemize}
			\item \textbf{Accesso}: Public;
			\item \textbf{Tipo di ritorno}: Void;
			\item \textbf{Descrizione}: metodo che aggiorna i contenuti della pagina home.
		\end{itemize}
		\item \textbf{goProfile}()
		\begin{itemize}
			\item \textbf{Accesso}: Public;
			\item \textbf{Tipo di ritorno}: Void;
			\item \textbf{Descrizione}: metodo che richiama Services::\-toPages::\-profilepage() per effettuare il reindirizzamento alla pagina profile.
		\end{itemize}
		\item \textbf{goEdit}(slideId)
		\begin{itemize}
			\item \textbf{Accesso}: Public;
			\item \textbf{Tipo di ritorno}: Void;
			\item \textbf{Descrizione}: metodo che richiama Services::\-toPages::\-edipage(slideId) per effettuare il reindirizzamento alla pagina di edit.
		\end{itemize}
		\item \textbf{goExecute}(slideId)
		\begin{itemize}
			\item \textbf{Accesso}: Public;
			\item \textbf{Tipo di ritorno}: Void;
			\item \textbf{Descrizione}: metodo che richiama Services::\-toPages::\-executionpage(slideId) per effettuare il reindirizzamento alla pagina di esecuzione.
		\end{itemize}
		\item \textbf{getSS}()
		\begin{itemize}
			\item \textbf{Accesso}: Public;
			\item \textbf{Tipo di ritorno}: Object;
			\item \textbf{Descrizione}: metodo che richiama Model::\-serverRelation::\-getPresentationsMeta() per visualizzare le presentazioni dell'utente corrente.
		\end{itemize}
		\item \textbf{deleteSlideShow}(slideId)
		\begin{itemize}
			\item \textbf{Accesso}: Public;
			\item \textbf{Tipo di ritorno}: Void;
			\item \textbf{Descrizione}: metodo che richiama Model::\-serverRelation::\-deletePresentation(slideId) per eliminare la presentazione slideId dal database. Se l'operazione ha successo, viene richiamato il metodo update().
		\end{itemize}
		\item \textbf{renameSlideShow}(nameSS, rename)
		\begin{itemize}
			\item \textbf{Accesso}: Public;
			\item \textbf{Tipo di ritorno}: Void;
			\item \textbf{Descrizione}: metodo che richiama Model::\-serverRelation::\-renamePresentation(nameSS, rename) per rinominare la presentazione slideId. Se l'operazione ha successo, viene richiamato il metodo update().
		\end{itemize}
		\item \textbf{createSlideShow}()
		\begin{itemize}
			\item \textbf{Accesso}: Public;
			\item \textbf{Tipo di ritorno}: Void;
			\item \textbf{Descrizione}: metodo che richiama Model::\-serverRelation::\-newPresentation() per creare una nuova presentazione. Se l'operazione ha successo, viene richiamato il metodo update().
		\end{itemize}
		\item \textbf{salvaManifest}(slideId)
		\begin{itemize}
			\item \textbf{Accesso}: Public;
			\item \textbf{Tipo di ritorno}: Void;
			\item \textbf{Descrizione}: metodo che richiama Model::\-??????????????????????????????????????????????????????????????????????????????????????????????? per salvare nel manifest la presentazione slideId.
		\end{itemize}
	\end{itemize}
	}
\subsubsection{Controller::ProfileController}{
	\textbf{Funzione}\\
	\indent Questa classe si occupa di gestire i segnali e le chiamate provenienti dalla pagina profilo di un utente.\\
	\textbf{Relazioni d'uso con altri moduli}\\
	\indent Questa classe utilizzerà le seguenti classi:
	\begin{itemize}
		\item Model::\-serverRelation::\-?????????????????????????????????????????????????????????????????????????????.
		\item Services::Utils;
		\item Services::Main;
		\item Services::toPages;
		\item Services::Upload;
		\item \$ scope:Object\\
			\indent Questo campo dati rappresenta l’oggetto che permette la comunicazione tra la view ed il controller, rendendo possibile l’accesso al model mantenendolo sincronizzato, implementando in questo modo il 2-way data binding.
	\end{itemize}
	\textbf{Attributi}\\
    \begin{itemize}
    	\item \textbf{user}()
		\begin{itemize}
			\item \textbf{Accesso}: Private;
			\item \textbf{Tipo}: Object;
			\item \textbf{Descrizione}: oggetto contenente username, password e nuova password derivanti dal form della pagina html.
		\end{itemize}
    	\item \textbf{getData}()
		\begin{itemize}
			\item \textbf{Accesso}: Private;
			\item \textbf{Tipo}: Object;
			\item \textbf{Descrizione}: metodo che ritorna un oggetto contenente i campi dati username, password e newpassword ricavati dallo \$ scope. Le password vengono criptate grazie a Services::\-Utils::\-encrypt(password).
		\end{itemize}
    \end{itemize}
	\textbf{Metodi}
	\begin{itemize}
		\item \textbf{changepassword}()
		\begin{itemize}
			\item \textbf{Accesso}: Public;
			\item \textbf{Tipo di ritorno}: Void;
			\item \textbf{Descrizione}: metodo che controlla se user è definito e, in caso affermativo, richiama Model::\-serverRelation::\-??????????????????????????????????????????????????????????????????????????????????????????????? per cambiare la password dell'utente.
		\end{itemize}
	\end{itemize}
}
\subsubsection{View::Pages::Execution}{
	\textbf{Funzione}\\
	\indent Questa classe si occuperà di gestire i segnali e le chiamate provenienti dalla pagina di esecuzione.\\
	\textbf{Relazioni d'uso con altri moduli}\\
	\indent Questa classe utilizzerà le seguenti classi:
	\begin{itemize}
		\item View::Pages::Execution;
		\item Services::SharedData;
		\item Services::Main;
		\item Services::toPages;
		\item Services::Utils;
		\item \$ scope:Object\\
			\indent Questo campo dati rappresenta l’oggetto che permette la comunicazione tra la view ed il controller, rendendo possibile l’accesso al model mantenendolo sincronizzato, implementando in questo modo il 2-way data binding;
		\item \$ route:Object\\
			\indent Servizio angular che permette la gestione del routing all'interno dell'applicazione.
	\end{itemize}
	\textbf{Metodi}
	\begin{itemize}
		\item Execution(string id): costruttore che dovrà inizializzare gli eventuali attibuti e la pagina View::Pages::Execution in base all'id della presentazione. Execution(string id) controllerà la presenza del token di sessione, usando il metodo goIndex() in caso negativo, e dialogherà con Model::MongoRelations::Loader::LoaderClass per caricare la presentazione dal database;
		\item \textbf{on \$ locationChangeSuccess}()
		\begin{itemize}
			\item \textbf{Accesso}: Public;
			\item \textbf{Tipo di ritorno}: Void;
			\item \textbf{Descrizione}: evento che permette l'interazione tra Angular.js e Impress.js. Dato che il framework Impress.js cambia l'url della pagina in modo dinamico, è necessario istruire il routing di Angular su come comportarsi, in modo tale che non ci sia alcun reindirizzamento inopportuno.
		\end{itemize}
		\item \textbf{translateImpress}(json)
		\begin{itemize}
			\item \textbf{Accesso}: Public;
			\item \textbf{Tipo di ritorno}: Void;
			\item \textbf{Descrizione}: metodo che richiama una funzione JavaScript per la traduzione dell'oggetto json, passato come parametro, in html eseguibile dal framework Impress.js.
		\end{itemize}
		\item \textbf{goHome}()
		\begin{itemize}
			\item \textbf{Accesso}: Public;
			\item \textbf{Tipo di ritorno}: Void;
			\item \textbf{Descrizione}: metodo che richiama Services::\-toPages::\-homepage() per effettuare il reindirizzamento alla pagina home.
		\end{itemize}
		\item \textbf{goEdit}()
		\begin{itemize}
			\item \textbf{Accesso}: Public;
			\item \textbf{Tipo di ritorno}: Void;
			\item \textbf{Descrizione}: metodo che richiama Services::\-toPages::\-editpage() per effettuare il reindirizzamento alla pagina di edit.
		\end{itemize}
	\end{itemize}
}
\subsubsection{Controller::EditController}{
	\textbf{Funzione}\\
	\indent Questa classe si occuperà di mostrare all'utente la possibilità di apportare modifiche ad una presentazione.\\\\
	\textbf{Relazioni d'uso con altri moduli}\\
	\indent Questa classe utilizzerà le seguenti classi:
	\begin{itemize}
		\item View::Pages::Edit;
		\item Model::SlideShow::SlideShowActions::Command:
		\begin{itemize}
			\item ConcreteTextInsertCommand;
			\item ConcreteFrameInsertCommand;
			\item ConcreteImageInsertCommand;
			\item ConcreteSVGInsertCommand;
			\item ConcreteAudioInsertCommand;
			\item ConcreteVideoInsertCommand;
			\item ConcreteBackgroundInsertCommand;
			\item ConcreteTextRemoveCommand;
			\item ConcreteFrameRemoveCommand;
			\item ConcreteImageRemoveCommand;
			\item ConcreteSVGRemoveCommand;
			\item ConcreteAudioRemoveCommand;
			\item ConcreteVideoRemoveCommand;
			\item ConcreteEditSizeCommand;
			\item ConcreteEditPositionCommand;
			\item ConcreteEditRotationCommand;
			\item ConcreteEditColorCommand;
			\item ConcreteEditBackgroundCommand;
			\item ConcreteEditFontCommand;
			\item ConcreteEditContentCommand;
			\item Invoker;
		\end{itemize}
		\item Services::Main;
		\item Services::toPages;
		\item Services::Utils;
		\item Services::SaredData;
		\item Services::Upload;
		\item Model::\-serverRelation::\-MongoRelation::\-Loader;
		\item \$ scope:Object\\
			\indent Questo campo dati rappresenta l’oggetto che permette la comunicazione tra la view ed il controller, rendendo possibile l’accesso al model mantenendolo sincronizzato, implementando in questo modo il 2-way data binding.
		\item \$ q::Object\\
			\indent Servizio Angular che permette di eseguire funzioni in modo asincrono;
		\item \$ mdSideNav::Object\\
			\indent Servizio Angular Material per il controllo della barra laterale;
		\item \$ mdBottomSheet::Object\\
			\indent Servizio Angular Material per il controllo del'oggetto mdBottomsSheet.
	\end{itemize}
	\textbf{Attributi}\\
	\begin{itemize}
		\item \textbf{inv}()
		\begin{itemize}
			\item \textbf{Accesso}: Private;
			\item \textbf{Tipo}: Object;
			\item \textbf{Descrizione}: oggetto che mantiene una istanza di Model::\-Command::\-Invoker.
		\end{itemize}
		\item \textbf{mongo}()
		\begin{itemize}
			\item \textbf{Accesso}: Private;
			\item \textbf{Tipo}: Object;
			\item \textbf{Descrizione}: oggetto che mantiene una istanza di Model::\-serverRelation::\-MongoRelation.
		\end{itemize}
    \end{itemize}
	\textbf{Metodi}
	\begin{itemize}
		\item \textbf{translateEdit}(json)
		\begin{itemize}
			\item \textbf{Accesso}: Public;
			\item \textbf{Tipo di ritorno}: Void;
			\item \textbf{Descrizione}: metodo che richiama una funzione JavaScript per la traduzione dell'oggetto json, passato come parametro, in html permettendo all'utente di modificare la presentazione.
		\end{itemize}
		\item \textbf{goExecute}()
		\begin{itemize}
			\item \textbf{Accesso}: Public;
			\item \textbf{Tipo di ritorno}: Void;
			\item \textbf{Descrizione}: metodo che richiama Services::\-toPages::\-executionpage() per effettuare il reindirizzamento alla pagina di execution.
		\end{itemize}
		\item \textbf{toggleList}()
		\begin{itemize}
			\item \textbf{Accesso}: Public;
			\item \textbf{Tipo di ritorno}: Void;
			\item \textbf{Descrizione}: metodo che gestisce \$ mdBottomSheet e \$ mdSidenav.
		\end{itemize}
		\item \textbf{showPathBottomSheet}(\$ event)
		\begin{itemize}
			\item \textbf{Accesso}: Public;
			\item \textbf{Tipo di ritorno}: Void;
			\item \textbf{Descrizione}: metodo che fa apparire \$ mdBottomSheet per la visualizzazione dei percorsi.
		\end{itemize}
		\item \textbf{show}(id)
		\begin{itemize}
			\item \textbf{Accesso}: Public;
			\item \textbf{Tipo di ritorno}: Void;
			\item \textbf{Descrizione}: metodo che gestisce la comparsa/scomparsa dei bottoni di gestione della presentazione (inserimento, rimozione, etc.) in base all'id dell'elemento html cliccato.
		\end{itemize}
		\item \textbf{inserisciFrame}()
		\begin{itemize}
			\item \textbf{Accesso}: Public;
			\item \textbf{Tipo di ritorno}: Void;
			\item \textbf{Descrizione}: metodo che inserisce un frame nel piano della presentazione, attraverso un'opportuna funziona javascript, e che richiama, utilizzando il metodo execute di inv, Model::\-SlideShow::\-SlideShowActions::\-Command::\-ConcreteFrameInsertCommand(spec) passandogli le specifiche del frame inserito.
		\end{itemize}
		\item \textbf{inserisciTesto}()
		\begin{itemize}
			\item \textbf{Accesso}: Public;
			\item \textbf{Tipo di ritorno}: Void;
			\item \textbf{Descrizione}: metodo che inserisce un elemento testo nel piano della presentazione, attraverso un'opportuna funziona javascript, e che richiama, utilizzando il metodo execute di inv, Model::\-SlideShow::\-SlideShowActions::\-Command::\-ConcreteTextInsertCommand(spec) passandogli le specifiche dell'elemento testo inserito.
		\end{itemize}
		\item \textbf{inserisciImmagini}(files)
		\begin{itemize}
			\item \textbf{Accesso}: Public;
			\item \textbf{Tipo di ritorno}: Void;
			\item \textbf{Descrizione}: metodo che prima richiama Services::\-Upload::\-isImage(frames) per controllare che le estensioni siano corrette, successivamente Services::\-Upload::\-uploadmedia(files) per l'upload dei file immagine. Se l'operazione ha successo, inserisce ogni immagine nel piano della presentazione, attraverso un'opportuna funziona javascript, e richiama, utilizzando il metodo execute di inv, Model::\-SlideShow::\-SlideShowActions::\-Command::\-ConcreteImageInsertCommand(spec) passandogli le specifiche degli elementi immagine inseriti.
		\end{itemize}
		\item \textbf{inserisciAudio}(files)
		\begin{itemize}
			\item \textbf{Accesso}: Public;
			\item \textbf{Tipo di ritorno}: Void;
			\item \textbf{Descrizione}: metodo che prima richiama Services::\-Upload::\-isAudio(frames) per controllare che le estensioni siano corrette, successivamente Services::\-Upload::\-uploadmedia(files) per l'upload dei file audio. Se l'operazione ha successo,  inserisce ogni audio nel piano della presentazione, attraverso un'opportuna funziona javascript, e richiama, utilizzando il metodo execute di inv, Model::\-SlideShow::\-SlideShowActions::\-Command::\-ConcreteAudioInsertCommand(spec) passandogli le specifiche degli elementi audio inseriti.
		\end{itemize}
		\item \textbf{inserisciVideo}(files)
		\begin{itemize}
			\item \textbf{Accesso}: Public;
			\item \textbf{Tipo di ritorno}: Void;
			\item \textbf{Descrizione}: metodo che prima richiama Services::\-Upload::\-isVideo(frames) per controllare che le estensioni siano corrette, successivamente Services::\-Upload::\-uploadmedia(files) per l'upload dei file video. Se l'operazione ha successo,  inserisce ogni video nel piano della presentazione, attraverso un'opportuna funziona javascript, e richiama, utilizzando il metodo execute di inv, Model::\-SlideShow::\-SlideShowActions::\-Command::\-ConcreteVideoInsertCommand(spec) passandogli le specifiche degli elementi video inseriti.
		\end{itemize}
		\item \textbf{rimuoviElemento}()
		\begin{itemize}
			\item \textbf{Accesso}: Public;
			\item \textbf{Tipo di ritorno}: Void;
			\item \textbf{Descrizione}: metodo che rimuove l'elemento corrente dal piano della presentazione e successivamente, in base al tipo dell'elemento, utilizzando il metodo execute di inv, richiama uno tra i seguenti:
			\begin{itemize}
				\item ConcreteTextRemoveCommand;
				\item ConcreteFrameRemoveCommand;
				\item ConcreteImageRemoveCommand;
				\item ConcreteAudioRemoveCommand;
				\item ConcreteVideoRemoveCommand.
			\end{itemize}
		\end{itemize}
		\item \textbf{cambiaColoreSfondo}(color)
		\begin{itemize}
			\item \textbf{Accesso}: Public;
			\item \textbf{Tipo di ritorno}: Void;
			\item \textbf{Descrizione}: metodo che assegna al background della presentazione il valore color, passato come parametro. Successivamente richiama, utilizzando la funzione execute di inv, Model::\-SlideShow::\-SlideShowActions::\-Command::\-ConcreteBackgroundInsertCommand(spec) passandogli le specifiche del background.
		\end{itemize}
		\item \textbf{cambiaImmagineSfondo}(files)
		\begin{itemize}
			\item \textbf{Accesso}: Public;
			\item \textbf{Tipo di ritorno}: Void;
			\item \textbf{Descrizione}: metodo che assegna al background della presentazione un nuovo sfondo in base al parametro files. Successivamente richiama, utilizzando la funzione execute di inv, Model::\-SlideShow::\-SlideShowActions::\-Command::\-ConcreteBackgroundInsertCommand(spec) passandogli le specifiche del background.
		\end{itemize}
		\item \textbf{rimuoviSfondo}()
		\begin{itemize}
			\item \textbf{Accesso}: Public;
			\item \textbf{Tipo di ritorno}: Void;
			\item \textbf{Descrizione}: metodo che rimuove colore e sfondo dal background e che successivamente, utilizzando la funzione execute di inv, richiama Model::\-SlideShow::\-SlideShowActions::\-Command::\-ConcreteBackgroundInsertCommand(spec) passandogli le specifiche del background.
		\end{itemize}
		\item \textbf{cambiaColoreSfondoFrame}(color)
		\begin{itemize}
			\item \textbf{Accesso}: Public;
			\item \textbf{Tipo di ritorno}: Void;
			\item \textbf{Descrizione}: metodo che assegna al background del frame selezionato il valore color, passato come parametro. Successivamente richiama, utilizzando la funzione execute di inv, Model::\-SlideShow::\-SlideShowActions::\-Command::\-ConcreteEditBackgroundCommand(spec) passandogli le specifiche del background del frame.
		\end{itemize}
		\item \textbf{cambiaImmagineSfondoFrame}(files)
		\begin{itemize}
			\item \textbf{Accesso}: Public;
			\item \textbf{Tipo di ritorno}: Void;
			\item \textbf{Descrizione}: metodo che assegna al background del frame selezionato un nuovo sfondo in base al parametro files. Successivamente richiama, utilizzando la funzione execute di inv, Model::\-SlideShow::\-SlideShowActions::\-Command::\-ConcreteEditBackgroundCommand(spec) passandogli le specifiche del background.
		\end{itemize}
		\item \textbf{rimuoviSfondoFrame}()
		\begin{itemize}
			\item \textbf{Accesso}: Public;
			\item \textbf{Tipo di ritorno}: Void;
			\item \textbf{Descrizione}: metodo che rimuove colore e sfondo dal background del frame selezionato e che successivamente, utilizzando la funzione execute di inv, richiama Model::\-SlideShow::\-SlideShowActions::\-Command::\-ConcreteEditBackgroundCommand(spec) passandogli le specifiche del background.
		\end{itemize}
		\item \textbf{mediaControl}()
		\begin{itemize}
			\item \textbf{Accesso}: Public;
			\item \textbf{Tipo di ritorno}: Void;
			\item \textbf{Descrizione}: metodo che attiva le funzionalità per la riproduzione di un file media.
		\end{itemize}
		\item \textbf{ruotaElemento}(value)
		\begin{itemize}
			\item \textbf{Accesso}: Public;
			\item \textbf{Tipo di ritorno}: Void;
			\item \textbf{Descrizione}: metodo che ruota l'elemento selezionato in base al parametro value. Successivamente richiama, utilizzando la funzione execute di inv, Model::\-SlideShow::\-SlideShowActions::\-Command::\-ConcreteEditRotationCommand(spec) passandogli le specifiche della nuova rotazione.
		\end{itemize}
		\item \textbf{muoviElemento}()
		\begin{itemize}
			\item \textbf{Accesso}: Public;
			\item \textbf{Tipo di ritorno}: Void;
			\item \textbf{Descrizione}: metodo che, in base al nuovo posizionamento dell'elemento selezionato all'interno del piano della presentazione, richiama, utilizzando la funzione execute di inv, Model::\-SlideShow::\-SlideShowActions::\-Command::\-ConcreteEditPositionCommand(spec) passandogli le specifiche della nuova posizione.
		\end{itemize}
		\item \textbf{ridimensionaElemento}()
		\begin{itemize}
			\item \textbf{Accesso}: Public;
			\item \textbf{Tipo di ritorno}: Void;
			\item \textbf{Descrizione}: metodo che, in base alla nuova dimensione dell'elemento selezionato all'interno del piano della presentazione, richiama, utilizzando la funzione execute di inv, Model::\-SlideShow::\-SlideShowActions::\-Command::\-ConcreteEditSizeCommand(spec) passandogli le specifiche della nuova dimensione.
		\end{itemize}
		\item \textbf{aggiungiMainPath}()
		\begin{itemize}
			\item \textbf{Accesso}: Public;
			\item \textbf{Tipo di ritorno}: Void;
			\item \textbf{Descrizione}: metodo che richiama, utilizzando la funzione execute di inv, Model::\-SlideShow::\-SlideShowActions::\-Command::\-AddToMainPathCommand(spec) passandogli le specifiche del frame da aggiungere al percorso principale.
		\end{itemize}
		\item \textbf{rimuoviMainPath}()
		\begin{itemize}
			\item \textbf{Accesso}: Public;
			\item \textbf{Tipo di ritorno}: Void;
			\item \textbf{Descrizione}: metodo che richiama, utilizzando la funzione execute di inv, Model::\-SlideShow::\-SlideShowActions::\-Command::\-RemoveFromMainPathCommand(spec) passandogli le specifiche del frame da togliere dal percorso principale.
		\end{itemize}
	\end{itemize}
}