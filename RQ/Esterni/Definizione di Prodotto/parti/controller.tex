\section {Specifica classi del front-end}
\subsection{Package Premi::App}

\subsection {Package Premi::Services}
\label{sec:services}
\textbf{\tipo}: i servizi sono degli oggetti che incapsulano del codice che si occupa di eseguire uno specifico compito, compito che poi sarà utilizzato all’interno di una o più parti dell’applicazione.\\\\
\textbf{\relaz}: Comunica con il controller per la gestione delle funzioni da eseguire e con il model per la gestione delle componenti necessarie.\\\\
\subsubsection{Services::Main}
\subsubsection{Services::Utils}
\subsubsection{Services::SharedData}
\subsubsection{Services::ToPages}


\subsection{Package Premi::Controller}{
		\label{sec:controller}
		Tutti i package seguenti appartengono al package Premi, quindi per ognuno di essi lo scope sarà: Premi::[nome package].\\\\
		\textbf{\tipo}: contiene le classi che gestiscono i segnali e le chiamate effettuati dalla View.\\
		\textbf{\relaz}: comunica con il Model per la gestione del profilo e delle presentazioni.\\
		\subsubsection{Controller::HeaderController}
		\label{sub:HeaderController}
		\textbf{Funzione}\\
		\indent Questa classe si occuperà di controllare l'Header dell'applicazione.\\
		\textbf{Relazioni d'uso con altri moduli}\\
		\indent Questa classe utilizzerà le seguenti classi:
		\begin{itemize}
		\item Services::Utils;
		\item Services::Main;
		\item Services::toPages.
		\end{itemize}
		\textbf{Attributi}\\
		\begin{itemize}
		\item scope:Object\\
		Questo campo dati rappresenta l’oggetto che permette la comunicazione tra la view ed il controller, rendendo possibile l’accesso al model mantenendolo sincronizzato, implementando in questo modo il 2-way data binding.
		\item rootScope:Object\\
		Questo campo dati rappresenta lo scope radice dell’applicazione. Tutti gli altri scope	discendono da questo.
		\end{itemize}
		\textbf{Metodi}
		\begin{itemize}
		\item +error()
		\item +who()
		\item +isToken()
		\item +goLogin()
		\item +goRegistrazione()
		\item +goHome()
		\item +goProfile()
		\item +logout()
		\end{itemize}
\subsubsection{Controller::AccessController}{
				\label{sub:AccessController}
				\textbf{Funzione}\\
					\indent Questa classe si occuperà di controllare che le credenziali di accesso siano corrette nel caso dell'autenticazione oppure di registrare un nuovo utente.\\
				\textbf{Relazioni d'uso con altri moduli}\\
					\indent Questa classe utilizzerà le seguenti classi:
				\begin{itemize}
					\item View::Pages::Index;
					\item Model::MongoRelations::AccessControl::Authentication;
					\item Model::MongoRelations::AccessControl::Registration.
					\item Services::Utils;
					\item Services::Main;
					\item Services::toPages.
				\end{itemize}
				\textbf{Attributi}\\
	            \begin{itemize}
	            \item scope:Object\\
	            Questo campo dati rappresenta l’oggetto che permette la comunicazione tra la view ed il controller, rendendo possibile l’accesso al model mantenendolo sincronizzato, implementando in questo modo il 2-way data binding.
	            \end{itemize}
				\textbf{Metodi}
					\begin{itemize}
                    \item +getData()
                    \item +reset()
                    \item +login()
                    \item +registration()
				\end{itemize}
			}
			\subsubsection{Controller::HomeController}{
					\label{sub:homecontroller}
					\textbf{Funzione}\\
					\indent Questa classe si occuperà di gestire i segnali e le chiamate provenienti dalla pagina View::Pages::Home.\\
					\textbf{Relazioni d'uso con altri moduli}\\
					\indent Questa classe utilizzerà le seguenti classi:
					\begin{itemize}
						\item View::Pages::Home;
						\item Model::MongoRelations::AccessControl::LoaderClass;
						\item Model::MongoRelations::AccessControl::Authentication;
						\item Services::Utils;
						\item Services::Main;
						\item Services::toPages.
					\end{itemize}
					\textbf{Attributi}\\
				    \begin{itemize}
					\item scope:Object\\
				    Questo campo dati rappresenta l’oggetto che permette la comunicazione tra la view ed il controller, rendendo possibile l’accesso al model mantenendolo sincronizzato, implementando in questo modo il 2-way data binding.
				    \item window::Object\\
					\end{itemize}
					\textbf{Metodi}
					\begin{itemize}
					\item +update():
					\item +goEdit(slideId):
					\item +goExecute(slideId):
					\item +goProfile():
					\item +getSS():
					\item +deleteSlideShow(slideId):
					\item +renameSlideShow(nameSS):
					\item +createSlideShow():
					\item +createSlideShow()
                    
					\end{itemize}
				}
				\subsubsection{Controller::ProfileController}{
					\textbf{Funzione}\\
					\indent Questa classe gestirà le operazioni e la logica applicativa riguardante la pagina profilo di un utente.\\\\
					\textbf{Relazioni d'uso con altri moduli}\\
					\indent Questa classe utilizzerà le seguenti classi:
					\begin{itemize}
						\item Model::MongoRelations::AccessControl::Authentication((DA COMPLETARE)).
						\item Services::Utils;
						\item Services::Main;
						\item Services::toPages;
						\item Services::Upload.
					\end{itemize}
					\textbf{Attributi}\\
					\begin{itemize}
					\item scope:Object\\
					Questo campo dati rappresenta l’oggetto che permette la comunicazione tra la view ed il controller, rendendo possibile l’accesso al model mantenendolo sincronizzato, implementando in questo modo il 2-way data binding.
					\item formData\\
					Questo campo dati rappresenta
					\end{itemize}
					\textbf{Metodi}
					\begin{itemize}
                    \item +getData()
                    \item +changepassword()
                    \item +uploadpedia(files)
					\end{itemize}
				}
				\subsubsection{View::Pages::Execution}{
					\textbf{Funzione}\\
					\indent Questa classe si occuperà di gestire i segnali che arrivano da View::Pages::Execution.\\\\
					\textbf{Relazioni d'uso con altri moduli}\\
					\indent Questa classe utilizzerà le seguenti classi:
					\begin{itemize}
						\item View::Pages::Execution;
						\item Model::MongoRelations::Loader::LoaderClass;
					\end{itemize}
					\textbf{Attributi}\\
					\indent Al momento non sono stati previsti degli attributi.\\\\
					\textbf{Metodi}
					\begin{itemize}
						\item Execution(string id): costruttore che dovrà inizializzare gli eventuali attibuti e la pagina View::Pages::Execution in base all'id della presentazione. Execution(string id) controllerà la presenza del token di sessione, usando il metodo goIndex() in caso negativo, e dialogherà con Model::MongoRelations::Loader::LoaderClass per caricare la presentazione dal database;
						\item +goHome(): metodo che reindirizza alla pagina View::Pages::Home;
						\item +goIndex(): metodo che reindirizza alla pagina View::Pages::Index;
						\item +goEdit(): metodo che reindirizza alla pagina View::Pages::Edit.
					\end{itemize}
				}
				\subsubsection{Controller::EditController}{
					\textbf{Funzione}\\
					\indent Questa classe si occuperà di mostrare all'utente la possibilità di apportare modifiche ad una presentazione.\\\\
					\textbf{Relazioni d'uso con altri moduli}\\
					\indent Questa classe utilizzerà le seguenti classi:
					\begin{itemize}
						\item View::Pages::Edit;
						\item Model::SlideShow::SlideShowActions::Command:
						\begin{itemize}
													\item ConcreteTextInsertCommand;
													\item ConcreteFrameInsertCommand;
													\item ConcreteImageInsertCommand;
													\item ConcreteSVGInsertCommand;
													\item ConcreteAudioInsertCommand;
													\item ConcreteVideoInsertCommand;
													\item ConcreteBackgroundInsertCommand;
													\item ConcreteTextRemoveCommand;
													\item ConcreteFrameRemoveCommand;
													\item ConcreteImageRemoveCommand;
													\item ConcreteSVGRemoveCommand;
													\item ConcreteAudioRemoveCommand;
													\item ConcreteVideoRemoveCommand;
													\item ConcreteBackgroundRemoveCommand;
													\item ConcreteEditSizeCommand;
													\item ConcreteEditPositionCommand;
													\item ConcreteEditRotationCommand;
													\item ConcreteEditColorCommand;
													\item ConcreteEditBackgroundCommand;
													\item ConcreteEditFontCommand;
													\item ConcreteEditContentCommand; [[[[[[[[[[[[[[[[ESISTE????]]]]]]]]]]]]]]]]
													\item Invoker;
					\end{itemize}
						\item Services::Main;
						\item Services::toPages;
						\item Services::Utils;
						\item Services::SaredData;
						\item Services::Upload;
						\item Model::MongoRelations::Loader::Autenticazione;
						\item Model::MongoRelations::Loader::Loaderclass;
					\end{itemize}
					\textbf{Attributi}\\
					\begin{itemize}
					\item scope:Object\\
					Questo campo dati rappresenta l’oggetto che permette la comunicazione tra la view ed il controller, rendendo possibile l’accesso al model mantenendolo sincronizzato, implementando in questo modo il 2-way data binding.
					\item q::Object\\
					\item mdSideNav::Object
					\item mdBottomSheet::Object
				    \end{itemize}
					\textbf{Metodi}
					\begin{itemize}
					\item goEdit(slideId):
					\item goExecute(slideId):
					\item toggleList():
					\item backgroundManage(bool):
					\item slideShowBackgroundManage(bool):
					\item pathsManage(bool):
					\item rotation(bool):
					\item show(id):
					\item mainPath()
					\item showPathBottomSheet(event)
					\item inserisciFrame()
					\item inserisciTesto()
					\item uploadmedia(files)
					\item inserisciImmagine(files)
					\item inserisciAudio(files)
					\item inserisciVideo(files)
					\item rimuoviFrame(id)
					\item rimuoviSfondo()
					\item cambiaColoreSfondo(color)
					\item cambiaColoreSfondoFrame(color)
					\item mediaControl()
					\item announceClick(index)
					\item setRotation(new rotation)
					\item addToMain()
					\end{itemize}
				}
				\subsection{Controller::BottomSheetController}
				}