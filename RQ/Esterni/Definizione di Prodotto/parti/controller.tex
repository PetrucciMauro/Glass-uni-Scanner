\section {Specifica classi del front-end}
\subsection{Premi::App}
\label{sec:premiapp}
		\textbf{Funzione}\\
		\indent Questa classe si occuperà di fare il boostrap dell'applicazione instanziando la rootscope  e iniettando tutti i moduli necessari.\\
		\textbf{Relazioni d'uso con altri moduli}\\
		\indent Questa classe utilizzerà le seguenti classi:
		\begin{itemize}
		\item Services::Utils;
		\item Services::Main;
		\item Services::toPages.
		\end{itemize}
		\textbf{Attributi}\\
		\begin{itemize}
		\item scope:Object\\
		Questo campo dati rappresenta l’oggetto che permette la comunicazione tra la view ed il controller, rendendo possibile l’accesso al model mantenendolo sincronizzato, implementando in questo modo il 2-way data binding;
		\item rootScope:Object\\
		Questo campo dati rappresenta lo scope radice dell’applicazione. Tutti gli altri scope	discendono da questo;
		\item \textdollar routeProvider::Object\\
		Questo campo dati rappresenta il servizio che collega tra loro controller, view e l'URL corrente nel browser;
		\item \textdollar mdIconProvider\\
		\item \textdollar mdThemingProvider\\
		\item \textdollar httpProvider\\
		\item \textdollar provide\\
		\item \textdollar locationProvider\\
		\end{itemize}
		\textbf{Metodi}
		\begin{itemize}
		\item run
		\item config
		\item interceptors
		\item decorator
		\end{itemize}

\subsection {Package Premi::Services}
\label{sec:services}
\textbf{\tipo}: i servizi sono degli oggetti che incapsulano del codice che si occupa di eseguire uno specifico compito, il quale sarà poi utilizzato all’interno di una o più parti dell’applicazione.\\
\textbf{\relaz}: comunica con il controller per la gestione delle funzioni da eseguire e con il model per la gestione delle componenti necessarie.\\
\subsubsection{Services::Main}{
		\label{sub:servicesMain}
		\textbf{Funzione}\\
		\indent Questa classe si occuperà di eseguire le funzioni base dell'applicazione, in particolare autenticazione e registrazione al server degli utenti.\\
		\textbf{Relazioni d'uso con altri moduli}\\
		\indent Questa classe utilizzerà le seguenti classi:
		\begin{itemize}
			\item Services::Utils;
			\item localStorage:Object\\
				\indent Servizio angular che permette il salvataggio in locale di oggetti necessari al garantimento delle funzioni dell'applicazione.
		\end{itemize}
		\textbf{Attributi}
		\begin{itemize}
			\item \textbf{login}
			\begin{itemize}
				\item \textbf{Accesso}: Private;
				\item \textbf{Tipo}: Oggetto;
				\item \textbf{Descrizione}: oggetto che mantiene la sessione corrente.
			\end{itemize}
		\end{itemize}
		\textbf{Metodi}
		\begin{itemize}
			\item \textbf{value}()
			\begin{itemize}
				\item \textbf{Accesso}: Private;
				\item \textbf{Tipo di ritorno}: Void;
				\item \textbf{Descrizione}: metodo che controlla se è stato effettuato un refresh della pagina, in tal caso ripristina login.
			\end{itemize}
			\item \textbf{register}(formData, success, error)
			\begin{itemize}
				\item \textbf{Accesso}: Public;
				\item \textbf{Tipo di ritorno}: Void;
				\item \textbf{Descrizione}: metodo che, attraverso l'oggetto formData contenente le credenziali di accesso, effettua la registrazione al server di un nuovo utente. Se l'operazione ha successo viene invocato success altrimenti error.
			\end{itemize}
			\item \textbf{login}(formData, success, error)
			\begin{itemize}
				\item \textbf{Accesso}: Public;
				\item \textbf{Tipo di ritorno}: Void;
				\item \textbf{Descrizione}: metodo che, attraverso l'oggetto formData contenente le credenziali di accesso, effettua l'autenticazione al server di un utente. Se l'operazione ha successo viene invocato success altrimenti error.
			\end{itemize}
			\item \textbf{logout}(success, error)
			\begin{itemize}
				\item \textbf{Accesso}: Public;
				\item \textbf{Tipo di ritorno}: Void;
				\item \textbf{Descrizione}: metodo che effettua il logout dal server. Se l'operazione ha successo viene invocato success altrimenti error.
			\end{itemize}
			\item \textbf{changepassword}(formData, success, error)
			\begin{itemize}
				\item \textbf{Accesso}: Public;
				\item \textbf{Tipo di ritorno}: Void;
				\item \textbf{Descrizione}: metodo che, attraverso l'oggetto formData contenente le credenziali di accesso e la nuova password, effettua il cambio della password di un utente. Se l'operazione ha successo viene invocato success altrimenti error.
			\end{itemize}
			\item \textbf{getToken}()
			\begin{itemize}
				\item \textbf{Accesso}: Public;
				\item \textbf{Tipo di ritorno}: JSONWebToken;
				\item \textbf{Descrizione}: metodo che ritorna il token di sessione.
			\end{itemize}
			\item \textbf{getUser}()
			\begin{itemize}
				\item \textbf{Accesso}: Public;
				\item \textbf{Tipo di ritorno}: Object;
				\item \textbf{Descrizione}: metodo che ritorna l'utente attualmente autenticato con il server.
			\end{itemize}
		\end{itemize} 
}
\subsubsection{Services::Upload}{
		\label{sub:servicesUpload}
		\textbf{Funzione}\\
		\indent Questa classe si occuperà di eseguire l'upload di file media nel database.\\
		\textbf{Relazioni d'uso con altri moduli}\\
		\indent Questa classe utilizzerà le seguenti classi:
		\begin{itemize}
			\item Services::Main;
			\item Services::Utils;
		\end{itemize}
		\textbf{Attributi}
		\begin{itemize}
			\item \textbf{image}
			\begin{itemize}
				\item \textbf{Accesso}: Private;
				\item \textbf{Tipo}: Array;
				\item \textbf{Descrizione}: array contenente i formati immagini accettati per l'upload.
			\end{itemize}
			\item \textbf{audio}
			\begin{itemize}
				\item \textbf{Accesso}: Private;
				\item \textbf{Tipo}: Array;
				\item \textbf{Descrizione}: array contenente i formati audio accettati per l'upload.
			\end{itemize}
			\item \textbf{video}
			\begin{itemize}
				\item \textbf{Accesso}: Private;
				\item \textbf{Tipo}: Array;
				\item \textbf{Descrizione}: array contenente i formati video accettati per l'upload.
			\end{itemize}
		\end{itemize}
		\textbf{Metodi}
		\begin{itemize}
			\item \textbf{uploadmedia}(files, success, error)
			\begin{itemize}
				\item \textbf{Accesso}: Public;
				\item \textbf{Tipo di ritorno}: Void;
				\item \textbf{Descrizione}: metodo che effettua l'upload dei file contenuti nell'array files passato come parametro. Se l'operazione ha successo viene invocato success altrimenti error.
			\end{itemize}
			\item \textbf{isImage}(files)
			\begin{itemize}
				\item \textbf{Accesso}: Public;
				\item \textbf{Tipo di ritorno}: Bool;
				\item \textbf{Descrizione}: metodo che ritorna true se i file contenuti nell'array files, passato come parametro, rispettano almeno uno tra i formati contenuti nell'array image, altrimenti ritorna false.
			\end{itemize}
			\item \textbf{isAudio}(files)
			\begin{itemize}
				\item \textbf{Accesso}: Public;
				\item \textbf{Tipo di ritorno}: Bool;
				\item \textbf{Descrizione}: metodo che ritorna true se i file contenuti nell'array files, passato come parametro, rispettano almeno uno tra i formati contenuti nell'array audio, altrimenti ritorna false.
			\end{itemize}
			\item \textbf{isVideo}(files)
			\begin{itemize}
				\item \textbf{Accesso}: Public;
				\item \textbf{Tipo di ritorno}: Bool;
				\item \textbf{Descrizione}: metodo che ritorna true se i file contenuti nell'array files, passato come parametro, rispettano almeno uno tra i formati contenuti nell'array video, altrimenti ritorna false.
			\end{itemize}
			\item \textbf{getFileUrl}()
			\begin{itemize}
				\item \textbf{Accesso}: Public;
				\item \textbf{Tipo di ritorno}: Stringa;
				\item \textbf{Descrizione}: metodo che ritorna il percorso di salvataggio dei file dell'utente corrente.
			\end{itemize}
		\end{itemize} 
}

\subsubsection{Services::Utils}{
		\label{sub:servicesUtils}
		\textbf{Funzione}\\
		\indent Questa classe si occuperà di eseguire piccole funzionalità utili ad ogni parte dell'applicazione.\\
		\textbf{Metodi}
		\begin{itemize}
			\item \textbf{decodeToken}(token)
			\begin{itemize}
				\item \textbf{Accesso}: Public;
				\item \textbf{Tipo di ritorno}: Object;
				\item \textbf{Descrizione}: metodo che decodifica token, passato come parametro, e ritorna l'oggetto utente corrispondente.
			\end{itemize}
			\item \textbf{grade}(password)
			\begin{itemize}
				\item \textbf{Accesso}: Public;
				\item \textbf{Tipo di ritorno}: String;
				\item \textbf{Descrizione}: metodo che determina la robustezza del parametro password. La lunghezza minima di una password è stata impostata a sei caratteri.
			\end{itemize}
			\item \textbf{hostname}()
			\begin{itemize}
				\item \textbf{Accesso}: Public;
				\item \textbf{Tipo di ritorno}: String;
				\item \textbf{Descrizione}: metodo che ritorna il dominio dell'applicazione.
			\end{itemize}
			\item \textbf{isUndefined}(object)
			\begin{itemize}
				\item \textbf{Accesso}: Public;
				\item \textbf{Tipo di ritorno}: Bool;
				\item \textbf{Descrizione}: metodo che ritorna true se il parametro object risulta indefinito, altrimento ritorna false.
			\end{itemize}
			\item \textbf{isObject}(object)
			\begin{itemize}
				\item \textbf{Accesso}: Public;
				\item \textbf{Tipo di ritorno}: Void;
				\item \textbf{Descrizione}: metodo che ritorna true se il parametro object risulta definito, altrimento ritorna false.
			\end{itemize}
			\item \textbf{encrypt}(string)
			\begin{itemize}
				\item \textbf{Accesso}: Public;
				\item \textbf{Tipo di ritorno}: String;
				\item \textbf{Descrizione}: metodo che ritorna il parametro string criptato. Il metodo di criptaggio scelto è lo SHA-1.
			\end{itemize}
		\end{itemize}
}

\subsubsection{Services::SharedData}{
		\label{sub:servicesSharedData}
		\textbf{Funzione}\\
		\indent Questa classe mantiene in memoria la presentazione sulla quale l'utente sta lavorando.\\
		\textbf{Relazioni d'uso con altri moduli}\\
		\indent Questa classe utilizzerà le seguenti classi:
		\begin{itemize}
			\item Services::Utils;
			\item Services::Main;
			\item localStorage:Object\\
				\indent Servizio angular che permette il salvataggio in locale di oggetti necessari al garantimento delle funzioni dell'applicazione.
		\end{itemize}
		\textbf{Attributi}\\
		\begin{itemize}
			\item \textbf{idExecution}
			\begin{itemize}
				\item \textbf{Accesso}: Private;
				\item \textbf{Tipo}: Object;
				\item \textbf{Descrizione}: oggetto che rappresenta l'attuale presentazione in esecuzione.
			\end{itemize}
			\item \textbf{idEdit}
			\begin{itemize}
				\item \textbf{Accesso}: Private;
				\item \textbf{Tipo}: Object;
				\item \textbf{Descrizione}: oggetto che rappresenta l'attuale presentazione aperta in modalità modifica.
			\end{itemize}
		\end{itemize}
		\textbf{Metodi}
		\begin{itemize}
			\item \textbf{forExecution}(idSlideShow)
			\begin{itemize}
				\item \textbf{Accesso}: Public;
				\item \textbf{Tipo di ritorno}: Object;
				\item \textbf{Descrizione}: metodo che, nel caso in cui il parametro idSlideShow sia definito, richiama il metodo Model::\-ServerRelation::\-MongoRelation::\-getPresentation(idSlideShow) assegnando il risultato a idExecution. In ogni caso idExecution viene ritornato.
			\end{itemize}
			\item \textbf{forEdit}(idSlideShow)
			\begin{itemize}
				\item \textbf{Accesso}: Public;
				\item \textbf{Tipo di ritorno}: Object;
				\item \textbf{Descrizione}: metodo che, nel caso in cui il parametro idSlideShow sia definito, richiama il metodo Model::\-ServerRelation::\-MongoRelation::\-getPresentation(idSlideShow) assegnando il risultato a idEdit. In ogni caso idEdit viene ritornato.
			\end{itemize}
		\end{itemize}
}
\subsubsection{Services::toPages}{
		\label{sub:servicestoPages}
		\textbf{Funzione}\\
		\indent Questa classe si occuperà di eseguire i reindirizzamenti alle pagine corrette.\\
		\textbf{Relazioni d'uso con altri moduli}\\
		\indent Questa classe utilizzerà le seguenti classi:
		\begin{itemize}
			\item Services::Utils;
			\item Services::Main;
			\item Services::SharedData;
			\item \$ http:Object\\
				\indent Servizio Angular che permette la comunicazione in remoto con un server.
			\item \$ location:Object\\
				\indent Servizio Angular che gestisce gli indirizzi URL.
		\end{itemize}
		\textbf{Metodi}
		\begin{itemize}
			\item \textbf{sendRequest}(dest, success, error)
			\begin{itemize}
				\item \textbf{Accesso}: Private;
				\item \textbf{Tipo di ritorno}: Object;
				\item \textbf{Descrizione}: metodo che ritorna una richiesta http all'indirizzo definito dal parametro dest. Se l'operazione ha successo viene invocato success altrimenti error.
			\end{itemize}
			\item \textbf{loginpage}()
			\begin{itemize}
				\item \textbf{Accesso}: Public;
				\item \textbf{Tipo di ritorno}: Object;
				\item \textbf{Descrizione}: metodo che permette di accedere alla pagina di Login. Esso richiama il metodo sendRequest() il quale, se ha successo, reindirizza alla pagina richiesta.
			\end{itemize}
			\item \textbf{registrazionepage}()
			\begin{itemize}
				\item \textbf{Accesso}: Public;
				\item \textbf{Tipo di ritorno}: Object;
				\item \textbf{Descrizione}: metodo che permette di accedere alla pagina di Registrazione. Esso richiama il metodo sendRequest() il quale, se ha successo, reindirizza alla pagina richiesta.
			\end{itemize}
			\item \textbf{homepage}()
			\begin{itemize}
				\item \textbf{Accesso}: Public;
				\item \textbf{Tipo di ritorno}: Object;
				\item \textbf{Descrizione}: metodo che permette di accedere alla pagina Home. Esso richiama il metodo sendRequest() il quale, se ha successo, reindirizza alla pagina richiesta.
			\end{itemize}
			\item \textbf{profilepage}()
			\begin{itemize}
				\item \textbf{Accesso}: Public;
				\item \textbf{Tipo di ritorno}: Object;
				\item \textbf{Descrizione}: metodo che permette di accedere alla pagina Profile. Esso richiama il metodo sendRequest() il quale, se ha successo, reindirizza alla pagina richiesta.
			\end{itemize}
			\item \textbf{editpage}(slideId)
			\begin{itemize}
				\item \textbf{Accesso}: Public;
				\item \textbf{Tipo di ritorno}: Object;
				\item \textbf{Descrizione}: metodo che permette di accedere alla pagina di Edit. Esso richiama il metodo sendRequest() il quale, se ha successo, reindirizza alla pagina richiesta e richiama il metodo SharedData.forEdit() passandogli il parametro slideId.
			\end{itemize}
			\item \textbf{executionpage}(slideId)
			\begin{itemize}
				\item \textbf{Accesso}: Public;
				\item \textbf{Tipo di ritorno}: Object;
				\item \textbf{Descrizione}: metodo che permette di accedere alla pagina di Execution. Esso richiama il metodo sendRequest() il quale, se ha successo, reindirizza alla pagina richiesta e richiama il metodo SharedData.forEdit() passandogli il parametro slideId.
			\end{itemize}
		\end{itemize} 
}

\subsection{Package Premi::Controller}{
		\label{sec:controller}
		Tutti i package seguenti appartengono al package Premi, quindi per ognuno di essi lo scope sarà: Premi::[nome package].\\\\
		\textbf{\tipo}: contiene le classi che gestiscono i segnali e le chiamate effettuati dalla View.\\
		\textbf{\relaz}: comunica con il Model per la gestione del profilo e delle presentazioni.\\
		\subsubsection{Controller::HeaderController}
		\label{sub:HeaderController}
		\textbf{Funzione}\\
		\indent Questa classe si occuperà di controllare l'Header dell'applicazione.\\
		\textbf{Relazioni d'uso con altri moduli}\\
		\indent Questa classe utilizzerà le seguenti classi:
		\begin{itemize}
		\item Services::Utils;
		\item Services::Main;
		\item Services::toPages.
		\end{itemize}
		\textbf{Attributi}\\
		\begin{itemize}
		\item scope:Object\\
		Questo campo dati rappresenta l’oggetto che permette la comunicazione tra la view ed il controller, rendendo possibile l’accesso al model mantenendolo sincronizzato, implementando in questo modo il 2-way data binding.
		\item rootScope:Object\\
		Questo campo dati rappresenta lo scope radice dell’applicazione. Tutti gli altri scope	discendono da questo.
		\end{itemize}
		\textbf{Metodi}
		\begin{itemize}
		\item +error()
		\item +who()
		\item +isToken()
		\item +goLogin()
		\item +goRegistrazione()
		\item +goHome()
		\item +goProfile()
		\item +logout()
		\end{itemize}
\subsubsection{Controller::AccessController}{
				\label{sub:AccessController}
				\textbf{Funzione}\\
					\indent Questa classe si occuperà di controllare che le credenziali di accesso siano corrette nel caso dell'autenticazione oppure di registrare un nuovo utente.\\
				\textbf{Relazioni d'uso con altri moduli}\\
					\indent Questa classe utilizzerà le seguenti classi:
				\begin{itemize}
					\item View::Pages::Index;
					\item Model::MongoRelations::AccessControl::Authentication;
					\item Model::MongoRelations::AccessControl::Registration.
					\item Services::Utils;
					\item Services::Main;
					\item Services::toPages.
				\end{itemize}
				\textbf{Attributi}\\
	            \begin{itemize}
	            \item scope:Object\\
	            Questo campo dati rappresenta l’oggetto che permette la comunicazione tra la view ed il controller, rendendo possibile l’accesso al model mantenendolo sincronizzato, implementando in questo modo il 2-way data binding.
	            \end{itemize}
				\textbf{Metodi}
					\begin{itemize}
                    \item +getData()
                    \item +reset()
                    \item +login()
                    \item +registration()
				\end{itemize}
			}
			\subsubsection{Controller::HomeController}{
					\label{sub:homecontroller}
					\textbf{Funzione}\\
					\indent Questa classe si occuperà di gestire i segnali e le chiamate provenienti dalla pagina View::Pages::Home.\\
					\textbf{Relazioni d'uso con altri moduli}\\
					\indent Questa classe utilizzerà le seguenti classi:
					\begin{itemize}
						\item View::Pages::Home;
						\item Model::MongoRelations::AccessControl::LoaderClass;
						\item Model::MongoRelations::AccessControl::Authentication;
						\item Services::Utils;
						\item Services::Main;
						\item Services::toPages.
					\end{itemize}
					\textbf{Attributi}\\
				    \begin{itemize}
					\item scope:Object\\
				    Questo campo dati rappresenta l’oggetto che permette la comunicazione tra la view ed il controller, rendendo possibile l’accesso al model mantenendolo sincronizzato, implementando in questo modo il 2-way data binding.
				    \item window::Object\\
					\end{itemize}
					\textbf{Metodi}
					\begin{itemize}
					\item +update():
					\item +goEdit(slideId):
					\item +goExecute(slideId):
					\item +goProfile():
					\item +getSS():
					\item +deleteSlideShow(slideId):
					\item +renameSlideShow(nameSS):
					\item +createSlideShow():
					\item +createSlideShow()
                    
					\end{itemize}
				}
				\subsubsection{Controller::ProfileController}{
					\textbf{Funzione}\\
					\indent Questa classe gestirà le operazioni e la logica applicativa riguardante la pagina profilo di un utente.\\\\
					\textbf{Relazioni d'uso con altri moduli}\\
					\indent Questa classe utilizzerà le seguenti classi:
					\begin{itemize}
						\item Model::MongoRelations::AccessControl::Authentication((DA COMPLETARE)).
						\item Services::Utils;
						\item Services::Main;
						\item Services::toPages;
						\item Services::Upload.
					\end{itemize}
					\textbf{Attributi}\\
					\begin{itemize}
					\item scope:Object\\
					Questo campo dati rappresenta l’oggetto che permette la comunicazione tra la view ed il controller, rendendo possibile l’accesso al model mantenendolo sincronizzato, implementando in questo modo il 2-way data binding.
					\item formData\\
					Questo campo dati rappresenta
					\end{itemize}
					\textbf{Metodi}
					\begin{itemize}
                    \item +getData()
                    \item +changepassword()
                    \item +uploadpedia(files)
					\end{itemize}
				}
				\subsubsection{View::Pages::Execution}{
					\textbf{Funzione}\\
					\indent Questa classe si occuperà di gestire i segnali che arrivano da View::Pages::Execution.\\\\
					\textbf{Relazioni d'uso con altri moduli}\\
					\indent Questa classe utilizzerà le seguenti classi:
					\begin{itemize}
						\item View::Pages::Execution;
						\item Model::MongoRelations::Loader::LoaderClass;
					\end{itemize}
					\textbf{Attributi}\\
					\indent Al momento non sono stati previsti degli attributi.\\\\
					\textbf{Metodi}
					\begin{itemize}
						\item Execution(string id): costruttore che dovrà inizializzare gli eventuali attibuti e la pagina View::Pages::Execution in base all'id della presentazione. Execution(string id) controllerà la presenza del token di sessione, usando il metodo goIndex() in caso negativo, e dialogherà con Model::MongoRelations::Loader::LoaderClass per caricare la presentazione dal database;
						\item +goHome(): metodo che reindirizza alla pagina View::Pages::Home;
						\item +goIndex(): metodo che reindirizza alla pagina View::Pages::Index;
						\item +goEdit(): metodo che reindirizza alla pagina View::Pages::Edit.
					\end{itemize}
				}
				\subsubsection{Controller::EditController}{
					\textbf{Funzione}\\
					\indent Questa classe si occuperà di mostrare all'utente la possibilità di apportare modifiche ad una presentazione.\\\\
					\textbf{Relazioni d'uso con altri moduli}\\
					\indent Questa classe utilizzerà le seguenti classi:
					\begin{itemize}
						\item View::Pages::Edit;
						\item Model::SlideShow::SlideShowActions::Command:
						\begin{itemize}
													\item ConcreteTextInsertCommand;
													\item ConcreteFrameInsertCommand;
													\item ConcreteImageInsertCommand;
													\item ConcreteSVGInsertCommand;
													\item ConcreteAudioInsertCommand;
													\item ConcreteVideoInsertCommand;
													\item ConcreteBackgroundInsertCommand;
													\item ConcreteTextRemoveCommand;
													\item ConcreteFrameRemoveCommand;
													\item ConcreteImageRemoveCommand;
													\item ConcreteSVGRemoveCommand;
													\item ConcreteAudioRemoveCommand;
													\item ConcreteVideoRemoveCommand;
													\item ConcreteBackgroundRemoveCommand;
													\item ConcreteEditSizeCommand;
													\item ConcreteEditPositionCommand;
													\item ConcreteEditRotationCommand;
													\item ConcreteEditColorCommand;
													\item ConcreteEditBackgroundCommand;
													\item ConcreteEditFontCommand;
													\item ConcreteEditContentCommand; [[[[[[[[[[[[[[[[ESISTE????]]]]]]]]]]]]]]]]
													\item Invoker;
					\end{itemize}
						\item Services::Main;
						\item Services::toPages;
						\item Services::Utils;
						\item Services::SaredData;
						\item Services::Upload;
						\item Model::MongoRelations::Loader::Autenticazione;
						\item Model::MongoRelations::Loader::Loaderclass;
					\end{itemize}
					\textbf{Attributi}\\
					\begin{itemize}
					\item scope:Object\\
					Questo campo dati rappresenta l’oggetto che permette la comunicazione tra la view ed il controller, rendendo possibile l’accesso al model mantenendolo sincronizzato, implementando in questo modo il 2-way data binding.
					\item q::Object\\
					\item mdSideNav::Object
					\item mdBottomSheet::Object
				    \end{itemize}
					\textbf{Metodi}
					\begin{itemize}
					\item goEdit(slideId):
					\item goExecute(slideId):
					\item toggleList():
					\item backgroundManage(bool):
					\item slideShowBackgroundManage(bool):
					\item pathsManage(bool):
					\item rotation(bool):
					\item show(id):
					\item mainPath()
					\item showPathBottomSheet(event)
					\item inserisciFrame()
					\item inserisciTesto()
					\item uploadmedia(files)
					\item inserisciImmagine(files)
					\item inserisciAudio(files)
					\item inserisciVideo(files)
					\item rimuoviFrame(id)
					\item rimuoviSfondo()
					\item cambiaColoreSfondo(color)
					\item cambiaColoreSfondoFrame(color)
					\item mediaControl()
					\item announceClick(index)
					\item setRotation(new rotation)
					\item addToMain()
					\end{itemize}
				}
				\subsection{Controller::BottomSheetController}
				}