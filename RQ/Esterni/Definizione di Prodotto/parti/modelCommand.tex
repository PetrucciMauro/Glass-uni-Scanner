\subsection{Classe Command}{
	\subsubsection{Classe Invoker}{
	\textbf{Funzione}\\
		\indent Classe, componente invoker del Design Pattern Command.\\
	\textbf{Scope}\\
		\indent Model::SlideShow::SlideShowActions::Command::Invoker.\\
	\textbf{Utilizzo}\\
		\indent Viene utilizzata per eseguire i comandi e memorizzarli all’interno di stack dedicate all’implementazione delle funzionalità di annulla e ripristina.\\
	\textbf{Attributi}
	\begin{itemize}
		\item \textbf{undoStack}
		\begin{itemize}
			\item \textbf{Accesso}: Private;
			\item \textbf{Tipo}: Array;
			\item \textbf{Descrizione}: contiene l’elenco dei comandi eseguiti e annullabili.
		\end{itemize}
		\item \textbf{redoStack}
		\begin{itemize}
			\item \textbf{Accesso}: Private;
			\item \textbf{Tipo}: Array;
			\item \textbf{Descrizione}: contiene l’elenco dei comandi annullati e ripristinabili.
		\end{itemize}
	\end{itemize}
	\noindent{\textbf{Metodi}}
	\begin{itemize}
		\item \textbf{execute}(AbstractCommand)
		\begin{itemize}
			\item \textbf{Accesso}: Public;
			\item \textbf{Tipo di ritorno}: Void;
			\item \textbf{Descrizione}: invoca il metodo doAction() del comando ricevuto e invoca
			undoStack.push(AbstractCommand) e redoStack.clear().
		\end{itemize}
		\item \textbf{undo}()
		\begin{itemize}
			\item \textbf{Accesso}: Public;
			\item \textbf{Tipo di ritorno}: Void;
			\item \textbf{Descrizione}: invoca il metodo undoAction() dell’ultimo comando in undoStack() e invoca command=undoStack.pop() e redoStack.push(command).
		\end{itemize}
		\item \textbf{redo}()
		\begin{itemize}
			\item \textbf{Accesso}: Public;
			\item \textbf{Tipo di ritorno}: Void;
			\item \textbf{Descrizione}: invoca il metodo doAction() dell’ultimo comando in redoStack() e invoca command=redoStack.pop() e undoStack.push(command).
		\end{itemize}
		\item \textbf{getUndoStack}()
		\begin{itemize}
			\item \textbf{Accesso}: Public;
			\item \textbf{Tipo di ritorno}: Boolean;
			\item \textbf{Descrizione}: ritorna true se undoStack non è vuoto, false altrimenti.
		\end{itemize}
		\item \textbf{getRedoStack}()
		\begin{itemize}
			\item \textbf{Accesso}: Public;
			\item \textbf{Tipo di ritorno}: Boolean;
			\item \textbf{Descrizione}: ritorna true se redoStack non è vuoto, false altrimenti.
		\end{itemize}
	\end{itemize}
	
	\subsubsection{Classe AbstractCommand}{
		\textbf{Funzione}\\
			\indent Classe concreta, i suoi elementi rappresentano un oggetto di tipo testo.\\
	   	\textbf{Scope}\\
			\indent Model::SlideShow::SlideShowActions::Command::AbstractCommand.\\
		\textbf{Utilizzo}\\
			\indent È classe base per i comandi di modifica, inserimento ed eliminazione degli elementi della presentazione.\\
		\textbf{Attributi}
		\begin{itemize}
			\item \textbf{id}
			\begin{itemize}
				\item \textbf{Accesso}: Private;
				\item \textbf{Tipo}: Integer;
				\item \textbf{Descrizione}: indica il codice identificativo dell’oggetto su cui viene eseguito il comando.
			\end{itemize}
			\item \textbf{executed}
			\begin{itemize}
				\item \textbf{Accesso}: Private;
				\item \textbf{Tipo}: Boolean;
				\item \textbf{Descrizione}: è settata a false di default, indica se il comando è stato eseguito.
			\end{itemize}
		\end{itemize}
		
		\noindent{\textbf{Metodi}}
		\begin{itemize}
			\item \textbf{doAction}()
			\begin{itemize}
				\item \textbf{Accesso}: Public;
				\item \textbf{Tipo di ritorno}: Void; [[[[[[[[[CORRETTO??]]]]]]]]]
				\item \textbf{Descrizione}: metodo virtuale implementato dalle sottoclassi. Svolge le operazioni a cui è dedicato il comando.
			\end{itemize}
			\item \textbf{undoAction}()
			\begin{itemize}
				\item \textbf{Accesso}: Public;
				\item \textbf{Tipo di ritorno}: Void;[[[[[[[[[[CORRETTO??]]]]]]]]]]
				\item \textbf{Descrizione}: metodo virtuale implementato dalle sottoclassi. Annulla le operazioni eseguite dal comando.
			\end{itemize}
		\end{itemize}
		\noindent{\textbf{Ereditata da}:}
		\begin{itemize}
			\item ConcreteTextInsertCommand (\S\ref{TextInsertCommand});
			\item ConcreteFrameInsertCommand (\S\ref{FrameInsertCommand});
			\item ConcreteImageInsertCommand (\S\ref{ImageInsertCommand});
			\item ConcreteSVGInsertCommand (\S\ref{SVGInsertCommand});
			\item ConcreteAudioInsertCommand (\S\ref{AudioInsertCommand});
			\item ConcreteVideoInsertCommand (\S\ref{VideoInsertCommand});
			\item ConcreteBackgroundInsertCommand (\S\ref{BackgroundInsertCommand});
			\item ConcreteTextRemoveCommand (\S\ref{TextRemoveCommand});
			\item ConcreteFrameRemoveCommand (\S\ref{FrameRemoveCommand});
			\item ConcreteImageRemoveCommand (\S\ref{ImageRemoveCommand});
			\item ConcreteSVGRemoveCommand (\S\ref{SVGRemoveCommand});
			\item ConcreteAudioRemoveCommand (\S\ref{AudioRemoveCommand});
			\item ConcreteVideoRemoveCommand (\S\ref{VideoRemoveCommand});
			\item ConcreteBackgroundRemoveCommand (\S\ref{BackgroundRemoveCommand});
			\item ConcreteEditSizeCommand (\S\ref{EditSizeCommand});
			\item ConcreteEditPositionCommand (\S\ref{EditPositionCommand});
			\item ConcreteEditRotationCommand (\S\ref{EditRotationCommand});
			\item ConcreteEditColorCommand (\S\ref{EditColorCommand});
			\item ConcreteEditBackgroundCommand (\S\ref{EditBackgroundCommand});
			\item ConcreteEditFontCommand (\S\ref{EditFontCommand});
			\item ConcreteEditContentCommand (\S\ref{EditContentCommand}).
		\end{itemize}
	}
\subsubsubsection{Classe ConcreteTextInsertCommand}{
	\label{TextInsertCommand}
	\textbf{Funzione}\\
		\indent Classe concreta, è interfaccia del Design Pattern Command.\\
   	\textbf{Scope}\\
		\indent Model::SlideShow::SlideShowActions::Command::AbstractCommand::\-ConcreteTextInsertCommand.\\
	\textbf{Utilizzo}\\
		\indent Viene costruito da Premi::Controller::EditController, riceve le coordinate di inserimento di un testo nella presentazione e invoca il metodo di Inserter insertText() passandogliele.\\
	\textbf{Attributi}
	\begin{itemize}
			\item \textbf{posX}
			\begin{itemize}
				\item \textbf{Accesso}: Private;
				\item \textbf{Tipo}: Double;
				\item \textbf{Descrizione}: rappresenta la posizione sull’asse delle x in cui si deve inserire il testo.
			\end{itemize}
			\item \textbf{posY}
			\begin{itemize}
				\item \textbf{Accesso}: Private;
				\item \textbf{Tipo}: Double;
				\item \textbf{Descrizione}: rappresenta la posizione sull’asse delle y in cui si deve inserire il testo.
			\end{itemize}
			\item \textbf{rotation}
			\begin{itemize}
				\item \textbf{Accesso}: Private;
				\item \textbf{Tipo}: Double;
				\item \textbf{Descrizione}: rappresenta i gradi di rotazione che deve avere l’oggetto inserito.
			\end{itemize}
	\end{itemize}
	\noindent{\textbf{Metodi}}
	\begin{itemize}
		\item \textbf{ConcreteTextInsertCommand}(posX: double, posY: double, rotation: double)
		\begin{itemize}
			\item \textbf{Accesso}: Public;
			\item \textbf{Tipo di ritorno}: Void;
			\item \textbf{Descrizione}: costruisce l’oggetto ConcreteTextInsertCommand e setta posX, posY e rotation.
		\end{itemize}
		\item \textbf{doAction}()
		\begin{itemize}
			\item \textbf{Accesso}: Public;
			\item \textbf{Tipo}: Void;
			\item \textbf{Descrizione}: invoca il metodo di Inserter insertText(posX, posY, degrees) passando come parametri i  campi dati posX, posY e rotation. insertText restituisce un int che rappresenta l’id dell’oggetto. Se executed è settato a false lo setta come true, altrimenti invoca il metodo update(id) di EditPresenter.
		\end{itemize}
		\item \textbf{undoAction}()
		\begin{itemize}
			\item \textbf{Accesso}: Public;
			\item \textbf{Tipo}: Void;
			\item \textbf{Descrizione}: invoca il metodo di Editor removeText(id) passando come parametro il campo id. Invoca il metodo remove(id) di EditPresenter.
		\end{itemize}
	\end{itemize}
	}
	
\subsubsubsection{Classe ConcreteFrameInsertCommand}{
	\label{FrameInsertCommand}
	\textbf{Funzione}\\
		\indent Classe concreta, è interfaccia del Design Pattern Command.\\
   	\textbf{Scope}\\
		\indent Model::SlideShow::SlideShowActions::Command::AbstractCommand::\-ConcreteFrameInsertCommand.\\
	\textbf{Utilizzo}\\
		\indent Viene costruito da Premi::Controller::EditController, riceve le coordinate di inserimento di un testo nella presentazione e invoca il metodo di Inserter insertFrame() passandogliele.\\
	\textbf{Attributi}
	\begin{itemize}
		\item \textbf{posX}
		\begin{itemize}
			\item \textbf{Accesso}: Private;
			\item \textbf{Tipo}: Double;
			\item \textbf{Descrizione}: rappresenta la posizione sull’asse delle x in cui si deve inserire il frame.
		\end{itemize}
		\item \textbf{posY}
		\begin{itemize}
			\item \textbf{Accesso}: Private;
			\item \textbf{Tipo}: Double;
			\item \textbf{Descrizione}: rappresenta la posizione sull’asse delle y in cui si deve inserire il frame.
		\end{itemize}
		\item \textbf{rotation}
		\begin{itemize}
			\item \textbf{Accesso}: Private;
			\item \textbf{Tipo}: Double;
			\item \textbf{Descrizione}: rappresenta i gradi di rotazione che deve avere l’oggetto inserito.
		\end{itemize}
	\end{itemize}
	
	\noindent{\textbf{Metodi}}
	\begin{itemize}
		\item \textbf{ConcreteFrameInsertCommand}(posX: double, posY: double, rotation: double)
		\begin{itemize}
			\item \textbf{Accesso}: Public;
			\item \textbf{Tipo di ritorno}: Void;
			\item \textbf{Descrizione}: costruisce l’oggetto ConcreteFrameInsertCommand e setta posX, posY e rotation.
		\end{itemize}
		\item \textbf{doAction}()
		\begin{itemize}
			\item \textbf{Accesso}: Public;
			\item \textbf{Tipo di ritorno}: Void;
			\item \textbf{Descrizione}: invoca il metodo di Inserter insertFrame(posX, posY, rotation) passando come parametri i  campi dati posX, posY e rotation. insertFrame restituisce un int che rappresenta l’id dell’oggetto. Se executed è settato a false lo setta come true, altrimenti invoca il metodo update(id) di EditController.
		\end{itemize}
		\item \textbf{undoAction}()
		\begin{itemize}
			\item \textbf{Accesso}: Public;
			\item \textbf{Tipo di ritorno}: Void;
			\item \textbf{Descrizione}: invoca il metodo di Editor removeFrame(id) passando come parametro il campo id. Invoca il metodo remove(id) di EditController.
		\end{itemize}
	\end{itemize}
	}
\subsubsubsection{Classe ConcreteImageInsertCommand}{
	\label{ImageInsertCommand}
	\textbf{Funzione}\\
		\indent Classe concreta, è interfaccia del Design Pattern Command.\\
   	\textbf{Scope}\\
		\indent Model::SlideShow::SlideShowActions::Command::AbstractCommand::\-ConcreteImageInsertCommand.\\
	\textbf{Utilizzo}\\
		\indent Viene costruito da Premi::Controller::EditController, riceve le coordinate di inserimento di un’immagine nella presentazione e invoca il metodo di Inserter insertImage() passandogliele.\\
	\textbf{Attributi}
	\begin{itemize}
		\item \textbf{posX}
		\begin{itemize}
			\item \textbf{Accesso}: Private;
			\item \textbf{Tipo}: Double;
			\item \textbf{Descrizione}: rappresenta la posizione sull’asse delle x in cui si deve inserire il frame.
		\end{itemize}
		\item \textbf{posY}
		\begin{itemize}
			\item \textbf{Accesso}: Private;
			\item \textbf{Tipo}: Double;
			\item \textbf{Descrizione}: rappresenta la posizione sull’asse delle y in cui si deve inserire il frame.
		\end{itemize}
		\item \textbf{rotation}
		\begin{itemize}
			\item \textbf{Accesso}: Private;
			\item \textbf{Tipo}: Double;
			\item \textbf{Descrizione}: rappresenta i gradi di rotazione che deve avere l’oggetto inserito.
		\end{itemize}
		\item \textbf{ref}
		\begin{itemize}
			\item \textbf{Accesso}: Private;
			\item \textbf{Tipo}: String;
			\item \textbf{Descrizione}: rappresenta l’url dell’oggetto inserito.
		\end{itemize}
	\end{itemize}
	
	\noindent{\textbf{Metodi}}
	\begin{itemize}
		\item \textbf{ConcreteImageInsertCommand}(posX: double, posY: double, rotation: double, ref: string)
		\begin{itemize}
			\item \textbf{Accesso}: Public;
			\item \textbf{Tipo di ritorno}: Void;
			\item \textbf{Descrizione}: costruisce l’oggetto ConcreteImageInsertCommand e setta posX, posY, rotation e ref.
		\end{itemize}
		\item \textbf{doAction}()
		\begin{itemize}
			\item \textbf{Accesso}: Public;
			\item \textbf{Tipo di ritorno}: Void;
			\item \textbf{Descrizione}: invoca il metodo di Inserter insertImage(posX, posY, rotation, ref) passando come parametri i  campi dati posX, posY, rotation e ref. insertImage restituisce un int che rappresenta l’id dell’oggetto. Se executed è settato a false lo setta come true, altrimenti invoca il metodo update(id) di EditController.
		\end{itemize}
		\item \textbf{undoAction}()
		\begin{itemize}
			\item \textbf{Accesso}: Public;
			\item \textbf{Tipo di ritorno}: Void;
			\item \textbf{Descrizione}: invoca il metodo di Editor removeImage(id) passando come parametro il campo id. Invoca il metodo remove(id) di EditController.
		\end{itemize}
	\end{itemize}
	}
\subsubsubsection{Classe ConcreteSVGInsertCommand}{
	\label{SVGInsertCommand}
	\textbf{Funzione}\\
		\indent Classe concreta, è interfaccia del Design Pattern Command.\\
   	\textbf{Scope}\\
		\indent Model::SlideShow::SlideShowActions::Command::AbstractCommand::\-ConcreteSVGInsertCommand.\\
	\textbf{Utilizzo}\\
		\indent Viene costruito da Premi::Controller::EditController, riceve le coordinate di inserimento di un SVG nella presentazione e invoca il metodo di Inserter insertSVG() passandogliele.\\
	\textbf{Attributi}
	\begin{itemize}
		\item \textbf{posX}
		\begin{itemize}
			\item \textbf{Accesso}: Private;
			\item \textbf{Tipo}: Double;
			\item \textbf{Descrizione}: rappresenta la posizione sull’asse delle x in cui si deve inserire il frame.
		\end{itemize}
		\item \textbf{posY}
		\begin{itemize}
			\item \textbf{Accesso}: Private;
			\item \textbf{Tipo}: Double;
			\item \textbf{Descrizione}: rappresenta la posizione sull’asse delle y in cui si deve inserire il frame.
		\end{itemize}
		\item \textbf{rotation}
		\begin{itemize}
			\item \textbf{Accesso}: Private;
			\item \textbf{Tipo}: Double;
			\item \textbf{Descrizione}: rappresenta i gradi di rotazione che deve avere l’oggetto inserito.
		\end{itemize}
		\item \textbf{shape}
		\begin{itemize}
			\item \textbf{Accesso}: Private;
			\item \textbf{Tipo}: Array;
			\item \textbf{Descrizione}: rappresenta la forma dell’oggetto inserito.
		\end{itemize}
		\item \textbf{color}
		\begin{itemize}
			\item \textbf{Accesso}: Private;
			\item \textbf{Tipo}: String;
			\item \textbf{Descrizione}: rappresenta il colore dell’oggetto inserito.
		\end{itemize}
	\end{itemize}
	
	\noindent{\textbf{Metodi}}
	\begin{itemize}
		\item \textbf{ConcreteSVGInsertCommand}(posX: double, posY: double, rotation: double, shape:array, color:string)
		\begin{itemize}
			\item \textbf{Accesso}: Public;
			\item \textbf{Tipo di ritorno}: Void;
			\item \textbf{Descrizione}: costruisce l’oggetto ConcreteSVGInsertCommand e setta posX, posY, rotation, shape, color.
		\end{itemize}
		\item \textbf{doAction}()
		\begin{itemize}
			\item \textbf{Accesso}: Public;
			\item \textbf{Tipo di ritorno}: Void;
			\item \textbf{Descrizione}: invoca il metodo di Inserter insertSVG(posX, posY, shape, color) passando come parametri i  campi dati posX, posY, rotation, shape e color. insertSVG restituisce un int che rappresenta l’id dell’oggetto. Se executed è settato a false lo setta come true, altrimenti invoca il metodo update(id) di EditController.
		\end{itemize}
		\item \textbf{undoAction}()
		\begin{itemize}
			\item \textbf{Accesso}: Public;
			\item \textbf{Tipo di ritorno}: Void;
			\item \textbf{Descrizione}: invoca il metodo di Editor removeSVG(id) passando come parametro il campo id. Invoca il metodo remove(id) di EditController.
		\end{itemize}
	\end{itemize}
	}
\subsubsubsection{Classe ConcreteAudioInsertCommand}{
	\label{AudioInsertCommand}
	\textbf{Funzione}\\
		\indent Classe concreta, è interfaccia del Design Pattern Command.\\
   	\textbf{Scope}\\
		\indent Model::SlideShow::SlideShowActions::Command::AbstractCommand::\-ConcreteAudioInsertCommand.\\
	\textbf{Utilizzo}\\
		\indent Viene costruito da Premi::Controller::EditController, riceve le coordinate di inserimento di un audio nella presentazione e invoca il metodo di Inserter insertAudio() passandogliele.\\
	\textbf{Attributi}
	\begin{itemize}
		\item \textbf{posX}
		\begin{itemize}
			\item \textbf{Accesso}: Private;
			\item \textbf{Tipo}: Double;
			\item \textbf{Descrizione}: rappresenta la posizione sull’asse delle x in cui si deve inserire il frame.
		\end{itemize}
		\item \textbf{posY}
		\begin{itemize}
			\item \textbf{Accesso}: Private;
			\item \textbf{Tipo}: Double;
			\item \textbf{Descrizione}: rappresenta la posizione sull’asse delle y in cui si deve inserire il frame.
		\end{itemize}
		\item \textbf{rotation}
		\begin{itemize}
			\item \textbf{Accesso}: Private;
			\item \textbf{Tipo}: Double;
			\item \textbf{Descrizione}: rappresenta i gradi di rotazione che deve avere l’oggetto inserito.
		\end{itemize}
		\item \textbf{ref}
		\begin{itemize}
			\item \textbf{Accesso}: Private;
			\item \textbf{Tipo}: String;
			\item \textbf{Descrizione}: rappresenta l’url dell’oggetto inserito.
		\end{itemize}
	\end{itemize}
	
	\noindent{\textbf{Metodi}}
	\begin{itemize}
		\item \textbf{ConcreteAudioInsertCommand}(posX: double, posY: double, rotation: double, ref: string)
		\begin{itemize}
			\item \textbf{Accesso}: Public;
			\item \textbf{Tipo di ritorno}: Void;
			\item \textbf{Descrizione}: costruisce l’oggetto ConcreteAudioInsertCommand e setta posX, posY, rotation, ref.
		\end{itemize}
		\item \textbf{doAction}()
		\begin{itemize}
			\item \textbf{Accesso}: Public;
			\item \textbf{Tipo di ritorno}: Void;
			\item \textbf{Descrizione}: invoca il metodo di Inserter insertAudio(posX, posY, rotation, ref) passando come parametri i  campi dati posX, posY, rotation e ref. insertAudio restituisce un int che rappresenta l’id dell’oggetto. Se executed è settato a false lo setta come true, altrimenti invoca il metodo update(id) di EditController.
		\end{itemize}
		\item \textbf{undoAction}()
		\begin{itemize}
			\item \textbf{Accesso}: Public;
			\item \textbf{Tipo di ritorno}: Void;
			\item \textbf{Descrizione}: invoca il metodo di Editor removeAudio(id) passando come parametro il campo id. Invoca il metodo remove(id) di EditController.
		\end{itemize}
	\end{itemize}
	}
\subsubsubsection{Classe ConcreteVideoInsertCommand}{
	\label{VideoInsertCommand}
	\textbf{Funzione}\\
		\indent Classe concreta, è interfaccia del Design Pattern Command.\\
   	\textbf{Scope}\\
		\indent Model::SlideShow::SlideShowActions::Command::AbstractCommand::\-ConcreteVideoInsertCommand.\\
	\textbf{Utilizzo}\\
		\indent Viene costruito da Premi::Controller::EditController, riceve le coordinate di inserimento di un video nella presentazione e invoca il metodo di Inserter insertVideo() passandogliele..\\
	\textbf{Attributi}
	\begin{itemize}
		\item \textbf{posX}
		\begin{itemize}
			\item \textbf{Accesso}: Private;
			\item \textbf{Tipo}: Double;
			\item \textbf{Descrizione}: rappresenta la posizione sull’asse delle x in cui si deve inserire il frame.
		\end{itemize}
		\item \textbf{posY}
		\begin{itemize}
			\item \textbf{Accesso}: Private;
			\item \textbf{Tipo}: Double;
			\item \textbf{Descrizione}: rappresenta la posizione sull’asse delle y in cui si deve inserire il frame.
		\end{itemize}
		\item \textbf{rotation}
		\begin{itemize}
			\item \textbf{Accesso}: Private;
			\item \textbf{Tipo}: Double;
			\item \textbf{Descrizione}: rappresenta i gradi di rotazione che deve avere l’oggetto inserito.
		\end{itemize}
		\item \textbf{ref}
		\begin{itemize}
			\item \textbf{Accesso}: Private;
			\item \textbf{Tipo}: String;
			\item \textbf{Descrizione}: rappresenta l’url dell’oggetto inserito.
		\end{itemize}
	\end{itemize}
	
	\noindent{\textbf{Metodi}}
	\begin{itemize}
		\item \textbf{ConcreteVideoInsertCommand}(posX: double, posY: double, rotation: double, ref: string)
		\begin{itemize}
			\item \textbf{Accesso}: Public;
			\item \textbf{Tipo di ritorno}: Void;
			\item \textbf{Descrizione}: costruisce l’oggetto ConcreteVideoInsertCommand e setta posX, posY, rotation, ref.
		\end{itemize}
		\item \textbf{doAction}()
		\begin{itemize}
			\item \textbf{Accesso}: Public;
			\item \textbf{Tipo di ritorno}: Void;
			\item \textbf{Descrizione}: ivoca il metodo di Inserter insertVideo(posX, posY, rotation, ref) passando come parametri i  campi dati posX, posY, rotation e ref. insertVideo restituisce un int che rappresenta l’id dell’oggetto. Se executed è settato a false lo setta come true, altrimenti invoca il metodo update(id) di EditController.
		\end{itemize}
		\item \textbf{undoAction}()
		\begin{itemize}
			\item \textbf{Accesso}: Public;
			\item \textbf{Tipo di ritorno}: Void;
			\item \textbf{Descrizione}: invoca il metodo di Editor removeVideo(id) passando come parametro il campo id. Invoca il metodo remove(id) di EditController.
		\end{itemize}
	\end{itemize}
	}
\subsubsubsection{Classe ConcreteBackgroundInsertCommand}{
	\label{BackgroundInsertCommand}
	\textbf{Funzione}\\
		\indent Classe concreta, è interfaccia del Design Pattern Command.\\
   	\textbf{Scope}\\
		\indent Model::SlideShow::SlideShowActions::Command::AbstractCommand::\-ConcreteBackgroundInsertCommand.\\
	\textbf{Utilizzo}\\
		\indent Viene costruito da Premi::Controller::EditController, riceve i parametri di inserimento di uno sfondo nella presentazione e invoca il metodo di Inserter insertBackground() passandogliele.\\
	\textbf{Attributi}
	\begin{itemize}
		\item \textbf{ref}
		\begin{itemize}
			\item \textbf{Accesso}: Private;
			\item \textbf{Tipo}: String;
			\item \textbf{Descrizione}: rappresenta l’url dell’oggetto inserito.
		\end{itemize}
		\item \textbf{color}
		\begin{itemize}
			\item \textbf{Accesso}: Private;
			\item \textbf{Tipo}: String;
			\item \textbf{Descrizione}: rappresenta il colore dello sfondo inserito.
		\end{itemize}
		\item \textbf{oldBackground}
		\begin{itemize}
			\item \textbf{Accesso}: Private;
			\item \textbf{Tipo}: SlideShowElements::Background;
			\item \textbf{Descrizione}: rappresenta il vecchio sfondo della presentazione.
		\end{itemize}
	\end{itemize}
	
	\noindent{\textbf{Metodi}}
	\begin{itemize}
		\item \textbf{ConcreteBackgroundInsertCommand}(ref: string, color:string)
		\begin{itemize}
			\item \textbf{Accesso}: Public;
			\item \textbf{Tipo di ritorno}: Void;
			\item \textbf{Descrizione}: costruisce l’oggetto ConcreteVideoInsertCommand e setta ref e color.
		\end{itemize}
		\item \textbf{doAction}()
		\begin{itemize}
			\item \textbf{Accesso}: Public;
			\item \textbf{Tipo di ritorno}: Void;
			\item \textbf{Descrizione}: invoca il metodo di Inserter insertBackground (ref, color) passando come parametri i  campi dati ref e color. insertBackground restituisce un int che rappresenta l’id dell’oggetto o eventualmente un oggetto di tipo Background a cui ConcreteBackgroundInsertCommand istanzia oldBackground e al cui id inizializza il campo id. Se executed è settato a false lo setta come true, altrimenti invoca il metodo update(id) di EditController.
		\end{itemize}
		\item \textbf{undoAction}()
		\begin{itemize}
			\item \textbf{Accesso}: Public;
			\item \textbf{Tipo di ritorno}: Void;
			\item \textbf{Descrizione}: invoca il metodo di Remover removeBackground(id) passando come parametro il campo id. Se il campo oldBackground è inizializzato invoca il metodo Inserter::insertBackground(SlideShowElement::Background) e invoca il metodo update(id) di EditController, altrimenti ne invoca il metodo remove(id).
		\end{itemize}
	\end{itemize}
	}
\subsubsubsection{Classe ConcreteTextRemoveCommand}{
	\label{TextRemoveCommand}
	\textbf{Funzione}\\
		\indent Classe concreta, è interfaccia del Design Pattern Command..\\
   	\textbf{Scope}\\
		\indent Model::SlideShow::SlideShowActions::Command::AbstractCommand::\-ConcreteTextRemoveCommand.\\
	\textbf{Utilizzo}\\
		\indent Viene costruito da Premi::Controller::EditController, riceve l’id dell’elemento testo da rimuovere dalla presentazione e invoca il metodo di Remover removeText(id).\\
	\textbf{Attributi}
	\begin{itemize}
		\item \textbf{text}
		\begin{itemize}
			\item \textbf{Accesso}: Private;
			\item \textbf{Tipo}: SlideShowElements::Text;
			\item \textbf{Descrizione}: è una copia dell’elemento testo rimosso. 
		\end{itemize}
	\end{itemize}
	
	\noindent{\textbf{Metodi}}
	\begin{itemize}
		\item \textbf{ConcreteTextRemoveCommand}(id:integer)
		\begin{itemize}
			\item \textbf{Accesso}: Public;
			\item \textbf{Tipo di ritorno}: Void;
			\item \textbf{Descrizione}: costruisce l’oggetto ConcreteTextRemoveCommand e setta l’attributo id.
		\end{itemize}
		\item \textbf{doAction}()
		\begin{itemize}
			\item \textbf{Accesso}: Public;
			\item \textbf{Tipo di ritorno}: Void;
			\item \textbf{Descrizione}: invoca il metodo Remover::removeText(id) passando come parametro l’id dell’elemento. removeText restituisce un oggetto di tipo SlideShowElements::Text. Se executed è settato a false lo setta come true, altrimenti invoca il metodo EditController::update(id).
		\end{itemize}
		\item \textbf{undoAction}()
		\begin{itemize}
			\item \textbf{Accesso}: Public;
			\item \textbf{Tipo di ritorno}: Void;
			\item \textbf{Descrizione}: invoca il metodo Inserter::insertText(Text) passando come parametro text. Invoca il metodo update(id) di EditController.
		\end{itemize}
	\end{itemize}
	}
\subsubsubsection{Classe ConcreteFrameRemoveCommand}{
	\label{FrameRemoveCommand}
	\textbf{Funzione}\\
		\indent Classe concreta, è interfaccia del Design Pattern Command.\\
   	\textbf{Scope}\\
		\indent Model::SlideShow::SlideShowActions::Command::AbstractCommand::\-ConcreteFrameRemoveCommand.\\
	\textbf{Utilizzo}\\
		\indent Viene costruito da Premi::Controller::EditController, riceve l’id del frame da rimuovere dalla presentazione e invoca il metodo Remover::removeFrame(id).\\
	\textbf{Attributi}
	\begin{itemize}
		\item \textbf{removedFrame}
		\begin{itemize}
			\item \textbf{Accesso}: Private;
			\item \textbf{Tipo}: SlideShowElements::Frame;
			\item \textbf{Descrizione}: è una copia dell’oggetto rimosso.
		\end{itemize}
	\end{itemize}
	
	\noindent{\textbf{Metodi}}
	\begin{itemize}
		\item \textbf{ConcreteFrameRemoveCommand}(id:integer)
		\begin{itemize}
			\item \textbf{Accesso}: Public;
			\item \textbf{Tipo di ritorno}: Void;
			\item \textbf{Descrizione}: costruisce l’oggetto ConcreteFrameRemoveCommand e setta i campi dati id.
		\end{itemize}
		\item \textbf{doAction}()
		\begin{itemize}
			\item \textbf{Accesso}: Public;
			\item \textbf{Tipo di ritorno}: Void;
			\item \textbf{Descrizione}: invoca il metodo Remover::removeFrame(id) passando come parametro il  campi dati id. removeFrame restituisce una copia dell’oggetto rimosso che verrà settata come campo dati removedFrame. Se executed è settato a false lo setta come true, altrimenti invoca il metodo update(id) di EditController.
		\end{itemize}
		\item \textbf{undoAction}()
		\begin{itemize}
			\item \textbf{Accesso}: Public;
			\item \textbf{Tipo di ritorno}: Void;
			\item \textbf{Descrizione}: invoca il metodo Inserter::insertFrame(Frame) passando come parametro il campo removedFrame. Invoca il metodo update(id) di EditController.
		\end{itemize}
	\end{itemize}
	}
\subsubsubsection{Classe ConcreteImageRemoveCommand}{
	\label{ImageRemoveCommand}
	\textbf{Funzione}\\
		\indent Classe concreta, è interfaccia del Design Pattern Command\\
   	\textbf{Scope}\\
		\indent Model::SlideShow::SlideShowActions::Command::AbstractCommand::\-ConcreteImageRemoveCommand.\\
	\textbf{Utilizzo}\\
		\indent Viene costruito da Premi::Controller::EditController, riceve l’id dell’immagine da rimuovere dalla presentazione e invoca il metodo Remover::removeImage(id).\\
	\textbf{Attributi}
	\begin{itemize}
		\item \textbf{removedImage}
		\begin{itemize}
			\item \textbf{Accesso}: Private;
			\item \textbf{Tipo}: SlideShowElements::Image;
			\item \textbf{Descrizione}: è una copia dell’oggetto rimosso
		\end{itemize}
	\end{itemize}
	
	\noindent{\textbf{Metodi}}
	\begin{itemize}
		\item \textbf{ConcreteImageRemoveCommand}(id: integer)
		\begin{itemize}
			\item \textbf{Accesso}: Public;
			\item \textbf{Tipo di ritorno}: Void;
			\item \textbf{Descrizione}: costruisce l’oggetto ConcreteImageRemoveCommand e setta il campo id.
		\end{itemize}
		\item \textbf{doAction}()
		\begin{itemize}
			\item \textbf{Accesso}: Public;
			\item \textbf{Tipo di ritorno}: Void;
			\item \textbf{Descrizione}: invoca il metodo Remover::removeImage(id) passando come parametro l’id dell’oggetto da rimuovere. removeImage restituisce un oggetto di tipo SlideShowElements::Image cui sarà settato il campo removedImage. Se executed è settato a false lo setta come true, altrimenti invoca il metodo update(id) di EditController.
		\end{itemize}
		\item \textbf{undoAction}()
		\begin{itemize}
			\item \textbf{Accesso}: Public;
			\item \textbf{Tipo di ritorno}: Void;
			\item \textbf{Descrizione}: invoca il metodo Inserter::insertImage(Image) passando come parametro il campo removedImage. Invoca il metodo update(id) di EditController.
		\end{itemize}
	\end{itemize}
	}
\subsubsubsection{Classe ConcreteSVGRemoveCommand}{
	\label{SVGRemoveCommand}
	\textbf{Funzione}\\
		\indent Classe concreta, è interfaccia del Design Pattern Command.\\
   	\textbf{Scope}\\
		\indent Model::SlideShow::SlideShowActions::Command::AbstractCommand::\-ConcreteSVGRemoveCommand.\\
	\textbf{Utilizzo}\\
		\indent Viene costruito da Premi::Controller::EditController, riceve l’id dell’elemento SVG da rimuovere dalla presentazione e invoca il metodo Remover::removeSVG().\\
	\textbf{Attributi}
	\begin{itemize}
		\item \textbf{removedSVG}
		\begin{itemize}
			\item \textbf{Accesso}: Private;
			\item \textbf{Tipo}: SlideShowElements::SVG;
			\item \textbf{Descrizione}: è una copia dell’oggetto che rappresenta l’elemento rimosso.
		\end{itemize}
	\end{itemize}
	
	\noindent{\textbf{Metodi}}
	\begin{itemize}
		\item \textbf{ConcreteSVGRemoveCommand}(id:integer)
		\begin{itemize}
			\item \textbf{Accesso}: Public;
			\item \textbf{Tipo di ritorno}: Void;
			\item \textbf{Descrizione}: costruisce l’oggetto ConcreteSVGRemoveCommand e setta il campo dati id.
		\end{itemize}
		\item \textbf{doAction}()
		\begin{itemize}
			\item \textbf{Accesso}: Public;
			\item \textbf{Tipo di ritorno}: Void;
			\item \textbf{Descrizione}: invoca il metodo Remover::removeSVG(id) passando come parametro il campo dati id.	removeSVG restituisce un elemento di tipo SlideShowElements::SVG che rappresenta l’elemento rimosso e a cui viene inizializzato il campo removedSVG. Se executed è settato a false lo setta come true, altrimenti invoca il metodo update(id) di EditController.
		\end{itemize}
		\item \textbf{undoAction}()
		\begin{itemize}
			\item \textbf{Accesso}: Public;
			\item \textbf{Tipo di ritorno}: Void;
			\item \textbf{Descrizione}: invoca il metodo di Inserter::insertSVG(SVG) passando come parametro il campo removedSVG. Invoca il metodo update(id) di EditController.
		\end{itemize}
	\end{itemize}
	}
\subsubsubsection{Classe ConcreteAudioRemoveCommand}{
	\label{AudioRemoveCommand}
	\textbf{Funzione}\\
		\indent Classe concreta, è interfaccia del Design Pattern Command.\\
   	\textbf{Scope}\\
		\indent Model::SlideShow::SlideShowActions::Command::AbstractCommand::\-ConcreteAudioRemoveCommand.\\
	\textbf{Utilizzo}\\
		\indent Viene costruito da Premi::Controller::EditController, riceve l’id dell’elemento audio da rimuovere dalla presentazione e invoca il metodo Remover::removeAudio(id).\\
	\textbf{Attributi}
	\begin{itemize}
		\item \textbf{removedAudio}
		\begin{itemize}
			\item \textbf{Accesso}: Private;
			\item \textbf{Tipo}: SlideShowElements::Audio;
			\item \textbf{Descrizione}: è una copia dell’oggetto che rappresenta l’elemento rimosso.
		\end{itemize}
	\end{itemize}
	
	\noindent{\textbf{Metodi}}
	\begin{itemize}
		\item \textbf{ConcreteAudioRemoveCommand}(id:integer)
		\begin{itemize}
			\item \textbf{Accesso}: Public;
			\item \textbf{Tipo di ritorno}: Void;
			\item \textbf{Descrizione}: costruisce l’oggetto ConcreteAudioRemoveCommand e setta il campo dati id.
		\end{itemize}
		\item \textbf{doAction}()
		\begin{itemize}
			\item \textbf{Accesso}: Public;
			\item \textbf{Tipo di ritorno}: Void;
			\item \textbf{Descrizione}: invoca il metodo Remover::removeAudio(id) passando come parametro il campo dati id. removeAudio restituisce una copia dell’oggetto Audio rimosso. Se executed è settato a false lo setta come true, altrimenti invoca il metodo update(id) di EditController.
		\end{itemize}
		\item \textbf{undoAction}()
		\begin{itemize}
			\item \textbf{Accesso}: Public;
			\item \textbf{Tipo di ritorno}: Void;
			\item \textbf{Descrizione}: invoca il metodo Inserter::insertAudio(Audio) passando come parametro il campo removedAudio. Invoca il metodo update(id) di EditController.
		\end{itemize}
	\end{itemize}
	}
\subsubsubsection{Classe ConcreteVideoRemoveCommand}{
	\label{VideoRemoveCommand}
	\textbf{Funzione}\\
		\indent Classe concreta, è interfaccia del Design Pattern Command.\\
   	\textbf{Scope}\\
		\indent Model::SlideShow::SlideShowActions::Command::AbstractCommand::\-ConcreteVideoRemoveCommand.\\
	\textbf{Utilizzo}\\
		\indent Viene costruito da Premi::Controller::EditController, riceve l’id dell’elemento video da rimuovere dalla presentazione e invoca il metodo Remover::removeVideo(id).\\
	\textbf{Attributi}
	\begin{itemize}
		\item \textbf{removedVideo}
		\begin{itemize}
			\item \textbf{Accesso}: Private;
			\item \textbf{Tipo}: SlideShowElements::Video;
			\item \textbf{Descrizione}: è una copia dell’oggetto che rappresenta l’elemento rimosso.
		\end{itemize}
	\end{itemize}
	
	\noindent{\textbf{Metodi}}
	\begin{itemize}
		\item \textbf{ConcreteVideoRemoveCommand}(id:integer)
		\begin{itemize}
			\item \textbf{Accesso}: Public;
			\item \textbf{Tipo di ritorno}: Void;
			\item \textbf{Descrizione}: costruisce l’oggetto ConcreteVideoRemoveCommand e setta il campo dati id.
		\end{itemize}
		\item \textbf{doAction}()
		\begin{itemize}
			\item \textbf{Accesso}: Public;
			\item \textbf{Tipo di ritorno}: Void;
			\item \textbf{Descrizione}: invoca il metodo Remover::removeVideo(id) passando come parametro il campo dati id. removeVideo restituisce una copia dell’oggetto Video rimosso. Se executed è settato a false lo setta come true, altrimenti invoca il metodo update(id) di EditController.
		\end{itemize}
		\item \textbf{undoAction}()
		\begin{itemize}
			\item \textbf{Accesso}: Public;
			\item \textbf{Tipo di ritorno}: Void;
			\item \textbf{Descrizione}: invoca il metodo Inserter::insertVideo(Video) passando come parametro il campo removedVideo. Invoca il metodo update(id) di EditController.
		\end{itemize}
	\end{itemize}
	}
\subsubsubsection{Classe ConcreteBackgroundRemoveCommand}{
	\label{BackgroundRemoveCommand}
	\textbf{Funzione}\\
		\indent Classe concreta, è interfaccia del Design Pattern Command.\\
   	\textbf{Scope}\\
		\indent Model::SlideShow::SlideShowActions::Command::AbstractCommand::\-ConcreteBackgroundRemoveCommand.\\
	\textbf{Utilizzo}\\
		\indent Viene costruito da Premi::Controller::EditController, riceve l’id dell’elemento sfondo da rimuovere dalla presentazione e invoca il metodo Remover::removeBackground(id).\\
	\textbf{Attributi}
	\begin{itemize}
		\item \textbf{removedBackground}
		\begin{itemize}
			\item \textbf{Accesso}: Private;
			\item \textbf{Tipo}: SlideShowElements:: Background;
			\item \textbf{Descrizione}: è una copia dell’oggetto che rappresenta l’elemento rimosso.
		\end{itemize}
	\end{itemize}
	
	\noindent{\textbf{Metodi}}
	\begin{itemize}
		\item \textbf{ConcreteBackgroundRemoveCommand}(id:integer)
		\begin{itemize}
			\item \textbf{Accesso}: Public;
			\item \textbf{Tipo di ritorno}: Void;
			\item \textbf{Descrizione}: costruisce l’oggetto ConcreteBackgroundRemoveCommand e setta il campo dati id.
		\end{itemize}
		\item \textbf{doAction}()
		\begin{itemize}
			\item \textbf{Accesso}: Public;
			\item \textbf{Tipo di ritorno}: Void;
			\item \textbf{Descrizione}: invoca il metodo Remover::removeBackground (id) passando come parametro il campo dati id. removeBackground restituisce una copia dell’oggetto Audio rimosso. Se executed è settato a false lo setta come true, altrimenti invoca il metodo update(id) di EditController.
		\end{itemize}
		\item \textbf{undoAction}()
		\begin{itemize}
			\item \textbf{Accesso}: Public;
			\item \textbf{Tipo di ritorno}: Void;
			\item \textbf{Descrizione}: invoca il metodo Inserter::insertBackground (Background) passando come parametro il campo removedBackground. Invoca il metodo update(id) di EditController.
		\end{itemize}
	\end{itemize}
	}
\subsubsubsection{Classe ConcreteEditPositionCommand}{
	\label{EditPositionCommand}
	\textbf{Funzione}\\
		\indent Classe concreta, è interfaccia del Design Pattern Command.\\
   	\textbf{Scope}\\
		\indent Model::SlideShow::SlideShowActions::Command::AbstractCommand::\-ConcreteEditPositionCommand.\\
	\textbf{Utilizzo}\\
		\indent Viene costruito da Premi::Controller::EditController, riceve l’id dell’elemento dalla presentazione e le coordinate x e y in cui deve essere spostato l’elemento, invoca il metodo Editor::editPosition(id, posX, posY).\\
	\textbf{Attributi}
	\begin{itemize}
		\item \textbf{type}
		\begin{itemize}
			\item \textbf{Accesso}: Private;
			\item \textbf{Tipo}: String;
			\item \textbf{Descrizione}: indica il tipo dell’elemento da modificare.
		\end{itemize}
		\item \textbf{oldPosX}
		\begin{itemize}
			\item \textbf{Accesso}: Private;
			\item \textbf{Tipo}: Double;
			\item \textbf{Descrizione}: indica la vecchia posizione sull’asse x dell’elemento.
		\end{itemize}
		\item \textbf{oldPosY}
		\begin{itemize}
			\item \textbf{Accesso}: Private;
			\item \textbf{Tipo}: Double;
			\item \textbf{Descrizione}: indica la vecchia posizione sull’asse y dell’elemento.
		\end{itemize}
		\item \textbf{newPosX}
		\begin{itemize}
			\item \textbf{Accesso}: Private;
			\item \textbf{Tipo}: Double;
			\item \textbf{Descrizione}: indica la nuova posizione sull’asse x dell’elemento.
		\end{itemize}
		\item \textbf{newPosY}
		\begin{itemize}
			\item \textbf{Accesso}: Private;
			\item \textbf{Tipo}: Double;
			\item \textbf{Descrizione}: indica la nuova posizione sull’asse y dell’elemento.
		\end{itemize}
	\end{itemize}
	
	\noindent{\textbf{Metodi}}
	\begin{itemize}
		\item \textbf{ConcreteEditPositionCommand}(id:integer, tipo:string, posX:double, posY:double)
		\begin{itemize}
			\item \textbf{Accesso}: Public;
			\item \textbf{Tipo di ritorno}: Void;
			\item \textbf{Descrizione}: costruisce l’oggetto ConcreteEditPositionCommand e setta il campo dati id, il campo dati type il campo dati newPosX e il campo dati newPosY con i parametri ricevuti.
		\end{itemize}
		\item \textbf{doAction}()
		\begin{itemize}
			\item \textbf{Accesso}: Public;
			\item \textbf{Tipo di ritorno}: Void;
			\item \textbf{Descrizione}: invoca il metodo Editor::editPosition(id, tipo, posX, posY) passando come parametri i campi dati id, type, new posX e newPosY. editPosition ritorna una coppia di double a cui ConcreteEditPositionCommand inizializza oldPosX e oldPosY. Se executed è settato a false lo setta come true, altrimenti invoca il metodo update(id) di EditController.
		\end{itemize}
		\item \textbf{undoAction}()
		\begin{itemize}
			\item \textbf{Accesso}: Public;
			\item \textbf{Tipo di ritorno}: Void;
			\item \textbf{Descrizione}: invoca il metodo Editor::editPosition(id, type, posX, posY) passando come parametro i campi dati id, type, oldPosX, oldPosY. Invoca il metodo update(id) di EditController.
		\end{itemize}
	\end{itemize}
	}
\subsubsubsection{Classe ConcreteEditRotationCommand}{
	\label{EditRotationCommand}
	\textbf{Funzione}\\
		\indent Classe concreta, è interfaccia del Design Pattern Command.\\
   	\textbf{Scope}\\
		\indent Model::SlideShow::SlideShowActions::Command::AbstractCommand::\-ConcreteEditRotationCommand.\\
	\textbf{Utilizzo}\\
		\indent Viene costruito da Premi::Controller::EditController, riceve l’id dell’elemento dalla presentazione e il grado a cui deve essere ruotato l’elemento, invoca il metodo Editor::editRotation(id, tipo, degrees).\\
	\textbf{Attributi}
	\begin{itemize}
		\item \textbf{type}
		\begin{itemize}
			\item \textbf{Accesso}: Private;
			\item \textbf{Tipo}: String;
			\item \textbf{Descrizione}: indica il tipo dell’elemento da modificare.
		\end{itemize}
		\item \textbf{oldDegrees}
		\begin{itemize}
			\item \textbf{Accesso}: Private;
			\item \textbf{Tipo}: Double;
			\item \textbf{Descrizione}: indica i vecchi gradi di rotazione dell’elemento.
		\end{itemize}
		\item \textbf{newDegrees}
		\begin{itemize}
			\item \textbf{Accesso}: Private;
			\item \textbf{Tipo}: Double;
			\item \textbf{Descrizione}: indica i nuovi gradi di rotazione dell’elemento.
		\end{itemize}
	\end{itemize}
	
	\noindent{\textbf{Metodi}}
	\begin{itemize}
		\item \textbf{ConcreteEditRotationCommand}(id:integer, tipo:string, degrees:double)
		\begin{itemize}
			\item \textbf{Accesso}: Public;
			\item \textbf{Tipo di ritorno}: Void;
			\item \textbf{Descrizione}: costruisce l’oggetto ConcreteEditRotationCommand e setta il campo dati id, il campo dati type e il campo dati newDegrees con i parametri ricevuti.
		\end{itemize}
		\item \textbf{doAction}()
		\begin{itemize}
			\item \textbf{Accesso}: Public;
			\item \textbf{Tipo di ritorno}: Void;
			\item \textbf{Descrizione}: invoca il metodo Editor::editRotation(id, tipo, degrees) passando come parametri i campi dati id, type e newDegrees. editRotation ritorna un double cui ConcreteEditRotationCommand inizializza oldDegrees. Se executed è settato a false lo setta come true, altrimenti invoca il metodo update(id) di EditController.
		\end{itemize}
		\item \textbf{undoAction}()
		\begin{itemize}
			\item \textbf{Accesso}: Public;
			\item \textbf{Tipo di ritorno}: Void;
			\item \textbf{Descrizione}: invoca il metodo Editor::editRotation(id, tipo, degrees) passando come parametro i campi dati id, tipo, oldRotation. Invoca il metodo update(id) di EditController.
		\end{itemize}
	\end{itemize}
	}
\subsubsubsection{Classe ConcreteEditSizeCommand}{
	\label{EditSizeCommand}
	\textbf{Funzione}\\
		\indent Classe concreta, è interfaccia del Design Pattern Command.\\
   	\textbf{Scope}\\
		\indent Model::SlideShow::SlideShowActions::Command::AbstractCommand::\-ConcreteEditSizeCommand.\\
	\textbf{Utilizzo}\\
		\indent Viene costruito da Premi::Controller::EditController, riceve l’id dell’elemento dalla presentazione e le dimensioni a cui deve essere ridimensionato l’elemento, invoca il metodo Editor::editSize(id, tipo, height, width).\\
	\textbf{Attributi}
	\begin{itemize}
		\item \textbf{type}
		\begin{itemize}
			\item \textbf{Accesso}: Private;
			\item \textbf{Tipo}: String;
			\item \textbf{Descrizione}: indica il tipo dell’elemento da modificare.
		\end{itemize}
		\item \textbf{oldHeight}
		\begin{itemize}
			\item \textbf{Accesso}: Private;
			\item \textbf{Tipo}: Double;
			\item \textbf{Descrizione}: indica la vecchia altezza dell’elemento.
		\end{itemize}
		\item \textbf{oldWidth}
		\begin{itemize}
			\item \textbf{Accesso}: Private;
			\item \textbf{Tipo}: Double;
			\item \textbf{Descrizione}: indica la vecchia larghezza dell’elemento.
		\end{itemize}
		\item \textbf{newHeight}
		\begin{itemize}
			\item \textbf{Accesso}: Private;
			\item \textbf{Tipo}: Double;
			\item \textbf{Descrizione}: indica la nuova altezza dell’elemento.
		\end{itemize}
		\item \textbf{newWidth}
		\begin{itemize}
			\item \textbf{Accesso}: Private;
			\item \textbf{Tipo}: Double;
			\item \textbf{Descrizione}: indica la nuova larghezza dell’elemento.
		\end{itemize}
	\end{itemize}
	
	\noindent{\textbf{Metodi}}
	\begin{itemize}
		\item \textbf{ConcreteEditSizeCommand}(id:integer, tipo:string, height:double, width:double)
		\begin{itemize}
			\item \textbf{Accesso}: Public;
			\item \textbf{Tipo di ritorno}: Void;
			\item \textbf{Descrizione}: costruisce l’oggetto ConcreteEditSizeCommand e setta il campo dati id, il campo dati type e i campi dati newHeight e newWidth con i parametri ricevuti.
		\end{itemize}
		\item \textbf{doAction}()
		\begin{itemize}
			\item \textbf{Accesso}: Public;
			\item \textbf{Tipo di ritorno}: Void;
			\item \textbf{Descrizione}: invoca il metodo Editor::editSize(id, tipo, height, width) passando come parametri i campi dati id, type, newHeight e newWidth. editSize ritorna una coppia di double a cui ConcreteEditSizeCommand inizializza oldHeight e oldWidth. Se executed è settato a false lo setta come true, altrimenti invoca il metodo update(id) di EditController.
		\end{itemize}
		\item \textbf{undoAction}()
		\begin{itemize}
			\item \textbf{Accesso}: Public;
			\item \textbf{Tipo di ritorno}: Void;
			\item \textbf{Descrizione}: invoca il metodo Editor::editSize(id, tipo, height, width) passando come parametro i campi dati id, tipo, oldHeight, oldWidth. Invoca il metodo update(id) di EditController.
		\end{itemize}
	\end{itemize}
	}
\subsubsubsection{Classe ConcreteEditBackgroundCommand}{
	\label{EditBackgroundCommand}
	\textbf{Funzione}\\
		\indent Classe concreta, è interfaccia del Design Pattern Command.\\
   	\textbf{Scope}\\
		\indent Model::SlideShow::SlideShowActions::Command::AbstractCommand::\-ConcreteEditBackgroundCommand.\\
	\textbf{Utilizzo}\\
		\indent Viene costruito da Premi::Controller::EditController, riceve l’id dell’elemento dalla presentazione e lo sfondo da applicargli, invoca il metodo Editor::editBackground(id, tipo, newRef, newColor).\\
	\textbf{Attributi}
	\begin{itemize}
		\item \textbf{type}
		\begin{itemize}
			\item \textbf{Accesso}: Private;
			\item \textbf{Tipo}: String;
			\item \textbf{Descrizione}: indica il tipo dell’elemento da modificare.
		\end{itemize}
		\item \textbf{oldRef}
		\begin{itemize}
			\item \textbf{Accesso}: Private;
			\item \textbf{Tipo}: String;
			\item \textbf{Descrizione}: indica il vecchio url dell’immagine di sfondo dell’elemento.
		\end{itemize}
		\item \textbf{oldColor}
		\begin{itemize}
			\item \textbf{Accesso}: Private;
			\item \textbf{Tipo}: String;
			\item \textbf{Descrizione}: indica il vecchio colore di sfondo dell’elemento.
		\end{itemize}
		\item \textbf{newRef}
		\begin{itemize}
			\item \textbf{Accesso}: Private;
			\item \textbf{Tipo}: String;
			\item \textbf{Descrizione}: indica il nuovo url dell’immagine di sfondo dell’elemento.
		\end{itemize}
		\item \textbf{newColor}
		\begin{itemize}
			\item \textbf{Accesso}: Private;
			\item \textbf{Tipo}: String;
			\item \textbf{Descrizione}: indica il nuovo colore di sfondo dell’elemento.
		\end{itemize}
	\end{itemize}
	
	\noindent{\textbf{Metodi}}
	\begin{itemize}
		\item \textbf{ConcreteEditBackgroundCommand}(id:integer, tipo:string, url:string, color:string)
		\begin{itemize}
			\item \textbf{Accesso}: Public;
			\item \textbf{Tipo di ritorno}: Void;
			\item \textbf{Descrizione}: costruisce l’oggetto ConcreteEditBackgroundCommand e setta il campo dati id, il campo dati type e i campi dati newUrl e newColor con i parametri ricevuti.
		\end{itemize}
		\item \textbf{doAction}()
		\begin{itemize}
			\item \textbf{Accesso}: Public;
			\item \textbf{Tipo di ritorno}: Void;
			\item \textbf{Descrizione}: invoca il metodo Editor::editBackground(id, tipo, newRef, newColor) passando come parametri i campi dati id, type, newRef e newColor. editBackground ritorna una ConcreteEditBackgroundCommand inizializza oldRef e oldColor. Se executed è settato a false lo setta come true, altrimenti invoca il metodo update(id) di EditController.
		\end{itemize}
		\item \textbf{undoAction}()
		\begin{itemize}
			\item \textbf{Accesso}: Public;
			\item \textbf{Tipo di ritorno}: Void;
			\item \textbf{Descrizione}: invoca il metodo Editor::editBackground(id, tipo, url, color) passando come parametro i campi dati id, tipo, oldUrl, oldColor. Invoca il metodo update(id) di EditController.
		\end{itemize}
	\end{itemize}
	}
\subsubsubsection{Classe ConcreteEditColorCommand}{
	\label{EditColorCommand}
	\textbf{Funzione}\\
		\indent Classe concreta, è interfaccia del Design Pattern Command.\\
   	\textbf{Scope}\\
		\indent Model::SlideShow::SlideShowActions::Command::AbstractCommand::\-ConcreteEditColorCommand.\\
	\textbf{Utilizzo}\\
		\indent Viene costruito da Premi::Controller::EditController, riceve l’id dell’elemento dalla presentazione e il colore da applicargli, invoca il metodo Editor::editColor(id, tipo, color).\\
	\textbf{Attributi}
	\begin{itemize}
		\item \textbf{type}
		\begin{itemize}
			\item \textbf{Accesso}: Private;
			\item \textbf{Tipo}: String;
			\item \textbf{Descrizione}: indica il tipo dell’elemento da modificare.
		\end{itemize}
		\item \textbf{oldColor}
		\begin{itemize}
			\item \textbf{Accesso}: Private;
			\item \textbf{Tipo}: String;
			\item \textbf{Descrizione}: indica il vecchio colore di sfondo dell’elemento.
		\end{itemize}
		\item \textbf{newColor}
		\begin{itemize}
			\item \textbf{Accesso}: Private;
			\item \textbf{Tipo}: String;
			\item \textbf{Descrizione}: indica il nuovo colore di sfondo dell’elemento.
		\end{itemize}
	\end{itemize}
	
	\noindent{\textbf{Metodi}}
	\begin{itemize}
		\item \textbf{ConcreteEditColorCommand}(id:integer, tipo:string, color:string)
		\begin{itemize}
			\item \textbf{Accesso}: Public;
			\item \textbf{Tipo di ritorno}: Void;
			\item \textbf{Descrizione}: costruisce l’oggetto ConcreteEditColorCommand e setta il campo dati id, il campo dati type e il campo dati newColor con i parametri ricevuti.
		\end{itemize}
		\item \textbf{doAction}()
		\begin{itemize}
			\item \textbf{Accesso}: Public;
			\item \textbf{Tipo di ritorno}: Void;
			\item \textbf{Descrizione}: invoca il metodo Editor::editColor(id, tipo, color) passando come parametri i campi dati id e newColor. editColor ritorna una variabile di tipo string a cui ConcreteEditColorCommand inizializza oldColor. Se executed è settato a false lo setta come true, altrimenti invoca il metodo update(id) di EditController.
		\end{itemize}
		\item \textbf{undoAction}()
		\begin{itemize}
			\item \textbf{Accesso}: Public;
			\item \textbf{Tipo di ritorno}: Void;
			\item \textbf{Descrizione}: invoca il metodo Editor::editColor(id, tipo, color) passando come parametro i campi dati id,tipo, oldColor. Invoca il metodo update(id) di EditController.
		\end{itemize}
	\end{itemize}
	}
\subsubsubsection{Classe ConcreteEditFontCommand}{
	\label{EditFontCommand}
	\textbf{Funzione}\\
		\indent Classe concreta, è interfaccia del Design Pattern Command.\\
   	\textbf{Scope}\\
		\indent Model::SlideShow::SlideShowActions::Command::AbstractCommand::\-ConcreteEditFontCommand.\\
	\textbf{Utilizzo}\\
		\indent Viene costruito da Premi::Controller::EditController, riceve l’id dell’elemento dalla presentazione e lo sfondo da applicargli, invoca il metodo Editor::editFont(id, tipo, font).\\
	\textbf{Attributi}
	\begin{itemize}
		\item \textbf{type}
		\begin{itemize}
			\item \textbf{Accesso}: Private;
			\item \textbf{Tipo}: String;
			\item \textbf{Descrizione}: indica il tipo dell’elemento da modificare.
		\end{itemize}
		\item \textbf{oldFont}
		\begin{itemize}
			\item \textbf{Accesso}: Private;
			\item \textbf{Tipo}: String;
			\item \textbf{Descrizione}: indica il vecchio font dell’elemento.
		\end{itemize}
		\item \textbf{newFont}
		\begin{itemize}
			\item \textbf{Accesso}: Private;
			\item \textbf{Tipo}: String;
			\item \textbf{Descrizione}: indica il nuovo font dell’elemento.
		\end{itemize}
	\end{itemize}
	
	\noindent{\textbf{Metodi}}
	\begin{itemize}
		\item \textbf{ConcreteEditFontCommand}(id:integer, tipo:string, font:string)
		\begin{itemize}
			\item \textbf{Accesso}: Public;
			\item \textbf{Tipo di ritorno}: Void;
			\item \textbf{Descrizione}: costruisce l’oggetto ConcreteEditFontCommand e setta il campo dati id, il campo dati type e il campo dati newFont con i parametri ricevuti.
		\end{itemize}
		\item \textbf{doAction}()
		\begin{itemize}
			\item \textbf{Accesso}: Public;
			\item \textbf{Tipo di ritorno}: Void;
			\item \textbf{Descrizione}: invoca il metodo Editor::editFont (id, tipo, font) passando come parametri i campi dati id, type, newFont. editFont ritorna una variabile di tipo stringa a cui ConcreteEditFontCommand inizializza oldFont. Se executed è settato a false lo setta come true, altrimenti invoca il metodo update(id) di EditController.
		\end{itemize}
		\item \textbf{undoAction}()
		\begin{itemize}
			\item \textbf{Accesso}: Public;
			\item \textbf{Tipo di ritorno}: Void;
			\item \textbf{Descrizione}: invoca il metodo Editor::editFont(id, tipo, font) passando come parametro i campi dati id, tipo, oldFont. Invoca il metodo update(id) di EditController.
		\end{itemize}
	\end{itemize}
	}
}
}