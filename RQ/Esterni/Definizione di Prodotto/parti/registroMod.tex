\Large{\textbf{Registro delle modifiche}}\\
\normalsize

%	Ordine di inserimento: dall'ultima versione alla prima
\renewcommand*{\arraystretch}{1.4}
\begin{longtable} [c]{|>{\centering\arraybackslash}m{2cm} | >{\centering\arraybackslash}m{4cm} | >{\centering\arraybackslash}m{3cm} | >{\centering\arraybackslash}m{6cm} |}
		\caption{Versionamento del documento \label{tab:versionamento}}\\
		 \hline
		 \textbf{Versione} & \textbf{Autore} & \textbf{Data} & \textbf{Descrizione}\\
		 \hline
		 \endfirsthead
		 \hline
		 \textbf{Versione} & \textbf{Autore} & \textbf{Data} & \textbf{Descrizione}\\
		 \hline
		\endhead
		 \hline
		 \endfoot
		 \hline
		 \endlastfoot
		 \hline		 
		 0.7.0 & \BM & 24-06-2015 & Aggiunta di contenuti. Inserimento del capitolo Package::Premi::Controller\\	
		 \hline		 
		 0.5.0 & \VG & 20-06-2015 & Aggiunta di contenuti. Inserimento del capitolo Package::Premi::Model\\	
		 \hline		 
		 0.4.0 & \FM & 15-06-2015 & Aggiunta di contenuti. Inserimento del capitolo Package::Premi::View\\	
		 \hline		 
		 0.3.0 & \TP & 12-06-2015 & Aggiunta di contenuti. Inserimento del capitolo Standard di Progetto\\		 
		 \hline		 
		 0.2.5 & \GP & 09-06-2015 & Aggiunta di contenuti. Inserimento del capitolo Introduzione e Descrizione generale\\		 
		 \hline
		 0.1.0 & \GP & 08-06-2015 & Stesura dello scheletro del documento\\		 
\end{longtable}

\newpage
\Large{\textbf{Storico }}\\
\normalsize \\

%	Per mettere più tabelle di storico basta copiare e incollare la seguente porzione di codice e modificarla in base ai dati nuovi
\noindent \textbf{RP -> RQ}
\label{tabVers1}
\begin{table}[h]
	\begin{tabular}{p{0.2\textwidth} p{0.7\textwidth}}
		\toprule \textbf{Versione 1..0.0}	&	\textbf{Nominativo}\\
		\midrule Redazione	& \FM, \VG, \TP, \BM\\
		\midrule Verifica & \PM \\
		\midrule Approvazione	& \BM\\
		\bottomrule
	\end{tabular}
	\caption{Storico ruoli RP -> RQ}
\end{table}
