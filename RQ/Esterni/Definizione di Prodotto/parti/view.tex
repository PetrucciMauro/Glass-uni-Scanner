\section{Package View}{
	\textbf{\tipo}: contiene le classi che istanzieranno gli oggetti per l'interfaccia grafica del software.\\
	\textbf{\relaz}: utilizza le classi contenute nel package Controller per le comunicazioni con il Model.\\
	\textbf{\attivita}: rappresenta l'intera GUI del nostro sistema.
\subsection{View::Pages}{
\textbf{\tipo}: contiene le pagine in Html, rappresentano l'interfaccia grafica vera e propria.\\
\textbf{\relaz}: utilizza le classi contenute nel package Controller.\\
\textbf{\attivita}: rappresenta le pagine fisiche del software.
}
\subsection{View::Pages::Index}{
	\textbf{Funzione}\\
		\indent Questa pagina si occuperà di mostrare all'utente header e footer validi per ogni pagina html e permettendogli di accedere alle pagine Login, Registrazione, Home e Profile e di poter effettuare il logout dal sistema .\\
	\textbf{Relazioni d'uso con altri moduli}\\
		\indent Questa pagina utilizzerà le seguenti classi:
	\begin{itemize}
		\item Controller::HeaderController.
	\end{itemize}
	\textbf{Input utente}
		\begin{itemize}
		\item bottoneAccedi(): richiama Controller::HeaderController::goLogin() che reindirizza alla pagina View::Pages::Login;
		\item bottoneRegistrati(): richiama Controller::HeaderController::goRegistrazione() che reindirizza alla pagina View::Pages::Registrazione;
		\item bottoneHome(): richiama Controller::HeaderController::goHome() che reindirizza alla pagina View::Pages::Home;
		\item bottoneProfilo(): richiama Controller::HeaderController::goProfile() che reindirizza alla pagina View::Pages::Profilo;
	\end{itemize}
	}
\subsection{View::Pages::Login}{
	\textbf{Funzione}\\
		\indent Questa pagina si occuperà di mostrare all'utente la possibilità di effettuare il login.\\
	\textbf{Relazioni d'uso con altri moduli}\\
		\indent Questa pagina utilizzerà le seguenti classi:
	\begin{itemize}
		\item Controller::AccessController.
	\end{itemize}
	\textbf{Input utente}
		\begin{itemize}
		\item bottoneLogin(): attiva il metodo Controller::AccessController::login() che controlla se i campi della form sono stati compilati correttamente. Se l'operazione ha successo, viene effettuato il reindirizzamento alla pagina View::Pages::Home;
	\end{itemize}
	}
	\subsection{View::Pages::Registrazione}{
	\textbf{Funzione}\\
		\indent Questa pagina si occuperà di mostrare all'utente la possibilità di effettuare la registrazione al sistema.\\
	\textbf{Relazioni d'uso con altri moduli}\\
		\indent Questa pagina utilizzerà le seguenti classi:
	\begin{itemize}
		\item Controller::AccessController.
	\end{itemize}
	\textbf{Input utente}
		\begin{itemize}
		\item bottoneRegistrati(): attiva il metodo Controller::AccessController::registration() che controlla se i campi della form sono stati compilati correttamente. Se l'operazione ha successo, viene effettuato il reindirizzamento alla pagina View::Pages::Home;
	\end{itemize}
	}
\subsection{View::Pages::Home}{
	\textbf{Funzione}\\
	\indent Questa pagina si occuperà di mostrare all'utente le presentazioni presenti sul proprio database dando la possibilità di eliminarle, eseguirle, scaricarle in locale, rinominarle, modificarle o crearne di nuove.\\
	\textbf{Relazioni d'uso con altri moduli}\\
	\indent Questa pagina utilizzerà le seguenti classi:
	\begin{itemize}
		\item Controller::HomeController
	\end{itemize}
	\textbf{Input utente}
	\begin{itemize}
		\item bottoneNuovaPresentazione(): bottone che richiama il metodo di Controller::HomeController::createSlideShow();
		\item bottoneElimina(): bottone che richiama il metodo di Controller::HomeController::deleteSlideShow() passandogli il nome della presentazione da eliminare;
		\item bottoneRinomina(): bottone che richiama il metodo di Controller::HomeController::renameSlideShow() passandogli il nome della presentazione da rinominare;
		\item bottoneEsegui(): bottone che richiama il metodo di Controller::HomeController::goExecute() passandogli il nome della presentazione da eseguire;
		\item bottoneEdit(): bottone che richiama il metodo di Controller::HomeController::goEdit() passandogli il nome della presentazione da modificare;
		\item bottoneSalva(): bottone che richiama il metodo di Controller::HomeController::salvaManifest() passandogli il nome della presentazione da salvare in locale.
	\end{itemize}
	}
\subsection{View::Pages::Profile}{
	\textbf{Funzione}\\
	\indent Questa pagina si occuperà di mostrare all'utente la possibilità di cambiare i propri dati personali.\\
	\textbf{Relazioni d'uso con altri moduli}\\
	\indent Questa pagina utilizzerà le seguenti classi:
	\begin{itemize}
		\item Controller::ProfileController
	\end{itemize}
	\textbf{Attributi}\\
	\indent Al momento non sono stati previsti degli attributi.\\
	\textbf{Input utente}
	\begin{itemize}
		\item bottoneCambiaPassword(): bottone che richiama il metodo di Controller::ProfileController::changePassword(). Il risultato dell'operazione viene in ogni caso comunicato all'utente.
	\end{itemize}
}
\subsection{View::Pages::Execution}{
	\textbf{Funzione}\\
	\indent Questa pagina si occuperà di gestire l'esecuzione di una presentazione utilizzando il framework Impress.js associato alla pagina.\\\\
	\textbf{Relazioni d'uso con altri moduli}\\
	\indent Questa pagina utilizzerà le seguenti classi:
	\begin{itemize}
		\item Controller::ExecutionController
	\end{itemize}
	\textbf{Attributi}\\
	\indent Al momento non sono stati previsti degli attributi.\\\\
	\textbf{Input utente}
	\begin{itemize}
		\item +next(): metodo che viene invocato premendo il tasto "freccia destra" e che invoca a sua volta il metodo impress().next() implementato all'interno del framework Impress.js e che visualizzerà il frame successivo della presentazione;
		\item +prev(): metodo che viene invocato premendo il tasto "freccia sinistra" e che invoca a sua volta il metodo impress().prev() implementato all'interno del framework Impress.js e che visualizzerà il frame precedente della presentazione;
		\item +bookmark(): metodo che viene invocato premendo il tasto "barra spaziatrice" e che invoca a sua volta il metodo impress().bookmark() implementato all'interno del framework Impress.js e che visualizzerà il frame con bookmark successivo.
	\end{itemize}
}
\subsection{View::Pages::Edit}{
	\textbf{Funzione}\\
	\indent Questa pagina si occuperà di mostrare all'utente la possibilità di apportare modifiche ad una presentazione.\\\\
	\textbf{Relazioni d'uso con altri moduli}\\
	\indent Questa pagina utilizzerà le seguenti classi:
	\begin{itemize}
		\item Controller::EditController
	\end{itemize}
	\textbf{Attributi}\\
	\indent Al momento non sono stati previsti degli attributi.\\\\
	\textbf{Input utente}
	\begin{itemize}
		\item bottoneInsertFrame(): bottone che fa comparire nel piano della presentazione il nuovo frame in base alla forma selezionata, e attiva il metodo in Controller::EditController che si occuperà di aggiornare le informazioni della presentazione;
		\item bottoneInsertMedia(): bottone che fa comparire nel piano della presentazione il nuovo file media e attiva il metodo in Controller::EditController che aggiornerà le informazioni della presentazione e caricherà il file richiesto nel server;
		\item dropMove(): evento che si attiva al rilascio dello spostamento di un elemento, attiva il metodo in Controller::EditController che si occuperà dell'aggiornamento delle informazioni della presentazione;
		\item bottoneInsertText(): bottone che fa comparire nel piano della presentazione il nuovo elemento testuale e attiva il metodo in Controller::EditController che aggiornerà le informazioni della presentazione;
		\item bottoneTextEdit(): bottone che modifica il testo selezionato e attiva il metodo in Controller::EditController che si occupa dell'aggiornamento dell'elemento testuale modificato; 
		\item bottoneDeleteElement(): bottone che fa scomparire l'elemento dal piano della presentazione e attiva il metodo di Controller::EditController che si occuperà dell'aggiornamento della presentazione;
		\item bottoneInsertChoice(): bottone che fa inserire all'utente il testo della scelta e fa scegliere il frame al quale farà riferimento, attiva il metodo di Controller::EditController che si occuperà dell'aggiornamento della presentazione;
		\item bottoneBookmark(): bottone che assegna o rimuove il bookmark al frame, attiva il metodo di Controller::EditController che si occuperà dell'aggiornamento della presentazione;
		\item dropSize(): evento che si attiva al rilascio del ridimensionamento di un elemento, attiva il metodo in Controller::EditController che si occuperà dell'aggiornamento delle informazioni della presentazione;
		\item dropRotation(): evento che si attiva al rilascio della rotazione di un elemento, attiva il metodo in Controller::EditController che si occuperà dell'aggiornamento delle informazioni della presentazione;
		\item bottonePath(): bottone che mostra all'utente l'ordine dei frame e da la possibilità di modificarlo, se viene modificato attiva il metodo in Controller::EditController che si occuperà dell'aggiornamento delle informazioni della presentazione;
		\item bottoneFrameBackground(): bottone assegna l'immagine caricata come sfondo del frame e che attiva il metodo in Controller::EditController per caricare il file nel server e lo imposta come sfondo del frame;
		\item bottoneBackground(): bottone che assegna l'immagine caricata come sfondo della presentazione e attiva il metodo in Controller::EditController per caricare il file nel server e lo imposta come sfondo della presentazione;
		\item bottoneInsertSvg(): bottone che fa comparire nel piano della presentazione il nuovo elemento svg selezionato e attiva il metodo in Controller::EditController che si occuperà di aggiornare le informazioni della presentazione;
		\item bottoneExecute(): bottone che reindirizza alla pagina View::Pages::Execute e attiva il metodo in Controller::EditController che salverà l'id della presentazione da eseguire nella pagina successiva;
	\end{itemize}
}

