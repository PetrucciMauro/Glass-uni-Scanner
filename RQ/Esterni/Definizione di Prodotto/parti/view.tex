\section{Package Premi::View}{
		\textbf{\tipo}: contiene le classi che istanzieranno gli oggetti per l'interfaccia grafica del software.
		\textbf{\relaz}: utilizza le classi contenute nel package Premi::Presenter per le comunicazioni con il Model.
		\textbf{\attivita}: rappresenta l'intera GUI del nostro sistema.
		\subsection{Premi::View::Pages}{
			\textbf{\tipo}: contiene le pagine in Html, rappresentano l'interfaccia grafica vera e propria.
			\textbf{\relaz}: utilizza le classi contenute nel package Premi::ViewJavascript per attivare le funzioni javascript associate a degli eventi e le classi contenute nel package ApacheServer::PhpFunctions per le chiamate Php al server.
			\textbf{\attivita}: rappresenta le pagine fisiche del software.
			}
			\subsection{Premi::View::Pages::Index}{
					\textbf{Funzione}\\
						Questa classe si occuperà di mostrare all'utente la possibilità di effettuare il login oppure di registrarsi al sistema.
					\textbf{Relazioni d'uso con altri moduli}\\
						Questa classe utilizzerà le seguenti classi:
					\begin{itemize}
						\item Premi::View::ViewJavascript::UserFunctions
						\item Premi::ApacheServer::PhpFunctions
					\end{itemize}
					\textbf{Attributi}\\
						Al momento non sono stati previsti degli attributi.
					\textbf{Metodi}\\
						+Index()\\
						costruttore che dovrà inizializzare gli attibuti\\
						+login(string name, string password)\\
						metodo che invia alla classe Premi::View::ViewJavascript::UserFunctions le due stringhe passate 
						
					
				}
		