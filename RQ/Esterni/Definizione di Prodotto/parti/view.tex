\section{Package Premi::View}{
		\textbf{\tipo}: contiene le classi che istanzieranno gli oggetti per l'interfaccia grafica del software.\\\\
		\textbf{\relaz}: utilizza le classi contenute nel package Premi::Controller per le comunicazioni con il Model.\\\\
		\textbf{\attivita}: rappresenta l'intera GUI del nostro sistema.
		\subsection{Premi::View::Pages}{
			\textbf{\tipo}: contiene le pagine in Html, rappresentano l'interfaccia grafica vera e propria.\\\\
			\textbf{\relaz}: utilizza le classi contenute nel package Premi::Controller.\\\\
			\textbf{\attivita}: rappresenta le pagine fisiche del software.
			}
			\subsection{Premi::View::Pages::Index}{
					\textbf{Funzione}\\
						\indent Questa pagina si occuperà di mostrare all'utente la possibilità di effettuare il login oppure di registrarsi al sistema.\\\\
					\textbf{Relazioni d'uso con altri moduli}\\
						\indent Questa pagina utilizzerà le seguenti classi:
					\begin{itemize}
						\item Premi::Controller::IndexController
					\end{itemize}
					\textbf{Attributi}\\
						\indent Al momento non sono stati previsti degli attributi.\\\\
					\textbf{Input utente}
						\begin{itemize}
						\item bottoneLogin(): attiva il metodo Premi::Controller::IndexController::login(username,password), se non sono compilati entrambi i campi restituisce errore, se va tutto bene reindirizza alla pagina Premi::View::Pages::Home;
						\item bottoneSubscription(): attiva il metodo Premi::Controller::IndexController::subscription(username,password), se non sono compilati entrambi i campi restituisce errore, se va tutto bene reindirizza alla pagina Premi::View::Pages::Home.
					\end{itemize}
				}
				\subsection{Premi::View::Pages::Home}{
						\textbf{Funzione}\\
						\indent Questa pagina si occuperà di mostrare all'utente le presentazioni presenti sul proprio database e per ognuno di esse darà la possibilità di eliminarle, eseguirle, scaricarle in locale, rinominarle, modificarle, crearne una nuova o effettuare il logout.\\\\
						\textbf{Relazioni d'uso con altri moduli}\\
						\indent Questa pagina utilizzerà le seguenti classi:
						\begin{itemize}
							\item Premi::Controller::HomeController
						\end{itemize}
						\textbf{Attributi}\\
						\indent Al momento non sono stati previsti degli attributi.\\\\
						\textbf{Input utente}
						\begin{itemize}
							\item bottoneDeleteSlideShow(): bottone che fa scomparire il riferimento alla presentazione di cui il bottone fa riferimento, attiva il metodo di Premi::Controller::HomeController che si occuperà dell'eliminazione della presentazione dal database;
							\item bottoneDownload(): bottone che attiva il Premi::Controller::HomeController che si occuperà dello scaricamento della presentazione nel Manifest e manda a schermo un segnale di avvenuto scaricamento della presentazione;
							\item bottoneRenameSlideShow(): bottone che attiva il Premi::Controller::HomeController che si occuperà della rinominazione della presentazione nel database e manda a schermo l'effettva modifica del nome della presentazione;
							\item bottoneExecute(): bottone che reindirizza alla pagina Premi::View::Pages::Execution e attiva il controller Premi::Controller::HomeController che si prende l'id della presentazione alla quale il bottone fa riferimento per la creazione della prossima pagina;
							\item bottoneEdit(): bottone che reindirizza alla pagina Premi::View::Pages::Edit e attiva il controller Premi::Controller::HomeController che si prende l'id della presentazione alla quale il bottone fa riferimento per la creazione della prossima pagina;
							\item bottoneLogout(): bottone che attiva il Premi::Controller::HomeController che si occuperà della distruzione del token di sessione e reindirizza alla pagina Premi::View::Pages::Index.
						\end{itemize}
					}
					\subsection{Premi::View::Pages::Profile}{
						\textbf{Funzione}\\
						\indent Questa pagina si occuperà di mostrare all'utente la possibilità di cambiare i propri dati personali, caricare, eliminare o rinominare i propri file media.\\\\
						\textbf{Relazioni d'uso con altri moduli}\\
						\indent Questa pagina utilizzerà le seguenti classi:
						\begin{itemize}
							\item Premi::Controller::ProfileController
						\end{itemize}
						\textbf{Attributi}\\
						\indent Al momento non sono stati previsti degli attributi.\\\\
						\textbf{Input utente}
						\begin{itemize}
							\item bottoneChangePassword(): bottone che, se i campi sono compilati erroneamente restituisce errore, altrimenti attiva il metodo in Premi::Controller::ProfileController che si occupa di cambiare la password dell'account e restituisce un messaggio a schermo di avvenuto cambiamento;
							\item bottoneUploadMedia(): bottone che attiva il metodo in Premi::Controller::ProfileController che caricherà il file richiesto nel server, se l'operazione va a buon fine la schermata si aggiorna con la nuova miniatura del file media;
							\item bottoneDeleteMedia(): bottone che fa scomparire il riferimento al file media di cui il bottone fa riferimento, attiva il metodo di Premi::Controller::ProfileController che si occuperà dell'eliminazione del file media dal server;
							\item bottoneRenameMedia(): bottone che attiva il Premi::Controller::ProfileController che si occuperà della rinominazione del file media presente nel server e manda a schermo l'effettva modifica del nome del file media.
						\end{itemize}
					}
					\subsection{Premi::View::Pages::Execution}{
						\textbf{Funzione}\\
						\indent Questa pagina si occuperà di gestire l'esecuzione di una presentazione utilizzando il framework Impress.js associato alla pagina.\\\\
						\textbf{Relazioni d'uso con altri moduli}\\
						\indent Questa pagina utilizzerà le seguenti classi:
						\begin{itemize}
							\item Premi::Controller::ExecutionController
						\end{itemize}
						\textbf{Attributi}\\
						\indent Al momento non sono stati previsti degli attributi.\\\\
						\textbf{Input utente}
						\begin{itemize}
							\item +next(): metodo che viene invocato premendo il tasto "freccia destra" e che invoca a sua volta il metodo impress().next() implementato all'interno del framework Impress.js e che visualizzerà il frame successivo della presentazione;
							\item +prev(): metodo che viene invocato premendo il tasto "freccia sinistra" e che invoca a sua volta il metodo impress().prev() implementato all'interno del framework Impress.js e che visualizzerà il frame precedente della presentazione;
							\item +bookmark(): metodo che viene invocato premendo il tasto "barra spaziatrice" e che invoca a sua volta il metodo impress().bookmark() implementato all'interno del framework Impress.js e che visualizzerà il frame con bookmark successivo.
						\end{itemize}
					}
					\subsection{Premi::View::Pages::Edit}{
						\textbf{Funzione}\\
						\indent Questa pagina si occuperà di mostrare all'utente la possibilità di apportare modifiche ad una presentazione.\\\\
						\textbf{Relazioni d'uso con altri moduli}\\
						\indent Questa pagina utilizzerà le seguenti classi:
						\begin{itemize}
							\item Premi::Controller::EditController
						\end{itemize}
						\textbf{Attributi}\\
						\indent Al momento non sono stati previsti degli attributi.\\\\
						\textbf{Input utente}
						\begin{itemize}
							\item bottoneInsertFrame(): bottone che fa comparire nel piano della presentazione il nuovo frame in base alla forma selezionata, e attiva il metodo in Premi::Controller::EditController che si occuperà di aggiornare le informazioni della presentazione;
							\item bottoneInsertMedia(): bottone che fa comparire nel piano della presentazione il nuovo file media e attiva il metodo in Premi::Controller::EditController che aggiornerà le informazioni della presentazione e caricherà il file richiesto nel server;
							\item dropMove(): evento che si attiva al rilascio dello spostamento di un elemento, attiva il metodo in Premi::Controller::EditController che si occuperà dell'aggiornamento delle informazioni della presentazione;
							\item bottoneInsertText(): bottone che fa comparire nel piano della presentazione il nuovo elemento testuale e attiva il metodo in Premi::Controller::EditController che aggiornerà le informazioni della presentazione;
							\item bottoneTextEdit(): bottone che modifica il testo selezionato e attiva il metodo in Premi::Controller::EditController che si occupa dell'aggiornamento dell'elemento testuale modificato; 
							\item bottoneDeleteElement(): bottone che fa scomparire l'elemento dal piano della presentazione e attiva il metodo di Premi::Controller::EditController che si occuperà dell'aggiornamento della presentazione;
							\item bottoneInsertChoice(): bottone che fa inserire all'utente il testo della scelta e fa scegliere il frame al quale farà riferimento, attiva il metodo di Premi::Controller::EditController che si occuperà dell'aggiornamento della presentazione;
							\item bottoneBookmark(): bottone che assegna o rimuove il bookmark al frame, attiva il metodo di Premi::Controller::EditController che si occuperà dell'aggiornamento della presentazione;
							\item dropSize(): evento che si attiva al rilascio del ridimensionamento di un elemento, attiva il metodo in Premi::Controller::EditController che si occuperà dell'aggiornamento delle informazioni della presentazione;
							\item dropRotation(): evento che si attiva al rilascio della rotazione di un elemento, attiva il metodo in Premi::Controller::EditController che si occuperà dell'aggiornamento delle informazioni della presentazione;
							\item bottonePath(): bottone che mostra all'utente l'ordine dei frame e da la possibilità di modificarlo, se viene modificato attiva il metodo in Premi::Controller::EditController che si occuperà dell'aggiornamento delle informazioni della presentazione;
							\item bottoneFrameBackground(): bottone assegna l'immagine caricata come sfondo del frame e che attiva il metodo in Premi::Controller::EditController per caricare il file nel server e lo imposta come sfondo del frame;
							\item bottoneBackground(): bottone che assegna l'immagine caricata come sfondo della presentazione e attiva il metodo in Premi::Controller::EditController per caricare il file nel server e lo imposta come sfondo della presentazione;
							\item bottoneInsertSvg(): bottone che fa comparire nel piano della presentazione il nuovo elemento svg selezionato e attiva il metodo in Premi::Controller::EditController che si occuperà di aggiornare le informazioni della presentazione;
							\item bottoneExecute(): bottone che reindirizza alla pagina Premi::View::Pages::Execute e attiva il metodo in Premi::Controller::EditController che salverà l'id della presentazione da eseguire nella pagina successiva;
						\end{itemize}
					}
					
		