\section{NodeServer}{

	Il seguente diagramma delle classi \`{e} stato esteso con le primitive:
	\begin{itemize}
	\item \textbf{<<Resource>>} : rappresenta una risorsa associata ad un certo url a cui sono disponibili dei servizi
	\item \textbf{<<Node>>} : rappresenta una parte di url a cui non sono disponibili servizi ma \`{e} utile per suddividere quest'ultimi
	\item \textbf{<<Server>>} : rappresenta la radice dei servizi offerti dal server
	\item \textbf{<<Path>>} : indica una aggiunta in coda all' url attuale per raggiungere una nuova risorsa o nodo
	\item \textbf{<<Middleware>>} : indica un middleware, un insieme di funzionalit\`{a} chiamate ogni qualvolta si accede a risorse attraversando questo elemento
	\end{itemize}
	
	\begin{figure}[H]
		\includegraphics[scale=0.8]{\imgs {nodeAPI}.pdf}
		\label{fig:nodeAPI}
		\caption{Servizi RESTfull offerti dal server nodeJs}
	\end{figure}
	
	\subsection{Risorse e Servizi}{
	
	\begin{itemize}
			
		\item \textbf{NodeServer:} radice dei servizi offerti dal server node:
			\begin{enumerate}
				\item server per pagine html, css e javascript associati
				\item servizi di autenticazione stateless
				\item file server per salvare sul server file statici multimediali(immagini, audio, video), richiedono autenticazione
				\item servizi di interazione con un database MongoDB dove sono persistentemente salvate le presentaizoni, richiedono autenticazione
			\end{enumerate}	
				
		\item \textbf{Register:}
			\begin{itemize}
			\item  /account/register POST
				\begin{itemize} 
				\item \textbf{descrizione:} verifica la presenza dello username nella collezione in mongoDB relativa agli account, se non \`{e} gi\`{a} presente inserisce username e password ricevuti altrimenti annulla l'operazione di inserimento e ritorna un messaggio di errore
				\item \textbf{parametri input body(application/json):} username : string, password : string
				\item \textbf{parametri output body(application/json):} success : boolean, message : string
				\end{itemize}
			\end{itemize}
			
		\item \textbf{Authenticate:}
			\begin{itemize}
			\item  /account/authenticate GET
				\begin{itemize} 
				\item \textbf{descrizione:} verifica username e password ricevuti e se presenti nella opportuna collezione in mongoDB ritorna un token per l'accesso ai servizi protetti, altrimenti ritorna solo un messaggio d'errore
				\item \textbf{parametri input header:} username : string, password : string
				\item \textbf{parametri output body(application/json):} success : boolean, message : string, token : string
				\end{itemize}
			\end{itemize}
				
		\item \textbf{ChangePassword:}
			\begin{itemize}
			\item  /account/changepassword POST
				\begin{itemize} 
				\item \textbf{descrizione:} verifica username e password ricevuti e se presenti nella opportuna collezione in mongoDB e modifica la password con newpassword, altrimenti ritorna solo un messaggio d'errore
				\item \textbf{parametri input body(application/json):} username : string, password : string, new password : string
				\item \textbf{parametri output body(application/json):} success : boolean, message : string
				\end{itemize}
			\end{itemize}
			
		\item \textbf{PublicPages:}
			\begin{itemize}
			\item  /publicpages/[file] GET
				\begin{itemize} 
				%\item \textbf{descrizione:} se presente il file [file] nella cartella /public_htdocs del server ritorna il file stesso 
				\item \textbf{parametri input body:} /
				\item \textbf{output body:} fileStatico
				\end{itemize}
			\end{itemize}
			
		\item \textbf{tokenMiddleware:} verifica che il token passato nel campo Authorization dell' Header sia valido, ne estrae lo username dell'utente e permette l'accesso ai servizi richiesti
		\item \textbf{PrivatePages:}
			\begin{itemize}
			\item  /private/htdocs/[file] GET
				\begin{itemize} 
				\item \textbf{descrizione:} se presente il file [file] nella cartella /private/htdocs del server ritorna il file stesso
				\item \textbf{parametri input header:} token : string
				\item \textbf{output body:} fileStatico
				\end{itemize}
			\end{itemize}
			
		\item \textbf{PresentationMeta:}
			\begin{itemize}
			\item  /private/api/presentations GET
				\begin{itemize} 
				\item \textbf{descrizione:} cerca in mongoDB nella collezione associata alle presentazioni dell'utente, ritorna un array i cui elementi sono array associativi con le meta-informazioni riguardanti le presentazioni
				\item \textbf{parametri input header:} token : string
				\item \textbf{output body(application/json):}  success : boolean, message : string, presentationMetas : array
				\end{itemize}
			\end{itemize}
			
		\item \textbf{NewPresentation:}
			\begin{itemize}
			\item  /private/api/presentations/[presentationName] POST
				\begin{itemize} 
				\item \textbf{descrizione:} crea una nuova presentazione con il nome [presentatioNname] se il nome non \`{e} gi\`{a} stato usata per un'altra presentazione dello stesso utente
				\item \textbf{parametri input header:} token : string
				\item \textbf{output body(application/json):}  success : boolean, message : string
				\end{itemize}
			\end{itemize}
			
		\item \textbf{Presentation:}
			\begin{itemize}
			\item  /private/api/presentations/[presentationName] GET
				\begin{itemize} 
				\item \textbf{descrizione:} recupera la presentazione se esistente associata al nome passato come ultima parte dell'url
				\item \textbf{parametri input header:} token : string
				\item \textbf{output body(application/json):}  success : boolean, message : string, presentation : object
				\end{itemize}
			\item  /private/api/presentations/[presentationName] DELETE
				\begin{itemize} 
				\item \textbf{descrizione:} elimina la associazione dell'utente con nome il valore di [presentationName]
				\item \textbf{parametri input header:} token : string
				\item \textbf{output body(application/json):}  success : boolean, message : string
				\end{itemize}
			\end{itemize}

		\item \textbf{RenamePresentation:}
			\begin{itemize}
			\item  /private/api/presentations/[presentationName]/[newname] POST
				\begin{itemize} 
				\item \textbf{descrizione:} rinomina la presentazione con il nome [presentatioNname] con il nome [newname]
				\item \textbf{parametri input header:} token : string
				\item \textbf{output body(application/json):}  success : boolean, message : string
				\end{itemize}
			\end{itemize}
			
		\item \textbf{CreateElement:}
			\begin{itemize}
			\item   /private/api/presentations/[presentationName]/[createElementPOST
				\begin{itemize} 
				\item \textbf{descrizione:} crea nella presentazione [presentationName] dell'utente un nuovo elemento, ovvero l'oggetto element passato in input nel corpo della chiamata
				\item \textbf{parametri input header:} token : string
				\item \textbf{input body(application/json):}  element : object
				\item \textbf{output body(application/json):}  success : boolean, message : string
				\end{itemize}
			\end{itemize}
			
		\item \textbf{PresentationElement:}
			\begin{itemize}
			\item    /private/api/presentations/[presentationName]/[idElement] PUT
				\begin{itemize} 
				\item \textbf{descrizione:} sostituisce nella presentazione dell'utente l'elemento con identificativo [idElement] con l'oggetto passato nel corpo in formato son						\item \textbf{parametri input header:} token : string
				\item \textbf{input body(application/json):}  element : object
				\item \textbf{output body(application/json):}  success : boolean, message : string
				\end{itemize}
			\item    /private/api/presentations/[presentationName]/[idElement] DELETE
				\begin{itemize} 
				\item \textbf{descrizione:} elimina dalla presentazione con nome [presentationName] l'elemento con identificativo [idElement]						
				\item \textbf{parametri input header:} token : string
				\item \textbf{output body(application/json):}  success : boolean, message : string
				\end{itemize}
			\end{itemize}
			
		\item \textbf{ImagesMeta:}
			\begin{itemize}
			\item   /private/api/files/image GET
				\begin{itemize} 
				\item \textbf{descrizione:} ritorna un array con oggetti rappresentanti informazioni sui file immagine dell'utente sul server
				\item \textbf{parametri input header:} token : string
				\item \textbf{output body(application/json):}  success : boolean, message : string, imageMetas : array
				\end{itemize}
			\end{itemize}
			
		\item \textbf{ImagesMeta:}
			\begin{itemize}
			\item   /private/api/files/image GET
				\begin{itemize} 
				\item \textbf{descrizione:} ritorna un array con oggetti rappresentanti informazioni sui file immagine dell'utente sul server
				\item \textbf{parametri input header:} token : string
				\item \textbf{output body(application/json):}  success : boolean, message : string, imageMetas : array
				\end{itemize}
			\end{itemize}
			
		\item \textbf{Image:}
			\begin{itemize}
			\item    /private/api/files/image/[imagename] POST
				\begin{itemize} 
				\item \textbf{descrizione:} caricare da locale un nuovo file immagine nella cartella /users/[username]/images					
				\item \textbf{parametri input header:} token : string
				\item \textbf{input body(multipart/form-data):} file   
				\item \textbf{output body(application/json):}  success : boolean, message : string
				\end{itemize}
			\item    /private/api/files/image/[imagename] DELETE
				\begin{itemize} 
				\item \textbf{descrizione:} elimina il file immagine [imagename] dalla cartella /users/[username]/images nel server				
				\item \textbf{parametri input header:} token : string
				\item \textbf{output body(application/json):}  success : boolean, message : string
				\end{itemize}
			\end{itemize}
			
		\item \textbf{RenameImage:}
			\begin{itemize}
			\item   /private/api/files/image/[imagename]/[newname] POST
				\begin{itemize} 
				\item \textbf{descrizione:} rinomina il file immagine [imagename] con il nuovo nome passato come valore di [newname] nella cartella /users/[username]/images
				\item \textbf{parametri input header:} token : string
				\item \textbf{output body(application/json):}  success : boolean, message : string
				\end{itemize}
			\end{itemize}
			
		\item \textbf{VideosMeta:}
			\begin{itemize}
			\item   /private/api/files/video GET
				\begin{itemize} 
				\item \textbf{descrizione:} ritorna un array con oggetti rappresentanti informazioni sui file video dell'utente sul server
				\item \textbf{parametri input header:} token : string
				\item \textbf{output body(application/json):}  success : boolean, message : string, videoMetas : array
				\end{itemize}
			\end{itemize}
			
		\item \textbf{Video:}
			\begin{itemize}
			\item    /private/api/files/video/[videoname] POST
				\begin{itemize} 
				\item \textbf{descrizione:} caricare da locale un nuovo file video nella cartella /users/[username]/videos					
				\item \textbf{parametri input header:} token : string
				\item \textbf{input body(multipart/form-data):} file   
				\item \textbf{output body(application/json):}  success : boolean, message : string
				\end{itemize}
			\item    /private/api/files/video/[videoname] DELETE
				\begin{itemize} 
				\item \textbf{descrizione:} elimina il file video [videoname] dalla cartella /users/[username]/videos nel server				
				\item \textbf{parametri input header:} token : string
				\item \textbf{output body(application/json):}  success : boolean, message : string
				\end{itemize}
			\end{itemize}
			
		\item \textbf{RenameImage:}
			\begin{itemize}
			\item   /private/api/files/video/[videoname]/[newname] POST
				\begin{itemize} 
				\item \textbf{descrizione:} rinomina il file video [videoname] con il nuovo nome passato come valore di [newname] nella cartella /users/[username]/videos
				\item \textbf{parametri input header:} token : string
				\item \textbf{output body(application/json):}  success : boolean, message : string
				\end{itemize}
			\end{itemize}
			
		\item \textbf{AudiosMeta:}
			\begin{itemize}
			\item   /private/api/files/audio GET
				\begin{itemize} 
				\item \textbf{descrizione:} ritorna un array con oggetti rappresentanti informazioni sui file audio dell'utente sul server
				\item \textbf{parametri input header:} token : string
				\item \textbf{output body(application/json):}  success : boolean, message : string, audioMetas : array
				\end{itemize}
			\end{itemize}
			
		\item \textbf{Audio:}
			\begin{itemize}
			\item    /private/api/files/audio/[audioname] POST
				\begin{itemize} 
				\item \textbf{descrizione:} caricare da locale un nuovo file video nella cartella /users/[username]/videos					
				\item \textbf{parametri input header:} token : string
				\item \textbf{input body(multipart/form-data):} file   
				\item \textbf{output body(application/json):}  success : boolean, message : string
				\end{itemize}
			\item    /private/api/files/audio/[audioname] DELETE
				\begin{itemize} 
				\item \textbf{descrizione:} elimina il file audio [audioname] dalla cartella /users/[username]/audios nel server				
				\item \textbf{parametri input header:} token : string
				\item \textbf{output body(application/json):}  success : boolean, message : string
				\end{itemize}
			\end{itemize}
			
		\item \textbf{RenameAudio:}
			\begin{itemize}
			\item   /private/api/files/audio/[audioname]/[newname] POST
				\begin{itemize} 
				\item \textbf{descrizione:} rinomina il file video [videoname] con il nuovo nome passato come valore di [newname] nella cartella /users/[username]/audios
				\item \textbf{parametri input header:} token : string
				\item \textbf{output body(application/json):}  success : boolean, message : string
				\end{itemize}
			\end{itemize}
			
	\end{itemize}
	}
}
	
	
	
	
	
	
	
	
	
	
	
	
	
	
	