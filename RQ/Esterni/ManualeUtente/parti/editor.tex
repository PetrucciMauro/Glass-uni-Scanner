\section{Editor}
Per poter accedere all'editor bisogner� essere autenticati e aver creato una presentazione. La schermata di editor si presenter� in questo modo:
	\begin{figure}[H]
		\centering
		\includegraphics[scale=0.75]{\imgs {editor}.jpg} %inserire il diagramma UML
		\label{editor}
		\caption{Schermata Editor}
	\end{figure}
\subsection{Creazione primo frame}
L'editor permette l'inserimento dentro l'area apposita di frame , nei quali si potranno inserire elementi multimediali. Per procedere alla creazione della slide � sufficiente:
\begin{itemize}
\item premere il pulsante per l'inserimento di un nuovo elemento;

\item posizionarsi con il mouse su Inserisci frame premere il tasto sinistro del mouse e successivamente trascinare il nuovo frame nell'apposita area;


\end{itemize}
\subsection{Inserimento testo}
L'editor permette l'inserimento di testi all'interno della presentazione. Per procedere all' inserimento del testo � sufficiente:
\begin{itemize}
\item premere il pulsante per l'inserimento di un nuovo elemento;

\item posizionarsi con il mouse su Inserisci testo premere il tasto sinistro del mouse e successivamente trascinare il riquadro nell'apposita area;

\item fare click sinistro con il mouse all'interno dell riquadro per poter inserire del testo.
\end{itemize}
\subsubsection{Modifica testo}
Sono possibili per il testo le seguenti modifiche:
\begin{itemize}


\item Gestione colore testo
\item Gestione dimensione testo
\item Gestione font testo
\end{itemize}
\subsection{Inserimento Elemento multimediale}
L'editor permette l'inserimento di elementi multimediali quali, immagini,audio e video. Per procedere all'inserimento di un Elemento multimediale � sufficiente:
\begin{itemize}
\item premere il pulsante per l'inserimento di un nuovo elemento;
\item premere il pulsante inserisci immagine/audio/video/
\item selezionare l'elemento multimediale desiderato e caricarlo
\item Immagini e video possono essere ridimensionati a piacimento 
\end{itemize}
\subsection{Gestione sfondo}
L'editor permette l'inserimento di un proprio sfondo personale per la presentazione. Per procedere all'inserimento di uno sfondo � sufficiente:
\begin{itemize}
\item premere il pulsante per la gestione dello sfondo;
\item preme il pulsante Scegli immagine di sfondo;
\item selezionare l'immagine che si vuole;
\end{itemize}
