\section{Autenticazione e gestione profilo}
\subsection{Registrazione}
\begin{enumerate}
\item Il primo passo per iniziare ad utilizzare \premi\ è la registrazione. È possibile registrarsi premendo il pulsante Registrati posto nella Homepage.
	\begin{figure}[H]
		\centering
		\includegraphics[scale=0.75]{\imgs {registrazione}.jpg} %inserire il diagramma UML
		\label{fig:registrazione}
		\caption{Homepage}
	\end{figure}
\item Verrà proposto un form da completare seguendo la procedura:
\begin{itemize}
\item unsername:Inserire lo username con il quale ci si vuole registrare;
\item password: Inserire una password, per motivi di sicurezza è richiesta una password più lunga di 6 caratteri;
\item registrati: Premere infine il pulsante Registrati per confermare i dati e poter iniziare ad utilizzare \premi.
\end{itemize}
\end{enumerate}
	\begin{figure}[H]
		\centering
		\includegraphics[scale=0.75]{\imgs {formregistrazione}.jpg} %inserire il diagramma UML
		\label{fig:formregistrazione}
		\caption{Form di registrazione}
	\end{figure}

\subsection{Autenticazione}
\begin{enumerate}
\item Per poter accedere in \premi bisogna essere prima di tutto registrati. Per effettuare la login basta compilare il form che si presenta in homepage o , in alternativa , premere il pulsante login.
	\begin{figure}[H]
		\centering
		\includegraphics[scale=0.75]{\imgs {registrazione}.jpg} %inserire il diagramma UML
		\label{fig:login}
		\caption{Homepage}
	\end{figure}
\item Verrà proposto un form da completare seguendo la procedura:
\begin{itemize}
\item unsername:Inserire lo username con il quale ci si vuole registrare;
\item password: Inserire una password, per motivi di sicurezza è richiesta una password più lunga di 6 caratteri;
\item login: Premere infine il pulsante login per confermare i dati e poter iniziare ad utilizzare \premi.
\end{itemize}

\end{enumerate}
	\begin{figure}[H]
		\centering
		\includegraphics[scale=0.75]{\imgs {formlogin}.jpg} %inserire il diagramma UML
		\label{formlogin}
		\caption{Form di autenticazione}
	\end{figure}
\subsection{Logout}
Per poter effettuare il logout è necessario essere registrati. Il logout permette di terminare la propria sessione di lavoro. Un volta effettuato il logout al successivo accesso alla piattaforma sarà richiesto di autenticarsi. Per effettuare il
logout cliccare sul pulsante Logout.

	\begin{figure}[H]
		\centering
		\includegraphics[scale=0.75]{\imgs {logout}.jpg} %inserire il diagramma UML
		\label{logout}
		\caption{Logout}
	\end{figure}
\subsection{Cambio password}
\begin{enumerate}
\item Per poter effettuare il cambio password è necessario essere registrati. Per poter eseguire il cambio password bisogna premere sul pulsante profilo posto nella barra di navigazione.

	\begin{figure}[H]
		\centering
		\includegraphics[scale=0.75]{\imgs {profilo}.jpg} %inserire il diagramma UML
		\label{profilo}
		\caption{Procedura cambio password}
	\end{figure}
\item Cambia Password: Premere il pulsante Cambia password, apparirà il form per il cambio password;
\item Password Attuale: Inserire la password attuale associata al proprio account;
\item Nuova Password: Inserire una nuova password, per motivi di sicurezza sono accettabili solo password più lunghe di 6 caratteri;
\item Conferma Nuova Password: Inserire nuovamente la password;
\item Submit: Preme il pulsante submit per confermare i dati e procedere al cambio password.
\end{enumerate}
	\begin{figure}[H]
		\centering
		\includegraphics[scale=0.75]{\imgs {cambiopassword}.jpg} %inserire il diagramma UML
		\label{cambiopassword}
		\caption{Form cambio password}
	\end{figure}