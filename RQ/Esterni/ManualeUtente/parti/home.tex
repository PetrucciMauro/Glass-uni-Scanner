\section{Home}
\subsection{Creazione nuova presentazione}
\begin{enumerate}
\item Per poter creare una nuova presentazione è necessario essere autenticati. Per poter creare una nuova presentazione Basterà premere il pulsante Nuova Presentazione
	\begin{figure}[H]
		\centering
		\includegraphics[scale=0.75]{\imgs {creanuovapres1}.jpg} %inserire il diagramma UML
		\label{creanuovapres1}
		\caption{Schermata creazione nuova presentazione 1}
	\end{figure}
\item Nel campo di compilazione inserire un nome per la nuova presentazione e successivamente preme il pulsante Submit
	\begin{figure}[H]
		\centering
		\includegraphics[scale=0.75]{\imgs {creanuovapres2}.jpg} %inserire il diagramma UML
		\label{creanuovapres2}
		\caption{Schermata creazione nuova presentazion 2}
	\end{figure}
\end{enumerate}
\subsection{Eliminazione Presentazione}
\begin{enumerate}
\item Per poter eliminare una presentazione è necessario essere autenticati e successivamente aver creato una presentazione. Per poter eliminare una nuova presentazione basterà premere sul pulsante elimina corrispondente al nome della presentazione che si vuole eliminare.
	\begin{figure}[H]
		\centering
		\includegraphics[scale=0.75]{\imgs {eliminapres}.jpg} %inserire il diagramma UML
		\label{eliminapres}
		\caption{Eliminazione presentazione}
	\end{figure}

\end{enumerate}
\subsection{Rinomina Presentazione}
\begin{enumerate}
\item Per poter rinominare una presentazione è necessario essere autenticati e successivamente aver creato una presentazione. Per poter rinominare una presentazione basterà preme sul pulsante eliminaa corrsipondente al nome della presentazione che si vuole rinominare;
	\begin{figure}[H]
		\centering
		\includegraphics[scale=0.75]{\imgs {rinominapresentazione1}.jpg} %inserire il diagramma UML
		\label{rinominapresentazione1}
		\caption{Schermata rinomina presentazione 1}
	\end{figure}
\item Inserire il nuovo nome della presentazione nella finestra che appare;
\item Premere Ok per confermare e rinominare la presentazione o Annulla per annullare.
	\begin{figure}[H]
		\centering
		\includegraphics[scale=0.75]{\imgs {rinominapresentazione2}.jpg} %inserire il diagramma UML
		\label{rinominapresentazione2}
		\caption{Schermata rinomina presentazione 2}
	\end{figure}
\end{enumerate}
\subsection{Salvataggio Presentazione}
\begin{enumerate}
\item Per poter salvare una presentazione è necessario essere autenticati e successivamente aver creato una presentazione. Per poter salvare una nuova presentazione basterà premere sul pulsante Salva corrispondente al nome della presentazione che si vuole salvare.
	\begin{figure}[H]
		\centering
		\includegraphics[scale=0.75]{\imgs {salvapres}.jpg} %inserire il diagramma UML
		\label{salvapres}
		\caption{Salvataggio presentazione}
	\end{figure}
\end{enumerate}
\subsection{Editare una presentazione}

\begin{enumerate}
\item Per poter editare una presentazione è necessario essere autenticati e successivamente aver creato una presentazione. Per poter editare una presentazione basterà premere sul pulsante Editor corrispondente al nome della presentazione che si vuole editare.
	\begin{figure}[H]
		\centering
		\includegraphics[scale=0.75]{\imgs {goeditor}.jpg} %inserire il diagramma UML
		\label{goeditor}
		\caption{Come editare una presentazione}
	\end{figure}
\end{enumerate}
\subsection{Esecuzione presentazione}
\begin{enumerate}
\item Per poter eseguire una presentazione è necessario essere autenticati e successivamente aver creato una presentazione. Per poter eseguire una presentazione basterà premere sul pulsante Esegui corrispondente al nome della presentazione che si vuole eseguire.
	\begin{figure}[H]
		\centering
		\includegraphics[scale=0.75]{\imgs {goesecuzione}.jpg} %inserire il diagramma UML
		\label{goesecuzione}
		\caption{Come eseguire una presentazione}
	\end{figure}
\end{enumerate}
