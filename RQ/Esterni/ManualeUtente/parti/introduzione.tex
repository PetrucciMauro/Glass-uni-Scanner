\section{Introduzione}
\subsection{Scopo del documento}
Questo documento è rivolto all’utente, ha lo scopo di illustrargli le procedure da seguire per svolgere le operazioni utente di \premi . All'utilizzatore non è richiesta alcuna conoscenza informatica poiché dovrà interfacciarsi, tramite web browser, alle funzionalità di \premi che vengono erogate con le stesse modalità di un normale sito internet.
\subsection{Scopo del Prodotto}
Lo scopo del Progetto è la realizzazione un Software\ped{g} per la creazione ed esecuzione di presentazioni multimediali favorendo l’uso di tecniche di storytelling e visualizzazione non lineare dei contenuti.
\subsection{Glossario}
Al fine di evitare ogni ambiguità di linguaggio e massimizzare la comprensione del documento, i termini tecnici, di dominio, gli acronimi e le parole che necessitano di essere chiarite, sono qui sotto riportate, Ogni occorrenza di vocaboli presenti nel Glossario è marcata da una “g” minuscola in pedice.
\begin{itemize}
\item \textbf{Editor}: programma per la modifica di contenuti testuali o multimediali. Un semplice editor è generalmente incluso in ogni sistema operativo;
\item \textbf{Frame}: ciascuna delle aree di schermo che visualizzano parti indipendenti di contenuto durante la visualizzazione di una presentazione;
\item \textbf{Software}: set di istruzioni interpretabili da una macchina che guidano le componenti di un computer a svolgere specifiche operazioni.

\end{itemize}



