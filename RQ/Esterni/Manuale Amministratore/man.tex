\documentclass[a4paper,12pt]{article}
\usepackage{../../Template/format}

%	VARIABILI
%	Percorsi file
\newcommand{\parti}{./parti/}
\newcommand{\imgs}{./images/} %se esistono
\newcommand{\temp}{../../Template/}

%	Informazioni documento
\newcommand{\titoloDoc}{Manuale Amministratore}
\newcommand{\dataCreazione}{19-08-2015}
\newcommand{\versione}{5.0.0} % versione corrente del documento
\newcommand{\dataUM}{19-08-2015} % data dell'ultima modifica
\newcommand{\stato}{Formale}
\newcommand{\uso}{Esterno}
\newcommand{\redaz}{\FM, \BM}
\newcommand{\verif}{\GP}
\newcommand{\appr}{\VG}
\newcommand{\sommario}{
	Il presente documento consiste del manuale per l'amministratore del sistema \premi.
	}
\title{\titoloDoc}

%	HEADER
\rhead{\titoloDoc \ v.\versione}

%	CORPO
\begin{document}
	\input{\temp firstpage} % prima pagina
	\newpage
	\input{\temp sommario} % breve sommario
	\newpage
	\input{\parti registroMod} % registro delle modifiche
	\newpage
	\tableofcontents % crea l'indice
	\newpage
	\listoffigures
	\listoftables
	\newpage
	
\section{Introduzione}
\subsection{Scopo del documento}
Questo documento rappresenta il manuale amministratore per il sistema software Premi, vengono descritte le procedure per la corretta installazione e avviamento del server nodeJs  e per l'avvio e configurazione del database MongoDB.

\subsection{Glossario}
Per ridurre il rischio di ambiguità di linguaggio e migliorare la comprensione del documento, i termini tecnici, di dominio e gli acronimi sono riportate nel documento Glossario v5.x.x

\section{Requisiti}
Per poter avviare correttamente il server deve essere installato sul PC ospitante il server le tecnologie MongoDB e nodeJs. I sistemi operativi supportati sono OSX Yosemite, Ubuntu dalla versione 12.04LTS, Windows7 o successivi.

\section{Installazione e Avvio}

Di seguito sono riportati i passi per la configurazione del database MongoDB e l'avviamento del server per i sistemi operativi OSX Yosemite, Ubuntu  12.04LTS e Windows7:

%\subsection{OSX Yosemite}
\begin{enumerate}
\item creare una cartella data per raccogliere in database creati con MongoDB
mongod --dbpath /usr/local/mongodb-data
\item avviamento mongodb con il comando "sudo mongod" (--port xxxxx se si vuole usare una porta diversa da quella di default), specificando la cartella per il savataggio e recupero dei darabase con l'opzione "mongod --dbpath /mia/cartella/"
\item  spostarsi da terminale nella cartella Premi (o quella in cui \`{e} presente lo script mongoConfig se spostato) ed accedere a mongoDB con il comando "mongo", avvenuto l'accesso configurare il database con il comando "load('mongoConfig.js')"
\item se \`{e} stato avviato mongod con una porta differente da qulla di default modificare il file config.js all'interno della cartella Premi specificando la porta con cui \`{e} stato avviato mongod
\item  avviamento server node spostandosi da terminale sulla cartella premi e lanciare il comando "node premi\_Server.js" % modifica porta avviamento server node
%\item configurazione premi per config porta

\end{enumerate}

	
\end{document}