\section{Requisiti}  %requisiti OS e pacchetti 
\subsection{Sistema operativo}
Il  computer utilizzato come server dovr\`{a} avere i seguenti requisiti in termini di sistema operativo:
\begin{itemize}
\item  Mac OS versione 10.10 o successiva
\item Ubuntu versione 12.04 o successiva
\item Windows7 e successivi
\end{itemize}

\subsection{Tecnologie e pacchetti}
Per poter avviare correttamente il sistema \premi, devono essere installate sul PC ospitante le seguenti tecnologie:
\begin{itemize}
\item  MongoDb dalla versione 3.0.5
\item  Node.js dalla versione 0.12.7
\item  npm dalla versione 2.11.3
\item  bower dalla versione 1.4.1
\item tutti i pacchetti npm listati nel file package.json con le versioni descritte nel file stesso
\item tutti i pacchetti bower listati nel file bower.json con le versioni descritte nel file stesso
\end{itemize}

\newpage
\section{Installazione} % recupero e installazione pacchetti

\subsection{Installazione tecnologie}
Di seguito le istruzioni per installare le tecnologie dette:
\begin{itemize}
\item \textbf{MongoDB}: recuperabile dalla pagina \textbf{https://www.mongodb.org}, sono disponibili diverse versioni scaricabili a seconda del sistema operativo, per l'installazione seguire il wizard di supporto
\item \textbf{Node.js}: recuperabile dalla pagina \textbf{https://nodejs.org/en/}, sono disponibili diverse versioni scaricabili a seconda del sistema operativo, per l'installazione seguire il wizard di supporto
\item \textbf{npm}: viene recuperato ed installato automaticamente con l'installazione di Node.js per ogni versione successiva alla 0.10.32, altrimenti recuperabile dalla pagina \\ \textbf{http://nodejs.org/download/}
\item \textbf{bower}: per l'installazione \`{e} necessario aver gi\`{a} installato npm, istruzioni per l'installazione alla pagina \textbf{http://bower.io}
\end{itemize}

\subsection{Installazione pacchetti}
Una volta installate le tecnologie npm e bower si devono recuperare i pacchetti descritti nei file package.json e bower.json.
Il recupero e l'installazione di questi pacchetti pu\`{o} essere svolta facilmente da riga di comando come segue:
\begin{itemize}
\item dalla cartella radice del sistema installare i pacchetti npm con il comando "\textbf{npm install}"
\item dalla cartella radice del sistema installare i pacchetti bower con il comando "\textbf{bower install}"
\end{itemize}

