\section{Requisiti}
Per poter avviare correttamente il Server\ped{g} deve essere installato sul PC\ped{g} ospitante il Server\ped{g} le tecnologie MongoDB e nodeJs. I sistemi operativi supportati sono OSX Yosemite, Ubuntu\ped{g} dalla versione 12.04LTS, Windows7 o successivi.

\section{Installazione e Avvio}

Di seguito sono riportati i passi per la configurazione del database MongoDB e l'avviamento del Server\ped{g} per i sistemi operativi OSX Yosemite, Ubuntu\ped{g}  12.04LTS e Windows7:

%\subsection{OSX Yosemite}
\begin{enumerate}
\item da terminale eseguire il comando npm install , per sistemi operativi che richiedono i permessi di root sarà necessario eseguire il comando sudo npm install, il comando eseguirà il download di tutti moduli necessari al funzionamento di \premi;
\item da terminale eseguire il comando bower install , il comando eseguirà il download del Framework\ped{g} angular con tutte le sue dipendenze;
\item creare una cartella data per raccogliere in database creati con MongoDB
mongod --dbpath /usr/local/mongodb-data
\item avviamento mongodb con il comando "sudo mongod" (--port xxxxx se si vuole usare una porta diversa da quella di default), specificando la cartella per il savataggio e recupero dei database con l'opzione "mongod --dbpath /mia/cartella/"
\item  spostarsi dal terminale nella cartella Premi (o quella in cui \`{e} presente lo script mongoConfig se spostato) ed accedere a mongoDB con il comando "mongo", avvenuto l'accesso configurare il database con il comando "load('mongoConfig.js')"
\item se \`{e} stato avviato mongod con una porta differente da qulla di default modificare il File\ped{g} config.js all'interno della cartella Premi specificando la porta con cui \`{e} stato avviato mongod
\item  avviamento Server\ped{g} node spostandosi da terminale sulla cartella premi e lanciare il comando "node premi\_Server\ped{g}.js" % modifica porta avviamento Server\ped{g} node
%\item configurazione premi per config porta

\end{enumerate}
