\subsection{UC 1.11 - Annulla/Ripristina infografica}{
	\label{uc1.11}
	\begin{figure}[H]
		\centering
		\includegraphics[scale=0.70]{\imgs {UC1.11}.jpg} %inserire il diagramma UML
		\label{fig:uc1.11}
		\caption{Caso d'uso 1.11: Annulla/Ripristina infografica}
	\end{figure}
	\textbf{Attori}: Utente mobile. \\
	\textbf{Descrizione}:L'utente pu� annullare un comando selezionato o ripristinare un comando annullato
	\textbf{Precondizione}:  l'utente ha effettuato l'accesso al sistema in modalit� modifica ad una  infografica \\
	\textbf{Postcondizione}: l'utente ha annullato un comando o ripristinato un comando annullato  \\
	
\subsubsection{UC 1.11.1 - Annulla modifica}{
		\label{uc1.11.1}
		\textbf{Attori}: utente. \\
		\textbf{Descrizione}:L'utente pu� annullare l'ultimo comando di modifica tra:. \\
		\begin{itemize}
			\item inserimento frame
			\item eliminazione frame
			\item spostamento frame
			\item modifica frame
			\item inserimento svg
			\item modifica svg
			\item eliminazione elemento
			\item inserimento bookmark
			\item cancellazione bookmark
			\item modifica nella definizione di percorso
		\end{itemize}
		\textbf{Precondizione}: Il sistema ha registrato almeno un comando di modifica della infografica da parte dell'utente. \\
		\textbf{Postcondizione}: Il sistema ha ripristinato lo stato precedente all'ultimo comando di modifica della infografica dell'utente, e ha memorizzato nello storico degli annullamenti il comando annullato.	\\
		\textbf{Scenario principale}:
		\begin{enumerate}
			\item l'utente seleziona l'azione annulla 
		\end{enumerate}
		}
		
\subsubsection{UC 1.11.2 - Ripristina modifica}{
		\label{uc1.11.2}
		\textbf{Attori}: Utente mobile. \\
		\textbf{Descrizione}: L'utente pu� ripristinare l'ultimo comando presente nello storico dei comandi annullati.
		\textbf{Precondizione}: Lo storico degli annullamenti di comando non � vuoto. \\
		\textbf{Postcondizione}: Il sistema ha rieseguito l'ultimo comando di modifica della infografica presente nello storico dei comandi annullati, ed � stato rimosso il comando rieseguito dallo storico.
		}


