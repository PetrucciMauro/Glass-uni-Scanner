\subsection{UC 1.17 - Modifica infografica}{
	\label{uc1.17}
	\begin{figure}[H]
		\centering
		\includegraphics[scale=0.6]{\imgs {UC1.17}.jpg} %inserire il diagramma UML
		\label{fig:uc1.17}
		\caption{Caso d'uso 1.17: Gestione delle infografiche}
	\end{figure}
	\textbf{Attori}: utente Desktop\ped{g}. \\
	\textbf{Descrizione}: l'utente è in grado di prendere i Frame\ped{g} di una presentazione e di inserirli in un unico documento stampabile. \\
	\textbf{Precondizione}: il sistema è acceso e funzionante.	\\
	\textbf{Postcondizione}: l'utente ha modificato un'Infografica\ped{g} che può aver salvato in locale o online.	\\
	\textbf{Scenario principale}:
	\begin{enumerate}
		\item Selezione della presentazione di cui produrre l’Infografica\ped{g} \S\hyperref[uc1.17.1]{(UC 1.17.1)};
		\item Selezione di un Template\ped{g} di Infografica\ped{g} \S\hyperref[uc1.17.2]{(UC 1.17.2)};
		\item Selezione di un Elemento\ped{g} dell’Infografica\ped{g} \S\hyperref[uc1.17.3]{(UC 1.17.3)};
		\item Modifica e rimozione di un Elemento\ped{g} grafico o di testo \S\hyperref[uc1.17.4]{(UC 1.17.4)};
		\item Salvataggio dell'Infografica\ped{g} \S\hyperref[uc1.17.5]{(UC 1.17.5)}.
	\end{enumerate}
	\textbf{Scenari alternativi}: 
	\begin{itemize}
		\item L'operazione viene annullata e non si apportano cambiamenti \S\hyperref[uc1.5]{(UC 1.5)}.
	\end{itemize}
	
	}
\subsubsection{UC 1.17.1 - Selezione della presentazione}{
	\label{uc1.17.1}
	\textbf{Attori}: utente Desktop\ped{g}. \\
	\textbf{Descrizione}: l'utente è in grado di selezionare la presentazione da cui produrre l'Infografica\ped{g}. \\
	\textbf{Precondizione}: il sistema è attivo e funzionante e mostra all'utente le presentazioni disponibili.	\\
	\textbf{Postcondizione}: nell'Editor\ped{g} viene caricata una presentazione sui quali oggetti si può andare a interagire.\\
	\textbf{Scenario principale}:
	\begin{enumerate}
		\item L'utente seleziona la presentazione di cui produrre l’Infografica\ped{g}.
	\end{enumerate}
	}
\subsubsection{UC 1.17.2 - Selezione di un template}{
	\label{uc1.17.2}
	\textbf{Attori}: utente Desktop\ped{g}. \\
	\textbf{Descrizione}: permettere all'utente di selezionare il Template\ped{g} che preferisce. \\
	\textbf{Precondizione}: il sistema è attivo e funzionante.\\
	\textbf{Postcondizione}: l'utente ha caricato nell'Editor\ped{g} un Template\ped{g} che potrà modificare per creare la sua Infografica\ped{g}.\\
	\textbf{Scenario principale}:
	\begin{enumerate}
		\item L'utente seleziona un Template\ped{g} tra quelli disponibili.
	\end{enumerate}	
	}
\subsubsection{UC 1.17.3 - Selezione di un Elemento\ped{g} dell'infografica}{
	\label{uc1.17.3}
	\textbf{Attori}: utente Desktop\ped{g}. \\
	\textbf{Descrizione}: l'utente è in grado di selezionare elementi\ped{g} specifici del Template\ped{g} dell'Infografica\ped{g} o di quelli da lui inseriti. \\
	\textbf{Precondizione}: il sistema è acceso e funzionante, un Template\ped{g} di Infografica\ped{g} è stato caricato nell'Editor\ped{g}.	\\
	\textbf{Postcondizione}: l'Elemento\ped{g} è stato selezionato.\\
	\textbf{Scenario principale}:
	\begin{enumerate}
		\item L'utente seleziona un Elemento\ped{g} dell'Infografica\ped{g}.
	\end{enumerate}		
	}
\subsubsection{UC 1.17.4 - Modifica infografica}{
	\label{uc1.17.4}
	\begin{figure}[H]
		\centering
		\includegraphics[scale=0.75]{\imgs {UC1.17.4}.jpg} %inserire il diagramma UML
		\label{fig:uc1.17.4}
		\caption{Caso d'uso 1.17.4: Modifica di un'infografica}
	\end{figure}
	\textbf{Attori}: utente Desktop\ped{g}. \\
	\textbf{Descrizione}: l'utente è in grado di selezionare elementi\ped{g} specifici dell'Infografica\ped{g}. \\
	\textbf{Precondizione}: il sistema è acceso e funzionante e un'Infografica\ped{g} è stata caricata in modalità modifica.	\\
	\textbf{Postcondizione}: l'Infografica\ped{g} è stata modificata.	\\
	\textbf{Scenario principale}:
	\begin{enumerate}
		\item Modifica di un Elemento\ped{g} dell'Infografica\ped{g} \S\hyperref[uc1.17.4.1]{(UC 1.17.4.1)};
		\item Rimozione dello sfondo \S\hyperref[uc1.17.4.2]{(UC 1.17.4.2)};
		\item Inserimento di uno sfondo \S\hyperref[uc1.17.4.3]{(UC 1.17.4.3)};
		\item Inserimento di un Elemento\ped{g} grafico \S\hyperref[uc1.17.4.4]{(UC 1.17.4.4)};
		\item Inserimento di un Elemento\ped{g} di testo \S\hyperref[uc1.17.4.5]{(UC 1.17.4.5)};
		\item Inserimento Frame\ped{g} \S\hyperref[uc1.17.4.6]{(UC 1.17.4.6)};
		\item Eliminazione di un Elemento\ped{g} \S\hyperref[uc1.17.4.7]{(UC 1.17.4.7)}.
	\end{enumerate}
}
\subsubsection{UC 1.17.4.1 - Modifica Elemento\ped{g} infografica}{
	\label{uc1.17.4.1}
	\begin{figure}[H]
		\centering
		\includegraphics[scale=0.75]{\imgs {UC1.17.4.1}.jpg} %inserire il diagramma UML
		\label{fig:uc1.17.4.1}
		\caption{Caso d'uso 1.17.4.1: Modifica di un Elemento\ped{g} dell'infografica}
	\end{figure}
	\textbf{Attori}: utente Desktop\ped{g}. \\
	\textbf{Descrizione}: l'utente è in grado di selezionare elementi\ped{g} specifici dell'Infografica\ped{g}. \\
	\textbf{Precondizione}: il sistema è acceso e funzionante ed un Elemento\ped{g} dell'Infografica\ped{g}  è stato selezionato.	\\
	\textbf{Postcondizione}: l'Elemento\ped{g} selezionato è stato modificato.	\\
	\textbf{Scenario principale}:
	\begin{enumerate}
		\item Per un Elemento\ped{g} grafico è possibile ingrandire o ridurre le dimensioni \S\hyperref[uc1.17.4.1.1]{(UC 1.17.4.1.1)};
		\item Per un Elemento\ped{g} testuale è possibile effettuare diverse modifiche \S\hyperref[uc1.17.4.1.2]{(UC 1.17.4.1.2)};
		\item In entrambi i casi è possibile cambiare la posizione dell'Elemento\ped{g} \S\hyperref[uc1.17.4.1.3]{(UC 1.17.4.1.3)}.
	\end{enumerate}
	}
\subsubsection{UC 1.17.4.1.1 - Modifica dimensioni di un Elemento\ped{g} grafico}{
	\label{uc1.17.4.1.1}
	\textbf{Attori}: utente Desktop\ped{g}. \\
	\textbf{Descrizione}: l'utente è in grado di aumentare e ridurre le dimensioni di un Elemento\ped{g} grafico. \\
	\textbf{Precondizione}: è stato selezionato un Elemento\ped{g} grafico.\\
	\textbf{Postcondizione}: l'Elemento\ped{g} selezionato ha cambiato le sue dimensioni.\\
	\textbf{Scenario principale}:
	\begin{enumerate}
		\item L'utente modifica le dimensioni dell'Elemento\ped{g} selezionato.
	\end{enumerate}		
	}
\subsubsection{UC 1.17.4.1.2 - Modifica Elemento\ped{g} testuale}{
	\label{uc1.17.4.1.2}
	\begin{figure}[H]
		\centering
		\includegraphics[scale=0.75]{\imgs {UC1.17.4.1.2}.jpg} %inserire il diagramma UML
		\label{fig:uc1.17.4.1.2}
		\caption{Caso d'uso 1.17.4.1.2: Modifica di un Elemento\ped{g} testuale}
	\end{figure}
	\textbf{Attori}: utente Desktop\ped{g}. \\
	\textbf{Descrizione}: l'utente è in grado di modificare un Elemento\ped{g} testuale. \\
	\textbf{Precondizione}: un Elemento\ped{g} testuale dell'Infografica\ped{g} è stato selezionato.	\\
	\textbf{Postcondizione}: l'Elemento\ped{g} selezionato testuale è stato modificato.	\\
	\textbf{Scenario principale}:
	\begin{enumerate}
		\item Cambiare il carattere \S\hyperref[uc1.17.4.1.2.1]{(UC 1.17.4.1.2.1)};
		\item Modificare le dimensioni del carattere \S\hyperref[uc1.17.4.1.2.2]{(UC 1.17.4.1.2.2)};
		\item Una formattazione (corsivo, grassetto, sottolineato) \S\hyperref[uc1.17.4.1.2.3]{(UC 1.17.4.1.2.3)};
		\item Cambiare colore del carattere \S\hyperref[uc1.17.4.1.2.4]{(UC 1.17.4.1.2.4)};
		\item Cambiare il colore di sfondo \S\hyperref[uc1.17.4.1.2.5]{(UC 1.17.4.1.2.5)}.
	\end{enumerate}
	}
\subsubsection{UC 1.17.4.1.2.1 - Modifica del carattere}{
	\label{uc1.17.4.1.2.1}
	\textbf{Attori}: utente Desktop\ped{g}. \\
	\textbf{Descrizione}: l'utente è in grado di cambiare il carattere del testo nell’Elemento\ped{g} testuale. \\
	\textbf{Precondizione}: l'Elemento\ped{g} testuale è stato selezionato e si trova ora in modalità modifica.\\
	\textbf{Postcondizione}: è stato cambiato il carattere dell'Elemento\ped{g} testo selezionato.\\
	\textbf{Scenario principale}:
	\begin{enumerate}
		\item L'utente modifica il carattere dell'Elemento\ped{g} testuale.
	\end{enumerate}		
	}
\subsubsection{UC 1.17.4.1.2.2 - Modifica delle dimensioni del carattere}{
	\label{uc1.17.4.1.2.2}
	\textbf{Attori}: utente Desktop\ped{g}. \\
	\textbf{Descrizione}: l'utente è in grado di cambiare le dimensioni delle lettere dell’Elemento\ped{g} testuale. \\
	\textbf{Precondizione}: l'Elemento\ped{g} testuale è stato selezionato e si trova ora in modalità modifica.\\
	\textbf{Postcondizione}: il testo ha cambiato le dimensioni del carattere.\\
	\textbf{Scenario principale}:
	\begin{enumerate}
		\item L'utente modifica le dimensioni del testo nell'Elemento\ped{g} testuale.
	\end{enumerate}	
	}
\subsubsection{UC 1.17.4.1.2.3 - Modifica della formattazione}{
	\label{uc1.17.4.1.2.3}
	\textbf{Attori}: utente Desktop\ped{g}. \\
	\textbf{Descrizione}: l'utente è in grado di cambiare la formattazione dell’Elemento\ped{g} testuale tra quelli disponibili (corsivo, grassetto, sottolineato) oppure tornare al testo non formattato. \\
	\textbf{Precondizione}: l'Elemento\ped{g} testuale è stato selezionato e si trova ora in modalità modifica.\\
	\textbf{Postcondizione}: il testo ha cambiato formato.\\
	\textbf{Scenario principale}:
	\begin{enumerate}
		\item L'utente modifica la formattazione dell'Elemento\ped{g} testuale.
	\end{enumerate}			
	}
\subsubsection{UC 1.17.4.1.2.4 - Modifica del colore}{
	\label{uc1.17.4.1.2.4}
	\textbf{Attori}: utente Desktop\ped{g}. \\
	\textbf{Descrizione}: l'utente è in grado di cambiare il colore del testo contenuto nell’Elemento\ped{g} testuale. \\
	\textbf{Precondizione}: l'Elemento\ped{g} testuale è stato selezionato e si trova ora in modalità modifica.\\
	\textbf{Postcondizione}: è stato cambiato il colore dei caratteri dell'Elemento\ped{g} testuale.\\
	\textbf{Scenario principale}:
	\begin{enumerate}
		\item L'utente modifica il colore del testo nell'Elemento\ped{g} testuale.
	\end{enumerate}			
	}
\subsubsection{UC 1.17.4.1.2.5 - Modifica sfondo}{
	\label{uc1.17.4.1.2.5}
	\textbf{Attori}: utente Desktop\ped{g}. \\
	\textbf{Descrizione}: l'utente è in grado di cambiare il colore dello sfondo dell’Elemento\ped{g} testuale (evidenziatura). \\
	\textbf{Precondizione}: l'Elemento\ped{g} testuale è stato selezionato e si trova ora in modalità modifica.\\
	\textbf{Postcondizione}: le lettere del testo sono state evidenziate.\\
	\textbf{Scenario principale}:
	\begin{enumerate}
		\item L'utente evidenzia il testo nell’Elemento\ped{g} testuale.
	\end{enumerate}			
	}
\subsubsection{UC 1.17.4.1.3 - Spostamento di un elemento}{
	\label{uc1.17.4.1.3}
	\textbf{Attori}: utente Desktop\ped{g}. \\
	\textbf{Descrizione}: l'utente è in grado di cambiare la posizione di un Elemento\ped{g}. \\
	\textbf{Precondizione}: l'Elemento\ped{g} è stato selezionato.\\
	\textbf{Postcondizione}: l'Elemento\ped{g} è stato spostato all’interno dell’Infografica\ped{g}.\\
	\textbf{Scenario principale}:
	\begin{enumerate}
		\item L'utente modifica la posizione dell'Elemento\ped{g}.
	\end{enumerate}			
	}
\subsubsection{UC 1.17.4.2 - Rimozione dello sfondo}{
	\label{uc1.17.4.2}
	\textbf{Attori}: utente Desktop\ped{g}. \\
	\textbf{Descrizione}: l'utente è in grado di rimuovere lo sfondo dell’Infografica\ped{g}. \\
	\textbf{Precondizione}: il sistema è acceso e funzionante ed è stata caricata un'Infografica\ped{g}.	\\
	\textbf{Postcondizione}: lo sfondo dell'Infografica\ped{g} è stato eliminato.\\
	\textbf{Scenario principale}:
	\begin{enumerate}
		\item L'utente rimuove lo sfondo dell'Infografica\ped{g}.
	\end{enumerate}			
	}
\subsubsection{UC 1.17.4.3 - Inserimento di uno sfondo}{
	\label{uc1.17.4.3}
	\textbf{Attori}: utente Desktop\ped{g}. \\
	\textbf{Descrizione}: l'utente è in grado di scegliere uno sfondo per l'Infografica\ped{g}. \\
	\textbf{Precondizione}: il sistema è acceso e funzionante ed è stata caricata un'Infografica\ped{g}.	\\
	\textbf{Postcondizione}: lo sfondo dell'Infografica\ped{g} è stato cambiato.\\
	\textbf{Scenario principale}:
	\begin{enumerate}
		\item L'utente inserisce uno sfondo nell'Infografica\ped{g}.
	\end{enumerate}			
	}
\subsubsection{UC 1.17.4.4 - Inserimento di un Elemento\ped{g} grafico}{
	\label{uc1.17.4.4}
	\textbf{Attori}: utente Desktop\ped{g}. \\
	\textbf{Descrizione}: l'utente è in grado di inserire un Elemento\ped{g} grafico. \\
	\textbf{Precondizione}: il sistema è acceso e funzionante ed è stata caricata un'Infografica\ped{g}.	\\
	\textbf{Postcondizione}: è stato inserito un nuovo Elemento\ped{g} nell'Infografica\ped{g}.\\
	\textbf{Scenario principale}:
	\begin{enumerate}
		\item Quando un Elemento\ped{g} grafico viene inserito si entra automaticamente in modalità modifica \S\hyperref[uc1.17.4.1]{(UC 1.17.4.1)}.
	\end{enumerate}
	}
\subsubsection{UC 1.17.4.5 - Inserimento di un Elemento\ped{g} testuale}{
	\label{uc1.17.4.5}
	\textbf{Attori}: utente Desktop\ped{g}. \\
	\textbf{Descrizione}: l'utente è in grado di inserire un Elemento\ped{g} testuale nell'Infografica\ped{g}. \\
	\textbf{Precondizione}: il Programma\ped{g} è acceso e funzionante ed è stata caricata un'Infografica\ped{g}.	\\
	\textbf{Postcondizione}: è stato inserito un Elemento\ped{g} testuale precedentemente assente nell'Infografica\ped{g}.	\\
	\textbf{Scenario principale}:
	\begin{enumerate}
		\item Quando viene inserito un Elemento\ped{g} testuale questo entra automaticamente in modalità modifica \S\hyperref[uc1.17.4.1.2]{(UC 1.17.4.1.2)}.
	\end{enumerate}
	}
\subsubsection{UC 1.17.4.6 - Inserimento di un frame}{
	\label{uc1.17.4.6}
	\textbf{Attori}: utente Desktop\ped{g}. \\
	\textbf{Descrizione}: l'utente è in grado di inserire un Frame\ped{g} come immagine vettoriale nell'Infografica\ped{g}. \\
	\textbf{Precondizione}: il Programma\ped{g} è acceso e funzionante, è stata caricata un'Infografica\ped{g} ed una presentazione di cui l'utente vuole fare l'Infografica\ped{g}.	\\
	\textbf{Postcondizione}: nell'Infografica\ped{g} è stato inserito un nuovo Frame\ped{g}.\\
	\textbf{Scenario principale}:
	\begin{enumerate}
		\item L'utente inserisce un nuovo Frame\ped{g} nell'Infografica\ped{g}.
	\end{enumerate}			
	}
\subsubsection{UC 1.17.4.7 - Eliminazione di un elemento}{
	\label{uc1.17.4.7}
	\textbf{Attori}: utente Desktop\ped{g}. \\
	\textbf{Descrizione}: l'utente è in grado di eliminare un Elemento\ped{g} dell'Infografica\ped{g}. \\
	\textbf{Precondizione}: il sistema è acceso e funzionante, è stata caricata un'Infografica\ped{g} ed è stato selezionato un Elemento\ped{g} dell'Infografica\ped{g}.	\\
	\textbf{Postcondizione}: è stato rimosso un Elemento\ped{g} dall'Infografica\ped{g}.\\
	\textbf{Scenario principale}:
	\begin{enumerate}
		\item L'utente elimina un Elemento\ped{g} dall'Infografica\ped{g}.
	\end{enumerate}			
	}
\subsubsection{UC 1.17.5 - Salvataggio infografica}{
	\label{uc1.17.5}
	\textbf{Attori}: utente Desktop\ped{g}. \\
	\textbf{Descrizione}: l'utente può salvare l'Infografica\ped{g} su cui stava lavorando. \\
	\textbf{Precondizione}: il sistema è acceso e funzionante e l'utente ha caricato un'Infografica\ped{g}.	\\
	\textbf{Postcondizione}: l'utente ha salvato l’Infografica\ped{g} su cui stava lavorando nel proprio spazio personale.\\
	\textbf{Scenario principale}:
	\begin{enumerate}
		\item L'utente salva l'Infografica\ped{g} nel proprio spazio personale.
	\end{enumerate}			
	}
\subsubsection{UC 1.17.7 - Esportazione infografica}{
	\label{uc1.17.7}
	\textbf{Attori}: utente Desktop\ped{g}. \\
	\textbf{Descrizione}: l'utente può esportare un'Infografica\ped{g} come File\ped{g} immagine nel proprio dispositivo. \\
	\textbf{Precondizione}: il sistema è acceso e funzionante e l'utente ha caricato un'Infografica\ped{g}.	\\
	\textbf{Postcondizione}: l'utente ha esportato l'Infografica\ped{g} sotto forma di File\ped{g} immagine nel proprio dispositivo.\\
	\textbf{Scenario principale}:
	\begin{enumerate}
		\item L'utente esporta l'Infografica\ped{g} come immagine nel proprio dispositivo.
	\end{enumerate}			
	}