\subsection{UC 1.17 - Modifica infografica}{
	\label{uc1.17}
	\begin{figure}[H]
		\centering
		\includegraphics[scale=0.6]{\imgs {UC1.17}.jpg} %inserire il diagramma UML
		\label{fig:uc1.17}
		\caption{Caso d'uso 1.17: Gestione delle infografiche}
	\end{figure}
	\textbf{Attori}: utente desktop. \\
	\textbf{Descrizione}: l'utente è in grado di prendere i frame di una presentazione e di inserirli in un unico documento stampabile. \\
	\textbf{Precondizione}: il sistema è acceso e funzionante.	\\
	\textbf{Postcondizione}: l'utente ha modificato un'infografica che può aver salvato in locale o online.	\\
	\textbf{Scenario principale}:
	\begin{enumerate}
		\item Selezione della presentazione di cui produrre l’infografica \S\hyperref[uc1.17.1]{(UC 1.17.1)};
		\item Selezione di un template di infografica \S\hyperref[uc1.17.2]{(UC 1.17.2)};
		\item Selezione di un elemento dell’infografica \S\hyperref[uc1.17.3]{(UC 1.17.3)};
		\item Modifica e rimozione di un elemento grafico o di testo \S\hyperref[uc1.17.4]{(UC 1.17.4)};
		\item Salvataggio dell'infografica \S\hyperref[uc1.17.5]{(UC 1.17.5)}.
	\end{enumerate}
	\textbf{Scenari alternativi}: 
	\begin{itemize}
		\item L'operazione viene annullata e non si apportano cambiamenti \S\hyperref[uc1.5]{(UC 1.5)}.
	\end{itemize}
	
	}
\subsubsection{UC 1.17.1 - Selezione della presentazione}{
	\label{uc1.17.1}
	\textbf{Attori}: utente desktop. \\
	\textbf{Descrizione}: l'utente è in grado di selezionare la presentazione da cui produrre l'infografica. \\
	\textbf{Precondizione}: il sistema è attivo e funzionante e mostra all'utente le presentazioni disponibili.	\\
	\textbf{Postcondizione}: nell'editor viene caricata una presentazione sui quali oggetti si può andare a interagire.\\
	\textbf{Scenario principale}:
	\begin{enumerate}
		\item L'utente seleziona la presentazione di cui produrre l’infografica.
	\end{enumerate}
	}
\subsubsection{UC 1.17.2 - Selezione di un template}{
	\label{uc1.17.2}
	\textbf{Attori}: utente desktop. \\
	\textbf{Descrizione}: permettere all'utente di selezionare il template che preferisce. \\
	\textbf{Precondizione}: il sistema è attivo e funzionante.\\
	\textbf{Postcondizione}: l'utente ha caricato nell'editor un template che potrà modificare per creare la sua infografica.\\
	\textbf{Scenario principale}:
	\begin{enumerate}
		\item L'utente seleziona un template tra quelli disponibili.
	\end{enumerate}	
	}
\subsubsection{UC 1.17.3 - Selezione di un elemento dell'infografica}{
	\label{uc1.17.3}
	\textbf{Attori}: utente desktop. \\
	\textbf{Descrizione}: l'utente è in grado di selezionare elementi specifici del template dell'infografica o di quelli da lui inseriti. \\
	\textbf{Precondizione}: il sistema è acceso e funzionante, un template di infografica è stato caricato nell'editor.	\\
	\textbf{Postcondizione}: l'elemento è stato selezionato.\\
	\textbf{Scenario principale}:
	\begin{enumerate}
		\item L'utente seleziona un elemento dell'infografica.
	\end{enumerate}		
	}
\subsubsection{UC 1.17.4 - Modifica infografica}{
	\label{uc1.17.4}
	\begin{figure}[H]
		\centering
		\includegraphics[scale=0.75]{\imgs {UC1.17.4}.jpg} %inserire il diagramma UML
		\label{fig:uc1.17.4}
		\caption{Caso d'uso 1.17.4: Modifica di un'infografica}
	\end{figure}
	\textbf{Attori}: utente desktop. \\
	\textbf{Descrizione}: l'utente è in grado di selezionare elementi specifici dell'infografica. \\
	\textbf{Precondizione}: il sistema è acceso e funzionante e un'infografica è stata caricata in modalità modifica.	\\
	\textbf{Postcondizione}: l'infografica è stata modificata.	\\
	\textbf{Scenario principale}:
	\begin{enumerate}
		\item Modifica di un elemento dell'infografica \S\hyperref[uc1.17.4.1]{(UC 1.17.4.1)};
		\item Rimozione dello sfondo \S\hyperref[uc1.17.4.2]{(UC 1.17.4.2)};
		\item Inserimento di uno sfondo \S\hyperref[uc1.17.4.3]{(UC 1.17.4.3)};
		\item Inserimento di un elemento grafico \S\hyperref[uc1.17.4.4]{(UC 1.17.4.4)};
		\item Inserimento di un elemento di testo \S\hyperref[uc1.17.4.5]{(UC 1.17.4.5)};
		\item Inserimento frame \S\hyperref[uc1.17.4.6]{(UC 1.17.4.6)};
		\item Eliminazione di un elemento \S\hyperref[uc1.17.4.7]{(UC 1.17.4.7)}.
	\end{enumerate}
}
\subsubsection{UC 1.17.4.1 - Modifica elemento infografica}{
	\label{uc1.17.4.1}
	\begin{figure}[H]
		\centering
		\includegraphics[scale=0.75]{\imgs {UC1.17.4.1}.jpg} %inserire il diagramma UML
		\label{fig:uc1.17.4.1}
		\caption{Caso d'uso 1.17.4.1: Modifica di un elemento dell'infografica}
	\end{figure}
	\textbf{Attori}: utente desktop. \\
	\textbf{Descrizione}: l'utente è in grado di selezionare elementi specifici dell'infografica. \\
	\textbf{Precondizione}: il sistema è acceso e funzionante ed un elemento dell'infografica  è stato selezionato.	\\
	\textbf{Postcondizione}: l'elemento selezionato è stato modificato.	\\
	\textbf{Scenario principale}:
	\begin{enumerate}
		\item Per un elemento grafico è possibile ingrandire o ridurre le dimensioni \S\hyperref[uc1.17.4.1.1]{(UC 1.17.4.1.1)};
		\item Per un elemento testuale è possibile effettuare diverse modifiche \S\hyperref[uc1.17.4.1.2]{(UC 1.17.4.1.2)};
		\item In entrambi i casi è possibile cambiare la posizione dell'elemento \S\hyperref[uc1.17.4.1.3]{(UC 1.17.4.1.3)}.
	\end{enumerate}
	}
\subsubsection{UC 1.17.4.1.1 - Modifica dimensioni di un elemento grafico}{
	\label{uc1.17.4.1.1}
	\textbf{Attori}: utente desktop. \\
	\textbf{Descrizione}: l'utente è in grado di aumentare e ridurre le dimensioni di un elemento grafico. \\
	\textbf{Precondizione}: è stato selezionato un elemento grafico.\\
	\textbf{Postcondizione}: l'elemento selezionato ha cambiato le sue dimensioni.\\
	\textbf{Scenario principale}:
	\begin{enumerate}
		\item L'utente modifica le dimensioni dell'elemento selezionato.
	\end{enumerate}		
	}
\subsubsection{UC 1.17.4.1.2 - Modifica elemento testuale}{
	\label{uc1.17.4.1.2}
	\begin{figure}[H]
		\centering
		\includegraphics[scale=0.75]{\imgs {UC1.17.4.1.2}.jpg} %inserire il diagramma UML
		\label{fig:uc1.17.4.1.2}
		\caption{Caso d'uso 1.17.4.1.2: Modifica di un elemento testuale}
	\end{figure}
	\textbf{Attori}: utente desktop. \\
	\textbf{Descrizione}: l'utente è in grado di modificare un elemento testuale. \\
	\textbf{Precondizione}: un elemento testuale dell'infografica è stato selezionato.	\\
	\textbf{Postcondizione}: l'elemento selezionato testuale è stato modificato.	\\
	\textbf{Scenario principale}:
	\begin{enumerate}
		\item Cambiare il carattere \S\hyperref[uc1.17.4.1.2.1]{(UC 1.17.4.1.2.1)};
		\item Modificare le dimensioni del carattere \S\hyperref[uc1.17.4.1.2.2]{(UC 1.17.4.1.2.2)};
		\item Una formattazione (corsivo, grassetto, sottolineato) \S\hyperref[uc1.17.4.1.2.3]{(UC 1.17.4.1.2.3)};
		\item Cambiare colore del carattere \S\hyperref[uc1.17.4.1.2.4]{(UC 1.17.4.1.2.4)};
		\item Cambiare il colore di sfondo \S\hyperref[uc1.17.4.1.2.5]{(UC 1.17.4.1.2.5)}.
	\end{enumerate}
	}
\subsubsection{UC 1.17.4.1.2.1 - Modifica del carattere}{
	\label{uc1.17.4.1.2.1}
	\textbf{Attori}: utente desktop. \\
	\textbf{Descrizione}: l'utente è in grado di cambiare il carattere del testo nell’elemento testuale. \\
	\textbf{Precondizione}: l'elemento testuale è stato selezionato e si trova ora in modalità modifica.\\
	\textbf{Postcondizione}: è stato cambiato il carattere dell'elemento testo selezionato.\\
	\textbf{Scenario principale}:
	\begin{enumerate}
		\item L'utente modifica il carattere dell'elemento testuale.
	\end{enumerate}		
	}
\subsubsection{UC 1.17.4.1.2.2 - Modifica delle dimensioni del carattere}{
	\label{uc1.17.4.1.2.2}
	\textbf{Attori}: utente desktop. \\
	\textbf{Descrizione}: l'utente è in grado di cambiare le dimensioni delle lettere dell’elemento testuale. \\
	\textbf{Precondizione}: l'elemento testuale è stato selezionato e si trova ora in modalità modifica.\\
	\textbf{Postcondizione}: il testo ha cambiato le dimensioni del carattere.\\
	\textbf{Scenario principale}:
	\begin{enumerate}
		\item L'utente modifica le dimensioni del testo nell'elemento testuale.
	\end{enumerate}	
	}
\subsubsection{UC 1.17.4.1.2.3 - Modifica della formattazione}{
	\label{uc1.17.4.1.2.3}
	\textbf{Attori}: utente desktop. \\
	\textbf{Descrizione}: l'utente è in grado di cambiare la formattazione dell’elemento testuale tra quelli disponibili (corsivo, grassetto, sottolineato) oppure tornare al testo non formattato. \\
	\textbf{Precondizione}: l'elemento testuale è stato selezionato e si trova ora in modalità modifica.\\
	\textbf{Postcondizione}: il testo ha cambiato formato.\\
	\textbf{Scenario principale}:
	\begin{enumerate}
		\item L'utente modifica la formattazione dell'elemento testuale.
	\end{enumerate}			
	}
\subsubsection{UC 1.17.4.1.2.4 - Modifica del colore}{
	\label{uc1.17.4.1.2.4}
	\textbf{Attori}: utente desktop. \\
	\textbf{Descrizione}: l'utente è in grado di cambiare il colore del testo contenuto nell’elemento testuale. \\
	\textbf{Precondizione}: l'elemento testuale è stato selezionato e si trova ora in modalità modifica.\\
	\textbf{Postcondizione}: è stato cambiato il colore dei caratteri dell'elemento testuale.\\
	\textbf{Scenario principale}:
	\begin{enumerate}
		\item L'utente modifica il colore del testo nell'elemento testuale.
	\end{enumerate}			
	}
\subsubsection{UC 1.17.4.1.2.5 - Modifica sfondo}{
	\label{uc1.17.4.1.2.5}
	\textbf{Attori}: utente desktop. \\
	\textbf{Descrizione}: l'utente è in grado di cambiare il colore dello sfondo dell’elemento testuale (evidenziatura). \\
	\textbf{Precondizione}: l'elemento testuale è stato selezionato e si trova ora in modalità modifica.\\
	\textbf{Postcondizione}: le lettere del testo sono state evidenziate.\\
	\textbf{Scenario principale}:
	\begin{enumerate}
		\item L'utente evidenzia il testo nell’elemento testuale.
	\end{enumerate}			
	}
\subsubsection{UC 1.17.4.1.3 - Spostamento di un elemento}{
	\label{uc1.17.4.1.3}
	\textbf{Attori}: utente desktop. \\
	\textbf{Descrizione}: l'utente è in grado di cambiare la posizione di un elemento. \\
	\textbf{Precondizione}: l'elemento è stato selezionato.\\
	\textbf{Postcondizione}: l'elemento è stato spostato all’interno dell’infografica.\\
	\textbf{Scenario principale}:
	\begin{enumerate}
		\item L'utente modifica la posizione dell'elemento.
	\end{enumerate}			
	}
\subsubsection{UC 1.17.4.2 - Rimozione dello sfondo}{
	\label{uc1.17.4.2}
	\textbf{Attori}: utente desktop. \\
	\textbf{Descrizione}: l'utente è in grado di rimuovere lo sfondo dell’infografica. \\
	\textbf{Precondizione}: il sistema è acceso e funzionante ed è stata caricata un'infografica.	\\
	\textbf{Postcondizione}: lo sfondo dell'infografica è stato eliminato.\\
	\textbf{Scenario principale}:
	\begin{enumerate}
		\item L'utente rimuove lo sfondo dell'infografica.
	\end{enumerate}			
	}
\subsubsection{UC 1.17.4.3 - Inserimento di uno sfondo}{
	\label{uc1.17.4.3}
	\textbf{Attori}: utente desktop. \\
	\textbf{Descrizione}: l'utente è in grado di scegliere uno sfondo per l'infografica. \\
	\textbf{Precondizione}: il sistema è acceso e funzionante ed è stata caricata un'infografica.	\\
	\textbf{Postcondizione}: lo sfondo dell'infografica è stato cambiato.\\
	\textbf{Scenario principale}:
	\begin{enumerate}
		\item L'utente inserisce uno sfondo nell'infografica.
	\end{enumerate}			
	}
\subsubsection{UC 1.17.4.4 - Inserimento di un elemento grafico}{
	\label{uc1.17.4.4}
	\textbf{Attori}: utente desktop. \\
	\textbf{Descrizione}: l'utente è in grado di inserire un elemento grafico. \\
	\textbf{Precondizione}: il sistema è acceso e funzionante ed è stata caricata un'infografica.	\\
	\textbf{Postcondizione}: è stato inserito un nuovo elemento nell'infografica.\\
	\textbf{Scenario principale}:
	\begin{enumerate}
		\item Quando un elemento grafico viene inserito si entra automaticamente in modalità modifica \S\hyperref[uc1.17.4.1]{(UC 1.17.4.1)}.
	\end{enumerate}
	}
\subsubsection{UC 1.17.4.5 - Inserimento di un elemento testuale}{
	\label{uc1.17.4.5}
	\textbf{Attori}: utente desktop. \\
	\textbf{Descrizione}: l'utente è in grado di inserire un elemento testuale nell'infografica. \\
	\textbf{Precondizione}: il programma è acceso e funzionante ed è stata caricata un'infografica.	\\
	\textbf{Postcondizione}: è stato inserito un elemento testuale precedentemente assente nell'infografica.	\\
	\textbf{Scenario principale}:
	\begin{enumerate}
		\item Quando viene inserito un elemento testuale questo entra automaticamente in modalità modifica \S\hyperref[uc1.17.4.1.2]{(UC 1.17.4.1.2)}.
	\end{enumerate}
	}
\subsubsection{UC 1.17.4.6 - Inserimento di un frame}{
	\label{uc1.17.4.6}
	\textbf{Attori}: utente desktop. \\
	\textbf{Descrizione}: l'utente è in grado di inserire un frame come immagine vettoriale nell'infografica. \\
	\textbf{Precondizione}: il programma è acceso e funzionante, è stata caricata un'infografica ed una presentazione di cui l'utente vuole fare l'infografica.	\\
	\textbf{Postcondizione}: nell'infografica è stato inserito un nuovo frame.\\
	\textbf{Scenario principale}:
	\begin{enumerate}
		\item L'utente inserisce un nuovo frame nell'infografica.
	\end{enumerate}			
	}
\subsubsection{UC 1.17.4.7 - Eliminazione di un elemento}{
	\label{uc1.17.4.7}
	\textbf{Attori}: utente desktop. \\
	\textbf{Descrizione}: l'utente è in grado di eliminare un elemento dell'infografica. \\
	\textbf{Precondizione}: il sistema è acceso e funzionante, è stata caricata un'infografica ed è stato selezionato un elemento dell'infografica.	\\
	\textbf{Postcondizione}: è stato rimosso un elemento dall'infografica.\\
	\textbf{Scenario principale}:
	\begin{enumerate}
		\item L'utente elimina un elemento dall'infografica.
	\end{enumerate}			
	}
\subsubsection{UC 1.17.5 - Salvataggio infografica}{
	\label{uc1.17.5}
	\textbf{Attori}: utente desktop. \\
	\textbf{Descrizione}: l'utente può salvare l'infografica su cui stava lavorando. \\
	\textbf{Precondizione}: il sistema è acceso e funzionante e l'utente ha caricato un'infografica.	\\
	\textbf{Postcondizione}: l'utente ha salvato l’infografica su cui stava lavorando nel proprio spazio personale.\\
	\textbf{Scenario principale}:
	\begin{enumerate}
		\item L'utente salva l'infografica nel proprio spazio personale.
	\end{enumerate}			
	}
\subsubsection{UC 1.17.7 - Esportazione infografica}{
	\label{uc1.17.7}
	\textbf{Attori}: utente desktop. \\
	\textbf{Descrizione}: l'utente può esportare un'infografica come file immagine nel proprio dispositivo. \\
	\textbf{Precondizione}: il sistema è acceso e funzionante e l'utente ha caricato un'infografica.	\\
	\textbf{Postcondizione}: l'utente ha esportato l'infografica sotto forma di file immagine nel proprio dispositivo.\\
	\textbf{Scenario principale}:
	\begin{enumerate}
		\item L'utente esporta l'infografica come immagine nel proprio dispositivo.
	\end{enumerate}			
	}