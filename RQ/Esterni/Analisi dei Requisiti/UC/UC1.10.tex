\subsection{UC 1.10 - Annulla/Ripristina presentazione desktop}{
	\label{uc1.11}
	\begin{figure}[H]
		\centering
		\includegraphics[scale=0.70]{\imgs {UC1.10}.jpg} %inserire il diagramma UML
		\label{fig:uc1.11}
		\caption{Caso d'uso 1.11: Annulla/Ripristina presentazione desktop}
	\end{figure}
	\textbf{Attori}: Utente desktop. \\
	\textbf{Descrizione}:L'utente pu� annullare un comando selezionato o ripristinare un comando annullato
	\textbf{Precondizione}:  l'utente ha effettuato l'accesso al sistema in modalit� modifica ad una presentazione in modalit� desktop \\
	\textbf{Postcondizione}: l'utente ha annullato un comando o ripristinato un comando annullato  \\
	
\subsubsection{UC 1.10.1 - Annulla modifica}{
		\label{uc1.10.1}
		\textbf{Attori}: Utente Desktop. \\
		\textbf{Descrizione}:L'utente pu� annullare l'ultimo comando di modifica tra:. \\
		\begin{itemize}
			\item inserimento frame
			\item eliminazione frame
			\item spostamento frame
			\item modifica frame
			\item inserimento svg
			\item modifica svg
			\item eliminazione elemento
			\item inserimento bookmark
			\item cancellazione bookmark
			\item modifica nella definizione di percorso
		\end{itemize}
		\textbf{Precondizione}: Il sistema ha registrato almeno un comando di modifica della presentazione da parte dell'utente. \\
		\textbf{Postcondizione}: Il sistema ha ripristinato lo stato precedente all'ultimo comando di modifica della presentazione dell'utente, e ha memorizzato nello storico degli annullamenti il comando annullato.	\\
		\textbf{Scenario principale}:
		\begin{enumerate}
			\item l'utente seleziona l'azione annulla 
		\end{enumerate}
		}
		
\subsubsection{UC 1.10.2 - Ripristina modifica}{
		\label{uc1.10.2}
		\textbf{Attori}: Utente Desktop. \\
		\textbf{Descrizione}: L'utente pu� ripristinare l'ultimo comando presente nello storico dei comandi annullati.
		\textbf{Precondizione}: Lo storico degli annullamenti di comando non � vuoto. \\
		\textbf{Postcondizione}: Il sistema ha rieseguito l'ultimo comando di modifica della presentazione presente nello storico dei comandi annullati, ed � stato rimosso il comando rieseguito dallo storico.
		}


