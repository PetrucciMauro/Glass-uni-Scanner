\subsection{UC 0.9 - Gestione template}{
	\label{uc0.9}
	\begin{figure}[H]
		\centering
		\includegraphics[scale=0.6]{\imgs {UC0.9}.jpg} %inserire il diagramma UML
		\label{fig:uc0.9}
		\caption{Caso d'uso 0.9: Gestione template}
	\end{figure}
	\textbf{Attori}: amministratore. \\
	\textbf{Descrizione}: l'amministratore ha effettuato il login e il sistema è correttamente funzionante. L'amministratore ha la possibilità di inserire nuovi template di infografica o di presentazione, nuovi oggetti grafici per comporre i template ed eliminare template presenti nel server.\\
	\textbf{Precondizione}: il sistema è in funzione e l'amministratore ha effettuato il login.\\
	\textbf{Postcondizione}: le operazioni svolte dall'amministratore sono state portate a termine con successo.\\
	\textbf{Scenario principale}:
	\begin{enumerate}
		\item Caricamento di un template di infografica o di presentazione \S\hyperref[uc0.9.1]{(UC0.9.1)};
		\item Caricamento di un elemento \S\hyperref[uc0.9.3]{(UC0.9.3)};
		\item Eliminazione di un template \S\hyperref[uc0.9.5]{(UC0.9.5)};
	\end{enumerate}
	\textbf{Scenari alternativi}:
	\begin{itemize}
		\item L'amministratore può ripristinare un template eliminato \S\hyperref[uc0.9.5]{(UC0.9.5)};
		\item L'amministratore può ripristinare le modifiche eseguite;
		\item Visualizzazione di un errore nel caso vengano inseriti un oggetto o un template già presenti nel database.
	\end{itemize}
	}
\subsubsection{UC0.9.1 - Caricamento di un template}{
	\label{uc0.9.1}
	\textbf{Attori}: amministratore. \\
	\textbf{Descrizione}: l'amministratore esegue il caricamento di un nuovo template nel database. \\
	\textbf{Precondizione}: il sistema  è funzionante e l'amministratore ha effettuato il login.	\\
	\textbf{Postcondizione}: il template è stato caricato correttamente nel database.	\\
	\textbf{Scenario principale}:
	\begin{enumerate}
		\item L'amministratore naviga nel proprio spazio di lavoro  per trovare il template da inserire;
		\item L'amministratore inserisce il template nel database.
	\end{enumerate}
	\textbf{Scenari alternativi}:
	\begin{itemize}
		\item Il template è già presente nel database e l'amministratore viene informato dell'errore \S\hyperref[uc0.9.2]{(UC 0.9.2)}.
	\end{itemize}
	}
\subsubsection{UC 0.9.3 - Caricamento di un elemento}{
	\label{uc0.9.3}
	\textbf{Attori}: amministratore. \\
	\textbf{Descrizione}: l'amministratore esegue il caricamento di un nuovo elemento nel database. \\
	\textbf{Precondizione}: il sistema  è funzionante e l'amministratore ha effettuato il login.	\\
	\textbf{Postcondizione}: l'elemento grafico è stato caricato correttamente nel database.	\\
	\textbf{Scenario principale}:
	\begin{enumerate}
		\item L'amministratore naviga nel proprio spazio di lavoro per trovare l'elemento grafico da inserire;
		\item L'amministratore inserisce l'elemento grafico nel database.
	\end{enumerate}
	\textbf{Scenari alternativi}:
	\begin{itemize}
		\item Il file grafico è già presente nel database e l'amministratore viene informato dell'errore \S\hyperref[uc0.9.2]{(UC 0.9.2)}.
	\end{itemize}
	}
\subsubsection{UC0.9.2 - Visualizza errore}{
	\label{uc0.9.2}
	\textbf{Attori}: amministratore. \\
	\textbf{Descrizione}: il sistema ha riscontrato un errore e deve renderlo noto all'amministratore. \\
	\textbf{Precondizione}: un template o un elemento è in fase di caricamento.	\\
	\textbf{Postcondizione}: l'amministratore ha visualizzato l'errore.	\\
	\textbf{Scenario principale}:
	\begin{enumerate}
		\item Viene visualizzato l'errore riscontrato dal sistema.
	\end{enumerate}
	}
\subsubsection{UC 0.9.4- Eliminazione di un template}{
	\label{uc0.9.4}
	\textbf{Attori}: amministratore. \\
	\textbf{Descrizione}: l'amministratore elimina uno dei template. \\
	\textbf{Precondizione}: il sistema è funzionante e l'amministratore decide di eliminare un template.	\\
	\textbf{Postcondizione}: il template è stato eliminato dal sistema.	\\
	\textbf{Scenario principale}:
	\begin{enumerate}
		\item L’amministratore elimina il template selezionato.
	\end{enumerate}
	\textbf{Scenari alternativi}:
	\begin{itemize}
		\item L'amministratore decide di ripristinare il template eliminato \S\hyperref[uc0.9.5]{(UC 0.9.5)}.
	\end{itemize}
	}
\subsubsection{UC 0.9.5 - Ripristino template eliminato}{
	\label{uc0.9.5}
	\textbf{Attori}: amministratore. \\
	\textbf{Descrizione}: l'amministratore ha deciso di ripristinare un template eliminato. \\
	\textbf{Precondizione}: lo storico dei template eliminati non è vuoto.	\\
	\textbf{Postcondizione}: il template è stato ripristinato.	\\
	\textbf{Scenario principale}:
	\begin{enumerate}
		\item Selezione template eliminato; 
		\item Ripristino del template eliminato.
	\end{enumerate}
	}