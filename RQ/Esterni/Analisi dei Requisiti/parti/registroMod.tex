\Large{\textbf{Registro delle modifiche}}\\
\normalsize

%	Ordine di inserimento: dall'ultima versione alla prima
\renewcommand*{\arraystretch}{1.4}
\begin{longtable} [c]{|>{\centering\arraybackslash}m{2cm} | >{\centering\arraybackslash}m{4cm} | >{\centering\arraybackslash}m{3cm} | >{\centering\arraybackslash}m{6cm} |}
		\caption{Versionamento del documento \label{tab:versionamento}}\\
		 \hline
		 \textbf{Versione} & \textbf{Autore} & \textbf{Data} & \textbf{Descrizione}\\
		 \hline
		 \endfirsthead
		 \hline
		 \textbf{Versione} & \textbf{Autore} & \textbf{Data} & \textbf{Descrizione}\\
		 \hline
		\endhead
		 \hline
		 \endfoot
		 \hline
		 \endlastfoot
		 3.0.0 & \FM & 19-08-2015 & Approvazione del documento\\
		 \hline 
		 2.1.0 & \PM & 18-07-2015 & Apportate le correzioni sul documento a seguito delle segnalazioni del committente\\
		 \hline
		 2.0.0 & \VG & 25-05-2015 & Approvazione del documento\\
  		 \hline	
		 1.8.0 & \BM & 24-05-2015 & Apportate piccole modifiche\\
 		 \hline	
		 1.5.0 & \BM & 20-05-2015 & Apportate verifiche segnalate dal verificatore \GP\\
		 \hline			 
		 1.2.5 & \BM & 09-05-2015 & Correzioni ortografiche e grammaticali\\
		 \hline	
		 1.2.0 & \FM, \BM & 08-05-2015 & Applicazione incrementi del committente\\
		 \hline			 
		 1.0.0 & \BM & 13-04-2015 & Approvazione del documento\\
		 \hline				 
		 0.7.0 & \BM, \GP, \VG, \PM & 12 aprile 2015 & Apportate le modifiche segnalate dai verificatori \PM {} e \BM\\
		 \hline			 
		 0.6.0 & \BM & 10-04-2015 & Aggiunta dei casi d'uso mancanti\\
		 \hline		 
		 0.5.0 & \BM & 6-04-2015 & Ordinamento del file e formattazione del testo\\
		 \hline
		 0.4.0 & \VG & 3-04-2015 & Aggiornamento dei casi d'uso 1.6 e 1.17\\
		 \hline
		 0.3.0 & \FM & 1-04-2015 & Aggiornamento del caso d'uso 1.3\\
		 \hline		 
		 0.2.0 & \BM & 24-03-2015 & Aggiunta dei contenuti. Inserimento di tutti i casi d'uso\\		 
		 \hline
		 0.1.0 & \BM & 20-03-2015 & Stesura dello scheletro del documento\\		 
\end{longtable}

\newpage
\Large{\textbf{Storico }}\\
\normalsize \\

%	Per mettere più tabelle di storico basta copiare e incollare la seguente porzione di codice e modificarla in base ai dati nuovi
\noindent \textbf{pre-RR}
\label{tabVers1}
\begin{table}[h]
	\begin{tabular}{p{0.2\textwidth} p{0.7\textwidth}}
		\toprule \textbf{Versione 1.0.0}	&	\textbf{Nominativo}\\
		\midrule Redazione	& Tutti i componenti del gruppo\\
		\midrule Verifica & \PM, \BM\\
		\midrule Approvazione	& \TP\\
		\bottomrule
	\end{tabular}
	\caption{Storico ruoli pre-RR}
\end{table}
\\
\normalsize \\
\noindent \textbf{RR -> RP}
\label{tabVers2}
\begin{table}[h]
	\begin{tabular}{p{0.2\textwidth} p{0.7\textwidth}}
		\toprule \textbf{Versione 2.0.0}	&	\textbf{Nominativo}\\
		\midrule Redazione	& \FM, \BM\\
		\midrule Verifica & \GP\\
		\midrule Approvazione	& \VG\\
		\bottomrule
	\end{tabular}
	\caption{Storico ruoli RR -> RP}
\end{table}

\noindent \textbf{RP -> RQ}
\label{tabVers3}
\begin{table}[h]
	\begin{tabular}{p{0.2\textwidth} p{0.7\textwidth}}
		\toprule \textbf{Versione 3.0.0}	&	\textbf{Nominativo}\\
		\midrule Redazione	& \PM, \GP\\
		\midrule Verifica & \TP\\
		\midrule Approvazione	& \FM\\
		\bottomrule
	\end{tabular}
	\caption{Storico ruoli RP -> RQ}
\end{table}