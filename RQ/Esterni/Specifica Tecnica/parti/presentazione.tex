\section{Descrizione architetturale}{
\subsection{Metodo e formalismi}{
	Si progetterà l'architettura del sistema secondo un approccio top-down, ovvero iniziando da una visione più astratta sul sistema ed aumentando di concretezza nelle iterazioni successive.
	Si passerà quindi alla definizione dei package e successivamente dei componenti di questi. Infine si andranno a definire le singole classi e interfacce specificando per ognuna:
	\begin{itemize}
	\item Tipo;
	\item Funzione;
	\item Classi o interfacce estese;
	\item Interfacce implementate;
	\item Relazioni con altre classi.
	\end{itemize}
	
	Verranno quindi illustrati i Design Pattern usati nella progettazione architetturale del sistema rimandano la spiegazione all'appendice (A1).\\
	Per i diagrammi di Package, classi e attività verrà usata la notazione UML 2.0.
}
\subsection{Architettura generale}{
	Il prodotto si presenta suddiviso in tre parti distinte: Model, View e Controller. Si è quindi cercato di implementare il design pattern architetturale MVC in modo da garantire un basso livello di accoppiamento. In figura 1 viene riportato il diagramma dei package, in seguito vengono elencate le componenti dell'applicativo con le relative caratteristiche e funzionalità generali, per una trattazione più approfondita si rimanda alle sezioni specifiche dei componenti.
	
	\begin{figure}
		\centering
		\includegraphics[scale=0.23]{\imgs {General}.jpg} %inserire il diagramma UML
		\label{fig:architettura}
		\caption{Architettura generale del sistema}
	\end{figure}
	\subsubsection{Model}{
		Contiene la rappresentazione dei dati, l'implementazione dei metodi da applicare ad essi e lo stato di questi ultimi; costituisce il cuore del software e risulta di fatto totalmente indipendente dagli altri due strati.
		}
	\subsubsection{View}{	
		Contiene tutti gli elementi della GUI, comprese le interfacce di comunicazione con le librerie grafiche esterne. Si limita a passare gli input inviati dall'utente allo strato che sta sotto di lei, il Controller, demandandone a quest'ultimo la	gestione.
		}
	\subsubsection{Controller}{
		E' il punto di incontro tra la View e il Model: i dati ricevuti da quest’ultimo sono elaborati per essere presentati alla View.
		}
	}
	
	
	
	
	
	\subsection{Servizi Api nodeAPI}{

		Il seguente diagramma delle classi \`{e} stato esteso con le primitive:
	\begin{itemize}
	\item \textbf{<<Resource>>} : rappresenta una risorsa associata ad un certo url a cui sono disponibili dei servizi
	\item \textbf{<<Node>>} : rappresenta una parte di URL a cui non sono disponibili servizi ma \`{e} utile per suddividere quest'ultimi
	\item \textbf{<<Server>>} : rappresenta la radice dei servizi offerti dal server
	\item \textbf{<<Path>>} : indica una aggiunta in coda all' URL attuale per raggiungere una nuova risorsa o nodo
	\item \textbf{<<Middleware>>} : indica un middleware, un insieme di funzionalit\`{a} chiamate ogni qualvolta si accede a risorse attraversando questo elemento
	\end{itemize}
	
			\begin{figure}[H]
			\centering
			\includegraphics[scale=0.6]{\imgs {nodeAPI}.pdf}
			\label{fig:nodeAPI}
			\caption{Servizio Api nodeApI}
		\end{figure}
		
		\begin{itemize}
			
		\item \textbf{NodeServer:} radice dei servizi offerti dal server:
			\begin{enumerate}
				\item server per pagine HTML e file statici associati
				\item servizi di autenticazione stateless
				\item servizi di upload e reperimento file statici multimediali per utente
				\item servizi di interazione con MongoDB per salvataggio persistente delle presentazioni
			\end{enumerate}	
				
		\item \textbf{Register:}
			\begin{itemize}
			\item  \textbf{POST} /account/register 
				\begin{itemize} 
				\item \textbf{descrizione:} inserisce nuovo utente in MongoDB, crea una nuova collezione 'presentations'+username, crea le cartelle per i file utente
				\end{itemize}
			\end{itemize}
			
		\item \textbf{Authenticate:}
			\begin{itemize}
			\item  \textbf{GET} /account/authenticate 
				\begin{itemize} 
				\item \textbf{descrizione:} verifica se username e password sono corretti e ritorna un token per l'accesso ai servizi protetti
				\end{itemize}
			\end{itemize}
				
		\item \textbf{ChangePassword:}
			\begin{itemize}
			\item  \textbf{POST} /account/changepassword 
				\begin{itemize} 
				\item \textbf{descrizione:} verifica la correttezza di username e password e modifica questa'ultima con la nuova 
				\end{itemize}
			\end{itemize}
			
		\item \textbf{PublicPages:}
			\begin{itemize}
			\item  \textbf{GET} /publicpages/[file] 
				\begin{itemize} 
				\item \textbf{descrizione:} se presente [file] nella cartella /public\_html del server ritorna il file stesso 
				\end{itemize}
			\end{itemize}
			
		\item \textbf{tokenMiddleware:} verifica che il token passato nel campo Authorization dell' Header sia valido, dal token ricava lo username dell'utente
		
		\item \textbf{PrivatePages:}
			\begin{itemize}
			\item  \textbf{GET} /private/htdocs/[file] 
				\begin{itemize} 
				\item \textbf{descrizione:} se presente [file] nella cartella /private\_html del server ritorna il file stesso
				\end{itemize}
			\end{itemize}
			
		\item \textbf{PresentationMeta:}
			\begin{itemize}
			\item  \textbf{GET} /private/api/presentations 
				\begin{itemize} 
				\item \textbf{descrizione:} cerca in MongoDB nella collezione associata alle presentazioni dell'utente, ritorna un array i cui elementi sono i campi meta delle presentazioni dell'utente
				\end{itemize}
			\end{itemize}
			
		\item \textbf{NewPresentation:}
			\begin{itemize}
			\item  \textbf{POST} /private/api/presentations/new/[presentationName] 
				\begin{itemize} 
				\item \textbf{descrizione:} se non esiste gi\`{a} crea una nuova presentazione con il nome [presentatioNname]
				\end{itemize}
			\end{itemize}
			
		\item \textbf{NewCopyPresentation:}
			\begin{itemize}
			\item  \textbf{POST} /private/api/presentations/new/[newPresentationName]/[oldPresentationName]
				\begin{itemize} 
				\item \textbf{descrizione:} crea una nuova presentazione con nome [newPresentationName] dalla presentazione con titolo [oldPresentationName]
				\end{itemize}
			\end{itemize}

			
		\item \textbf{Presentation:}
			\begin{itemize}
			\item  \textbf{GET} /private/api/presentations/[presentationName]
				\begin{itemize} 
				\item \textbf{descrizione:} recupera se presente la presentazione dell'utente associata al titolo passato nell'URL
				\end{itemize}
				
			\item  \textbf{DELETE} /private/api/presentations/[presentationName]
				\begin{itemize} 
				\item \textbf{descrizione:} elimina se presente la presentazione dell'utente associata al titolo passato nell'URL
				\end{itemize}
			\end{itemize}

		\item \textbf{RenamePresentation:}
			\begin{itemize}
			\item  \textbf{POST} /private/api/presentations/[presentationName]/rename/[newname] 
				\begin{itemize} 
				\item \textbf{descrizione:} rinomina se presente la presentazione dell'utente associata al titolo passato nell'URL con il nome [newname]
				\end{itemize}
			\end{itemize}
			
		\item \textbf{PresentationElement:}
			\begin{itemize}
			\item   \textbf{POST} /private/api/presentations/[presentationName]/element
				\begin{itemize} 
				\item \textbf{descrizione:} inserisce nella presentazione dell'utente individuata da [presentationName] l'oggetto element passato nel body della richiesta
				\end{itemize}
				
			\item  \textbf{PUT} /private/api/presentations/[presentationName]/element
				\begin{itemize} 
				\item \textbf{descrizione:} sostituisce nella presentazione dell'utente l'elemento passato nel body della richiesta
				\end{itemize}
			\end{itemize}
				
		\item \textbf{DeleteElement:}
			\begin{itemize}
			\item   \textbf{DELETE} /private/api/presentations/[presentationName]/delete/[type/[id\_element]]
				\begin{itemize} 
				\item \textbf{descrizione:} elimina dalla presentazione con il titolo [presentationName] l'elemento con identificativo [idElement]						
				\end{itemize}
			\end{itemize}
			
	%%%%%%%%%%%%%%%%%%%%%%%%%%%%%%%%
	%  FILE_SERVER
	%%%%%%%%%%%%%%%%%%%%%%%%%%%%%%%%
			
		\item \textbf{GetImage:}
			\begin{itemize}
			\item    \textbf{GET} /files/[user]/image/[imagename]
				\begin{itemize} 
				\item \textbf{descrizione:} ritorna il file [imagename] nella cartella /users/[username]/image				\end{itemize}
			\end{itemize}
			
		\item \textbf{GetAudio:}
			\begin{itemize}
			\item    \textbf{GET} /files/[user]/audio/[audioname]
				\begin{itemize} 
				\item \textbf{descrizione:} ritorna il file [audioname] nella cartella /users/[username]/audios				\end{itemize}		
			\end{itemize}
			
		\item \textbf{GetVideo:}
			\begin{itemize}
			\item    \textbf{GET} /files/[user]/video/[videoname]
				\begin{itemize} 
				\item \textbf{descrizione:} ritorna il file [videoname] nella cartella /users/[username]/videos					
				\end{itemize}
			\end{itemize}
	
			
			
		\item \textbf{ImagesMeta:}
			\begin{itemize}
			\item   \textbf{GET} /private/api/files/image 
				\begin{itemize} 
				\item \textbf{descrizione:} ritorna un array con i nomi dei file immagine dell'utente
				\end{itemize}
			\end{itemize}
			
		\item \textbf{Image:}
			\begin{itemize}
			\item    \textbf{POST} /private/api/files/image/[imagename]
				\begin{itemize} 
				\item \textbf{descrizione:} caricare da locale un nuovo file immagine nella cartella /users/[username]/images					
				\end{itemize}
			\item   \textbf{DELETE} /private/api/files/image/[imagename]
				\begin{itemize} 
				\item \textbf{descrizione:} elimina il file immagine [imagename] dalla cartella /users/[username]/images			
				\end{itemize}
			\end{itemize}
			
		\item \textbf{RenameImage:}
			\begin{itemize}
			\item   \textbf{POST} /private/api/files/image/[imagename]/[newname] 
				\begin{itemize} 
				\item \textbf{descrizione:} rinomina il file immagine [imagename] con [newname] nella cartella /users/[username]/images
				\end{itemize}
			\end{itemize}
			
		\item \textbf{AudiosMeta:}
			\begin{itemize}
			\item   \textbf{GET} /private/api/files/audio
				\begin{itemize} 
				\item \textbf{descrizione:} ritorna un array con i nomi dei file audio dell'utente
				\end{itemize}
			\end{itemize}
			
		\item \textbf{Audio:}
			\begin{itemize}
			\item    \textbf{POST} /private/api/files/audio/[audioname]
				\begin{itemize} 
				\item \textbf{descrizione:} caricare da locale un nuovo file immagine nella cartella /users/[username]/audios					

				\end{itemize}
			\item   \textbf{DELETE} /private/api/files/audio/[audioname]
				\begin{itemize} 
				\item \textbf{descrizione:} elimina il file audio [audioname] dalla cartella /users/[username]/audios			

				\end{itemize}
			\end{itemize}
			
		\item \textbf{RenameAudio:}
			\begin{itemize}
			\item   \textbf{POST} /private/api/files/audio/[audioname]/[newname] 
				\begin{itemize} 
				\item \textbf{descrizione:} rinomina il file audio [audioname] con [newname] nella cartella /users/[username]/audios

				\end{itemize}
			\end{itemize}
						
		\item \textbf{VideosMeta:}
			\begin{itemize}
			\item   \textbf{GET} /private/api/files/video
				\begin{itemize} 
				\item \textbf{descrizione:} ritorna un array con i nomi dei file video dell'utente
	
				\end{itemize}
			\end{itemize}
			
		\item \textbf{Video:}
			\begin{itemize}
			
			\item    \textbf{POST} /private/api/files/video/[videoname]
				\begin{itemize} 
				\item \textbf{descrizione:} caricare da locale un nuovo file immagine nella cartella /users/[username]/videos					

				\end{itemize}
			\item   \textbf{DELETE} /private/api/files/video/[videoname]
				\begin{itemize} 
				\item \textbf{descrizione:} elimina il file video [videoname] dalla cartella /users/[username]/videos			

				\end{itemize}
			\end{itemize}
			
		\item \textbf{RenameVideo:}
			\begin{itemize}
			\item   \textbf{POST} /private/api/files/video/[videoname]/[newname] 
				\begin{itemize} 
				\item \textbf{descrizione:} rinomina il file video [videoname] con [newname] nella cartella /users/[username]/videos

				\end{itemize}
			\end{itemize}

			
	\end{itemize}
		
	}
	
}