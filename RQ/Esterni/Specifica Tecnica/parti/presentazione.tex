\section{Descrizione architetturale}{
\subsection{Metodo e formalismi}{
	Si progetterà l'architettura del sistema secondo un approccio top-down, ovvero iniziando da una visione più astratta sul sistema ed aumentando di concretezza nelle iterazioni successive.
	Si passerà quindi alla definizione dei package e successivamente dei componenti di questi. Infine si andranno a definire le singole classi e interfacce specificando per ognuna:
	\begin{itemize}
	\item Tipo;
	\item Funzione;
	\item Classi o interfacce estese;
	\item Interfacce implementate;
	\item Relazioni con altre classi.
	\end{itemize}
	
	Verranno quindi illustrati i Design Pattern usati nella progettazione architetturale del sistema rimandano la spiegazione all'appendice (A1).\\
	Per i diagrammi di Package, classi e attività verrà usata la notazione UML 2.0.
}
\subsection{Architettura generale}{
	Il prodotto si presenta suddiviso in tre parti distinte: Model, View e Controller. Si è quindi cercato di implementare il design pattern architetturale MVC in modo da garantire un basso livello di accoppiamento. In figura 1 viene riportato il diagramma dei package, in seguito vengono elencate le componenti dell'applicativo con le relative caratteristiche e funzionalità generali, per una trattazione più approfondita si rimanda alle sezioni specifiche dei componenti.
	
	\begin{figure}
		\centering
		\includegraphics[scale=0.23]{\imgs {General}.jpg} %inserire il diagramma UML
		\label{fig:architettura}
		\caption{Architettura generale del sistema}
	\end{figure}
	\subsubsection{Model}{
		Contiene la rappresentazione dei dati, l'implementazione dei metodi da applicare ad essi e lo stato di questi ultimi; costituisce il cuore del software e risulta di fatto totalmente indipendente dagli altri due strati.
		}
	\subsubsection{View}{	
		Contiene tutti gli elementi della GUI, comprese le interfacce di comunicazione con le librerie grafiche esterne. Si limita a passare gli input inviati dall'utente allo strato che sta sotto di lei, il Controller, demandandone a quest'ultimo la	gestione.
		}
	\subsubsection{Controller}{
		E' il punto di incontro tra la View e il Model: i dati ricevuti da quest’ultimo sono elaborati per essere presentati alla View.
		}
	}
	
	\subsection{Servizi Api nodeAPI}{
		\begin{figure}[H]
			\centering
			\includegraphics[scale=0.6]{\imgs {nodeAPI}.pdf}
			\label{fig:nodeAPI}
			\caption{Servizio Api nodeApI}
		\end{figure}
		\begin{itemize}
			
			\item \textbf{NodeApiServer:} radice dei servizi offerti da nodeApi, ovvero servizi di autenticazione e di interazione con il database MongoDB per salvare in modo persistente le presentazioni degli utenti in remoto
			
			\item \textbf{User:} servizi disponibili al path /users, offre funzionalità di registrazione e autenticazione di un utente attraverso token scambiati dal client al server ad ogni richiesta di servizio
			
			\item \textbf{apiMiddleware:} al path /api è presente un middleware per proteggere i servizi Presentations, Presentation ed Element da accessi di utenti non autenticati
			
			\item \textbf{Presentations:} al path /api/presentations è disponibile un servizio per ottenere meta-informazioni sulle presentazioni create dall'utente 
			
			\item \textbf{Presentation:} al path /api/presentations/presentation sono disponibili servizi per creare una nuova presentazione e per recuperare o eliminare una presentazione dell'utente
			
			\item \textbf{Element:} al path /api/presentations/presentation/element sono disponibili servizi per inserire, sostituire o eliminare un elemento in una presentazione dell'utente
		\end{itemize}
	}
	
}