\section{Design Pattern e Pattern Architetturali}{
	\subsection{MVC}{
			\begin{figure}[H]
				\centering
				\includegraphics[scale=0.6]{\imgs {MVC}.png}
				\label{fig:MVC}
				\caption{Model View Controller}
			\end{figure}
			\begin{itemize}
				\item \textbf{Scopo dell’utilizzo}: è stato scelto il pattern MVC per separare la logica dell'applicazione dalla rappresentazione grafica;
				\item \textbf{Contesto d’utilizzo}: il pattern MVC viene utilizzato per l'architettura generale dell'applicazione.
				Ogni modifica effettuata dall'utente sulla View viene inviata al Model tramite il Controller e viceversa.
			\end{itemize}
		}	
	\subsection{Command}{
			\begin{figure}[H]
				\centering
				\includegraphics[scale=0.38]{\imgs {CommandPackagePattern}.pdf}
				\label{fig:commandclassdiagram}
				\caption{Diagramma delle classi del package Command}
			\end{figure}
		\begin{figure}[H]
			\centering
			\includegraphics[scale=0.5]{\imgs {Command}.pdf}
			\label{fig:commandsequencediagram}
			\caption{diagramma di sequenza del Pattern Command}
		\end{figure}
		\begin{itemize}
			\item \textbf{Descrizione}: viene utilizzato quando c'è la necessità di disaccoppiare l’invocazione di un comando dai suoi dettagli implementativi, separando colui che invoca il comando da colui che esegue l’operazione.
			
			Tale operazione viene realizzata attraverso questa catena: Client->Invocatore->Ricevitore
			
			Il Client non è tenuto a conoscere i dettagli del comando ma il suo compito è solo quello di chiamare il metodo dell' Invocatore che si occuperà di intermediare l’operazione.
			L’Invocatore ha l’obiettivo di incapsulare, nascondere i dettagli della chiamata come nome del metodo e parametri.
			Il Ricevitore utilizza i parametri ricevuti per eseguire l’operazione
			\item \textbf{Scopo dell’utilizzo}: si è scelto di utilizzare il pattern Command perché poter accodare o mantenere uno storico delle operazioni e	gestire operazioni cancellabili;
			\item \textbf{Contesto d’utilizzo}: viene utilizzato in fase di modifica delle presentazioni.
		\end{itemize}
		\subsubsection{Premi::Model::SlideShow::SlideShowActions::Command}{
			Il package Premi::Model::SlideShow::SlideShowActions::Command implementa il pattern Command, tuttavia il client è esterno al package ed è individuabile nella classe\\ Premi::Controller::Presentazione::Edit, che invoca il costruttore delle sottoclassi di\\ Premi::Model::SlideShow::SlideShowActions::Command::AbstractCommand e dà l'oggetto creato in pasto a Premi::Model::SlideShow::SlideShowActions::Command::Invoker, che rappresenta, appunto, la componente invoker del pattern e che mette l'oggetto della sottoclasse di AbstractCommand in un contenitore denominato undo, invoca quindi il metodo Invoker::execute() che a sua volta esegue concretamente il comando.\\
			Premi::Controller::Presentazione::Edit può invocare il metodo unexecute() di Invoker che a sua volta invoca il metodo AbstractCommand::undoCommand() nell'ultimo oggetto inserito nel membro contenitore undo. Questo metodo esegue le operazioni necessarie per annullare tutte le modifiche apportate dal comando. Quindi Invoker toglie il comando dal contenitore undo e lo inserisce nel contenitore redo. Quando Premi::Controller::Presentazione::Edit invoca il metodo Invoker::execute(), l'oggetto Invoker esegue il comando e lo sposta nuovamente dal membro contenitore redo e lo mette nel membro undo.    
		}
	}
