\subsection{Presenter}{
	\textbf{\tipo}: fanno parte di questo livello i package che gestiscono i segnali e le chiamate effettuati dalla view.\\
	\textbf{\relaz}: comunica con il Model per rendere possibile la gestione del profilo e la gestione delle presentazioni da parte dell'utente.\\
	
	\subsubsection{Presenter::EditPresenter}{
		\textbf{\tipo}: Lo scopo di questa classe è di gestire i segnali e le chiamate delle pagine View::Pages::DesktopEdit e View::Pages::MobileEdit.\\	
		\textbf{\relaz}:
		\begin{itemize}
			\item Model::SlideShow::SlideShowActions::Command::ConcreteTextInsertCommand <- EditPresenter costruisce un comando e lo dà in pasto a Model:Invoker;
			\item ConcreteFrameInsertCommand <- EditPresenter costruisce un comando e lo dà in pasto a Model:Invoker;
			\item ConcreteImageInsertCommand <- EditPresenter costruisce un comando e lo dà in pasto a Model:Invoker;
			\item ConcreteSVGInsertCommand <- EditPresenter costruisce un comando e lo dà in pasto a Model:Invoker;
			\item ConcreteAudioInsertCommand <- EditPresenter costruisce un comando e lo dà in pasto a Model:Invoker;
			\item ConcreteVideoInsertCommand <- EditPresenter costruisce un comando e lo dà in pasto a Model:Invoker;
			\item ConcreteBackgroundInsertCommand <- EditPresenter costruisce un comando e lo dà in pasto a Model:Invoker;
			\item ConcreteTextRemoveCommand <- EditPresenter costruisce un comando e lo dà in pasto a Model:Invoker;
			\item ConcreteFrameRemoveCommand <- EditPresenter costruisce un comando e lo dà in pasto a Model:Invoker;
			\item ConcreteImageRemoveCommand <- EditPresenter costruisce un comando e lo dà in pasto a Model:Invoker;
			\item ConcreteSVGRemoveCommand <- EditPresenter costruisce un comando e lo dà in pasto a Model:Invoker;
			\item ConcreteAudioRemoveCommand <- EditPresenter costruisce un comando e lo dà in pasto a Model:Invoker;
			\item ConcreteVideoRemoveCommand <- EditPresenter costruisce un comando e lo dà in pasto a Model:Invoker;
			\item ConcreteBackgroundRemoveCommand <- EditPresenter costruisce un comando e lo dà in pasto a Model:Invoker;
			\item ConcreteEditSizeCommand <- EditPresenter costruisce un comando e lo dà in pasto a Model:Invoker;
			\item ConcreteEditPositionCommand <- EditPresenter costruisce un comando e lo dà in pasto a Model:Invoker;
			\item ConcreteEditRotationCommand <- EditPresenter costruisce un comando e lo dà in pasto a Model:Invoker;
			\item ConcreteEditColorCommand <- EditPresenter costruisce un comando e lo dà in pasto a Model:Invoker;
			\item ConcreteEditBackgroundCommand <- EditPresenter costruisce un comando e lo dà in pasto a Model:Invoker;
			\item ConcreteEditFontCommand <- EditPresenter costruisce un comando e lo dà in pasto a Model:Invoker;
			\item ConcreteEditContentCommand <- EditPresenter costruisce un comando e lo dà in pasto a Model:Invoker;
			\item Invoker <- EditPresenter costruisce l’oggetto di classe Invoker. Invoca il metodo execute() di Invoker, passando come parametro un oggetto di classe Command oppure invoca il metodo unexecute() di Invoker;
			\item Model::ApacheManager::FileManager <- EditPresenter invoca i metodi uploadFile() di FileManager quando viene inserito nella presentazione un file non ancora presente nel server;
			\item Model::ApacheRelations::ResourceGetter <- la classe della view invoca il metodo save() presente in  ExecutionPresenter che a sua volta invoca il metodo getResources() di ResourceGetter che aggiorna il file manifest con tutti gli elementi della presentazione e lo ricarica.
			\item View::Pages::DesktopEdit e View::Pages::MobileEdit -> invocano i metodi di EditPresenter, passando i parametri degli oggetti modificati;
			\item View::Pages::DesktopEdit e View::Pages::MobileEdit <- quando il logout ha successo EditPresenter comunica alla view di effettuare una redirect verso Index;
			\item Model::MongoRelations::Loader::Autenticazione <- Quando la view invia una richiesta di logout, EditPresenter invoca il metodo di Autenticazione deAuthenticate(), che termina la sessione.
			\item Model::MongoRelations::Loader::Loaderclass <- EditPresenter invoca i metodi forniti da LoaderClass che restituiscono l'oggetto JSON della presentazione. 
			\item View::ViewJavascript <- EditPresenter invoca le funzioni Javascript passando l'oggetto presentazione, destinate alla sua traduzione in formato HTML. Inoltre, EditPresenter invoca le funzioni ViewJavascript ogniqualvolta sia richiesta una modifica della View a seguito di una modifica del model, nello specifico nelle operazioni di annulla e ripristina, gestite dal package Model::SlideShow::SlideShowActions::Command.
			\item Model::SlideShow::SlideShowActions::Command -> le sottoclassi concrete di AbstractCommand invocano il metodo di update di EditPresenter, tale metodo invoca quindi le funzioni di ViewJavascript destinate alla modifica della pagina HTML.
		\end{itemize} 
		\textbf{\interfacce}: La pagina DesktopEdit o la pagina MobileEdit invia a EditPresenter comunica l’avvenuta modifica o la rimozione di un elemento della presentazione o l’inserimento di un nuovo elemento invocando i metodi corrispondenti di EditPresenter. EditPresenter istanzia un oggetto di una sottoclasse di Model::SlideShow::SlideShowActions::Command::AbstractCommand e lo dà in pasto a Model::Invoker. Eventualmente EditPresenter, dopo che la View ha invocato il metodo undo() di EditPresenter, può semplicemente annullare il comando appena eseguito invocando il metodo unexecute di Invoker.
		La pagina web può, inoltre richiedere il caricamento di una presentazione o la creazione di una nuova presentazione a EditPresenter, che, tramite invocazione dei metodi di Model::MongoRelations::Loader::LoaderClass, caricherà dal database.
		\\
	}
	\subsubsection{Presenter::HomePresenter}{
				\textbf{\tipo}: Lo scopo di questa classe è di gestire i segnali e le chiamate provenienti dalla pagina View::Pages::Home.\\	
				\textbf{\relaz}:
				\begin{itemize}
					\item View::Pages::Home -> costruisce HomePresenter, ne invoca i metodi passando i parametri dell’utente;
					\item Model::MongoRelations::Loader::LoaderClass <- HomePresenter invoca un metodo di LoaderClass che restituisce l’elenco dei titoli delle presentazioni dell’utente;
					\item Model::MongoRelations::Loader::Autenticazione <- Quando la view invia una richiesta di logout, HomePresenter invoca il metodo deAuthenticate() fornito da Autenticazione, che termina la sessione;
					\item Model::ApacheRelations::ResourceGetter <- la classe della view invoca il metodo save() presente in  HomePresenter passando per parametro un array di id di presentazioni che l'utente intende scaricare in locale, a sua volta HomePresenter invoca il metodo update() di ResourceGetter che controlla se esiste già un file manifest dopodiché lo aggiorna con tutti i riferimenti alle pagine da scaricare e lo ricarica. 					
				\end{itemize} 
			}
		\subsubsection{Presenter::ExecutionPresenter}{
				\textbf{\tipo}: Lo scopo di questa classe è di gestire i segnali delle pagine View::Pages::Execution verso il model.\\	
				\textbf{\relaz}:
					\begin{itemize}
						\item View::Pages::Execution -> costruisce ExecutionPresenter, ne invoca i metodi passando i parametri della presentazione da caricare;
						\item Model::MongoRelations::Loader::LoaderClass <- ExecutionPresenter passa i parametri di caricamento al Loader che carica la presentazione attraverso nodeAPI e la restituisce a ExecutionPresenter;
						\item View::ViewJavascript <- ExecutionPresenter invoca le funzioni di ViewJavascript preposte alla traduzione della presentazione nel codice HTML interpretabile da Impress.
					\end{itemize}
		}
		
		\subsubsection{Presenter::IndexPresenter}{
				\textbf{\tipo}: Lo scopo di questa classe è di gestire i segnali e le chiamate della pagina View::Pages::Index.\\	
				\textbf{\relaz}:
					\begin{itemize}
						\item Model::MongoRelations::Loader::Autenticazione <- Quando la view invia una richiesta di login, HomePresenter invoca il metodo authenticate() fornito da Autenticazione, se il login ha successo IndexPresenter invia alla view una richiesta di redirect alla pagina Home;
						\item Model::MongoRelations::Loader::Registrazione <- Quando la view invia una richiesta di registrazione, HomePresenter invoca il metodo register() fornito da Registrazione, se la registrazione ha successo viene eseguito il login e IndexPresenter invia alla view una richiesta di redirect alla pagina Home;
					\end{itemize}
		}
				
		\subsubsection{Presenter::ProfilePresenter}{
						\textbf{\tipo}: Lo scopo di questa classe è di gestire i segnali e le chiamate della pagina View::Pages::Presenter.\\	
						\textbf{\relaz}:
							\begin{itemize}
								\item Model::ApacheRelations::FileManager <- EditPresenter invoca i metodi di FileManager per caricare un file nel server, per modificarne il nome o per eliminarlo dal server;
								\item Model::MongoRelations::Loader::Autenticazione <- Quando la view invia una richiesta di logout, ProfilePresenter invoca il metodo di Autenticazione deAuthenticate(), che termina la sessione. ProfilePresenter invia quindi una richiesta di redirect alla pagina Index.
							\end{itemize}
				}
			}