\subsection{Model}{
	\textbf{\tipo}: Questo Package è la parte Model dell'architettura MVP.\\
	\textbf{\relaz}: è in relazione con il package Presenter e con NodeAPI.\\
	\textbf{Package contenuti}: 
	\begin{itemize}
	\item Model::SlideShow;
    \item Model::MongoRelations;
	\end{itemize}
}
\subsubsection{Model::SlideShow}{
		\textbf{\tipo}: All’interno di questo Package si trovano le classi che si riferiscono alla costruzione, alla distruzione e alla modifica degli elementi della presentazione oltre alle classi che rappresentano gli elementi stessi della presentazione.\\
        \textbf{\relaz}: il package è in relazione con Presenter da cui riceve le chiamate relative a inserimento, eliminazione e modifica degli elementi. Inoltre comunica con il package Model::MongoRelations, inviando a questi i segnali per la modifica in tempo reale dei dati presenti nel database e inserendo gli oggetti che rappresentano gli elementi nel campo dati presentazione presente all'interno di Model::MongoRelations::Loader::LoaderClass.\\
    }
\subsubsection{Model::SlideShow::SlideShowActions}{
		\textbf{\tipo}: All’interno di questo Package si trovano le classi che si occupano della costruzione, dell'inserimento, della rimozione e della modifica degli elementi della presentazione.\\
        \textbf{\relaz}: il package è in relazione con\\ Model::SlideShow::SlideShowActions::Command che ne invoca le funzioni passando i relativi parametri per l'inserimento, la rimozione e la modifica degli elementi. Inoltre comunica con il package Model::MongoRelations, inviando a questi i segnali per la modifica in tempo reale dei dati presenti nel database.\\
    }
    
	\subsubsection{Model::SlideShow::SlideShowActions::InsertEditRemove}{
		Tutti i componenti seguenti appartengono al package InsertEditRemove, quindi lo scope sarà Model::SlideShow::SlideShowActions::InsertEditRemove::<componente>.
		\begin{figure}[H]
			\centering
			\includegraphics[scale=0.3]{\imgs {inserteditremove}.pdf}
			\label{fig:ier}
			\caption{InsertEditRemove}
		\end{figure}
		\textbf{\tipo}: all’interno di questo Package sono implementate le classi statiche destinate all'inserimento, alla rimozione e alla modifica degli elementi della presentazione.\\
		\textbf{\relaz}:il package è in relazione con\\ Model::SlideShow::SlideShowActions::Command che invoca i metodi delle classi del package.
        Inoltre Model::SlideShow::SlideShowActions::InsertEditRemove::Inserter si occupa di costruire gli oggetti presenti nelle classi del package Model::SlideShow::SlideShowElements.
        InsertEditRemove è in relazione, infine, con il package Model::MongoRelations::DBSynch, infatti tramite chiamate asincrone la classe Inserter costruisce un oggetto Observer e un ConcreteSubject a esso associato.
    }
		\subsubsubsection{Editor}{
			\textbf{\tipo}: Classe statica che offre i metodi destinati all'eliminazione degli elementi all’interno di una presentazione.\\
			 È il componente receiver del Design Pattern Command.\\
			\textbf{\relaz}:
			\begin{itemize}
				\item Model::SlideShow::SlideShowActions::Command::ConcreteEditSizeCommand -> invoca il metodo editSize() messo a disposizione da Editor;
				\item Model::SlideShow::SlideShowActions::Command::ConcreteEditPositionCommand -> invoca il metodo editPosition() messo a disposizione da Editor;
				\item Model::SlideShow::SlideShowActions::Command::ConcreteEditRotationCommand -> invoca il metodo editRotation() messo a disposizione da Editor;
				\item Model::SlideShow::SlideShowActions::Command::ConcreteEditColorCommand -> invoca il metodo editColor() messo a disposizione da Editor;
				\item Model::SlideShow::SlideShowActions::Command::ConcreteEditFontCommand -> invoca il metodo editFont() messo a disposizione da Editor;
				\item Model::SlideShow::SlideShowActions::Command::ConcreteEditBackgroundCommand -> invoca il metodo editBackground() messo a disposizione da Remover;
				
				\item Model::SlideShow::SlideShowElements::Text <- Editor invoca i metodi di set degli oggetti di classe Text;
				\item Model::SlideShow::SlideShowElements::Frame <- Editor invoca i metodi di set degli oggetti di classe Frame;
				\item Model::SlideShow::SlideShowElements::Image <- Editor invoca i metodi di set degli oggetti di classe Image;
				\item Model::SlideShow::SlideShowElements::SVG <- Editor invoca i metodi di set degli oggetti di classe SVG;
				\item Model::SlideShow::SlideShowElements::Audio <- Editor invoca i metodi di set degli oggetti di classe Audio;
				\item Model::SlideShow::SlideShowElements::Video <- Editor invoca i metodi di set degli oggetti di classe Video;
				\item Model::SlideShow::SlideShowElements::Background <- Editor invoca i metodi di set degli oggetti di classe Background;
			\end{itemize} 
		}
	\subsubsubsection{Inserter}{
		\textbf{\tipo}: Classe statica che offre dei metodi per l’inserimento di elementi all’interno di una presentazione.\\
		 È il componente receiver del Design Pattern Command.\\	
		\textbf{\relaz}:
		\begin{itemize}
			\item Model::SlideShow::SlideShowActions::Command::ConcreteTextInsertCommand -> invoca il metodo insertText() messo a disposizione da Inserter;
			\item ConcreteFrameInsertCommand -> invoca il metodo insertFrame() messo a disposizione da Inserter;
			\item ConcreteImageInsertCommand -> invoca il metodo insertImage() messo a disposizione da Inserter;
			\item ConcreteSVGInsertCommand -> invoca il metodo insertSVG() messo a disposizione da Inserter;
			\item ConcreteAudioInsertCommand -> invoca il metodo insertAudio() messo a disposizione da Inserter;
			\item ConcreteVideoInsertCommand -> invoca il metodo insertVideo() messo a disposizione da Inserter;
			\item ConcreteBackgroundInsertCommand -> invoca il metodo insertBackground() messo a disposizione da Inserter;
            \item Model::SlideShow::SlideShowElements::Text <- Inserter costruisce gli oggetti di classe Text;
            \item Frame <- Inserter costruisce gli oggetti di classe Frame;
	         \item Image <- Inserter costruisce gli oggetti di classe Image;
	         \item SVG <- Inserter costruisce gli oggetti di classe SVG;
	         \item Audio <- Inserter costruisce gli oggetti di classe Audio;
	         \item Video <- Inserter costruisce gli oggetti di classe Video;
	         \item Background <- Inserter costruisce gli oggetti di classe Background;
            \item Model::MongoRelations::Loader::Caricatore <- Inserter inserisce gli oggetti JSON nel campo dati contenitore presentazione.
            \item Model::MongoRelations::DBSynch <- Inserter costruisce un ConcreteSubject e un ConcreteObserver. L'elemento costruito da Inserter ha un riferimento al ConcreteSubject così creato.
		\end{itemize} 

	}
	
	\subsubsubsection{Remover}{
		\textbf{\tipo}: Classe statica che offre i metodi destinati all'eliminazione degli elementi all’interno di una presentazione.\\	
		\textbf{\interfacce}: È il componente receiver del Design Pattern Command.\\
		\textbf{\relaz}:
		\begin{itemize}
			\item Model::SlideShow::SlideShowActions::Command::ConcreteTextRemoveCommand -> invoca il metodo removeText() messo a disposizione da Remover;
			\item Model::SlideShow::SlideShowActions::Command::ConcreteFrameRemoveCommand -> invoca il metodo removeFrame() messo a disposizione da Remover;
			\item ConcreteImageRemoveCommand -> invoca il metodo removeImage() messo a disposizione da Remover;
			\item ConcreteSVGRemoveCommand -> invoca il metodo removeSVG() messo a disposizione da Remover;
			\item ConcreteAudioRemoveCommand -> invoca il metodo removeAudio() messo a disposizione da Remover;
			\item ConcreteVideoRemoveCommand -> invoca il metodo removeVideo() messo a disposizione da Remover;
			\item ConcreteBackgroundRemoveCommand -> invoca il metodo removeBackground() messo a disposizione da Remover;
           \item Model::SlideShow::SlideShowElements::Text <- Editor invoca i metodi di set degli oggetti di classe Text;
           \item Frame <- Editor invoca i metodi di set degli oggetti di classe Frame;
			\item Image <- Editor invoca i metodi di set degli oggetti di classe Image;
			\item SVG <- Editor invoca i metodi di set degli oggetti di classe SVG;
			\item Audio <- Editor invoca i metodi di set degli oggetti di classe Audio;
			\item Video <- Editor invoca i metodi di set degli oggetti di classe Video;
			\item Background <- Editor invoca i metodi di set degli oggetti di classe Background;
			\item Text <- Editor invoca i metodi di set degli oggetti di classe Text;
			\item Frame <- Editor invoca i metodi di set degli oggetti di classe Frame;
			\item Image <- Editor invoca i metodi di set degli oggetti di classe Image;
			\item SVG <- Editor invoca i metodi di set degli oggetti di classe SVG;
			\item Audio <- Editor invoca i metodi di set degli oggetti di classe Audio;
			\item Video <- Editor invoca i metodi di set degli oggetti di classe Video;
			\item Background <- Editor invoca i metodi di set degli oggetti di classe Background;
		\end{itemize} 
	}	
	}
   \subsubsection{Model::SlideShow::SlideShowActions::Command}{
   		Tutti i componenti seguenti appartengono al package Command, quindi lo scope sarà Model::SlideShow::SlideShowActions::Command::<componente>.
	   	\begin{figure}[H]
	   		\centering
	   		\includegraphics[scale=0.3]{\imgs {CommandPackage}.pdf}
	   		\label{fig:cp}
	   		\caption{Command Package}
	   	\end{figure}
		\textbf{\tipo}:All’interno di questo Package viene implementato il Design Pattern command, utile per la gestione di funzioni di annullamento e ripristino.\\
		\textbf{\relaz}:. All’interno del Model, il package è in relazione con
		\begin{itemize}
		\item Model::SlideShow::SlideShowActions::InsertEditRemove;
		\end{itemize}
		Presenter::EditPresenter costruisce gli oggetti delle sottoclassi di AbstractCommand, inoltre quando viene invocato il metodo "undo()" di un comando concreto, questo invoca il metodo update() di EditPresenter.\\
	\subsubsubsection{Invoker}{
		\textbf{\tipo}: È componente invoker del Design Pattern Command, il suo scopo è tenere traccia delle modifiche atomiche apportate alla presentazione (modifica di elemento, eliminazione di elemento e inserimento di elemento) per poter implementare le funzioni di annulla/ripristina.\\	
		\textbf{\relaz}:
		\begin{itemize}
			\item Presenter::MobileEdit->crea un oggetto di una sottoclasse di\\ Model::SlideShow::SlideShowActions::Command::AbstractCommand passandolo all’Invoker che ne invoca il metodo execute() e lo inserisce nello stack “undostack”, richiama il metodo che svuota lo stack “redostack”.\\
			Può inoltre invocare il  metodo “undo()” dell’Invoker che provvede a richiamare il metodo "undoaction()" del comando sulla cima dello stack “undostack” e a spostarlo quindi nello stack “redostack”. Alternativamente invoca il  metodo “redo()” dell’Invoker che provvede a invocare il metodo "doaction()" del comando sulla cima dello stack “redostack” e a spostarlo quindi nello stack “undostack”;
			\item Presenter::DesktopEdit->si comporta in modo analogo a MobileEdit;
			\item Model::SlideShow::SlideShowActions::Command::AbstractCommand <- Invoker invoca il metodo doaction() dell'oggetto della sottoclasse di AbstractCommand. Alternativamente invoca il metodo undoaction().
		\end{itemize} 
		\textbf{\interfacce}: Viene invocato per effettuare le operazioni di modifica alla presentazione, a sua volta invoca i metodi doaction() o undoaction() di una classe derivata da Model::SlideShow::SlideShowActions::Command::AbstractCommand per eseguire materialmente il comando. Quando un comando viene eseguito, Invoker lo salva in un array \$undostack[ ].\\
	}
	\subsubsubsection{AbstractCommand}{
				\textbf{\tipo}: È interfaccia astratta del Design Pattern Command, è classe base per i comandi di modifica, inserimento ed eliminazione.\\	
				\textbf{\relaz}: 
				\begin{itemize}
                    \item Model::Invoker -> esegue materialmente il comando, richiamandone il metodo doaction(); inoltre provvede ad annullare l’ultima operazione invocandone il metodo undoaction().
				\end{itemize}	
                \textbf{\interfacce}:Viene utilizzata per applicare un generico parametro di trasformazione ad un oggetto della presentazione, questo parametro verrà poi specificato dalle classi concrete.\\
                \textbf{\figli}: 
                    \begin{itemize}
                    \item ConcreteTextInsertCommand;
                    \item ConcreteFrameInsertCommand;
                    \item ConcreteImageInsertCommand;
                    \item ConcreteSVGInsertCommand;
                    \item ConcreteAudioInsertCommand;
                    \item ConcreteVideoInsertCommand;
                    \item ConcreteBackgroundInsertCommand;
                    \item ConcreteTextRemoveCommand;
                    \item ConcreteFrameRemoveCommand;
                    \item ConcreteImageRemoveCommand;
                    \item ConcreteSVGRemoveCommand;
                    \item ConcreteAudioRemoveCommand;
                    \item ConcreteVideoRemoveCommand;
                    \item ConcreteBackgroundRemoveCommand;
                    \item ConcreteEditSizeCommand;
                    \item ConcreteEditPositionCommand;
                    \item ConcreteEditRotationCommand;
                    \item ConcreteEditColorCommand;
                    \item ConcreteEditBackgroundCommand;
                    \item ConcreteEditFontCommand;
                    \item ConcreteEditContentCommand.
                    \end{itemize}
                    }
    \subsubsubsection{ConcreteTextInsertCommand}{
				\textbf{\tipo}: È classe concreta del Design Pattern Command, rappresenta un comando per inserire un nuovo elemento testuale nella presentazione.\\	
				\textbf{\relaz}: 
				\begin{itemize}
					\item Presenter::EditPresenter -> invoca il costruttore della classe e passa l’oggetto così creato all’Invoker;
					\item Model::SlideShow::SlideShowActions::Command::Invoker -> invoca il metodo doaction() del comando e lo inserisce nel campo dati "undostack" e ne setta il valore del campo dati booleano "executed" a true, o ne invoca il metodo di annullamento undoaction() e lo inserisce nel campo dati "redostack";
                    \item Model::SlideShow::SlideShowActions::InsertEditRemove::Inserter <- invoca il metodo insertText(...) della classe statica per l’inserimento di un elemento;
                    \item Premi::Presenter::EditPresenter <- l'oggetto invoca il metodo update() di EditPresenter quando viene invocato il metodo doaction() e il campo dati booleano "executed" ha valore true, o quando viene invocato il metodo undoaction().
				\end{itemize}	
                \textbf{\base}: 
                    \begin{itemize}
                    \item Model::SlideShow::SlideShowActions::Command::AbstractCommand.
                    \end{itemize}
                    }
                    \subsubsubsection{ConcreteFrameInsertCommand}{
				\textbf{\tipo}: È classe concreta del Design Pattern Command, rappresenta un comando per inserire un nuovo elemento frame nella presentazione.\\	
				\textbf{\relaz}: 
				\begin{itemize}
					\item Presenter::EditPresenter -> invoca il costruttore della classe e passa l’oggetto così creato all’Invoker;
					\item Model::SlideShow::SlideShowActions::Command::Invoker -> invoca il metodo doaction() del comando e lo inserisce nel campo dati "undostack" e ne setta il valore del campo dati booleano "executed" a true, o ne invoca il metodo di annullamento undoaction() e lo inserisce nel campo dati "redostack";
                    \item Model::SlideShow::SlideShowActions::InsertEditRemove::Inserter <- invoca il metodo insertFrame(...) della classe statica per l’inserimento di un elemento frame nella presentazione;
                    \item Premi::Presenter::EditPresenter <- l'oggetto invoca il metodo update() di EditPresenter quando viene invocato il metodo doaction() e il campo dati booleano "executed" ha valore true, o quando viene invocato il metodo undoaction().
				\end{itemize}	
                \textbf{\base}: 
                    \begin{itemize}
                    \item Model::SlideShow::SlideShowActions::Command::AbstractCommand.
                    \end{itemize}
                    }
                    \subsubsubsection{ConcreteImageInsertCommand}{
				\textbf{\tipo}: È classe concreta del Design Pattern Command, rappresenta un comando per inserire un nuovo elemento immagine nella presentazione.\\	
				\textbf{\relaz}: 
				\begin{itemize}
					\item Presenter::EditPresenter -> invoca il costruttore della classe e passa l’oggetto così creato all’Invoker;
					\item Model::SlideShow::SlideShowActions::Command::Invoker -> invoca il metodo doaction() del comando e lo inserisce nel campo dati "undostack" e ne setta il valore del campo dati booleano "executed" a true, o ne invoca il metodo di annullamento undoaction() e lo inserisce nel campo dati "redostack";
                    \item Model::SlideShow::SlideShowActions::InsertEditRemove::Inserter <- invoca il metodo insertImage(...) della classe statica per l’inserimento di un elemento immagine nella presentazione;
                    \item Premi::Presenter::EditPresenter <- l'oggetto invoca il metodo update() di EditPresenter quando viene invocato il metodo doaction() e il campo dati booleano "executed" ha valore true, o quando viene invocato il metodo undoaction().
				\end{itemize}	
                \textbf{\base}: 
                    \begin{itemize}
                    \item Model::SlideShow::SlideShowActions::Command::AbstractCommand.
                    \end{itemize}
                    }
                    \subsubsubsection{ConcreteSVGInsertCommand}{
				\textbf{\tipo}: È classe concreta del Design Pattern Command, rappresenta un comando per inserire un nuovo elemento SVG nella presentazione.\\	
				\textbf{\relaz}: 
				\begin{itemize}
					\item Presenter::EditPresenter -> invoca il costruttore della classe e passa l’oggetto così creato all’Invoker;
					\item Model::SlideShow::SlideShowActions::Command::Invoker -> invoca il metodo doaction() del comando e lo inserisce nel campo dati "undostack" e ne setta il valore del campo dati booleano "executed" a true, o ne invoca il metodo di annullamento undoaction() e lo inserisce nel campo dati "redostack";
                    \item Model::SlideShow::SlideShowActions::InsertEditRemove::Inserter <- invoca il metodo insertSVG(...) della classe statica per l’inserimento di un elemento SVG nella presentazione;
                    \item Premi::Presenter::EditPresenter <- l'oggetto invoca il metodo update() di EditPresenter quando viene invocato il metodo doaction() e il campo dati booleano "executed" ha valore true, o quando viene invocato il metodo undoaction().
				\end{itemize}	
                \textbf{\base}: 
                    \begin{itemize}
                    \item Model::SlideShow::SlideShowActions::Command::AbstractCommand.
                    \end{itemize}
                    }
                \subsubsubsection{ConcreteAudioInsertCommand}{
				\textbf{\tipo}: È classe concreta del Design Pattern Command, rappresenta un comando per inserire un nuovo elemento audio nella presentazione.\\	
				\textbf{\relaz}: 
				\begin{itemize}
					\item Presenter::EditPresenter -> invoca il costruttore della classe e passa l’oggetto così creato all’Invoker;
					\item Model::SlideShow::SlideShowActions::Command::Invoker -> invoca il metodo doaction() del comando e lo inserisce nel campo dati "undostack" e ne setta il valore del campo dati booleano "executed" a true, o ne invoca il metodo di annullamento undoaction() e lo inserisce nel campo dati "redostack";
                    \item Model::SlideShow::SlideShowActions::InsertEditRemove::Inserter <- invoca il metodo insertAudio(...) della classe statica per l’inserimento di un elemento audio nella presentazione;
                    \item Premi::Presenter::EditPresenter <- l'oggetto invoca il metodo update() di EditPresenter quando viene invocato il metodo doaction() e il campo dati booleano "executed" ha valore true, o quando viene invocato il metodo undoaction().
				\end{itemize}	
                \textbf{\interfacce}: Viene utilizzata per gestire le richieste di inserimento di un nuovo elemento Audio.\\
                \textbf{\base}: 
                    \begin{itemize}
                    \item Model::SlideShow::SlideShowActions::Command::AbstractCommand.
                    \end{itemize}
                    }
                \subsubsubsection{ConcreteVideoInsertCommand}{
				\textbf{\tipo}: È classe concreta del Design Pattern Command, rappresenta un comando per inserire un nuovo elemento video nella presentazione.\\	
				\textbf{\relaz}: 
				\begin{itemize}
					\item Presenter::EditPresenter -> invoca il costruttore della classe e passa l’oggetto così creato all’Invoker;
					\item Model::SlideShow::SlideShowActions::Command::Invoker -> invoca il metodo doaction() del comando  e ne setta il valore del campo dati booleano "executed" a true, o ne invoca il metodo di annullamento undoaction() e lo inserisce nel campo dati "redostack";
                    \item Model::SlideShow::SlideShowActions::InsertEditRemove::Inserter <- invoca il metodo insertVideo(...) della classe statica per l’inserimento di un elemento video nella presentazione;
                    \item Premi::Presenter::EditPresenter <- l'oggetto invoca il metodo update() di EditPresenter quando viene invocato il metodo doaction() e il campo dati booleano "executed" ha valore true, o quando viene invocato il metodo undoaction().
				\end{itemize}	
                \textbf{\base}: 
                    \begin{itemize}
                    \item Model::SlideShow::SlideShowActions::Command::AbstractCommand.
                    \end{itemize}
                    }
                \subsubsubsection{ConcreteBackgroundInsertCommand}{
				\textbf{\tipo}: È classe concreta del Design Pattern Command, rappresenta un comando per inserire un nuovo elemento video nella presentazione.\\	
				\textbf{\relaz}: 
				\begin{itemize}
					\item Presenter::EditPresenter -> invoca il costruttore della classe e passa l’oggetto così creato all’Invoker;
					\item Model::SlideShow::SlideShowActions::Command::Invoker -> invoca il metodo doaction() del comando e lo inserisce nel campo dati "undostack" e ne setta il valore del campo dati booleano "executed" a true, o ne invoca il metodo di annullamento undoaction() e lo inserisce nel campo dati "redostack";
                    \item Model::SlideShow::SlideShowActions::InsertEditRemove::Inserter <- invoca il metodo insertBackground(...) della classe statica per l’inserimento di un elemento sfondo nella presentazione;
                    \item Premi::Presenter::EditPresenter <- l'oggetto invoca il metodo update() di EditPresenter quando viene invocato il metodo doaction() e il campo dati booleano "executed" ha valore true, o quando viene invocato il metodo undoaction().
				\end{itemize}
                \textbf{\base}: 
                    \begin{itemize}
                    \item Model::SlideShow::SlideShowActions::Command::AbstractCommand.
                    \end{itemize}
                    }        
     \subsubsubsection{ConcreteTextRemoveCommand}{
				\textbf{\tipo}: È classe concreta del Design Pattern Command, rappresenta un comando per rimuovere un elemento dalla presentazione.\\	
				\textbf{\relaz}: 
				\begin{itemize}
					\item Presenter::EditPresenter -> invoca il costruttore della classe e passa l’oggetto così creato all’Invoker;
					\item Model::SlideShow::SlideShowActions::Command::Invoker -> invoca il metodo doaction() del comando e lo inserisce nel campo dati "undostack" e ne setta il valore del campo dati booleano "executed" a true, o ne invoca il metodo di annullamento undoaction() e lo inserisce nel campo dati "redostack";
                    \item Model::SlideShow::SlideShowActions::InsertEditRemove::Remover <- invoca il metodo removeText(...) della classe statica per la rimozione di un elemento testuale nella presentazione;
                    \item Premi::Presenter::EditPresenter <- l'oggetto invoca il metodo update() di EditPresenter quando viene invocato il metodo doaction() e il campo dati booleano "executed" ha valore true, o quando viene invocato il metodo undoaction().
				\end{itemize}	
                \textbf{\base}: 
                    \begin{itemize}
                    \item Model::SlideShow::SlideShowActions::Command::AbstractCommand.
                    \end{itemize}
                    }
        \subsubsubsection{ConcreteFrameRemoveCommand}{
				\textbf{\tipo}: È classe concreta del Design Pattern Command, rappresenta un comando per rimuovere un elemento frame dalla presentazione.\\	
				\textbf{\relaz}: 
				\begin{itemize}
					\item Presenter::EditPresenter -> invoca il costruttore della classe e passa l’oggetto così creato all’Invoker;
					\item Model::SlideShow::SlideShowActions::Command::Invoker -> invoca il metodo doaction() del comando e lo inserisce nel campo dati "undostack" e ne setta il valore del campo dati booleano "executed" a true, o ne invoca il metodo di annullamento undoaction() e lo inserisce nel campo dati "redostack";
                    \item Model::SlideShow::SlideShowActions::InsertEditRemove::Remover <- invoca il metodo removeFrame(...) della classe statica per la rimozione di un elemento frame dalla presentazione;
                    \item Premi::Presenter::EditPresenter <- l'oggetto invoca il metodo update() di EditPresenter quando viene invocato il metodo doaction() e il campo dati booleano "executed" ha valore true, o quando viene invocato il metodo undoaction().
				\end{itemize}	
                \textbf{\base}: 
                    \begin{itemize}
                    \item Model::SlideShow::SlideShowActions::Command::AbstractCommand.
                    \end{itemize}
                    }                   
                    \subsubsubsection{ConcreteImageRemoveCommand}{
				\textbf{\tipo}: È classe concreta del Design Pattern Command, rappresenta un comando per rimuovere un elemento immagine dalla presentazione.\\	
				\textbf{\relaz}: 
				\begin{itemize}
					\item Presenter::EditPresenter -> invoca il costruttore della classe e passa l’oggetto così creato all’Invoker;
					\item Model::SlideShow::SlideShowActions::Command::Invoker -> Invoker invoca il metodo doaction() del comando e lo inserisce nel campo dati "undostack" e ne setta il valore del campo dati booleano "executed" a true, o ne invoca il metodo di annullamento undoaction() e lo inserisce nel campo dati "redostack";
                    \item Model::SlideShow::SlideShowActions::InsertEditRemove::Remover <- invoca il metodo removeImage(...) della classe statica per l’eliminazione di un elemento immagine dalla presentazione;
                    \item Premi::Presenter::EditPresenter <- l'oggetto invoca il metodo update() di EditPresenter quando viene invocato il metodo doaction() e il campo dati booleano "executed" ha valore true, o quando viene invocato il metodo undoaction().
				\end{itemize}	
                \textbf{\base}: 
                    \begin{itemize}
                    \item Model::SlideShow::SlideShowActions::Command::AbstractCommand.
                    \end{itemize}
                    }               
                    \subsubsubsection{ConcreteSVGRemoveCommand}{
				\textbf{\tipo}: È classe concreta del Design Pattern Command, rappresenta un comando per rimuovere un elemento SVG dalla presentazione.\\	
				\textbf{\relaz}: 
				\begin{itemize}
					\item Presenter::EditPresenter -> invoca il costruttore della classe e passa l’oggetto così creato all’Invoker;
					\item Model::SlideShow::SlideShowActions::Command::Invoker -> Invoker invoca il metodo doaction() del comando e lo inserisce nel campo dati "undostack" e ne setta il valore del campo dati booleano "executed" a true, o ne invoca il metodo di annullamento undoaction() e lo inserisce nel campo dati "redostack";
                    \item Model::SlideShow::SlideShowActions::InsertEditRemove::Remover <- invoca il metodo removeSVG(...) della classe statica per l’eliminazione di un elemento SVG dalla presentazione;
                    \item Premi::Presenter::EditPresenter <- l'oggetto invoca il metodo update() di EditPresenter quando viene invocato il metodo doaction() e il campo dati booleano "executed" ha valore true, o quando viene invocato il metodo undoaction().
				\end{itemize}	
                \textbf{\base}: 
                    \begin{itemize}
                    \item Model::SlideShow::SlideShowActions::Command::AbstractCommand.
                    \end{itemize}
                    }
                    \subsubsubsection{ConcreteAudioRemoveCommand}{
				\textbf{\tipo}: È classe concreta del Design Pattern Command, rappresenta un comando per rimuovere un elemento audio dalla presentazione.\\	
				\textbf{\relaz}: 
				\begin{itemize}
					\item Presenter::EditPresenter -> invoca il costruttore della classe e passa l’oggetto così creato all’Invoker;
					\item Model::SlideShow::SlideShowActions::Command::Invoker -> Invoker invoca il metodo doaction() del comando e lo inserisce nel campo dati "undostack" e ne setta il valore del campo dati booleano "executed" a true, o ne invoca il metodo di annullamento undoaction() e lo inserisce nel campo dati "redostack";
                    \item Model::SlideShow::SlideShowActions::InsertEditRemove::Remover <- invoca il metodo removeAudio(...) della classe statica per l’eliminazione di un elemento immagine dalla presentazione;
                    \item Premi::Presenter::EditPresenter <- l'oggetto invoca il metodo update() di EditPresenter quando viene invocato il metodo doaction() e il campo dati booleano "executed" ha valore true, o quando viene invocato il metodo undoaction().
				\end{itemize}	
                \textbf{\base}: 
                    \begin{itemize}
                    \item Model::SlideShow::SlideShowActions::Command::AbstractCommand.
                    \end{itemize}
                    }
                    \subsubsubsection{ConcreteVideoRemoveCommand}{
				\textbf{\tipo}: È classe concreta del Design Pattern Command, rappresenta un comando per rimuovere un elemento video dalla presentazione.\\	
				\textbf{\relaz}: 
				\begin{itemize}
					\item Presenter::EditPresenter -> invoca il costruttore della classe e passa l’oggetto così creato all’Invoker;
					\item Model::SlideShow::SlideShowActions::Command::Invoker -> Invoker invoca il metodo doaction() del comando e lo inserisce nel campo dati "undostack" e ne setta il valore del campo dati booleano "executed" a true, o ne invoca il metodo di annullamento undoaction() e lo inserisce nel campo dati "redostack";
                    \item Model::SlideShow::SlideShowActions::InsertEditRemove::Remover <- invoca il metodo removeVideo(...) della classe statica per l’eliminazione di un elemento video dalla presentazione;
                    \item Premi::Presenter::EditPresenter <- l'oggetto invoca il metodo update() di EditPresenter quando viene invocato il metodo doaction() e il campo dati booleano "executed" ha valore true, o quando viene invocato il metodo undoaction().
				\end{itemize}
                \textbf{\base}: 
                    \begin{itemize}
                    \item Model::SlideShow::SlideShowActions::Command::AbstractCommand.
                    \end{itemize}
                    }
                    \subsubsubsection{ConcreteBackgroundRemoveCommand}{
				\textbf{\tipo}: È classe concreta del Design Pattern Command, rappresenta un comando per rimuovere lo sfondo della presentazione.\\	
				\textbf{\relaz}: 
				\begin{itemize}
					\item Presenter::EditPresenter -> invoca il costruttore della classe e passa l’oggetto così creato all’Invoker;
					\item Model::SlideShow::SlideShowActions::Command::Invoker -> Invoker invoca il metodo doaction() del comando e lo inserisce nel campo dati "undostack" e ne setta il valore del campo dati booleano "executed" a true, o ne invoca il metodo di annullamento undoaction() e lo inserisce nel campo dati "redostack";
                    \item Model::SlideShow::SlideShowActions::InsertEditRemove::Remover <- invoca il metodo removeBackground(...) della classe statica per l’eliminazione dell'elemento sfondo dalla presentazione;
                    \item Premi::Presenter::EditPresenter <- l'oggetto invoca il metodo update() di EditPresenter quando viene invocato il metodo doaction() e il campo dati booleano "executed" ha valore true, o quando viene invocato il metodo undoaction().
				\end{itemize}	
                \textbf{\base}: 
                    \begin{itemize}
                    \item Model::SlideShow::SlideShowActions::Command::AbstractCommand.
                    \end{itemize}
                    }
                        \subsubsubsection{ConcreteEditSizeCommand}{
				\textbf{\tipo}: È classe concreta del Design Pattern Command, rappresenta un comando per modificare le dimensioni di un elemento della presentazione.\\	
				\textbf{\relaz}: 
				\begin{itemize}
					\item Presenter::EditPresenter -> invoca il costruttore della classe e passa l’oggetto così creato all’Invoker;
					\item Model::SlideShow::SlideShowActions::Command::Invoker -> Invoker invoca il metodo doaction() del comando e lo inserisce nel campo dati "undostack" e ne setta il valore del campo dati booleano "executed" a true, o ne invoca il metodo di annullamento undoaction() e lo inserisce nel campo dati "redostack";
                    \item Model::SlideShow::SlideShowActions::InsertEditRemove::Editor <- il comando invoca il metodo editSize(...) della classe statica per la modifica dei campi dati relativi alle dimensioni dell'oggetto nella presentazione;
                    \item Premi::Presenter::EditPresenter <- l'oggetto invoca il metodo update() di EditPresenter quando viene invocato il metodo doaction() e il campo dati booleano "executed" ha valore true, o quando viene invocato il metodo undoaction().
				\end{itemize}	
                \textbf{\base}: 
                    \begin{itemize}
                    \item Model::SlideShow::SlideShowActions::Command::AbstractCommand.
                    \end{itemize}
                    }
                    \subsubsubsection{ConcreteEditPositionCommand}{
				\textbf{\tipo}: È classe concreta del Design Pattern Command, rappresenta un comando per modificare la posizione di un elemento della presentazione.\\	
				\textbf{\relaz}: 
				\begin{itemize}
					\item Presenter::EditPresenter -> invoca il costruttore della classe e passa l’oggetto così creato all’Invoker;
					\item Model::SlideShow::SlideShowActions::Command::Invoker -> Invoker invoca il metodo doaction() del comando e lo inserisce nel campo dati "undostack" e ne setta il valore del campo dati booleano "executed" a true, o ne invoca il metodo di annullamento undoaction() e lo inserisce nel campo dati "redostack";
                    \item Model::SlideShow::SlideShowActions::InsertEditRemove::Editor <- il comando invoca il metodo editPosition(...) della classe statica per la modifica dei campi dati relativi alla posizione dell'oggetto nella presentazione;
                    \item Premi::Presenter::EditPresenter <- l'oggetto invoca il metodo update() di EditPresenter quando viene invocato il metodo doaction() e il campo dati booleano "executed" ha valore true, o quando viene invocato il metodo undoaction().
				\end{itemize}	
                \textbf{\base}: 
                    \begin{itemize}
                    \item Model::SlideShow::SlideShowActions::Command::AbstractCommand.
                    \end{itemize}
                    }
                    \subsubsubsection{ConcreteEditColorCommand}{
				\textbf{\tipo}: È classe concreta del Design Pattern Command, rappresenta un comando per modificare il colore di un elemento della presentazione.\\	
				\textbf{\relaz}: 
				\begin{itemize}
					\item Presenter::EditPresenter -> invoca il costruttore della classe e passa l’oggetto così creato all’Invoker;
					\item Model::SlideShow::SlideShowActions::Command::Invoker -> Invoker invoca il metodo doaction() del comando e lo inserisce nel campo dati "undostack" e ne setta il valore del campo dati booleano "executed" a true, o ne invoca il metodo di annullamento undoaction() e lo inserisce nel campo dati "redostack";
                    \item Model::SlideShow::SlideShowActions::InsertEditRemove::Editor <- il comando invoca il metodo editColor(...) della classe statica per la modifica del campo dati relativo al colore dell'oggetto della presentazione;
                    \item Premi::Presenter::EditPresenter <- l'oggetto invoca il metodo update() di EditPresenter quando viene invocato il metodo doaction() e il campo dati booleano "executed" ha valore true, o quando viene invocato il metodo undoaction().
				\end{itemize}	
                \textbf{\base}: 
                    \begin{itemize}
                    \item Model::SlideShow::SlideShowActions::Command::AbstractCommand.
                    \end{itemize}
                    }
                     \subsubsubsection{ConcreteEditBackgroundCommand}{
				\textbf{\tipo}: È classe concreta del Design Pattern Command, rappresenta un comando per modificare lo sfondo di un elemento frame della presentazione.\\	
				\textbf{\relaz}: 
				\begin{itemize}
					\item Presenter::EditPresenter -> invoca il costruttore della classe e passa l’oggetto così creato all’Invoker;
					\item Model::SlideShow::SlideShowActions::Command::Invoker -> Invoker invoca il metodo doaction() del comando e lo inserisce nel campo dati "undostack" e ne setta il valore del campo dati booleano "executed" a true, o ne invoca il metodo di annullamento undoaction() e lo inserisce nel campo dati "redostack";
                    \item Model::SlideShow::SlideShowActions::InsertEditRemove::Editor <- il comando invoca il metodo editBackground(...) della classe statica per la modifica del campo dati relativo allo sfondo dell'oggetto della presentazione;
                    \item Premi::Presenter::EditPresenter <- l'oggetto invoca il metodo update() di EditPresenter quando viene invocato il metodo doaction() e il campo dati booleano "executed" ha valore true, o quando viene invocato il metodo undoaction().
				\end{itemize}	
                \textbf{\base}: 
                    \begin{itemize}
                    \item Model::SlideShow::SlideShowActions::Command::AbstractCommand.
                    \end{itemize}
                    }
                     \subsubsubsection{ConcreteEditRotationCommand}{
				\textbf{\tipo}: È classe concreta del Design Pattern Command, rappresenta un comando per modificare l'orientamento di un elemento della presentazione.\\	
				\textbf{\relaz}: 
				\begin{itemize}
					\item Presenter::EditPresenter -> invoca il costruttore della classe e passa l’oggetto così creato all’Invoker;
					\item Model::SlideShow::SlideShowActions::Command::Invoker -> Invoker invoca il metodo doaction() del comando e lo inserisce nel campo dati "undostack" e ne setta il valore del campo dati booleano "executed" a true, o ne invoca il metodo di annullamento undoaction() e lo inserisce nel campo dati "redostack";
                    \item Model::SlideShow::SlideShowActions::InsertEditRemove::Editor <- il comando invoca il metodo editRotation(...) della classe statica per la modifica del campo dati relativo all'orientamento dell'oggetto della presentazione;
                    \item Premi::Presenter::EditPresenter <- l'oggetto invoca il metodo update() di EditPresenter quando viene invocato il metodo doaction() e il campo dati booleano "executed" ha valore true, o quando viene invocato il metodo undoaction().
				\end{itemize}	
                \textbf{\base}: 
                    \begin{itemize}
                    \item Model::SlideShow::SlideShowActions::Command::AbstractCommand.
                    \end{itemize}
                    }
                    \subsubsubsection{ConcreteEditFontCommand}{
				\textbf{\tipo}: È classe concreta del Design Pattern Command, rappresenta un comando per modificare il carattere di un elemento testuale della presentazione.\\	
				\textbf{\relaz}: 
				\begin{itemize}
					\item Presenter::EditPresenter -> invoca il costruttore della classe e passa l’oggetto così creato all’Invoker;
					\item Model::SlideShow::SlideShowActions::Command::Invoker -> Invoker invoca il metodo doaction() del comando e lo inserisce nel campo dati "undostack" e ne setta il valore del campo dati booleano "executed" a true, o ne invoca il metodo di annullamento undoaction() e lo inserisce nel campo dati "redostack";
                    \item Model::SlideShow::SlideShowActions::InsertEditRemove::Editor <- il comando invoca il metodo editColor(...) della classe statica per la modifica dei campi dati relativi al font dell'oggetto testuale della presentazione;
                    \item Premi::Presenter::EditPresenter <- l'oggetto invoca il metodo update() di EditPresenter quando viene invocato il metodo doaction() e il campo dati booleano "executed" ha valore true, o quando viene invocato il metodo undoaction().
				\end{itemize}	
                \textbf{\base}: 
                    \begin{itemize}
                    \item Model::SlideShow::SlideShowActions::Command::AbstractCommand.
                    \end{itemize}
                    }
                    }
     \subsubsection{Model::SlideShow::SlideShowElements}{
     	Tutti i componenti seguenti appartengono al package SlideShowElements, quindi lo scope sarà Model::SlideShow::SlideShowElements::<componente>.
               	\begin{figure}[H]
           			\centering
          			\includegraphics[scale=0.28]{\imgs {slideshowelements}.pdf}
         			\label{fig:sse}
    				\caption{SlideShowElements}
               	\end{figure}
		\textbf{\tipo}:Di questo package fanno parte le classi degli elementi della presentazione e la classe che definisce la presentazione stessa.\\
		\textbf{\relaz}:Model::SlideShow::SlideShowElements è in comunicazione con 
        \begin{itemize}
        \item Model::SlideShow::SlideShowActions::Insert, i cui oggetti durante la modifica della presentazione istanziano oggetti di tipo SlideShowElement;
        \item Model::Remove, i cui oggetti rimuovono da MongoRelations::Caricatore gli oggetti di tipo SlideShowElement e li distruggono;
        \item Model::SlideShow::SlideShowActions::EditElements, i cui oggetti invocano metodi degli oggetti SlideShowElement che ne impostano i campi;
        \item Model::DBSynch, i metodi di set degli oggetti delle classi del package SlideShowElements, infatti, invocano il metodo notify dell'observer contenuto nel package DBSynch a cui l'oggetto è associato.
  		\end{itemize}

    \subsubsubsection{SlideShowElement}{
				\textbf{\tipo}: Gli oggetti della classe SlideShowElement rappresentano gli elementi della presentazione.\\	
				\textbf{\relaz}: 
				\begin{itemize}
					\item Model::SlideShow::SlideShowActions::InsertEditRemove::Inserter -> invoca il costruttore delle sottoclassi di SlideShowElement e li inserisce nel campo dati "presentazione" all'interno di Model::MongoRelations::Loader::LoaderClass;
                    \item Model::SlideShow::SlideShowActions::InsertEditRemove::Editor -> gli oggetti delle sue sottoclassi richiamano le funzioni delle sottoclassi di SlideShowElement che gestiscono l’impostazione dei campi dati;
                    \item Model::SlideShow::SlideShowActions::Remove::Remover -> gli oggetti delle sue sottoclassi rimuovono dai contenitori di SlideShow gli oggetti di classe SlideShowElement e ne richiamano i distruttori;
                    \item Model::DBSynch::ConcreteObserver <- i metodi di set degli oggetti delle sottoclassi di SlideShowElements invocano il metodo notify() dell'observer contenuto nel package DBSynch di cui l'oggetto tiene un riferimento.
				\end{itemize}	
                \textbf{\interfacce}:\\ Model::SlideShow::SlideShowActions::InsertEditRemove::Inserter istanzia oggetti di sottoclassi di SlideShowElement e li inserisce nel campo dati contenitore presentazione all’interno di\\ Model::MongoRelations::Model:LoaderClass\\
                \textbf{\figli}: 
                    \begin{itemize}
                    \item Model::SlideShow::Text;
                    \item Model::SlideShow::Frame;
                    \item Model::SlideShow::Image;
                    \item Model::SlideShow::SVG;
                    \item Model::SlideShow::Audio;
                    \item Model::SlideShow::Video;
                    \item Model::SlideShow::Background.
                    \end{itemize}
                    }
     \subsubsubsection{Text}{
				\textbf{\tipo}: Gli oggetti della classe Text rappresentano gli elementi di tipo testuale della presentazione.\\
				\textbf{\relaz}: 
				\begin{itemize}
					\item Model::SlideShow::SlideShowActions::InsertEditRemove::Inserter -> invoca il costruttore di Text e inserisce l’oggetto nel campo dati contenitore all’interno dell’oggetto della classe Model::MongoRelations::Loader::Caricatore;
                    \item Model::SlideShow::SlideShowActions::InsertEditRemove::Remover -> rimuove l’oggetto Text dal campo dati presentazione all’interno di Model::MongoRelations::Loader::LoaderClass, ne invoca quindi il distruttore;
                    \item Model::SlideShow::SlideShowActions::InsertEditRemove::Editor -> invoca i metodi che modificano i campi dati dell'oggetto;
                    \item Model::DBSynch::ConcreteObserver <- i metodi di set della classe invocano il metodo notify() dell'observer contenuto nel package DBSynch di cui l'oggetto della classe tiene un riferimento.
				\end{itemize}	
                \textbf{\interfacce}: Gli oggetti della classe Text vengono istanziati da Model::SlideShow::SlideShowActions::Insert::ConcreteTextInserter e inseriti nel campo dati contenitore presentazione all’interno di \\Model::MongoRelations::Loader::LoaderClass.\\
                \textbf{\base}: 
                    \begin{itemize}
                    \item Model::SlideShow::SlideShowElement.
                    \end{itemize}
                    }
           \subsubsubsection{Frame}{
				\textbf{\tipo}: Gli oggetti della classe Frame rappresentano gli elementi di tipo frame della presentazione.\\
				\textbf{\relaz}: 
				\begin{itemize}
					\item Model::SlideShow::SlideShowActions::InsertEditRemove::Inserter -> invoca il costruttore di Frame e inserisce l’oggetto nel campo dati contenitore all’interno dell’oggetto della classe Model::MongoRelations::Loader::Caricatore;
                    \item Model::SlideShow::SlideShowActions::InsertEditRemove::Remover -> rimuove l’oggetto Frame dal campo dati presentazione all’interno di Model::MongoRelations::Loader::LoaderClass, ne invoca quindi il distruttore;
                    \item Model::SlideShow::SlideShowActions::InsertEditRemove::Editor -> invoca i metodi che modificano i campi dati dell'oggetto;
                    \item Model::DBSynch::ConcreteObserver <- i metodi di set della classe invocano il metodo notify() dell'observer contenuto nel package DBSynch di cui l'oggetto della classe tiene un riferimento.\end{itemize}	
                \textbf{\interfacce}: Gli oggetti della classe Frame vengono istanziati da Model::SlideShow::SlideShowActions::InsertEditRemove::Inserter inseriti nel campo dati contenitore presentazione all’interno di \\Model::MongoRelations::Loader::LoaderClass.\\
                \textbf{\base}: 
                    \begin{itemize}
                    \item Model::SlideShow::SlideShowElement.
                    \end{itemize}
                    }
                    \subsubsubsection{Image}{
				\textbf{\tipo}: Gli oggetti della classe Image rappresentano gli elementi di tipo immagine della presentazione.\\
				\textbf{\relaz}: 
				\begin{itemize}
					\item Model::SlideShow::SlideShowActions::InsertEditRemove::Inserter -> invoca il costruttore di Image e inserisce l’oggetto nel campo dati contenitore all’interno dell’oggetto della classe Model::MongoRelations::Loader::Caricatore;
                    \item Model::SlideShow::SlideShowActions::InsertEditRemove::Remover -> rimuove l’oggetto Image dal campo dati presentazione all’interno di Model::MongoRelations::Loader::LoaderClass, ne invoca quindi il distruttore;
                    \item Model::SlideShow::SlideShowActions::InsertEditRemove::Editor ->  invoca i metodi che modificano i campi dati dell'oggetto;
                    \item Model::DBSynch::ConcreteObserver <- i metodi di set della classe invocano il metodo notify() dell'observer contenuto nel package DBSynch di cui l'oggetto della classe tiene un riferimento.
				\end{itemize}	
                \textbf{\interfacce}: Gli oggetti della classe Image vengono istanziati da Model::SlideShow::SlideShowActions::InsertEditRemove::Inserter  inseriti nel campo dati contenitore presentazione all’interno di \\Model::MongoRelations::Loader::LoaderClass.\\
                \textbf{\base}: 
                    \begin{itemize}
                    \item Model::SlideShow::SlideShowElement.
                    \end{itemize}
                    }
                    \subsubsubsection{SVG}{
				\textbf{\tipo}: Gli oggetti della classe SVG rappresentano gli elementi di tipo SVG della presentazione.\\
				\textbf{\relaz}: 
				\begin{itemize}
					\item Model::SlideShow::SlideShowActions::InsertEditRemove::Inserter -> invoca il costruttore di SVG e inserisce l’oggetto nel campo dati contenitore all’interno dell’oggetto della classe Model::MongoRelations::Loader::Caricatore;
                    \item Model::SlideShow::SlideShowActions::InsertEditRemove::Remover -> rimuove l’oggetto SVG dal campo dati presentazione all’interno di Model::MongoRelations::Loader::LoaderClass, ne invoca quindi il distruttore;
                    \item Model::SlideShow::SlideShowActions::InsertEditRemove::Editor -> invoca i metodi che modificano i campi dati dell'oggetto;
                    \item Model::DBSynch::ConcreteObserver <- i metodi di set della classe invocano il metodo notify() dell'observer contenuto nel package DBSynch di cui l'oggetto della classe tiene un riferimento.
				\end{itemize}	
                \textbf{\interfacce}: Gli oggetti della classe SVG vengono istanziati da Model::SlideShow::SlideShowActions::InsertEditRemove::Inserter e da questi inseriti nel campo dati contenitore presentazione all’interno di \\Model::MongoRelations::Loader::LoaderClass.\\
                \textbf{\base}: 
                    \begin{itemize}
                    \item Model::SlideShow::SlideShowElement.
                    \end{itemize}
                    }
                    \subsubsubsection{Audio}{
				\textbf{\tipo}: Gli oggetti della classe Audio rappresentano gli elementi di tipo audio della presentazione.\\
				\textbf{\relaz}: 
				\begin{itemize}
					\item Model::SlideShow::SlideShowActions::InsertEditRemove::Inserter -> invoca il costruttore di Audio e inserisce l’oggetto nel campo dati contenitore all’interno dell’oggetto della classe\\ Model::MongoRelations::Loader::Caricatore;
                    \item Model::SlideShow::SlideShowActions::InsertEditRemove::Remover -> rimuove l’oggetto Audio dal campo dati presentazione all’interno di Model::MongoRelations::Loader::LoaderClass, ne invoca quindi il distruttore;
                    \item Model::SlideShow::SlideShowActions::InsertEditRemove::Editor -> invoca i metodi che modificano i campi dati dell'oggetto;
                    \item Model::DBSynch::ConcreteObserver <- i metodi di set della classe invocano il metodo notify() dell'observer contenuto nel package DBSynch di cui l'oggetto della classe tiene un riferimento.
				\end{itemize}	
                \textbf{\interfacce}: Gli oggetti della classe Audio vengono istanziati da Model::SlideShow::SlideShowActions::InsertEditRemove::Inserter e da questi inseriti nel campo dati contenitore presentazione all’interno di\\ Model::MongoRelations::Loader::LoaderClass.\\
                \textbf{\base}:  
                    \begin{itemize}
                    \item Model::SlideShow::SlideShowElements::SlideShowElement.
                    \end{itemize}
                    }
                    \subsubsubsection{Video}{
				\textbf{\tipo}: Gli oggetti della classe Video rappresentano gli elementi di tipo video della presentazione.\\
				\textbf{\relaz}: 
				\begin{itemize}
					\item Model::SlideShow::SlideShowActions::InsertEditRemove::Inserter -> invoca il costruttore di Video e inserisce l’oggetto nel campo dati contenitore all’interno dell’oggetto della classe Model::MongoRelations::Loader::Caricatore;
                    \item Model::SlideShow::SlideShowActions::InsertEditRemove::Remover -> rimuove l’oggetto Video dal campo dati presentazione all’interno di Model::MongoRelations::Loader::LoaderClass, ne invoca quindi il distruttore;
                     \item Model::SlideShow::SlideShowActions::InsertEditRemove::Editor -> invoca i metodi che modificano i campi dati dell'oggetto;
                    \item Model::DBSynch::ConcreteObserver <- i metodi di set della classe invocano il metodo notify() dell'observer contenuto nel package DBSynch di cui l'oggetto della classe tiene un riferimento.
				\end{itemize}	
                \textbf{\interfacce}: Gli oggetti della classe Video vengono istanziati da Model::SlideShow::SlideShowActions::InsertEditRemove::Inserter e da questi inseriti nel campo dati contenitore presentazione all’interno di \\Model::MongoRelations::Loader::LoaderClass.\\
                \textbf{\base}: 
                    \begin{itemize}
                    \item Model::SlideShow::SlideShowElements::SlideShowElement.
                    \end{itemize}
                    }     
                 \subsubsection{Model::SlideShow::Background}{
                				\textbf{\tipo}: Gli oggetti della classe Background rappresentano lo sfondo della presentazione.\\
                				\textbf{\relaz}: 
                				\begin{itemize}
                					\item Model::SlideShow::SlideShowActions::InsertEditRemove::Inserter -> invoca il costruttore di Background e inserisce l’oggetto nel campo dati contenitore all’interno dell’oggetto della classe Model::MongoRelations::Loader::Caricatore;
                                    \item Model::SlideShow::SlideShowActions::InsertEditRemove::Remover -> rimuove l’oggetto Video dal campo dati presentazione all’interno di Model::MongoRelations::Loader::LoaderClass, ne invoca quindi il distruttore;
                                    \item Model::SlideShow::SlideShowActions::InsertEditRemove::Editor -> invoca i metodi che modificano i campi dati dell'oggetto;
                    \item Model::DBSynch::ConcreteObserver <- i metodi di set della classe invocano il metodo notify() dell'observer contenuto nel package DBSynch di cui l'oggetto della classe tiene un riferimento.
                				\end{itemize}	
                                \textbf{\interfacce}: Gli oggetti della classe Background vengono istanziati da Model::SlideShow::SlideShowActions::InsertEditRemove::Inserter e da questi inseriti nel campo dati contenitore presentazione all’interno di\\ Model::MongoRelations::Loader::LoaderClass.\\
                                \textbf{\base}: 
                                    \begin{itemize}
                                    \item Model::SlideShow::SlideShowElements::SlideShowElement.
                                    \end{itemize}
                                    }              
}

\subsubsection{Model::MongoRelations}{
	%\begin{figure}[H]
	%	\centering
	%	\includegraphics[scale=0.6]{\imgs {MongoRelations}.pdf}
	%	\label{fig:sr}
	%	\caption{MongoRelations}
	%\end{figure}
		\textbf{\tipo}: package, racchiude le funzionalità del sistema che interagiscono con i servizi di interazione con il database MongoDB esposti dall' interfaccia nodeApi.\\
		\textbf{\relaz}:
		\begin{itemize}
		\item relazioni verso \textbf{nodeApi} del quale si utilizzano i servizi dalla interfaccia;
		\item relazioni da \textbf{Presenter} per il recupero o la creazione di una presentazione dal database MongoDB;
		\item relazioni da \textbf{Model::SlideShow} che utilizza la rappresentazione locale della presentazione e i servizi per l'interazione a nodeApi.
		\end{itemize} 
}

\subsubsection{Model::MongoRelations::Loader}{
	\begin{figure}[H]
		\centering
		\includegraphics[scale=0.4]{\imgs {Loader}.pdf}
		\label{fig:srl}
		\caption{MongoRelationsLoader}
	\end{figure}
		\textbf{\tipo}: package, racchiude le funzioni di recupero o creazione di una presentazione dal server attraverso i servizi nodeApi, una volta ottenuta la presentazione e' esposta per le modifiche provenienti da altri package nel Model 
		\textbf{\relaz}: 
		\begin{itemize}
		\item relazione verso \textbf{nodeApi} per il recupero della presentazione dal database MongoDB
		\item relazione da \textbf{Model::SlideShow::InsertEditRemove} a cui viene esposta la rappresentazione della presentazione locale per essere modificata
		\end{itemize}

\subsubsubsection{LoaderClass}{
				\textbf{\tipo}: Classe la cui funzione è recuperare una presentazione dal database remoto ed esporla alle modifiche da parte di altri metodi definiti nel model, creare una nuova presentazione\\	
				\textbf{\relaz}: 
				\begin{itemize}
				\item relazione verso \textbf{nodeApi} per il recupero della presentazione dal database MongoDB
				\item relazione da \textbf{Model::SlideShow::InsertEditRemove} a cui viene esposta la rappresentazione della presentazione locale per essere modificata
				\end{itemize}	
                    }
}

\subsubsection{Model::MongoRelations::AccessControl}{

	\begin{figure}[H]
		\centering
		\includegraphics[scale=0.5]{\imgs {accessControll}.pdf}
		\label{fig:cd}
		\caption{DbSync}
	\end{figure}

		\textbf{\tipo}: package, racchiude le funzioni di registrazione dell’utente e autenticazione tramite token ai servizi esposti da nodeApi.\\
		\textbf{\relaz}: 
		\begin{itemize}
		\item relazioni verso \textbf{nodeApi} a cui vengono passati i parametri per la registrazione e l'autenticazione dell'utente
		\item relazioni da \textbf{Presenter} da cui si ricevono i parametri in input dell'utente
		\item relazioni da \textbf{DBSynch} a cui viene esposto il token ricevuto da nodeApi dopo la autenticazione
		\end{itemize}
		
        \subsubsubsection{Autenticazione}{
				\textbf{\tipo}: Classe, fornisce le funzionalità di autenticazione e deautenticazione.\\	
				\textbf{\relaz}: 
				\begin{itemize}
					\item relazione verso \textbf{nodeApi} per il recupero del token passando i parametri di autenticazione dell'utente
                    			\item relazione da \textbf{Presenter} da cui riceve in input i parametri dell'utente per la autenticazione 
					\item relazione da \textbf{Model::MongoRelations::DBSynch} a cui espone il token per poter usare i servizi di nodeApi
				\end{itemize}	
                    }
        \subsubsubsection{Registrazione}{
				\textbf{\tipo}: Classe, fornisce le funzionalità di registrazione.\\	
				\textbf{\relaz}: 
				\begin{itemize}
					\item relazione verso \textbf{nodeApi} per il la registrazione dell'utente presso il database MongoDB
                    			\item relazione da \textbf{Presenter} da cui riceve in input i parametri dell'utente per la registrazione
				\end{itemize}	
            }
}

\subsubsection{Model::MongoRelations::DBSynch}{

	\begin{figure}[H]
		\centering
		\includegraphics[scale=0.6]{\imgs {ClassDiagram1}.pdf}
		\label{fig:cd}
		\caption{DbSync}
	\end{figure}
	
		\textbf{\tipo}: il package ha lo scopo di raccogliere le funzionalità di aggiornamento delle presentazioni in remoto tramite un pattern Observer e chiamate asincrone ai servizi di nodeApi\\
		\textbf{\relaz}:
		\begin{itemize}
		\item relazioni da e verso \textbf{Model::SlideShow::SlideShowElements} in cui si detiene un riferimento nei confronti dei ConcreteSubjects e viceversa
		\item relazioni verso \textbf{Model::MongoRelations::AccessControll} da cui riceve il token per accedere ai servizi nodeApi
		\item relazioni verso \textbf{nodeApi} verso cui inoltra le chiamate di update degli elementi della presentazione modificata
		\end{itemize}
       
       \subsubsubsection{Observer}{
				\textbf{\tipo}: Interfaccia, espone il metodo update() che una volta recuperato l'elemento modificato chiama il metodo in nodeApi per l'aggiornamento della presentazione nel database MongoDB\\	
				\textbf{\relaz}: 
				\begin{itemize}
					\item associazione con Subject che detiene un riferimento dell'Observer
					\item interfaccia realizzata da ConcreteObserver che definisce il metodo update()
				\end{itemize}	
            }
            
            \subsubsubsection{ConcreteObserver}{
				\textbf{\tipo}: Classe, concretizza l’interfaccia Observer\\	
				\textbf{\relaz}: 
				\begin{itemize}
					\item realizza l’interfaccia Observer definendone il metodo update()
					\item detiene un riferimento a Subject
					\item relazione verso nodeApi di cui ne chiama i servizi per l'aggiornamento di una presentazione
				\end{itemize}	
            }
            
             \subsubsubsection{Subject}{
				\textbf{\tipo}: Classe astratta, definisce una classe astratta per i diversi tipi di Subject a seconda degli elementi da osservare. Definisce i metodi attach(Observer), detach(Observer) e notify() per aggiungere, togliere e notificare il ConcreteObserver associato.\\	
				\textbf{\relaz}: 
				\begin{itemize}
					\item associazione da ConcreteObserver che ne detiene un riferimento; 
					\item realizzata dalle classi: SubjectAudio, SubjectVideo, SubjectText, SubjectFrame, SubjectSvg, SubjectImg che definiscono il metodo getElement() utilizzato da ConcreteObserver per ottenere l’oggetto modificato.
				\end{itemize}	
            }
            
             \subsubsubsection{SubjectAudio}{
				\textbf{\tipo}: Classe, fornisce un’implementazione di Subject permettendo di applicare il pattern "Observer" e mantenere gli oggetti le cui modifiche sono da notificare in un altro package.\\	
				\textbf{\relaz}: 
				\begin{itemize}
					\item implementa Subject definendo il metodo getElement(), associazione da e verso la classe Model::SlideShow::SlideShowElements::Audio di cui detiene un riferimento.
				\end{itemize}	
            }
            
            \subsubsubsection{SubjectVideo}{
				\textbf{\tipo}: Classe, fornisce un’implementazione di Subject permettendo di applicare il pattern "Observer" e mantenere gli oggetti le cui modifiche sono da notificare in un altro package.\\	
				\textbf{\relaz}: 
				\begin{itemize}
					\item implementa Subject definendo il metodo getElement(), associazione da e verso la classe Model::SlideShow::SlideShowElements::Video di cui detiene un riferimento.
				\end{itemize}	
            }
            
            \subsubsubsection{SubjectText}{
				\textbf{\tipo}: Classe, fornisce un’implementazione di Subject permettendo di applicare il pattern "Observer" e mantenere gli oggetti le cui modifiche sono da notificare in un altro package.\\	
				\textbf{\relaz}: 
				\begin{itemize}
					\item implementa Subject definendo il metodo getElement(), associazione da e verso la classe Model::SlideShow::SlideShowElements::Text di cui detiene un riferimento.
				\end{itemize}	
            }
            
            \subsubsubsection{SubjectFrame}{
				\textbf{\tipo}: Classe, fornisce un’implementazione di Subject permettendo di applicare il pattern "Observer" e mantenere gli oggetti le cui modifiche sono da notificare in un altro package.\\	
				\textbf{\relaz}: 
				\begin{itemize}
					\item implementa Subject definendo il metodo getElement(), associazione da e verso la classe Model::SlideShow::SlideShowElements::Frame di cui detiene un riferimento.
				\end{itemize}	
            }
            
            \subsubsubsection{SubjectImg}{
				\textbf{\tipo}: Classe, fornisce un’implementazione di Subject permettendo di applicare il pattern "Observer" e mantenere gli oggetti le cui modifiche sono da notificare in un altro package.\\	
				\textbf{\relaz}: 
				\begin{itemize}
					\item implementa Subject definendo il metodo getElement(), associazione da e verso la classe Model::SlideShow::SlideShowElements::Image di cui detiene un riferimento.
				\end{itemize}	
            }
            
            \subsubsubsection{SubjectSVG}{
				\textbf{\tipo}: Classe, fornisce un’implementazione di Subject permettendo di applicare il pattern "Observer" e mantenere gli oggetti le cui modifiche sono da notificare in un altro package.\\	
				\textbf{\relaz}: 
				\begin{itemize}
					\item implementa Subject definendo il metodo getElement(), associazione da e verso la classe Model::SlideShow::SlideShowElements::SVG di cui detiene un riferimento.
				\end{itemize}	
            }
            
             \subsubsubsection{SubjectBackground}{
				\textbf{\tipo}: Classe, fornisce un’implementazione di Subject permettendo di applicare il pattern "Observer" e mantenere gli oggetti le cui modifiche sono da notificare in un altro package.\\		
				\textbf{\relaz}: 
				\begin{itemize}
					\item implementa Subject definendo il metodo getElement(), associazione da e verso la classe Model::SlideShow::SlideShowElements::Background di cui detiene un riferimento.
				\end{itemize}	
            }

}

\subsubsection{Model::ApacheRelations}{
		\textbf{\tipo}: Compito di questo package è di gestire l'interazione con il server Apache.\\
		\begin{itemize}
			\item Il package comunica con la view ricevendo chiamate da View::Pages::Profile.
			\item I componenti del package ApacheRelations hanno relazioni di dipendenza nei confronti del package nodeApi del quale utilizzano i servizi esposti dall’interfaccia; c’è dipenda tra il package ApacheRelations ed il package MongoRelations.\\
		\end{itemize}

\subsubsection{Model::ApacheRelations::ApacheServerManager}{
	Il componente seguente appartiene al package ApacheServerManager, quindi lo scope sarà Model::ApacheRelations::ApacheServerManager::<componente>.
		\textbf{\tipo}: Compito di questo package è di gestire l'interazione con il server Apache per le operazioni di modifica dei files.\\
		\textbf{\relaz}:
		\begin{itemize}
			\item Il package comunica con la view ricevendo chiamate da View::Pages::Profile.
		\end{itemize}

			\subsubsubsection{FileManager}{
			\textbf{\tipo}: Lo scopo di questa classe è di gestire le chiamate della pagina View::Pages::Profile per l'inserimento, cancellazione e rinominazione di file sul server Apache.\\
			\textbf{\relaz}:
			\begin{itemize}
			 	\item View::Pages::Profile:UploadMedia -> costruisce ApacheManager, ne invoca i metodi passando i parametri dell'utente ed i parametri del file da caricare; 
			 	\item View::Pages::Profile::DeleteMedia -> costruisce ApacheManager, ne invoca i metodi passando i parametri dell'utente ed i e l'id del file media da eliminare;
			 	\item View::Pages::Profile::RenameMedia -> costruisce ApacheManager, ne invoca i metodi passando i parametri dell'utente, l'id e il nuovo nome del file media da rinominare; 
			 	\item Model::Caricamento::Uploader <- ApacheManager passa i parametri di caricamento ad Uploader che istanzia l'oggetto sul server.
			 	\item Presenter::EditPresenter <- ApacheManager passa lo username dell'utente che sta svolgendo operazioni sul file, il file ed il tipo del file al server Apache; questo se l'operazione è andata a buon fine, ritorna un segnale ad ApacheManager, che lo trasmette ad EditPresenter.
			\end{itemize}
			\textbf{\interfacce}: La pagina Profile costruisce FileManager per fare modifiche ai file. 
			}}
			\subsubsection{Premi::ApacheRelations::ResourceGetter}{
							La seguente classe appartiene al package ApacheRelations, quindi lo scope sarà \\Model::ApacheRelations::<componente>.
						   	\textbf{\tipo}: Questo package ha lo scopo di rendere disponibili le presentazioni in locale tramite chiamate a funzioni o servizi del server Apache.    \\
						   	\textbf{\relaz}:
						   	\begin{itemize}
						   		\item dipendenza con Premi::Model::MongoRelations::Loader::LoaderClass <- ResourceGetter invoca i metodi forniti da LoaderClass che restituiscono ;
						   		\item definisce il metodo getResources();
						   	\end{itemize}
	}
}