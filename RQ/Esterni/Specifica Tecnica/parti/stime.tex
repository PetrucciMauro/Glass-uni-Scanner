\section{Stime di fattibilità e di bisogno di risorse}{
	L'architettura definita precedentemente ha raggiunto un livello di dettaglio sufficiente per fornire una stima sulla fattibilità e di bisogno di risorse. L'analisi dell'architettura progettata ha permesso di constatare che le tecnologie che si è scelto di adottare risultano sufficientemente adeguate per la realizzazione del prodotto e riescono a ricoprire le esigenze progettuali.\\
	Poiché tutti gli strumenti da utilizzare nello sviluppo sono gratuiti, il bisogno di
	risorse non si dimostra essere particolarmente problematico.\\
	Si è deciso di utilizzare HTML5, CSS3 e Javascript (e le sue librerie) per lo sviluppo della parte web.\\
	Per la parte di database si è scelto l'utilizzo di MEAN e delle librerie Express.js e Node.js per una migliore interazione con MongoDB.\\
	Per la parte di esecuzione delle presentazioni è stato scelto Impress.js, framework che permette l'esecuzione in maniera non lineare come richiesto.\\
	Per la parte di modifica delle presentazioni verrà utilizzato il framework Angular.js per lo spostamento in tempo reale degli elementi delle presentazioni.
	Per l'esposizione verrà utilizzato un server apache per la esposizione online dell'applicazione e l'utilizzo della tecnologia HTML5 Manifest che semplifica la gestione delle presentazioni offline.
	}