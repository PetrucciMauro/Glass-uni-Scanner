\section{Introduzione}
\subsection{Scopo del documento}
Il presente documento ha lo scopo di definire la progettazione ad alto livello del Progetto\ped{g} Premi.\\
Verrà presentata l'architettura generale secondo la quale saranno organizzate le varie componenti Software\ped{g} e saranno descritti i Design Pattern utilizzati.
\subsection{Scopo del Prodotto}
Lo scopo del Progetto\ped{g} è la realizzazione un Software\ped{g} per la creazione ed esecuzione di presentazioni multimediali favorendo l’uso di tecniche di storytelling e visualizzazione non lineare dei contenuti.
\subsection{Glossario}
Al fine di evitare ogni ambiguità di linguaggio e massimizzare la comprensione dei documenti, i termini tecnici, di dominio, gli acronimi e le parole che necessitano di essere chiarite, sono riportate nel documento \href{run:../../Esterni/\fGlossario}{\fEscapeGlossario}. Ogni occorrenza di vocaboli presenti nel Glossario è marcata da una “g” minuscola in pedice.
\subsection{Riferimenti}

\subsubsection{Normativi}
\begin{itemize}
\item Capitolato d’appalto C4: \premi: Software\ped{g} di presentazione “better than Prezi” \\
\url{http://www.math.unipd.it/~tullio/IS-1/2014/Progetto/C4.pdf};
\item Norme di Progetto\ped{g}: \href{run:../../Interni/\fNormeDiProgetto}{\fEscapeNormeDiProgetto};
\item Analisi dei Requisiti\ped{g}: \href{run:../../Interni/\fAnalisiDeiRequisiti}{\fEscapeAnalisiDeiRequisiti};
\item Piano di qualifica: \href{run:../../Interni/\fPianoDiQualifica}{\fEscapePianoDiQualifica};
\item Piano di Progetto\ped{g}: \href{run:../../Interni/\fPianoDiProgetto}{\fEscapePianoDiProgetto}.
\end{itemize}
\subsubsection{Informativi}
\begin{itemize}
\item \textbf{Design Patterns: Elements of Reusable Object-Oriented Software\ped{g}, Addison Wesley, 1995};
\item Descrizione dei Design Pattern\\
\url{http://sourcemaking.com/design_patterns};
\item Ingegneria del Software\ped{g} - Ian Sommerville - 9a Edizione (2010):
\item Slide del docente per l'anno accademico 2014/2015 reperibili al sito \\
\url{http://www.math.unipd.it/~tullio/IS-1/2014/};
\item MEAN: \url{http://www.mean.io/}; \textbf{MEAN Web Development, Amos Q. Haviv, 2014}; 
\item MongoDB: \url{http://docs.mongodb.org/manual/};
\item Angular.js: \url{https://docs.angularjs.org/tutorial};
\item Express.js: \url{http://expressjs.com/};
\item Node.js: \url{https://nodejs.org/documentation/};
\item jQuery: \url{http://api.jquery.com/} ;
\item Impress.js: \url{https://github.com/bartaz/impress.js/}.

\end{itemize}
