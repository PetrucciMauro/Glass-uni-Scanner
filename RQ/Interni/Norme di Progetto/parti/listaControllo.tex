\appendix



\section{Lista di controllo}
\label{sec:ListaControllo}
Durante l'applicazione del Walkthrough ai documenti, sono state riportate le tipologie
di errori più frequenti. La lista di controllo risultante è la seguente:
\begin{itemize}

\item \textbf{Norme stilistiche}:
\begin{itemize}
\item Elenco puntato: non inizia con la lettera maiuscola;
\item Elenco puntato: non termina con il punto e virgola oppure con il punto se è
l’ultimo Elemento\ped{g};
\item Elenco numerato: non termina con il punto e virgola oppure con il punto se
è l’ultimo Elemento\ped{g};
\item Nome proprio di persona: non rispetta la norma Cognome Nome;
\item Parole Proponente\ped{g} e Committente\ped{g}: non vengono scritte con la maiuscola
iniziale.
\end{itemize}
\item \textbf{Italiano}:
\begin{itemize}

\item Periodi: frasi troppo lunghe rendono i concetti di difficile comprensione;
\item Doppie negazioni: evitare l’utilizzo di doppie negazioni perché complicano la
comprensione della frase;
\item Punto e virgola: evitare l’uso del punto e virgola quando è necessario usare
il punto;
\item Proponente\ped{g} e Committente\ped{g}: non si deve confondere il loro significato.
\end{itemize}
\item \textbf{\LaTeX}:
\begin{itemize}
\item Lettere accentate nelle variabili: non viene utilizzato il comando apposito;
\item Carattere di spaziatura: non deve essere utilizzato all’interno dei Tag\ped{g};
\item Macro \LaTeX: non viene scritta usando l'apposito comando.
\end{itemize}
\item \textbf{UML}:
\begin{itemize}
\item Il sistema non deve mai essere un attore;
\item Controllo ortografico: deve essere effettuato in modo dettagliato a causa dell'impossibilità di automatizzare i controlli sui diagrammi;
\item Direzione delle frecce non corrette;
\item Consistenza della nomenclatura tra i diagrammi e le descrizioni testuali nei documenti.
\end{itemize}
La seguente lista di controllo vuole riassumere invece gli errori più frequenti rilevati
durante il Walkthrough del tracciamento Requisiti\ped{g} :

\item \textbf{Tracciamento requisiti}:
\begin{itemize}

\item Ad ogni caso d’uso deve corrispondere almeno un Requisito\ped{g};
\item Ad ogni Requisito\ped{g} deve corrispondere almeno una fonte;
\item La fonte “Capitolato” non deve comparire nei Requisiti\ped{g} interni;
\item Deve esserci copertura totale del capitolato nei Requisiti\ped{g}.

\end{itemize}
\end{itemize}
