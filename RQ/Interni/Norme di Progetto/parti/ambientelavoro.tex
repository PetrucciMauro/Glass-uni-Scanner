\section{Ambiente di lavoro} 
\subsection{Risorse}{
\subsubsection{Risorse\ped{g} necessarie:}{
	\subsubsubsection{Risorse\ped{g} umane}{
		I ruoli necessari a garantire la qualità del prodotto sono:
		\begin{itemize}
			\item Responsabile di Progetto\ped{g};
			\item Amministratore;
			\item Verificatore;
			\item Programmatore. 
		\end{itemize}
	}
	\subsubsubsection{Risorse\ped{g} Hardware}{
		Saranno necessari:
		\begin{itemize}
			\item Computer con installato Software\ped{g} necessario allo sviluppo del Progetto\ped{g} in tutte le sue fasi;
			\item Luoghi in cui svolgere riunioni, preferibilmente dotati di connessione ad Internet.
		\end{itemize}
	}
	\subsubsubsection{Risorse\ped{g} software}{
		Saranno necessari:
		\begin{itemize}
			\item Strumenti per automatizzare i test;
			\item Framework\ped{g} per eseguire test di unità;
			\item Piattaforma di versionamento per la creazione e gestione di Ticket\ped{g};
			\item Debugger per i linguaggi di programmazione scelti;
			\item Browser\ped{g} come piattaforma di testing dell'applicazione da sviluppare;
			\item Strumenti per effettuare l'analisi statica\ped{g} del Codice\ped{g} per misurare le metriche\ped{g}.
		\end{itemize}
	}
}
\subsubsection{Risorse\ped{g} disponibili}{
	Sono disponibili:
	\begin{itemize}
		\item Computer personali dei membri del gruppo;
		\item Computer presenti nelle aule informatiche del Dipartimento di Matematica;
		\item Aule disponibili per incontri nel Dipartimento di Matematica;
		\item Un dispositivo Raspberry Pi 2 Model B, utilizzato come Server\ped{g} per programmi\ped{g} organizzativi e di testing.
	\end{itemize}
	\subsubsubsection{Risorse\ped{g} software}{
	    \begin{itemize}
	     			\item Strumenti per il coordinamento \S\ref{sec:strumentiCoordinamento};
	     			\item Strumenti per i documenti \S\ref{sec:strumentiDocumenti};
	     			\item Strumenti per la Codifica\ped{g} \S\ref{sec:strumentiCodifica};
	     			\item Strumenti verifica \S\ref{sec:strumentiVerifica}.
	    \end{itemize}

	}
}
}

\subsection{Sistemi Operativi}

L’intero sviluppo del Progetto\ped{g} viene svolto in ambienti Unix-Like e Windows\ped{g}, nello specifico, Ubuntu\ped{g}, Mac, Windows\ped{g} . Tale scelta è maturata dopo aver appurato che le tecnologie utilizzate per lo sviluppo del Progetto\ped{g} sono indipendenti dall’ambiente di sviluppo e di impiego.

