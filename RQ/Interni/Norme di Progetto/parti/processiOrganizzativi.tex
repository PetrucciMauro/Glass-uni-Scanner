\section{Processi\ped{g} Organizzativi}
\subsection{Applicazione PDCA}
Per poter garantire un costante miglioramento dei Processi\ped{g} il gruppo \gruppo\ adotterà il PDCA con le seguenti modalità:\\\\
\textbf{Plan - Pianificazione}\\
Individuazione delle linee guida che permettono di risolvere un problema o di migliorare un Processo\ped{g} attraverso la successiva pianificazione delle azioni da svolgere:
\begin{itemize}
\item Definizione degli obiettivi;
\item Raccolta dei dati;
\item Determinazione degli interventi necessari;
\item Determinazione dei risultati attesi;
\item Definizione delle responsabilità per la fase di attuazione;
\item Pianificazione delle azioni da svolgere.
\end{itemize}
\textbf{Do - Applicazione}\\
Preparazione degli interventi definiti e implementazione delle procedure per attuarli. Applicazione delle azioni programmate e verifica del funzionamento del piano.
\begin{itemize}
\item Applicazione delle azioni e procedure;
\item Implementazione soluzioni per un periodo di prova;
\item Verifica dell'adeguatezza delle soluzioni adottate rispetto agli obiettivi attesi;
\item Formazione interna sulle nuove modalità operative a fronte delle soluzioni adottate.\\
\end{itemize}
\textbf{Check - Controllo e verifica dei risultati}\\
Verificare della corretta esecuzione delle azioni nei tempi previsti e con gli obiettivi prefissati.
Se si è raggiunto l’obiettivo definito, si procede alla fase successiva, altrimenti è necessario ripetere un nuovo ciclo PDCA sullo stesso problema analizzandone criticamente le varie fasi.
In seguito alla prima applicazione delle soluzioni progettate, queste vengono sottoposte ad un controllo nel tempo per verificare la sostenibilità di quanto realizzato ed risolvere eventuali punti di criticità o anomalie se queste persistono.\\\\
\textbf{Act - Implementazione delle soluzioni}\\
Una volta che le soluzioni adottate hanno dimostrato di funzionare, è opportuno procedere a:
\begin{itemize}
\item Standardizzare il miglioramento ottenuto applicandolo in via definitiva;
\item Individuare eventuali esigenze di studio da parte dei componenti del gruppo per rendere operative le soluzioni adottate;
\item Continuare a monitorare la situazione ripetendo il ciclo più volte fino a raggiungere i miglioramenti desiderati;
\item Individuare altre opportunità di miglioramento.
\end{itemize}
\subsection{Gestione di progetto}
Le responsabilità di gestione dell’intero Progetto\ped{g}, dalla nascita alla conclusione, sono da
attribuire al \emph{Responsabile di Progetto}.
Quest’ultimo dovrà garantire un corretto sviluppo delle attività utilizzando, qualora sia possibile, degli strumenti che gli consentano di:
\begin{itemize}
\item Pianificare, coordinare e controllare le attività;
\item Gestire e controllare le Risorse\ped{g};
\item Analizzare e gestire i rischi;
\item Elaborare i dati.
\end{itemize}
\subsubsection{Pianificazione delle attività}
Per pianificare le attività il \emph{Responsabile di Progetto} deve realizzare un diagramma di
Gantt  per ciascuna fase indicata nella sezione Pianificazione del \href{run:../../Esterni/\fPianoDiProgetto}{\fEscapePianoDiProgetto} , utilizzando Redmine come descritto nella sezione \S\ref{sec:Pianificazione}.
\subsubsection{Coordinazione e controllo delle attività}
Per coordinare e controllare le attività il \emph{Responsabile} di Progetto\ped{g} deve riportare la struttura creata con Redmine sfruttando il suo sistema di Ticketing\ped{g}
come descritto nella sezione \S\ref{sec:protocolloSviluppo}.
In questo modo ciascun componente del gruppo sarà avvisato delle attività ad esso assegnate e potrà inserire lo stato delle stesse permettendo al \emph{Responsabile} di verificare immediatamente l’avanzamento del Progetto\ped{g}.

\subsubsection{Gestione e controllo delle risorse}
Per gestire e controllare le Risorse\ped{g} il \emph{Responsabile} di Progetto\ped{g} deve utilizzare Redmine come indicato nella sezione \S\ref{sec:protocolloSviluppo} che gli consente anche di verificare l’avanzamento di ogni Processo\ped{g} come riportato nel \href{run:../../Esterni/\fPianoDiProgetto}{\fEscapePianoDiProgetto} .
\subsubsection{Analisi e Gestione dei rischi}
Durante l’avanzamento del Progetto\ped{g} il \emph{Responsabile di Progetto} deve monitorare costantemente il verificarsi dei rischi descritti nel \href{run:../../Esterni/\fPianoDiProgetto}{\fEscapePianoDiProgetto} ed eventuali nuovi rischi, attuando le contromisure descritte e riportando gli effettivi riscontri.

\subsubsection{Elaborazione dati}
Il \emph{Responsabile di Progetto} deve sfruttare i fogli di calcolo\ped{g}, come descritto nella sezione \S\ref{sec:fogliDiCalcolo}, per elaborare i dati raccolti durante lo sviluppo del Progetto\ped{g} e riportarli nel \emph{Piano di Progetto}.

\subsubsection{Delega}
\label{sec:delega}
Il \emph{Responsabile di Progetto}, nel caso in cui abbia redatto una parte di un documento, può delegare l'approvazione di tale documento ad un \emph{Verificatore}.
\subsubsection{Responsabilità di sotto-progetto}
Ogni macro-attività può essere assegnata dal \emph{Responsabile} ad un responsabile di sotto-Progetto\ped{g}, i cui compiti saranno l’assegnazione delle singole attività alle Risorse\ped{g} rese disponibili e la gestione dei cambiamenti.
\subsubsubsection{Assegnazione attività}
Per assegnare attività alle Risorse\ped{g} disponibili, il responsabile di sotto-Progetto\ped{g} dovrà seguire le procedure di Ticketing\ped{g} descritte in \S\ref{sec:realizzazioneControllo}.
\subsubsubsection{Gestione dei cambiamenti}
In caso di errori, in seguito alla notifica da parte del \emph{Verificatore} tramite Ticket\ped{g}, il responsabile di sotto-Progetto\ped{g} dovrà assegnare la correzione mediante la procedura di Ticketing\ped{g} descritta in \S\ref{sec:realizzazioneControllo}. Al termine della correzione, sarà compito del responsabile di sotto-Progetto\ped{g} accettare o respingere la modifica, e richiederne eventualmente il rifacimento.
Nel caso la correzione riguardi un’attività di Codifica\ped{g}, sarà compito del responsabile di sotto-Progetto\ped{g} programmare una nuova esecuzione dei test di unità e di integrazione correlati al modulo modificato.

\subsection{Collaborazione}{
\subsubsection{Comunicazioni}
	\subsubsubsection{Comunicazioni interne}{
		Per le comunicazioni interne è stato aperto un gruppo privato su Facebook accessibile ai singoli membri del team. \begin{center}
			\url{https://www.facebook.com/groups/1709354699290988}
		\end{center} 
		Inoltre ogni membro del team dovrà annotare i propri impegni sullo strumento Google Calendar, il quale verrà utilizzato per segnare qualsiasi tipo di impegno: di gruppo e individuale.\\\\
		In caso di comunicazioni vocali o videoconferenze verrà utilizzato Skype.
		
	 }
	\subsubsubsection{Comunicazioni esterne}{
	Per quanto riguarda le comunicazioni esterne (verso Committente\ped{g} e/o Proponente\ped{g}) è stata creata una casella di posta elettronica dedicata, gestita dal \emph{Responsabile di progetto}: \begin{center}
		\href{mailto:\mail}{\mail} \end{center} è compito del \emph{Responsabile} gestire le informazioni in entrata e in uscita avvisando il proprio gruppo e il Committente/Proponente\ped{g} di eventuali comunicazioni rispettivamente in entrata e in uscita.
		}
}

\subsubsection{Riunioni}
	\subsubsubsection{Interne}{
		\begin{itemize}
			\item Ogni membro del gruppo può richiedere una riunione interna tramite un post all’interno del gruppo su Facebook (tramite l’uso del Tag\ped{g} [Richiesta Riunione Interna $x$] con $x$ numero incrementato di 1 rispetto alla richiesta precedente). Questa richiesta  in base alle risposte degli altri componenti verrà presa in esame dal \emph{Responsabile};
			\item Una volta valutate le motivazioni della richiesta il \emph{Responsabile} controlla sul calendario del gruppo le disponibilità dei vari componenti;
			\item Il \emph{Responsabile}, entro 1 giorno lavorativo pubblica una nuova discussione con Tag\ped{g} [Esito Richiesta Interna x], dove annuncia in caso positivo l’orario ed il luogo della riunione, altrimenti annulla la riunione oppure rimanda la richiesta al successivo incontro:;
			\item Nel caso in cui, per diversi motivi, alla riunione non potessero presenziare più di due membri, si procede a fissare una nuova riunione (vedi punto 2 e seguenti).
		\end{itemize}
		\subsubsubsection{Casi Particolari}{
			Per le richieste di riunioni interne vicine (5 giorni lavorativi) ad una Milestone\ped{g}, se approvate dal \emph{Responsabile}, verranno indette il giorno stesso o il seguente.
		}
	}
	\subsubsubsection{Esterne}{
		Per le riunioni esterne (quindi gli incontri con il Proponente/Committente\ped{g}) la prassi è la medesima delle riunioni interne; può essere avanzata da qualsiasi membro del gruppo con il Tag\ped{g} [Richiesta Riunione esterna $x$].
		In questo caso il \emph{Responsabile} avrà il duplice compito di valutare la richiesta dopo aver consultato il calendario e di contattare  il Committente\ped{g}, per accordarsi su tempi e luogo dell’incontro, che verranno poi riferiti sulla piattaforma di comunicazioni interne tramite il Tag\ped{g} [Esito Richiesta Riunione Esterna $x$].
		}
	\subsubsubsection{Esito}{
		Ad ogni riunione (sia interna che esterna) il \emph{Responsabile} ha il dovere di assicurarsi che venga redatto un verbale che riassuma gli argomenti trattati durante l’incontro e tutte le eventuali decisioni prese; i membri del gruppo hanno l’obbligo di applicare le eventuali modifiche o correzioni decise durante la riunione ed è del \emph{Responsabile} il dovere che i problemi emersi durante il verbale siano stati risolti.
		}
		
\subsubsection{Repository\ped{g} e strumenti per la condivisione di file}

\subsubsubsection{Repository}
Sono stati creati due Repository\ped{g} Git:
\begin{itemize}


\item documents.git disponibile all’Indirizzo\ped{g}:\\
\begin{center}\url{https://github.com/PetrucciMauro/documents}\\\end{center}
conterrà i sorgenti \LaTeX \ e gli script necessari alla stesura dei documenti;
\item premi.git disponibile all’Indirizzo\ped{g}:\\
\begin{center}
\url{https://github.com/PetrucciMauro/Premi}\\
\end{center}
conterrà i sorgenti dell’applicazione.\\
\end{itemize}
Una volta terminata la fase di lavorazione di un documento, verrà creato un branch di verifica. In questo modo i Verificatori potranno lavorare parallelamente al resto del gruppo ed effettuare il merge  delle loro modifiche, una volta terminato il lavoro di verifica.
Il meccanismo di verifica e approvazione è descritto in dettaglio nella sezione \S\ref{sec:VerificaDocumenti}.


\subsubsubsection{Condivisione file}
Per la condivisione informale di File\ped{g} e per il lavoro collaborativo su documenti di supporto, si usa la piattaforma di condivisione File\ped{g} online Google Drive.
Trattandosi di strumenti informali, non si definiscono procedure rigorose d’uso e se ne lascia la descrizione alle sezioni \S\ref{sec:condivisioneFile}.


\subsection{Strumenti}

\subsubsection{Coordinamento}
\label{sec:strumentiCoordinamento}
è stato predisposto un Server\ped{g} dedicato sul quale sono installate alcune applicazioni WEB\ped{g}
che facilitano la gestione del Progetto\ped{g}. Per connettersi al Server\ped{g}, l'Indirizzo\ped{g} è il seguente:\\
\begin{center}
\url{http://gioberry.no-ip.org/}
\end{center}
\subsubsubsection{Software\ped{g} di gestione del progetto} 
\label{subsec:Software di gestione del prodotto}
Come piattaforma di gestione del Progetto\ped{g} è stato scelto \textbf{Redmine}, che fornisce:
\begin{itemize}
\item Un sistema flessibile di gestione dei Ticket\ped{g};
\item Il grafico Gantt delle attività;
\item Un calendario per organizzare i compiti;
\item La visualizzazione del Repository\ped{g} associato al Progetto\ped{g};
\item Un sistema di rendicontazione del tempo.
\end{itemize}


\subsubsubsection{Versionamento}


Come strumento di versionamento si è deciso di utilizzare \textbf{Git}.
Git è uno strumento di versionamento veloce e di facile apprendimento che
rappresenta uno dei migliori strumenti attualmente esistenti.\\ Per lo sviluppo collaborativo abbiamo deciso di appoggiarci al servizio \textbf{Github} che fornisce non solo un Repository\ped{g} Git, ma anche strumenti utili alla collaborazione fra più persone, come il servizio di \textbf{Ticket}, \textbf{Wiki} e \textbf{Milestone}.\\
Per quanto riguarda l’uso di Git sui computer di sviluppo, si è deciso l’uso
della versione ufficiale rilasciata dal team di sviluppo di Git(2.3.3).\\
Per interfacciarsi con il Repository\ped{g} viene utilizzato \textbf{SmartGit}, un client multi piattaforma che permette di utilizzare Git in maniera rapida.\\
Si descrive ora la procedura di corretto utilizzo del Programma\ped{g} SmartGit .
\begin{itemize}

\item 	\textbf{Clonare il repository}: è possibile clonare il Repository\ped{g} remoto in locale attraverso la seguente procedura:

\begin{itemize}
\item Premere nel menu in alto il pulsante Repository\ped{g} e successivamente Clone;
\item Nel riquadro comparso, inserire il Link\ped{g} del repository\\ \url{https://github.com/PetrucciMauro/documents.git}\\
oppure\\
\url{https://github.com/PetrucciMauro/Premi.git}\\
successivamente premere il pulsante \textbf{next};
\item Nella schermata successiva lasciare spuntati entrambi i box e premere \textbf{next};
\item Selezionare la posizione in cui verrà salvata la versione locale del Repository\ped{g}.
\end{itemize}
\item \textbf{Sincronizzare il repository}: Dalla schermata principale premere il pulsante pull; 
\item \textbf{Salvare una modifica in locale}: Dalla schermata principale premendo il pulsante Commit\ped{g} e inserendo nell'apposita textbox un "Messaggio di commit", si salvano le modifiche effettuate ai File\ped{g};
\item \textbf{Inviare le modifiche al Repository\ped{g} remoto}: Dalla schermata principale premere il pulsante Push e, successivamente alla comparsa del nuovo riquadro, ancora push, ciò comporterà l'invio delle modifiche ai File\ped{g} al Repository\ped{g} remoto.

\end{itemize}

\subsubsubsection{Software\ped{g} di Integrazione Continua}

Si è scelto di adottare \textbf{Travis} per applicare l’integrazione continua allo sviluppo del Progetto\ped{g}.\\ 
Tale Software\ped{g} permette di pianificare ed eseguire dei compiti da eseguire sui File\ped{g} sorgente.\\
Mette inoltre a disposizione un cruscotto su cui è possibile visualizzare lo stato del Codice\ped{g} prodotto. Tale Software\ped{g} è infatti in grado di interagire con il Software\ped{g} di versionamento, e se disponibile con Software\ped{g} per l’esecuzione di test sul Codice\ped{g} prodotto.\\ 
La configurazione di Travis verrà eseguita dall'/emph{amministratore} che si occuperà di scrivere il File\ped{g} .travis.yml , File\ped{g} di configurazione in cui vengono riportati tutti i comandi che Travis dovrà eseguire per il testing del Software\ped{g}.

\subsubsubsection{Condivisione dei file}
  \label{sec:condivisioneFile}
  Si è inoltre scelto di utilizzare degli strumenti online che permettono di condividere File\ped{g}
  in modo semplice e veloce e che consentono di organizzare gli appuntamenti personali
  dei singoli componenti del gruppo.
\subsubsubsection{Google Drive}
  In questa piattaforma di condivisione File\ped{g} verranno salvati i documenti che:
  \begin{itemize}
  
  
  \item Non necessitano di controllo di versione;
  \item Hanno bisogno di grande interattività tra i componenti del gruppo;
  \item Possono essere acceduti tramite l’uso di un semplice Browser\ped{g}.
   \end{itemize}
  Questo strumento permetterà a 2 o più componenti del gruppo di interagire lavorando sugli stessi documenti contemporaneamente. Google Drive viene utilizzato come strumento di supporto allo sviluppo della documentazione e del Software\ped{g} presente su Git .
  


\subsubsubsection{Google Calendar}
 
Google Calendar viene utilizzato all’interno del gruppo per gestire le risorse umane\ped{g}. In
particolare tale strumento viene utilizzato per notificare in quali giorni un determinato
membro non può essere disponibile e per segnalare date rilevanti per il gruppo, come
ad esempio le date delle riunioni.
\subsubsection{Pianificazione}
\label{sec:Pianificazione}
Per pianificare le attività legate allo sviluppo del Progetto\ped{g} e la gestione delle Risorse\ped{g} si è scelto di utilizzare \emph{Redmine}.
\emph{Redmine} è un Programma\ped{g} per il project management. I motivi per cui è stato scelto questo Software\ped{g} vengono descritti nella sezione \S\ref{subsec:Software di gestione del prodotto}


\subsection{Protocollo per lo sviluppo dell'applicazione}
\label{sec:protocolloSviluppo} 
Per procedere con uno sviluppo controllato dei documenti e del Codice\ped{g} si è scelto di adottare il sistema di Ticketing\ped{g} \textbf{Redmine}.\\ 
Le motivazioni alla base della scelta di tale Software\ped{g} sono descritte nella sezione \S\ref{subsec:Software di gestione del prodotto}.

\subsubsection{Creare un nuovo progetto} 

La creazione di un Progetto\ped{g} è compito del \emph{Responsabile di Progetto}.\\ 
Un nuovo Progetto\ped{g} rappresenta una macro-attività caratterizzata da molte sotto-attività supervisionate da un responsabile.\\\
Per creare un nuovo Progetto\ped{g}:
\begin{itemize}
\item Aprire \textbf{Progetti}; 
\item Selezionare \textbf{Nuovo progetto}; 
\item Assegnare un \textbf{Nome} breve ma significativo; 
\item Nel caso in cui si voglia creare un sotto-Progetto\ped{g} indicare il nome del Progetto\ped{g} padre dall’omonimo campo; 
\item \textbf{Identificativo}: scrivere in minuscolo ed indicare Codice\ped{g} della fase a cui si riferisce;
\item Lasciare inalterati gli altri campi. 
\end{itemize}
 
\subsubsection{Creazione ticket}
 
  I Ticket\ped{g} vengono creati da:
 \begin{itemize}
 

    \item \textbf{Responsabile di Progetto}: crea i Ticket\ped{g} più importanti che rappresentano le macro fasi evidenziate dalla pianificazione; 
	\item \textbf{Responsabile di Sotto-progetto}: crea i Ticket\ped{g} per i Processi\ped{g} non pianificati inizialmente, che si evidenziano necessari per l’avanzamento del sotto-Progetto\ped{g} assegnato; 
	\item \textbf{Verificatore}: crea i Ticket\ped{g} per segnalare errori ed imprecisioni trovate durante il Processo\ped{g} di verifica. 
 \end{itemize}


I Ticket\ped{g} possono essere di tre tipologie:
\begin{itemize}


\item \textbf{Ticket\ped{g} di pianificazione}: rappresentano le macro-attività di maggiore importanza. Sono organizzate in una gerarchia con vari livelli di priorità.
 Tali attività vengono create da: 
\begin{itemize}
\item \emph{Responsabile di Progetto}: durante la pianificazione identifica le attività più importati e generali; 
\item \emph{Responsabile di Sotto-progetto}: durante lo svolgimento delle attività può scomporre in sotto-problemi l’attività indicata dal Responsabile di Progetto\ped{g}. 
\end{itemize}


\item \textbf{Ticket\ped{g} di realizzazione e controllo}: tutti i documenti redatti, durante la stesura attraversano due stadi: 
\begin{itemize}
\item \textbf{Realizzazione}: un redattore del documento effettua una prima stesura; 
\item \textbf{Controllo}: un redattore, diverso da quello della precedente fase, esegue un primo controllo sui contenuti della parte scritta. 
\end{itemize}


\item \textbf{Ticket\ped{g} di verifica}: rappresentano gli errori identificati dai Verificatori durante 
il controllo che la realizzazione dell'attività sia conforme a quanto richiesto e che 
rispetti tutte le norme.
\end{itemize}



\subsubsubsection{Ticket\ped{g} di pianificazione}

\begin{itemize}


\item Selezionare \textbf{Nuova segnalazione} da menù principale; 
\item \textbf{Tracker}: indicare la natura del Ticket\ped{g}: 
	\begin{itemize}
	\item \textbf{Documento}: stesura di un documento. Il tipo di attività svolta dal redattore del documento viene definito durante la rendicontazione; 
	\item \textbf{Codifica}: stesura di Codice\ped{g}; 
	\item \textbf{Verifica}: macro-attività di verifica sul prodotto dei sotto-Processi\ped{g}. 


	\end{itemize}

\item \textbf{Oggetto}: descrizione breve e significativa; 
\item \textbf{Descrizione}: descrizione comprensibile e con riferimenti esterni mediante Link\ped{g} se necessario; 
\item \textbf{Stato}: Plan; 
\item \textbf{Attività principale}: se si vuole creare una \textbf{sotto-attività} indicare l’id del Ticket\ped{g} 
padre; 
\item \textbf{Categoria}: PDCA, solo se il Ticket\ped{g} viene creato dal \emph{Responsabile di Progetto}; 
\item \textbf{Assegnato a}: inserire il nome del responsabile; 
\item \textbf{Osservatori}: aggiungere eventuali collaboratori.
\end{itemize}  


\subsubsubsection{Ticket\ped{g} di realizzazione e controllo} 
\label{sec:realizzazioneControllo}
		\begin{itemize}
		
		\item Selezionare \textbf{Nuova segnalazione} da menù principale; 
		\item \textbf{Tracker}: indicare la natura del Ticket\ped{g}: 
		\begin{itemize}
	\item \textbf{Documento}: stesura di un documento. Il tipo di attività svolta dal redattore del documento viene definito durante la rendicontazione; 
	\item \textbf{Codifica}: stesura di Codice\ped{g}; 
	\item \textbf{Verifica}: attività di verifica sui prodotti dei Processi\ped{g}. 

	\end{itemize}

\item \textbf{Oggetto}: descrizione breve e significativa secondo il principio: nome Ticket\ped{g} padre attività da svolgere (realizzazione o controllo); 
\item \textbf{Descrizione}: descrizione comprensibile e con riferimenti esterni mediante Link\ped{g} se 
necessario; 

\item \textbf{Stato}: New; 
\item \textbf{Attività principale}: se si vuole creare una \textbf{sotto-attività} indicare l’id del Ticket\ped{g} 
padre; 
\item \textbf{Inizio}: dare una data di inizio presunta; 
\item \textbf{Scadenza}: dare una data di fine presunta; 
\item \textbf{Assegnato a}: inserire il nome del responsabile; 
\item \textbf{Osservatori}: aggiungere eventuali collaboratori. 
\end{itemize} 

\subsubsubsection{Ticket\ped{g} di verifica}
\label{sec:TicketVerifica}

Un \emph{Verificatore} per creare un \emph{Ticket\ped{g} di verifica} deve: 
\begin{enumerate}
\item Assicurarsi che esista all’interno del Progetto\ped{g} l’attività \emph{Verifica}.
Su tale attività vi devono essere due sotto-attività: “Verifica - realizzazione”, 
“Verifica - approvazione”.\\
Tutti i Ticket\ped{g} creati devono essere sotto-attività di: “Verifica - realizzazione”; 
\item Creare quindi il Ticket\ped{g} secondo le seguenti direttive: 
		\begin{itemize}
		
		
		\item Selezionare \textbf{Nuova segnalazione} da menù principale; 
		\item \textbf{Tracker}: Bug; 
		\item \textbf{Oggetto}: descrizione breve e significativa dell’errore incontrato; 
		\item \textbf{Descrizione}: descrivere in modo dettagliato e chiaro: la natura e la posizione dell’errore; 
		\item \textbf{Stato}: New; 
		\item \textbf{Attività principale}: tutti i Ticket\ped{g} devono essere figli del Ticket\ped{g} “Verifica - realizzazione” del Progetto\ped{g} su cui si sta eseguendo la verifica; 
		\item \textbf{Assegnato a}: inserire il nome del responsabile del Progetto\ped{g} padre (es. 
		responsabile delle \emph{Norme di Progetto}). 
		\end{itemize}
Tutti i campi non segnalati sono da lasciare vuoti. 
Sarà poi compito del responsabile del Progetto\ped{g} padre decidere a chi assegnare la correzione dell’errore. Nel caso in cui l’errore segnalato non sia considerato valido dal 
\emph{Responsabile del sotto-progetto} verrà confermato il rifiuto dal \emph{Responsabile di Progetto}. 

\end{enumerate}


\subsubsubsection{Dipendenze temporali}


Dopo la creazione del Ticket\ped{g}, per aggiungere \textbf{dipendenze temporali} tra i Ticket\ped{g}:
\begin{itemize}
\item Andare su \textbf{segnalazioni}; 
\item Aprire il Link\ped{g} alla segnalazione a cui aggiungere la dipendenza; 
\item Nella sezione \textbf{segnalazioni correlate} premere \textbf{aggiungi}; 
\item Scegliere \textbf{segue} e indicare il numero della segnalazione che lo blocca ed eventuali giorni di slack. 

\end{itemize} 

Tutti i campi non segnalati sono da lasciare vuoti. 

\subsubsection{Aggiornamento ticket}

Esistendo due tipologie di Ticket\ped{g}, viene qui definito la procedura per effettuare l’aggiornamento di entrambe.

\subsubsubsection{Ticket\ped{g} di pianificazione}

\begin{itemize}
\item Andare sul menù \textbf{Segnalazioni}; 
\item Selezionare il Ticket\ped{g} di interesse; 
\item Cliccare il Link\ped{g} \textbf{Aggiorna}; 
\item Commentare ciò che si è fatto sulla form \textbf{Note}; 
\item Cambiare lo stato del Ticket\ped{g} secondo la seguente logica:
		\begin{itemize}
		\item \textbf{Do}: quando un Ticket\ped{g} è in questo stato indica che una o più persone stanno 
		lavorando su tale attività; 
		\item \textbf{Check}: quando un Ticket\ped{g} è in questo stato indica che una o più persone 
		stanno lavorando sulla verifica di tale attività; 
		\item \textbf{Act}: l’attività è stata conclusa e verificata, e ne sono state tratte le conclusioni adeguate. 
		
		\end{itemize} 

\item Se viene concluso, aggiornare lo stato del Ticket\ped{g} di pianificazione padre. 

\end{itemize}


\subsubsubsection{Ticket\ped{g} di realizzazione e controllo}

\begin{itemize}
\item Andare sul menù \textbf{Segnalazioni}; 
\item Selezionare il Ticket\ped{g} di interesse; 
\item Cliccare il Link\ped{g} \textbf{Aggiorna}; 
\item Indicare il tempo impiegato in ore; 
\item Indicare il tipo di attività svolta; 
\item Commentare ciò che si è fatto sulla form \textbf{Note}; 
\item Cambiare lo stato del Ticket\ped{g} secondo la seguente logica: 
		\begin{itemize}
		\item \textbf{In Progress}: quando un Ticket\ped{g} è in questo stato indica che una o più persone 
		stanno lavorando su tale attività. La percentuale di completamento deve 
		essere impostata tra lo 0\% ed il 90\%; 
		\item \textbf{Closed}: l’attività è stata conclusa. La percentuale di completamento dell’attività è al 100\%. 
		 
		\end{itemize} 
\item Aggiornare lo stato del Ticket\ped{g} di pianificazione padre secondo tali principi: 
		\begin{itemize}
		\item Ticket\ped{g} figlio passa da New a In Progress: il Ticket\ped{g} padre passa da Plan a Do, 
		o da Do a Check; 
		\item Ticket\ped{g} figlio passa a Closed: il Ticket\ped{g} padre deve essere in Do o Check; 
		\item Tutti i Ticket\ped{g} figli vengono chiusi: il Ticket\ped{g} padre passa ad Act.
		\end{itemize}

\end{itemize}



\subsubsubsection{Ticket\ped{g} di verifica}

\begin{itemize}
\item Andare sul menù \textbf{Segnalazioni}; 
\item Selezionare il Ticket\ped{g} di interesse; 
\item Cliccare il Link\ped{g} \textbf{Aggiorna}; 
\item Indicare il tempo impiegato in ore; 
\item Indicare Verifica come tipo di attività svolta; 
\item Commentare le correzione nella form \textbf{Note}; 
\item Cambiare lo stato del Ticket\ped{g} secondo la seguente logica:
		\begin{itemize}
		\item \textbf{In Progress}: quando un Ticket\ped{g} è in questo stato indica che una o più persone stanno lavorando su tale attività. La percentuale di completamento deve essere impostata tra lo 0\% ed il 90\%; 
		\item \textbf{Closed}: l’attività è stata conclusa. La percentuale di completamento dell’attività è al 100\%; 
		\item \textbf{Rejected}: l’attività di verifica è stata rifiutata dal \emph{Responsabile del sotto progetto} in accordo con il \emph{Responsabile di Progetto}. 
		
		\end{itemize}

\item Aggiornare lo stato del Ticket\ped{g} di pianificazione padre secondo tali principi:
		\begin{itemize}
		\item Ticket\ped{g} figlio passa da New a In Progress: il Ticket\ped{g} padre passa da Plan a Do, 
		o da Do a Check; 
		\item Ticket\ped{g} figlio passa a Closed: il Ticket\ped{g} padre deve essere in Do o Check; 
		\item Tutti i Ticket\ped{g} figli vengono chiusi: il Ticket\ped{g} padre passa ad Act. 
		
		\end{itemize} 

\end{itemize} 
\begin{figure}[H]
  \centering
    \includegraphics[width=0.60\textwidth]{\imgs CreazioneTicket}
  \caption{Diagramma attività - Creazione nuovo ticket}
  \label{fig:CreazioneTicket}
\end{figure}

\begin{figure}[H]
  \centering
    \includegraphics[width=0.75\textwidth]{\imgs AggiornamentoTicket}
  \caption{Diagramma attività -Aggiornamento ticket}
  \label{fig:gull}
\end{figure}




\subsubsection{Consigli di utilizzo}
 
	Per avere una immediata visualizzazione dei Ticket\ped{g} assegnati, è consigliato personalizzare 
	la pagina personale: 
	\begin{itemize}
		\item Andare alla \textbf{Pagina personale}; 
		\item Cliccare il Link\ped{g} \textbf{Personalizza la pagina}; 
		\item Dal menù a tendina \textbf{La mia pagina di blocco}, selezionare \textbf{Le mie segnalazioni} 
		e premere il pulsante verde +; 
		\item Ripetere il punto precedente per aggiungere \textbf{Segnalazioni osservate}. 
	
	\end{itemize}

	Per avere una visualizzazione più chiara delle segnalazioni si consiglia di ordinarle per 
	oggetto. Tale risultato può essere ottenuto premendo \textbf{Oggetto} dalla pagina \textbf{Segnalazioni}.