\section{Configurazione iniziale}
Una volta installate le tecnologie e i pacchetti\ped{g} utilizzati dal sistema si dovr\`{a} impostare opportunamente le variabili d'ambiente\ped{g} per gli eseguibili mongod e mongo.
A questo punto si dovr\`{a} configurare il server\ped{g} come segue:

\subsection{Mac OS e Ubuntu}

\begin{enumerate}

\item creare una cartella per raccogliere i database creati con MongoDB
\item avviamento mongodb con il comando "\textbf{sudo mongod}" (\textbf{--port xxxxx} se si vuole usare una porta diversa da quella di default), specificando la cartella per il savataggio e recupero dei database con l'opzione "\textbf{mongod --dbpath /cartella/dbs/}"
\item  spostarsi dal terminale nella cartella Premi (o quella in cui \`{e} presente lo script mongoConfig se spostato) ed accedere a mongoDB con il comando "\textbf{mongo}", avvenuto l'accesso configurare il database con il comando "\textbf{load('mongoConfig.js')}"
\item se \`{e} stato avviato mongod con una porta differente da quella di default modificare il file config.js all'interno della cartella Premi specificando la porta con cui \`{e} stato avviato mongod

\end{enumerate}


\subsection{Windows}

\begin{enumerate}

\item creare una cartella per raccogliere i database creati con MongoDB
\item avviamento mongodb con il comando "\textbf{mongod}" (\textbf{--port xxxxx} se si vuole usare una porta diversa da quella di default), specificando la cartella per il savataggio e recupero dei database con l'opzione "\textbf{mongod --dbpath /cartella/dbs/}"
\item  spostarsi dal terminale nella cartella Premi (o quella in cui \`{e} presente lo script mongoConfig se spostato) ed accedere a mongoDB con il comando "\textbf{mongo}", avvenuto l'accesso configurare il database con il comando "\textbf{load('mongoConfig.js')}"
\item se \`{e} stato avviato mongod con una porta differente da quella di default modificare il file config.js all'interno della cartella Premi specificando la porta con cui \`{e} stato avviato mongod

\end{enumerate}

\newpage

\section{Avvio}
Una volta avvenuta la configurazione iniziale si pu\`{o} avviare il server con i passi descritti in seguito:

\begin{itemize}
\item avviamento server MongoDB con il comando "\textbf{mongod --port xxxxx --dbpath /percorso/cartella/dbs}"
\item  avviamento Server Node.js spostandosi da terminale alla radice della cartella premi e lanciare il comando "\textbf{node premi\_Server.js}" 
\end{itemize}


\section{Arresto}
Una volta avviato, per arrestare il sistema basta arrestare il server MongoDb e il server\ped{g} Node.js con il comando \textbf{ctrl + c} nelle finestre di terminale in cui sono stati avviati i due server\ped{g}.
