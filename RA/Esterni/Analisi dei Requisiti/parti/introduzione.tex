\section{Introduzione}
\subsection{Scopo del documento}
L’analisi del capitolato d’appalto C4 e il successivo incontro con il Proponente\ped{g} hanno evidenziato un insieme di Requisiti\ped{g} che il presente documento ha lo scopo di elencare e descrivere in modo dettagliato. Lo scopo di tale documento è quindi presentare le funzionalità che offrirà il prodotto.
\subsection{Scopo del Prodotto}
Lo scopo del Progetto\ped{g} è la realizzazione un Software\ped{g} per la creazione ed esecuzione di presentazioni multimediali favorendo l’uso di tecniche di storytelling e visualizzazione non lineare dei contenuti.
\subsection{Glossario}
Al fine di evitare ogni ambiguità di linguaggio e massimizzare la comprensione dei documenti, i termini tecnici, di dominio, gli acronimi e le parole che necessitano di essere chiarite, sono riportate nel documento \href{run:../../Esterni/\fGlossario}{\fEscapeGlossario}. Ogni occorrenza di vocaboli presenti nel Glossario è marcata da una “g” minuscola in pedice.
\subsection{Riferimenti}
\subsubsection{Normativi}

\begin{itemize}
\item Regole del Progetto\ped{g} didattico, reperibili all'Indirizzo\ped{g}:\\ \url{http://www.math.unipd.it/~tullio/IS-1/2014/Progetto/PD01.pdf}
\item Vincoli di organigramma, consultabili all’Indirizzo\ped{g}:\\ \url{http://www.math.unipd.it/~tullio/IS-1/2014/Progetto/PD01b.html}
\item Norme di Progetto\ped{g}: \href{run:../../Interni/\fNormeDiProgetto}{\fEscapeNormeDiProgetto};
\item Capitolato d’appalto C4: Premi: Software\ped{g} di presentazione “better than Prezi” \\
\url{http://www.math.unipd.it/~tullio/IS-1/2014/Progetto/C4.pdf}.
\end{itemize}

\subsubsection{Informativi}
\begin{itemize}
\item Slide dell'insegnamento Ingegneria del Software\ped{g} modulo A:\\
\begin{itemize}
\item Il ciclo di vita\ped{g} del Software\ped{g};
\item Gestione di Progetto\ped{g}.
\end{itemize}
\url{http://www.math.unipd.it/~tullio/IS-1/2014/} ;

\item Ingegneria del Software\ped{g} - Ian Sommerville - 9a Edizione (2010).
\end{itemize}

