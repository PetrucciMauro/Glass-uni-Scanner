\subsection{UC 0.10 - Login Account utente}{
	\label{uc0.10}
	\begin{figure}[H]
		\centering
		\includegraphics[scale=0.75]{\imgs {UC0.10}.jpg} %inserire il diagramma UML
		\label{fig:uc0.10}
		\caption{Caso d'uso 0.10: Login\ped{g} Account\ped{g} utente}
	\end{figure}
	\textbf{Attori}: utente non autenticato\ped{g}. \\
	\textbf{Descrizione}: l'utente inserisce il proprio username e la propria password, se queste risultano corrette l'utente viene autenticato. \\
	\textbf{Precondizione}: il sistema è attivo e funzionante; l'utente non ancora autenticato vuole effettuare il Login\ped{g}.	\\
	\textbf{Postcondizione}: l'utente viene autenticato ed entra nel sistema.	\\
	\textbf{Scenario principale}:
	\begin{enumerate}
		\item Inserimento username \S\hyperref[uc0.10.1]{(UC 0.10.1)};
		\item Inserimento password \S\hyperref[uc0.10.2]{(UC 0.10.2)};
	\end{enumerate}
	\textbf{Scenario alternativo}:
	\begin{itemize}
		\item Le credenziali non risultano corrette, ne viene data comunicazione all'utente che potrà reinserirle.
	\end{itemize}
	}
	\subsubsection{UC0.10.1 - Inserimento username}{
		\label{uc0.10.1}
		\textbf{Attori}: utente non autenticato\ped{g}. \\
		\textbf{Descrizione}: l'utente inserisce il proprio username.	\\
		\textbf{Precondizione}: l'utente  è pronto ad inserire il proprio username per effettuare il Login\ped{g}.	\\
		\textbf{Postcondizione}: l'utente  ha inserito il proprio username per effettuare il Login\ped{g}.\\
		\textbf{Scenario principale}:
		\begin{enumerate}
			\item L'utente inserisce il proprio username.
		\end{enumerate}
		}
	\subsubsection{UC0.10.2 - Inserimento password}{
		\label{uc0.10.2}
		\textbf{Attori}: utente non autenticato\ped{g}. \\
		\textbf{Descrizione}: l'utente  inserisce la propria password.	\\
		\textbf{Precondizione}: l'utente  è pronto ad inserire la propria password per effettuare il Login\ped{g}.	\\
		\textbf{Postcondizione}: l'utente  ha inserito la propria password per effettuare il Login\ped{g}.	\\
		\textbf{Scenario principale}:
		\begin{enumerate}
			\item L'utente inserisce la propria password.
		\end{enumerate}
		}