\subsection{UC 0.9 - Gestione template}{
	\label{uc0.9}
	\begin{figure}[H]
		\centering
		\includegraphics[scale=0.6]{\imgs {UC0.9}.jpg} %inserire il diagramma UML
		\label{fig:uc0.9}
		\caption{Caso d'uso 0.9: Gestione template}
	\end{figure}
	\textbf{Attori}: amministratore. \\
	\textbf{Descrizione}: l'amministratore ha effettuato il Login\ped{g} e il sistema è correttamente funzionante. L'amministratore ha la possibilità di inserire nuovi Template\ped{g} di Infografica\ped{g} o di presentazione, nuovi oggetti grafici per comporre i Template\ped{g} ed eliminare Template\ped{g} presenti nel Server\ped{g}.\\
	\textbf{Precondizione}: il sistema è in Funzione\ped{g} e l'amministratore ha effettuato il Login\ped{g}.\\
	\textbf{Postcondizione}: le operazioni svolte dall'amministratore sono state portate a termine con successo.\\
	\textbf{Scenario principale}:
	\begin{enumerate}
		\item Caricamento di un Template\ped{g} di Infografica\ped{g} o di presentazione \S\hyperref[uc0.9.1]{(UC0.9.1)};
		\item Caricamento di un Elemento\ped{g} \S\hyperref[uc0.9.3]{(UC0.9.3)};
		\item Eliminazione di un Template\ped{g} \S\hyperref[uc0.9.5]{(UC0.9.5)};
	\end{enumerate}
	\textbf{Scenari alternativi}:
	\begin{itemize}
		\item L'amministratore può ripristinare un Template\ped{g} eliminato \S\hyperref[uc0.9.5]{(UC0.9.5)};
		\item L'amministratore può ripristinare le modifiche eseguite;
		\item Visualizzazione di un errore nel caso vengano inseriti un oggetto o un Template\ped{g} già presenti nel database.
	\end{itemize}
	}
\subsubsection{UC0.9.1 - Caricamento di un template}{
	\label{uc0.9.1}
	\textbf{Attori}: amministratore. \\
	\textbf{Descrizione}: l'amministratore esegue il caricamento di un nuovo Template\ped{g} nel database. \\
	\textbf{Precondizione}: il sistema  è funzionante e l'amministratore ha effettuato il Login\ped{g}.	\\
	\textbf{Postcondizione}: il Template\ped{g} è stato caricato correttamente nel database.	\\
	\textbf{Scenario principale}:
	\begin{enumerate}
		\item L'amministratore naviga nel proprio spazio di lavoro  per trovare il Template\ped{g} da inserire;
		\item L'amministratore inserisce il Template\ped{g} nel database.
	\end{enumerate}
	\textbf{Scenari alternativi}:
	\begin{itemize}
		\item Il Template\ped{g} è già presente nel database e l'amministratore viene informato dell'errore \S\hyperref[uc0.9.2]{(UC 0.9.2)}.
	\end{itemize}
	}
\subsubsection{UC0.9.2 - Visualizza errore caricamento template}{
	\label{uc0.9.2}
	\textbf{Attori}: amministratore. \\
	\textbf{Descrizione}: il sistema ha riscontrato un errore nel caricamento d'un template e deve renderlo noto all'amministratore. \\
	\textbf{Precondizione}: un Template\ped{g} è in fase di caricamento.	\\
	\textbf{Postcondizione}: l'amministratore ha visualizzato l'errore del caricamento d'un template.	\\
	\textbf{Scenario principale}:
	\begin{enumerate}
		\item Viene visualizzato l'errore sul caricamento d'un template riscontrato dal sistema.
	\end{enumerate}
	}
\subsubsection{UC 0.9.3 - Caricamento di un elemento}{
	\label{uc0.9.3}
	\textbf{Attori}: amministratore. \\
	\textbf{Descrizione}: l'amministratore esegue il caricamento di un nuovo Elemento\ped{g} nel database. \\
	\textbf{Precondizione}: il sistema  è funzionante e l'amministratore ha effettuato il Login\ped{g}.	\\
	\textbf{Postcondizione}: l'Elemento\ped{g} grafico è stato caricato correttamente nel database.	\\
	\textbf{Scenario principale}:
	\begin{enumerate}
		\item L'amministratore naviga nel proprio spazio di lavoro per trovare l'Elemento\ped{g} grafico da inserire;
		\item L'amministratore inserisce l'Elemento\ped{g} grafico nel database.
	\end{enumerate}
	\textbf{Scenari alternativi}:
	\begin{itemize}
		\item Il File\ped{g} grafico è già presente nel database e l'amministratore viene informato dell'errore \S\hyperref[uc0.9.2]{(UC 0.9.2)}.
	\end{itemize}
	}
\subsubsection{UC0.9.6 - Visualizza errore caricamento elemento}{
	\label{uc0.9.2}
	\textbf{Attori}: amministratore. \\
	\textbf{Descrizione}: il sistema ha riscontrato un errore sul caricamento d'un elemento e deve renderlo noto all'amministratore. \\
	\textbf{Precondizione}: un Elemento\ped{g} è in fase di caricamento.	\\
	\textbf{Postcondizione}: l'amministratore ha visualizzato l'errore relativo al caricamento d'un elemento.	\\
	\textbf{Scenario principale}:
	\begin{enumerate}
		\item Viene visualizzato l'errore riscontrato dal sistema sul caricamento d'un elemento.
	\end{enumerate}
	}
\subsubsection{UC 0.9.4- Eliminazione di un template}{
	\label{uc0.9.4}
	\textbf{Attori}: amministratore. \\
	\textbf{Descrizione}: l'amministratore elimina uno dei Template\ped{g}. \\
	\textbf{Precondizione}: il sistema è funzionante e l'amministratore decide di eliminare un Template\ped{g}.	\\
	\textbf{Postcondizione}: il Template\ped{g} è stato eliminato dal sistema.	\\
	\textbf{Scenario principale}:
	\begin{enumerate}
		\item L’amministratore elimina il Template\ped{g} selezionato.
	\end{enumerate}
	\textbf{Scenari alternativi}:
	\begin{itemize}
		\item L'amministratore decide di ripristinare il Template\ped{g} eliminato \S\hyperref[uc0.9.5]{(UC 0.9.5)}.
	\end{itemize}
	}
\subsubsection{UC 0.9.5 - Ripristino Template eliminato}{
	\label{uc0.9.5}
	\textbf{Attori}: amministratore. \\
	\textbf{Descrizione}: l'amministratore ha deciso di ripristinare un Template\ped{g} eliminato. \\
	\textbf{Precondizione}: lo storico dei Template\ped{g} eliminati non è vuoto.	\\
	\textbf{Postcondizione}: il Template\ped{g} è stato ripristinato.	\\
	\textbf{Scenario principale}:
	\begin{enumerate}
		\item Selezione Template\ped{g} eliminato; 
		\item Ripristino del Template\ped{g} eliminato.
	\end{enumerate}
	}