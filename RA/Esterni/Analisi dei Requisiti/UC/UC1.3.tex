\subsection{UC 1.3 - Modifica presentazione da desktop}{
	\label{uc1.3}
	\begin{figure}[H]
		\centering
		\includegraphics[scale=0.75]{\imgs {UC1.3}.jpg} %inserire il diagramma UML
		\label{fig:uc1.3}
		\caption{Caso d'uso 1.3: Modifica Desktop\ped{g} della presentazione}
	\end{figure}
	\textbf{Attori}: utente Desktop\ped{g} \\
	\textbf{Descrizione}: l'utente ha la possibilità di modificare una presentazione presente nel proprio spazio sul Server\ped{g}. \\
	\textbf{Precondizione}: il sistema ha una presentazione caricata correttamente e aperta in modalità modifica.	\\
	\textbf{Postcondizione}: l'utente ha modificato la presentazione selezionata e salvato le modifiche in locale.	\\
	\textbf{Scenario principale}:
	\begin{enumerate}
		\item L'utente può inserire un nuovo Frame\ped{g} vuoto \S\hyperref[uc1.3.1]{(UC 1.3.1)};
		\item L'utente può spostare i Frame\ped{g} nel piano della presentazione\ped{g} \S\hyperref[uc1.3.2]{(UC 1.3.2)};
		\item L'utente può modificare i Frame\ped{g} inseriti \S\hyperref[uc1.3.3]{(UC 1.3.3)};
		\item L'utente può eliminare i Frame\ped{g} inseriti \S\hyperref[uc1.3.4]{(UC 1.3.4)};
		\item L'utente può impostare uno sfondo \S\hyperref[uc1.3.5]{(UC 1.3.5)};
		\item L'utente può definire i percorsi\ped{g} della presentazione \S\hyperref[uc1.3.6]{(UC 1.3.6)};
		\item L'utente può inserire e cancellare Bookmark\ped{g} \S\hyperref[uc1.3.7]{(UC 1.3.7)};
		\item L'utente può impostare le opzioni di esecuzione della presentazione \S\hyperref[uc1.3.8]{(UC 1.3.8)};
		\item L'utente può definire i tempi di permanenza sui Frame\ped{g} \S\hyperref[uc1.3.9]{(UC 1.3.9)};
		\item L'utente può inserire un Elemento\ped{g} SVG \S\hyperref[uc1.3.10]{(UC 1.3.10)};
		\item L'utente può modifica un Elemento\ped{g} SVG \S\hyperref[uc1.3.11]{(UC 1.3.11)};
		\item L'utente può eliminare un Elemento\ped{g} \S\hyperref[uc1.3.12]{(UC 1.3.12)};
		\item L'utente può modificare la rotazione di un Elemento\ped{g} \S\hyperref[uc1.3.13]{(UC 1.3.13)};
	\end{enumerate}
	\textbf{Scenari alternativi}: 
	\begin{itemize}
		\item L'operazione viene annullata e non si apportano cambiamenti \S\hyperref[uc1.5]{(UC 1.5)}.
	\end{itemize}
	}
\subsubsection{UC 1.3.1 - Inserimento nuovo frame}{
	\label{uc1.3.1}
	\begin{figure}[H]
		\centering
		\includegraphics[scale=0.75]{\imgs {UC1.3.1}.jpg} %inserire il diagramma UML
		\label{fig:uc1.3.1}
		\caption{Caso d'uso 1.3.1: Inserimento di un nuovo frame}
	\end{figure}
	\textbf{Attori}: utente Desktop\ped{g} \\
	\textbf{Descrizione}: l'utente può inserire un nuovo Frame\ped{g} nel piano della presentazione\ped{g}. \\
	\textbf{Precondizione}: il sistema ha una presentazione aperta in modalità modifica.	\\
	\textbf{Postcondizione}: l'utente ha inserito nel piano della presentazione\ped{g} un nuovo Frame\ped{g}.	\\
	\textbf{Scenario principale}:
	\begin{enumerate}
		\item L'utente seleziona un Frame\ped{g} nel menu a lato del piano della presentazione\ped{g} \S\hyperref[uc1.3.1.1]{(UC 1.3.1.1)};
		\item L'utente sposta il Frame\ped{g} all'interno del piano della presentazione\ped{g} \S\hyperref[uc1.3.1.2]{(UC 1.3.1.2)};
		\item L'utente rilascia il Frame\ped{g}.
	\end{enumerate}
}
\subsubsection{UC 1.3.1.1 - Selezione frame}{
	\label{uc1.3.1.1}
	\textbf{Attori}: utente Desktop\ped{g} \\
	\textbf{Descrizione}: l'utente può selezionare uno dei Frame\ped{g} definiti dal sistema. \\
	\textbf{Precondizione}: il sistema ha una presentazione aperta in modalità modifica.	\\
	\textbf{Postcondizione}: l'utente ha selezionato dal menu il Frame\ped{g} desiderato.	\\
	\textbf{Scenario principale}:
	\begin{enumerate}
		\item L'utente seleziona l'azione "nuovo frame";
		\item L'utente seleziona uno dei Frame\ped{g} disponibili.
	\end{enumerate}
	}
\subsubsection{UC 1.3.1.2 - Spostamento Frame\ped{g} nel piano della presentazione}{
	\label{uc1.3.1.2}
	\textbf{Attori}: utente Desktop\ped{g} \\
	\textbf{Descrizione}: l'utente può spostare un nuovo Frame\ped{g} nel piano della presentazione\ped{g}. \\
	\textbf{Precondizione}: il sistema ha una presentazione aperta in modalità modifica.	\\
	\textbf{Postcondizione}: l'utente ha inserito nel piano di modifica un nuovo Frame\ped{g}.	\\
	\textbf{Scenario principale}:
	\begin{enumerate}
		\item L'utente sposta il Frame\ped{g} nel piano della presentazione\ped{g};
		\item L'utente rilascia la selezione sul Frame\ped{g}.
	\end{enumerate}
	}
\subsubsection{UC 1.3.2 - Spostamento frame}{
	\label{uc1.3.2}
	\textbf{Attori}: utente Desktop\ped{g} \\
	\textbf{Descrizione}: l'utente può spostare un Frame\ped{g} nel piano della presentazione\ped{g}. \\
	\textbf{Precondizione}: il sistema ha una presentazione aperta in modalità modifica.	\\
	\textbf{Postcondizione}: l'utente ha spostato un Frame\ped{g} nel piano della presentazione\ped{g}.	\\
	\textbf{Scenario principale}:
	\begin{enumerate}
		\item L'utente seleziona un Frame\ped{g} nel piano della presentazione\ped{g};
		\item L'utente sposta il Frame\ped{g} nel piano della presentazione\ped{g};
		\item L'utente rilascia il Frame\ped{g} selezionato.
	\end{enumerate}
	}
\subsubsection{UC 1.3.3 - Modifica Desktop\ped{g} di un frame}{
	\label{uc1.3.3}
	\begin{figure}[H]
		\centering
		\includegraphics[scale=0.6]{\imgs {UC1.3.3}.jpg} %inserire il diagramma UML
		\label{fig:uc1.3.3}
		\caption{Caso d'uso 1.3.3: Modifica Desktop\ped{g} di un frame}
	\end{figure}
	\textbf{Attori}: utente Desktop\ped{g} \\
	\textbf{Descrizione}: l'utente Desktop\ped{g} ha scelto l'opzione di modifica di un Frame\ped{g}: può scegliere di inserire o modificare un Elemento\ped{g} (che può essere del testo, un'immagine, o un video), di spostare un Elemento\ped{g}, di eliminare un Elemento\ped{g}, di inserire o eliminare una scelta, di modificare la dimensione o la forma del Frame\ped{g}, lo spessore o il colore del bordo e lo sfondo. \\
	\textbf{Precondizione}: il sistema ha una presentazione aperta in modalità modifica.	\\
	\textbf{Postcondizione}: nella presentazione è presente un Frame\ped{g} modificato.	\\
	\textbf{Scenario principale}:
	\begin{enumerate}
		\item Inserimento di un Elemento\ped{g} testo \S\hyperref[uc1.3.3.1]{(UC 1.3.3.1)};
		\item Inserimento di un Elemento\ped{g} immagine \S\hyperref[uc1.3.3.2]{(UC 1.3.3.2)};
		\item Inserimento di un Elemento\ped{g} video \S\hyperref[uc1.3.3.3]{(UC 1.3.3.3)};
		\item Modifica di un Elemento\ped{g} testo \S\hyperref[uc1.3.3.4]{(UC 1.3.3.4)};
		\item Modifica di un Elemento\ped{g} immagine \S\hyperref[uc1.3.3.5]{(UC 1.3.3.5)};
		\item Modifica di un Elemento\ped{g} video \S\hyperref[uc1.3.3.6]{(UC 1.3.3.6)};
		\item Spostamento di un Elemento\ped{g} \S\hyperref[uc1.3.3.7]{(UC 1.3.3.7)};
		\item Modifica rotazione di un Frame\ped{g} \S\hyperref[uc1.3.3.8]{(UC 1.3.3.8)};
		\item Inserimento di un Elemento\ped{g} scelta \S\hyperref[uc1.3.3.9]{(UC 1.3.3.9)};
		\item Modifica di un Elemento\ped{g} scelta \S\hyperref[uc1.3.3.10]{(UC 1.3.3.10)};
		\item Modifica della dimensione del Frame\ped{g} \S\hyperref[uc1.3.3.11]{(UC 1.3.3.11)};
		\item Modifica della forma del Frame\ped{g} \S\hyperref[uc1.3.3.12]{(UC 1.3.3.12)};
		\item Modifica dello spessore del bordo del Frame\ped{g} \S\hyperref[uc1.3.3.13]{(UC 1.3.3.13)};
		\item Modifica del colore del bordo del Frame\ped{g} \S\hyperref[uc1.3.3.14]{(UC 1.3.3.14)};
		\item Modifica dello sfondo del Frame\ped{g} \S\hyperref[uc1.3.3.15]{(UC 1.3.3.15)};
	\end{enumerate}
	}
\subsubsection{UC 1.3.3.1 - Inserimento di un Elemento\ped{g} testo}{
	\label{uc1.3.3.1}
	\textbf{Attori}: utente Desktop\ped{g} \\
	\textbf{Descrizione}: l'utente Desktop\ped{g} inserisce un Elemento\ped{g} di tipo testo all'interno del Frame\ped{g}. \\
	\textbf{Precondizione}: il sistema ha una presentazione aperta in modalità modifica e l'utente desidera aggiungere un nuovo Elemento\ped{g} testuale.	\\
	\textbf{Postcondizione}: nel Frame\ped{g} è presente un nuovo Elemento\ped{g} testo.	\\
	\textbf{Scenario principale}:
	\begin{enumerate}
		\item L'utente Desktop\ped{g} seleziona l'opzione di inserimento testo;
		\item L'utente Desktop\ped{g} inserisce il testo desiderato.
	\end{enumerate}
	}
\subsubsection{UC 1.3.3.2 - Inserimento di un Elemento\ped{g} immagine}{
	\label{uc1.3.3.2}
	\textbf{Attori}: utente Desktop\ped{g} \\
	\textbf{Descrizione}: l'utente Desktop\ped{g} inserisce un Elemento\ped{g} di tipo immagine all'interno del Frame\ped{g}. \\
	\textbf{Precondizione}: il sistema ha una presentazione aperta in modalità modifica e l'utente desidera aggiungere un Elemento\ped{g} immagine.	\\
	\textbf{Postcondizione}: nel Frame\ped{g} è presente un nuovo Elemento\ped{g} immagine.	\\
	\textbf{Scenario principale}:
	\begin{enumerate}
		\item L'utente Desktop\ped{g} seleziona l'opzione di inserimento immagine;
		\item L'utente Desktop\ped{g} inserisce l'immagine desiderata tra quelle rese disponibili dal sistema oppure tra quelle presenti in locale o nello spazio sul Server\ped{g}.
	\end{enumerate}
	}
\subsubsection{UC 1.3.3.3 - Inserimento di un Elemento\ped{g} audio/video}{
	\label{uc1.3.3.3}
	\textbf{Attori}: utente Desktop\ped{g} \\
	\textbf{Descrizione}: l'utente Desktop\ped{g} inserisce un Elemento\ped{g} di tipo audio o video all'interno del Frame\ped{g}. \\
	\textbf{Precondizione}: il sistema ha una presentazione aperta in modalità modifica e l'utente desidera aggiungere un Elemento\ped{g} audio o video.	\\
	\textbf{Postcondizione}: nel Frame\ped{g} è presente un nuovo Elemento\ped{g} audio o video.	\\
	\textbf{Scenario principale}:
	\begin{enumerate}
		\item L'utente Desktop\ped{g} seleziona l'opzione di inserimento audio o video;
		\item L'utente Desktop\ped{g} inserisce l'audio o il video desiderato tra quelli presenti in locale oppure come Link\ped{g} ad uno presente in rete.
	\end{enumerate}
	}
\subsubsection{UC 1.3.3.4 - Modifica di un Elemento\ped{g} testo}{
	\label{uc1.3.3.4}
	\textbf{Attori}: utente Desktop\ped{g} \\
	\textbf{Descrizione}: l'utente Desktop\ped{g} modifica un Elemento\ped{g} testo presente all'interno del Frame\ped{g}. \\
	\textbf{Precondizione}: il sistema ha una presentazione aperta in modalità modifica e l'utente desidera modificare il contenuto di un Elemento\ped{g} testo.	\\
	\textbf{Postcondizione}: nel Frame\ped{g} è presente un Elemento\ped{g} testo modificato.	\\
	\textbf{Scenario principale}:
	\begin{enumerate}
		\item L'utente Desktop\ped{g} seleziona un Elemento\ped{g} testo;
		\item L'utente Desktop\ped{g} modifica il testo selezionato.
	\end{enumerate}
	}
\subsubsection{UC 1.3.3.5 - Modifica dimensione di un Elemento\ped{g} immagine}{
	\label{uc1.3.3.5}
	\textbf{Attori}: utente Desktop\ped{g} \\
	\textbf{Descrizione}: l'utente Desktop\ped{g} ridimensiona un Elemento\ped{g} immagine presente all'interno del Frame\ped{g}. \\
	\textbf{Precondizione}: il sistema ha una presentazione aperta in modalità modifica e l'utente desidera modificare la dimensione di un Elemento\ped{g} immagine.	\\
	\textbf{Postcondizione}: nel Frame\ped{g} è presente un Elemento\ped{g} immagine ridimensionato.	\\
	\textbf{Scenario principale}:
	\begin{enumerate}
		\item L'utente Desktop\ped{g} seleziona un Elemento\ped{g} immagine;
		\item L'utente Desktop\ped{g} ridimensiona l'Elemento\ped{g} immagine.
	\end{enumerate}
	}
\subsubsection{UC 1.3.3.6 - Modifica dimensione di un Elemento\ped{g} audio/video}{
	\label{uc1.3.3.6}
	\textbf{Attori}: utente Desktop\ped{g} \\
	\textbf{Descrizione}: l'utente Desktop\ped{g} ridimensiona un Elemento\ped{g} audio o video presente all'interno del Frame\ped{g}. \\
	\textbf{Precondizione}: il sistema ha una presentazione aperta in modalità modifica e l'utente desidera modificare la dimensione di un Elemento\ped{g} audio o video.	\\
	\textbf{Postcondizione}: nel Frame\ped{g} è presente un Elemento\ped{g} audio o un video ridimensionato.	\\
	\textbf{Scenario principale}:
	\begin{enumerate}
		\item L'utente Desktop\ped{g} seleziona un Elemento\ped{g} audio o video;
		\item L'utente Desktop\ped{g} ridimensiona l'Elemento\ped{g} audio o video.
	\end{enumerate}
	}
\subsubsection{UC 1.3.3.7 - Spostamento di un elemento}{
	\label{uc1.3.3.7}
	\textbf{Attori}: utente Desktop\ped{g} \\
	\textbf{Descrizione}: l'utente Desktop\ped{g} sposta un Elemento\ped{g} qualsiasi all'interno del Frame\ped{g}. \\
	\textbf{Precondizione}: il sistema ha una presentazione aperta in modalità modifica e l'utente desidera modificare la posizione di un Elemento\ped{g}.	\\
	\textbf{Postcondizione}: nel Frame\ped{g} è presente un Elemento\ped{g} spostato.	\\
	\textbf{Scenario principale}:
	\begin{enumerate}
		\item L'utente Desktop\ped{g} seleziona un Elemento\ped{g};
		\item L'utente Desktop\ped{g} trascina l'Elemento\ped{g} selezionato all'interno del Frame\ped{g}.
	\end{enumerate}
	}
\subsubsection{UC 1.3.3.8 - Modifica rotazione di un frame}{
	\label{uc1.3.3.8}
	\textbf{Attori}: utente Desktop\ped{g} \\
	\textbf{Descrizione}: l'utente Desktop\ped{g} modifica la rotazione di un Frame\ped{g}. \\
	\textbf{Precondizione}: il sistema ha una presentazione aperta in modalità modifica e l'utente desidera modificare la rotazione di un Frame\ped{g}.	\\
	\textbf{Postcondizione}: il Frame\ped{g} ha cambiato la propria rotazione.	\\
	\textbf{Scenario principale}:
	\begin{enumerate}
		\item L'utente Desktop\ped{g} seleziona un Frame\ped{g};
		\item L'utente Desktop\ped{g} ne modifica la rotazione.
	\end{enumerate}
	}
\subsubsection{UC 1.3.3.9 - Inserimento di un Elemento\ped{g} scelta}{
	\label{uc1.3.3.9}
	\textbf{Attori}: utente Desktop\ped{g} \\
	\textbf{Descrizione}: l'utente Desktop\ped{g} inserisce all'interno di un Frame\ped{g} un Elemento\ped{g} scelta verso un altro Frame\ped{g} non appartenente allo stesso Percorso\ped{g} di presentazione del Frame\ped{g}. \\
	\textbf{Precondizione}: il sistema ha una presentazione aperta in modalità modifica e l'utente desidera aggiungere un Elemento\ped{g} scelta.	\\
	\textbf{Postcondizione}: l'utente Desktop\ped{g} ha inserito una nuova scelta.	\\
	\textbf{Scenario principale}:
	\begin{enumerate}
		\item L'utente Desktop\ped{g} seleziona l'opzione di inserimento scelta;
		\item L'utente Desktop\ped{g} seleziona un altro Frame\ped{g};
		\item L'utente Desktop\ped{g} assegna il nome all'Elemento\ped{g} scelta.
	\end{enumerate}
	}
\subsubsection{UC 1.3.3.10 - Modifica di un Elemento\ped{g} scelta}{
	\label{uc1.3.3.10}
	\textbf{Attori}: utente Desktop\ped{g} \\
	\textbf{Descrizione}: l'utente Desktop\ped{g} modifica il testo assegnato ad un Elemento\ped{g} scelta presente all'interno del Frame\ped{g}. \\
	\textbf{Precondizione}: il sistema ha una presentazione aperta in modalità modifica e l'utente desidera modificare il nome assegnato ad un Elemento\ped{g} a scelta.	\\
	\textbf{Postcondizione}: l'utente Desktop\ped{g} ha modificato il nome di un Elemento\ped{g} scelta.	\\
	\textbf{Scenario principale}:
	\begin{enumerate}
		\item L'utente Desktop\ped{g} seleziona un Elemento\ped{g} scelta;
		\item L'utente Desktop\ped{g} modifica il testo del nome assegnato all'Elemento\ped{g} scelta.
	\end{enumerate}
	}
\subsubsection{UC 1.3.3.11 - Modifica della dimensione del frame}{
	\label{uc1.3.3.11}
	\textbf{Attori}: utente Desktop\ped{g} \\
	\textbf{Descrizione}: l'utente Desktop\ped{g} ridimensiona il Frame\ped{g}. \\
	\textbf{Precondizione}: il sistema presenta un Frame\ped{g} selezionato e l'utente desidera modificarne la dimensione.	\\
	\textbf{Postcondizione}: la presentazione contiene un Frame\ped{g} con una nuova dimensione.	\\
	\textbf{Scenario principale}:
	\begin{enumerate}
		\item L'utente Desktop\ped{g} seleziona l'opzione per la modifica della dimensione del Frame\ped{g};
		\item L'utente Desktop\ped{g} ridimensiona il Frame\ped{g}.
	\end{enumerate}
	}
\subsubsection{UC 1.3.3.12 - Modifica della forma di un frame}{
	\label{uc1.3.3.12}
	\textbf{Attori}: utente Desktop\ped{g} \\
	\textbf{Descrizione}: l'utente Desktop\ped{g} modifica la forma del Frame\ped{g}. \\
	\textbf{Precondizione}: il sistema presenta un Frame\ped{g} selezionato e l'utente desidera modificarne la forma.	\\
	\textbf{Postcondizione}: la presentazione contiene un Frame\ped{g} con una nuova forma.	\\
	\textbf{Scenario principale}:
	\begin{enumerate}
		\item L'utente Desktop\ped{g} seleziona l'opzione per la modifica della forma del Frame\ped{g};
		\item L'utente Desktop\ped{g} seleziona la nuova forma del Frame\ped{g}.
	\end{enumerate}}
\subsubsection{UC 1.3.3.13 - Modifica dello spessore del bordo del frame}{
	\label{uc1.3.3.13}
	\textbf{Attori}: utente Desktop\ped{g} \\
	\textbf{Descrizione}: l'utente Desktop\ped{g} modifica lo spessore del bordo del Frame\ped{g}. \\
	\textbf{Precondizione}: il sistema presenta un Frame\ped{g} selezionato e l'utente desidera modificarne lo spessore del bordo.	\\
	\textbf{Postcondizione}: la presentazione contiene un Frame\ped{g} con un bordo dal nuovo spessore.	\\
	\textbf{Scenario principale}:
	\begin{enumerate}
		\item L'utente Desktop\ped{g} seleziona l'opzione per la modifica dello spessore del bordo del Frame\ped{g};
		\item L'utente Desktop\ped{g} ridimensiona lo spessore del bordo del Frame\ped{g}.
	\end{enumerate}
	}
\subsubsection{UC 1.3.3.14 - Modifica del colore del bordo del frame}{
	\label{uc1.3.3.14}
	\textbf{Attori}: utente Desktop\ped{g} \\
	\textbf{Descrizione}: l'utente Desktop\ped{g} modifica il colore del bordo del Frame\ped{g}. \\
	\textbf{Precondizione}: il sistema presenta un Frame\ped{g} selezionato e l'utente desidera modificarne il colore del bordo.	\\
	\textbf{Postcondizione}: la presentazione contiene un Frame\ped{g} con un bordo con un nuovo colore.	\\
	\textbf{Scenario principale}:
	\begin{enumerate}
		\item L'utente Desktop\ped{g} seleziona l'opzione per la modifica del colore del bordo del Frame\ped{g};
		\item L'utente Desktop\ped{g} seleziona il nuovo colore del bordo del Frame\ped{g}.
	\end{enumerate}
	}
\subsubsection{UC 1.3.3.15 - Modifica dello sfondo del frame}{
	\label{uc1.3.3.15}
	\textbf{Attori}: utente Desktop\ped{g} \\
	\textbf{Descrizione}: l'utente Desktop\ped{g} modifica lo sfondo del Frame\ped{g}. \\
	\textbf{Precondizione}: il sistema presenta un Frame\ped{g} selezionato e l'utente desidera modificarne lo sfondo.	\\
	\textbf{Postcondizione}: la presentazione contiene un Frame\ped{g} con un nuovo sfondo.	\\
	\textbf{Scenario principale}:
	\begin{enumerate}
		\item L'utente Desktop\ped{g} seleziona l'opzione per la modifica dello sfondo del Frame\ped{g};
		\item L'utente Desktop\ped{g} può selezionare un nuovo colore di sfondo del Frame\ped{g};
		\item L'utente Desktop\ped{g} può selezionare una nuova immagine di sfondo del Frame\ped{g}.
	\end{enumerate}
	}
\subsubsection{UC 1.3.4 - Eliminazione frame}{
	\label{uc1.3.4}
	\textbf{Attori}: utente Desktop\ped{g} \\
	\textbf{Descrizione}: l'utente può eliminare un Frame\ped{g} dal piano della presentazione\ped{g}. \\
	\textbf{Precondizione}: il sistema ha una presentazione aperta in modalità modifica.	\\
	\textbf{Postcondizione}: l'utente ha eliminato un Frame\ped{g} dal piano della presentazione\ped{g}.	\\
	\textbf{Scenario principale}:
	\begin{enumerate}
		\item L'utente seleziona un Frame\ped{g} nel piano della presentazione\ped{g};
		\item L'utente seleziona l'azione di cancellazione del Frame\ped{g};
		\item L'utente conferma la cancellazione del Frame\ped{g}.
	\end{enumerate}
	\textbf{Scenari alternativi}: 
	\begin{itemize}
		\item L'utente non conferma l'azione di cancellazione e si ritorna alla precondizione.
	\end{itemize}
}
\subsubsection{UC 1.3.5 - Impostazione sfondo}{
	\label{uc1.3.5}
	\begin{figure}[H]
		\centering
		\includegraphics[scale=0.75]{\imgs {UC1.3.5}.jpg} %inserire il diagramma UML
		\label{fig:uc1.3.5}
		\caption{Caso d'uso 1.3.5: Impostazione dello sfondo}
	\end{figure}
	\textbf{Attori}: utente Desktop\ped{g} \\
	\textbf{Descrizione}: l'utente può impostare un'immagine o un colore di sfondo alla presentazione. \\
	\textbf{Precondizione}: il sistema ha una presentazione aperta in modalità modifica.	\\
	\textbf{Postcondizione}: l'utente ha impostato una sfondo sul piano della presentazione\ped{g}.	\\
	\textbf{Scenario principale}:
	\begin{enumerate}
		\item L'utente inserisce un'immagine o un colore di sfondo in un'area definita \S\hyperref[uc1.3.5.1]{(UC 1.3.5.1)};
		\item L'utente può spostare l'area di sfondo sul piano della presentazione\ped{g} \S\hyperref[uc1.3.5.2]{(UC 1.3.5.2)};
		\item L'utente può ridimensionare l'area di sfondo sul piano della presentazione\ped{g} \S\hyperref[uc1.3.5.3]{(UC 1.3.5.3)};
		\item L'utente conferma l'inserimento dello sfondo.
	\end{enumerate}
	\textbf{Scenari alternativi}: 
	\begin{itemize}
		\item L'utente non conferma l'inserimento dello sfondo e si ritorna alla precondizione.
	\end{itemize}
}
\subsubsection{UC 1.3.5.1 - Inserimento sfondo}{
	\label{uc1.3.5.1}
	\begin{figure}[H]
		\centering
		\includegraphics[scale=0.75]{\imgs {UC1.3.5.1}.jpg} %inserire il diagramma UML
		\label{fig:uc1.3.5.1}
		\caption{Caso d'uso 1.3.5.1: Inserimento di uno sfondo}
	\end{figure}
	\textbf{Attori}: utente Desktop\ped{g} \\
	\textbf{Descrizione}: l'utente può inserire una immagine o un colore di sfondo nel piano della presentazione\ped{g}. \\
	\textbf{Precondizione}: il sistema ha una presentazione aperta in modalità modifica.	\\
	\textbf{Postcondizione}: l'utente ha inserito uno sfondo nel piano della presentazione\ped{g}.	\\
	\textbf{Scenario principale}:
	\begin{enumerate}
		\item L'utente può selezionare un'immagine da usare come sfondo \S\hyperref[uc1.3.5.1.1]{(UC 1.3.5.1)};
		\item L'utente può selezionare un colore da usare come sfondo \S\hyperref[uc1.3.5.1.2]{(UC 1.3.5.2)};
		\item L'utente definisce un'area nel piano della presentazione\ped{g} per ospitare lo sfondo \S\hyperref[uc1.3.5.1.3]{(UC 1.3.5.3)};
		\item L'utente conferma l'inserimento dello sfondo \S\hyperref[uc1.3.5.1.4]{(UC 1.3.5.4)}.
	\end{enumerate}
}
\subsubsection{UC 1.3.5.1.1 - Selezione immagine sfondo}{
	\label{uc1.3.5.1.1}
	\textbf{Attori}: utente Desktop\ped{g} \\
	\textbf{Descrizione}: l'utente può selezionare un'immagine per lo sfondo. \\
	\textbf{Precondizione}: il sistema ha una presentazione aperta in modalità modifica e l'utente ha selezionato l'opzione per impostare lo sfondo.	\\
	\textbf{Postcondizione}: l'utente ha selezionato un'immagine per lo sfondo.	\\
	\textbf{Scenario principale}:
	\begin{enumerate}
		\item L'utente scorre i File\ped{g} immagine resi disponibili dal sistema oppure può scegliere di caricare un'immagine presente in locale;
		\item L'utente seleziona un'immagine;
		\item L'utente conferma la selezione.
	\end{enumerate}
}
\subsubsection{UC 1.3.5.1.2 - Selezione colore sfondo}{
	\label{uc1.3.5.1.2}
	\textbf{Attori}: utente Desktop\ped{g} \\
	\textbf{Descrizione}: l'utente può selezionare un colore per lo sfondo. \\
	\textbf{Precondizione}: il sistema ha una presentazione aperta in modalità modifica e l'utente ha selezionato l'opzione per impostare il colore di sfondo.	\\
	\textbf{Postcondizione}: l'utente ha selezionato un colore per lo sfondo.	\\
	\textbf{Scenario principale}:
	\begin{enumerate}
		\item L'utente scorre i colori definiti;
		\item L'utente seleziona un colore;
		\item L'utente conferma la selezione.
	\end{enumerate}
}
\subsubsection{UC 1.3.5.1.3 - Definizione area sfondo}{
	\label{uc1.3.5.1.3}
	\textbf{Attori}: utente Desktop\ped{g} \\
	\textbf{Descrizione}: l'utente può definire un'area per lo sfondo. \\
	\textbf{Precondizione}: il sistema ha una presentazione aperta in modalità modifica e l'utente ha selezionato un'immagine o un colore per lo sfondo.	\\
	\textbf{Postcondizione}: l'utente ha definito un'area per lo sfondo.	\\
	\textbf{Scenario principale}:
	\begin{enumerate}
		\item L'utente seleziona un'area nel piano di presentazione;
		\item L'utente conferma la definizione dell'area .
	\end{enumerate}
}
\subsubsection{UC 1.3.5.1.4 - Conferma sfondo}{
	\label{uc1.3.5.1.4}
	\textbf{Attori}: utente Desktop\ped{g} \\
	\textbf{Descrizione}: l'utente deve confermare l'inserimento di uno sfondo. \\
	\textbf{Precondizione}: l'utente ha selezionato un colore o un'immagine per lo sfondo e ne ha definito l'area.	\\
	\textbf{Postcondizione}: l'utente ha inserito uno sfondo.	\\
	\textbf{Scenario principale}:
	\begin{enumerate}
		\item L'utente conferma l'inserimento dello sfondo.
	\end{enumerate}
}
\subsubsection{UC 1.3.5.2 - Spostamento sfondo}{
	\label{uc1.3.5.2}
	\textbf{Attori}: utente Desktop\ped{g} \\
	\textbf{Descrizione}: l'utente può spostare l'area di sfondo nel piano della presentazione\ped{g}. \\
	\textbf{Precondizione}: il sistema ha una presentazione aperta in modalità modifica e vi è presente un'area di sfondo.	\\
	\textbf{Postcondizione}: l'utente ha spostato uno sfondo nel piano della presentazione\ped{g}.	\\
	\textbf{Scenario principale}:
	\begin{enumerate}
		\item L'utente seleziona l'area di sfondo;
		\item L'utente sposta l'area di sfondo sul piano della presentazione\ped{g}.
	\end{enumerate}
}
\subsubsection{UC 1.3.5.3 - Ridimensionamento sfondo}{
	\label{uc1.3.5.3}
	\textbf{Attori}: utente Desktop\ped{g} \\
	\textbf{Descrizione}: l'utente può ridimensionare l'area di sfondo nel piano della presentazione\ped{g}. \\
	\textbf{Precondizione}: il sistema ha una presentazione aperta in modalità modifica e vi è presente un'area di sfondo.	\\
	\textbf{Postcondizione}: l'utente ha ridimensionato uno sfondo nel piano della presentazione\ped{g}.	\\
	\textbf{Scenario principale}:
	\begin{enumerate}
		\item L'utente seleziona l'area dello sfondo;
		\item L'utente seleziona l'azione 'ridimensionamento';
		\item L'utente ridefinisce l'area dello sfondo;
		\item L'utente conferma la nuova area.
	\end{enumerate}
}
\subsubsection{UC 1.3.6 - Definizione percorsi\ped{g} di visualizzazione}{
	\label{uc1.3.6}
	\begin{figure}[H]
		\centering
		\includegraphics[scale=0.75]{\imgs {UC1.3.6}.jpg} %inserire il diagramma UML
		\label{fig:uc1.3.6}
		\caption{Caso d'uso 1.3.6: Definizione dei percorsi\ped{g} di visualizzazione}
	\end{figure}
	\textbf{Attori}: utente Desktop\ped{g} \\
	\textbf{Descrizione}: l'utente può definire la sequenza dei Frame\ped{g} visualizzati di una presentazione. \\
	\textbf{Precondizione}: il sistema è entrato in modalità modifica percorsi\ped{g} e la presentazione caricata presenta almeno un Frame\ped{g}.	\\
	\textbf{Postcondizione}: l'utente ha definito la sequenza dei Frame\ped{g} visualizzati in fase di esecuzione della presentazione aperta.	\\
	\textbf{Scenario principale}:
	\begin{enumerate}
		\item L'utente passa alla modalità "modifica percorsi\ped{g} di visualizzazione";
		\item L'utente imposta un Frame\ped{g} come primo Elemento\ped{g} visualizzato di una presentazione;
		\item L'utente può definire nuove transizioni tra Frame\ped{g};
		\item L'utente può escludere un Frame\ped{g} parte di un Percorso\ped{g};
		\item L'utente può eliminare una transizione.
	\end{enumerate}
}
\subsubsection{UC 1.3.6.1 - Imposta Frame\ped{g} iniziale}{
	\label{uc1.3.6.1}
	\textbf{Attori}: utente Desktop\ped{g} \\
	\textbf{Descrizione}: l'utente può impostare il primo Frame\ped{g} visualizzato in fase di esecuzione di una presentazione. \\
	\textbf{Precondizione}: il sistema presenta un Frame\ped{g} selezionato e l'utente desidera impostarlo come primo Frame\ped{g}.	\\
	\textbf{Postcondizione}: l'utente ha impostato il primo Frame\ped{g} da visualizzare in fase di esecuzione della presentazione aperta.	\\
	\textbf{Scenario principale}:
	\begin{enumerate}
		\item L'utente seleziona un Frame\ped{g} sul piano della presentazione\ped{g};
		\item L'utente seleziona l'azione "inizio" a lato del Frame\ped{g}.
	\end{enumerate}
}
\subsubsection{UC 1.3.6.2 - Nuova transizione}{
	\label{uc1.3.6.2}
	\textbf{Attori}: utente Desktop\ped{g} \\
	\textbf{Descrizione}: l'utente può definire una nuova transizione tra due Frame\ped{g}. \\
	\textbf{Precondizione}: il sistema presenta un Frame\ped{g} selezionato e l'utente desidera impostare il successivo nel Percorso\ped{g} di visualizzazione.	\\
	\textbf{Postcondizione}: l'utente ha definito una transizione tra due Frame\ped{g}.	\\
	\textbf{Scenario principale}:
	\begin{enumerate}
		\item L'utente seleziona una Frame\ped{g} nel piano della presentazione\ped{g};
		\item L'utente seleziona l'azione "aggiungi transizione";
		\item L'utente seleziona il Frame\ped{g} destinazione.
	\end{enumerate}
}
\subsubsection{UC 1.3.6.3 - Escludi Frame\ped{g} dal percorso}{
	\label{uc1.3.6.3}
	\textbf{Attori}: utente Desktop\ped{g} \\
	\textbf{Descrizione}: l'utente può escludere la visualizzazione di un Frame\ped{g} dal Percorso\ped{g}. \\
	\textbf{Precondizione}: il sistema presenta un Frame\ped{g} selezionato e l'utente desidera escluderlo dal Percorso\ped{g} di presentazione.	\\
	\textbf{Postcondizione}: l'utente ha rimosso un Frame\ped{g} dal Percorso\ped{g} di visualizzazione.	\\
	\textbf{Scenario principale}:
	\begin{enumerate}
		\item L'utente seleziona un Frame\ped{g} nel piano della presentazione\ped{g};
		\item L'utente seleziona l'azione per l'esclusione del Frame\ped{g} dal Percorso\ped{g} di presentazione;
		\item L'utente conferma l'eliminazione del Frame\ped{g}.
	\end{enumerate}
}
\subsubsection{UC 1.3.6.4 - Elimina transizione}{
	\label{uc1.3.6.4}
	\textbf{Attori}: utente Desktop\ped{g} \\
	\textbf{Descrizione}: l'utente può eliminare una transizione tra due Frame\ped{g} del piano di presentazione. \\
	\textbf{Precondizione}: il sistema presenta una transizione selezionata e l'utente desidera eliminarla.	\\
	\textbf{Postcondizione}: l'utente ha eliminato una transizione tra due Frame\ped{g}.	\\
	\textbf{Scenario principale}:
	\begin{enumerate}
		\item L'utente seleziona una transizione nel piano della presentazione\ped{g};
		\item L'utente seleziona l'azione "cancella transizione".
	\end{enumerate}
}
\subsubsection{UC 1.3.7 - Gestione Desktop\ped{g} bookmark}{
	\label{uc1.3.7}
	\begin{figure}[H]
		\centering
		\includegraphics[scale=0.75]{\imgs {UC1.3.7}.jpg} %inserire il diagramma UML
		\label{fig:uc1.3.7}
		\caption{Caso d'uso 1.3.7: Gestione Desktop\ped{g} dei bookmark}
	\end{figure}
	\textbf{Attori}: utente Desktop\ped{g} \\
	\textbf{Descrizione}: l'utente può inserire e cancellare Bookmark\ped{g} da un Frame\ped{g}. \\
	\textbf{Precondizione}: il sistema ha una presentazione aperta in modalità modifica.	\\
	\textbf{Postcondizione}: l'utente ha inserito o eliminato Bookmark\ped{g} da un Frame\ped{g}.	\\
	\textbf{Scenario principale}:
	\begin{enumerate}
		\item L'utente può inserire un nuovo Bookmark\ped{g} in un Frame\ped{g};
		\item L'utente può eliminare un Bookmark\ped{g} da un Frame\ped{g}.
	\end{enumerate}
}
\subsubsection{UC 1.3.7.1 - Inserimento bookmark}{
	\label{uc1.3.7.1}
	\textbf{Attori}: utente Desktop\ped{g} \\
	\textbf{Descrizione}: l'utente può inserire un nuovo Bookmark\ped{g} su un Frame\ped{g}. \\
	\textbf{Precondizione}: il sistema ha una presentazione aperta in modalità modifica e l'utente desidera aggiungere un nuovo Bookmark\ped{g}.	\\
	\textbf{Postcondizione}: l'utente ha inserito un Bookmark\ped{g} ad un Frame\ped{g}.	\\
	\textbf{Scenario principale}:
	\begin{enumerate}
		\item L'utente seleziona un Frame\ped{g} associato al Frame\ped{g} in cui inserire il Bookmark\ped{g};
		\item L'utente selezione l'azione inserisci Bookmark\ped{g} a lato del numero del Frame\ped{g} corrispondente.
	\end{enumerate}
}
\subsubsection{UC 1.3.7.2 - Eliminazione bookmark}{
	\label{uc1.3.7.2}
	\textbf{Attori}: utente Desktop\ped{g} \\
	\textbf{Descrizione}: l'utente può cancellare Bookmark\ped{g} da un Frame\ped{g}. \\
	\textbf{Precondizione}: il sistema ha una presentazione aperta in modalità modifica e l'utente desidera rimuovere un Bookmark\ped{g}.	\\
	\textbf{Postcondizione}: l'utente ha eliminato un Bookmark\ped{g} da un Frame\ped{g}.	\\
	\textbf{Scenario principale}:
	\begin{enumerate}
		\item L'utente seleziona il Frame\ped{g} associato al Frame\ped{g} da cui eliminare il Bookmark\ped{g};
		\item L'utente seleziona l'azione "elimina bookmark" a lato del Frame\ped{g} e del numero del Frame\ped{g} a cui è associato il Bookmark\ped{g} da eliminare.
	\end{enumerate}
}
\subsubsection{UC 1.3.8 - Definizione effetti di transizione}{
	\label{uc1.3.8}
	\begin{figure}[H]
		\centering
		\includegraphics[scale=0.75]{\imgs {UC1.3.8}.jpg} %inserire il diagramma UML
		\label{fig:uc1.3.8}
		\caption{Caso d'uso 1.3.8: Definizione degli effetti di transizione tra frame}
	\end{figure}
	\textbf{Attori}: utente Desktop\ped{g} \\
	\textbf{Descrizione}: l'utente può definire gli effetti di transizione da un Frame\ped{g} ad un altro; l'effetto verrà attribuito al Frame\ped{g} selezionato e verrà applicato nel momento in cui si passerà da quel Frame\ped{g} a quello successivo nel Percorso\ped{g} di presentazione. \\
	\textbf{Precondizione}: il sistema ha una presentazione aperta in modalità modifica; l'utente ha selezionato l'opzione per la definizione degli effetti di transizione.	\\
	\textbf{Postcondizione}: l'utente ha modificato l'effetto di transizione per le quelle desiderate.	\\
	\textbf{Scenario principale}:
	\begin{enumerate}
		\item L'utente seleziona il Frame\ped{g} a cui modificare l'effetto di transizione;
		\item L'utente seleziona l'azione "effetto transizione";
		\item L'utente seleziona un effetto tra quelli definiti;
		\item L'utente può selezionare la velocità di transizione;
		\item L'utente conferma il cambiamento.
	\end{enumerate}
	\textbf{Scenari alternativi}:
	\begin{itemize}
		\item L'utente non conferma la modifica e si ritorna alla precondizione.
	\end{itemize}
	}
\subsubsection{UC 1.3.8.1 - Selezione transizione}{
	\label{uc1.3..8.1}
	\textbf{Attori}: utente Desktop\ped{g} \\
	\textbf{Descrizione}: l'utente vuole selezionare una transizione da applicare ad un Frame\ped{g}. \\
	\textbf{Precondizione}: il sistema ha una presentazione aperta in modalità modifica e l'utente ha selezionato l'opzione effetti transazioni.	\\
	\textbf{Postcondizione}: l'utente ha selezionato una transizione.	\\
	\textbf{Scenario principale}:
	\begin{enumerate}
		\item L'utente seleziona una transizione per il Frame\ped{g} selezionato.
	\end{enumerate}
}
\subsubsection{UC 1.3.8.2 - Selezione effetto}{
	\label{uc1.3.8.2}
	\textbf{Attori}: utente Desktop\ped{g} \\
	\textbf{Descrizione}: l'utente può selezionare un effetto grafico di transizione da associare ad una transizione. \\
	\textbf{Precondizione}: il sistema ha una transizione selezionata e l'utente ha selezionato l'opzione effetti transazioni.	\\
	\textbf{Postcondizione}: l'utente ha selezionato un effetto di transizione per una transizione.	\\
	\textbf{Scenario principale}:
	\begin{enumerate}
		\item L'utente seleziona un effetto per una transizione.
	\end{enumerate}
}
\subsubsection{UC 1.3.8.3 - Conferma effetto}{
	\label{uc1.3.8.3}
	\textbf{Attori}: utente Desktop\ped{g} \\
	\textbf{Descrizione}: l'utente può confermare la modifica di un effetto grafico su di una transizione. \\
	\textbf{Precondizione}: l'utente una transizione selezionata e l'utente ha selezionato l'opzione effetti transazioni, un effetto grafico e una velocità di visualizzazione per la transizione considerata.	\\
	\textbf{Postcondizione}: l'utente ha confermato la modifica di un effetto grafico su di una transizione.\\
	\textbf{Scenario principale}:
	\begin{enumerate}
		\item L'utente conferma l'effetto selezionato.
	\end{enumerate}
}
\subsubsection{UC 1.3.8.4 - Imposta velocità transizione}{
	\label{uc1.3..8.4}
	\textbf{Attori}: utente Desktop\ped{g} \\
	\textbf{Descrizione}: l'impostare una velocità di visualizzazione di un effetto grafico su una transizione. \\
	\textbf{Precondizione}: il sistema ha una transizione selezionata e l'utente ha selezionato l'opzione effetti transazioni e un effetto grafico per la transizione considerata.	\\
	\textbf{Postcondizione}: l'utente ha impostato la velocità di visualizzazione di un effetto grafico su una transizione.\\
	\textbf{Scenario principale}:
	\begin{enumerate}
		\item L'utente imposta la velocità di visualizzazione di un effetto grafico.
	\end{enumerate}
}
\subsubsection{UC 1.3.9 - Definizione tempo di permanenza}{
	\label{uc1.3.9}
	\textbf{Attori}: utente Desktop\ped{g} \\
	\textbf{Descrizione}: l'utente può definire il temo di permanenza per ogni Frame\ped{g} in modalità esecuzione automatica di una presentazione. \\
	\textbf{Precondizione}: il sistema ha una presentazione aperta in modalità modifica.	\\
	\textbf{Postcondizione}: l'utente ha modificato il tempo di permanenza sui Frame\ped{g} in modalità esecuzione automatica della presentazione.	\\
	\textbf{Scenario principale}:
	\begin{enumerate}
		\item L'utente seleziona la voce in menu "tempo di permanenza";
		\item L'utente sceglie il tempo di permanenza in secondi;
		\item L'utente conferma la scelta.
	\end{enumerate}
	\textbf{Scenari alternativi}:
	\begin{itemize}
		\item L'utente non conferma la scelta  e si ritorna alla precondizione.
	\end{itemize}
}
\subsubsection{UC 1.3.10 - Inserimento di un Elemento\ped{g} SVG}{
	\label{uc1.3.10}
	\textbf{Attori}: utente Desktop\ped{g} \\
	\textbf{Descrizione}: l'utente Desktop\ped{g} inserisce un Elemento\ped{g} di tipo SVG nel piano della presentazione\ped{g} o all'interno di un Frame\ped{g}. \\
	\textbf{Precondizione}: il sistema ha una presentazione aperta in modalità modifica e l'utente desidera aggiungere un Elemento\ped{g} SVG.	\\
	\textbf{Postcondizione}: nel piano della presentazione\ped{g} o all'interno di un Frame\ped{g} è presente un nuovo Elemento\ped{g} SVG.	\\
	\textbf{Scenario principale}:
	\begin{enumerate}
		\item L'utente Desktop\ped{g} seleziona l'opzione di inserimento Elemento\ped{g} SVG;
		\item L'utente Desktop\ped{g} inserisce l'Elemento\ped{g} SVG desiderato tra quelli presenti in locale.
	\end{enumerate}
	}
\subsubsection{UC 1.3.11 - Modifica di un Elemento\ped{g} SVG}{
	\label{uc1.3.11}
	\begin{figure}[H]
		\centering
		\includegraphics[scale=0.75]{\imgs {UC1.3.11}.jpg} %inserire il diagramma UML
		\label{fig:uc1.3.11}
		\caption{Caso d'uso 1.3.11: Modifica di un Elemento\ped{g} SVG}
	\end{figure}
	\textbf{Attori}: utente Desktop\ped{g} \\
	\textbf{Descrizione}: l'utente Desktop\ped{g} può modificare dimensione e colore di un Elemento\ped{g} SVG. \\
	\textbf{Precondizione}: il sistema ha una presentazione aperta in modalità modifica e l'utente desidera modificare un Elemento\ped{g} SVG.	\\
	\textbf{Postcondizione}: l'Elemento\ped{g} SVG è stato modificato come l'utente desiderava.	\\
	\textbf{Scenario principale}:
	\begin{enumerate}
		\item L'utente Desktop\ped{g} può modificare le dimensioni di un Elemento\ped{g} SVG \S\hyperref[uc1.3.11.1]{(UC 1.3.11.1)};
		\item L'utente Desktop\ped{g} può modificare il colore di un Elemento\ped{g} SVG \S\hyperref[uc1.3.11.2]{(UC 1.3.11.2)}.
	\end{enumerate}
	}
\subsubsection{UC 1.3.11.1 - Modifica dimensioni di un Elemento\ped{g} SVG}{
	\label{uc1.3.11.1}
	\textbf{Attori}: utente Desktop\ped{g} \\
	\textbf{Descrizione}: l'utente Desktop\ped{g} può modificare le dimensioni di un Elemento\ped{g} SVG. \\
	\textbf{Precondizione}: il sistema ha una presentazione aperta in modalità modifica e l'utente desidera modificare le dimensioni di un Elemento\ped{g} SVG.	\\
	\textbf{Postcondizione}: le dimensioni dell'Elemento\ped{g} SVG sono state modificate.	\\
	\textbf{Scenario principale}:
	\begin{enumerate}
		\item L'utente Desktop\ped{g} seleziona un Elemento\ped{g} SVG;
		\item L'utente Desktop\ped{g} ridimensiona l'Elemento\ped{g} SVG.
	\end{enumerate}
	}
\subsubsection{UC 1.3.11.2 - Modifica colore di un Elemento\ped{g} SVG}{
	\label{uc1.3.11.2}
	\textbf{Attori}: utente Desktop\ped{g} \\
	\textbf{Descrizione}: l'utente Desktop\ped{g} può modificare il colore di un Elemento\ped{g} SVG. \\
	\textbf{Precondizione}: il sistema ha una presentazione aperta in modalità modifica e l'utente desidera modificare il colore di un Elemento\ped{g} SVG.	\\
	\textbf{Postcondizione}: il colore dell'Elemento\ped{g} SVG è stato cambiato.	\\
	\textbf{Scenario principale}:
	\begin{enumerate}
		\item L'utente Desktop\ped{g} seleziona un Elemento\ped{g} SVG;
		\item L'utente Desktop\ped{g} seleziona l'opzione per la modifica del colore;
		\item L'utente Desktop\ped{g} modifica il colore con quello desiderato.
	\end{enumerate}
	}
\subsubsection{UC 1.3.12 - Eliminazione di un elemento}{
		\label{uc1.3.12}
		\textbf{Attori}: utente Desktop\ped{g} \\
		\textbf{Descrizione}: l'utente Desktop\ped{g} elimina un Elemento\ped{g} qualsiasi dal piano della presentazione\ped{g}. \\
		\textbf{Precondizione}: il sistema ha una presentazione aperta in modalità modifica e l'utente desidera eliminare un Elemento\ped{g}.	\\
		\textbf{Postcondizione}: l'utente Desktop\ped{g} ha eliminato un Elemento\ped{g}.	\\
		\textbf{Scenario principale}:
		\begin{enumerate}
			\item L'utente Desktop\ped{g} seleziona un Elemento\ped{g};
			\item L'utente Desktop\ped{g} elimina l'Elemento\ped{g} selezionato.
		\end{enumerate}
		}
\subsubsection{UC 1.3.13 - Modifica rotazione di un elemento}{
	\label{uc1.3.13}
	\textbf{Attori}: utente Desktop\ped{g} \\
	\textbf{Descrizione}: l'utente Desktop\ped{g} modifica la rotazione di un Elemento\ped{g}. \\
	\textbf{Precondizione}: il sistema ha una presentazione aperta in modalità modifica e l'utente desidera modificare la rotazione di un Elemento\ped{g}.	\\
	\textbf{Postcondizione}: l'utente Desktop\ped{g} ha cambiato la rotazione di un Elemento\ped{g}.	\\
	\textbf{Scenario principale}:
	\begin{enumerate}
		\item L'utente Desktop\ped{g} seleziona un Elemento\ped{g};
		\item L'utente Desktop\ped{g} ne modifica la rotazione.
	\end{enumerate}
	}