\subsection{UC 1.12 - Annulla/Ripristina presentazione mobile}{
	\label{uc1.12}
	\begin{figure}[H]
		\centering
		\includegraphics[scale=0.70]{\imgs {UC1.12}.jpg} %inserire il diagramma UML
		\label{fig:uc1.12}
		\caption{Caso d'uso 1.12: Annulla/Ripristina presentazione mobile}
	\end{figure}
	\textbf{Attori}: Utente mobile\ped{g}. \\
	\textbf{Descrizione}:L'utente può annullare un comando selezionato o ripristinare un comando annullato
	\textbf{Precondizione}:  l'utente ha effettuato l'accesso al sistema in modalità modifica ad una presentazione in modalità mobile \\
	\textbf{Postcondizione}: l'utente ha annullato un comando o ripristinato un comando annullato  \\
	
\subsubsection{UC 1.12.1 - Annulla modifica}{
		\label{uc1.12.1}
		\textbf{Attori}: utente mobile\ped{g} \\
		\textbf{Descrizione}:L'utente può annullare l'ultimo comando di modifica tra:. \\
		\begin{itemize}
			\item eliminazione Frame\ped{g}
			\item spostamento Frame\ped{g}
			\item modifica Frame\ped{g}
			\item inserimento Bookmark\ped{g}
			\item cancellazione Bookmark\ped{g}
			\item modifica nella definizione di Percorso\ped{g}
		\end{itemize}
		\textbf{Precondizione}: Il sistema ha registrato almeno un comando di modifica della presentazione da parte dell'utente. \\
		\textbf{Postcondizione}: Il sistema ha ripristinato lo stato precedente all'ultimo comando di modifica della presentazione dell'utente, e ha memorizzato nello storico degli annullamenti il comando annullato.	\\
		\textbf{Scenario principale}:
		\begin{enumerate}
			\item l'utente seleziona l'azione annulla 
		\end{enumerate}
		}
		
\subsubsection{UC 1.12.2 - Ripristina modifica}{
		\label{uc1.12.2}
		\textbf{Attori}: Utente mobile\ped{g}. \\
		\textbf{Descrizione}: L'utente può ripristinare l'ultimo comando presente nello storico dei comandi annullati.
		\textbf{Precondizione}: Lo storico degli annullamenti di comando non è vuoto. \\
		\textbf{Postcondizione}: Il sistema ha rieseguito l'ultimo comando di modifica della presentazione presente nello storico dei comandi annullati, ed è stato rimosso il comando rieseguito dallo storico.
		}



