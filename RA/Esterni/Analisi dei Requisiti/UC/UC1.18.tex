\subsection{UC 1.18 - Esecuzione presentazione}{
	\label{uc1.18}
	\begin{figure}[H]
		\centering
		\includegraphics[scale=0.70]{\imgs {UC1.18}.jpg} %inserire il diagramma UML
		\label{fig:uc1.18}
		\caption{Caso d'uso 1.18: Esecuzione della presentazione}
	\end{figure}
	\textbf{Attori}: utente. \\
	\textbf{Descrizione}: l'utente non autenticato\ped{g} può eseguire una presentazione salvata in locale mentre un utente autenticato\ped{g} può eseguire una presentazione salvata nel proprio Account\ped{g}. L'esecuzione può avvenire in maniera automatica oppure controllandone manualmente l'avanzamento. \\
	\textbf{Precondizione}: una presentazione è già stata caricata e l'utente può avviarla nella modalità desiderata. \\
	\textbf{Postcondizione}: la presentazione è stata eseguita secondo il metodo scelto.	\\
	\textbf{Scenario principale}:
	\begin{enumerate}
		\item Scelta del metodo con cui eseguirla: automatica \S\hyperref[uc1.18.2]{(UC 1.18.2)} o manuale \S\hyperref[uc1.18.1]{(UC 1.18.1)}.
	\end{enumerate}
	\textbf{Scenari alternativi}: 
	\begin{itemize}
		\item Possibilità di passare da manuale ad automatico e viceversa \S\hyperref[uc1.18.3]{(UC 1.18.3)}  \S\hyperref[uc1.18.4]{(UC 1.18.4)}.
	\end{itemize}
	}
\subsubsection{UC 1.18.1 - Esecuzione manuale}{
		\label{uc1.18.1}
		\begin{figure}[H]
			\centering
			\includegraphics[scale=0.75]{\imgs {UC1.18.1}.jpg} %inserire il diagramma UML
			\label{fig:uc1.18.1}
			\caption{Caso d'uso 1.18.1: Esecuzione manuale della presentazione}
		\end{figure}
		\textbf{Attori}: utente. \\
		\textbf{Descrizione}: l'utente può scorrere la presentazione, scegliere il Percorso\ped{g} da intraprendere a run-time, cambiare livello di visualizzazione, tornare a Frame\ped{g} precedenti anche attraverso l'utilizzo dei Bookmark\ped{g} o passare alla presentazione automatica \\
		\textbf{Precondizione}: il sistema è funzionante e l'utente ha avviato la presentazione in manuale. \\
		\textbf{Postcondizione}: l'utente ha avviato la presentazione nel modo desiderato.	\\
		\textbf{Scenario principale}:
		\begin{enumerate}
			\item L'utente scorre la presentazione \S\hyperref[uc1.18.1.1]{(UC 1.18.1.1)};
			\item L'utente può scegliere il Percorso\ped{g} su cui continuare \S\hyperref[uc1.18.1.2]{(UC 1.18.1.2)};
			\item L'utente può saltare al Bookmark\ped{g} successivo \S\hyperref[uc1.18.1.3]{(UC 1.18.1.3)};
			\item L'utente può cambiare il livello di visualizzazione \S\hyperref[uc1.18.1.4]{(UC 1.18.1.4)};
			\item L'utente può ingrandire il contenuto \S\hyperref[uc1.18.1.18]{(UC 1.18.1.18)};
			\item L'utente può riprodurre File\ped{g} media \S\hyperref[uc1.18.1.6]{(UC 1.18.1.6)}.
		\end{enumerate}
		\textbf{Scenario alternativo}: 
		\begin{itemize}
			\item L'utente può avviare la presentazione automatica da qualsiasi Frame\ped{g} \S\hyperref[uc1.18.3]{(UC 1.18.3)}.
		\end{itemize}
		}
	\subsubsection{UC 1.18.1.1 - Scorrimento in avanti e indietro dei frame}{
		\label{uc1.18.1.1}
		\textbf{Attori}: utente. \\
		\textbf{Descrizione}: l'utente vuole scorrere la presentazione al Frame\ped{g} successivo o precedente rispetto a quello in presentazione. \\
		\textbf{Precondizione}: un Frame\ped{g} è in presentazione.	\\
		\textbf{Postcondizione}: il Frame\ped{g} in presentazione è cambiato passando a quello successivo o precedente in base alla scelta dell'utente.\\
		\textbf{Scenario principale}:
		\begin{enumerate}
			\item L'utente fa avanzare o retrocedere la presentazione al Frame\ped{g} successivo o al Frame\ped{g} precedente.
		\end{enumerate}
	}
	\subsubsection{UC 1.18.1.2 - Scelta all'interno del frame}{
		\label{uc1.18.1.2}
		\textbf{Attori}: utente. \\
		\textbf{Descrizione}: il Frame\ped{g} su cui si trova l’utente permette di scegliere con quale Percorso\ped{g} continuare e l'utente sceglie come far continuare la presentazione. \\
		\textbf{Precondizione}: il Frame\ped{g} in presentazione presenta più d'una scelta possibile per il Frame\ped{g} successivo.	\\
		\textbf{Postcondizione}: la presentazione può continuare con il primo Frame\ped{g} del Percorso\ped{g} che l'utente ha scelto.\\
		\textbf{Scenario principale}:
		\begin{enumerate}
			\item L'utente sceglie il Percorso\ped{g} con cui proseguire la presentazione.
		\end{enumerate}
	}
	\subsubsection{UC 1.18.1.3 - Salto al Bookmark successivo}{
		\label{uc1.18.1.3}
		\textbf{Attori}: utente. \\
		\textbf{Descrizione}: l'utente può andare a presentare un altro Frame\ped{g} saltando una parte della presentazione. \\
		\textbf{Precondizione}: il Frame\ped{g} attuale presenta un Bookmark\ped{g} ad un altro Frame\ped{g}.	\\
		\textbf{Postcondizione}: la presentazione continua dal Frame\ped{g} puntato dal Bookmark\ped{g}.	\\
		\textbf{Scenario principale}:
		\begin{enumerate}
			\item L'utente salta al Bookmark\ped{g} successivo.
		\end{enumerate}
	}
	\subsubsection{UC 1.18.1.4 - Passaggio ad un livello superiore}{
		\label{uc1.18.1.4}
		\textbf{Attori}: utente. \\
		\textbf{Descrizione}: l'utente, se lo desidera, può salire al Frame\ped{g} padre appena superiore. \\
		\textbf{Precondizione}: il Frame\ped{g} in presentazione è figlio di almeno un Frame\ped{g} padre (non è quindi la radice).	\\
		\textbf{Postcondizione}: il Frame\ped{g} in presentazione è passata al Frame\ped{g} padre.\\
		\textbf{Scenario principale}:
		\begin{enumerate}
			\item L'utente passa al livello superiore della presentazione.
		\end{enumerate}
	}
	\subsubsection{UC 1.18.1.5 - Possibilità di ingrandimento del contenuto}{
		\label{uc1.18.1.18}
		\textbf{Attori}: utente. \\
		\textbf{Descrizione}: l'utente può ingrandire il contenuto presente nel Frame\ped{g}, che sia un'immagine, un video o del testo. \\
		\textbf{Precondizione}: il Frame\ped{g} in presentazione contiene uno o più elementi\ped{g} di contenuto.	\\
		\textbf{Postcondizione}: viene inquadrato lo zoom dell'immagine, del video o del testo selezionato; l'utente può in qualsiasi momento ritornare al Frame\ped{g} di cui il contenuto fa parte.\\
		\textbf{Scenario principale}:
		\begin{enumerate}
			\item L'utente allarga il contenuto voluto.
		\end{enumerate}
	}
	\subsubsection{UC 1.18.1.6 - Riproduzione File media audio/video}{
		\label{uc1.18.1.6}
		\begin{figure}[H]
			\centering
			\includegraphics[scale=0.75]{\imgs {UC1.18.1.6}.jpg} %inserire il diagramma UML
			\label{fig:uc1.18.1.6}
			\caption{Caso d'uso 1.18.1.6: Riproduzione di File\ped{g} media audio/video}
		\end{figure}
		\textbf{Attori}: utente. \\
		\textbf{Descrizione}: durante una presentazione manuale il Frame\ped{g} corrente contiene un File\ped{g} media audio o video che l'utente può riprodurre. \\
		\textbf{Precondizione}: il Frame\ped{g} in presentazione contiene un File\ped{g} media da riprodurre.	\\
		\textbf{Postcondizione}: il File\ped{g} media è in riproduzione o l'utente ha deciso di proseguire coi Frame\ped{g} successivi.	\\
		\textbf{Scenario principale}:
		\begin{enumerate}
			\item L'utente può avviare la riproduzione di un File\ped{g} media, dall'inizio \S\hyperref[uc1.18.1.6.1]{(UC 1.18.1.6.1)} o da un punto specifico  \S\hyperref[uc1.18.1.6.4]{(UC 1.18.1.6.4)};
			\item L'utente può mettere in pausa la riproduzione del File\ped{g} media \S\hyperref[uc1.18.1.6.2]{(UC 1.18.1.6.2)} e riprenderla successivamente  \S\hyperref[uc1.18.1.6.3]{(UC 1.18.1.6.3)};
			\item L'utente può Arrestare\ped{g} la riproduzione del File\ped{g} media \S\hyperref[uc1.18.1.6.5]{(UC 1.18.1.6.5)}.
		\end{enumerate}
	}
	\subsubsection{UC 1.18.1.6.1 - Avvio manuale della riproduzione di un File media}{
		\label{uc1.18.1.6.1}
		\textbf{Attori}: utente. \\
		\textbf{Descrizione}: l'utente può avviare la riproduzione del File\ped{g} media presente nel Frame\ped{g} in qualsiasi momento. \\
		\textbf{Precondizione}: il Frame\ped{g} in presentazione contiene almeno un File\ped{g} media.	\\
		\textbf{Postcondizione}: il File\ped{g} media è in riproduzione.\\
		\textbf{Scenario principale}:
		\begin{enumerate}
			\item L'utente avvia la riproduzione del File\ped{g} media.
		\end{enumerate}
	}
	\subsubsection{UC 1.18.1.6.2 - Sospensione della riproduzione del File media}{
		\label{uc1.18.1.6.2}
		\textbf{Attori}: utente. \\
		\textbf{Descrizione}: l'utente può mettere in pausa il File\ped{g} media in esecuzione. \\
		\textbf{Precondizione}: è in riproduzione un File\ped{g} media.	\\
		\textbf{Postcondizione}: il File\ped{g} media è in pausa.\\
		\textbf{Scenario principale}:
		\begin{enumerate}
			\item L'utente sospende la riproduzione del File\ped{g} media.
		\end{enumerate}		
	}
	\subsubsection{UC 1.18.1.6.3 - Ripresa dell'esecuzione del File media}{
		\label{uc1.18.1.6.3}
		\textbf{Attori}: utente. \\
		\textbf{Descrizione}: l'utente può riprendere la riproduzione del File\ped{g} media. \\
		\textbf{Precondizione}: il File\ped{g} media in riproduzione è stato messo in pausa.	\\
		\textbf{Postcondizione}: il File\ped{g} media ha ripreso la sua riproduzione.\\
		\textbf{Scenario principale}:
		\begin{enumerate}
			\item L'utente riprende la riproduzione del File\ped{g} media.
		\end{enumerate}		
	}
	\subsubsection{UC 1.18.1.6.4 - Riproduzione del File media da qualsiasi minuto}{
		\label{uc1.18.1.6.4}
		\textbf{Attori}: utente. \\
		\textbf{Descrizione}: l'utente può far riprodurre il File\ped{g} media da un qualsiasi minuto. \\
		\textbf{Precondizione}: un File\ped{g} media è in riproduzione o è stato sospeso.	\\
		\textbf{Postcondizione}: il File\ped{g} media è in riproduzione dal minuto scelto dall'utente.\\
		\textbf{Scenario principale}:
		\begin{enumerate}
			\item L'utente riproduce il File\ped{g} media da un minuto qualsiasi. 
		\end{enumerate}		
	}
	\subsubsection{UC 1.18.1.6.5 - Arresto della riproduzione in corso}{
		\label{uc1.18.1.6.5}
		\textbf{Attori}: utente. \\
		\textbf{Descrizione}: l'utente può Arrestare\ped{g} la riproduzione del File\ped{g} media in corso. Se il File\ped{g} media è stato ingrandito allora la presentazione ritorna al Frame\ped{g} di cui il File\ped{g} media faceva parte; se l'utente decidesse di riavviarne la riproduzione, essa ripartirà dall'inizio.\\
		\textbf{Precondizione}: un File\ped{g} media è in riproduzione o è stato sospeso.	\\
		\textbf{Postcondizione}: è stata arrestata l'esecuzione del File\ped{g} media e la presentazione è tornata al Frame\ped{g} di partenza.\\
		\textbf{Scenario principale}:
		\begin{enumerate}
			\item L'utente arresta la riproduzione del File\ped{g} media.
		\end{enumerate}		
	}
\subsubsection{UC 1.18.2 - Esecuzione automatica}{
	\label{uc1.18.2}
	\begin{figure}[H]
		\centering
		\includegraphics[scale=0.75]{\imgs {UC1.18.2}.jpg} %inserire il diagramma UML
		\label{fig:uc1.18.2}
		\caption{Caso d'uso 1.18.2: Esecuzione automatica della presentazione}
	\end{figure}
	\textbf{Attori}: utente. \\
	\textbf{Descrizione}: l'utente può avviare la presentazione, metterla in pausa, arrestarla o passare alla modalità manuale in ogni momento. Nel caso in cui ci fossero Frame\ped{g} contenenti delle scelte allora la presentazione rimane in attesa della decisione dell'utente, se entrare in una delle scelte o proseguire con il Percorso\ped{g} normale. Quando l'utente avrà preso la sua decisione, l'esecuzione riprenderà in maniera automatica. \\
	\textbf{Precondizione}: il sistema è funzionante e attende la scelta dell'utente. \\
	\textbf{Postcondizione}: l'utente ha avviato la presentazione in modalità automatica.	\\
	\textbf{Scenario principale}:
	\begin{enumerate}
		\item L'utente avvia l'esecuzione della presentazione \S\hyperref[uc1.18.2.1]{(UC 1.18.2.1)};
		\item L'utente mette in pausa l'esecuzione della presentazione \S\hyperref[uc1.18.2.4]{(UC 1.18.2.4)}  e può riprenderla successivamente  \S\hyperref[uc1.18.2.2]{(UC 1.18.2.2)};
		\item L'utente arresta l'esecuzione della presentazione \S\hyperref[uc1.18.2.3]{(UC 1.18.2.3)};
		\item L'utente può decidere il tempo di durata presentazione dei Frame\ped{g} \S\hyperref[uc1.18.2.5]{(UC 1.18.2.5)};
		\item L'utente può gestire i File\ped{g} media durante l'esecuzione \S\hyperref[uc1.18.2.6]{(UC 1.18.2.6)}.
	\end{enumerate}
	\textbf{Scenario alternativo}: 
	\begin{itemize}
		\item L'utente decide di passare alla modalità manuale \S\hyperref[uc1.18.4]{(UC 1.18.4)}.
	\end{itemize}
	}
	\subsubsection{UC 1.18.2.1 - Avvia presentazione}{
		\label{uc1.18.2.1}
		\textbf{Attori}: utente. \\
		\textbf{Descrizione}: l'utente può avviare la presentazione automatica dal primo Frame\ped{g} o da un altro. \\
		\textbf{Precondizione}: una presentazione è stata caricata e un Frame\ped{g} è selezionato.	\\
		\textbf{Postcondizione}: la presentazione passa da un Frame\ped{g} all'altro, fermandosi in ognuno per il tempo impostato e arrestandosi sui Frame\ped{g} che presentano scelte.	\\
		\textbf{Scenario principale}:
		\begin{enumerate}
			\item L'utente avvia la presentazione in modalità automatica.
		\end{enumerate}
	}
	\subsubsection{UC 1.18.2.2 - Riprendi presentazione}{
		\label{uc1.18.2.2}
		\textbf{Attori}: utente. \\
		\textbf{Descrizione}: l'utente può riprendere la presentazione automatica in seguito ad una sua sospensione. \\
		\textbf{Precondizione}: l'esecuzione automatica della presentazione automatica è stata temporaneamente sospesa.	\\
		\textbf{Postcondizione}: la presentazione riprende dal Frame\ped{g} in cui era stata sospesa.\\
		\textbf{Scenario principale}:
		\begin{enumerate}
			\item L'utente riprende l'esecuzione della presentazione in modalità automatica.
		\end{enumerate}		
	}
	\subsubsection{UC 1.18.2.3 - Arresta presentazione}{
		\label{uc1.18.2.3}
		\textbf{Attori}: utente. \\
		\textbf{Descrizione}: l'utente può Arrestare\ped{g} e chiudere la presentazione. \\
		\textbf{Precondizione}: la presentazione è in esecuzione in modalità automatica.	\\
		\textbf{Postcondizione}: la presentazione viene chiusa e il sistema ritorna alla pagina principale.\\
		\textbf{Scenario principale}:
		\begin{enumerate}
			\item L'utente arresta l'esecuzione della presentazione in modalità automatica.
		\end{enumerate}		
	}
	\subsubsection{UC 1.18.2.4 - Sospendi presentazione}{
		\label{uc1.18.2.4}
		\textbf{Attori}: utente. \\
		\textbf{Descrizione}: l'utente può mettere in pausa l'esecuzione automatica della presentazione e potrà riprenderla dallo stesso punto in qualsiasi momento. \\
		\textbf{Precondizione}: è in esecuzione la presentazione in modalità automatica.	\\
		\textbf{Postcondizione}: l'esecuzione automatica della presentazione è stata sospesa e viene mostrato il Frame\ped{g} che era in esecuzione al momento della sospensione.\\
		\textbf{Scenario principale}:
		\begin{enumerate}
			\item L'utente sospende l'esecuzione della presentazione in modalità automatica;
			\item L'utente può riprendere l'esecuzione automatica della presentazione \S\hyperref[uc1.18.2.2]{(UC 1.18.2.2)}.
		\end{enumerate}		
	}
	\subsubsection{UC 1.18.2.5 - Imposta tempo scorrimento frame}{
		\label{uc1.18.2.5}
		\textbf{Attori}: utente. \\
		\textbf{Descrizione}: l'utente può impostare un tempo di visualizzazione che vale per ogni Frame\ped{g} nell'esecuzione corrente della presentazione; tale tempo sovrascrive quello impostato durante la modifica della presentazione. \\
		\textbf{Precondizione}: la presentazione è in modalità manuale.	\\
		\textbf{Postcondizione}: la presentazione automatica può essere avviata con il tempo di scorrimento impostato dall'utente.\\
		\textbf{Scenario principale}:
		\begin{enumerate}
			\item L'utente imposta il tempo di scorrimento dei Frame\ped{g} per l'esecuzione corrente della presentazione.
		\end{enumerate}		}
	\subsubsection{UC 1.18.2.6 - Riproduzione File media audio/video}{
		\label{uc1.18.2.6}
		\begin{figure}
			\centering
			\includegraphics[scale=0.75]{\imgs {UC1.18.2.6}.jpg} %inserire il diagramma UML
			\label{fig:uc1.18.2.6}
			\caption{Caso d'uso 1.18.2.6: Riproduzione dei File\ped{g} media}
		\end{figure}
		\textbf{Attori}: utente. \\
		\textbf{Descrizione}: durante una presentazione automatica, se nel Frame\ped{g} sono presenti uno o più File\ped{g} media essi verranno avviati da sinistra verso destra e dall'alto verso il basso in modo automatico; l'utente può saltare la riproduzione di tali File\ped{g} media e la presentazione automatica continua col Frame\ped{g} successivo. \\
		\textbf{Precondizione}: la presentazione è in esecuzione automatica e il Frame\ped{g} in presentazione contiene uno o più File\ped{g} media  audio/video.	\\
		\textbf{Postcondizione}: il File\ped{g} media è in riproduzione o l'utente ha deciso di proseguire coi Frame\ped{g} successivi.	\\
		\textbf{Scenario principale}:
		\begin{enumerate}
			\item La presentazione automatica avvia la riproduzione del File\ped{g} media \S\hyperref[uc1.18.2.6.1]{(UC 1.18.2.6.1)};
			\item L'utente può saltare la riproduzione del File\ped{g} media \S\hyperref[uc1.18.2.6.5]{(UC 1.18.2.6.5)};
			\item L'utente può mettere in pausa la riproduzione del File\ped{g} media \S\hyperref[uc1.18.2.6.2]{(UC 1.18.2.6.2)} e riprenderla dal punto in cui l'aveva sospesa \S\hyperref[uc1.18.2.6.3]{(UC 1.18.2.6.3)} o da punti diversi \S\hyperref[uc1.18.2.6.4]{(UC 1.18.2.6.4)};
			\item L'utente ha la possibilità di ingrandire un video \S\hyperref[uc1.18.2.6.6]{(UC 1.18.2.6.6)}.
		\end{enumerate}
	}
	\subsubsection{UC 1.18.2.6.1 - Avvio automatico riproduzione di un File media}{
		\label{uc1.18.2.6.1}
		\textbf{Attori}: utente. \\
		\textbf{Descrizione}: il File\ped{g} media presente nel Frame\ped{g} viene riprodotto automaticamente; se più File\ped{g} media sono presenti allora verranno riprodotti tutti uno alla volta (ordine: da sinistra a destra e dall'alto al basso). \\
		\textbf{Precondizione}: la presentazione è in modalità esecuzione automatica e il Frame\ped{g} in presentazione contiene uno o più File\ped{g} media audio/video.	\\
		\textbf{Postcondizione}: il File\ped{g} media è in riproduzione.\\
		\textbf{Scenario principale}:
		\begin{enumerate}
			\item L'esecuzione automatica del File\ped{g} media ne riproduce i contenuti.
		\end{enumerate}				
	}
	\subsubsection{UC 1.18.2.6.2 - Sospensione della riproduzione del File media}{
		\label{uc1.18.2.6.2}
		\textbf{Attori}: utente. \\
		\textbf{Descrizione}: l'utente può mettere in pausa la riproduzione del File\ped{g} media in corso. \\
		\textbf{Precondizione}: è in riproduzione un File\ped{g} media.	\\
		\textbf{Postcondizione}: è stata sospesa la riproduzione del File\ped{g} media.\\
		\textbf{Scenario principale}:
		\begin{enumerate}
			\item L'utente sospende la riproduzione del File\ped{g} media.
		\end{enumerate}				
	}
	\subsubsection{UC 1.18.2.6.3 - Ripresa dell'esecuzione del File media}{
		\label{uc1.18.2.6.3}
		\textbf{Attori}: utente. \\
		\textbf{Descrizione}: l'utente riprende la produzione del File\ped{g} media dopo averla sospesa. \\
		\textbf{Precondizione}: è stata sospesa la riproduzione del File\ped{g} media nel Frame\ped{g} correnteecuzione del File\ped{g} media nel Frame\ped{g} corrente.\\
		\textbf{Scenario principale}:
		\begin{enumerate}
			\item L'utente riprende la riproduzione del File\ped{g} media.
		\end{enumerate}				
	}
	\subsubsection{UC 1.18.2.6.4 - Riproduzione del File media da qualsiasi minuto}{
		\label{uc1.18.2.6.4}
		\textbf{Attori}: utente. \\
		\textbf{Descrizione}: l'utente può far riprodurre il File\ped{g} media da qualsiasi minuto
		\textbf{Precondizione}: un File\ped{g} media è in riproduzione o è stato sospeso.	\\
		\textbf{Postcondizione}: il File\ped{g} media è in riproduzione dal minuto scelto dall'utente.\\
		\textbf{Scenario principale}:
		\begin{enumerate}
			\item L'utente riproduce il File\ped{g} media da un minuto qualsiasi. 
		\end{enumerate}				
	}
	\subsubsection{UC 1.18.2.6.5 - Salto della riproduzione in corso}{
		\label{uc1.18.2.6.5}
		\textbf{Attori}: utente. \\
		\textbf{Descrizione}: l'utente può saltare la riproduzione in corso e far proseguire normalmente la presentazione automatica. \\
		\textbf{Precondizione}: un File\ped{g} media è in riproduzione o è stato sospeso.	\\
		\textbf{Postcondizione}: l'esecuzione automatica della presentazione riproduce un altro File\ped{g} media audio/video se presente oppure riprende la sua normale esecuzione.\\
		\textbf{Scenario principale}:
		\begin{enumerate}
			\item L'utente salta la riproduzione del File\ped{g} media.
		\end{enumerate}						
	}
	\subsubsection{UC 1.18.2.6.6 - Visualizzazione a schermo intero di un video}{
		\label{uc1.18.2.6.6}
		\textbf{Attori}: utente. \\
		\textbf{Descrizione}: l'utente può riprodurre il video a schermo intero. \\
		\textbf{Precondizione}: un video è in riproduzione o è stato sospeso.	\\
		\textbf{Postcondizione}: lo schermo presenta il video selezionato a schermo intero.\\
		\textbf{Scenario principale}:
		\begin{enumerate}
			\item L'utente allarga le dimensioni del video.
		\end{enumerate}	
		\textbf{Scenario alternativo}:
		\begin{itemize}
			\item L'utente può uscire dalla visualizzazione a schermo intero in qualsiasi momento.
		\end{itemize}
	}
	\subsubsection{UC 1.18.3 - Passaggio alla modalità di esecuzione automatica}{
		\label{uc1.18.3}
		\textbf{Attori}: utente. \\
		\textbf{Descrizione}: l'utente vuole passare dalla presentazione in modalità manuale a quella in modalità automatica. \\
		\textbf{Precondizione}: la presentazione è in modalità esecuzione manuale.	\\
		\textbf{Postcondizione}: La presentazione è stata avviata in modalità automatica partendo dal Frame\ped{g} corrente.\\
		\textbf{Scenario principale}:
		\begin{enumerate}
			\item L'utente passa alla riproduzione automatica della presentazione.
		\end{enumerate}						
	}
	\subsubsection{UC 1.18.4 - Passaggio alla modalità di esecuzione manuale}{
		\label{uc1.18.4}
		\textbf{Attori}: utente. \\
		\textbf{Descrizione}: l'utente vuole passare dalla presentazione in modalità automatica a quella in modalità manuale. \\
		\textbf{Precondizione}: la presentazione è in modalità esecuzione automatica.	\\
		\textbf{Postcondizione}: la presentazione è stata avviata in modalità manuale partendo dal Frame\ped{g} corrente.\\
		\textbf{Scenario principale}:
		\begin{enumerate}
			\item L'utente passa alla riproduzione manuale della presentazione.
		\end{enumerate}						
	}