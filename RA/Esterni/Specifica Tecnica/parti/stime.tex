\section{Stime di fattibilità e di bisogno di risorse}{
	L'architettura definita precedentemente ha raggiunto un livello di dettaglio sufficiente per fornire una stima sulla fattibilità e di bisogno di Risorse\ped{g}. L'analisi dell'architettura progettata ha permesso di constatare che le tecnologie che si è scelto di adottare risultano sufficientemente adeguate per la realizzazione del prodotto e riescono a ricoprire le esigenze progettuali.\\
	Poiché tutti gli strumenti da utilizzare nello sviluppo sono gratuiti, il bisogno di
	Risorse\ped{g} non si dimostra essere particolarmente problematico.\\
	Si è deciso di utilizzare HTML5, CSS3 e Javascript (e le sue librerie) per lo sviluppo della parte WEB\ped{g}.\\
	Per la parte di database si è scelto l'utilizzo di MEAN e delle librerie Express.js e Node.js per una migliore interazione con MongoDB.\\
	Per la parte di esecuzione delle presentazioni è stato scelto Impress.js, Framework\ped{g} che permette l'esecuzione in maniera non lineare come richiesto.\\
	Per la parte di modifica delle presentazioni verranno utilizzati Javascript e il Framework\ped{g} Angular.js per lo spostamento in tempo reale degli elementi\ped{g} delle presentazioni.
	Infine è stato inoltre considerato l'utilizzo della tecnologia HTML5 Manifest per la gestione delle presentazioni offline.
	}