\section{Standard di progetto}{
	\subsection{Standard di progettazione architetturale}{
		Gli standard di progettazione architetturale sono definiti nel documento \href{run:../../Esterni/\fSpecificaTecnica}{\fEscapeSpecificaTecnica}.}
	\subsection{Standard di documentazione del codice}{
		Gli standard per la scrittura di documentazione del Codice\ped{g} sono definiti nelle \href{run:../../Interni/\fNormeDiProgetto}{\fEscapeNormeDiProgetto}}
	\subsection{Standard di denominazione di entità e relazioni}{
		Tutti gli elementi\ped{g} (package, classi, metodi o attributi) definiti, devono avere denominazioni chiare ed autoesplicative. Nel caso il nome risulti lungo, è preferibile preferire la chiarezza alla lunghezza. \\
		Sono ammesse abbreviazioni se:
		\begin{itemize}
			\item immediatamente comprensibili;
			\item non ambigue;
			\item sufficientemente contestualizzate.
		\end{itemize}
		Le regole tipografiche relative ai nomi delle entità sono definite nelle \href{run:../../Interni/\fNormeDiProgetto}{\fEscapeNormeDiProgetto}.}
	\subsection{Standard di programmazione}{
		Gli standard di programmazione sono definiti e descritti nelle \href{run:../../Interni/\fNormeDiProgetto}{\fEscapeNormeDiProgetto}.}
	\subsection{Strumenti di lavoro}{
		Gli strumenti da adottare e le procedure per utilizzarli correttamente durante la realizzazione del prodotto Software\ped{g} sono definiti nelle \href{run:../../Interni/\fNormeDiProgetto}{\fEscapeNormeDiProgetto}.}
}