	\subsection{Classe SlideShowElements}{
		\textbf{Funzione}\\
			\indent Classe astratta, base delle classi usate per rappresentare gli elementi\ped{g} della presentazione.\\
		\textbf{Scope}\\
			\indent Model::SlideShow::SlideShowElements.\\
		\textbf{Utilizzo}\\
			\indent Contiene gli attributi e i metodi comuni degli oggetti che rappresentano gli elementi\ped{g} della presentazione.\\
		\textbf{Attributi}
		\begin{itemize}
			\item \textbf{id}
			\begin{itemize}
				\item \textbf{Accesso}: Private;
				\item \textbf{Tipo}: Integer;
				\item \textbf{Descrizione}: indica l’identificativo univoco dell’Elemento\ped{g}.
			\end{itemize}
			\item \textbf{yIndex}
			\begin{itemize}
				\item \textbf{Accesso}: Private;
				\item \textbf{Tipo}: Double;
				\item \textbf{Descrizione}: rappresenta la posizione sull’asse delle y dell’Elemento\ped{g} rispetto alla presentazione.
			\end{itemize}
			\item \textbf{xIndex}
			\begin{itemize}
				\item \textbf{Accesso}: Private;
				\item \textbf{Tipo}: Double;
				\item \textbf{Descrizione}: rappresenta la posizione sull’asse delle x dell’Elemento\ped{g} rispetto alla presentazione.
			\end{itemize}
			\item \textbf{zIndex}
			\begin{itemize}
				\item \textbf{Accesso}: Private;
				\item \textbf{Tipo}: Integer;
				\item \textbf{Descrizione}: rappresenta la posizione dell’Elemento\ped{g} rispetto agli altri elementi\ped{g} della presentazione.
			\end{itemize}
			\item \textbf{rotation}
			\begin{itemize}
				\item \textbf{Accesso}: Private;
				\item \textbf{Tipo}: Double;
				\item \textbf{Descrizione}: rappresenta il grado di rotazione dell’oggetto.
			\end{itemize}
			\item \textbf{height}
			\begin{itemize}
				\item \textbf{Accesso}: Private;
				\item \textbf{Tipo}: Double;
				\item \textbf{Descrizione}: rappresenta l’altezza dell’oggetto.
			\end{itemize}
			\item \textbf{width}
			\begin{itemize}
				\item \textbf{Accesso}: Private;
				\item \textbf{Tipo}: Double;
				\item \textbf{Descrizione}: rappresenta la larghezza dell’oggetto.
			\end{itemize}
			\item \textbf{waste}
			\begin{itemize}
				\item \textbf{Accesso}: Private;
				\item \textbf{Tipo}: Double;
				\item \textbf{Descrizione}: campo dati utilizzato dal traduttore Impress per mettere gli elementi\ped{g} nel modo corretto.
			\end{itemize}
		\end{itemize}
		
		\noindent{\textbf{Ereditata da}:}
		\begin{itemize}
			\item Text (\S\ref{Text});
			\item Image (\S\ref{Image});
			\item Frame\ped{g} (\S\ref{Frame});
			\item SVG (\S\ref{SVG});
			\item Background (\S\ref{Background});
			\item Audio (\S\ref{Audio});
			\item Video (\S\ref{Video}).
		\end{itemize}
		
		\subsubsection{Classe Text}{
			\label{Text}
			\textbf{Funzione}\\
				\indent Classe concreta, i suoi elementi\ped{g} rappresentano un oggetto di tipo testo.\\
		   	\textbf{Scope}\\
				\indent Model::SlideShow::SlideShowElements::Text.\\
			\textbf{Utilizzo}\\
				\indent Il costruttore viene invocato da InsertEditRemove::insertText().\\
			\textbf{Attributi}
			\begin{itemize}
				\item \textbf{font}
				\begin{itemize}
					\item \textbf{Accesso}: Private;
					\item \textbf{Tipo}: String;
					\item \textbf{Descrizione}: rappresenta il Font\ped{g} dell’oggetto.
				\end{itemize}
				\item \textbf{fontSize}
				\begin{itemize}
					\item \textbf{Accesso}: Private;
					\item \textbf{Tipo}: String;
					\item \textbf{Descrizione}: rappresenta la dimensione del Font\ped{g} dell’oggetto.
				\end{itemize}
				\item \textbf{content}
				\begin{itemize}
					\item \textbf{Accesso}: Private;
					\item \textbf{Tipo}: String;
					\item \textbf{Descrizione}: rappresenta il contenuto del testo.
				\end{itemize}
				\item \textbf{color}
				\begin{itemize}
					\item \textbf{Accesso}: Private;
					\item \textbf{Tipo}: String;
					\item \textbf{Descrizione}: rappresenta il colore dell’oggetto.
				\end{itemize}
				\item \textbf{type}
				\begin{itemize}
					\item \textbf{Accesso}: Private;
					\item \textbf{Tipo}: String;
					\item \textbf{Descrizione}: rappresenta il tipo dell'oggetto.
				\end{itemize}
			\end{itemize}
		}
	\subsubsection{Classe Image}{
		\label{Image}
		\textbf{Funzione}\\
			\indent Classe concreta, i suoi elementi\ped{g} rappresentano un oggetto di tipo immagine.\\
	   	\textbf{Scope}\\
			\indent Model::SlideShow::SlideShowElements::Image.\\
		\textbf{Utilizzo}\\
			\indent Il costruttore viene invocato da InsertEditRemove::insertImage().\\
		\textbf{Attributi}
		\begin{itemize}
				\item \textbf{url}
				\begin{itemize}
					\item \textbf{Accesso}: Private;
					\item \textbf{Tipo}: String;
					\item \textbf{Descrizione}: rappresenta il Percorso\ped{g} dell’oggetto.
				\end{itemize}
				\item \textbf{type}
				\begin{itemize}
					\item \textbf{Accesso}: Private;
					\item \textbf{Tipo}: String;
					\item \textbf{Descrizione}: rappresenta il tipo dell'oggetto.
				\end{itemize}
		\end{itemize}
		}
		
	\subsubsection{Classe Frame}{
		\label{Frame}
		\textbf{Funzione}\\
			\indent Classe concreta, i suoi elementi\ped{g} rappresentano un oggetto di tipo Frame\ped{g}.\\
	   	\textbf{Scope}\\
			\indent Model::SlideShow::SlideShowElements::Frame.\\
		\textbf{Utilizzo}\\
			\indent Il costruttore viene invocato da InsertEditRemove::insertFrame().\\
		\textbf{Attributi}
		\begin{itemize}
			\item \textbf{bookmark}
			\begin{itemize}
				\item \textbf{Accesso}: Private;
				\item \textbf{Tipo}: Bool;
				\item \textbf{Descrizione}: è a 1 se il Frame\ped{g} è un Bookmark\ped{g}, 0 altrimenti.
			\end{itemize}
			\item \textbf{ref}
			\begin{itemize}
				\item \textbf{Accesso}: Private;
				\item \textbf{Tipo}: String;
				\item \textbf{Descrizione}: contiene il riferimento dell’immagine di sfondo Frame\ped{g}.
			\end{itemize}
			\item \textbf{color}
			\begin{itemize}
				\item \textbf{Accesso}: Private;
				\item \textbf{Tipo}: String;
				\item \textbf{Descrizione}: rappresenta il colore dello sfondo del Frame\ped{g}.
			\end{itemize}
			\item \textbf{type}
			\begin{itemize}
				\item \textbf{Accesso}: Private;
				\item \textbf{Tipo}: String;
				\item \textbf{Descrizione}: rappresenta il tipo dell'oggetto.
			\end{itemize}
		\end{itemize}
		}
	\subsubsection{Classe SVG}{
		\label{SVG}
		\textbf{Funzione}\\
			\indent Classe concreta, i suoi elementi\ped{g} rappresentano un oggetto di tipo SVG.\\
	   	\textbf{Scope}\\
			\indent Model::SlideShow::SlideShowElements::SVG.\\
		\textbf{Utilizzo}\\
			\indent Il costruttore viene invocato da InsertEditRemove::insertSVG().\\
		\textbf{Attributi}
		\begin{itemize}
			\item \textbf{color}
			\begin{itemize}
				\item \textbf{Accesso}: Private;
				\item \textbf{Tipo}: String;
				\item \textbf{Descrizione}: rappresenta il colore dell’oggetto.
			\end{itemize}
			\item \textbf{shape}
			\begin{itemize}
				\item \textbf{Accesso}: Private;
				\item \textbf{Tipo}: Array;
				\item \textbf{Descrizione}: rappresenta le coordinate della forma dell’oggetto.
			\end{itemize}
			\item \textbf{type}
			\begin{itemize}
				\item \textbf{Accesso}: Private;
				\item \textbf{Tipo}: String;
				\item \textbf{Descrizione}: rappresenta il tipo dell'oggetto.
			\end{itemize}
		\end{itemize}
		}
	\subsubsection{Classe Audio}{
		\label{Audio}
		\textbf{Funzione}\\
			\indent Classe concreta, i suoi elementi\ped{g} rappresentano un oggetto di tipo immagina.\\
	   	\textbf{Scope}\\
			\indent Model::SlideShow::SlideShowElements::Audio.\\
		\textbf{Utilizzo}\\
			\indent Il costruttore viene invocato da InsertEditRemove::insertAudio().\\
		\textbf{Attributi}
		\begin{itemize}
			\item \textbf{url}
			\begin{itemize}
				\item \textbf{Accesso}: Private;
				\item \textbf{Tipo}: String;
				\item \textbf{Descrizione}: rappresenta il Percorso\ped{g} dell’oggetto.
			\end{itemize}
			\item \textbf{type}
			\begin{itemize}
				\item \textbf{Accesso}: Private;
				\item \textbf{Tipo}: String;
				\item \textbf{Descrizione}: rappresenta il tipo dell'oggetto.
			\end{itemize}
		\end{itemize}
		}
	
	\subsubsection{Classe Video}{
		\label{Video}
		\textbf{Funzione}\\
			\indent Classe concreta, i suoi elementi\ped{g} rappresentano un oggetto di tipo video.\\
	   	\textbf{Scope}\\
			\indent Model::SlideShow::SlideShowElements::Video.\\
		\textbf{Utilizzo}\\
			\indent Il costruttore viene invocato da InsertEditRemove::insertVideo().\\
		\textbf{Attributi}
		\begin{itemize}
			\item \textbf{url}
			\begin{itemize}
				\item \textbf{Accesso}: Private;
				\item \textbf{Tipo}: String;
				\item \textbf{Descrizione}: rappresenta il Percorso\ped{g} dell’oggetto.
			\end{itemize}
			\item \textbf{type}
			\begin{itemize}
				\item \textbf{Accesso}: Private;
				\item \textbf{Tipo}: String;
				\item \textbf{Descrizione}: rappresenta il tipo dell'oggetto.
			\end{itemize}
		\end{itemize}
		}

	\subsubsection{Classe Background}{
		\label{Background}
		\textbf{Funzione}\\
			\indent Classe concreta, i suoi elementi\ped{g} rappresentano lo sfondo.\\
	   	\textbf{Scope}\\
			\indent Model::SlideShow::SlideShowElements::Background.\\
		\textbf{Utilizzo}\\
			\indent Il costruttore viene invocato da InsertEditRemove::insertBackground().\\
		\textbf{Attributi}
		\begin{itemize}
			\item \textbf{image}
			\begin{itemize}
				\item \textbf{Accesso}: Private;
				\item \textbf{Tipo}: String;
				\item \textbf{Descrizione}: rappresenta il riferimento dell’immagine dello sfondo.
			\end{itemize}
			\item \textbf{color}
			\begin{itemize}
				\item \textbf{Accesso}: Private;
				\item \textbf{Tipo}: String;
				\item \textbf{Descrizione}: rappresenta il colore dello sfondo.
			\end{itemize}
			\item \textbf{type}
			\begin{itemize}
				\item \textbf{Accesso}: Private;
				\item \textbf{Tipo}: String;
				\item \textbf{Descrizione}: rappresenta il tipo dell'oggetto.
			\end{itemize}
		\end{itemize}

		}
	}
\subsection{Classe InsertEditRemove}{
		\textbf{Funzione}\\
			\indent Classe concreta in cui vengono implementati gli algoritmi di inserimento, modifica e rimozione degli elementi\ped{g} nella presentazione.\\
	   	\textbf{Scope}\\
			\indent Model::SlideShow::SlideShowActions::InsertEditRemove.\\
		\textbf{Utilizzo}\\
			\indent Viene utilizzata dalla classe command per eseguire i comandi di inserimento.\\
		\textbf{Attributi}
		\begin{itemize}
			\item \textbf{presentazione}
			\begin{itemize}
				\item \textbf{Accesso}: Private;
				\item \textbf{Descrizione}: oggetto json che contiene gli oggetti delle classi che rappresentano gli elementi\ped{g} della presentazione.
			\end{itemize}
		\end{itemize}
		\noindent{\textbf{Metodi}}
		\begin{itemize}
			\item \textbf{constructPresentazione}(newPresentazione)
			\begin{itemize}
				\item \textbf{Accesso}: Public;
				\item \textbf{Tipo di ritorno}: Void;
				\item \textbf{Descrizione}: setta Presentazione a newPresentazione.
			\end{itemize}
			\item \textbf{getPresentazione}()
			\begin{itemize}
				\item \textbf{Accesso}: Public;
				\item \textbf{Tipo di ritorno}: SlideShowElement;
				\item \textbf{Descrizione}: ritorna Presentazione.
			\end{itemize}
			\item \textbf{getPresentationName}()
			\begin{itemize}
				\item \textbf{Accesso}: Public;
				\item \textbf{Tipo di ritorno}: SlideShowElement;
				\item \textbf{Descrizione}: ritorna il nome di Presentazione.
			\end{itemize}
			\item \textbf{getIdPresentazione}()
			\begin{itemize}
				\item \textbf{Accesso}: Public;
				\item \textbf{Tipo di ritorno}: SlideShowElement;
				\item \textbf{Descrizione}: ritorna l'id di Presentazione. Questo metodo è utile per alleggerire le operazioni di diversi metodi del controller e del model che hanno bisogno di accedere esclusivamente al campo id dell'oggetto presentazione. 
			\end{itemize}
			\item \textbf{getFrames}()
			\begin{itemize}
				\item \textbf{Accesso}: Public;
				\item \textbf{Tipo di ritorno}: Array di oggetti di tipo Frame;
				\item \textbf{Descrizione}: ritorna i Frame\ped{g} della presentazione.
			\end{itemize}
			\item \textbf{getTexts}()
			\begin{itemize}
				\item \textbf{Accesso}: Public;
				\item \textbf{Tipo di ritorno}: Array di oggetti di tipo Text;
				\item \textbf{Descrizione}: ritorna gli elementi\ped{g} testo della presentazione.
			\end{itemize}
			\item \textbf{getImages}()
			\begin{itemize}
				\item \textbf{Accesso}: Public;
				\item \textbf{Tipo di ritorno}: Array di oggetti di tipo Image;
				\item \textbf{Descrizione}: ritorna gli elementi\ped{g} immagine della presentazione.
			\end{itemize}
			\item \textbf{getAudios}()
			\begin{itemize}
				\item \textbf{Accesso}: Public;
				\item \textbf{Tipo di ritorno}: Array di oggetti di tipo Audio;
				\item \textbf{Descrizione}: ritorna gli elementi\ped{g} audio della presentazione.
			\end{itemize}
			\item \textbf{getVideos}()
			\begin{itemize}
				\item \textbf{Accesso}: Public;
				\item \textbf{Tipo di ritorno}: Array di oggetti di tipo Video;
				\item \textbf{Descrizione}: ritorna gli elementi\ped{g} video della presentazione.
			\end{itemize}
			\item \textbf{getSVGs}()
			\begin{itemize}
				\item \textbf{Accesso}: Public;
				\item \textbf{Tipo di ritorno}: Array di oggetti di tipo SVG;
				\item \textbf{Descrizione}: ritorna gli elementi\ped{g} SVG della presentazione.
			\end{itemize}
			\item \textbf{getBackground}()
			\begin{itemize}
				\item \textbf{Accesso}: Public;
				\item \textbf{Tipo di ritorno}: SlideShowElement;
				\item \textbf{Descrizione}: ritorna il background della presentazione.
			\end{itemize}
			\item \textbf{getPaths}()
			\begin{itemize}
				\item \textbf{Accesso}: Public;
				\item \textbf{Tipo di ritorno}: SlideShowElement;
				\item \textbf{Descrizione}: ritorna i percorsi\ped{g} della presentazione.
			\end{itemize}
			\item \textbf{getElement}(id)
			\begin{itemize}
				\item \textbf{Accesso}: Public;
				\item \textbf{Tipo di ritorno}: SlideShowElement;
				\item \textbf{Descrizione}: ritorna l'Elemento\ped{g} id della presentazione.
			\end{itemize}
			\item \textbf{insertText}(spec:object)
			\begin{itemize}
				\item \textbf{Accesso}: Public;
				\item \textbf{Tipo di ritorno}: Void;
				\item \textbf{Descrizione}: costruisce un oggetto di tipo Text() passando il parametro spec. Inserisce l’oggetto così costruito nell’oggetto Presentazione.
			\end{itemize}
			\item \textbf{insertFrame}(spec:object)
			\begin{itemize}
				\item \textbf{Accesso}: Public;
				\item \textbf{Tipo di ritorno}: Integer;
				\item \textbf{Descrizione}: costruisce un oggetto di tipo Frame() passando il parametro spec. Inserisce l’oggetto così costruito nell’oggetto Presentazione.
			\end{itemize}
			\item \textbf{insertImage}(spec:object)
			\begin{itemize}
				\item \textbf{Accesso}: Public;
				\item \textbf{Tipo di ritorno}: Integer;
				\item \textbf{Descrizione}: costruisce un oggetto di tipo Image() passando il parametro spec. Inserisce l’oggetto così costruito nell’oggetto Presentazione.
			\end{itemize}
			\item \textbf{insertSVG}(spec:object)
			\begin{itemize}
				\item \textbf{Accesso}: Public;
				\item \textbf{Tipo di ritorno}: Integer;
				\item \textbf{Descrizione}: costruisce un oggetto di tipo SVG() passando il parametro spec. Inserisce l’oggetto così costruito nell’oggetto Presentazione.
			\end{itemize}
			\item \textbf{insertAudio}(spec:object)
			\begin{itemize}
				\item \textbf{Accesso}: Public;
				\item \textbf{Tipo di ritorno}: Integer;
				\item \textbf{Descrizione}: costruisce un oggetto di tipo Audio() passando il parametro spec. Inserisce l’oggetto così costruito nell’oggetto Presentazione.
			\end{itemize}
			\item \textbf{insertVideo}(spec:object)
			\begin{itemize}
				\item \textbf{Accesso}: Public;
				\item \textbf{Tipo di ritorno}: Integer;
				\item \textbf{Descrizione}: costruisce un oggetto di tipo Video() passando il parametro spec. Inserisce l’oggetto così costruito nell’oggetto Presentazione.
			\end{itemize}
			\item \textbf{insertBackground}(spec:object)
			\begin{itemize}
				\item \textbf{Accesso}: Public;
				\item \textbf{Tipo di ritorno}: Integer;
				\item \textbf{Descrizione}: costruisce un oggetto di tipo Background() passando il parametro spec e salvandosi in un oggetto oldBackground il background precedente. Imposta il nuovo background alla presentazione e ritorna oldBackground.
			\end{itemize}
			\item \textbf{removeText}(id:integer)
			\begin{itemize}
				\item \textbf{Accesso}: Public;
				\item \textbf{Tipo di ritorno}: Text;
				\item \textbf{Descrizione}: copia l’oggetto con il campo dati id corrispondente e lo rimuove dall’oggetto Presentazione. Restituisce l’oggetto copiato, tale copia verrà poi utilizzata dal command che ha invocato questo metodo per poter ripristinare l'elemento eliminato in caso venisse richiesta una operazione di annulla / ripristina.
			\end{itemize}
			\item \textbf{removeFrame}(id:integer)
			\begin{itemize}
				\item \textbf{Accesso}: Public;
				\item \textbf{Tipo di ritorno}: Frame\ped{g};
				\item \textbf{Descrizione}: copia l’oggetto con il campo dati id corrispondente, accede al suo campo prev e ne identifica i predecessore, pone prev.next=next. Rimuove l’oggetto dall’oggetto Presentazione e restituisce l’oggetto copiato, tale copia verrà poi utilizzata dal command che ha invocato questo metodo per poter ripristinare l'elemento eliminato in caso venisse richiesta una operazione di annulla / ripristina.
			\end{itemize}
			\item \textbf{removeImage}(id:integer)
			\begin{itemize}
				\item \textbf{Accesso}: Public;
				\item \textbf{Tipo di ritorno}: Image;
				\item \textbf{Descrizione}: copia l’oggetto con il campo dati id corrispondente e lo rimuove dall’oggetto Presentazione. Restituisce l’oggetto copiato, tale copia verrà poi utilizzata dal command che ha invocato questo metodo per poter ripristinare l'elemento eliminato in caso venisse richiesta una operazione di annulla / ripristina.
			\end{itemize}
			\item \textbf{removeSVG}(id:integer)
			\begin{itemize}
				\item \textbf{Accesso}: Public;
				\item \textbf{Tipo di ritorno}: SVG;
				\item \textbf{Descrizione}: copia l’oggetto con il campo dati id corrispondente e lo rimuove dall’oggetto Presentazione. Restituisce l’oggetto copiato, tale copia verrà poi utilizzata dal command che ha invocato questo metodo per poter ripristinare l'elemento eliminato in caso venisse richiesta una operazione di annulla / ripristina.
			\end{itemize}
			\item \textbf{removeAudio}(id:integer)
			\begin{itemize}
				\item \textbf{Accesso}: Public;
				\item \textbf{Tipo di ritorno}: Audio;
				\item \textbf{Descrizione}: copia l’oggetto con il campo dati id corrispondente e lo rimuove dall’oggetto Presentazione. Restituisce l’oggetto copiato, tale copia verrà poi utilizzata dal command che ha invocato questo metodo per poter ripristinare l'elemento eliminato in caso venisse richiesta una operazione di annulla / ripristina.
			\end{itemize}
			\item \textbf{removeVideo}(id:integer)
			\begin{itemize}
				\item \textbf{Accesso}: Public;
				\item \textbf{Tipo di ritorno}: Video;
				\item \textbf{Descrizione}: copia l’oggetto con il campo dati id corrispondente e lo rimuove dall’oggetto Presentazione. Restituisce l’oggetto copiato, tale copia verrà poi utilizzata dal command che ha invocato questo metodo per poter ripristinare l'elemento eliminato in caso venisse richiesta una operazione di annulla / ripristina.
			\end{itemize}
			\item \textbf{editPosition}(spec:object)
			\begin{itemize}
				\item \textbf{Accesso}: Public;
				\item \textbf{Tipo di ritorno}: SlideShowElement;
				\item \textbf{Descrizione}: scorre il campo dati che contiene gli oggetti di tipo SlideShowElements in Presentazione per trovare l’oggetto con il campo spec.id corrispondente, crea un oggetto oldPosition per contenere i valori di xIndex e di yIndex dell’oggetto trovato e imposta xIndex e yIndex con i valori passati tramite spec. Restituisce oldPosition, questo verrà poi utilizzato dal command che ha invocato questo metodo per poter eliminare le modifiche effettuate in caso venisse richiesta una operazione di annulla / ripristina.
			\end{itemize}
			\item \textbf{editRotation}(spec:object)
			\begin{itemize}
				\item \textbf{Accesso}: Public;
				\item \textbf{Tipo di ritorno}: SlideShowElement;
				\item \textbf{Descrizione}: scorre il campo dati che contiene gli oggetti di tipo SlideShowElements in Presentazione per trovare l’oggetto con il campo spec.id corrispondente, crea un oggetto oldRotation per contenere il valore di rotation dell’oggetto trovato e imposta rotation con il valore passato tramite spec. Restituisce oldRotation, questo verrà poi utilizzato dal command che ha invocato questo metodo per poter eliminare le modifiche effettuate in caso venisse richiesta una operazione di annulla / ripristina.
			\end{itemize}
			\item \textbf{editSize}(spec:object)
			\begin{itemize}
				\item \textbf{Accesso}: Public;
				\item \textbf{Tipo di ritorno}: SlideShowElement;
				\item \textbf{Descrizione}: scorre il campo dati che contiene gli oggetti di tipo SlideShowElements in Presentazione per trovare l’oggetto con il campo spec.id corrispondente, crea un oggetto oldSize per contenere i valori di height e di width dell’oggetto trovato e imposta height e width con i valori passati tramite spec. Restituisce oldSize, questo verrà poi utilizzato dal command che ha invocato questo metodo per poter eliminare le modifiche effettuate in caso venisse richiesta una operazione di annulla / ripristina.
			\end{itemize}
			\item \textbf{editBackground}(spec:object)
			\begin{itemize}
				\item \textbf{Accesso}: Public;
				\item \textbf{Tipo di ritorno}: SlideShowElement;
				\item \textbf{Descrizione}: scorre il campo dati che contiene gli oggetti di tipo SlideShowElements in Presentazione per trovare l’oggetto con il campo spec.id corrispondente. Se lo trova crea un oggetto oldBackground per salvare i valori di color e ref dell’oggetto trovato e imposta color e ref con i valori passati tramite spec. Restituisce oldBackground, questo verrà poi utilizzato dal command che ha invocato questo metodo per poter eliminare le modifiche effettuate in caso venisse richiesta una operazione di annulla / ripristina.
			\end{itemize}
			\item \textbf{editColor}(spec:object)
			\begin{itemize}
				\item \textbf{Accesso}: Public;
				\item \textbf{Tipo di ritorno}: SlideShowElement;
				\item \textbf{Descrizione}: scorre il campo dati che contiene gli oggetti di tipo SlideShowElements in Presentazione per trovare l’oggetto con il campo spec.id corrispondente, crea un oggetto oldColor per contenere il valore di color dell’oggetto trovato e imposta color con il valore passato tramite spec. Restituisce oldColor, questo verrà poi utilizzato dal command che ha invocato questo metodo per poter eliminare le modifiche effettuate in caso venisse richiesta una operazione di annulla / ripristina.
			\end{itemize}
			\item \textbf{editShape}(spec:object)
			\begin{itemize}
				\item \textbf{Accesso}: Public;
				\item \textbf{Tipo di ritorno}: SlideShowElement;
				\item \textbf{Descrizione}: scorre il campo dati che contiene gli oggetti di tipo SlideShowElements in Presentazione per trovare l’oggetto con il campo spec.id corrispondente, crea un oggetto oldShape per contenere il valore di shape dell’oggetto trovato e imposta shape con il valore passato tramite spec. Restituisce oldShape, questo verrà poi utilizzato dal command che ha invocato questo metodo per poter eliminare le modifiche effettuate in caso venisse richiesta una operazione di annulla / ripristina.
			\end{itemize}
			\item \textbf{editContent}(spec:object)
			\begin{itemize}
				\item \textbf{Accesso}: Public;
				\item \textbf{Tipo di ritorno}: SlideShowElement;
				\item \textbf{Descrizione}: scorre il campo dati che contiene gli oggetti di tipo SlideShowElements::text in Presentazione per trovare l’oggetto con il campo spec.id corrispondente. Se lo trova copia il campo content in un oggetto oldContent e imposta content con il valore passato tramite spec. Restituisce oldContent, questo verrà poi utilizzato dal command che ha invocato questo metodo per poter eliminare le modifiche effettuate in caso venisse richiesta una operazione di annulla / ripristina.
			\end{itemize}
			\item \textbf{editFont}(spec:object)
			\begin{itemize}
				\item \textbf{Accesso}: Public;
				\item \textbf{Tipo di ritorno}: SlideShowElement;
				\item \textbf{Descrizione}: scorre il campo dati che contiene gli oggetti di tipo SlideShowElements::text in Presentazione per trovare l’oggetto con il campo spec.id corrispondente. Se lo trova copia il campo Font\ped{g} e fontSize in un oggetto oldFont e imposta Font\ped{g} con il valore di newFont e fontSize con i valori passati tramite spec. Restituisce oldFont, questo verrà poi utilizzato dal command che ha invocato questo metodo per poter eliminare le modifiche effettuate in caso venisse richiesta una operazione di annulla / ripristina.
			\end{itemize}
			\item \textbf{portaAvanti}(spec:object)
			\begin{itemize}
				\item \textbf{Accesso}: Public;
				\item \textbf{Tipo di ritorno}: Void;
				\item \textbf{Descrizione}: scorre il campo dati che contiene gli oggetti di tipo SlideShowElements in Presentazione per trovare l’oggetto con il campo spec.id corrispondente. Se lo trova aumenta il valore di zIndex.
			\end{itemize}
			\item \textbf{portaDietro}(spec:object)
			\begin{itemize}
				\item \textbf{Accesso}: Public;
				\item \textbf{Tipo di ritorno}: Void;
				\item \textbf{Descrizione}: scorre il campo dati che contiene gli oggetti di tipo SlideShowElements in Presentazione per trovare l’oggetto con il campo spec.id corrispondente. Se lo trova diminuisce il valore di zIndex.
			\end{itemize}
			\item \textbf{addFrameToMainPath}(spec:object)
			\begin{itemize}
				\item \textbf{Accesso}: Public;
				\item \textbf{Tipo di ritorno}: Void;
				\item \textbf{Descrizione}: aggiunge il Frame\ped{g} spec.id nel Percorso\ped{g} principale nella posizione spec.pos.
			\end{itemize}
			\item \textbf{removeFrameFromMainPath}(id:integer)
			\begin{itemize}
				\item \textbf{Accesso}: Public;
				\item \textbf{Tipo di ritorno}: Void;
				\item \textbf{Descrizione}: scorre il Percorso\ped{g} principale in Presentazione per trovare l’oggetto con il campo id corrispondente. Se lo trova copia la posizione in un oggetto oldFrame e elimina dal Percorso\ped{g} il Frame\ped{g}. Restituisce oldFrame, questo verrà poi utilizzato dal command che ha invocato questo metodo per poter eliminare le modifiche effettuate in caso venisse richiesta una operazione di annulla / ripristina.
			\end{itemize}
			\item \textbf{addChoicePath}(id)
			\begin{itemize}
				\item \textbf{Accesso}: Public;
				\item \textbf{Tipo di ritorno}: Integer;
				\item \textbf{Descrizione}: metodo che aggiunge un nuovo Percorso\ped{g} a scelta id a Presentazione e ritorna l'id del Percorso\ped{g}.
			\end{itemize}
			\item \textbf{deleteChoicePath}(pathId)
			\begin{itemize}
				\item \textbf{Accesso}: Public;
				\item \textbf{Tipo di ritorno}: SlideShowElement;
				\item \textbf{Descrizione}: metodo che elimina il Percorso\ped{g} a scelta pathId da Presentazione e ritorna l'oggetto eliminato, questo verrà poi utilizzato dal command che ha invocato questo metodo per poter eliminare le modifiche effettuate in caso venisse richiesta una operazione di annulla / ripristina.
			\end{itemize}
			\item \textbf{addFrameToChoicePath}(spec:object)
			\begin{itemize}
				\item \textbf{Accesso}: Public;
				\item \textbf{Tipo di ritorno}: Void;
				\item \textbf{Descrizione}: aggiunge il Frame\ped{g} spec.id al Percorso\ped{g} a scelta spec.pathId.
			\end{itemize}
			\item \textbf{removeFrameFromChoicePath}(spec:object)
			\begin{itemize}
				\item \textbf{Accesso}: Public;
				\item \textbf{Tipo di ritorno}: SlideShowElement;
				\item \textbf{Descrizione}: scorre il Percorso\ped{g} a scelta spec.pathId in Presentazione per trovare l’oggetto con il campo spec.id corrispondente. Se lo trova copia posizione, id e id del Percorso\ped{g} in un oggetto obj e elimina dal Percorso\ped{g} il Frame\ped{g}. Restituisce obj, questo verrà poi utilizzato dal command che ha invocato questo metodo per poter eliminare le modifiche effettuate in caso venisse richiesta una operazione di annulla / ripristina.
			\end{itemize}
		\end{itemize}
		
	}
