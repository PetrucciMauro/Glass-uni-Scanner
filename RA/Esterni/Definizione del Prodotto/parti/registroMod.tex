\Large{\textbf{Registro delle modifiche}}\\
\normalsize

%	Ordine di inserimento: dall'ultima versione alla prima
\renewcommand*{\arraystretch}{1.4}
\begin{longtable} [c]{|>{\centering\arraybackslash}m{2cm} | >{\centering\arraybackslash}m{4cm} | >{\centering\arraybackslash}m{3cm} | >{\centering\arraybackslash}m{6cm} |}
		\caption{Versionamento del documento \label{tab:versionamento}}\\
		 \hline
		 \textbf{Versione} & \textbf{Autore} & \textbf{Data} & \textbf{Descrizione}\\
		 \hline
		 \endfirsthead
		 \hline
		 \textbf{Versione} & \textbf{Autore} & \textbf{Data} & \textbf{Descrizione}\\
		 \hline
		\endhead
		 \hline
		 \endfoot
		 \hline
		 \endlastfoot
		 \hline
		 1.0.5 & \GP & 09-09-2015 & Inserimento sezioni relative al tracciamento; inserito HomeOffline, HomeOfflineController, ExecutionOffline, ExecutionOfflineController \\
  		 \hline
  		 1.0.4 & \BM& 09-09-2015 & Correzione grammaticale classe InsertEditRemove\\
  		 \hline
  		 1.0.4 & \PM & 09-09-2015 & Aggiornamento NodeServer \\
  		 \hline
  		 1.0.3 & \VG & 09-09-2015 & Aggiornato Command: Invoker; Model: SlideShowElements \\
 		 \hline
  		 1.0.2 & \VG & 08-09-2015 & Aggiornata descrizione Model, classe InsertEditRemove  \\
 		 \hline
  		 1.0.1 & \TP & 08-09-2015 & Aggiornata Descrizione Generale, classi della View: Edit, Execution, Profile, Home, Login; NodeServer; controller: ProfileController, Services; Node \\
 		 \hline	
		 1.0.0 & \FM & 20-08-2015 & Approvazione Documento \\
		 \hline
		 0.8.5 & \PM & 18-08-2015 & Modifica del nome del file in "Definizione di Prodotto" \\
		 \hline
		 0.8.4 & \PM & 18-08-2015 & Inserimento sezioni relative al tracciamento; aggiornati test; aggiunta consuntivo per la fase di codifica \\
		 \hline
		 0.8.3 & \BM & 10-08-2015 & Aggiornato Premi::Controller, Premi:: View \\
 		 \hline
		 0.8.2 & \BM & 07-08-2015 & Aggiornato Premi::Controller, Premi:: View \\
		 \hline
		 0.8.1 & \FM & 05-08-2015 & Aggiornato diagramma Loader \\
		 \hline
 		 0.8.0 & \BM & 04-08-2015 & Aggiornato Controller::goRegistrazione, goHome, AccessController \\
 		 \hline
 		 0.7.11 & \FM & 03-08-2015 & Corretto output di NodeServer::PresentationMeta: {GET} /private/api/presentations\\
 		 \hline
 		 0.7.10 & \FM & 02-08-2015 & Corretto serverRelation::MongoRelation, schema di NodeServer, definizione di NodeServer \\
 		 \hline
 		 0.7.9 & \BM, \PM & 31-07-2015 & Aggiornata specifica classi del front-end, Premi::App\\
 		 \hline
 		 0.7.8 & \PM & 28-07-2015 & Inserimento services di Angular in Premi::App\\
 		 \hline
 		 0.7.7 & \PM & 27-07-2015 & Aggiunta di contenuti. Inserimento di Premi::Services, modifica di Premi::Controller\\
 		 \hline
 		 0.7.6 & \VG & 19-07-2015 & Aggiornamento capitolo Premi::Model, metodi di inserimento oggetti della presentazione.\\
 		 \hline
 		 0.7.5 & \TP & 14-07-2015 & Aggiornamento metodi di serverRelation::Registration, MongoRelation, Loader, FileServerRelation, Authentication\\
 		 \hline
 		 0.7.4 & \TP & 02-07-2015 & Aggiornamento node\_server::Premi\_Server\\
 		 \hline
 		 0.7.3 & \TP & 30-06-2015 & Aggiornamento schema del backEndProgettazione\\
 		 \hline
   		 0.7.2 & \BM & 28-06-2015 & Aggiornamento Controller::InsertFrame\\	
  		 \hline
  		 0.7.1 & \TP & 27-06-2015 & Aggiornamento Model:: ServerRelations, Model::MongoRelations\\	
 		 \hline	 	 
		 0.7.0 & \BM & 24-06-2015 & Aggiunta di contenuti. Inserimento del capitolo Package::Premi::Controller\\	
		 \hline		 
		 0.5.0 & \VG & 20-06-2015 & Aggiunta di contenuti. Inserimento del capitolo Package::Premi::Model\\	
		 \hline		 
		 0.4.0 & \FM & 15-06-2015 & Aggiunta di contenuti. Inserimento del capitolo Package::Premi::View\\	
		 \hline		 
		 0.3.0 & \TP & 12-06-2015 & Aggiunta di contenuti. Inserimento del capitolo Standard di Progetto\\		 
		 \hline		 
		 0.2.0 & \GP & 09-06-2015 & Aggiunta di contenuti. Inserimento del capitolo Introduzione e Descrizione generale\\		 
		 \hline
		 0.1.0 & \GP & 08-06-2015 & Stesura dello scheletro del documento\\		 
\end{longtable}

\newpage
\Large{\textbf{Storico }}\\
\normalsize \\

%	Per mettere più tabelle di storico basta copiare e incollare la seguente porzione di codice e modificarla in base ai dati nuovi
\noindent \textbf{RP -> RQ}
\label{tabVers1}
\begin{table}[h]
	\begin{tabular}{p{0.2\textwidth} p{0.7\textwidth}}
		\toprule \textbf{Versione 1..0.0}	&	\textbf{Nominativo}\\
		\midrule Redazione	& \FM, \VG, \TP, \BM\\
		\midrule Verifica & \PM \\
		\midrule Approvazione	& \FM\\
		\bottomrule
	\end{tabular}
	\caption{Storico ruoli RP -> RQ}
\end{table}
