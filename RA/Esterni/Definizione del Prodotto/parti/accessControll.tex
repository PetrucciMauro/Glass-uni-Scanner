	\subsubsection{Registration}{
	
	\begin{figure}[H]
		\includegraphics[scale=0.8]{\imgs {Registration}.pdf}
		\label{fig:nodeAPI}
		\caption{Diagramma classe Model::serverRelation::accessControll::Registration}
	\end{figure}
		\textbf{Attributi:}
			\begin{itemize}
			\item \textbf{-messageState}
				\begin{itemize}
				\item \textbf{Accesso:} private
				\item \textbf{Tipo:} string
				\item \textbf{Descrizione:} messaggio ritornato dall'ultima chiamata di autenticazione al Server\ped{g}
				\end{itemize}
			\end{itemize}
	
		\textbf{Metodi:}
			\begin{itemize}
			\item \textbf{register(user : string, password : string) : bool}
				\begin{itemize}
				\item \textbf{Accessibilit\`{a}:} public
				\item \textbf{Descrizione:} chiama il servizio /account/register POST con passando i parametri attuali, scrive in messageState lo stato ritornato dalla chiamata a nodeApi, ritorna true solo se la chiamata va a buon fine
				\item \textbf{Tipo di ritorno:} bool
				\end{itemize}
			
			\item \textbf{getMessage() : string}
				\begin{itemize}
				\item \textbf{Accessibilit\`{a}:} public
				\item \textbf{Descrizione:} ritorna il contenuto di messageState
				\item \textbf{Tipo di ritorno:} string
				\end{itemize}

			\end{itemize}
	}
	
	\subsubsection{Authentication}{
	
	\begin{figure}[H]
		\includegraphics[scale=0.8]{\imgs {Authentication}.pdf}
		\label{fig:nodeAPI}
		\caption{Diagramma classe Model::serverRelation::accessControll::Authentication}
	\end{figure}
	
	\textbf{Attributi:}
			\begin{itemize}
			\item \textbf{-messageState}
				\begin{itemize}
				\item \textbf{Accesso:} private
				\item \textbf{Tipo:} string
				\item \textbf{Descrizione:} messaggio ritornato dall'ultima chiamata di autenticazione al Server\ped{g}
				\end{itemize}
				
			\item \textbf{-token}
				\begin{itemize}
				\item \textbf{Accesso:} private
				\item \textbf{Tipo:} string
				\item \textbf{Descrizione:} stringa per l?autenticazione ai servizi Server\ped{g} nodeJs
				\item \textbf{Note:} valore ?empty? se non \`{e} ancora stato richiesto il token, \`{e} stato cancellato, oppure l? autenticazione \`{e} fallita
				\end{itemize}
			\end{itemize}
	
		\textbf{Metodi:}
		\begin{itemize}
		\item \textbf{authenticate(user : string, password : string) : bool}
			\begin{itemize}
			\item \textbf{Accessibilit\`{a}:} public
			\item \textbf{Descrizione:} chiama il servizio Account\ped{g}/authenticate, ritorna true se la chiamata \`{e} andata a buon fine, in questo caso mette nel campo token il token ritornato dal Server\ped{g} nodeJs

			\item \textbf{Tipo di ritorno:} bool
			\end{itemize}
			
		\item \textbf{deAuthenticate() : bool}
			\begin{itemize}
			\item \textbf{Accessibilit\`{a}:} public
			\item \textbf{Descrizione:} de-autentica l?utente dai servizi Server\ped{g} cancellando il contenuto del campo dati token, ritorna true se l?operazione va a buon fine
			\item \textbf{Tipo di ritorno:} bool
			\end{itemize}
			
		\item \textbf{changePassword(user:string, password:string, newpassword:string) : bool}
			\begin{itemize}
			\item \textbf{Accessibilit\`{a}:} public
			\item \textbf{Descrizione:} modifica la password dell'utente specificato
			\item \textbf{Tipo di ritorno:} bool
			\end{itemize}
			
		\item \textbf{getToken() : string}
			\begin{itemize}
			\item \textbf{Accessibilit\`{a}:} public
			\item \textbf{Descrizione:} ritorna il contenuto del campo dati token
			\item \textbf{Tipo di ritorno:} string
			\end{itemize}
			
		\item \textbf{getMessage() : string}
			\begin{itemize}
			\item \textbf{Accessibilit\`{a}:} public
			\item \textbf{Descrizione:} ritorna il contenuto di messageState
			\item \textbf{Tipo di ritorno:} string
			\end{itemize}

		\end{itemize}
	}
	
	
	
	
	
	
	
	
	
	