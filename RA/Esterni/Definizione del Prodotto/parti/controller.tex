\section{Package Premi::\-Controller}{
	\label{sec:controller}
	Tutti i package seguenti appartengono al package Premi, quindi per ognuno di essi lo scope sarà: Premi::\-[nome package].\\\\
	\textbf{\tipo}: contiene le classi che gestiscono i segnali e le chiamate effettuati dalla View. Il Controller è interamente implementato utilizzando la tecnologia Angular.js.\\
	\textbf{\relaz}: comunica con la View ricevendone i segnali e mantenendola aggiornata e con il Model per recuperare informazioni, modificarle e salvare i dati correttamente nel server.\\

	\subsection{Controller::\-premiApp}{
		\label{sec:premiapp}
		\textbf{Funzione}\\
		\indent Questa classe si occuperà di fare il bootstrap dell'applicazione instanziando la rootscope  e iniettando tutti i moduli necessari. Verranno configurate tutte le impostazioni necessarie relative al funzionamento di Angular.\\
		\textbf{Relazioni d'uso con altri moduli}\\
		\indent Questa classe utilizzerà le seguenti classi:
		\begin{itemize}
			\item Controller;
			\item View::Pages;
			\item ngRoute:Object\\
				\indent Modulo Angular che inietta i servizi necessari per il routing;
			\item ngMaterial:Object\\
				\indent Modulo Angular che inietta i servizi necessari per l'utilizzo di Angular Material;
			\item ngStorage:Object\\
				\indent Modulo Angular che inietta i servizi necessari per l'utilizzo dello storage locale.
			\item \textdollar routeProvider:Object\\
				\indent Questo campo dati rappresenta il servizio che collega tra loro controller, view e l'URL\ped{g} corrente nel Browser\ped{g};
			\item \textdollar mdIconProvider e \textdollar mdThemingProvider\\
				\indent Moduli di Angular Material
			\item \textdollar httpProvider\\
				\indent Provider Angular per configurare il servizio http;
			\item \textdollar provide\\
				\indent Modulo Angular utilizzato per la gestione delle eccezioni;
			\item \textdollar locationProvider\\
				\indent Provider Angular utilizzato per rendere disponibile il servizio di reindirizzamento delle pagine.
		\end{itemize}
	}

	\subsection {Controller::\-Services}{
		\label{sec:services}
		\textbf{\tipo}: questo package contiene tutte le funzionalità di base richieste dall'applicazione tra cui l'autenticazione al server, il reindirizzamento delle pagine e l'upload dei file nel server. \\
		\textbf{\relaz}: comunica con gli altri package del controller per la gestione delle funzioni\ped{g} da eseguire e con il model per la gestione delle componenti necessarie.

		\subsubsection{Services::\-Main}{
			\label{sub:servicesMain}
			\textbf{Funzione}\\
			\indent Questa classe si occuperà di eseguire le funzioni\ped{g} base dell'applicazione, in particolare autenticazione e registrazione al Server\ped{g} degli utenti.\\
			\textbf{Relazioni d'uso con altri moduli}\\
			\indent Questa classe utilizzerà le seguenti classi:
			\begin{itemize}
				\item Controller::Services::\-Utils;
				\item Model::serverRelation::accessControl::Authentication;
				\item Model::serverRelation::accessControl::Registration;
				\item localStorage:Object\\
					\indent Servizio angular che permette il salvataggio in locale di oggetti necessari al  garantire delle funzioni\ped{g} dell'applicazione.
			\end{itemize}
			\textbf{Attributi}
			\begin{itemize}
				\item \textbf{login}
				\begin{itemize}
					\item \textbf{Accesso}: Private;
					\item \textbf{Tipo}: Oggetto;
					\item \textbf{Descrizione}: oggetto che mantiene la sessione corrente.
				\end{itemize}
			\end{itemize}
			\textbf{Metodi}
			\begin{itemize}
				\item \textbf{value}()
				\begin{itemize}
					\item \textbf{Accesso}: Private;
					\item \textbf{Tipo di ritorno}: Void;
					\item \textbf{Descrizione}: metodo che controlla se è stato effettuato un refresh della pagina, in tal caso ripristina Login\ped{g}.
				\end{itemize}
				\item \textbf{register}(formData, success, error)
				\begin{itemize}
					\item \textbf{Accesso}: Public;
					\item \textbf{Tipo di ritorno}: Void;
					\item \textbf{Descrizione}: metodo che, attraverso l'oggetto formData contenente le credenziali di accesso, effettua la registrazione al Server\ped{g} di un nuovo utente richiamando il metodo register() di Registration. Se l'operazione ha successo, viene autenticato l'utente ed invocato success altrimenti error.
				\end{itemize}
				\item \textbf{login}(formData, success, error)
				\begin{itemize}
					\item \textbf{Accesso}: Public;
					\item \textbf{Tipo di ritorno}: Void;
					\item \textbf{Descrizione}: metodo che, attraverso l'oggetto formData contenente le credenziali di accesso, effettua l'autenticazione al Server\ped{g} di un utente richiamando il metodo authenticate() di Authentication. Se l'operazione ha successo viene invocato success altrimenti error.
				\end{itemize}
				\item \textbf{logout}(success, error)
				\begin{itemize}
					\item \textbf{Accesso}: Public;
					\item \textbf{Tipo di ritorno}: Void;
					\item \textbf{Descrizione}: metodo che effettua il Logout\ped{g} dal Server\ped{g} richiamando il metodo deauthenticate() di Authentication. Se l'operazione ha successo viene invocato success altrimenti error.
				\end{itemize}
				\item \textbf{changepassword}(formData, success, error)
				\begin{itemize}
					\item \textbf{Accesso}: Public;
					\item \textbf{Tipo di ritorno}: Void;
					\item \textbf{Descrizione}: metodo che, attraverso l'oggetto formData contenente le credenziali di accesso e la nuova password, effettua il cambio della password di un utente richiamando il metodo changepassword() di Authentication. Se l'operazione ha successo viene invocato success altrimenti error.
				\end{itemize}
				\item \textbf{getToken}()
				\begin{itemize}
					\item \textbf{Accesso}: Public;
					\item \textbf{Tipo di ritorno}: JSONWebToken;
					\item \textbf{Descrizione}: metodo che ritorna il token di sessione richiamando getToken() di Authentication.
				\end{itemize}
				\item \textbf{getUser}()
				\begin{itemize}
					\item \textbf{Accesso}: Public;
					\item \textbf{Tipo di ritorno}: Object;
					\item \textbf{Descrizione}: metodo che ritorna l'utente attualmente autenticato con il Server\ped{g}.
				\end{itemize}
			\end{itemize} 
		}
		\subsubsection{Services::\-Upload}{
			\label{sub:servicesUpload}
			\textbf{Funzione}\\
			\indent Questa classe si occuperà di eseguire l'upload di File\ped{g} media nel database.\\
			\textbf{Relazioni d'uso con altri moduli}\\
			\indent Questa classe utilizzerà le seguenti classi:
			\begin{itemize}
				\item Controller::Services::\-Main;
				\item Controller::Services::\-Utils.
			\end{itemize}
			\textbf{Attributi}
			\begin{itemize}
				\item \textbf{image}
				\begin{itemize}
					\item \textbf{Accesso}: Private;
					\item \textbf{Tipo}: Array;
					\item \textbf{Descrizione}: array contenente i formati immagini accettati per l'upload.
				\end{itemize}
				\item \textbf{audio}
				\begin{itemize}
					\item \textbf{Accesso}: Private;
					\item \textbf{Tipo}: Array;
					\item \textbf{Descrizione}: array contenente i formati audio accettati per l'upload.
				\end{itemize}
				\item \textbf{video}
				\begin{itemize}
					\item \textbf{Accesso}: Private;
					\item \textbf{Tipo}: Array;
					\item \textbf{Descrizione}: array contenente i formati video accettati per l'upload.
				\end{itemize}
			\end{itemize}
			\textbf{Metodi}
			\begin{itemize}
				\item \textbf{uploadmedia}(files\ped{g}, callback)
				\begin{itemize}
					\item \textbf{Accesso}: Public;
					\item \textbf{Tipo di ritorno}: Void;
					\item \textbf{Descrizione}: metodo che invia una richiesta XMLHttpRequest a \textit{[hostname]\-/private/api/files/\-[image|audio|video]/\-[nome\_file]} effettuando l'upload dei File\ped{g} contenuti nell'array files\ped{g} passato come parametro. Se l'operazione ha successo viene richiamato callback, altrimenti viene lanciato un errore.
				\end{itemize}
				\item \textbf{isImage}(files\ped{g})
				\begin{itemize}
					\item \textbf{Accesso}: Public;
					\item \textbf{Tipo di ritorno}: Bool;
					\item \textbf{Descrizione}: metodo che ritorna true se i File\ped{g} contenuti nell'array files\ped{g}, passato come parametro, rispettano almeno uno tra i formati contenuti nell'array image, altrimenti ritorna false.
				\end{itemize}
				\item \textbf{isAudio}(files\ped{g})
				\begin{itemize}
					\item \textbf{Accesso}: Public;
					\item \textbf{Tipo di ritorno}: Bool;
					\item \textbf{Descrizione}: metodo che ritorna true se i File\ped{g} contenuti nell'array files\ped{g}, passato come parametro, rispettano almeno uno tra i formati contenuti nell'array audio, altrimenti ritorna false.
				\end{itemize}
				\item \textbf{isVideo}(files\ped{g})
				\begin{itemize}
					\item \textbf{Accesso}: Public;
					\item \textbf{Tipo di ritorno}: Bool;
					\item \textbf{Descrizione}: metodo che ritorna true se i File\ped{g} contenuti nell'array files\ped{g}, passato come parametro, rispettano almeno uno tra i formati contenuti nell'array video, altrimenti ritorna false.
				\end{itemize}
				\item \textbf{getFileUrl}(File\ped{g})
				\begin{itemize}
					\item \textbf{Accesso}: Public;
					\item \textbf{Tipo di ritorno}: String;
					\item \textbf{Descrizione}: metodo che ritorna il Percorso\ped{g} di salvataggio del parametro File\ped{g} rispetto all'utente corrente.
				\end{itemize}
			\end{itemize} 
		}
		\subsubsection{Services::\-Utils}{
			\label{sub:servicesUtils}
			\textbf{Funzione}\\
			\indent Questa classe si occuperà di eseguire piccole funzionalità utili ad ogni parte dell'applicazione.\\
			\textbf{Metodi}
			\begin{itemize}
				\item \textbf{decodeToken}(token)
				\begin{itemize}
					\item \textbf{Accesso}: Public;
					\item \textbf{Tipo di ritorno}: Object;
					\item \textbf{Descrizione}: metodo che decodifica token, passato come parametro, e ritorna l'oggetto utente corrispondente.
				\end{itemize}
				\item \textbf{grade}(password)
				\begin{itemize}
					\item \textbf{Accesso}: Public;
					\item \textbf{Tipo di ritorno}: String;
					\item \textbf{Descrizione}: metodo che determina la robustezza del parametro password. La lunghezza minima di una password è stata impostata a sei caratteri.
				\end{itemize}
				\item \textbf{hostname}()
				\begin{itemize}
					\item \textbf{Accesso}: Public;
					\item \textbf{Tipo di ritorno}: String;
					\item \textbf{Descrizione}: metodo che ritorna il dominio dell'applicazione.
				\end{itemize}
				\item \textbf{isUndefined}(object)
				\begin{itemize}
					\item \textbf{Accesso}: Public;
					\item \textbf{Tipo di ritorno}: Bool;
					\item \textbf{Descrizione}: metodo che ritorna true se il parametro object risulta indefinito, altrimenti ritorna false.
				\end{itemize}
				\item \textbf{isObject}(object)
				\begin{itemize}
					\item \textbf{Accesso}: Public;
					\item \textbf{Tipo di ritorno}: Void;
					\item \textbf{Descrizione}: metodo che ritorna true se il parametro object risulta definito, altrimenti ritorna false.
				\end{itemize}
				\item \textbf{encrypt}(string)
				\begin{itemize}
					\item \textbf{Accesso}: Public;
					\item \textbf{Tipo di ritorno}: String;
					\item \textbf{Descrizione}: metodo che ritorna il parametro string criptato. Il metodo di criptaggio scelto è lo SHA-1.
				\end{itemize}
			\end{itemize}
		}
		\subsubsection{Services::\-SharedData}{
			\label{sub:servicesSharedData}
			\textbf{Funzione}\\
			\indent Questa classe mantiene in memoria la presentazione sulla quale l'utente sta lavorando.\\
			\textbf{Relazioni d'uso con altri moduli}\\
			\indent Questa classe utilizzerà le seguenti classi:
			\begin{itemize}
				\item Controller::Services::\-Utils;
				\item Controller::Services::\-Main;
				\item Model::serverRelation::mongoRelation;
				\item localStorage:Object\\
					\indent Servizio angular che permette il salvataggio in locale di oggetti necessari al  garantire delle funzioni\ped{g} dell'applicazione.
			\end{itemize}
			\textbf{Attributi}\\
			\begin{itemize}
				\item \textbf{myPresentation}
				\begin{itemize}
					\item \textbf{Accesso}: Private;
					\item \textbf{Tipo}: Object;
					\item \textbf{Descrizione}: oggetto che rappresenta l'attuale presentazione aperta.
				\end{itemize}
			\end{itemize}
			\textbf{Metodi}
			\begin{itemize}
				\item \textbf{getPresentazione}(idSlideShow)
				\begin{itemize}
					\item \textbf{Accesso}: Public;
					\item \textbf{Tipo di ritorno}: Object;
					\item \textbf{Descrizione}: metodo che, nel caso in cui il parametro idSlideShow sia definito, richiama il metodo Model::\-serverRelation::\-mongoRelation::\-getPresentation() passandogli il parametro idSlideShow e assegnando il risultato a myPresentation. In ogni caso myPresentation viene ritornato.
				\end{itemize}
			\end{itemize}
		}
		\subsubsection{Services::\-toPages}{
			\label{sub:servicestoPages}
			\textbf{Funzione}\\
			\indent Questa classe si occuperà di eseguire i reindirizzamenti alle pagine corrette.\\
			\textbf{Relazioni d'uso con altri moduli}\\
			\indent Questa classe utilizzerà le seguenti classi:
			\begin{itemize}
				\item Controller::Services::\-Utils;
				\item Controller::Services::\-Main;
				\item Controller::Services::\-SharedData;
				\item \$http:Object\\
					\indent Servizio Angular che permette la comunicazione in remoto con un Server\ped{g}.
				\item \$location:Object\\
					\indent Servizio Angular che gestisce gli indirizzi\ped{g} URL\ped{g}.
			\end{itemize}
			\textbf{Metodi}
			\begin{itemize}
				\item \textbf{sendRequest}(dest, success, error)
				\begin{itemize}
					\item \textbf{Accesso}: Private;
					\item \textbf{Tipo di ritorno}: Object;
					\item \textbf{Descrizione}: metodo che ritorna una richiesta http all'Indirizzo\ped{g} definito dal parametro dest. Se l'operazione ha successo viene invocato success altrimenti error.
				\end{itemize}
				\item \textbf{loginpage}()
				\begin{itemize}
					\item \textbf{Accesso}: Public;
					\item \textbf{Tipo di ritorno}: Object;
					\item \textbf{Descrizione}: metodo che permette di accedere alla pagina di Login\ped{g}. Esso richiama il metodo sendRequest(), per effettuare una richiesta a \textit{/publicpages/login}, il quale, se ha successo, reindirizza alla pagina richiesta.
				\end{itemize}
				\item \textbf{registrazionepage}()
				\begin{itemize}
					\item \textbf{Accesso}: Public;
					\item \textbf{Tipo di ritorno}: Object;
					\item \textbf{Descrizione}: metodo che permette di accedere alla pagina di Registrazione. Esso richiama il metodo sendRequest(), per effettuare una richiesta a \textit{/publicpages/registrazione}, il quale, se ha successo, reindirizza alla pagina richiesta.
				\end{itemize}
				\item \textbf{homepage}()
				\begin{itemize}
					\item \textbf{Accesso}: Public;
					\item \textbf{Tipo di ritorno}: Object;
					\item \textbf{Descrizione}: metodo che permette di accedere alla pagina Home. Esso richiama il metodo sendRequest(), per effettuare una richiesta a \textit{/private/home}, il quale, se ha successo, reindirizza alla pagina richiesta.
				\end{itemize}
				\item \textbf{profilepage}()
				\begin{itemize}
					\item \textbf{Accesso}: Public;
					\item \textbf{Tipo di ritorno}: Object;
					\item \textbf{Descrizione}: metodo che permette di accedere alla pagina Profile. Esso richiama il metodo sendRequest(), per effettuare una richiesta a \textit{/private/profile}, il quale, se ha successo, reindirizza alla pagina richiesta.
				\end{itemize}
				\item \textbf{editpage}(slideId)
				\begin{itemize}
					\item \textbf{Accesso}: Public;
					\item \textbf{Tipo di ritorno}: Object;
					\item \textbf{Descrizione}: metodo che permette di accedere alla pagina di Edit. Esso richiama il metodo sendRequest(), per effettuare una richiesta a \textit{/private/edit}, il quale, se ha successo, reindirizza alla pagina richiesta e richiama il metodo SharedData.forEdit() passandogli il parametro slideId.
				\end{itemize}
				\item \textbf{executionpage}(slideId)
				\begin{itemize}
					\item \textbf{Accesso}: Public;
					\item \textbf{Tipo di ritorno}: Object;
					\item \textbf{Descrizione}: metodo che permette di accedere alla pagina di Execution. Esso richiama il metodo sendRequest(), per effettuare una richiesta a \textit{/private/execution}, il quale, se ha successo, reindirizza alla pagina richiesta e richiama il metodo SharedData.forEdit() passandogli il parametro slideId.
				\end{itemize}
			\end{itemize} 
		}
	}

	\subsubsection{Controller::\-HeaderController}
		\label{sub:HeaderController}
		\textbf{Funzione}\\
		\indent Questa classe si occuperà di controllare l'Header dell'applicazione.\\
		\textbf{Relazioni d'uso con altri moduli}\\
		\indent Questa classe utilizzerà le seguenti classi:
		\begin{itemize}
			\item View::\-Pages::\-Index;
			\item Controller::Services::\-Utils;
			\item Controller::Services::\-Main;
			\item Controller::Services::\-toPages;
			\item \$scope:Object\\
				\indent Oggetto Angular che lega Controller e View. Nello specifico, permette l'esecuzione di espressioni che mantengono aggiornata la View e l'implementazione del 2-way data binding. Possono esistere più scope, strutturati in modo gerarchico, i quali simulano la struttura DOM dell'applicazione;
			\item \$rootScope:Object\\
				\indent Questo campo dati rappresenta lo scope radice dell’applicazione. Tutti gli altri scope discendono da questo.
		\end{itemize}

		\textbf{Metodi}
		\begin{itemize}
			\item \textbf{goLogin}()
			\begin{itemize}
				\item \textbf{Accesso}: Public;
				\item \textbf{Tipo di ritorno}: Void;
				\item \textbf{Descrizione}: metodo che richiama loginpage() di toPages per effettuare il reindirizzamento alla pagina di Login\ped{g}.
			\end{itemize}
			\item \textbf{goRegistrazione}()
			\begin{itemize}
				\item \textbf{Accesso}: Public;
				\item \textbf{Tipo di ritorno}: Void;
				\item \textbf{Descrizione}: metodo che richiama registrazionepage() di toPages per effettuare il reindirizzamento alla pagina di registrazione.
			\end{itemize}
			\item \textbf{goHome}()
			\begin{itemize}
				\item \textbf{Accesso}: Public;
				\item \textbf{Tipo di ritorno}: Void;
				\item \textbf{Descrizione}: metodo che richiama homepage() di toPages per effettuare il reindirizzamento alla pagina home.
			\end{itemize}
			\item \textbf{goProfile}()
			\begin{itemize}
				\item \textbf{Accesso}: Public;
				\item \textbf{Tipo di ritorno}: Void;
				\item \textbf{Descrizione}: metodo che richiama profilepage() di toPages per effettuare il reindirizzamento alla pagina profile.
			\end{itemize}
			\item \textbf{who}()
			\begin{itemize}
				\item \textbf{Accesso}: Public;
				\item \textbf{Tipo di ritorno}: String;
				\item \textbf{Descrizione}: metodo che ritorna lo username dell'utente attualmente autenticato richiamando il metodo getUser() di Main.
			\end{itemize}
			\item \textbf{isToken}()
			\begin{itemize}
				\item \textbf{Accesso}: Public;
				\item \textbf{Tipo di ritorno}: Boolean;
				\item \textbf{Descrizione}: metodo che verifica l'effettiva autenticazione dell'utente richiamando il metodo getToken() di Main.
			\end{itemize}
			\item \textbf{logout}()
			\begin{itemize}
				\item \textbf{Accesso}: Public;
				\item \textbf{Tipo di ritorno}: Void;
				\item \textbf{Descrizione}: metodo che richiama il metodo logout() di Main per effettuare il Logout\ped{g} dal Server\ped{g}. Se l'operazione va a buon fine, viene effettuato il reindirizzamento alla pagina di Login\ped{g} richiamando il metodo loginpage() di toPages.
			\end{itemize}
			\item \textbf{error}()
			\begin{itemize}
				\item \textbf{Accesso}: Public;
				\item \textbf{Tipo di ritorno}: String;
				\item \textbf{Descrizione}: metodo che ritorna l'errore individuato; esso deve essere posto all'interno di \$rootScope.error.
			\end{itemize}
		\end{itemize}

	\subsubsection{Controller::\-AccessController}{
		\label{sub:AccessController}
		\textbf{Funzione}\\
			\indent Questa classe si occuperà di controllare che le credenziali di accesso siano corrette nel caso dell'autenticazione oppure di registrare un nuovo utente.\\
		\textbf{Relazioni d'uso con altri moduli}\\
			\indent Questa classe utilizzerà le seguenti classi:
		\begin{itemize}
			\item View::\-Pages::\-Login\ped{g};
			\item View::\-Pages::\-Registrazione;
			\item Controller::Services::\-Utils;
			\item Controller::Services::\-Main;
			\item Controller::Services::\-toPages;
			\item \$scope:Object\\
	    		\indent Oggetto Angular che lega Controller e View. Nello specifico, permette l'esecuzione di espressioni che mantengono aggiornata la View e l'implementazione del 2-way data binding. Possono esistere più scope, strutturati in modo gerarchico, i quali simulano la struttura DOM dell'applicazione.
		\end{itemize}
		\textbf{Attributi}\\
	    \begin{itemize}
	    	\item \textbf{user}
			\begin{itemize}
				\item \textbf{Accesso}: Private;
				\item \textbf{Tipo}: Object;
				\item \textbf{Descrizione}: oggetto contenente username e password derivanti dal form della pagina html.
			\end{itemize}
			\item \textbf{usernameError}
			\begin{itemize}
				\item \textbf{Accesso}: Public;
				\item \textbf{Tipo}: String;
				\item \textbf{Descrizione}: stringa contenente l'eventuale errore accorso con l'inserimento dello username.
			\end{itemize}
			\item \textbf{passwordError}
			\begin{itemize}
				\item \textbf{Accesso}: Public;
				\item \textbf{Tipo}: String;
				\item \textbf{Descrizione}: stringa contenente l'eventuale errore accorso con l'inserimento della password.
			\end{itemize}
	    \end{itemize}
		\textbf{Metodi}
		\begin{itemize}
			\item \textbf{getData}()
			\begin{itemize}
				\item \textbf{Accesso}: Private;
				\item \textbf{Tipo}: Object;
				\item \textbf{Descrizione}: metodo che ritorna un oggetto contenente i campi dati username e password ricavati dallo \$scope. La password viene criptata grazie al metodo encrypt(password) di Utils.
			\end{itemize}
			\item \textbf{reset}()
			\begin{itemize}
				\item \textbf{Accesso}: Public;
				\item \textbf{Tipo di ritorno}: Void;
				\item \textbf{Descrizione}: metodo che cancella i valori delle variabili all'interno di \$scope.
			\end{itemize}
			\item \textbf{login}()
			\begin{itemize}
				\item \textbf{Accesso}: Public;
				\item \textbf{Tipo di ritorno}: Void;
				\item \textbf{Descrizione}: metodo che controlla se user è definito e, in caso affermativo, richiama login(data) di Main per effettuare il Login\ped{g} al Server\ped{g}. Se l'operazione ha successo viene effettuato il reindirizzamento alla pagina home richiamando il metodo homepage() di toPages.
			\end{itemize}
	        \item \textbf{registration}()
			\begin{itemize}
				\item \textbf{Accesso}: Public;
				\item \textbf{Tipo di ritorno}: Void;
				\item \textbf{Descrizione}: metodo che controlla se user è definito e, in caso affermativo, richiama register(data) di Main per effettuare la registrazione al Server\ped{g}. Se l'operazione ha successo viene effettuato il reindirizzamento alla pagina home richiamando il metodo homepage() di toPages.
			\end{itemize}
		\end{itemize}
	}
	\subsubsection{Controller::\-HomeController}{
		\label{sub:homecontroller}
		\textbf{Funzione}\\
		\indent Questa classe si occuperà di gestire i segnali e le chiamate provenienti dalla pagina Home.\\
		\textbf{Relazioni d'uso con altri moduli}\\
		\indent Questa classe utilizzerà le seguenti classi:
		\begin{itemize}
			\item View::\-Pages::\-Home;
			\item Model::\-serverRelation::\-mongoRelation;
			\item Services::\-Utils;
			\item Services::\-Main;
			\item Services::\-toPages;
			\item \$scope:Object\\
				\indent Oggetto Angular che lega Controller e View. Nello specifico, permette l'esecuzione di espressioni che mantengono aggiornata la View e l'implementazione del 2-way data binding. Possono esistere più scope, strutturati in modo gerarchico, i quali simulano la struttura DOM dell'applicazione.
		\end{itemize}
		\textbf{Attributi}\\
		\begin{itemize}
			\item \textbf{mongo}
			\begin{itemize}
				\item \textbf{Accesso}: Private;
				\item \textbf{Tipo}: Object;
				\item \textbf{Descrizione}: oggetto che mantiene una istanza di MongoRelation.
			\end{itemize}
	    \end{itemize}
		\textbf{Metodi}
		\begin{itemize}
			\item \textbf{update}()
			\begin{itemize}
				\item \textbf{Accesso}: Public;
				\item \textbf{Tipo di ritorno}: Void;
				\item \textbf{Descrizione}: metodo che aggiorna i contenuti della pagina home.
			\end{itemize}
			\item \textbf{goEdit}(slideId)
			\begin{itemize}
				\item \textbf{Accesso}: Public;
				\item \textbf{Tipo di ritorno}: Void;
				\item \textbf{Descrizione}: metodo che richiama editpage(slideId) di toPages per effettuare il reindirizzamento alla pagina di edit.
			\end{itemize}
			\item \textbf{goExecute}(slideId)
			\begin{itemize}
				\item \textbf{Accesso}: Public;
				\item \textbf{Tipo di ritorno}: Void;
				\item \textbf{Descrizione}: metodo che richiama executionpage(slideId) di toPages per effettuare il reindirizzamento alla pagina di esecuzione.
			\end{itemize}
			\item \textbf{getSS}()
			\begin{itemize}
				\item \textbf{Accesso}: Public;
				\item \textbf{Tipo di ritorno}: Object;
				\item \textbf{Descrizione}: metodo che richiama getPresentationsMeta() di mongoRelation per visualizzare le presentazioni dell'utente corrente.
			\end{itemize}
			\item \textbf{deleteSlideShow}(slideId)
			\begin{itemize}
				\item \textbf{Accesso}: Public;
				\item \textbf{Tipo di ritorno}: Void;
				\item \textbf{Descrizione}: metodo che richiama deletePresentation(slideId) di mongoRelation per eliminare la presentazione slideId dal database. Se l'operazione ha successo, viene richiamato il metodo update().
			\end{itemize}
			\item \textbf{renameSlideShow}(nameSS, rename)
			\begin{itemize}
				\item \textbf{Accesso}: Public;
				\item \textbf{Tipo di ritorno}: Void;
				\item \textbf{Descrizione}: metodo che richiama renamePresentation(nameSS, rename) di mongoRelation per rinominare la presentazione slideId. Se l'operazione ha successo, viene richiamato il metodo update().
			\end{itemize}
			\item \textbf{createSlideShow}()
			\begin{itemize}
				\item \textbf{Accesso}: Public;
				\item \textbf{Tipo di ritorno}: Void;
				\item \textbf{Descrizione}: metodo che richiama newPresentation() di mongoRelation per creare una nuova presentazione. Se l'operazione ha successo, viene richiamato il metodo update().
			\end{itemize}
			\item \textbf{setStyleSS}(slide)
			\begin{itemize}
				\item \textbf{Accesso}: Public;
				\item \textbf{Tipo di ritorno}: CSS Object;
				\item \textbf{Descrizione}: metodo che ricava il background da slide e ritorna l'oggetto con le proprietà CSS corrispondenti. Se l'immagine di background non è definita, allora nel CSS non verrà riportata tale proprietà.
			\end{itemize}
		\end{itemize}
		}
	\subsubsection{Controller::\-ProfileController}{
		\textbf{Funzione}\\
		\indent Questa classe si occupa di gestire i segnali e le chiamate provenienti dalla pagina profilo di un utente.\\
		\textbf{Relazioni d'uso con altri moduli}\\
		\indent Questa classe utilizzerà le seguenti classi:
		\begin{itemize}
			\item Services::\-Utils;
			\item Services::\-Main;
			\item Services::\-toPages;
			\item \$scope:Object\\
				\indent Oggetto Angular che lega Controller e View. Nello specifico, permette l'esecuzione di espressioni che mantengono aggiornata la View e l'implementazione del 2-way data binding. Possono esistere più scope, strutturati in modo gerarchico, i quali simulano la struttura DOM dell'applicazione.
		\end{itemize}
		\textbf{Attributi}\\
	    \begin{itemize}
	    	\item \textbf{user}
			\begin{itemize}
				\item \textbf{Accesso}: Private;
				\item \textbf{Tipo}: Object;
				\item \textbf{Descrizione}: oggetto contenente username, password e nuova password derivanti dal form della pagina html.
			\end{itemize}
	    \end{itemize}
		\textbf{Metodi}
		\begin{itemize}
			\item \textbf{getData}()
			\begin{itemize}
				\item \textbf{Accesso}: Private;
				\item \textbf{Tipo}: Object;
				\item \textbf{Descrizione}: metodo che ritorna un oggetto contenente i campi dati username, password e newpassword ricavati dallo \$scope. Le password vengono criptate grazie a encrypt(password) di Utils.
			\end{itemize}
			\item \textbf{changepassword}()
			\begin{itemize}
				\item \textbf{Accesso}: Public;
				\item \textbf{Tipo di ritorno}: Void;
				\item \textbf{Descrizione}: metodo che controlla se user è definito e, in caso affermativo, richiama il metodo changepassword() di Main per cambiare la password dell'utente, passandogli i dati da modificare.
			\end{itemize}
		\end{itemize}
	}
	\subsubsection{Controller::\-ExecutionController}{
		\textbf{Funzione}\\
		\indent Questa classe si occuperà di gestire i segnali e le chiamate provenienti dalla pagina di esecuzione.\\
		\textbf{Relazioni d'uso con altri moduli}\\
		\indent Questa classe utilizzerà le seguenti classi:
		\begin{itemize}
			\item View::\-Pages::\-Execution;
			\item Services::\-SharedData;
			\item Services::\-Main;
			\item Services::\-toPages;
			\item Services::\-Utils;
			\item \$scope:Object\\
				\indent Oggetto Angular che lega Controller e View. Nello specifico, permette l'esecuzione di espressioni che mantengono aggiornata la View e l'implementazione del 2-way data binding. Possono esistere più scope, strutturati in modo gerarchico, i quali simulano la struttura DOM dell'applicazione;
			\item \$route:Object\\
				\indent Servizio angular che permette la gestione del routing all'interno dell'applicazione.
		\end{itemize}
		\textbf{Metodi}
		\begin{itemize}
			\item \textbf{on \$locationChangeSuccess}()
			\begin{itemize}
				\item \textbf{Accesso}: Public;
				\item \textbf{Tipo di ritorno}: Void;
				\item \textbf{Descrizione}: evento che permette l'interazione di Angular.js e Impress.js. Dato che il Framework\ped{g} Impress.js cambia l'URL\ped{g} della pagina in modo dinamico, è necessario istruire il routing di Angular su come comportarsi, in modo tale che non ci sia alcun reindirizzamento inopportuno.
			\end{itemize}
			\item \textbf{translateImpress}(json)
			\begin{itemize}
				\item \textbf{Accesso}: Public;
				\item \textbf{Tipo di ritorno}: Void;
				\item \textbf{Descrizione}: metodo che richiama una Funzione\ped{g} JavaScript\ped{g} per la traduzione dell'oggetto json, ricavato tramite getPresentazione() di SharedData, in html eseguibile dal Framework\ped{g} Impress.js.
			\end{itemize}
			\item \textbf{goHome}()
			\begin{itemize}
				\item \textbf{Accesso}: Public;
				\item \textbf{Tipo di ritorno}: Void;
				\item \textbf{Descrizione}: metodo che richiama homepage() di toPages per effettuare il reindirizzamento alla pagina home. Nonostante tale metodo sia già presente in HeaderController, è necessario avere la possibilità di accedere alla home, in quanto l'header viene nascosto per permettere la piena funzionalità ad Impress.js e visualizzare la presentazione a pieno schermo.
			\end{itemize}
			\item \textbf{goEdit}()
			\begin{itemize}
				\item \textbf{Accesso}: Public;
				\item \textbf{Tipo di ritorno}: Void;
				\item \textbf{Descrizione}: metodo che richiama editpage() di toPages per effettuare il reindirizzamento alla pagina di edit.
			\end{itemize}
		\end{itemize}
	}
	\subsubsection{Controller::\-EditController}{
		\textbf{Funzione}\\
		\indent Questa classe si occuperà di mostrare all'utente la possibilità di apportare modifiche ad una presentazione.\\\\
		\textbf{Relazioni d'uso con altri moduli}\\
		\indent Questa classe utilizzerà le seguenti classi:
		\begin{itemize}
			\item View::\-Pages::\-Edit;
			\item View::Pages::[javascript\_functions];
			\item Model::\-SlideShow::\-SlideShowActions::\-Command:
			\begin{itemize}
				\item Invoker;
				\item ConcreteTextInsertCommand;
				\item ConcreteFrameInsertCommand;
				\item ConcreteImageInsertCommand;
				\item ConcreteSVGInsertCommand;
				\item ConcreteAudioInsertCommand;
				\item ConcreteVideoInsertCommand;
				\item ConcreteBackgroundInsertCommand;
				\item ConcreteTextRemoveCommand;
				\item ConcreteFrameRemoveCommand;
				\item ConcreteImageRemoveCommand;
				\item ConcreteSVGRemoveCommand;
				\item ConcreteAudioRemoveCommand;
				\item ConcreteVideoRemoveCommand;
				\item ConcreteEditSizeCommand;
				\item ConcreteEditPositionCommand;
				\item ConcreteEditRotationCommand;
				\item ConcreteEditColorCommand;
				\item ConcreteEditBackgroundCommand;
				\item ConcreteEditFontCommand;
				\item ConcreteEditContentCommand;
				\item ConcretePortaAvantiCommand;
				\item ConcretePortaDietroCommand;
				\item ConcreteAddToMainPathCommand;
				\item ConcreteRemoveFromMainPathCommand
			\end{itemize}
			\item Model::\-serverRelation::\-Loader;
			\item Services::\-Main;
			\item Services::\-toPages;
			\item Services::\-Utils;
			\item Services::\-SharedData;
			\item Services::\-Upload;
			\item \$scope:Object\\
				\indent Oggetto Angular che lega Controller e View. Nello specifico, permette l'esecuzione di espressioni che mantengono aggiornata la View e l'implementazione del 2-way data binding. Possono esistere più scope, strutturati in modo gerarchico, i quali simulano la struttura DOM dell'applicazione;
			\item \$interval:Object\\
				\indent servizio Angular che permette di eseguire determinate operazioni ad ogni intervallo di tempo T;
			\item \$q::\-Object\\
				\indent Servizio Angular che permette di eseguire funzioni\ped{g} in modo asincrono;
			\item \$mdSideNav::\-Object\\
				\indent Servizio Angular Material per il controllo della barra laterale;
			\item \$mdBottomSheet::\-Object\\
				\indent Servizio Angular Material per il controllo dell'oggetto mdBottomsSheet.
		\end{itemize}
		\textbf{Attributi}\\
		\begin{itemize}
			\item \textbf{inv}
			\begin{itemize}
				\item \textbf{Accesso}: Private;
				\item \textbf{Tipo}: Object;
				\item \textbf{Descrizione}: oggetto che mantiene una istanza di Invoker.
			\end{itemize}
			\item \textbf{mongo}
			\begin{itemize}
				\item \textbf{Accesso}: Private;
				\item \textbf{Tipo}: Object;
				\item \textbf{Descrizione}: oggetto che mantiene una istanza di mongoRelation.
			\end{itemize}
			\item \textbf{canRemoveBookmark}
			\begin{itemize}
				\item \textbf{Accesso}: Public;
				\item \textbf{Tipo}: Boolean;
				\item \textbf{Descrizione}: oggetto utilizzato per il data-binding, in base al quale, se settato a true, il frame selezionato non ha un bookmark e per questo è possibile aggiungerlo.
			\end{itemize}
			\item \textbf{canRemoveBookmark}
			\begin{itemize}
				\item \textbf{Accesso}: Public;
				\item \textbf{Tipo}: Boolean;
				\item \textbf{Descrizione}: oggetto utilizzato per il data-binding, in base al quale, se settato a true, il frame selezionato ha un bookmark e per questo è possibile la sua rimozione.
			\end{itemize}
			\item \textbf{allFonts}
			\begin{itemize}
				\item \textbf{Accesso}: Private;
				\item \textbf{Tipo}: Object;
				\item \textbf{Descrizione}: oggetto che contiene tutti caratteri possibili che un elemento testo può avere.
			\end{itemize}
	    \end{itemize}
		\textbf{Metodi}
		\begin{itemize}
			\item \textbf{translateEdit}(json)
			\begin{itemize}
				\item \textbf{Accesso}: Public;
				\item \textbf{Tipo di ritorno}: Void;
				\item \textbf{Descrizione}: metodo che permette la traduzione dell'oggetto json, ricavato tramite getPresentazione() di SharedData, in html permettendo all'utente di modificare la presentazione.
			\end{itemize}
			\item \textbf{goExecute}()
			\begin{itemize}
				\item \textbf{Accesso}: Public;
				\item \textbf{Tipo di ritorno}: Void;
				\item \textbf{Descrizione}: metodo che richiama executionpage() di toPages per effettuare il reindirizzamento alla pagina di execution.
			\end{itemize}
			\item \textbf{toggleList}()
			\begin{itemize}
				\item \textbf{Accesso}: Public;
				\item \textbf{Tipo di ritorno}: Void;
				\item \textbf{Descrizione}: metodo che gestisce \$mdBottomSheet e \$mdSidenav.
			\end{itemize}
			\item \textbf{showPathBottomSheet}(\$event)
			\begin{itemize}
				\item \textbf{Accesso}: Public;
				\item \textbf{Tipo di ritorno}: Void;
				\item \textbf{Descrizione}: metodo che fa apparire \$mdBottomSheet per la visualizzazione dei percorsi\ped{g}.
			\end{itemize}
			\item \textbf{show}(id)
			\begin{itemize}
				\item \textbf{Accesso}: Public;
				\item \textbf{Tipo di ritorno}: Void;
				\item \textbf{Descrizione}: metodo che gestisce la comparsa/scomparsa dei bottoni di gestione della presentazione (inserimento, rimozione, etc.) in base all'id dell'Elemento\ped{g} html cliccato.
			\end{itemize}
			\item \textbf{salvaPresentazione}()
			\begin{itemize}
				\item \textbf{Accesso}: Public;
				\item \textbf{Tipo di ritorno}: Void;
				\item \textbf{Descrizione}: metodo che richiama update() di Loader per salvare la presentazione nel database. Questo metodo viene utilizzato in congiunta ad \$interval in modo da assicurare un salvataggio continuo della presentazione ma senza creare troppo traffico in rete.
			\end{itemize}
			\item \textbf{inserisciFrame}(spec:object)
			\begin{itemize}
				\item \textbf{Accesso}: Public;
				\item \textbf{Tipo di ritorno}: Void;
				\item \textbf{Descrizione}: metodo che inserisce un Frame\ped{g} nel piano della presentazione\ped{g}, attraverso la Funzione\ped{g} JavaScript\ped{g} inserisciFrame(spec) di Edit inserisciFrame(spec), e che richiama, utilizzando il metodo execute di inv, ConcreteFrameInsertCommand() del Command passandogli le specifiche del Frame\ped{g} inserito. Infine, richiama addInsert() di Loader passandogli l'identificativo del Frame\ped{g} inserito. Nel caso in cui il parametro spec sia definito, significa che è stata inviata una richiesta di undo/redo da Command, per cui il metodo si occuperà solamente di aggiornare la view.
			\end{itemize}
			\item \textbf{inserisciTesto}(spec:object)
			\begin{itemize}
				\item \textbf{Accesso}: Public;
				\item \textbf{Tipo di ritorno}: Void;
				\item \textbf{Descrizione}: metodo che inserisce un Elemento\ped{g} testo nel piano della presentazione\ped{g}, attraverso la Funzione\ped{g} JavaScript\ped{g} inserisciTesto(spec) di Edit, e che richiama, utilizzando il metodo execute di inv, ConcreteTextInsertCommand() di Command passandogli le specifiche dell'Elemento\ped{g} testo inserito. Infine, richiama addInsert() di Loader passandogli l'identificativo del testo inserito. Nel caso in cui il parametro spec sia definito, significa che è stata inviata una richiesta di undo/redo da Command, per cui, il metodo si occuperà solamente di aggiornare la view.
			\end{itemize}
			\item \textbf{inserisciImmagini}(files\ped{g}, spec)
			\begin{itemize}
				\item \textbf{Accesso}: Public;
				\item \textbf{Tipo di ritorno}: Void;
				\item \textbf{Descrizione}: metodo che prima richiama isImage(frames) di Upload per controllare che le estensioni\ped{g} siano corrette, successivamente uploadmedia(files\ped{g}, callback) di Upload per il caricamento dei File\ped{g} immagine nel Server\ped{g}. Se l'operazione ha successo, viene invocato callback() il quale inserisce ogni immagine nel piano della presentazione\ped{g}, attraverso la Funzione\ped{g} javascipt inserisciImmagine(percorso\_File\ped{g}, spec) di Edit, e richiama, utilizzando il metodo execute di inv, ConcreteImageInsertCommand() di Command passandogli le specifiche degli elementi\ped{g} immagine inseriti. Infine, richiama addInsert() di Loader passandogli l'identificativo dell'immagine inserita. Nel caso in cui il parametro spec sia definito, significa che è stata inviata una richiesta di undo/redo da Command, per cui, il metodo si occuperà solamente di aggiornare la view.
			\end{itemize}
			\item \textbf{inserisciAudio}(files\ped{g}, spec)
			\begin{itemize}
				\item \textbf{Accesso}: Public;
				\item \textbf{Tipo di ritorno}: Void;
				\item \textbf{Descrizione}: metodo che prima richiama isAudio(frames) di Upload per controllare che le estensioni\ped{g} siano corrette, successivamente uploadmedia(files\ped{g}, callback) di Upload per il caricamento dei File\ped{g} audio nel Server\ped{g}. Se l'operazione ha successo, viene invocato callback() il quale inserisce ogni audio nel piano della presentazione\ped{g}, attraverso la Funzione\ped{g} JavaScript\ped{g} inserisciAudio(percorso\_File\ped{g}, spec) di Edit, e richiama, utilizzando il metodo execute di inv, ConcreteAudioInsertCommand() di Command passandogli le specifiche degli elementi\ped{g} audio inseriti. Infine, richiama addInsert() di Loader passandogli l'identificativo dell'audio inserito. Nel caso in cui il parametro spec sia definito, significa che è stata inviata una richiesta di undo/redo da Command, per cui, il metodo si occuperà solamente di aggiornare la view.
			\end{itemize}
			\item \textbf{inserisciVideo}(files\ped{g}, spec)
			\begin{itemize}
				\item \textbf{Accesso}: Public;
				\item \textbf{Tipo di ritorno}: Void;
				\item \textbf{Descrizione}: metodo che prima richiama isVideo(frames) di Upload per controllare che le estensioni\ped{g} siano corrette, successivamente uploadmedia(files\ped{g}, callback) di Upload per il caricamento dei File\ped{g} video nel Server\ped{g}. Se l'operazione ha successo, viene invocato callback() il quale inserisce ogni video nel piano della presentazione\ped{g}, attraverso la Funzione\ped{g} JavaScript\ped{g} inserisciVideo(percorso\_File\ped{g}, spec) di Edit, e richiama, utilizzando il metodo execute di inv, ConcreteVideoInsertCommand() di Command passandogli le specifiche degli elementi\ped{g} video inseriti. Infine, richiama addInsert() di Loader passandogli l'identificativo del video inserito. Nel caso in cui il parametro spec sia definito, significa che è stata inviata una richiesta di undo/redo da Command, per cui, il metodo si occuperà solamente di aggiornare la view.
			\end{itemize}
			\item \textbf{dragMedia}(files\ped{g}, spec)
			\begin{itemize}
				\item \textbf{Accesso}: Public;
				\item \textbf{Tipo di ritorno}: Void;
				\item \textbf{Descrizione}: metodo apposito per la gestione del drag and drop di File\ped{g} media all'interno della presentazione. Gestisce l'inserimento di immagini, audio e video richiamando una tra le seguenti funzioni\ped{g} JavaScript\ped{g} di Edit:
				\begin{itemize}
					\item inserisciImmagine(percorso\_File\ped{g}, spec);
					\item inserisciAudio(percorso\_File\ped{g}, spec);
					\item inserisciVideo(percorso\_File\ped{g}, spec).
				\end{itemize}
				 e successivamente uno tra i seguenti metodi di Command, dandolo in pasto al metodo execute() di inv:
				\begin{itemize}
					\item ConcreteImageInsertCommand;
					\item ConcreteAudioInsertCommand;
					\item ConcreteVideoInsertCommand.
				\end{itemize}
				Infine, richiama addInsert() di Loader passandogli l'identificativo del video inserito.
			\end{itemize}
			\item \textbf{getMediaSpec}(ele, tipo, URL\ped{g})
			\begin{itemize}
				\item \textbf{Accesso}: Private;
				\item \textbf{Tipo di ritorno}: Object;
				\item \textbf{Descrizione}: metodo che ritorna le specifiche dell'Elemento\ped{g} ele inserito nel piano della presentazione\ped{g}, da passare al Command. Tale metodo viene richiamato dai metodi adibiti all'inserimento di immagini, audio o video.
			\end{itemize}
			\item \textbf{rimuoviElemento}(spec:object)
			\begin{itemize}
				\item \textbf{Accesso}: Public;
				\item \textbf{Tipo di ritorno}: Void;
				\item \textbf{Descrizione}: metodo che rimuove l'Elemento\ped{g} corrente dal piano della presentazione\ped{g} richiamando la Funzione\ped{g} JavaScript\ped{g} elimina(id\_Elemento\ped{g}) di Edit e successivamente, in base al tipo dell'Elemento\ped{g}, utilizzando il metodo execute di inv, richiama uno tra i seguenti:
				\begin{itemize}
					\item ConcreteTextRemoveCommand;
					\item ConcreteFrameRemoveCommand;
					\item ConcreteImageRemoveCommand;
					\item ConcreteAudioRemoveCommand;
					\item ConcreteVideoRemoveCommand.
				\end{itemize}
				 Infine, richiama addDelete() di Loader passandogli l'identificativo dell'Elemento\ped{g} eliminato. Nel caso in cui il parametro spec sia definito, significa che è stata inviata una richiesta di undo/redo da Command, per cui, il metodo si occuperà solamente di aggiornare la view.
			\end{itemize}
			\item \textbf{updateSfondo}(spec:object)
			\begin{itemize}
				\item \textbf{Accesso}: Public;
				\item \textbf{Tipo di ritorno}: Void;
				\item \textbf{Descrizione}: metodo che viene richiamato solo in seguito ad una operazione di undo/redo e che per questo, si occupa solamente di aggiornare il background della presentazione nella view.
			\end{itemize}
			\item \textbf{cambiaColoreSfondo}(color)
			\begin{itemize}
				\item \textbf{Accesso}: Public;
				\item \textbf{Tipo di ritorno}: Void;
				\item \textbf{Descrizione}: metodo che assegna al background della presentazione il valore color, passato come parametro. Successivamente richiama, utilizzando la Funzione\ped{g} execute di inv, ConcreteBackgroundInsertCommand() di Command passandogli le specifiche del background. Infine, richiama addUpdate() di Loader passandogli l'identificativo dell'Elemento\ped{g} background modificato.
			\end{itemize}
			\item \textbf{cambiaImmagineSfondo}(files\ped{g})
			\begin{itemize}
				\item \textbf{Accesso}: Public;
				\item \textbf{Tipo di ritorno}: Void;
				\item \textbf{Descrizione}: metodo che assegna al background della presentazione un nuovo sfondo in base al parametro files\ped{g}. Successivamente richiama, utilizzando la Funzione\ped{g} execute di inv, ConcreteBackgroundInsertCommand() di Command passandogli le specifiche del background. Infine, richiama addUpdate() di Loader passandogli l'identificativo dell'Elemento\ped{g} background modificato.
			\end{itemize}
			\item \textbf{rimuoviSfondo}()
			\begin{itemize}
				\item \textbf{Accesso}: Public;
				\item \textbf{Tipo di ritorno}: Void;
				\item \textbf{Descrizione}: metodo che rimuove colore e sfondo dal background e che successivamente, utilizzando la Funzione\ped{g} execute di inv, richiama ConcreteBackgroundInsertCommand() di Command passandogli le specifiche del background. Infine, richiama addUpdate() di Loader passandogli l'identificativo dell'Elemento\ped{g} background modificato.
			\end{itemize}
			\item \textbf{updateSfondoFrame}(spec:object)
			\begin{itemize}
				\item \textbf{Accesso}: Public;
				\item \textbf{Tipo di ritorno}: Void;
				\item \textbf{Descrizione}: metodo che viene richiamato solo in seguito ad una operazione di undo/redo e che per questo, si occupa solamente di aggiornare il background del Frame\ped{g} spec.id nella view.
			\end{itemize}
			\item \textbf{cambiaColoreSfondoFrame}(color)
			\begin{itemize}
				\item \textbf{Accesso}: Public;
				\item \textbf{Tipo di ritorno}: Void;
				\item \textbf{Descrizione}: metodo che assegna al background del Frame\ped{g} selezionato il valore color, passato come parametro. Successivamente richiama, utilizzando la Funzione\ped{g} execute di inv, ConcreteEditBackgroundCommand() di Command passandogli le specifiche del background del Frame\ped{g}. Infine, richiama addUpdate() di Loader passandogli l'identificativo del Frame\ped{g} modificato.
			\end{itemize}
			\item \textbf{cambiaImmagineSfondoFrame}(files\ped{g})
			\begin{itemize}
				\item \textbf{Accesso}: Public;
				\item \textbf{Tipo di ritorno}: Void;
				\item \textbf{Descrizione}: metodo che assegna al background del Frame\ped{g} selezionato un nuovo sfondo in base al parametro files\ped{g}. Successivamente richiama, utilizzando la Funzione\ped{g} execute di inv, ConcreteEditBackgroundCommand() di Command passandogli le specifiche del background. Infine, richiama addUpdate() di Loader passandogli l'identificativo del Frame\ped{g} modificato.
			\end{itemize}
			\item \textbf{rimuoviSfondoFrame}()
			\begin{itemize}
				\item \textbf{Accesso}: Public;
				\item \textbf{Tipo di ritorno}: Void;
				\item \textbf{Descrizione}: metodo che rimuove colore e sfondo dal background del Frame\ped{g} selezionato e che successivamente, utilizzando la Funzione\ped{g} execute di inv, richiama ConcreteEditBackgroundCommand() di Command passandogli le specifiche del background. Infine, richiama addUpdate() di Loader passandogli l'identificativo del Frame\ped{g} modificato.
			\end{itemize}
			\item \textbf{cambiaColoreTesto}(color)
			\begin{itemize}
				\item \textbf{Accesso}: Public;
				\item \textbf{Tipo di ritorno}: Void;
				\item \textbf{Descrizione}: metodo che cambia il colore dei caratteri dell'Elemento\ped{g} testo selezionato, in base al parametro color, e che successivamente, utilizzando la Funzione\ped{g} execute di inv, richiama ConcreteEditColorCommand() di Command passandogli le specifiche del colore. Infine, richiama addUpdate() di Loader passandogli l'identificativo dell'Elemento\ped{g} testo modificato.
			\end{itemize}
			\item \textbf{cambiaSizeTesto}(value)
			\begin{itemize}
				\item \textbf{Accesso}: Public;
				\item \textbf{Tipo di ritorno}: Void;
				\item \textbf{Descrizione}: metodo che cambia la dimensione dei caratteri dell'Elemento\ped{g} testo selezionato, in base al parametro value, e che successivamente, utilizzando la Funzione\ped{g} execute di inv, richiama ConcreteEditFontCommand() di Command passandogli le specifiche del testo modificato. Infine, richiama addUpdate() di Loader passandogli l'identificativo dell'Elemento\ped{g} testo modificato.
			\end{itemize}
			\item \textbf{cambiaFontTesto}(Font\ped{g})
			\begin{itemize}
				\item \textbf{Accesso}: Public;
				\item \textbf{Tipo di ritorno}: Void;
				\item \textbf{Descrizione}: metodo che cambia la dimensione dei caratteri dell'Elemento\ped{g} testo selezionato, in base al parametro Font\ped{g}, e che successivamente, utilizzando la Funzione\ped{g} execute di inv, richiama ConcreteEditFontCommand() di Command passandogli le specifiche del testo modificato. Infine, richiama addUpdate() di Loader passandogli l'identificativo dell'Elemento\ped{g} testo modificato.
			\end{itemize}
			\item \textbf{aggiornaTesto}(textId, textContent, spec)
			\begin{itemize}
				\item \textbf{Accesso}: Public;
				\item \textbf{Tipo di ritorno}: Void;
				\item \textbf{Descrizione}: metodo che cambia il contenuto dell'Elemento\ped{g} testo textId, in base al parametro textContent, e che successivamente, utilizzando la Funzione\ped{g} execute di inv, richiama ConcreteEditContentCommand() di Command passandogli le specifiche del testo modificato. Infine, richiama addUpdate() di Loader passandogli l'identificativo dell'Elemento\ped{g} testo modificato. Nel caso in cui il parametro spec sia definito, significa che è stata inviata una richiesta di undo/redo da Command, per cui, il metodo si occuperà solamente di aggiornare la view.
			\end{itemize}
			\item \textbf{mediaControl}()
			\begin{itemize}
				\item \textbf{Accesso}: Public;
				\item \textbf{Tipo di ritorno}: Void;
				\item \textbf{Descrizione}: metodo che attiva le funzionalità per la riproduzione di un File\ped{g} media.
			\end{itemize}
			\item \textbf{ruotaElemento}(value, spec)
			\begin{itemize}
				\item \textbf{Accesso}: Public;
				\item \textbf{Tipo di ritorno}: Void;
				\item \textbf{Descrizione}: metodo che ruota l'Elemento\ped{g} selezionato in base al parametro value richiamando la Funzione\ped{g} JavaScript\ped{g} rotate(id\_Elemento\ped{g}, value) di Edit. Successivamente richiama, utilizzando la Funzione\ped{g} execute di inv, ConcreteEditRotationCommand() di Command passandogli le specifiche della nuova rotazione. Infine, richiama addUpdate() di Loader passandogli l'identificativo dell'Elemento\ped{g} modificato. Nel caso in cui il parametro spec sia definito, significa che è stata inviata una richiesta di undo/redo da Command, per cui, il metodo si occuperà solamente di aggiornare la view.
			\end{itemize}
			\item \textbf{muoviElemento}(spec:object)
			\begin{itemize}
				\item \textbf{Accesso}: Public;
				\item \textbf{Tipo di ritorno}: Void;
				\item \textbf{Descrizione}: metodo che, in base al nuovo posizionamento dell'Elemento\ped{g} selezionato all'interno del piano della presentazione\ped{g}, richiama, utilizzando la Funzione\ped{g} execute di inv, ConcreteEditPositionCommand() di Command passandogli le specifiche della nuova posizione. Infine, richiama addUpdate() di Loader passandogli l'identificativo dell'Elemento\ped{g} modificato. Nel caso in cui il parametro spec sia definito, significa che è stata inviata una richiesta di undo/redo da Command, per cui, il metodo si occuperà solamente di aggiornare la view.
			\end{itemize}
			\item \textbf{ridimensionaElemento}(spec:object)
			\begin{itemize}
				\item \textbf{Accesso}: Public;
				\item \textbf{Tipo di ritorno}: Void;
				\item \textbf{Descrizione}: metodo che, in base alla nuova dimensione dell'Elemento\ped{g} selezionato all'interno del piano della presentazione\ped{g}, richiama, utilizzando la Funzione\ped{g} execute di inv, ConcreteEditSizeCommand(spec) di Command passandogli le specifiche della nuova dimensione. Infine, richiama addUpdate() di Loader passandogli l'identificativo dell'Elemento\ped{g} modificato. Nel caso in cui il parametro spec sia definito, significa che è stata inviata una richiesta di undo/redo da Command, per cui, il metodo si occuperà solamente di aggiornare la view.
			\end{itemize}
			\item \textbf{aggiungiMainPath}(spec:object)
			\begin{itemize}
				\item \textbf{Accesso}: Public;
				\item \textbf{Tipo di ritorno}: Void;
				\item \textbf{Descrizione}: metodo che aggiunge il Frame\ped{g} selezionato al Percorso\ped{g} principale salvato nell'oggetto mainPath di Edit, e che richiama, utilizzando la Funzione\ped{g} execute di inv, AddToMainPathCommand() di Command passandogli le specifiche del Frame\ped{g} da aggiungere al Percorso\ped{g} principale. Infine, richiama addPaths() di Loader. Nel caso in cui il parametro spec sia definito, significa che è stata inviata una richiesta di undo/redo da Command, per cui, il metodo si occuperà solamente di aggiornare la view.
			\end{itemize}
			\item \textbf{rimuoviMainPath}(spec:object)
			\begin{itemize}
				\item \textbf{Accesso}: Public;
				\item \textbf{Tipo di ritorno}: Void;
				\item \textbf{Descrizione}: metodo che rimuove il Frame\ped{g} selezionato dal Percorso\ped{g} principale salvato nell'oggetto mainPath di Edit, e che richiama, utilizzando la Funzione\ped{g} execute di inv, RemoveFromMainPathCommand() di Command passandogli le specifiche del Frame\ped{g} da togliere dal Percorso\ped{g} principale. Infine, richiama addPaths() di Loader. Nel caso in cui il parametro spec sia definito, significa che è stata inviata una richiesta di undo/redo da Command, per cui, il metodo si occuperà solamente di aggiornare la view.
			\end{itemize}
			\item \textbf{inPath}(id)
			\begin{itemize}
				\item \textbf{Accesso}: Public;
				\item \textbf{Tipo di ritorno}: Boolean;
				\item \textbf{Descrizione}: metodo che controlla se un frame è all'interno del percorso principale attraverso il metodo getPaths() di InsertEditRemove. Ritorna true se il frame è nel path, altrimenti false.
			\end{itemize}
			\item \textbf{portaAvanti}(spec:object)
			\begin{itemize}
				\item \textbf{Accesso}: Public;
				\item \textbf{Tipo di ritorno}: Void;
				\item \textbf{Descrizione}: metodo che aggiorna il valore zIndex dell'Elemento\ped{g} correntemente selezionato attraverso la Funzione\ped{g} JavaScript\ped{g} portaAvanti(id\_Elemento\ped{g}) di Edit, e che richiama, utilizzando la Funzione\ped{g} execute di inv, concretePortaAvantiCommand() di Command passandogli le specifiche con l'Elemento\ped{g} da aggiornare. Nel caso in cui il parametro spec sia definito, significa che è stata inviata una richiesta di undo/redo da Command, per cui, il metodo si occuperà solamente di aggiornare la view.
			\end{itemize}
			\item \textbf{portaDietro}(spec:object)
			\begin{itemize}
				\item \textbf{Accesso}: Public;
				\item \textbf{Tipo di ritorno}: Void;
				\item \textbf{Descrizione}: metodo che aggiorna il valore zIndex dell'Elemento\ped{g} correntemente selezionato attraverso la Funzione\ped{g} JavaScript\ped{g} mandaDietro(id\_Elemento\ped{g}) di Edit, e che richiama, utilizzando la Funzione\ped{g} execute di inv, concretePortaDietroCommand() di Command passandogli le specifiche con l'Elemento\ped{g} da aggiornare. Nel caso in cui il parametro spec sia definito, significa che è stata inviata una richiesta di undo/redo da Command, per cui, il metodo si occuperà solamente di aggiornare la view.
			\end{itemize}
			\item \textbf{impostaPrimoSfondo}()
			\begin{itemize}
				\item \textbf{Accesso}: Private;
				\item \textbf{Tipo di ritorno}: Void;
				\item \textbf{Descrizione}: metodo che viene richiamato nel caso in cui la presentazione sia vuota, oppure se il valore background non è impostato. Esso richiama ConcreteBackgroundInsertCommand() di Command passandogli le dimensioni del background. Infine, richiama addUpdate() di Loader passandogli l'identificativo dell'Elemento\ped{g} background modificato.
			\end{itemize}
			\item \textbf{updateBookmark}(id)
			\begin{itemize}
				\item \textbf{Accesso}: Private;
				\item \textbf{Tipo di ritorno}: Void;
				\item \textbf{Descrizione}: metodo che aggiorna il campo Bookmark\ped{g} del Frame\ped{g} id. Esso richiama ConcreteEditBookmarkCommand() di Command passandogli le specifiche del Frame\ped{g}. Infine, richiama addUpdate() di Loader passandogli l'identificativo dell'Elemento\ped{g} background modificato.
			\end{itemize}
			\item \textbf{AddBookmark}(spec:object)
			\begin{itemize}
				\item \textbf{Accesso}: Private;
				\item \textbf{Tipo di ritorno}: Void;
				\item \textbf{Descrizione}: metodo che mantiene aggiornata l'icona del pulsante Bookmark\ped{g} nella view, in base al parametro spec.
			\end{itemize}
			\item \textbf{RemoveBookmark}(spec:object)
			\begin{itemize}
				\item \textbf{Accesso}: Private;
				\item \textbf{Tipo di ritorno}: Void;
				\item \textbf{Descrizione}: metodo che mantiene aggiornata l'icona del pulsante Bookmark\ped{g} nella view, in base al parametro spec.
			\end{itemize}
			\item \textbf{annullaModifica}()
			\begin{itemize}
				\item \textbf{Accesso}: Public;
				\item \textbf{Tipo di ritorno}: Void;
				\item \textbf{Descrizione}: metodo che annulla una modifica effettuata richiamando il metodo undo() di inv e che richiama uno tra i metodi addInsert, addUpdate, addDelete o addPaths di Loader in base al tipo di azione effettuata con l'undo.
			\end{itemize}
			\item \textbf{ripristinaModifica}()
			\begin{itemize}
				\item \textbf{Accesso}: Public;
				\item \textbf{Tipo di ritorno}: Void;
				\item \textbf{Descrizione}: metodo che ripristina una modifica annullata richiamando il metodo redo() di inv e che richiama uno tra i metodi addInsert, addUpdate, addDelete o addPaths di Loader in base al tipo di azione effettuata con il redo.
			\end{itemize}
			\item \textbf{getUndoStack}()
			\begin{itemize}
				\item \textbf{Accesso}: Public;
				\item \textbf{Tipo di ritorno}: Object;
				\item \textbf{Descrizione}: metodo che ritorna lo stack dell'undo richiamando il metodo getUndoStack() di inv.
			\end{itemize}
			\item \textbf{getRedoStack}()
			\begin{itemize}
				\item \textbf{Accesso}: Public;
				\item \textbf{Tipo di ritorno}: Object;
				\item \textbf{Descrizione}: metodo che ritorna lo stack del redo richiamando il metodo getRedoStack() di inv.
			\end{itemize}
		\end{itemize}
	}