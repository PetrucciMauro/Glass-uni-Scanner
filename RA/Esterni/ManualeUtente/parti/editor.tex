\section{Editor}
Per poter accedere all'Editor\ped{g} bisognerà essere autenticati e aver creato una presentazione. La schermata di Editor\ped{g} si presenterà in questo modo:
	\begin{figure}[H]
		\centering
		\includegraphics[scale=0.25]{\imgs {editor}.png} %inserire il diagramma UML
		\label{editor}
		\caption{Schermata editor}
	\end{figure}
\subsection{Inserimento Frame}
L'Editor\ped{g} permette di inserire i Frame\ped{g} che compongono la presentazione in un'area apposita; all'interno dei Frame\ped{g} possono essere inseriti anche elementi\ped{g} multimediali. Per procedere alla creazione della slide è sufficiente:
\begin{itemize}
\item premere il pulsante per l'inserimento di un nuovo Elemento\ped{g};

\item posizionarsi con il mouse su Inserisci Frame\ped{g} premere il tasto sinistro del mouse e successivamente trascinare il nuovo Frame\ped{g} nell'apposita area;
	\begin{figure}[H]
		\centering
		\includegraphics[scale=0.25]{\imgs {inserimentoframe}.png} %inserire il diagramma UML
		\label{inserimentoframe}
		\caption{Inserimento frame}
	\end{figure}
\end{itemize}
\subsubsection{Modifica Frame}
Per ogni Frame\ped{g} all'interno della presentazioni son possibili le seguenti azioni:
\begin{itemize}
\item Spostamento: posizionandosi sopra il Frame\ped{g} è possibile spostarlo usando il cursore del mouse;
\item Ridimensionamento: posizionandosi in uno degli angoli del Frame\ped{g} è possibile stringerlo od allargarlo;
\item Cambio dello sfondo.
\end{itemize}
\subsection{Inserimento testo}
L'Editor\ped{g} permette l'inserimento di testi all'interno della presentazione. Per procedere all' inserimento del testo è sufficiente:
\begin{itemize}
\item premere il pulsante per l'inserimento di un nuovo Elemento\ped{g};

\item posizionarsi con il mouse su Inserisci testo premere il tasto sinistro del mouse e successivamente trascinare il riquadro nell'apposita area;

\item fare click sinistro con il mouse all'interno del riquadro per poter inserire del testo.
	\begin{figure}[H]
		\centering
		\includegraphics[scale=0.25]{\imgs {inserimentotesto}.png} %inserire il diagramma UML
		\label{inserimentotesto}
		\caption{Inserimento testo}
	\end{figure}
\end{itemize}
\subsubsection{Modifica testo}
Sono possibili per il testo le seguenti modifiche:
\begin{itemize}
\item Modifica del colore del testo;
\item Modifica della  dimensione del testo;
\item Scelta del Font\ped{g} testo;
\item Rotazione del testo.
\end{itemize}
\subsection{Inserimento Elemento\ped{g} multimediale}
L'Editor\ped{g} permette l'inserimento di elementi\ped{g} multimediali quali, immagini,audio e video. Per procedere all'inserimento di un Elemento\ped{g} multimediale è sufficiente:
\begin{itemize}
\item Premere il pulsante per l'inserimento di un nuovo Elemento\ped{g};
\item Premere il pulsante inserisci immagine/audio/video/;
\item Selezionare l'Elemento\ped{g} multimediale desiderato e caricarlo.
\end{itemize}

	\begin{figure}[H]
		\centering
		\includegraphics[scale=0.25]{\imgs {inserimentomultimediale}.png} %inserire il diagramma UML
		\label{inserimentomultimediale}
		\caption{Inserimento Elemento\ped{g} multimediale}
	\end{figure}
\subsection{Gestione sfondo}
L'Editor\ped{g} permette l'inserimento di un proprio sfondo personale per la presentazione. Per procedere all'inserimento di uno sfondo è sufficiente:
\begin{itemize}
\item premere il pulsante per la gestione dello sfondo;
\item preme il pulsante Scegli immagine di sfondo;
\item selezionare l'immagine che si vuole.
\end{itemize}
	\begin{figure}[H]
		\centering
		\includegraphics[scale=0.25]{\imgs {gestionesfondo}.png} %inserire il diagramma UML
		\label{gestionesfondo}
		\caption{Gestione sfondo}
	\end{figure}
\subsection{Impostare Percorso\ped{g} per la presentazione}
L'Editor\ped{g} permette di impostare un Percorso\ped{g} per la propria presentazione per fare ciò è necessario aver inserito almeno un Frame\ped{g} nell'Editor\ped{g} e successivamente:
\begin{itemize}
\item Selezionare il Frame\ped{g} da inserire nel Percorso\ped{g}, una volta selezionato nella barra degli strumenti appariranno nuovi pulsant,i;
	\begin{figure}[H]
		\centering
		\includegraphics[scale=0.25]{\imgs {framedainserire}.png} %inserire il diagramma UML
		\label{framedainserire}
		\caption{Frame da inserire nel percorso principale}
	\end{figure}
		\begin{figure}[H]
			\centering
			\includegraphics[scale=0.25]{\imgs {frameselezionato}.png} %inserire il diagramma UML
			\label{frameselezionato}
			\caption{Frame da inserire selezionato}
		\end{figure}
\item Premere il pulsante Aggiungi a un Percorso\ped{g};
\item Premere il pulsante Aggiungi Frame\ped{g} al Percorso\ped{g} principale, a questo punto i bordi del Frame\ped{g} saranno diventati rossi ad indicare che si è selezionato il Percorso\ped{g} principale di esecuzione.
	\begin{figure}[H]
		\centering
		\includegraphics[scale=0.25]{\imgs {impostarepercorso}.png} %inserire il diagramma UML
		\label{impostarepercorso}
		\caption{Aggiunta a Percorso\ped{g} principale}
	\end{figure}
\end{itemize}
\subsubsection{Visualizzazione percorso}
Dopo aver inserito più d'un Frame\ped{g} nel Percorso\ped{g} di esecuzione, lo si può visualizzare aprendo il menu laterale, a questo punto bisognerà premere il pulsante \textbf{Percorso\ped{g} principale} per poter visualizzare il Percorso\ped{g} con i Frame\ped{g} ordinati come sono stati inseriti; spostandoli con il mouse si potrà cambiare l'ordine di esecuzione; sempre mediante il mouse si possono eliminare Frame\ped{g} dal Percorso\ped{g}.
	\begin{figure}[H]
		\centering
		\includegraphics[scale=0.25]{\imgs {menupercorso}.png} %inserire il diagramma UML
		\label{menupercorso}
		\caption{Menu Percorso\ped{g} principale}
	\end{figure}