\section{Autenticazione e gestione profilo}
\subsection{Registrazione}
\begin{enumerate}
\item Il primo passo per iniziare ad utilizzare \premi\ è la registrazione. È possibile registrarsi premendo il pulsante \textbf{Registrati} posto nella Homepage.
	\begin{figure}[H]
		\centering
		\includegraphics[scale=0.25]{\imgs {registrazione}.png} %inserire il diagramma UML
		\label{fig:registrazione}
		\caption{Homepage}
	\end{figure}
\item Verrà proposta una form da completare seguendo la procedura:
\begin{itemize}
\item username: inserire lo username con il quale ci si vuole registrare;
\item password: inserire una password, per motivi di sicurezza si chiede sia più lunga di 6 caratteri;
\item registrati: premere infine il pulsante \textbf{Registrati} per confermare i dati e poter iniziare ad utilizzare \premi.
\end{itemize}
\end{enumerate}
	\begin{figure}[H]
		\centering
		\includegraphics[scale=0.75]{\imgs {formregistrazione}.png} %inserire il diagramma UML
		\label{fig:formregistrazione}
		\caption{Form di registrazione}
	\end{figure}

\subsection{Autenticazione}
\begin{enumerate}
\item Per poter accedere in \premi bisogna essere prima di tutto registrati. Per effettuare la Login\ped{g} basta compilare la form nell'Homepage o, in alternativa, premere il pulsante \textbf{login}.
	\begin{figure}[H]
		\centering
		\includegraphics[scale=0.25]{\imgs {login}.png} %inserire il diagramma UML
		\label{fig:login}
		\caption{Pagina Autenticazione}
	\end{figure}
\item Verrà proposto un form da completare seguendo la procedura:
\begin{itemize}
\item username:Inserire lo username con il quale ci si vuole registrare;
\item password: Inserire la password scelta; per motivi di sicurezza questa ha più di 6 caratteri;
\item Login\ped{g}: Premere infine il pulsante Login\ped{g} per confermare i dati e poter iniziare ad utilizzare \premi.
\end{itemize}

\end{enumerate}
	\begin{figure}[H]
		\centering
		\includegraphics[scale=0.75]{\imgs {formlogin}.png} %inserire il diagramma UML
		\label{formlogin}
		\caption{Form di autenticazione}
	\end{figure}
\subsection{Logout}
Per poter effettuare il Logout\ped{g} è necessario essere registrati ed autenticati. Il Logout\ped{g} permette di terminare la propria sessione di lavoro. Un volta effettuato il Logout\ped{g} al successivo accesso alla piattaforma sarà richiesto di autenticarsi. Per effettuare il Logout\ped{g} cliccare sul pulsante \textbf{Logout}.

	\begin{figure}[H]
		\centering
		\includegraphics[scale=0.25]{\imgs {logout}.png} %inserire il diagramma UML
		\label{logout}
		\caption{Logout}
	\end{figure}
\subsection{Cambio password}
\begin{enumerate}
\item Per poter effettuare il cambio password è necessario essere registrati. Per poter eseguire il cambio password bisogna premere sul pulsante \textbf{profilo} posto nella barra di navigazione.

	\begin{figure}[H]
		\centering
		\includegraphics[scale=0.25]{\imgs {profilo}.png} %inserire il diagramma UML
		\label{profilo}
		\caption{Procedura cambio password}
	\end{figure}
\item Cambia Password: Premere il pulsante \textbf{Cambia password}, apparirà la form per effettuare il cambio della password;
\item Password Attuale: Inserire la password attuale associata al proprio Account\ped{g};
\item Nuova Password: Inserire una nuova password, per motivi di sicurezza sono accettabili solo password più lunghe di 6 caratteri;
\item Conferma Nuova Password: Inserire nuovamente la password;
\item Submit: Preme il pulsante submit per confermare i dati e procedere al cambio password.
\end{enumerate}
	\begin{figure}[H]
		\centering
		\includegraphics[scale=0.25]{\imgs {cambiopassword}.png} %inserire il diagramma UML
		\label{cambiopassword}
		\caption{Form cambio password}
	\end{figure}