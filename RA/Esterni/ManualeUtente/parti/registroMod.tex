\Large{\textbf{Registro delle modifiche}}\\
\normalsize

%	Ordine di inserimento: dall'ultima versione alla prima
\renewcommand*{\arraystretch}{1.4}
\begin{longtable} [c]{|>{\centering\arraybackslash}m{2cm} | >{\centering\arraybackslash}m{4cm} | >{\centering\arraybackslash}m{3cm} | >{\centering\arraybackslash}m{6cm} |}
		\caption{Versionamento del documento \label{tab:versionamento}}\\
		 \hline
		 \textbf{Versione} & \textbf{Autore} & \textbf{Data} & \textbf{Descrizione}\\
		 \hline
		 \endfirsthead
		 \hline
		 \textbf{Versione} & \textbf{Autore} & \textbf{Data} & \textbf{Descrizione}\\
		 \hline
		\endhead
		 \hline
		 \endfoot
		 \hline
		 \endlastfoot
		 1.1.0 & \PM & 27-08-2015 & Apportate modifiche a seguito delle segnalazioni del committente\\
		 \hline
		 1.0.0 & \FM & 21-08-2015 & Approvazione del documento\\
		 \hline
	     0.2.0 & \PM & 20-08-2015 & Inserimento immagini descrittive nel documento\\		
		 \hline
		 0.1.0 & \BM & 19-08-2015 & Stesura dello scheletro del documento\\		 
\end{longtable}

\newpage
\Large{\textbf{Storico }}\\
\normalsize \\

%	Per mettere più tabelle di storico basta copiare e incollare la seguente porzione di codice e modificarla in base ai dati nuovi
\noindent \textbf{RQ}
\label{tabVers1}
\begin{table}[h]
	\begin{tabular}{p{0.2\textwidth} p{0.7\textwidth}}
		\toprule \textbf{Versione 1.0.0}	&	\textbf{Nominativo}\\
		\midrule Redazione	& \PM\\
		\midrule Verifica & \TP\\
		\midrule Approvazione	& \FM\\
		\bottomrule
	\end{tabular}
	\caption{Storico ruoli RQ}
\end{table}

\noindent \textbf{RQ->RA}
\label{tabVers2}
\begin{table}[h]
	\begin{tabular}{p{0.2\textwidth} p{0.7\textwidth}}
		\toprule \textbf{Versione 2.0.0}	&	\textbf{Nominativo}\\
		\midrule Redazione	& \PM\\
		\midrule Verifica & \TP\\
		\midrule Approvazione	& \FM\\
		\bottomrule
	\end{tabular}
	\caption{Storico ruoli RA}
\end{table}

