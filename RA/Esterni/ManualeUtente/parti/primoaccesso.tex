\section{Primo accesso}
Per poter accedere a \premi\ è necessario soddisfare i prerequisiti riportati nella sezione \ref{sec:prerequisiti} e successivamente collegarsi al sito \url{www.premi.it}
\subsection{Registrazione}
\begin{enumerate}
\item Il primo passo per iniziare ad utilizzare \premi\ è la registrazione. È possibile registrarsi premendo il pulsante \textbf{Registrati} posto nella Homepage.
	\begin{figure}[H]
		\centering
		\includegraphics[scale=0.25]{\imgs {registrazione}.png} %inserire il diagramma UML
		\label{fig:registrazione}
		\caption{Homepage}
	\end{figure}
\item Verrà proposta una form da completare seguendo la procedura:
\begin{itemize}
\item username: inserire lo username con il quale ci si vuole registrare;
\item password: inserire una password, per motivi di sicurezza si chiede sia più lunga di 6 caratteri;
\item registrati: premere infine il pulsante \textbf{Registrati} per confermare i dati e poter iniziare ad utilizzare \premi.
\end{itemize}
\end{enumerate}
	\begin{figure}[H]
		\centering
		\includegraphics[scale=0.75]{\imgs {formregistrazione}.png} %inserire il diagramma UML
		\label{fig:formregistrazione}
		\caption{Form di registrazione}
	\end{figure}