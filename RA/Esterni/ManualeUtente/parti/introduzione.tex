\section{Introduzione}
\subsection{Scopo del documento}
Questo documento è rivolto all’utente, ha lo scopo di illustrargli le procedure da seguire per svolgere le operazioni utente di \premi . All'utilizzatore non è richiesta alcuna conoscenza informatica poiché dovrà interfacciarsi, tramite web browser\ped{g}, alle funzionalità di \premi che vengono erogate con le stesse modalità di un normale sito internet.
\subsection{Scopo del Prodotto}
Lo scopo del Progetto\ped{g} è la realizzazione un Software\ped{g} per la creazione ed esecuzione di presentazioni multimediali favorendo l’uso di tecniche di storytelling e visualizzazione non lineare dei contenuti.
\subsection{Prerequisiti}
\label{sec:prerequisiti}
Essendo \premi un applicazione web, l'utente non deve disporre di elevati requisiti hardware, deve solo possedere una connessione ad internet ed un web browser ($Chrome \geq 40.x$),
\subsection{Glossario}
Ogni occorrenza di termini tecnici, di dominio e gli acronimi sono marcati con una ''g'' in pedice.Le relative definizioni sono riportate di seguito.
\begin{itemize}
\item \textbf{Account:} L'insieme di funzionalità, strumenti e contenuti attribuiti ad un nome utente all'interno del programma;
\item \textbf{Arrestare:} Interruzione permanente di un'attività;
\item \textbf{Bookmark:} Segnalibro che permette di spostarsi rapidamente da un frame ad un altro al livello più esterno;
\item \textbf{Browser:} Un browser è un programma che consente di usufruire dei servizi di connettività in Rete e di navigare sul World Wide Web;
\item \textbf{Desktop:} Si intende una tipologia di computer contraddistinto dall'essere general purpose, monoutente, destinato ad un utilizzo non in mobilità e principalmente produttivo di dimensioni tali per cui l'installazione in una scrivania risulta la più appropriata per un comodo utilizzo;
\item \textbf{Dispositivo Mobile:} Dispositivo portatile che permette di ricevere, elaborare ed esportare dati senza usare una connessione internet cablata. Di norma si usa per indicare sia smartphone che tablet;
\item \textbf{Editor:} Programma\ped{g} per la modifica di contenuti testuali o multimediali. Un semplice Editor\ped{g} è generalmente incluso in ogni sistema operativo;
\item \textbf{Elemento:} Elemento presente all'interno di un frame il quale può essere un file media, un SVG o un testo;
\item \textbf{Elemento scelta:} Elemento presente all'interno di un frame che, selezionandolo, permette di spostarsi in un nuovo percorso di presentazione;
\item \textbf{File:} Contenitore di informazioni in forma digitale che si basa su un sistema di archiviazione durevole, è cioè a disposizione di altri programmi anche quando il programma che l'ha creato termina la sua esecuzione;
\item \textbf{File media:} Tutti i file con contenuto multimediale: immagine, video e audio;
\item \textbf{File system:} Struttura organizzativa, all'interno di un sistema operativo, che regola il funzionamento dei nomi di file, la loro memorizzazione e il loro recupero;
\item \textbf{Font:} Serie di caratteri distinti per stile;
\item \textbf{Frame:} ciascuna delle aree di schermo che visualizzano parti indipendenti di contenuto durante la visualizzazione di una presentazione;
\item \textbf{Link:} Rinvio ad un'unità informativa su supporto digitale ad un'altra. È ciò che caratterizza la non linearità dell'informazione propria di un ipertesto;
\item \textbf{Linux:} Famiglia di sistemi operativi di tipo Unix-like, rilasciati sotto varie possibili distribuzioni, aventi le caratteristica comune di utilizzare come nucle il kernel Linux;
\item \textbf{Login:} Autenticazione dell'utente nel sistema mediante l'inserimento di un username e password;
\item \textbf{Logout:} Uscita dal sistema da parte dell'utente;
\item \textbf{Mac OS:} Famiglia di sistemi operativi creati per computer Macintosh, prodotti da Apple;
\item \textbf{PC:} Personal Computer;
\item \textbf{Percorso:} Nome che contiene in forma esplicita informazioni sulla posizione di un file all'interno del sistema;
\item \textbf{Piano della presentazione:} Piano nel quale sono viualizzati tutti gli elementi di una presentazione;
\item \textbf{Requisito:} Ciascuna delle qualità necessarie e richieste per uno scopo determinato;
\item \textbf{Server:} Componente o sottosistema informatico di elaborazione che fornisce, a livello logico e a livello fisico, un qualunque tipo di servizio ad altre componenti(tipicamente chaiamte client, cioe cliente) che ne fanno richiesta attraverso una rete di computer, all'interno di un sistema informatico o direttamente in locale su un computer;
\item \textbf{Smartphone:} Uno smartphone è un dispositivo mobile che abbina funzionalità di telefono cellulare a quelle di dati personali;
\item \textbf{Software}: Set di istruzioni interpretabili da una macchina che guidano le componenti di un computer a svolgere specifiche operazioni.
\item \textbf{Sospendere:} Interruzione temporanea di un'attività;
\item \textbf{Tablet:} Dispositivo mobile con dimensioni dello schermo maggiori ai 7", con una potenza di calcolo e numero di porte di comunicazione simili a quelle di uno smartphone;
\item \textbf{URL:} Uniform Resource Locator: sequenza di caratteri che identifica univocamente l'indirizzo di una risorsa in Interne, tipicamente presente su un host server, come ad esempio un documento, un'immagine, un video;


\end{itemize}



