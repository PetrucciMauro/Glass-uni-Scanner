\section{Introduzione}
\subsection{Scopo del documento}
Questo documento è rivolto all’utente, ha lo scopo di illustrargli le procedure da seguire per svolgere le operazioni utente di \premi . All'utilizzatore non è richiesta alcuna conoscenza informatica poiché dovrà interfacciarsi, tramite web browser\ped{g}, alle funzionalità di \premi che vengono erogate con le stesse modalità di un normale sito internet.
\subsection{Scopo del Prodotto}
Lo scopo del Progetto\ped{g} è la realizzazione un Software\ped{g} per la creazione ed esecuzione di presentazioni multimediali favorendo l’uso di tecniche di storytelling e visualizzazione non lineare dei contenuti.
\subsection{Prerequisiti}
\label{sec:prerequisiti}
L'utente deve possedere una connessione ad internet ed un web browser ($Chrome \geq 40.x, Firefox \geq 40.x$).
\subsection{Glossario}
Ogni occorrenza di termini tecnici, di dominio e gli acronimi sono marcati con una ''g'' in pedice.Le relative definizioni sono riportate di seguito.
\begin{itemize}
\item \textbf{Editor}: Programma\ped{g} per la modifica di contenuti testuali o multimediali. Un semplice Editor\ped{g} è generalmente incluso in ogni sistema operativo;
\item \textbf{Frame}: ciascuna delle aree di schermo che visualizzano parti indipendenti di contenuto durante la visualizzazione di una presentazione;
\item \textbf{Software}: set di istruzioni interpretabili da una macchina che guidano le componenti di un computer a svolgere specifiche operazioni.
\end{itemize}



