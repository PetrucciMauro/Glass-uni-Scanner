\documentclass[a4paper,12pt]{article}
\usepackage{../../Template/format}

%	VARIABILI
%	Percorsi file
\newcommand{\parti}{./parti/}
\newcommand{\imgs}{./images/} %se esistono
\newcommand{\temp}{../../Template/}

%	Informazioni documento
\newcommand{\titoloDoc}{Piano di Progetto}
\newcommand{\dataCreazione}{2 marzo 2015}
\newcommand{\versione}{2.0.0} % versione corrente del documento
\newcommand{\dataUM}{25 maggio 2015} % data dell'ultima modifica
\newcommand{\stato}{Formale}
\newcommand{\uso}{Esterno}
\newcommand{\appr}{\VG}
\newcommand{\redaz}{\PM, \BM}
\newcommand{\verif}{\TP}
\newcommand{\sommario}{
	Il presente documento riporta il Piano di Progetto per il capitolato \premi.
	}
\title{\titoloDoc}

%	HEADER
\rhead{\titoloDoc \ v.\versione}

%	CORPO
\begin{document}
	\input{\temp firstpage} % prima pagina
	\newpage
	\input{\temp sommario} % breve sommario
	\newpage
	\input{\parti registroMod} % registro delle modifiche
	\newpage
	\tableofcontents % crea l'indice
	\newpage
	\listoffigures
	\listoftables
	\newpage
	\input{\parti introduzione}
	\newpage
	\input{\parti analisiRischi}
	\newpage
	\input{\parti cicloVita}
	\newpage
	\input{\parti scadenze}
	\newpage
	\input{\parti pianificazione}
	\newpage
	\input{\parti sudCompiti}
	\newpage
	\input{\parti prospettoEconomico}
	\begin{appendices}
		\newpage
		\input{\parti organigramma}
	\end{appendices}
\end{document}