\section{Consuntivo e preventivo a finire}
Questa sezione contiene il prospetto economico che riporta le spese effettivamente sostenute. Vengono riportate le ore impiegate per svolgere i compiti preventivati, sia per
ruolo che per persona. In base alle differenza di ore tra il preventivo e il consuntivo, detta conguaglio, avremmo un bilancio:
\begin{itemize}
\item \textbf{Positivo:} se il preventivo ha superato il consuntivo;
\item \textbf{Negativo:} se il consuntivo ha superato il preventivo;
\item \textbf{In pari:} se consuntivo e preventivo coincidono.
\end{itemize}

\subsection{Analisi}

Si riporta di seguito il consuntivo della fase di \textbf{Analisi.}\\
La tabella sottostante riporta le ore effettivamente impiegate e tra parentesi la differenza di ore tra preventivo e consuntivo, divise per ruolo. Come si pu� notare dal valore riportato nella riga \textbf{Differenza dei totali}, intesa come differenza tra preventivo e consuntivo, il segno negativo indica che � stata impiegata un'ora in pi� per svolgere le attivit� programmate con un bilancio in passivo di 10\euro.

	\begin{table}[H]
		\centering
	  \begin{tabular}{p{\dimexpr 0.3\linewidth-2\tabcolsep}p{\dimexpr 0.2\linewidth-2\tabcolsep}
		    							p{\dimexpr 0.2\linewidth-2\tabcolsep}}
		   \toprule Ruolo & Ore & Costi \\
		   \midrule
		   Responsabile & 55(-2) & 1650(-60) \\
		   Amministratore & 16(-1) & 320(-20) \\
		   Analista & 76(+3) & 1900(+75) \\
		   Programmatore & 0 & 0 \\
		   Progettista & 0 & 0 \\
		   Verificatore & 32(+1) & 480(+15) \\
		   \hline
		   Totale consuntivo & 180 & 4360 \\
		   Totale preventivo & 179 & 4350 \\
		   Differenza dei totali & -1 & -10 \\
		   \bottomrule
	 \end{tabular}
	 	\label{tab:costuntivoRequisiti}
	 	\caption{Differenza preventivo-consuntivo per ruolo, fase di Analisi}
	\end{table}
	 	
Nella tabella seguente sono riportate le differenze tra le ore di lavoro previste per ogni componente con quelle realmente impiegate.


	\begin{table}[H]
		\centering
	  \begin{tabular}{l*{7}{c}r}
		   \toprule 
		   Nominativo & Re & Am & An & Pt & Ve & Pr & Ore totali \\
		   \midrule
		   \BM  & 10(0) & 1(0) & 8(-1) &  & 7(-1) &  & 26(-2)   \\
		   \FM  & 3(+1) &  & 16(+2) &  &  13(+1) &  &  32(+4)  \\
		   \GP  & 5(-2) &  & 15(+2) &  &  2(+1) &  &  22(+1)  \\
		   \PM  & 7(0)  & 11(-1) & 8(-1) &  &  1(+2)&  &  23(0)  \\ 
		   \TP  & 18(-1)& 2 & 15(+2) &  &  7(+2) & &  37(+1)  \\ 
           \VG  & 12(0) & 2(0) & 14(-1) &  &  7(-2) &  &  35(-3) \\ 
		   \bottomrule
	 \end{tabular}
	 	\label{tab:consuntivoComponentiAnalisi}
	 	\caption{Differenza preventivo-consuntivo per ruolo, fase di Analisi}
	\end{table}
	
\subsubsection{Conclusioni}
L'attuazione delle attivit� pianificate e riportate nel Gantt di \ref{fig:pianorequisiti} si � discostata leggermente da quanto pianificato nella \textbf{Progettazione}.
Il gruppo ha impiegato, in totale , un'ora in pi� per completare la fase di \textbf{Analisi} provocando cos� un deficit nel bilancio di 10\euro.\\
Tale passivo non andr� ad influenzare il costo totale del progetto in quanto le ore impiegate in questa fase non vengono poste a carico del Proponente.

\subsection{Progettazione}

Si riporta di seguito il consuntivo della fase di \textbf{Progettazione} che andr� ad incidere sulle fasi successive.\\
La tabella sottostante riporta le ore effettivamente impiegate e tra parentesi la differenza di ore tra preventivo e consuntivo, divise per ruolo. Come si pu� notare dal valore riportato nella riga \textbf{Differenza dei totali}, intesa come differenza tra preventivo e consuntivo, il segno negativo indica che � stata impiegata un'ora in pi� per svolgere le attivit� programmate con un bilancio in passivo di 11\euro.

	\begin{table}[H]
		\centering
	  \begin{tabular}{p{\dimexpr 0.3\linewidth-2\tabcolsep}p{\dimexpr 0.2\linewidth-2\tabcolsep}
		    							p{\dimexpr 0.2\linewidth-2\tabcolsep}}
		   \toprule Ruolo & Ore & Costi \\
		   \midrule
		   Responsabile & 13(-1) & 390(-30) \\
		   Amministratore & 10(0) & 200(0) \\
		   Analista & 6(-1) & 150(-25) \\
		   Programmatore & 0 & 0 \\
		   Progettista & 82(+3) & 1804(+66) \\
		   Verificatore & 29(0) & 435(0) \\
		   \hline
		   Totale consuntivo & 142 & 3020 \\
		   Totale preventivo & 141 & 3009 \\
		   Differenza dei totali & -1 & -11 \\
		   \bottomrule
	 \end{tabular}
	 	\label{tab:costuntivoProgettazione}
	 	\caption{Differenza preventivo-consuntivo per ruolo, fase di Progettazione}
	\end{table}
	 	
Nella tabella seguente sono riportate le differenze tra le ore di lavoro previste per ogni componente con quelle realmente impiegate.


	\begin{table}[H]
		\centering
	  \begin{tabular}{l*{7}{c}r}
		   \toprule 
		   Nominativo & Re & Am & An & Pt & Ve & Pr & Ore totali \\
		   \midrule
		   \BM  &  & 10(0) &  &  & 6(+3)&  & 16(+3)   \\
		   \FM  &  &  & 6(-1) & 22(+1)  &  13(+1) &  &  28(0)  \\
		   \GP  &  &  &  & 8(0)  &  14(-4) &  &  22(-4)  \\
		   \PM  & 9(0)  &  & & 12(+2) &  7(+1)&  &  28(+3)  \\ 
		   \TP  & &  &  & 24(-4)  &  2(0) & &  26(-4)  \\ 
           \VG  & 5(-1) &  &  &  &  16(-4) &  &  21(+3) \\ 
		   \bottomrule
	 \end{tabular}
	 	\label{tab:consuntivoComponentiProgettazione}
	 	\caption{Differenza preventivo-consuntivo per ruolo, fase di Progettazione}
	\end{table}
	
\subsubsection{Conclusioni}
L'attuazione delle attivit� pianificate e riportate nel Gantt di \ref{fig:pianoprogettazione} si � discostata leggermente da quanto pianificato nella \textbf{Progettazione}.
Il gruppo ha impiegato, in totale , un'ora in pi� per completare la fase di \textbf{Progettazione} provocando cos� un deficit nel bilancio di 11\euro.\\
Tale passivo andr� ad influenzare la fase di codifica che comporter� quindi una ridistribuzione delle ore per rimanere dentro il bilancio previsto.




