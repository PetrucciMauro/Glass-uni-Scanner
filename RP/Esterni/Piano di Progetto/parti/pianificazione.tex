\section{Pianificazione}{
	Date le scadenze in (\S \ref{sec:scadenze}) si è diviso il progetto\ped{g} in 4 stati di sviluppo:
	\begin{itemize}
		\item \textbf{Analisi dei Requisiti} (AN);
		\item \textbf{Progettazione Architetturale} (PA);
		\item \textbf{Progettazione in dettaglio e Codifica} (PDC);
		\item \textbf{Verifica e Validazione} (VV).
	\end{itemize}
	
	Per ogni stato del progetto\ped{g} sono state individuate attività e sotto-attività a cui sono state associate le risorse\ped{g} del gruppo.\\
	Per ogni stato del processo\ped{g} è stato riportato il Gantt con evidenziato in colore rosso il cammino critico. Le attività parte di questo cammino saranno monitorate con maggiore attenzione in quanto un ritardo su queste attività	sarebbe dannoso per l'efficienza del gruppo e porterebbe a ritardi nell'avanzamento dello stato del progetto\ped{g}.\\
	Si è deciso di non riportare i diagrammi PERT poiché le attività critiche sono già state evidenziate nel GANT come detto in precedenza.
	Per ogni stato del progetto\ped{g} è stato riportato il diagramma WBS\ped{g} così da rendere esplicita la composizione delle attività e rendere immediata la costruzione del prospetto economico.
	
\newpage
\subsection{Analisi dei Requisiti}{
	\textbf{Periodo}: da 3-03-2015 a 3-04-2015. \\
	
	I documenti redatti in questo periodo sono:
	\begin{itemize}
		\item \textbf{Norme di Progetto}: il Responsabile e l'Amministratore scrivono il documento "Norme di Progetto" che norma le attività da svolgersi nel corso del progetto\ped{g};
		\item \textbf{Studio di Fattibilità}: viene valutata la fattibilità dei capitolati proposti;
		\item \textbf{Analisi dei Requisiti}: a partire dallo studio di fattibilità vengono analizzati in profondità i requisiti\ped{g} per il capitolato scelto;
		\item \textbf{Piano di progetto}: il responsabile di progetto\ped{g} durante questa fase si prende carico di definire un piano di progetto\ped{g} in cui vengono definite le macro-attività da svolgersi durante questo e i successivi stati di sviluppo del sistema. Successivamente alle attività saranno associate risorse\ped{g} così da poter redigere un prospetto economico per il proponente\ped{g};
		\item \textbf{Piano di Qualifica}: Analista, Verificatore e Responsabile redigono il piano di qualifica;
		\item \textbf{Glossario}: scritto dai redattori dei documenti in modo incrementale;
		\item \textbf{Lettera di Presentazione}: Lettera di Presentazione del gruppo da consegnare al committente\ped{g} per partecipare alla gara d'appalto.
	\end{itemize}

	\begin{landscape}
		\thispagestyle{empty}
		\begin{figure}[H]
			\parbox[c][\textwidth][s]{\linewidth}{
			\centering
			\vspace*{\fill}
			\fbox{\includegraphics[width=0.90\hsize]{\imgs pianoRequisiti.jpg}}
			\vspace*{\fill}
			\caption{Piano dei Requisiti}
			\label{fig:pianorequisiti}
			}
		\end{figure}
	\end{landscape}

	\renewcommand*{\arraystretch}{1.4}
	\begin{longtable} [c]{| l | l | l | l |}
		\caption{Attività e ruoli Piano dei Requisiti\ped{g} \label{tab:pianorequisiti}}\\
		 \hline
		 \textbf{Macro-Attività} & \textbf{Attività} & \textbf{Ruolo} & \textbf{Ore}\\
		 \hline
		 \endfirsthead
		 \hline
		 \textbf{Macro-Attività} & \textbf{Attività} & \textbf{Ruolo} & \textbf{Ore}\\
		 \hline
		\endhead
		 \hline
		 \endfoot
		 \hline
		 \endlastfoot
		Norme di Progetto\ped{g} & Comunicazioni & Responsabile 1 & 1\\
		Norme di Progetto\ped{g} & Requisiti\ped{g} & Responsabile 2 & 4\\
		Norme di Progetto\ped{g} & Progettazione & Responsabile 3 & 3\\
		Norme di Progetto\ped{g} & Codifica\ped{g} & Responsabile 4 & 3\\
		Norme di Progetto\ped{g} & Documentazione & Responsabile 3 & 5\\
		Norme di Progetto\ped{g} & Strumenti	&	Responsabile 5	&	4\\
		&	&	Amministratore1	&	12\\
		&	&	Amministratore 2	&	2\\
		Norme di Progetto\ped{g} & Norme ticketing\ped{g} &	Responsabile 1	&	1\\
		&	&	Amministratore 3 & 1\\
		Norme di Progetto\ped{g} & Repository\ped{g} &	Responsabile 4	&	1\\
		&	&	Amministratore 1	&	1\\
		Norme di Progetto\ped{g} & Verifica & Verificatore 3 & 2\\
		&	&	Verificatore 3 & 2\\
		&	&	Responsabile 3 & 1\\
		Analisi dei Requisiti\ped{g} & Descrizione & Analista1 & 6\\
		Analisi dei Requisiti\ped{g} & casi d'uso\ped{g} & Analista 1 & 9\\
		&	&	Analista 2 & 8\\
		&	&	Analista 3 & 9\\
		&	&	Analista 4 & 8\\
		&	&	Analista 5 & 8\\
		&	&	Analista 6 & 8\\
		Analisi dei Requisiti\ped{g} & Specifica & Analista 1 & 7\\
		&	&	Analista 3 & 7\\
		&	&	Analista 4 & 6\\
		Analisi dei Requisiti\ped{g} & Verifica & Verificatore 1 & 2\\
		&	&	Verificatore 2 & 3\\
		&	&	Responsabile 3 & 1\\
		Piano di Progetto\ped{g} & Calendario & Responsabile 2 & 2\\
		Piano di Progetto\ped{g} & Organigramma & Responsabile 6 & 1\\
		Piano di Progetto\ped{g} & Ciclo di vita\ped{g} & Responsabile 2 & 2\\
		Piano di Progetto\ped{g} & Analisi dei rischi & Responsabile 2 & 2\\
		Piano di Progetto\ped{g} & Individuazione attività & Responsabile 2 & 6\\
		Piano di Progetto\ped{g} & Assegnamento risorse\ped{g} & Responsabile 5 & 4\\
		&	&	Responsabile 2 & 2\\
		Piano di Progetto\ped{g} & Prospetto economico & Responsabile5 & 2\\
		Piano di Progetto\ped{g} & Verifica &  Verificatore 5 & 3\\
		&	&	Responsabile 1 & 1\\
		Piano di Qualifica & Visione generale & Verificatore 3 & 5\\
		&	&	Responsabile 4 & 3\\
		&	&	Responsabile 6 & 3\\
		Piano di Qualifica & Gestione amministrativa della revisione & & 2\\
		Piano di Qualifica & Resoconto verifiche & Verificatore 2 & 2\\
		Piano di Qualifica & Pianificazione validazione\ped{g} requisiti\ped{g} & Verificatore 5 & 10\\
		Piano di Qualifica & Verifica & Verificatore 4 & 2\\
		&	&	Responsabile 6 & 1\\
		Glossario & Stesura & / & 10\\
		Glossario & Verifica & Verificatore & 1\\
	\end{longtable}
}

\newpage
\subsection{Progettazione}{
	\textbf{Periodo}: dal 7-04-2015 al 30-04-2015. \\
	
	Le attività da svolgere in questo periodo saranno:
	\begin{itemize}
		\item Incremento e verifica dei documenti portati in Revisione dei Requisiti\ped{g};
		\item Stesura del documento \textbf{Specifica Tecnica} in cui il Progettista esporrà le scelte progettuali di alto livello del sistema e i design pattern che saranno utilizzati nel sistema. Si verificherà il tracciamento dal sistema ai requisiti\ped{g} e dai requisiti\ped{g} al sistema.
	\end{itemize}
	
	\begin{landscape}
		\thispagestyle{empty}
		\begin{figure}[H]
			\parbox[c][\textwidth][s]{\linewidth}{
			\centering
			\vspace*{\fill}
			\fbox{\includegraphics[width=\hsize]{\imgs pianoProgettazione.jpg}}
			\vspace*{\fill}
			\label{fig:pianoprogettazione}
			\caption{Piano di Progettazione}}
		\end{figure}
	\end{landscape}

	\renewcommand*{\arraystretch}{1.4}
	\begin{longtable} [c]{| l | l | l | l |}
		\caption{Attività e ruoli Piano di Progettazione \label{tab:pianoprogettazione}}\\
		 \hline
		 \textbf{Macro-Attività} & \textbf{Attività} & \textbf{Ruolo} & \textbf{Ore}\\
		 \hline
		 \endfirsthead
		 \hline
		 \textbf{Macro-Attività} & \textbf{Attività} & \textbf{Ruolo} & \textbf{Ore}\\
		 \hline
			\endhead
		 \hline
		 \endfoot
		 \hline
		 \endlastfoot
		 Norme di Progetto\ped{g} & Incremento & Responsabile 2 & 6\\
		 &	&	Amministratore & 10\\
		 Norme di Progetto\ped{g} & Verifica & Verificatore 2 & 2 \\
		 Analisi dei Requisiti\ped{g} & Incremento & Analista & 6 \\
		 Analisi dei Requisiti\ped{g} & Verifica & Verificatore 1 & 1 \\
		 Specifica Tecnica & Strumenti e tecnologie & Progettista 5 & 6 \\
		 Specifica Tecnica & Notazione & Progettista 2 & 2 \\
		 Specifica Tecnica & Architettura & Progettista 2 & 6\\
		 &	&	Progettista 3 & 6\\
		 &	&	Progettista 4 & 8\\
		 Specifica Tecnica & Design Patterns & Progettista 1 & 8\\
		 &	&	Progettista 2 & 8\\
		 &	&	Progettista 3 & 8\\
		 Specifica Tecnica & Componenti & Progettista 2 & 8\\
		 &	&	Progettista 3 & 8\\
		 &	&	Progettista 4 & 8\\
		 Specifica Tecnica & Tracciamento & Verificatore 1 & 5\\
		 &	&	Verificatore 4 & 5\\
		Specifica Tecnica & Verifica & Verificatore 3 & 5\\
		 &	&	Responsabile 2 & 1\\
		Piano di Qualifica & Incremento & Responsabile 1 & 4 \\
		Piano di Qualifica & Pianificazione Test & Progettista 5 & 6\\
		&	&	Verificatore 1 & 5\\
		&	&	Verificatore 2 & 6\\
		&	&	Responsabile 1 & 4\\
		&	&	Amministratore & 6\\
		Piano di Qualifica & Esito verifiche & Verificatore 3 & 4 \\
		Piano di Qualifica & Verifica & Verificatore 3 & 5\\
		&	&	Responsabile 1 & 1\\
		Glossario & Incremento & / & 6 \\
		Glossario & Verifica & Verificatore 4 & 2 \\
		Piano di Progetto\ped{g} & Consuntivo & Responsabile 2 & 2\\
	\end{longtable}

	\begin{landscape}
		\thispagestyle{empty}
		\begin{figure}[H]
			\parbox[c][\textwidth][s]{\linewidth}{
			\centering
			\vspace*{\fill}
			\fbox{\includegraphics[width=\hsize,height=0.35\vsize]{\imgs wbsProgettazione.jpg}}
			\vspace*{\fill}
			\label{fig:wbsProgettazione}
			\caption{Work breakdown structure della fase di Progettazione}}
		\end{figure}
	\end{landscape}
}

\newpage
\subsection{Progettazione in dettaglio e Codifica}{
	\textbf{Periodo}: dal 4-05-2015 a 13-06-2015. \\
	 
	 Le attività della parte \textbf{Progettazione in dettaglio e Codifica} sono:
	 \begin{itemize}
		 \item \textbf{Definizione di Prodotto}: che contiene la descrizione approfondita delle componenti del prodotto;
		 \item \textbf{Codifica}: sviluppo del codice\ped{g} del prodotto da parte dei programmatori che devono seguire quanto riportato nel documento Definizione di Prodotto;
		 \item \textbf{Esecuzione test automatici}: esecuzione automatica dei test di unità e integrazione e rapporto sul risultato;
		 \item \textbf{Incremento Specifica Tecnica}: incremento del documento di specifica tecnica con la progettazione riguardante i requisiti\ped{g} non obbligatori;
		 \item \textbf{Manualistica}: verranno reddatti i documenti sul prodotto per l'utente finale.
	 \end{itemize}

	\begin{landscape}
		\thispagestyle{empty}
		\begin{figure}[H]
			\parbox[c][\textwidth][s]{\linewidth}{
			\centering
			\vspace*{\fill}
			\fbox{\includegraphics[width=\hsize]{\imgs pianoCodifica.jpg}}
			\vspace*{\fill}
			\label{fig:pianoprogettazdettcodifica}
			\caption{Piano di Progettazione in dettaglio e Codifica}}
		\end{figure}
	\end{landscape}
	
	\renewcommand*{\arraystretch}{1.4}
	\begin{longtable} [c]{| l | l | l | l |}
		\caption{Attività e ruoli Piano di Progettazione in dettaglio e Codifica\ped{g} \label{tab:pianoprogettazdettcodifica}}\\
		 \hline
		 \textbf{Macro-Attività} & \textbf{Attività} & \textbf{Ruolo} & \textbf{Ore}\\
		 \hline
		 \endfirsthead
		 \hline
		 \textbf{Macro-Attività} & \textbf{Attività} & \textbf{Ruolo} & \textbf{Ore}\\
		 \hline
				\endhead
		 \hline
		 \endfoot
		 \hline
		 \endlastfoot
		 Norme di Progetto\ped{g} & Incremento & Responsabile & 8\\
		 &	&	Amministratore & 2\\
		 Norme di Progetto\ped{g} & Verifica & Verificatore 4 & 1 \\
		 Piano di Progetto\ped{g} & Rivalutazione & Responsabile & 2 \\
		 Piano di Progetto\ped{g} & Verifica & Verificatore 1 & 1 \\
		 Specifica Tecnica & Incremento Design Patterns & Progettista 1 & 4\\
		 &	&	Progettista 2 & 4\\
		 Specifica Tecnica & Incremento Componenti & Progettista 1 & 8\\
		 &	&	Progettista 2 & 8\\
		 Specifica Tecnica & Verifica & Verificatore 4 & 4 \\
		 Definizione Prodotto & Specifica Componenti Base & Progettista 3 & 10\\
		 &	&	Progettista 4 & 10\\
		 &	&	Progettista 5 & 10\\
		 &	&	Progettista 6 & 10\\
		 Definizione Prodotto & Specifica - Verifica Base & Verificatore 1 & 6\\
		 &	&	Verificatore 2 & 2\\
		 Definizione Prodotto & Specifica Componenti Incremento & Progettista 1 & 10\\
		 &	&	Progettista 2 & 10\\
		 Definizione Prodotto & Specifica - Verifica Incremento & Verificatore 2 & 4 \\
		 Definizione Prodotto & Specifica - Tracciamento Base & Progettista 3 & 4\\
		 &	&	Verificatore 1 & 6\\
		 Definizione Prodotto & Specifica - Tracciamento Incremento & Progettista 1 & 2\\
		 &	&	Verificatore 3 & 4\\
		 Definizione Prodotto & Verifica & Verificatore 2 & 4 \\
		 Codifica\ped{g} & Base & Programmatore 1 & 20\\
		 &	&	Programmatore 2 & 19\\
		 &	&	Programmatore 3 & 17\\
		 &	&	Programmatore 4 & 21\\
		 &	&	Programmatore 5 & 24\\
		 &	&	Programmatore 6 & 20\\
		 Codifica\ped{g} & Incremento & Programmatore 1 & 6\\
		 &	&	Programmatore 2 & 7\\
		 &	&	Programmatore 3 & 10\\
		 &	&	Programmatore 4 & 8\\
		 Codifica\ped{g} & Preparazione test unità base & Verificatore 1 & 15\\
		 &	&	Verificatore 4 & 15\\
		 Codifica\ped{g} & Verifica Base & Verificatore 2 & 13\\
		 &	&	Verificatore 3 & 11\\
		 Codifica\ped{g} & Preparazione test unità incremento & Verificatore 5 & 10\\
		 &	&	Verificatore 6 & 5\\
		 Codifica\ped{g} & Verifica Incremento & Verificatore 5 & 5\\
		 &	&	Verificatore 6 & 5\\
		 Piano di Progetto\ped{g} & Consuntivo & Responsabile & 2 \\
		 Piano di Qualifica & Resoconto attività verifica & Verificatore 4 & 4\\
		 &	&	Verificatore 3 & 4\\
		 Piano di Qualifica & Verifica & Verificatore 4 & 2 \\
	\end{longtable}

	\begin{landscape}
		\thispagestyle{empty}
		\begin{figure}[H]
			\parbox[c][\textwidth][s]{\linewidth}{
			\centering
			\vspace*{\fill}
			\fbox{\includegraphics[width=\hsize,height=0.35\vsize]{\imgs wbsCodifica.jpg}}
			\vspace*{\fill}
			\label{fig:wbsCodifica}
			\caption{Work breakdown structure della fase di Progettazione in dettaglio e Codifica}}
		\end{figure}
	\end{landscape}
}

\newpage
\subsection{Verifica e Validazione}{
	\textbf{Periodo}: dal 19-06-2015 al 01-07-2015.
	
	Le attività svolte in questo periodo saranno:
	\begin{itemize}
		\item \textbf{Esecuzione test}: non eseguiti durante il periodo di Progettazione e Codifica\ped{g};
		\item \textbf{Validazione}: del sistema rispetto ai metodi previsti in Piano di Qualifica per ogni requisito\ped{g} del sistema;
		\item \textbf{Incremento manualistica}: destinata all'utente finale e documentazione per il rilascio.
	\end{itemize}

	\begin{landscape}
		\thispagestyle{empty}	
		\begin{figure}[H]
			\parbox[c][\textwidth][s]{\linewidth}{
			\centering
			\vspace*{\fill}
			\fbox{\includegraphics[width=\hsize]{\imgs pianoAccettazione.jpg}}
			\vspace*{\fill}
			\label{fig:pianoaccettazione}
			\caption{Piano di Accettazione}}
		\end{figure}
	\end{landscape}
	
	\renewcommand*{\arraystretch}{1.4}
	\begin{longtable} [c]{| l | l | l | l |}
		\caption{Attività e ruoli Piano di Accettazione \label{tab:pianoaccettazione}}\\
		 \hline
		 \textbf{Macro-Attività} & \textbf{Attività} & \textbf{Ruolo} & \textbf{Ore}\\
		 \hline
		 \endfirsthead
		 \hline
		 \textbf{Macro-Attività} & \textbf{Attività} & \textbf{Ruolo} & \textbf{Ore}\\
		 \hline
			\endhead
		 \hline
		 \endfoot
		 \hline
		 \endlastfoot
		Norme & Incremento & Responsabile 1 & 2 \\
		Norme & Verifica & Verificatore 1 & 1 \\
		Glossario & Incremento & / & 4 \\
		Glossario & Verifica & Verificatore 2 & 1 \\
		Piano di Qualifica & Incremento Verifiche & Verificatore 2 & 4 \\
		Piano di Qualifica & Validazione\ped{g} Interna & Programmatore 1 & 9 \\
		&	&	Verificatore 1	&	4\\
		&	&	Progettista 1	&	8\\
		&	&	Responsabile 2	&	6\\
		&	&	Amministratore	&	4\\
		Piano di Qualifica & Verifica & Verificatore 3 & 2 \\
		Definizione Prodotto & Incremento & Progettista 2 & 12 \\
		Definizione Prodotto & Verifica & Verificatore 3 & 2\\
		&	&	Verificatore 6 & 1\\
		Specifica Tecnica & Incremento & Progettista3 & 8 \\
		Specifica Tecnica & Verifica & Verificatore 4 & 5\\
		&	&	Verificatore 5 & 1\\
		Manualistica & Incremento & Progettista 4 & 6\\
		&	&	Progettista 5 & 4\\
		&	&	Amministratore & 3\\
		&	&	Responsabile 1 & 2\\
		Manualistica & Verifica & Verificatore 5  & 2 \\
		&	&	Verificatore 4 & 2 \\
		Piano di Progetto\ped{g} & Consuntivo & Responsabile 1 & 2 \\
	\end{longtable}

	\begin{landscape}
		\thispagestyle{empty}
		\begin{figure}[H]
			\parbox[c][\textwidth][s]{\linewidth}{
			\centering
			\vspace*{\fill}
			\fbox{\includegraphics[width=\hsize,height=0.35\vsize]{\imgs wbsAccettazione.jpg}}
			\vspace*{\fill}
			\label{fig:wbsAccettazione}
			\caption{Work breakdown structure della fase di Accettazione}}
		\end{figure}
	\end{landscape}
	
}