	\subsection{View}{
		\begin{figure}[h]
			\centering
%			\includegraphics[scale=0.6]{\imgs {view}.png} %inserire il diagramma UML
			\label{fig:view}
			\caption{View}
		\end{figure}
		\textbf{\tipo}: questo livello costituisce l'interfaccia del software utilizzabile dagli utenti mediante pagine web.\\
		\textbf{\relaz}: il componente è costituito dal package Pages e comunica con il Controller per rendere possibile la gestione del proprio profilo, la gestione delle presentazioni e per controllare i dati in transito per il sistema, dovuti all'interazione dell'utente con lo stesso.\\
		
		\subsubsection{Premi::View::Pages::IndexPage}{
			\textbf{\tipo}: la classe IndexPage definisce la struttura, e la conseguente visualizazione, della pagina web che consente ad un utente di effettuare login e registrazione al sistema e di passare alla visualizzazione della classe Loader.\\
			\textbf{\relaz}: la classe IndexPage utilizza i metodi messi a disposizione dalla classe [[[[[[[[[[[[[[[[[[CONTROLLER LOGIN]]]]]]]]]]]]]]]]]], contenuta nel package Controller, per verificare i dati inseriti durante la fase di autenticazione, per inviare i dati relativi alla registrazione e per visualizzare eventuali errori emersi nella fase di autenticazione/registrazione.\\
			\textbf{\interfacce}: i metodi implementati nella classe IndexPage sono i seguenti:
			\begin{itemize}
				\item IndexPage::Login invia al controller i dati della login e, se corretti, manda alla pagina Home;
				\item IndexPage::Subscribe invia al controller i dati della registrazione e, se corretti, manda alla pagina Home;
				\item IndexPage::Manifest manda alla pagina Manifest.
			\end{itemize} 
			\textbf{\attivita}: la classe definisce la struttura della pagina web che consente agli utenti di autenticarsi e registrarsi al sistema. Essa resta in attesa che un utente inserisca i dati necessari per l’autenticazione o la registrazione al sistema oppure che l'utente decida di andare nella pagina Loader.\\
		}
		\subsubsection{Premi::View::Pages::Home}{
			\textbf{\tipo}: la classe Home definisce la struttura, e la conseguente visualizzazione, della pagina web che mostra ad un utente le presentazioni presenti sul server e i comandi principali di gestione del profilo e gestione presentazioni.\\	
			\textbf{\relaz}: la classe Home, utilizza i metodi messi a disposizione delle seguenti classi contenute nel package Controller: 
			\begin{itemize}
				\item [[CONTROLLER ELIMINAZIONE PRESENT.]] per l'eliminazione delle presentazioni dal server;
				\item [[[[[[[[[[CONTROLLER SCARICAMENTO MANIFEST]]]]]]]]]] per scaricare una presentazione in locale;
				\item [[[[[[[[[[CONTROLLER LOGOUT]]]]]]]]] per effettuare il logout.
			\end{itemize} 
			\textbf{\interfacce}: i metodi implementati nella classe Home sono i seguenti:
			\begin{itemize}
				\item Home::Delete invia al controller l'id della presentazione da eliminare;
				\item Home::Download invia al controller l'id della presetazione da scaricare in locale;
				\item Home::Execute manda alla pagina Execution con l'id della presentazione da eseguire
				\item Home::NewSlideShow manda alla pagina Edit con la richiesta di una nuova presentazione;
				\item Home::EditSlideShow manda alla pagina Edit con l'id della presentazione da modificare;
				\item Home::Logout manda al controller la richiesta di logout e manda alla pagina Index.				
			\end{itemize} 
			\textbf{\attivita}: la classe definisce la struttura della pagina web che consente agli utenti di visualizzare le anteprime delle proprie presentazioni, crearne di nuove, modificarle, eliminarle, scaricarle in locale e andare alla pagina Profile, effettuare il logout.\\
		}
		\subsubsection{Premi::View::Pages::Manifest}{
			\textbf{\tipo}: la classe Manifest definisce la struttura, e la conseguente visualizzazione, della pagina web che mostra ad un utente le presentazioni scaricate in locale e da la possibilità di eseguirle.\\
			\textbf{\interfacce}: i metodi implementati nella classe Manifest sono i seguenti:
			\begin{itemize}
				\item Manifest::ExecuteManifest esegue la presentazione selezionata utilizzando la pagina html già presente in locale e il framework impress.js;
				\item Manifest::DeleteManifest elimina la presentazione salvate in locale;
			\end{itemize} 
			\textbf{\attivita}: la classe definisce la struttura della pagina web che consente agli utenti di visualizzare le anteprime delle proprie presentazioni, eseguirle e eliminarle dalla posizione in locale.\\\\
		}
		\subsubsection{Premi::View::Pages::Profile}{
			\textbf{\tipo}: la classe Profile definisce la struttura della pagina web che consente agli utenti di modificare i propri dati di profilo e gestire i file media caricati nel server \\
			\textbf{\relaz}: la classe Profile utilizza i metodi messi a disposizione dalle seguenti classi presenti nel package Controller:
			\begin{itemize}
				\item [[[[[[CONTROLLER INSERIMENTO FILE MEDIA]]]]]] per il caricamento di file media nel server;
				\item [[[[[[CONTROLLER ELIMINAZ. FILE MEDIA]]]]]] per la loro eliminazione dal server;
				\item [[[[[[CONTROLLER PASSWORD]]]]]] per la modifica della password;
				\item [[[[[[CONTROLLER RINOMINA FILE MEDIA]]]]]] per rinominarli.
			\end{itemize}
			\textbf{\interfacce}: i metodi implementati nella classe Profile sono i seguenti:
			\begin{itemize}
				\item Profile::ChangePassword invia al controller la nuova password;
				\item Profile::UploadMedia invia al controller le informazioni sul nuovo file media caricato sul server;
				\item Profile::DeleteMedia invia al controller l'id del file media da eliminare;
				\item Profile::RenameMedia invia al controller l'id e il nuovo nome del file media.
			\end{itemize}
			\textbf{\attivita}: la classe Profile definisce la struttura, e la conseguente visualizzazione, della pagina web che mostra ad un utente i dati del proprio profilo, i propri file caricati e la possibilità di modificarli.\\
		}
		\subsubsection{Premi::View::Pages::Execution}{
			\textbf{\tipo}: la classe Execution definisce la struttura, e la conseguente visualizzazione, della pagina web che mostra ad un utente l'esecuzione di una presentazione.\\
			\textbf{\relaz}: questa classe è gestita dal framework esterno Impress.js utilizzato; utilizza i metodi messi a disposizione delle classi [[[[[[[[[CONTROLLER ESECUZIONE PRESENTAZIONE.]]]]]]]]] per creare la pagina che verrà eseguita da Impress.js.\\
			\textbf{\interfacce}: i metodi implementati nella classe Execution sono gestiti dal framework Impress.js con l'aggiunta e la modifica delle seguenti 4 funzioni all'interno del framework:
			\begin{itemize}
				\item Execution::Next va al frame successivo della presentazione;
				\item Execution::Prev va al frame precedente;
				\item Execution::Bookmark va al frame con bookmark successivo.
			\end{itemize}
			\textbf{\attivita}: La classe definisce la struttura della pagina web che consente agli utenti di eseguire la presentazione spostandosi con la tastiera avanti e indietro, passare al capitolo successivo oppure selezionare un nuovo percorso.\\
		}
		\subsubsection{Premi::View::Pages::Edit}{
			\textbf{\tipo}: la classe Edit è divisa in due sottoclassi, che sono visualizzazioni di pagine web diverse a seconda del dispositivo dalla quale viene visualizzata, Desktop o Mobile.\\
		}
		\subsubsection{Premi::View::Pages::EditDesktop}{
			\textbf{\tipo}: la classe EditDesktop definisce la struttura, e la conseguente visualizzazione, della pagina web che mostra da dispositivo desktop ad un utente  l'editor di modifica di una presentazione.\\
			\textbf{\relaz}: mandi principali di gestione del profilo e gestione presentazioni.\\	
			\textbf{\relaz}: la classe Home, utilizza i metodi messi a disposizione dalle seguenti classi presenti nel package Controller:
			\begin{itemize}
				\item [[[[[[[[[CONTROLLER CARICA EDITOR]]]]]]]]] per caricare la presentazione da modificare;
				\item [[[[[[[[[CONTROLLER INSERIMENTO]]]]]]]]] per l'inserimento di nuovi elementi;
				\item [[[[[[[[[CONTROLLER SPOSTAMENTO]]]]]]]]] per lo spostamento di nuovi elementi;
				\item [[[[[[[[[CONTROLLER ELIMINAZIONE]]]]]]]]] per l'eliminazione elementi;
				\item [[[[[[[[[CONTROLLER MODIFICA ELEMENTI]]]]]]]]] per le modifiche  effettuate agli elementi ;
				\item [[[[[[[[[CONTROLLER MODIFICA PERCORSO]]]]]]]]] per cambiare il percorso della presentazione.
			\end{itemize}
			\textbf{\interfacce}: i metodi implementati nella classe EditDesktop sono i seguenti:
			\begin{itemize}
				\item EditDesktop::InsertFrame invia al controller la richiesta di inserimento di un nuovo frame, la sua forma, le coordinate di posizione;
				\item EditDesktop::InsertMedia invia al controller la richiesta di inserimento di un nuovo file media, le sue informazioni e le coordinate di posizione e di rotazione;
				\item EditDesktop::MoveElement invia al controller l'id dell'elemento spostato e le sue nuove coordinate;
				\item EditDesktop::InsertText invia al controller la richiesta di inserimento di un nuovo elemento di testo, il suo contenuto, la sua formattazione e le sue coordinate;
				\item EditDesktop::TextEdit invia al controller l'id dell'elemento di testo e il suo nuovo contenuto;
				\item EditDesktop::DeleteElement invia al controller l'id dell'elemento eliminato;
				\item EditDesktop::InsertChoice invia al controller la richiesta di inserimento di una nuova scelta e l'id del frame a cui è indirizzata la scelta;
				\item EditDesktop::Bookmark invia al controller l'id del frame al quale viene associato o rimosso (a seconda dello stato in quel momento) un bookmark;
				\item EditDesktop::ChangeSize invia al controller l'id dell'elemento al quale vengono cambiate le dimensioni e le nuove misure;
				\item EditDesktop::ChangeRotation invia al controller l'id dell'elemento al quale viene cambiata la rotazione la percentuale di rotazione;
				\item EditDesktop::ChangePath invia al controller l'id del percorso modificato e il nuovo ordine dei frame.
				\item EditDesktop::FrameBackground invia al controller la richiesta di inserimento di un nuovo sfondo ad un frame, l'id del frame e le informazioni dell'immagine;
				\item EditDesktop::Background invia al controller la richiesta di inserimento di un nuovo sfondo alla presentazione e le informazioni dell'immagine;
				\item EditDesktop::InsertSVG invia al controller al richiesta di inserimento di un nuovo elemento SVG, la sua forma, il suo colore e le coordinate di posizione e di rotazione.
			\end{itemize}
			\textbf{\attivita}: La classe definisce la struttura della pagina web che consente agli utenti di modificare una presentazione (inserendo, spostando, modificando o eliminando elementi), cambiare il percorso, assegnare bookmark ai frame e inserire elementi scelta.\\
			\textbf{\base}: Premi::View::Pages::Edit.
		}
		\subsubsection{Premi::View::Pages::EditMobile}{
			\textbf{\tipo}: la classe EditMobile. definisce la struttura, e la conseguente visualizzazione, della pagina web che mostra  da dispositivo mobile ad un utente l'editor di modifica mobile di una presentazione. \\
			\textbf{\relaz}: la classe MobileEdit utilizza i metodi messi a disposizione dalle seguenti classi presenti nel package Controller:
			\begin{itemize}
				\item [[[[[[[[[CONTROLLER CARICA EDITOR MOBILE]]]]]]]]] per caricare la presentazione da modificare;
				\item [[[[[[[[CONTROLLER INSERIMENTO TESTO]]]]]]]] per l'inserimento di un elemento testuale;
				\item [[[[[CONTROLLER MODIFICA TESTO]]]]] per la modifica di un elemento testuale;
				\item [[[[[CONTROLLER INSERIMENTO BOOKMARK]]]]] per l'inserimento di un nuovo bookmark;
				\item [[[[[CONTROLLER REMOVE BOOKMARK]]]]] per rimuovere un bookmark.
			\end{itemize}
		\textbf{\interfacce}: i metodi implementati nella classe EditMobile sono i seguenti:
		\begin{itemize}
			\item EditDesktop::InsertText invia al controller la richiesta di inserimento di un nuovo elemento di testo, il suo contenuto, la sua formattazione e le sue coordinate;
			\item EditDesktop::TextEdit invia al controller l'id dell'elemento di testo e il suo nuovo contenuto;
			\item EditDesktop::Bookmark invia al controller l'id del frame al quale viene associato o rimosso (a seconda dello stato in quel momento) un bookmark;
		\end{itemize}
			\textbf{\attivita}: La classe definisce la struttura della pagina web che consente agli utenti di modificare una presentazione (modificando un elemento testo) e assegnare bookmark ai frame..\\
			\textbf{\base}: Premi::View::Pages::Edit.
		}