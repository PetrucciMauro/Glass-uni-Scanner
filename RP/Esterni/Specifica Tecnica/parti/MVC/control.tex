\subsection{Controller}{
	\textbf{\tipo}: fanno parte di questo livello i package che gestiscono i segnali e le chiamate effettuati dalla view verso la struttura dati.\\
	\textbf{\relaz}: il componente è costituito dai package Presentazione e Utente, comunica con il Model per rendere possibile la gestione del profilo e la gestione delle presentazioni da parte dell'utente.\\
	\textbf{Package contenuti}: 
	\begin{itemize}
	\item Premi::Controller::Presentazione
    \item Premi::Controller::Utente
	\end{itemize}
	\subsubsection{Premi::Controller::Presentazione}{
		\textbf{\tipo}: fanno parte di questo package tutte le classi di controller con cui interagiscono le pagine dedicate alla gestione delle presentazioni.\\
		\textbf{\relaz}: il package comunica con la view ricevendo chiamate da Premi::View::Pages::MobileEdit, Premi::View::Pages::DesktopEdit, Premi::View::Pages::Execution e Premi::View::Pages::Home. Comunica, invece, con il model inviando segnali e chiamate ai package Premi::Model::Inserimento, Premi::Model::Eliminazione, Premi::Model::Modifica, Premi::Model::Command, Premi::Model::Builder e alle classi Premi::Model::Invoker, Premi::Model::MongoHandler.\\
	
	\subsubsubsection{Premi::Controller::Presentazione::EditController}{
		\textbf{\tipo}: Lo scopo di questa classe è di gestire i segnali delle pagine Premi::View::Pages::DesktopEdit e Premi::View::Pages::MobileEdit verso il model.\\	
		\textbf{\relaz}:
		\begin{itemize}
			\item Premi.View.Pages.DesktopEdit e Premi.View.Pages.MobileEdit -> costruiscono EditController, ne invocano i metodi passando i parametri degli oggetti modificati;
			\item Premi::Model::Command <- EditController costruisce un comando e lo dà in pasto a Premi::Model:Invoker;
			\item Premi::Model::Invoker <- EditController costruisce l’oggetto di classe Invoker. Invoca il metodo execute() di Invoker, passando come paramentro un oggetto di classe Command oppure invoca il metodo unexecute() di Invoker;
			\item Premi::Model::Presentazione::Loader <- EditController costruisce l’oggetto di classe Loader, passando come parametro i riferimenti alla presentazione da caricare.
			
		\end{itemize} 
		\textbf{\interfacce}: La pagina DesktopEdit o la pagina MobileEdit invia a EditController un segnale comunicando l’avvenuta modifica o la rimozione di un elemento della presentazione o l’inserimento di un nuovo elemento. EditController istanzia un oggetto di classe Premi::Model::Command e lo dà in pasto a Premi::Model::Invoker. Eventualmente EditController può semplicemente annullare il comando appena eseguito invocando il metodo unexecute di Invoker.
		La pagina web può, inoltre richiedere il caricamento di una presentazione o la creazione di una nuova presentazione a EditController, che, tramite invocazione di Premi::Model::Presentazione::Loader, caricherà dal database.
		\\
	}
	\subsubsubsection{Premi::Controller::Presentazione::HomeController}{
				\textbf{\tipo}: Lo scopo di questa classe è di gestire i segnali della pagina Premi::View::Pages::Home verso la struttura dati.\\	
				\textbf{\relaz}:
				\begin{itemize}
					\item Premi.View.Pages.Home -> costruisce HomeController, ne invoca i metodi passando i parametri dell’utente;
					\item Premi::Model::MongoHandler <- HomeController invoca un metodo di MongoHandler che restituisce l’elenco dei titoli delle presentazioni dell’utente;					
				\end{itemize} 
				\textbf{\interfacce}: La pagina Home costruisce HomeController e richiede l’elenco delle presentazioni dell’utente.\\
			}
		\subsubsubsection{Premi::Controller::Presentazione::ExecutionController}{
				\textbf{\tipo}: Lo scopo di questa classe è di gestire i segnali delle pagine Premi::View::Pages::Execution verso il model.\\	
				\textbf{\relaz}:
					\begin{itemize}
						\item Premi.View.Pages.Execution -> costruiscono ExecutionController, ne invocano i metodi passando i parametri della presentazione da caricare;
						\item Premi::Model::Presentazione::Loader <- ExecutionController passa i parametri di caricamento al Loader che istanzia un oggetto di classe Premi::Model::Presentazione::SlideShow e lo restituisce a Loader.
					\end{itemize}
				\textbf{\interfacce}: La pagina Execution costruisce ExecutionController per caricare la presentazione.\\
		}
	\subsubsection{Premi::Controller::ApacheManager}{
			\textbf{\tipo}: Lo scopo di questa classe è di gestire i segnali della pagina Premi::View::Pages::Profile per l'inserimento, cancellazione e rinominazione di file sul server Apache.\\
			\textbf{\relaz}:
			\begin{itemize}
			 	\item Premi::View::Pages::Profile:UploadMedia -> costruisce ApacheManager, ne invoca i metodi passando i parametri dell'utente ed i parametri del file da caricare; 
			 	\item Premi::View::Pages::Profile::DeleteMedia -> costruisce ApacheManager, ne invoca i metodi passando i parametri dell'utente ed i e l'id del file media da eliminare;
			 	\item Premi::View::Pages::Profile::RenameMedia -> costruisce ApacheManager, ne invoca i metodi passando i parametri dell'utente, l'id e il nuovo nome del file media da rinominare; 
			 	\item Premi::Model::Caricamento::Uploader <- ApacheManager passa i parametri di caricamento ad Uploader che istanzia l'oggetto sul server.
			\end{itemize}
		\textbf{\interfacce}: La pagine Profile costruisce ApacheManager per fare modifiche ai file.
	}
}
	