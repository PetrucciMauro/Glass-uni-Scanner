\subsection{Presenter}{
	\textbf{\tipo}: fanno parte di questo livello i package che gestiscono i segnali e le chiamate effettuati dalla view.\\
	\textbf{\relaz}: comunica con il Model per rendere possibile la gestione del profilo e la gestione delle presentazioni da parte dell'utente.\\
	
	\subsubsection{Presenter::EditPresenter}{
		\textbf{\tipo}: Lo scopo di questa classe è di gestire i segnali e le chiamate delle pagine View::Pages::DesktopEdit e View::Pages::MobileEdit.\\	
		\textbf{\relaz}:
		\begin{itemize}
			
			\item Model::SlideShow::SlideShowActions::Command::ConcreteTextInsertCommand <- EditPresenter costruisce un comando e lo dà in pasto a Model:Invoker;
			\item Model::SlideShow::SlideShowActions::Command::ConcreteFrameInsertCommand <- EditPresenter costruisce un comando e lo dà in pasto a Model:Invoker;
			\item Model::SlideShow::SlideShowActions::Command::ConcreteImageInsertCommand <- EditPresenter costruisce un comando e lo dà in pasto a Model:Invoker;
			\item Model::SlideShow::SlideShowActions::Command::ConcreteSVGInsertCommand <- EditPresenter costruisce un comando e lo dà in pasto a Model:Invoker;
			\item Model::SlideShow::SlideShowActions::Command::ConcreteAudioInsertCommand <- EditPresenter costruisce un comando e lo dà in pasto a Model:Invoker;
			\item Model::SlideShow::SlideShowActions::Command::ConcreteVideoInsertCommand <- EditPresenter costruisce un comando e lo dà in pasto a Model:Invoker;
			\item Model::SlideShow::SlideShowActions::Command::ConcreteBackgroundInsertCommand <- EditPresenter costruisce un comando e lo dà in pasto a Model:Invoker;
			\item Model::SlideShow::SlideShowActions::Command::ConcreteTextRemoveCommand <- EditPresenter costruisce un comando e lo dà in pasto a Model:Invoker;
			\item Model::SlideShow::SlideShowActions::Command::ConcreteFrameRemoveCommand <- EditPresenter costruisce un comando e lo dà in pasto a Model:Invoker;
			\item Model::SlideShow::SlideShowActions::Command::ConcreteImageRemoveCommand <- EditPresenter costruisce un comando e lo dà in pasto a Model:Invoker;
			\item Model::SlideShow::SlideShowActions::Command::ConcreteSVGRemoveCommand <- EditPresenter costruisce un comando e lo dà in pasto a Model:Invoker;
			\item Model::SlideShow::SlideShowActions::Command::ConcreteAudioRemoveCommand <- EditPresenter costruisce un comando e lo dà in pasto a Model:Invoker;
			\item Model::SlideShow::SlideShowActions::Command::ConcreteVideoRemoveCommand <- EditPresenter costruisce un comando e lo dà in pasto a Model:Invoker;
			\item Model::SlideShow::SlideShowActions::Command::ConcreteBackgroundRemoveCommand <- EditPresenter costruisce un comando e lo dà in pasto a Model:Invoker;
			\item Model::SlideShow::SlideShowActions::Command::ConcreteEditSizeCommand <- EditPresenter costruisce un comando e lo dà in pasto a Model:Invoker;
			\item Model::SlideShow::SlideShowActions::Command::ConcreteEditPositionCommand <- EditPresenter costruisce un comando e lo dà in pasto a Model:Invoker;
			\item Model::SlideShow::SlideShowActions::Command::ConcreteEditRotationCommand <- EditPresenter costruisce un comando e lo dà in pasto a Model:Invoker;
			\item Model::SlideShow::SlideShowActions::Command::ConcreteEditColorCommand <- EditPresenter costruisce un comando e lo dà in pasto a Model:Invoker;
			\item Model::SlideShow::SlideShowActions::Command::ConcreteEditBackgroundCommand <- EditPresenter costruisce un comando e lo dà in pasto a Model:Invoker;
			\item Model::SlideShow::SlideShowActions::Command::ConcreteEditFontCommand <- EditPresenter costruisce un comando e lo dà in pasto a Model:Invoker;
			\item Model::SlideShow::SlideShowActions::Command::ConcreteEditContentCommand <- EditPresenter costruisce un comando e lo dà in pasto a Model:Invoker;
			\item Model::SlideShow::SlideShowActions::Command::Invoker <- EditPresenter costruisce l’oggetto di classe Invoker. Invoca il metodo execute() di Invoker, passando come paramentro un oggetto di classe Command oppure invoca il metodo unexecute() di Invoker;
			\item Model::ApacheManager::FileManager <- EditPresenter invoca i metodi uploadFile() di FileManager quando viene inserito nella presentazione un file non ancora presente nel server;
			\item Model::Manifest::ManifestManager <- la classe della view invoca il metodo save() presente in  ExecutionPresenter che a sua volta invoca il metodo update() di ManifestManager che aggiorna il file manifest con tutti gli elementi della presentazione e lo ricarica.  												
			\item View::Pages::DesktopEdit e View::Pages::MobileEdit -> costruiscono EditPresenter, ne invocano i metodi passando i parametri degli oggetti modificati;
			\item View::Pages::DesktopEdit e View::Pages::MobileEdit <- quando il logout ha successo EditPresenter comunica alla view di effettuare una redirect verso Index;
			\item Model::ServerRelations::Loader::Autenticazione <- Quando la view invia una richiesta di logout EditPresenter invoca il metodo di Autenticazione deAuthenticate(), che termina la sessione.
			
		\end{itemize} 
		\textbf{\interfacce}: La pagina DesktopEdit o la pagina MobileEdit invia a EditPresenter comunica l’avvenuta modifica o la rimozione di un elemento della presentazione o l’inserimento di un nuovo elemento invocando i metodi corrispondenti di EditPresenter. EditPresenter istanzia un oggetto di una sottoclasse di Model::SlideShow::SlideShowActions::Command::AbstractCommand e lo dà in pasto a Model::Invoker. Eventualmente EditPresenter, dopo che la View ha invocato il metodo undo() di EditPresenter, può semplicemente annullare il comando appena eseguito invocando il metodo unexecute di Invoker.
		La pagina web può, inoltre richiedere il caricamento di una presentazione o la creazione di una nuova presentazione a EditPresenter, che, tramite invocazione dei metodi di Model::ServerRelations::Loader::Costruttore, caricherà dal database.
		\\
	}
	\subsubsection{Presenter::HomePresenter}{
				\textbf{\tipo}: Lo scopo di questa classe è di gestire i segnali e le chiamate provenienti dalla pagina View::Pages::Home.\\	
				\textbf{\relaz}:
				\begin{itemize}
					\item View::Pages::Home -> costruisce HomePresenter, ne invoca i metodi passando i parametri dell’utente;
					\item Model::ServerRelations::Loader::Costruttore <- HomePresenter invoca un metodo di Costruttore che restituisce l’elenco dei titoli delle presentazioni dell’utente(??????);
					\item Model::ServerRelations::Loader::Autenticazione <- Quando la view invia una richiesta di logout, HomePresenter invoca il metodo deAuthenticate() fornito da Autenticazione, che termina la sessione;
					\item Model::Manifest::ManifestManager <- la classe della view invoca il metodo save() presente in  HomePresenter passando per parametro un array di id di presentazioni che l'utente intende scaricare in locale, a sua volta HomePresenter invoca il metodo update() di ManifestManager che controlla se esiste già un file manifest dopodiché lo aggiorna con tutti i riferimenti alle pagine da scaricare e lo ricarica. 					
				\end{itemize} 
				\textbf{\interfacce}: La pagina Home costruisce HomePresenter e richiede l’elenco delle presentazioni dell’utente.\\
			}
		\subsubsection{Presenter::ExecutionPresenter}{
				\textbf{\tipo}: Lo scopo di questa classe è di gestire i segnali delle pagine View::Pages::Execution verso il model.\\	
				\textbf{\relaz}:
					\begin{itemize}
						\item View::Pages::Execution -> costruisce ExecutionPresenter, ne invoca i metodi passando i parametri della presentazione da caricare;
						\item Model::ServerRelations::Loader::Costruttore <- ExecutionPresenter passa i parametri di caricamento al Loader che carica la presentazione attraverso nodeAPI e lo traduce in html ritornando il codice a ExecutionPresenter;
					\end{itemize}
				\textbf{\interfacce}: La pagina Execution costruisce ExecutionPresenter per caricare la presentazione.\\
		}
		
		\subsubsection{Presenter::IndexPresenter}{
						\textbf{\tipo}: Lo scopo di questa classe è di gestire i segnali e le chiamate della pagina View::Pages::Index.\\	
						\textbf{\relaz}:
							\begin{itemize}
								\item Model::ServerRelations::Loader::Autenticazione <- Quando la view invia una richiesta di login, HomePresenter invoca il metodo authenticate() fornito da Autenticazione, se il login ha successo IndexPresenter invia alla view una richiesta di redirect alla pagina Home;
								\item Model::ServerRelations::Loader::Registrazione <- Quando la view invia una richiesta di registrazione, HomePresenter invoca il metodo register() fornito da Registrazione, se la registrazione ha successo viene eseguito il login e IndexPresenter invia alla view una richiesta di redirect alla pagina Home;
							\end{itemize}
						\textbf{\interfacce}: La pagina Index costruisce IndexPresenter per svolgere le operazioni di login e logout.\\
				}
				
						\subsubsection{Presenter::ProfilePresenter}{
										\textbf{\tipo}: Lo scopo di questa classe è di gestire i segnali e le chiamate della pagina View::Pages::Presenter.\\	
										\textbf{\relaz}:
											\begin{itemize}
												\item Model::ApacheManager::FileManager <- EditPresenter invoca i metodi di FileManager per caricare un file nel server, per modificarne il nome o per eliminarlo dal server;
												\item Model::ServerRelations::Loader::Caricatore <- EditPresenter invoca i metodi di questa classe per cambiare il nome di una presentazione;
												\item Model::ServerRelations::Loader::Autenticazione <- Quando la view invia una richiesta di logout, ProfilePresenter invoca il metodo di Autenticazione deAuthenticate(), che termina la sessione. ProfilePresenter invia quindi una richiesta di redirect alla pagina Index.
											\end{itemize}
										\textbf{\interfacce}: La pagina Execution costruisce ExecutionPresenter per caricare la presentazione.\\
								}
							}
	