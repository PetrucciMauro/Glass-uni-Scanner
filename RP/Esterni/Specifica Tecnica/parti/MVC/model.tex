\subsection{Model}{
	\textbf{\tipo}: è la parte Model dell'architettura MVC.\\
	\textbf{\relaz}: ??????????????????????.\\
	\textbf{Package contenuti}: 
	\begin{itemize}
	\item Premi::Model::Inserimento;
    \item Premi::Model::Rimozione;
    \item Premi::Model::Modifica;
    \item Premi::Model::Command;
    \item Premi::Model::Invoker;
    \item Premi::Model::Builder;
    \item Premi::Model::Presentazione;
    \item Premi::Model::MongoHandler.
	\end{itemize}
	\subsubsection{Premi::Controller::Inserimento}{
		\textbf{\tipo}: All’interno di questo Package viene implementato il Design Pattern template per l’inserimento di nuovi elementi nella presentazione.\\
		\textbf{\relaz}:. Il package è in relazione con Premi::Model::Command da cui riceve i segnali e i parametri di inserimento dell’elemento. Inoltre comunica con il package Premi::Model::Presentazione, istanziando gli oggetti delle sottoclassi di SlideShowElement e inserendoli in SlideShow.\\
	
	\subsubsubsection{Premi::Model::Inserimento::Inserter}{
		\textbf{\tipo}: Classe astratta definita per l’implementazione del Design Pattern template, per l’inserimento di elementi all’interno di una presentazione.\\	
		\textbf{\relaz}:
		\begin{itemize}
			\item Premi::Model::Command::ConcreteConcreteInsertCommand -> utilizza i metodi messi a disposizione da Inserter e concretizzati dalle sue sottoclassi che a loro volta invocano le funzioni della classe Premi::Model::Presentazione::SlideShow per l’impostazione dei campi relativi. 
		\end{itemize} 
		\textbf{\interfacce}: Definisce le operazioni primitive astratte che le classi concrete sottostanti andranno a sovraccaricare e implementa il metodo template che rappresenta lo scheletro dell'algoritmo per l’inserimento di un elemento nella presentazione.
È il componente receiver del Design Pattern Command.\\
        \textbf{\figli}: 
        \begin{itemize}
            \item Premi::Model::Inserimento::ConcreteTextInserter;
            \item Premi::Model::Inserimento::ConcreteFrameInserter;
            \item Premi::Model::Inserimento::ConcreteSvgInserter;
            \item Premi::Model::Inserimento::ConcreteImageInserter;
            \item Premi::Model::Inserimento::ConcreteVideoInserter;
            \item Premi::Model::Inserimento::ConcreteAudioInserter.
        \end{itemize}
	}
	\subsubsubsection{Premi::Controller::Presentazione::Inserimento::ConcreteTextInserter}{
				\textbf{\tipo}: Classe che rappresenta un algoritmo di inserimento di un elemento testuale all’interno di una presentazione. È uno dei componenti concreti del Design Pattern Template.\\	
				\textbf{\relaz}: 
				\begin{itemize}
					\item Premi::Model::ConcreteInsertCommand -> invoca i metodi per inserire un nuovo elemento di tipo testo nella presentazione.
				\end{itemize} 
				\textbf{\interfacce}: Viene invocato per inserire elementi testuali in una presentazione.\\
                \textbf{\base}: 
                    \begin{itemize}
                    \item Premi::Model::Inserimento::Inserter.
                    \end{itemize}
			}
    \subsubsubsection{Premi::Model::Inserimento::ConcreteFrameInserter}{
				\textbf{\tipo}: Classe che rappresenta un algoritmo di inserimento di un elemento frame all’interno di una presentazione.
È uno dei componenti concreti del Design Pattern Template.\\	
				\textbf{\relaz}: 
				\begin{itemize}
					\item Premi::Model::ConcreteInsertCommand -> invoca i metodi per inserire un nuovo elemento di tipo Frame nella presentazione.
				\end{itemize} 
				\textbf{\interfacce}: Viene invocato per inserire elementi di tipo frame in una presentazione.\\
                \textbf{\base}: 
                    \begin{itemize}
                    \item Premi::Model::Inserimento::Inserter.
                    \end{itemize}
			}
       \subsubsubsection{Premi::Model::Inserimento::ConcreteSvgInserter}{
				\textbf{\tipo}: Classe che rappresenta un algoritmo di inserimento di un elemento svg all’interno di una presentazione.
È uno dei componenti concreti del Design Pattern Template.\\	
				\textbf{\relaz}: 
				\begin{itemize}
					\item Premi::Model::ConcreteInsertCommand -> invoca i metodi per inserire un nuovo elemento di tipo SVG nella presentazione.
				\end{itemize} 
				\textbf{\interfacce}: Viene invocato per inserire elementi svg in una presentazione.\\
                \textbf{\base}: 
                    \begin{itemize}
                    \item Premi::Model::Inserimento::Inserter.
                    \end{itemize}
			}
       \subsubsubsection{Premi::Model::Inserimento::ConcreteImageInserter}{
				\textbf{\tipo}: Classe che rappresenta un algoritmo di inserimento di un elemento immagine all’interno di una presentazione.
È uno dei componenti concreti del Design Pattern Template.\\	
				\textbf{\relaz}: 
				\begin{itemize}
					\item Premi::Model::ConcreteInsertCommand -> invoca i metodi per inserire un nuovo elemento di tipo immagine nella presentazione.
				\end{itemize} 
				\textbf{\interfacce}: Viene invocato per inserire elementi di tipo immagine in una presentazione.\\
                \textbf{\base}: 
                    \begin{itemize}
                    \item Premi::Model::Inserimento::Inserter.
                    \end{itemize}
			}
            \subsubsubsection{Premi::Model::Inserimento:: ConcreteVideoInserter}{
				\textbf{\tipo}: Classe che rappresenta un algoritmo di inserimento di un elemento video all’interno di una presentazione.
È uno dei componenti concreti del Design Pattern Template.\\	
				\textbf{\relaz}: 
				\begin{itemize}
					\item Premi::Model::ConcreteInsertCommand -> invoca i metodi per inserire un nuovo elemento di tipo video nella presentazione.
				\end{itemize} 
				\textbf{\interfacce}:Viene invocato per inserire elementi di tipo video in una presentazione.\\
                \textbf{\base}: 
                    \begin{itemize}
                    \item Premi::Model::Inserimento::Inserter.
                    \end{itemize}
			}
            \subsubsubsection{Premi::Model::Inserimento:: ConcreteAudioInserter}{
				\textbf{\tipo}: Classe che rappresenta un algoritmo di inserimento di un elemento di tipo audio all’interno di una presentazione.
È uno dei componenti concreti del Design Pattern Template.\\	
				\textbf{\relaz}: 
				\begin{itemize}
					\item Premi::Model::ConcreteInsertCommand -> invoca i metodi per inserire un nuovo elemento di tipo audio nella presentazione.
				\end{itemize} 
				\textbf{\interfacce}: Viene invocato per inserire elementi di tipo audio in una presentazione.\\
                \textbf{\base}: 
                    \begin{itemize}
                    \item Premi::Model::Inserimento::Inserter.
                    \end{itemize}
			}
	}
    
    	\subsubsection{Premi::Controller::Eliminazione}{
		\textbf{\tipo}: All’interno di questo Package viene implementato il Design Pattern template per l’eliminazione di elementi dalla presentazione.\\
		\textbf{\relaz}:Il package è in relazione con Premi::Model::Command da cui riceve i segnali e i parametri di eliminazione dell’elemento. Inoltre comunica con il package Premi::Model::Presentazione, rimuovendo dall’oggetto di classe SlideShow gli oggetti delle sottoclassi di SlideShowElement e distruggendoli.\\
	
	\subsubsubsection{Premi::Model::Eliminazione::Remover}{
		\textbf{\tipo}: Classe astratta definita per l’implementazione del Design Pattern template, per l’eliminazione di elementi all’interno di una presentazione.\\	
		\textbf{\relaz}:
		\begin{itemize}
			\item Premi::Model::Command::ConcreteConcreteRemoveCommand -> utilizza i metodi messi a disposizione da Remover e concretizzati dalle sue sottoclassi che a loro volta invocano le funzioni della classe.
		\end{itemize} 
		\textbf{\interfacce}: Definisce le operazioni primitive astratte che le classi concrete sottostanti andranno a sovraccaricare o definire e implementa il metodo template che rappresenta lo scheletro dell'algoritmo per l’eliminazione di un elemento nella presentazione.\\
È il componente receiver del Design Pattern Command.\\
        \textbf{\figli}: 
        \begin{itemize}
            \item Premi::Model::Eliminazione::ConcreteTextRemover;
            \item Premi::Model::Eliminazione::ConcreteFrameRemover;
            \item Premi::Model::Eliminazione::ConcreteSvgRemover;
            \item Premi::Model::Eliminazione::ConcreteImageRemover;
            \item Premi::Model::Eliminazione::ConcreteVideoRemover;
            \item Premi::Model::Eliminazione::ConcreteAudioRemover.
        \end{itemize}
	}
	\subsubsubsection{Premi::Model::Eliminazione:: ConcreteTextRemover}{
				\textbf{\tipo}: Classe che implementa un algoritmo di eliminazione di un elemento testuale all’interno di una presentazione.
È uno dei componenti concreti del Design Pattern Template.\\	
				\textbf{\relaz}: 
				\begin{itemize}
					\item Premi::Model::ConcreteRemoveCommand -> invoca i metodi per eliminare un elemento di tipo testo dalla presentazione.
				\end{itemize} 
				\textbf{\interfacce}: Viene invocato per eliminare elementi testuali in una presentazione.\\
                \textbf{\base}: 
                    \begin{itemize}
                    \item Premi::Model::Eliminazione::Remover.
                    \end{itemize}
			}
    \subsubsubsection{Premi::Model::Eliminazione:: ConcreteFrameRemover}{
				\textbf{\tipo}: Classe che implementa un algoritmo di eliminazione di un elemento di tipo frame all’interno di una presentazione.
È uno dei componenti concreti del Design Pattern Template.\\	
				\textbf{\relaz}: 
				\begin{itemize}
					\item Premi::Model::ConcreteRemoveCommand -> invoca i metodi per eliminare un elemento di tipo frame dalla presentazione.
				\end{itemize} 
				\textbf{\interfacce}: Viene invocato per eliminare elementi di tipo frame da una presentazione.\\
                \textbf{\base}: 
                    \begin{itemize}
                    \item Premi::Model::Eliminazione::Remover.
                    \end{itemize}
			}
       \subsubsubsection{Premi::Model::Eliminazione:: ConcreteSVGtRemover}{
				\textbf{\tipo}: Classe che implementa un algoritmo di eliminazione di un elemento di tipo SVG all’interno di una presentazione.
È uno dei componenti concreti del Design Pattern Template.\\	
				\textbf{\relaz}: 
				\begin{itemize}
					\item Premi::Model::ConcreteRemoveCommand -> invoca i metodi per eliminare un elemento di tipo SVG dalla presentazione.
				\end{itemize} 
				\textbf{\interfacce}: Viene invocato per eliminare elementi SVG da una presentazione.\\
                \textbf{\base}: 
                    \begin{itemize}
                    \item Premi::Model::Eliminazione::Remover.
                    \end{itemize}
			}
       \subsubsubsection{Premi::Model::Eliminazione:: ConcreteImageRemover}{
				\textbf{\tipo}: Classe che implementa un algoritmo di eliminazione di un elemento immagine all’interno di una presentazione.
È uno dei componenti concreti del Design Pattern Template.\\	
				\textbf{\relaz}: 
				\begin{itemize}
					\item Premi::Model::ConcreteRemoveCommand -> invoca i metodi per eliminare un elemento di tipo immagine dalla presentazione.
				\end{itemize} 
				\textbf{\interfacce}: Viene invocato per eliminare elementi immagine in una presentazione.\\
                \textbf{\base}: 
                    \begin{itemize}
                    \item Premi::Model::Eliminazione::Remover.
                    \end{itemize}
			}
            \subsubsubsection{Premi::Model::Eliminazione:: ConcreteVideoRemover}{
				\textbf{\tipo}: Classe che implementa un algoritmo di eliminazione di un elemento di tipo video all’interno di una presentazione.
È uno dei componenti concreti del Design Pattern Template.\\	
				\textbf{\relaz}: 
				\begin{itemize}
					\item Premi::Model::ConcreteRemoveCommand -> invoca i metodi per eliminare un elemento di tipo video dalla presentazione.
				\end{itemize} 
				\textbf{\interfacce}: Viene invocato per eliminare elementi di tipo video da una presentazione.\\
                \textbf{\base}: 
                    \begin{itemize}
                    \item Premi::Model::Eliminazione::Remover.
                    \end{itemize}
			}
            \subsubsubsection{Premi::Model::Eliminazione:: ConcreteAudioRemover}{
				\textbf{\tipo}: Classe che implementa un algoritmo di eliminazione di un elemento di tipo audio all’interno di una presentazione.
È uno dei componenti concreti del Design Pattern Template.\\	
				\textbf{\relaz}: 
				\begin{itemize}
					\item Premi::Model::ConcreteRemoveCommand -> invoca i metodi per eliminare un elemento di tipo audio dalla presentazione.
				\end{itemize} 
				\textbf{\interfacce}: Viene invocato per eliminare elementi di tipo audio da una presentazione.\\
                \textbf{\base}: 
                    \begin{itemize}
                    \item Premi::Model::Eliminazione::Remover.
                    \end{itemize}
			}
	}
}
	