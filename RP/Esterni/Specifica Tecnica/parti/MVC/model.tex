\subsection{Model}{
	\begin{figure}[H]
		\centering
		\includegraphics[scale=0.6]{\imgs {Model}.pdf}
		\label{fig:model}
		\caption{Model}
	\end{figure}
	\textbf{\tipo}: è la parte Model dell'architettura MVC.\\
	\textbf{\relaz}: è in relazione con il package Presenter e con NodeAPI.\\
	\textbf{Package contenuti}: 
	\begin{itemize}
	\item Premi::Model::SlideShow;
    \item Premi::Model::ServerRelations;
	\end{itemize}
}
\subsection{Premi::Model::SlideShow}{
		\textbf{\tipo}: All’interno di questo Package si trovano le classi che si riferiscono alla costruzione e alla modifica degli elementi della presentazione oltre alle classi che rappresentano gli elementi stessi della presentazione.\\
        \textbf{\relaz}: il package è in relazione con Premi::Presenter::EditPresenter da cui riceve i segnali e i parametri di inserimento e modifica degli elementi. Inoltre comunica con il package Premi::Model::ServerRelations, inviando a questi i segnali per la modifica in tempo reale dei dati presenti nel database.\\
    }
\subsection{Premi::Model::SlideShow::ModificaSlideShow}{
		\textbf{\tipo}: All’interno di questo Package si trovano le classi che definiscono gli algoritmi di modifica, inserimento e rimozione degli elementi della presentazione.\\
        \textbf{\relaz}: il package è in relazione con Premi::Presenter::EditPresenter da cui riceve i segnali e i parametri di inserimento e modifica degli elementi. Inoltre comunica con il package Premi::Model::ServerRelations, inviando a questi i segnali per la modifica in tempo reale dei dati presenti nel database.\\
    }
\subsection{Premi::Model::SlideShow::SlideShowActions}{
		\textbf{\tipo}: All’interno di questo Package si trovano le classi che si riferiscono alla costruzione, all'inserimento, alla rimozione e alla modifica degli elementi della presentazione.\\
        \textbf{\relaz}: il package è in relazione con Premi::Model::SlideShow::SlideShowActions::Command da cui riceve i segnali e i parametri di inserimento e modifica degli elementi. Inoltre comunica con il package Premi::Model::ServerRelations, inviando a questi i segnali per la modifica in tempo reale dei dati presenti nel database.\\
    }
    
	\subsection{Premi::Model::SlideShow::SlideShowActions::InsertEditRemove}{
		\begin{figure}[H]
			\centering
			\includegraphics[scale=0.6]{\imgs {inserteditremove}.pdf}
			\label{fig:ier}
			\caption{InsertEditRemove}
		\end{figure}
		\textbf{\tipo}: all’interno di questo Package sono implementate le classi statiche destinate all'inserimento, alla rimozione e alla modifica degli elementi della presentazione.\\
		\textbf{\relaz}:il package è in relazione con Premi::Model::SlideShow::SlideShowActions::Command che invoca i metodi delle classi del package.//
        Inoltre Premi::Model::SlideShow::SlideShowActions::InsertEditRemove::Inserter si occupa di costruire gli oggetti presenti nelle classi del package Premi::Model::SlideShow::SlideShowElements.
    }
		\subsubsection{Premi::Model::SlideShow::SlideShowActions::InsertEditRemove::Editor}{
			\textbf{\tipo}: Classe statica che offre i metodi destinati all'eliminazione degli elementi all’interno di una presentazione.\\	
			\textbf{\relaz}:
			\begin{itemize}
				\item Premi::Model::SlideShow::SlideShowActions::Command::ConcreteEditSizeCommand -> invoca il metodo editSize() messo a disposizione da Editor;
				\item Premi::Model::SlideShow::SlideShowActions::Command::ConcreteEditPositionCommand -> invoca il metodo editPosition() messo a disposizione da Editor;
				\item Premi::Model::SlideShow::SlideShowActions::Command::ConcreteEditRotationCommand -> invoca il metodo editRotation() messo a disposizione da Editor;
				\item Premi::Model::SlideShow::SlideShowActions::Command::ConcreteEditColorCommand -> invoca il metodo editColor() messo a disposizione da Editor;
				\item Premi::Model::SlideShow::SlideShowActions::Command::ConcreteEditFontCommand -> invoca il metodo editFont() messo a disposizione da Editor;
				\item Premi::Model::SlideShow::SlideShowActions::Command::ConcreteEditBackgroundCommand -> invoca il metodo editBackground() messo a disposizione da Remover;
				
				\item Premi::Model::SlideShow::SlideShowElements::Text <- Editor invoca i metodi di set degli oggetti di classe Text;
				\item Premi::Model::SlideShow::SlideShowElements::Frame <- Editor invoca i metodi di set degli oggetti di classe Frame;
				\item Premi::Model::SlideShow::SlideShowElements::Image <- Editor invoca i metodi di set degli oggetti di classe Image;
				\item Premi::Model::SlideShow::SlideShowElements::SVG <- Editor invoca i metodi di set degli oggetti di classe SVG;
				\item Premi::Model::SlideShow::SlideShowElements::Audio <- Editor invoca i metodi di set degli oggetti di classe Audio;
				\item Premi::Model::SlideShow::SlideShowElements::Video <- Editor invoca i metodi di set degli oggetti di classe Video;
				\item Premi::Model::SlideShow::SlideShowElements::Background <- Editor invoca i metodi di set degli oggetti di classe Background;
			\end{itemize} 
			\textbf{\interfacce}: È il componente receiver del Design Pattern Command.\\
		}
	\subsubsection{Premi::Model::SlideShow::SlideShowActions::InsertEditRemove::Inserter}{
		\textbf{\tipo}: Classe statica che offre dei metodi per l’inserimento di elementi all’interno di una presentazione.\\	
		\textbf{\relaz}:
		\begin{itemize}
			\item Premi::Model::SlideShow::SlideShowActions::Command::ConcreteTextInsertCommand -> invoca il metodo insertText() messo a disposizione da Inserter;
			\item Premi::Model::SlideShow::SlideShowActions::Command::ConcreteFrameInsertCommand -> invoca il metodo insertFrame() messo a disposizione da Inserter;
			\item Premi::Model::SlideShow::SlideShowActions::Command::ConcreteImageInsertCommand -> invoca il metodo insertImage() messo a disposizione da Inserter;
			\item Premi::Model::SlideShow::SlideShowActions::Command::ConcreteSVGInsertCommand -> invoca il metodo insertSVG() messo a disposizione da Inserter;
			\item Premi::Model::SlideShow::SlideShowActions::Command::ConcreteAudioInsertCommand -> invoca il metodo insertAudio() messo a disposizione da Inserter;
			\item Premi::Model::SlideShow::SlideShowActions::Command::ConcreteVideoInsertCommand -> invoca il metodo insertVideo() messo a disposizione da Inserter;
			\item Premi::Model::SlideShow::SlideShowActions::Command::ConcreteBackgroundInsertCommand -> invoca il metodo insertBackground() messo a disposizione da Inserter;
            \item Premi::Model::SlideShow::SlideShowElements::Text <- Inserter costruisce gli oggetti di classe Text;
            \item Premi::Model::SlideShow::SlideShowElements::Frame <- Inserter costruisce gli oggetti di classe Frame;
            \item Premi::Model::SlideShow::SlideShowElements::Image <- Inserter costruisce gli oggetti di classe Image;
            \item Premi::Model::SlideShow::SlideShowElements::SVG <- Inserter costruisce gli oggetti di classe SVG;
            \item Premi::Model::SlideShow::SlideShowElements::Audio <- Inserter costruisce gli oggetti di classe Audio;
            \item Premi::Model::SlideShow::SlideShowElements::Video <- Inserter costruisce gli oggetti di classe Video;
            \item Premi::Model::SlideShow::SlideShowElements::Background <- Inserter costruisce gli oggetti di classe Background;
            \item Premi::Model::ServerRelations::Loader::Caricatore <- Inserter inserisce gli oggetti json nel campo dati contenitore presentazione.
		\end{itemize} 
		\textbf{\interfacce}: È il componente receiver del Design Pattern Command.\\
	}
	
	\subsubsection{Premi::Model::SlideShow::SlideShowActions::InsertEditRemove::Remover}{
		\textbf{\tipo}: Classe statica che offre i metodi destinati all'eliminazione degli elementi all’interno di una presentazione.\\	
		\textbf{\relaz}:
		\begin{itemize}
			\item Premi::Model::SlideShow::SlideShowActions::Command::ConcreteTextRemoveCommand -> invoca il metodo removeText() messo a disposizione da Remover;
			\item Premi::Model::SlideShow::SlideShowActions::Command::ConcreteFrameRemoveCommand -> invoca il metodo removeFrame() messo a disposizione da Remover;
			\item Premi::Model::SlideShow::SlideShowActions::Command::ConcreteImageRemoveCommand -> invoca il metodo removeImage() messo a disposizione da Remover;
			\item Premi::Model::SlideShow::SlideShowActions::Command::ConcreteSVGRemoveCommand -> invoca il metodo removeSVG() messo a disposizione da Remover;
			\item Premi::Model::SlideShow::SlideShowActions::Command::ConcreteAudioRemoveCommand -> invoca il metodo removeAudio() messo a disposizione da Remover;
			\item Premi::Model::SlideShow::SlideShowActions::Command::ConcreteVideoRemoveCommand -> invoca il metodo removeVideo() messo a disposizione da Remover;
			\item Premi::Model::SlideShow::SlideShowActions::Command::ConcreteBackgroundRemoveCommand -> invoca il metodo removeBackground() messo a disposizione da Remover;
           \item Premi::Model::SlideShow::SlideShowElements::Text <- Editor invoca i metodi di set degli oggetti di classe Text;
           \item Premi::Model::SlideShow::SlideShowElements::Frame <- Editor invoca i metodi di set degli oggetti di classe Frame;
           \item Premi::Model::SlideShow::SlideShowElements::Image <- Editor invoca i metodi di set degli oggetti di classe Image;
           \item Premi::Model::SlideShow::SlideShowElements::SVG <- Editor invoca i metodi di set degli oggetti di classe SVG;
           \item Premi::Model::SlideShow::SlideShowElements::Audio <- Editor invoca i metodi di set degli oggetti di classe Audio;
           \item Premi::Model::SlideShow::SlideShowElements::Video <- Editor invoca i metodi di set degli oggetti di classe Video;
           \item Premi::Model::SlideShow::SlideShowElements::Background <- Editor invoca i metodi di set degli oggetti di classe Background;
           \item Premi::Model::SlideShow::SlideShowElements::Text <- Editor invoca i metodi di set degli oggetti di classe Text;
           \item Premi::Model::SlideShow::SlideShowElements::Frame <- Editor invoca i metodi di set degli oggetti di classe Frame;
           \item Premi::Model::SlideShow::SlideShowElements::Image <- Editor invoca i metodi di set degli oggetti di classe Image;
           \item Premi::Model::SlideShow::SlideShowElements::SVG <- Editor invoca i metodi di set degli oggetti di classe SVG;
           \item Premi::Model::SlideShow::SlideShowElements::Audio <- Editor invoca i metodi di set degli oggetti di classe Audio;
           \item Premi::Model::SlideShow::SlideShowElements::Video <- Editor invoca i metodi di set degli oggetti di classe Video;
           \item Premi::Model::SlideShow::SlideShowElements::Background <- Editor invoca i metodi di set degli oggetti di classe Background;
		\end{itemize} 
		\textbf{\interfacce}: È il componente receiver del Design Pattern Command.\\
	}	
	}
   \subsection{Premi::Model::SlideShow::SlideShowActions::Command}{
	   	\begin{figure}[H]
	   		\centering
	   		\includegraphics[scale=0.6]{\imgs {CommandPackage}.pdf}
	   		\label{fig:cp}
	   		\caption{Command Package}
	   	\end{figure}
		\textbf{\tipo}:All’interno di questo Package viene implementato il Design Pattern command, utile per la gestione di funzioni di annullamento e ripristino.\\
		\textbf{\relaz}:. All’interno del Model, il package è in relazione con Premi::Model::SlideShow::SlideShowActions::Insert, Premi::Model::Remove e Premi::Model::SlideShow::SlideShowActions::EditElements. Il package comunica, inoltre, con il presenter, infatti le sue classi sono generate da Premi::Presenter::SlideShow::EditPresenter.\\
	\subsubsection{Premi::Model::SlideShow::SlideShowActions::Command::Invoker}{
		\textbf{\tipo}: È componente invoker del Design Pattern Command, il suo scopo è tenere traccia delle modifiche atomiche apportate alla presentazione (modifica di elemento, eliminazione di elemento e inserimento di elemento) per poter implementare le funzioni di annulla/ripristina.\\	
		\textbf{\relaz}:
		\begin{itemize}
			\item Premi::Presenter::MobileEdit->crea un oggetto di una sottoclasse di Premi::Model::SlideShow::SlideShowActions::Command::AbstractCommand passandolo all’Invoker che lo esegue e lo inserisce nello stack “undo”, richiama il metodo che svuota lo stack “redo”.\\
			Può inoltre invocare il  metodo “unexecute” dell’Invoker che provvede a richiamare il metodo undo del comando sulla cima dello stack “undo” e a spostarlo quindi nello stack “redo”. Alternativamente invoca il  metodo “redo” dell’Invoker che provvede a eseguire il comando sulla cima dello stack “redo” e a spostarlo quindi nello stack “undo”;
			\item Premi::Presenter::DesktopEdit->si comporta in modo analogo a MobileEdit;
			\item Premi::Model::SlideShow::SlideShowActions::Command::AbstractCommand <- Invoker invoca il metodo execute() dell'oggetto della sottoclasse di AbstractCommand. Alternativamente invoca il metodo undo().
		\end{itemize} 
		\textbf{\interfacce}: Viene invocato per effettuare le operazioni di modifica alla presentazione, a sua volta invoca una classe derivata da Premi::Model::SlideShow::SlideShowActions::Command per eseguire materialmente il comando. Quando un comando viene eseguito, Invoker lo salva in un array \$undo[ ], insieme ai parametri necessari a riportare la presentazione allo stato precedente.\\
	}
	\subsubsection{Premi::Model::SlideShow::SlideShowActions::Command::AbstractCommand}{
				\textbf{\tipo}: È interfaccia astratta del Design Pattern Command, è classe base per i comandi di modifica, inserimento ed eliminazione.\\	
				\textbf{\relaz}: 
				\begin{itemize}
                    \item Premi::Model::Invoker -> esegue materialmente il comando, richiamandone i metodi di esecuzione; inoltre provvede ad annullare l’ultima operazione 
				\end{itemize}	
                \textbf{\interfacce}:Viene utilizzata per applicare un generico parametro di trasformazione ad un oggetto della presentazione, questo parametro verrà poi specificato dalle classi concrete.\\
                \textbf{\figli}: 
                    \begin{itemize}
                    \item Premi::Model::SlideShow::SlideShowActions::Command::ConcreteTextInsertCommand;
                    \item Premi::Model::SlideShow::SlideShowActions::Command::ConcreteFrameInsertCommand;
                    \item Premi::Model::SlideShow::SlideShowActions::Command::ConcreteImageInsertCommand;
                    \item Premi::Model::SlideShow::SlideShowActions::Command::ConcreteSVGInsertCommand;
                    \item Premi::Model::SlideShow::SlideShowActions::Command::ConcreteAudioInsertCommand;
                    \item Premi::Model::SlideShow::SlideShowActions::Command::ConcreteVideoInsertCommand;
                    \item Premi::Model::SlideShow::SlideShowActions::Command::ConcreteBackgroundInsertCommand;
                    \item Premi::Model::SlideShow::SlideShowActions::Command::ConcreteTextRemoveCommand;
                    \item Premi::Model::SlideShow::SlideShowActions::Command::ConcreteFrameRemoveCommand;
                    \item Premi::Model::SlideShow::SlideShowActions::Command::ConcreteImageRemoveCommand;
                    \item Premi::Model::SlideShow::SlideShowActions::Command::ConcreteSVGRemoveCommand;
                    \item Premi::Model::SlideShow::SlideShowActions::Command::ConcreteAudioRemoveCommand;
                    \item Premi::Model::SlideShow::SlideShowActions::Command::ConcreteVideoRemoveCommand;
                    \item Premi::Model::SlideShow::SlideShowActions::Command::ConcreteBackgroundRemoveCommand;
                    \item Premi::Model::SlideShow::SlideShowActions::Command::ConcreteEditSizeCommand;
                    \item Premi::Model::SlideShow::SlideShowActions::Command::ConcreteEditPositionCommand;
                    \item Premi::Model::SlideShow::SlideShowActions::Command::ConcreteEditRotationCommand;
                    \item Premi::Model::SlideShow::SlideShowActions::Command::ConcreteEditColorCommand;
                    \item Premi::Model::SlideShow::SlideShowActions::Command::ConcreteEditBackgroundCommand;
                    \item Premi::Model::SlideShow::SlideShowActions::Command::ConcreteEditFontCommand;
                    \item Premi::Model::SlideShow::SlideShowActions::Command::ConcreteEditContentCommand.
                    \end{itemize}
                    }
    \subsubsection{Premi::Model::SlideShow::SlideShowActions::Command::ConcreteTextInsertCommand}{
				\textbf{\tipo}: È classe concreta del Design Pattern Command, rappresenta un comando per inserire un nuovo elemento testuale nella presentazione.\\	
				\textbf{\relaz}: 
				\begin{itemize}
					\item Premi::Presenter::SlideShow::EditPresenter -> invoca il costruttore della classe e passa l’oggetto così creato all’Invoker;
					\item Premi::Model::SlideShow::SlideShowActions::Command::Invoker -> invoca il metodo doaction() del comando o ne invoca il metodo di annullamento;
                    \item Premi::Model::SlideShow::SlideShowActions::InsertEditRemove::Inserter <- invoca il metodo insertText() della classe statica per l’inserimento di un elemento.
				\end{itemize}	
                \textbf{\base}: 
                    \begin{itemize}
                    \item Premi::Model::SlideShow::SlideShowActions::Command::AbstractCommand.
                    \end{itemize}
                    }
                    \subsubsection{Premi::Model::SlideShow::SlideShowActions::Command::ConcreteFrameInsertCommand}{
				\textbf{\tipo}: È classe concreta del Design Pattern Command, rappresenta un comando per inserire un nuovo elemento frame nella presentazione.\\	
				\textbf{\relaz}: 
				\begin{itemize}
					\item Premi::Presenter::SlideShow::EditPresenter -> invoca il costruttore della classe e passa l’oggetto così creato all’Invoker;
					\item Premi::Model::SlideShow::SlideShowActions::Command::Invoker -> invoca il metodo doaction() del comando o ne invoca il metodo di annullamento;
                    \item Premi::Model::SlideShow::SlideShowActions::InsertEditRemove::Inserter <- invoca il metodo insertFrame() della classe statica per l’inserimento di un elemento frame nella presentazione.
				\end{itemize}	
                \textbf{\base}: 
                    \begin{itemize}
                    \item Premi::Model::SlideShow::SlideShowActions::Command::AbstractCommand.
                    \end{itemize}
                    }
                    \subsubsection{Premi::Model::SlideShow::SlideShowActions::Command::ConcreteImageInsertCommand}{
				\textbf{\tipo}: È classe concreta del Design Pattern Command, rappresenta un comando per inserire un nuovo elemento immagine nella presentazione.\\	
				\textbf{\relaz}: 
				\begin{itemize}
					\item Premi::Presenter::SlideShow::EditPresenter -> invoca il costruttore della classe e passa l’oggetto così creato all’Invoker;
					\item Premi::Model::SlideShow::SlideShowActions::Command::Invoker -> invoca il metodo doaction() del comando o ne invoca il metodo di annullamento;
                    \item Premi::Model::SlideShow::SlideShowActions::InsertEditRemove::Inserter <- invoca il metodo insertImage() della classe statica per l’inserimento di un elemento immagine nella presentazione.
				\end{itemize}	
                \textbf{\base}: 
                    \begin{itemize}
                    \item Premi::Model::SlideShow::SlideShowActions::Command::AbstractCommand.
                    \end{itemize}
                    }
                    \subsubsection{Premi::Model::SlideShow::SlideShowActions::Command::ConcreteSVGInsertCommand}{
				\textbf{\tipo}: È classe concreta del Design Pattern Command, rappresenta un comando per inserire un nuovo elemento SVG nella presentazione.\\	
				\textbf{\relaz}: 
				\begin{itemize}
					\item Premi::Presenter::SlideShow::EditPresenter -> invoca il costruttore della classe e passa l’oggetto così creato all’Invoker;
					\item Premi::Model::SlideShow::SlideShowActions::Command::Invoker -> invoca il metodo doaction() del comando o ne invoca il metodo di annullamento;
                    \item Premi::Model::SlideShow::SlideShowActions::InsertEditRemove::Inserter <- invoca il metodo insertSVG() della classe statica per l’inserimento di un elemento SVG nella presentazione.
				\end{itemize}	
                \textbf{\base}: 
                    \begin{itemize}
                    \item Premi::Model::SlideShow::SlideShowActions::Command::AbstractCommand.
                    \end{itemize}
                    }
                \subsubsection{Premi::Model::SlideShow::SlideShowActions::Command::ConcreteAudioInsertCommand}{
				\textbf{\tipo}: È classe concreta del Design Pattern Command, rappresenta un comando per inserire un nuovo elemento audio nella presentazione.\\	
				\textbf{\relaz}: 
				\begin{itemize}
					\item Premi::Presenter::SlideShow::EditPresenter -> invoca il costruttore della classe e passa l’oggetto così creato all’Invoker;
					\item Premi::Model::SlideShow::SlideShowActions::Command::Invoker -> invoca il metodo doaction() del comando o ne invoca il metodo di annullamento;
                    \item Premi::Model::SlideShow::SlideShowActions::InsertEditRemove::Inserter <- invoca il metodo insertAudio() della classe statica per l’inserimento di un elemento audio nella presentazione.
				\end{itemize}	
                \textbf{\interfacce}: Viene utilizzata per gestire le richieste di inserimento di un nuovo elemento Audio.\\
                \textbf{\base}: 
                    \begin{itemize}
                    \item Premi::Model::SlideShow::SlideShowActions::Command::AbstractCommand.
                    \end{itemize}
                    }
                \subsubsection{Premi::Model::SlideShow::SlideShowActions::Command::ConcreteVideoInsertCommand}{
				\textbf{\tipo}: È classe concreta del Design Pattern Command, rappresenta un comando per inserire un nuovo elemento video nella presentazione.\\	
				\textbf{\relaz}: 
				\begin{itemize}
					\item Premi::Presenter::SlideShow::EditPresenter -> invoca il costruttore della classe e passa l’oggetto così creato all’Invoker;
					\item Premi::Model::SlideShow::SlideShowActions::Command::Invoker -> invoca il metodo doaction() del comando o ne invoca il metodo di annullamento;
                    \item Premi::Model::SlideShow::SlideShowActions::InsertEditRemove::Inserter <- invoca il metodo insertVideo() della classe statica per l’inserimento di un elemento video nella presentazione.
				\end{itemize}	
                \textbf{\base}: 
                    \begin{itemize}
                    \item Premi::Model::SlideShow::SlideShowActions::Command::AbstractCommand.
                    \end{itemize}
                    }
                \subsubsection{Premi::Model::SlideShow::SlideShowActions::Command::ConcreteBackgroundInsertCommand}{
				\textbf{\tipo}: È classe concreta del Design Pattern Command, rappresenta un comando per inserire un nuovo elemento video nella presentazione.\\	
				\textbf{\relaz}: 
				\begin{itemize}
					\item Premi::Presenter::SlideShow::EditPresenter -> invoca il costruttore della classe e passa l’oggetto così creato all’Invoker;
					\item Premi::Model::SlideShow::SlideShowActions::Command::Invoker -> invoca il metodo doaction() del comando o ne invoca il metodo di annullamento;
                    \item Premi::Model::SlideShow::SlideShowActions::InsertEditRemove::Inserter <- invoca il metodo insertBackground() della classe statica per l’inserimento di un elemento sfondo nella presentazione.
				\end{itemize}
                \textbf{\base}: 
                    \begin{itemize}
                    \item Premi::Model::SlideShow::SlideShowActions::Command::AbstractCommand.
                    \end{itemize}
                    }        
     \subsubsection{Premi::Model::SlideShow::SlideShowActions::Command::ConcreteTextRemoveCommand}{
				\textbf{\tipo}: È classe concreta del Design Pattern Command, rappresenta un comando per rimuovere un elemento dalla presentazione.\\	
				\textbf{\relaz}: 
				\begin{itemize}
					\item Premi::Presenter::SlideShow::EditPresenter -> invoca il costruttore della classe e passa l’oggetto così creato all’Invoker;
					\item Premi::Model::SlideShow::SlideShowActions::Command::Invoker -> invoca il metodo doaction() del comando o ne invoca il metodo di annullamento;
                    \item Premi::Model::SlideShow::SlideShowActions::InsertEditRemove::Remover <- invoca il metodo removeText() della classe statica per la rimozione di un elemento testuale nella presentazione.
				\end{itemize}	
                \textbf{\base}: 
                    \begin{itemize}
                    \item Premi::Model::SlideShow::SlideShowActions::Command::AbstractCommand.
                    \end{itemize}
                    }
        \subsubsection{Premi::Model::SlideShow::SlideShowActions::Command::ConcreteFrameRemoveCommand}{
				\textbf{\tipo}: È classe concreta del Design Pattern Command, rappresenta un comando per rimuovere un elemento frame dalla presentazione.\\	
				\textbf{\relaz}: 
				\begin{itemize}
					\item Premi::Presenter::SlideShow::EditPresenter -> invoca il costruttore della classe e passa l’oggetto così creato all’Invoker;
					\item Premi::Model::SlideShow::SlideShowActions::Command::Invoker -> invoca il metodo doaction() del comando o ne invoca il metodo di annullamento;
                    \item Premi::Model::SlideShow::SlideShowActions::InsertEditRemove::Remover <- invoca il metodo removeFrame() della classe statica per la rimozione di un elemento frame dalla presentazione.
				\end{itemize}	
                \textbf{\base}: 
                    \begin{itemize}
                    \item Premi::Model::SlideShow::SlideShowActions::Command::AbstractCommand.
                    \end{itemize}
                    }                   
                    \subsubsection{Premi::Model::SlideShow::SlideShowActions::Command::ConcreteImageRemoveCommand}{
				\textbf{\tipo}: È classe concreta del Design Pattern Command, rappresenta un comando per rimuovere un elemento immagine dalla presentazione.\\	
				\textbf{\relaz}: 
				\begin{itemize}
					\item Premi::Presenter::SlideShow::EditPresenter -> invoca il costruttore della classe e passa l’oggetto così creato all’Invoker;
					\item Premi::Model::SlideShow::SlideShowActions::Command::Invoker -> Invoker invoca il metodo doaction() del comando o ne invoca il metodo di annullamento;
                    \item Premi::Model::SlideShow::SlideShowActions::InsertEditRemove::Remover <- invoca il metodo removeImage() della classe statica per l’eliminazione di un elemento immagine dalla presentazione.
				\end{itemize}	
                \textbf{\base}: 
                    \begin{itemize}
                    \item Premi::Model::SlideShow::SlideShowActions::Command::AbstractCommand.
                    \end{itemize}
                    }               
                    \subsubsection{Premi::Model::SlideShow::SlideShowActions::Command::ConcreteSVGRemoveCommand}{
				\textbf{\tipo}: È classe concreta del Design Pattern Command, rappresenta un comando per rimuovere un elemento SVG dalla presentazione.\\	
				\textbf{\relaz}: 
				\begin{itemize}
					\item Premi::Presenter::SlideShow::EditPresenter -> invoca il costruttore della classe e passa l’oggetto così creato all’Invoker;
					\item Premi::Model::SlideShow::SlideShowActions::Command::Invoker -> Invoker invoca il metodo doaction() del comando o ne invoca il metodo di annullamento;
                    \item Premi::Model::SlideShow::SlideShowActions::InsertEditRemove::Remover <- invoca il metodo removeSVG() della classe statica per l’eliminazione di un elemento SVG dalla presentazione.
				\end{itemize}	
                \textbf{\base}: 
                    \begin{itemize}
                    \item Premi::Model::SlideShow::SlideShowActions::Command::AbstractCommand.
                    \end{itemize}
                    }
                    \subsubsection{Premi::Model::SlideShow::SlideShowActions::Command::ConcreteAudioRemoveCommand}{
				\textbf{\tipo}: È classe concreta del Design Pattern Command, rappresenta un comando per rimuovere un elemento audio dalla presentazione.\\	
				\textbf{\relaz}: 
				\begin{itemize}
					\item Premi::Presenter::SlideShow::EditPresenter -> invoca il costruttore della classe e passa l’oggetto così creato all’Invoker;
					\item Premi::Model::SlideShow::SlideShowActions::Command::Invoker -> Invoker invoca il metodo doaction() del comando o ne invoca il metodo di annullamento;
                    \item Premi::Model::SlideShow::SlideShowActions::InsertEditRemove::Remover <- invoca il metodo removeAudio() della classe statica per l’eliminazione di un elemento immagine dalla presentazione.
				\end{itemize}	
                \textbf{\base}: 
                    \begin{itemize}
                    \item Premi::Model::SlideShow::SlideShowActions::Command::AbstractCommand.
                    \end{itemize}
                    }
                    \subsubsection{Premi::Model::SlideShow::SlideShowActions::Command::ConcreteVideoRemoveCommand}{
				\textbf{\tipo}: È classe concreta del Design Pattern Command, rappresenta un comando per rimuovere un elemento video dalla presentazione.\\	
				\textbf{\relaz}: 
				\begin{itemize}
					\item Premi::Presenter::SlideShow::EditPresenter -> invoca il costruttore della classe e passa l’oggetto così creato all’Invoker;
					\item Premi::Model::SlideShow::SlideShowActions::Command::Invoker -> Invoker invoca il metodo doaction() del comando o ne invoca il metodo di annullamento;
                    \item Premi::Model::SlideShow::SlideShowActions::InsertEditRemove::Remover <- invoca il metodo removeVideo() della classe statica per l’eliminazione di un elemento video dalla presentazione.
				\end{itemize}
                \textbf{\base}: 
                    \begin{itemize}
                    \item Premi::Model::SlideShow::SlideShowActions::Command::AbstractCommand.
                    \end{itemize}
                    }
                    \subsubsection{Premi::Model::SlideShow::SlideShowActions::Command::ConcreteBackgroundRemoveCommand}{
				\textbf{\tipo}: È classe concreta del Design Pattern Command, rappresenta un comando per rimuovere lo sfondo della presentazione.\\	
				\textbf{\relaz}: 
				\begin{itemize}
					\item Premi::Presenter::SlideShow::EditPresenter -> invoca il costruttore della classe e passa l’oggetto così creato all’Invoker;
					\item Premi::Model::SlideShow::SlideShowActions::Command::Invoker -> Invoker invoca il metodo doaction() del comando o ne invoca il metodo di annullamento;
                    \item Premi::Model::SlideShow::SlideShowActions::InsertEditRemove::Remover <- invoca il metodo removeBackground() della classe statica per l’eliminazione dell'elemento sfondo dalla presentazione.
				\end{itemize}	
                \textbf{\base}: 
                    \begin{itemize}
                    \item Premi::Model::SlideShow::SlideShowActions::Command::AbstractCommand.
                    \end{itemize}
                    }
                        \subsubsection{Premi::Model::SlideShow::SlideShowActions::Command::ConcreteEditSizeCommand}{
				\textbf{\tipo}: È classe concreta del Design Pattern Command, rappresenta un comando per modificare le dimensioni di un elemento della presentazione.\\	
				\textbf{\relaz}: 
				\begin{itemize}
					\item Premi::Presenter::SlideShow::EditPresenter -> invoca il costruttore della classe e passa l’oggetto così creato all’Invoker;
					\item Premi::Model::SlideShow::SlideShowActions::Command::Invoker -> Invoker invoca il metodo doaction() del comando o ne invoca il metodo di annullamento;
                    \item Premi::Model::SlideShow::SlideShowActions::InsertEditRemove::Editor <- invoca il metodo editSize() della classe statica per la modifica dei campi dati relativi alle dimensioni dell'oggetto nella presentazione.
				\end{itemize}	
                \textbf{\base}: 
                    \begin{itemize}
                    \item Premi::Model::SlideShow::SlideShowActions::Command::AbstractCommand.
                    \end{itemize}
                    }
                    \subsubsection{Premi::Model::SlideShow::SlideShowActions::Command::ConcreteEditPositionCommand}{
				\textbf{\tipo}: È classe concreta del Design Pattern Command, rappresenta un comando per modificare la posizione di un elemento della presentazione.\\	
				\textbf{\relaz}: 
				\begin{itemize}
					\item Premi::Presenter::SlideShow::EditPresenter -> invoca il costruttore della classe e passa l’oggetto così creato all’Invoker;
					\item Premi::Model::SlideShow::SlideShowActions::Command::Invoker -> Invoker invoca il metodo doaction() del comando o ne invoca il metodo di annullamento;
                    \item Premi::Model::SlideShow::SlideShowActions::InsertEditRemove::Editor <- invoca il metodo editPosition() della classe statica per la modifica dei campi dati relativi alla posizione dell'oggetto nella presentazione.
				\end{itemize}	
                \textbf{\base}: 
                    \begin{itemize}
                    \item Premi::Model::SlideShow::SlideShowActions::Command::AbstractCommand.
                    \end{itemize}
                    }
                    \subsubsection{Premi::Model::SlideShow::SlideShowActions::Command::ConcreteEditColorCommand}{
				\textbf{\tipo}: È classe concreta del Design Pattern Command, rappresenta un comando per modificare il colore di un elemento della presentazione.\\	
				\textbf{\relaz}: 
				\begin{itemize}
					\item Premi::Presenter::SlideShow::EditPresenter -> invoca il costruttore della classe e passa l’oggetto così creato all’Invoker;
					\item Premi::Model::SlideShow::SlideShowActions::Command::Invoker -> Invoker invoca il metodo doaction() del comando o ne invoca il metodo di annullamento;
                    \item Premi::Model::SlideShow::SlideShowActions::InsertEditRemove::Editor <- invoca il metodo editColor() della classe statica per la modifica del campo dati relativo al colore dell'oggetto della presentazione.
				\end{itemize}	
                \textbf{\base}: 
                    \begin{itemize}
                    \item Premi::Model::SlideShow::SlideShowActions::Command::AbstractCommand.
                    \end{itemize}
                    }
                     \subsubsection{Premi::Model::SlideShow::SlideShowActions::Command::ConcreteEditBackgroundCommand}{
				\textbf{\tipo}: È classe concreta del Design Pattern Command, rappresenta un comando per modificare lo sfondo di un elemento frame della presentazione.\\	
				\textbf{\relaz}: 
				\begin{itemize}
					\item Premi::Presenter::SlideShow::EditPresenter -> invoca il costruttore della classe e passa l’oggetto così creato all’Invoker;
					\item Premi::Model::SlideShow::SlideShowActions::Command::Invoker -> Invoker invoca il metodo doaction() del comando o ne invoca il metodo di annullamento;
                    \item Premi::Model::SlideShow::SlideShowActions::InsertEditRemove::Editor <- invoca il metodo editBackground() della classe statica per la modifica del campo dati relativo allo sfondo dell'oggetto della presentazione.
				\end{itemize}	
                \textbf{\base}: 
                    \begin{itemize}
                    \item Premi::Model::SlideShow::SlideShowActions::Command::AbstractCommand.
                    \end{itemize}
                    }
                     \subsubsection{Premi::Model::SlideShow::SlideShowActions::Command::ConcreteEditRotationCommand}{
				\textbf{\tipo}: È classe concreta del Design Pattern Command, rappresenta un comando per modificare l'orientamento di un elemento della presentazione.\\	
				\textbf{\relaz}: 
				\begin{itemize}
					\item Premi::Presenter::SlideShow::EditPresenter -> invoca il costruttore della classe e passa l’oggetto così creato all’Invoker;
					\item Premi::Model::SlideShow::SlideShowActions::Command::Invoker -> Invoker invoca il metodo doaction() del comando o ne invoca il metodo di annullamento;
                    \item Premi::Model::SlideShow::SlideShowActions::InsertEditRemove::Editor <- invoca il metodo editRotation() della classe statica per la modifica del campo dati relativo all'orientamento dell'oggetto della presentazione.
				\end{itemize}	
<<<<<<< e88e6bcbe29175cd959243ec0f79391e3cae0b11
                \textbf{\interfacce}: Viene utilizzata per gestire i Signal riguardanti la modifica dell'orientamento di un elemento;\\
=======
>>>>>>> 5d4b765697ecf4c596dcad1eb3d4f52bf617fc3d
                \textbf{\base}: 
                    \begin{itemize}
                    \item Premi::Model::SlideShow::SlideShowActions::Command::AbstractCommand.
                    \end{itemize}
                    }
                    \subsubsection{Premi::Model::SlideShow::SlideShowActions::Command::ConcreteEditFontCommand}{
				\textbf{\tipo}: È classe concreta del Design Pattern Command, rappresenta un comando per modificare il carattere di un elemento testuale della presentazione.\\	
				\textbf{\relaz}: 
				\begin{itemize}
					\item Premi::Presenter::SlideShow::EditPresenter -> invoca il costruttore della classe e passa l’oggetto così creato all’Invoker;
					\item Premi::Model::SlideShow::SlideShowActions::Command::Invoker -> Invoker invoca il metodo doaction() del comando o ne invoca il metodo di annullamento;
                    \item Premi::Model::SlideShow::SlideShowActions::InsertEditRemove::Editor <- invoca il metodo editColor() della classe statica per la modifica dei campi dati relativi al font dell'oggetto testuale della presentazione.
				\end{itemize}	
                \textbf{\base}: 
                    \begin{itemize}
                    \item Premi::Model::SlideShow::SlideShowActions::Command::AbstractCommand.
                    \end{itemize}
                    }
                    }
                     \subsubsection{Premi::Model::SlideShow::SlideShowElements}{
                     		\begin{figure}[H]
                     			\centering
                     			\includegraphics[scale=0.6]{\imgs {slideshowelements}.pdf}
                     			\label{fig:sse}
                     			\caption{SlideShowElements}
                     		\end{figure}
		\textbf{\tipo}:Di questo package fanno parte le classi degli elementi della presentazione e la classe che definisce la presentazione stessa. Sì tratta del package centrale del software.\\
		\textbf{\relaz}:.Premi::Model::SlideShow::SlideShowElements è in comunicazione con 
        \begin{itemize}
        \item Premi::Model::SlideShow::SlideShowActions::Insert, i cui oggetti durante la modifica della presentazione istanziano oggetti di tipo SlideShowElement;
        \item Premi::Model::Remove, i cui oggetti rimuovono da Premi::ServerRelations::Caricatore gli oggetti di tipo SlideShowElement e li distruggono;
        \item Premi::Model::SlideShow::SlideShowActions::EditElements, i cui oggetti invocano metodi degli oggetti SlideShowElement che ne impostano i campi;
  		\end{itemize}

    \subsubsection{Premi::Model::SlideShow::SlideShowElements::SlideShowElement}{
				\textbf{\tipo}: Gli oggetti della classe SlideShowElement rappresentano gli elementi della presentazione.\\	
				\textbf{\relaz}: 
				\begin{itemize}
					\item Premi::Model::SlideShow::SlideShowActions::Insert::Inserter-> invoca il costruttore delle sottoclassi di SlideShowElement e li inserisce nei campi dati contenitori all’interno di Premi::Model::SlideShow::SlideShow;
                    \item Premi::Model::SlideShow::SlideShowActions::EditElements:Editor -> gli oggetti delle sue sottoclassi richiamano le funzioni delle sottoclassi di SlideShowElement che gestiscono l’impostazione dei campi dati;
                    \item Premi::Model::SlideShow::SlideShowActions::Remove::Remover -> gli oggetti delle sue sottoclassi rimuovono dai contenitori di SlideShow gli oggetti di classe SlideShowElement e ne richiamano i distruttori.
				\end{itemize}	
                \textbf{\interfacce}: Premi::Model::SlideShow::SlideShowActions::Insert::Inserter instanzia oggetti di sottoclassi di SlideShowElement e li inserisce nel campo dati contenitore presentazione all’interno di Premi::Model::ServerRelations::Model:Costruttore\\
                \textbf{\figli}: 
                    \begin{itemize}
                    \item Premi::Model::SlideShow::Text;
                    \item Premi::Model::SlideShow::Frame;
                    \item Premi::Model::SlideShow::Image;
                    \item Premi::Model::SlideShow::SVG;
                    \item Premi::Model::SlideShow::Audio;
                    \item Premi::Model::SlideShow::Video;
                    \item Premi::Model::SlideShow::Background.
                    \end{itemize}
                    }
     \subsubsection{Premi::Model::SlideShow::SlideShowElements::Text}{
				\textbf{\tipo}: Gli oggetti della classe Text rappresentano gli elementi di tipo testuale della presentazione.\\
				\textbf{\relaz}: 
				\begin{itemize}
					\item Premi::Model::SlideShow::SlideShowActions::InsertEditRemove::Inserter -> invoca il costruttore di Text e inserisce l’oggetto nel campo dati contenitore all’interno dell’oggetto della classe Premi::Model::ServerRelations::Loader::Caricatore;
                    \item Premi::Model::SlideShow::SlideShowActions::InsertEditRemove::Remover -> rimuove l’oggetto Text dal campo dati presentazione all’interno di Premi::Model::ServerRelations::Loader::Costruttore, ne invoca quindi il distruttore;
                    \item Premi::Model::SlideShow::SlideShowActions::InsertEditRemove::Editor -> invoca i metodi che modificano i campi dati dell'oggetto.
				\end{itemize}	
                \textbf{\interfacce}: Gli oggetti della classe Text vengono istanziati da Premi::Model::SlideShow::SlideShowActions::Insert::ConcreteTextInserter e inseriti nel campo dati contenitore presentazione all’interno di Premi::Model::ServerRelations::Loader::Costruttore.\\
                \textbf{\base}: 
                    \begin{itemize}
                    \item Premi::Model::SlideShow::SlideShowElement.
                    \end{itemize}
                    }
           \subsubsection{Premi::Model::SlideShow::SlideShowElements::Frame}{
				\textbf{\tipo}: Gli oggetti della classe Frame rappresentano gli elementi di tipo frame della presentazione.\\
				\textbf{\relaz}: 
				\begin{itemize}
					\item Premi::Model::SlideShow::SlideShowActions::InsertEditRemove::Inserter -> invoca il costruttore di Frame e inserisce l’oggetto nel campo dati contenitore all’interno dell’oggetto della classe Premi::Model::ServerRelations::Loader::Caricatore;
                    \item Premi::Model::SlideShow::SlideShowActions::InsertEditRemove::Remover -> rimuove l’oggetto Frame dal campo dati presentazione all’interno di Premi::Model::ServerRelations::Loader::Costruttore, ne invoca quindi il distruttore;
                    \item Premi::Model::SlideShow::SlideShowActions::InsertEditRemove::Editor -> invoca i metodi che modificano i campi dati dell'oggetto.
				\end{itemize}	
                \textbf{\interfacce}: Gli oggetti della classe Frame vengono istanziati da Premi::Model::SlideShow::SlideShowActions::InsertEditRemove::Inserter inseriti nel campo dati contenitore presentazione all’interno di Premi::Model::ServerRelations::Loader::Costruttore.\\
                \textbf{\base}: 
                    \begin{itemize}
                    \item Premi::Model::SlideShow::SlideShowElement.
                    \end{itemize}
                    }
                    \subsubsection{Premi::Model::SlideShow::SlideShowElements::Image}{
				\textbf{\tipo}: Gli oggetti della classe Image rappresentano gli elementi di tipo immagine della presentazione.\\
				\textbf{\relaz}: 
				\begin{itemize}
					\item Premi::Model::SlideShow::SlideShowActions::InsertEditRemove::Inserter -> invoca il costruttore di Image e inserisce l’oggetto nel campo dati contenitore all’interno dell’oggetto della classe Premi::Model::ServerRelations::Loader::Caricatore;
                    \item Premi::Model::SlideShow::SlideShowActions::InsertEditRemove::Remover -> rimuove l’oggetto Image dal campo dati presentazione all’interno di Premi::Model::ServerRelations::Loader::Costruttore, ne invoca quindi il distruttore;
                    \item Premi::Model::SlideShow::SlideShowActions::InsertEditRemove::Editor ->  invoca i metodi che modificano i campi dati dell'oggetto.
				\end{itemize}	
                \textbf{\interfacce}: Gli oggetti della classe Image vengono istanziati da Premi::Model::SlideShow::SlideShowActions::InsertEditRemove::Inserter  inseriti nel campo dati contenitore presentazione all’interno di Premi::Model::ServerRelations::Loader::Costruttore.\\
                \textbf{\base}: 
                    \begin{itemize}
                    \item Premi::Model::SlideShow::SlideShowElement.
                    \end{itemize}
                    }
                    \subsubsection{Premi::Model::SlideShow::SlideShowElements::SVG}{
				\textbf{\tipo}: Gli oggetti della classe SVG rappresentano gli elementi di tipo SVG della presentazione.\\
				\textbf{\relaz}: 
				\begin{itemize}
					\item Premi::Model::SlideShow::SlideShowActions::InsertEditRemove::Inserter -> invoca il costruttore di SVG e inserisce l’oggetto nel campo dati contenitore all’interno dell’oggetto della classe Premi::Model::ServerRelations::Loader::Caricatore;
                    \item Premi::Model::SlideShow::SlideShowActions::InsertEditRemove::Remover -> rimuove l’oggetto SVG dal campo dati presentazione all’interno di Premi::Model::ServerRelations::Loader::Costruttore, ne invoca quindi il distruttore;
                    \item Premi::Model::SlideShow::SlideShowActions::InsertEditRemove::Editor -> invoca i metodi che modificano i campi dati dell'oggetto.
				\end{itemize}	
                \textbf{\interfacce}: Gli oggetti della classe SVG vengono istanziati da Premi::Model::SlideShow::SlideShowActions::InsertEditRemove::Inserter e da questi inseriti nel campo dati contenitore presentazione all’interno di Premi::Model::ServerRelations::Loader::Costruttore.\\
                \textbf{\base}: 
                    \begin{itemize}
                    \item Premi::Model::SlideShow::SlideShowElement.
                    \end{itemize}
                    }
                    \subsubsection{Premi::Model::SlideShow::SlideShowElements::Audio}{
				\textbf{\tipo}: Gli oggetti della classe Audio rappresentano gli elementi di tipo audio della presentazione.\\
				\textbf{\relaz}: 
				\begin{itemize}
					\item Premi::Model::SlideShow::SlideShowActions::InsertEditRemove::Inserter -> invoca il costruttore di Audio e inserisce l’oggetto nel campo dati contenitore all’interno dell’oggetto della classe Premi::Model::ServerRelations::Loader::Caricatore;
                    \item Premi::Model::SlideShow::SlideShowActions::InsertEditRemove::Remover -> rimuove l’oggetto Audio dal campo dati presentazione all’interno di Premi::Model::ServerRelations::Loader::Costruttore, ne invoca quindi il distruttore;
                    \item Premi::Model::SlideShow::SlideShowActions::InsertEditRemove::Editor -> invoca i metodi che modificano i campi dati dell'oggetto.
				\end{itemize}	
                \textbf{\interfacce}: Gli oggetti della classe Audio vengono istanziati da Premi::Model::SlideShow::SlideShowActions::InsertEditRemove::Inserter e da questi inseriti nel campo dati contenitore presentazione all’interno di Premi::Model::ServerRelations::Loader::Costruttore.\\
                \textbf{\base}:  
                    \begin{itemize}
                    \item Premi::Model::SlideShow::SlideShowElements::SlideShowElement.
                    \end{itemize}
                    }
                    \subsubsection{Premi::Model::SlideShow::SlideShowElements::Video}{
				\textbf{\tipo}: Gli oggetti della classe Video rappresentano gli elementi di tipo video della presentazione.\\
				\textbf{\relaz}: 
				\begin{itemize}
					\item Premi::Model::SlideShow::SlideShowActions::InsertEditRemove::Inserter -> invoca il costruttore di Video e inserisce l’oggetto nel campo dati contenitore all’interno dell’oggetto della classe Premi::Model::ServerRelations::Loader::Caricatore;
                    \item Premi::Model::SlideShow::SlideShowActions::InsertEditRemove::Remover -> rimuove l’oggetto Video dal campo dati presentazione all’interno di Premi::Model::ServerRelations::Loader::Costruttore, ne invoca quindi il distruttore;
                     \item Premi::Model::SlideShow::SlideShowActions::InsertEditRemove::Editor -> invoca i metodi che modificano i campi dati dell'oggetto.
				\end{itemize}	
                \textbf{\interfacce}: Gli oggetti della classe Video vengono istanziati da Premi::Model::SlideShow::SlideShowActions::InsertEditRemove::Inserter e da questi inseriti nel campo dati contenitore presentazione all’interno di Premi::Model::ServerRelations::Loader::Costruttore.\\
                \textbf{\base}: 
                    \begin{itemize}
                    \item Premi::Model::SlideShow::SlideShowElements::SlideShowElement.
                    \end{itemize}
                    }     
                 \subsubsection{Premi::Model::SlideShow::Background}{
                				\textbf{\tipo}: Gli oggetti della classe Background rappresentano lo sfondo della presentazione.\\
                				\textbf{\relaz}: 
                				\begin{itemize}
                					\item Premi::Model::SlideShow::SlideShowActions::InsertEditRemove::Inserter -> invoca il costruttore di Background e inserisce l’oggetto nel campo dati contenitore all’interno dell’oggetto della classe Premi::Model::ServerRelations::Loader::Caricatore;
                                    \item Premi::Model::SlideShow::SlideShowActions::InsertEditRemove::Remover -> rimuove l’oggetto Video dal campo dati presentazione all’interno di Premi::Model::ServerRelations::Loader::Costruttore, ne invoca quindi il distruttore;
                                    \item Premi::Model::SlideShow::SlideShowActions::InsertEditRemove::Editor -> invoca i metodi che modificano i campi dati dell'oggetto.
                				\end{itemize}	
                                \textbf{\interfacce}: Gli oggetti della classe Background vengono istanziati da Premi::Model::SlideShow::SlideShowActions::InsertEditRemove::Inserter e da questi inseriti nel campo dati contenitore presentazione all’interno di Premi::Model::ServerRelations::Loader::Costruttore.\\
                                \textbf{\base}: 
                                    \begin{itemize}
                                    \item Premi::Model::SlideShow::SlideShowElements::SlideShowElement.
                                    \end{itemize}
                                    }              
}


\subsection{Premi::Model::ServerRelations}{\\
	\begin{figure}[H]
		\centering
		\includegraphics[scale=0.6]{\imgs {ServerRelations}.pdf}
		\label{fig:sr}
		\caption{ServerRelations}
	\end{figure}
		\textbf{\tipo}: il package racchiude le funzionalità del sistema che interagiscono direttamente con i servizi web esposti dalla interfaccia nodeApi.\\
		\textbf{\relaz}: i componenti del package serverRelations hanno relazioni di dipendenza nei confronti del package nodeApi del quale utilizzano i servizi esposti dall’interfaccia; c’e’ dipenda tra il package serverRelations ed altri package del model.\\
}

\subsection{Premi::Model::ServerRelations::Loader}{
	\begin{figure}[H]
		\centering
		\includegraphics[scale=0.6]{\imgs {serverrelationsloader}.pdf}
		\label{fig:srl}
		\caption{ServerRelationsLoader}
	\end{figure}
		\textbf{\tipo}: il package racchiude le funzioni di recupero di una presentazione dal server attraversi i servizi nodeApi e traduzione della presentazione in elementi html che compongono la view della presentazione recuperata.\\
		\textbf{\relaz}: relazione di dipendenza con l’interfaccia dei servizi nodeApi per il recupero della presentazione.\\

\subsubsection{Premi::Model::ServerRelations::Loader::Costruttore}{
				\textbf{\tipo}: Classe la cui funzione è recuperare una presentazione dal database remoto o creare una nuova presentazione,  caricare la presentazione in formato html così da poter essere modificata o eseguita dall’utente.\\	
				\textbf{\relaz}: 
				\begin{itemize}
					\item nodeAPI <- dipendenza nei confronti del package nodeApi di cui chiama i servizi http in modo sincrono.
				\end{itemize}	
                    }
}

\subsection{Premi::Model::ServerRelations::AccessControl}{
	\begin{figure}[H]
		\centering
		\includegraphics[scale=0.6]{\imgs {AccessControl}.pdf}
		\label{fig:accCnt}
		\caption{AccessControl}
	\end{figure}
		\textbf{\tipo}: il package racchiude le funzioni di registrazione dell’utente e autenticazione tramite token ai servizi nodeApi.\\
		\textbf{\relaz}: dipendenza nei confronti dei servizi resi disponibili dall’interfaccia nodeApi; altri package in ServerRelations utilizzano questo package per recuperare il token per accedere ai servizi nodeApi di interazione con le presentazioni in remoto.\\
        \subsubsection{Premi::Model::ServerRelations::AccessControll::Autenticazione}{
				\textbf{\tipo}: Classe che fornisce funzionalità di autenticazione e deautenticazione ai servizi offerti da nodeApi attraverso passaggio di token.\\	
				\textbf{\relaz}: 
				\begin{itemize}
					\item nodeAPI <- dipendenza nei confronti di nodeApi di cui chiama in modo sincrono i servizi.
                    \item Premi::Presenter::Pagine::IndexPresenter -> invoca i metodi di Autenticazione per permettere all'utente di effettuare il login.
				\end{itemize}	
                    }
        \subsubsection{Premi::Model::ServerRelations::AccessControll::Registrazione}{
				\textbf{\tipo}: Classe, fornisce funzionalità di registrazione all’utente.\\	
				\textbf{\relaz}: 
				\begin{itemize}
					\item nodeAPI <- dipendenza nei confronti di nodeApi di cui chiama in modo sincrono i servizi.
                    \item Premi::Presenter::Pagine::IndexPresenter -> invoca i metodi di Registrazione per permettere all'utente di registrarsi al servizio.
				\end{itemize}	
            }
}

\subsection{Premi::Model::ServerRelations::DbConsistency}{
	\begin{figure}[H]
		\centering
		\includegraphics[scale=0.6]{\imgs {ClassDiagram1}.pdf}
		\label{fig:cd}
		\caption{DbConsistency}
	\end{figure}
		\textbf{\tipo}: il package ha lo scopo di raccogliere le funzionalità di aggiornamento delle presentazioni in remoto tramite un pattern observer e chiamate asincrone ai servizi di nodeApi\\
		\textbf{\relaz}:dipendenza con il package nodeApi; dipendenza nei confronti di altri package in Model per il recupero dello stato degli elementi della presentazione.\\
       
       \subsubsection{Premi::Model::ServerRelations::DbConsistency::Observer}{
				\textbf{\tipo}: Interfaccia, espone il metodo update(), utile per l’implementazione del design pattern “Observer”.\\	
				\textbf{\relaz}: 
				\begin{itemize}
					\item associazione con Subject per rendere effettiva la notify(); realizzata da ConcreteObserver che definisce il metodo update().
				\end{itemize}	
            }
            
            
            \subsubsection{Premi::Model::ServerRelations::DbConsistency::ConcreteObserver}{
				\textbf{\tipo}: Classe, concretizza l’interfaccia Observer, utile ad implementare il pattern “Observer”.\\	
				\textbf{\relaz}: 
				\begin{itemize}
					\item realizza l’interfaccia Observer definendone il metodo update(); associazione verso Subject.
				\end{itemize}	
            }
            
             \subsubsection{Premi::Model::ServerRelations::DbConsistency::Subject}{
				\textbf{\tipo}: Classe astratta, definisce una classe astratta per i diversi tipi di subject a seconda degli elementi da osservare. Definisce i metodi attach(Observer), detach(Observer) e notify() per implementare il pattern "Observer".\\	
				\textbf{\relaz}: 
				\begin{itemize}
					\item associazione da ConcreteObserver; classe astratta realizzata dalle classi: SubjectAudio, SubjectVideo, SubjectText, SubjectFrame, SubjectSvg, SubjectImg che definiscono il metodo getElement() utilizzato da ConcreteObserver per ottenere l’oggetto modificato.
				\end{itemize}	
            }
            
             \subsubsection{Premi::Model::ServerRelations::DbConsistency::SubjectAudio}{
				\textbf{\tipo}: Classe, fornisce un’implementazione di Subject permettendo di applicare il pattern "Observer".\\	
				\textbf{\relaz}: 
				\begin{itemize}
					\item implementa Subject definendo il metodo getElement(), associazione con la classe Premi::Model::SlideShow::SlideShowElements::Audio di cui detiene un riferimento.
				\end{itemize}	
            }
            
            \subsubsection{Premi::Model::ServerRelations::DbConsistency::SubjectAudio}{
				\textbf{\tipo}: Classe, fornisce un’implementazione di Subject permettendo di applicare il pattern "Observer".\\	
				\textbf{\relaz}: 
				\begin{itemize}
					\item implementa Subject definendo il metodo getElement(), associazione con la classe Premi::Model::SlideShow::SlideShowElements::Audio di cui detiene un riferimento.
				\end{itemize}	
            }
            
            \subsubsection{Premi::Model::ServerRelations::DbConsistency::SubjectVideo}{
				\textbf{\tipo}: Classe, fornisce un’implementazione di Subject permettendo di applicare il pattern "Observer".\\	
				\textbf{\relaz}: 
				\begin{itemize}
					\item implementa Subject definendo il metodo getElement(), associazione con la classe Premi::Model::SlideShow::SlideShowElements::Video di cui detiene un riferimento.
				\end{itemize}	
            }
            
            \subsubsection{Premi::Model::ServerRelations::DbConsistency::SubjectText}{
				\textbf{\tipo}: Classe, fornisce un’implementazione di Subject permettendo di applicare il pattern "Observer".\\	
				\textbf{\relaz}: 
				\begin{itemize}
					\item implementa Subject definendo il metodo getElement(), associazione con la classe Premi::Model::SlideShow::SlideShowElements::Text di cui detiene un riferimento.
				\end{itemize}	
            }
            
            \subsubsection{Premi::Model::ServerRelations::DbConsistency::SubjectFrame}{
				\textbf{\tipo}: Classe, fornisce un’implementazione di Subject permettendo di applicare il pattern "Observer".\\	
				\textbf{\relaz}: 
				\begin{itemize}
					\item implementa Subject definendo il metodo getElement(), associazione con la classe Premi::Model::SlideShow::SlideShowElements::Frame di cui detiene un riferimento.
				\end{itemize}	
            }
            
            \subsubsection{Premi::Model::ServerRelations::DbConsistency::SubjectImg}{
				\textbf{\tipo}: Classe, fornisce un’implementazione di Subject permettendo di applicare il pattern "Observer".\\	
				\textbf{\relaz}: 
				\begin{itemize}
					\item implementa Subject definendo il metodo getElement(), associazione con la classe Premi::Model::SlideShow::SlideShowElements::Image di cui detiene un riferimento.
				\end{itemize}	
            }
            
            \subsubsection{Premi::Model::ServerRelations::DbConsistency::SubjectSVG}{
				\textbf{\tipo}: Classe, fornisce un’implementazione di Subject permettendo di applicare il pattern "Observer".\\	
				\textbf{\relaz}: 
				\begin{itemize}
					\item implementa Subject definendo il metodo getElement(), associazione con la classe Premi::Model::SlideShow::SlideShowElements::SVG di cui detiene un riferimento.
				\end{itemize}	
            }
            
             \subsubsection{Premi::Model::ServerRelations::DbConsistency::SubjectBackground}{
				\textbf{\tipo}: Classe, fornisce un’implementazione di Subject permettendo di applicare il pattern "Observer".\\	
				\textbf{\relaz}: 
				\begin{itemize}
					\item implementa Subject definendo il metodo getElement(), associazione con la classe Premi::Model::SlideShow::SlideShowElements::Background di cui detiene un riferimento.
				\end{itemize}	
            }

}
\subsection{Premi::Model::Manifest}{
   	\textbf{\tipo}: Questo package ha lo scopo di rendere disponibili le presentazioni in locale tramite chiamate ai servizi di nodeApi e ai metodi definiti in Premi::Model::ServerRelation::Loader. \\
   	\textbf{\relaz}:
   	\begin{itemize}
   		\item definisce il metodo GestoreManifest(); relazione di dipendenza 
   	\end{itemize}

	\subsubsection{Premi::Model::Manifest::GestoreManifest}{
		\textbf{\tipo}: classe, fornisce i servizi raccolti nel package;\\
		\textbf{\relaz}:
		\begin{itemize}
			\item definisce il metodo insertElement(), addPage(), update(), associazione con la classe Premi::Model::ServerRelation::Loader.
		\end{itemize}
        }
}
\subsection{Premi::Model::GestioneFileServer} {
   	\textbf{\tipo}: Questo package racchiude le funzioni che permettono all'utente di interfacciarsi coi propri file presenti sul server Apache.
   	\textbf{\relaz}:
   	\begin{itemize}
   		\item relazione di dipendenza con l'interfaccia dei servizi di Apache.
   	\end{itemize}
   	\textbf{\interfacce}: Viene utilizzata per gestire le chiamate asincrone verso il server, per realizzare le operazioni di inserimento, cancellazione, cambio del nome di un file.\\
	
	\subsubsection{Premi::Model::GestioneFileServer::GestioneFile}{
		\textbf{\tipo}: classe che fornisce un'implementazione del package, per permettere il caricamento di file sul server, eliminazione di file dal server e rinominazione di file presenti sul server; \\
		\textbf{\relaz}:
		\begin{itemize}
			\item Premi::Presenter::EditPresenter, Premi::View::Pages::Profile, interfaccia del server Apache;
		\end{itemize}
	}
}

	