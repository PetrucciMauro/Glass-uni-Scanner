\subsection{Model}{
	\textbf{\tipo}: è la parte Model dell'architettura MVC.\\
	\textbf{\relaz}: ??????????????????????.\\
	\textbf{Package contenuti}: 
	\begin{itemize}
	\item Premi::Model::Inserimento;
    \item Premi::Model::Rimozione;
    \item Premi::Model::Modifica;
    \item Premi::Model::Command;
    \item Premi::Model::Invoker;
    \item Premi::Model::Builder;
    \item Premi::Model::Presentazione;
    \item Premi::Model::MongoHandler.
	\end{itemize}
	\subsubsection{Premi::Model::Inserimento}{
		\textbf{\tipo}: All’interno di questo Package viene implementato il Design Pattern template per l’inserimento di nuovi elementi nella presentazione.\\
		\textbf{\relaz}:. Il package è in relazione con Premi::Model::Command da cui riceve i segnali e i parametri di inserimento dell’elemento. Inoltre comunica con il package Premi::Model::Presentazione, istanziando gli oggetti delle sottoclassi di SlideShowElement e inserendoli in SlideShow.\\
	
	\subsubsubsection{Premi::Model::Inserimento::Inserter}{
		\textbf{\tipo}: Classe astratta definita per l’implementazione del Design Pattern template, per l’inserimento di elementi all’interno di una presentazione.\\	
		\textbf{\relaz}:
		\begin{itemize}
			\item Premi::Model::Command::ConcreteConcreteInsertCommand -> utilizza i metodi messi a disposizione da Inserter e concretizzati dalle sue sottoclassi che a loro volta invocano le funzioni della classe Premi::Model::Presentazione::SlideShow per l’impostazione dei campi relativi. 
		\end{itemize} 
		\textbf{\interfacce}: Definisce le operazioni primitive astratte che le classi concrete sottostanti andranno a sovraccaricare e implementa il metodo template che rappresenta lo scheletro dell'algoritmo per l’inserimento di un elemento nella presentazione.
È il componente receiver del Design Pattern Command.\\
        \textbf{\figli}: 
        \begin{itemize}
            \item Premi::Model::Inserimento::ConcreteTextInserter;
            \item Premi::Model::Inserimento::ConcreteFrameInserter;
            \item Premi::Model::Inserimento::ConcreteSvgInserter;
            \item Premi::Model::Inserimento::ConcreteImageInserter;
            \item Premi::Model::Inserimento::ConcreteVideoInserter;
            \item Premi::Model::Inserimento::ConcreteAudioInserter.
        \end{itemize}
	}
	\subsubsubsection{Premi::Controller::Presentazione::Inserimento::ConcreteTextInserter}{
				\textbf{\tipo}: Classe che rappresenta un algoritmo di inserimento di un elemento testuale all’interno di una presentazione. È uno dei componenti concreti del Design Pattern Template.\\	
				\textbf{\relaz}: 
				\begin{itemize}
					\item Premi::Model::ConcreteInsertCommand -> invoca i metodi per inserire un nuovo elemento di tipo testo nella presentazione.
				\end{itemize} 
				\textbf{\interfacce}: Viene invocato per inserire elementi testuali in una presentazione.\\
                \textbf{\base}: 
                    \begin{itemize}
                    \item Premi::Model::Inserimento::Inserter.
                    \end{itemize}
			}
    \subsubsubsection{Premi::Model::Inserimento::ConcreteFrameInserter}{
				\textbf{\tipo}: Classe che rappresenta un algoritmo di inserimento di un elemento frame all’interno di una presentazione.
È uno dei componenti concreti del Design Pattern Template.\\	
				\textbf{\relaz}: 
				\begin{itemize}
					\item Premi::Model::ConcreteInsertCommand -> invoca i metodi per inserire un nuovo elemento di tipo Frame nella presentazione.
				\end{itemize} 
				\textbf{\interfacce}: Viene invocato per inserire elementi di tipo frame in una presentazione.\\
                \textbf{\base}: 
                    \begin{itemize}
                    \item Premi::Model::Inserimento::Inserter.
                    \end{itemize}
			}
       \subsubsubsection{Premi::Model::Inserimento::ConcreteSvgInserter}{
				\textbf{\tipo}: Classe che rappresenta un algoritmo di inserimento di un elemento svg all’interno di una presentazione.
È uno dei componenti concreti del Design Pattern Template.\\	
				\textbf{\relaz}: 
				\begin{itemize}
					\item Premi::Model::ConcreteInsertCommand -> invoca i metodi per inserire un nuovo elemento di tipo SVG nella presentazione.
				\end{itemize} 
				\textbf{\interfacce}: Viene invocato per inserire elementi svg in una presentazione.\\
                \textbf{\base}: 
                    \begin{itemize}
                    \item Premi::Model::Inserimento::Inserter.
                    \end{itemize}
			}
       \subsubsubsection{Premi::Model::Inserimento::ConcreteImageInserter}{
				\textbf{\tipo}: Classe che rappresenta un algoritmo di inserimento di un elemento immagine all’interno di una presentazione.
È uno dei componenti concreti del Design Pattern Template.\\	
				\textbf{\relaz}: 
				\begin{itemize}
					\item Premi::Model::ConcreteInsertCommand -> invoca i metodi per inserire un nuovo elemento di tipo immagine nella presentazione.
				\end{itemize} 
				\textbf{\interfacce}: Viene invocato per inserire elementi di tipo immagine in una presentazione.\\
                \textbf{\base}: 
                    \begin{itemize}
                    \item Premi::Model::Inserimento::Inserter.
                    \end{itemize}
			}
            \subsubsubsection{Premi::Model::Inserimento:: ConcreteVideoInserter}{
				\textbf{\tipo}: Classe che rappresenta un algoritmo di inserimento di un elemento video all’interno di una presentazione.
È uno dei componenti concreti del Design Pattern Template.\\	
				\textbf{\relaz}: 
				\begin{itemize}
					\item Premi::Model::ConcreteInsertCommand -> invoca i metodi per inserire un nuovo elemento di tipo video nella presentazione.
				\end{itemize} 
				\textbf{\interfacce}:Viene invocato per inserire elementi di tipo video in una presentazione.\\
                \textbf{\base}: 
                    \begin{itemize}
                    \item Premi::Model::Inserimento::Inserter.
                    \end{itemize}
			}
            \subsubsubsection{Premi::Model::Inserimento:: ConcreteAudioInserter}{
				\textbf{\tipo}: Classe che rappresenta un algoritmo di inserimento di un elemento di tipo audio all’interno di una presentazione.
È uno dei componenti concreti del Design Pattern Template.\\	
				\textbf{\relaz}: 
				\begin{itemize}
					\item Premi::Model::ConcreteInsertCommand -> invoca i metodi per inserire un nuovo elemento di tipo audio nella presentazione.
				\end{itemize} 
				\textbf{\interfacce}: Viene invocato per inserire elementi di tipo audio in una presentazione.\\
                \textbf{\base}: 
                    \begin{itemize}
                    \item Premi::Model::Inserimento::Inserter.
                    \end{itemize}
			}
	}
    
    	\subsubsection{Premi::Model::Eliminazione}{
		\textbf{\tipo}: All’interno di questo Package viene implementato il Design Pattern template per l’eliminazione di elementi dalla presentazione.\\
		\textbf{\relaz}:Il package è in relazione con Premi::Model::Command da cui riceve i segnali e i parametri di eliminazione dell’elemento. Inoltre comunica con il package Premi::Model::Presentazione, rimuovendo dall’oggetto di classe SlideShow gli oggetti delle sottoclassi di SlideShowElement e distruggendoli.\\
	
	\subsubsubsection{Premi::Model::Eliminazione::Remover}{
		\textbf{\tipo}: Classe astratta definita per l’implementazione del Design Pattern template, per l’eliminazione di elementi all’interno di una presentazione.\\	
		\textbf{\relaz}:
		\begin{itemize}
			\item Premi::Model::Command::ConcreteConcreteRemoveCommand -> utilizza i metodi messi a disposizione da Remover e concretizzati dalle sue sottoclassi che a loro volta invocano le funzioni della classe.
		\end{itemize} 
		\textbf{\interfacce}: Definisce le operazioni primitive astratte che le classi concrete sottostanti andranno a sovraccaricare o definire e implementa il metodo template che rappresenta lo scheletro dell'algoritmo per l’eliminazione di un elemento nella presentazione.\\
È il componente receiver del Design Pattern Command.\\
        \textbf{\figli}: 
        \begin{itemize}
            \item Premi::Model::Eliminazione::ConcreteTextRemover;
            \item Premi::Model::Eliminazione::ConcreteFrameRemover;
            \item Premi::Model::Eliminazione::ConcreteSvgRemover;
            \item Premi::Model::Eliminazione::ConcreteImageRemover;
            \item Premi::Model::Eliminazione::ConcreteVideoRemover;
            \item Premi::Model::Eliminazione::ConcreteAudioRemover.
        \end{itemize}
	}
	\subsubsubsection{Premi::Model::Eliminazione:: ConcreteTextRemover}{
				\textbf{\tipo}: Classe che implementa un algoritmo di eliminazione di un elemento testuale all’interno di una presentazione.
È uno dei componenti concreti del Design Pattern Template.\\	
				\textbf{\relaz}: 
				\begin{itemize}
					\item Premi::Model::ConcreteRemoveCommand -> invoca i metodi per eliminare un elemento di tipo testo dalla presentazione.
				\end{itemize} 
				\textbf{\interfacce}: Viene invocato per eliminare elementi testuali in una presentazione.\\
                \textbf{\base}: 
                    \begin{itemize}
                    \item Premi::Model::Eliminazione::Remover.
                    \end{itemize}
			}
    \subsubsubsection{Premi::Model::Eliminazione:: ConcreteFrameRemover}{
				\textbf{\tipo}: Classe che implementa un algoritmo di eliminazione di un elemento di tipo frame all’interno di una presentazione.
È uno dei componenti concreti del Design Pattern Template.\\	
				\textbf{\relaz}: 
				\begin{itemize}
					\item Premi::Model::ConcreteRemoveCommand -> invoca i metodi per eliminare un elemento di tipo frame dalla presentazione.
				\end{itemize} 
				\textbf{\interfacce}: Viene invocato per eliminare elementi di tipo frame da una presentazione.\\
                \textbf{\base}: 
                    \begin{itemize}
                    \item Premi::Model::Eliminazione::Remover.
                    \end{itemize}
			}
       \subsubsubsection{Premi::Model::Eliminazione:: ConcreteSVGtRemover}{
				\textbf{\tipo}: Classe che implementa un algoritmo di eliminazione di un elemento di tipo SVG all’interno di una presentazione.
È uno dei componenti concreti del Design Pattern Template.\\	
				\textbf{\relaz}: 
				\begin{itemize}
					\item Premi::Model::ConcreteRemoveCommand -> invoca i metodi per eliminare un elemento di tipo SVG dalla presentazione.
				\end{itemize} 
				\textbf{\interfacce}: Viene invocato per eliminare elementi SVG da una presentazione.\\
                \textbf{\base}: 
                    \begin{itemize}
                    \item Premi::Model::Eliminazione::Remover.
                    \end{itemize}
			}
       \subsubsubsection{Premi::Model::Eliminazione:: ConcreteImageRemover}{
				\textbf{\tipo}: Classe che implementa un algoritmo di eliminazione di un elemento immagine all’interno di una presentazione.
È uno dei componenti concreti del Design Pattern Template.\\	
				\textbf{\relaz}: 
				\begin{itemize}
					\item Premi::Model::ConcreteRemoveCommand -> invoca i metodi per eliminare un elemento di tipo immagine dalla presentazione.
				\end{itemize} 
				\textbf{\interfacce}: Viene invocato per eliminare elementi immagine in una presentazione.\\
                \textbf{\base}: 
                    \begin{itemize}
                    \item Premi::Model::Eliminazione::Remover.
                    \end{itemize}
			}
            \subsubsubsection{Premi::Model::Eliminazione:: ConcreteVideoRemover}{
				\textbf{\tipo}: Classe che implementa un algoritmo di eliminazione di un elemento di tipo video all’interno di una presentazione.
È uno dei componenti concreti del Design Pattern Template.\\	
				\textbf{\relaz}: 
				\begin{itemize}
					\item Premi::Model::ConcreteRemoveCommand -> invoca i metodi per eliminare un elemento di tipo video dalla presentazione.
				\end{itemize} 
				\textbf{\interfacce}: Viene invocato per eliminare elementi di tipo video da una presentazione.\\
                \textbf{\base}: 
                    \begin{itemize}
                    \item Premi::Model::Eliminazione::Remover.
                    \end{itemize}
			}
            \subsubsubsection{Premi::Model::Eliminazione:: ConcreteAudioRemover}{
				\textbf{\tipo}: Classe che implementa un algoritmo di eliminazione di un elemento di tipo audio all’interno di una presentazione.
È uno dei componenti concreti del Design Pattern Template.\\	
				\textbf{\relaz}: 
				\begin{itemize}
					\item Premi::Model::ConcreteRemoveCommand -> invoca i metodi per eliminare un elemento di tipo audio dalla presentazione.
				\end{itemize} 
				\textbf{\interfacce}: Viene invocato per eliminare elementi di tipo audio da una presentazione.\\
                \textbf{\base}: 
                    \begin{itemize}
                    \item Premi::Model::Eliminazione::Remover.
                    \end{itemize}
			}
	}
   \subsubsection{Premi::Model::Modifica}{
		\textbf{\tipo}: All’interno di questo Package viene implementato il Design Pattern strategy per la modifica di elementi della presentazione.\\
		\textbf{\relaz}:Il package è in relazione con Premi::Model::Command da cui riceve i segnali e i parametri di modifica dell’elemento. Inoltre comunica con il package Premi::Model::Presentazione, modificando nell’oggetto di classe SlideShow gli oggetti delle sottoclassi di SlideShowElement.\\
	
	\subsubsubsection{Premi::Model::Modifica::Editor}{
		\textbf{\tipo}: Interfaccia per la componente strategy del Design Pattern Strategy per la selezione dell'algoritmo di modifica della presentazione.\\	
		\textbf{\relaz}:
		\begin{itemize}
			\item Premi::Model::Command::ConcreteEditCommand -> Invoca i costruttori delle sottoclassi di Editor; 
            \item Premi::Model::Presentazione::SlideShow <- scorre gli elementi dei membri contenitori all’interno di SlideShow per trovare l’elemento da modificare;
            \item Premi::Model::Presentazione::SlideShowElement <- invoca le funzioni della classe SlideShowElement per modificare opportunamente i campi dell’elemento.
		\end{itemize} 
		\textbf{\interfacce}: Permette di selezionare dinamicamente ed in modo estensibile l'algoritmo di modifica della presentazione.\\
        \textbf{\figli}: 
        \begin{itemize}
            \item Premi::Model::Modifica::Editor::EditorPosition;
            \item Premi::Model::Modifica::Editor::EditorSize;
            \item Premi::Model::Modifica::Editor::EditorContent;
            \item Premi::Model::Modifica::Editor::EditorRotate;
            \item Premi::Model::Modifica::Editor::EditorColor;
            \item Premi::Model::Modifica::Editor::EditorShape.
        \end{itemize}
	}
	\subsubsubsection{Premi::Model::Modifica::EditorPosition}{
				\textbf{\tipo}: Classe concreta del Design Pattern Strategy per la modifica dei campi inerenti alla posizione di un elemento della presentazione.\\	
				\textbf{\relaz}: 
				\begin{itemize}
					\item Premi::Model::Command::ConcreteEditCommand -> Invoca il costruttore di EditorPosition;
                    \item Premi::Model::Presentazione::SlideShow<- scorre gli elementi dei membri contenitori all’interno di SlideShow per trovare l’elemento da modificare; 
                    \item Premi::Model::Presentazione::SlideShowElement <- invoca le funzioni della classe SlideShowElement per modificare opportunamente i campi relativi alla posizione dell’elemento.
				\end{itemize}	\textbf{\interfacce}:Premi::Model::Command::ConcreteEditCommand invoca il costruttore e la funzione di esecuzione dell’operazione di modifica, EditorPosition invocherà quindi i metodi di modifica delle coordinate forniti all’interno della sottoclasse di Premi::Model::Presentazione::SlideShowElement di cui fa parte l’oggetto da modificare.\\
                \textbf{\base}: 
                    \begin{itemize}
                    \item Premi::Model::Modifica::Editor.
                    \end{itemize}
                    }
    \subsubsubsection{Premi::Model::Modifica::EditorSize}{
				\textbf{\tipo}: Classe concreta del Design Pattern Strategy per la modifica dei campi inerenti alla dimensione di un elemento della presentazione.\\	
				\textbf{\relaz}: 
				\begin{itemize}
					\item Premi::Model::Command::ConcreteEditCommand -> Invoca il costruttore di EditorSize;
                    \item Premi::Model::Presentazione::SlideShow<- scorre gli elementi dei membri contenitori all’interno di SlideShow per trovare l’elemento da modificare; 
                    \item Premi::Model::Presentazione::SlideShowElement <- invoca le funzioni della classe SlideShowElement per modificare opportunamente i campi relativi alla dimensione dell’elemento.
				\end{itemize}	\textbf{\interfacce}:Premi::Model::Command::ConcreteEditCommand invoca il costruttore e la funzione di esecuzione dell’operazione di modifica, EditorSize invocherà quindi i metodi di modifica delle dimensioni forniti all’interno della sottoclasse di Premi::Model::Presentazione::SlideShowElement di cui fa parte l’oggetto da modificare.\\
                \textbf{\base}: 
                    \begin{itemize}
                    \item Premi::Model::Modifica::Editor.
                    \end{itemize}
                    }
       \subsubsubsection{Premi::Model::Modifica::EditorRotate}{
				\textbf{\tipo}: Classe concreta del Design Pattern Strategy per la modifica dei campi inerenti all'inclinazione di un elemento della presentazione.\\	
				\textbf{\relaz}: 
				\begin{itemize}
					\item Premi::Model::Command::ConcreteEditCommand -> Invoca il costruttore di EditorRotate;
                    \item Premi::Model::Presentazione::SlideShow<- scorre gli elementi dei membri contenitori all’interno di SlideShow per trovare l’elemento da modificare; 
                    \item Premi::Model::Presentazione::SlideShowElement <- invoca le funzioni della classe SlideShowElement per modificare opportunamente i campi relativi all'inclinazione dell’elemento.
				\end{itemize}	\textbf{\interfacce}:Premi::Model::Command::ConcreteEditCommand invoca il costruttore e la funzione di esecuzione dell’operazione di modifica, EditorPosition invocherà quindi i metodi di modifica dell'inclinazione forniti all’interno della sottoclasse di Premi::Model::Presentazione::SlideShowElement di cui fa parte l’oggetto da modificare.\\
                \textbf{\base}: 
                    \begin{itemize}
                    \item Premi::Model::Modifica::Editor.
                    \end{itemize}
                    }
       \subsubsubsection{Premi::Model::Modifica::EditorContent}{
				\textbf{\tipo}: Classe concreta del Design Pattern Strategy per la modifica dei campi inerenti al contenuto di un elemento di tipo testuale della presentazione.\\	
				\textbf{\relaz}: 
				\begin{itemize}
					\item Premi::Model::Command::ConcreteEditCommand -> Invoca il costruttore di EditorContent;
                    \item Premi::Model::Presentazione::SlideShow<- scorre gli elementi del membro contenitore all’interno di SlideShow per trovare l’elemento testuale da modificare; 
                    \item Premi::Model::Presentazione::SlideShowElement <- invoca le funzioni della classe SlideShowElement per modificare opportunamente i campi relativi alla contenuto dell’elemento testuale.
				\end{itemize}	\textbf{\interfacce}:Premi::Model::Command::ConcreteEditCommand invoca il costruttore e la funzione di esecuzione dell’operazione di modifica, EditorContent invocherà quindi i metodi di modifica del contenuto forniti all’interno della classe Premi::Model::Presentazione::Text.\\
                \textbf{\base}: 
                    \begin{itemize}
                    \item Premi::Model::Modifica::Editor.
                    \end{itemize}
                    }
            \subsubsubsection{Premi::Model::Modifica::EditorShape}{
				\textbf{\tipo}: Classe concreta del Design Pattern Strategy per la modifica dei campi inerenti alla forma di un elemento SVG della presentazione.\\	
				\textbf{\relaz}: 
				\begin{itemize}
					\item Premi::Model::Command::ConcreteEditCommand -> Invoca il costruttore di EditorShape;
                    \item Premi::Model::Presentazione::SlideShow<- scorre gli elementi dei membri contenitori all’interno di SlideShow per trovare l’elemento SVG da modificare; 
                    \item Premi::Model::Presentazione::SlideShowElement <- invoca le funzioni della classe SlideShowElement per modificare opportunamente i campi relativi alla forma dell’elemento SVG.
				\end{itemize}	\textbf{\interfacce}:Premi::Model::Command::ConcreteEditCommand invoca il costruttore e la funzione di esecuzione dell’operazione di modifica, EditorShape invocherà quindi i metodi di modifica della forma forniti all’interno della classe Premi::Model::Presentazione::SVG.\\
                \textbf{\base}: 
                    \begin{itemize}
                    \item Premi::Model::Modifica::Editor.
                    \end{itemize}
                    }
            \subsubsubsection{Premi::Model::Modifica::EditorColor}{
				\textbf{\tipo}: Classe concreta del Design Pattern Strategy per la modifica dei campi inerenti al colore di un elemento SVG della presentazione.\\	
				\textbf{\relaz}: 
				\begin{itemize}
					\item Premi::Model::Command::ConcreteEditCommand -> Invoca il costruttore di EditorColor;
                    \item Premi::Model::Presentazione::SlideShow<- scorre gli elementi del membro contenitore all’interno di SlideShow per trovare l’elemento SVG da modificare; 
                    \item Premi::Model::Presentazione::SlideShowElement <- invoca le funzioni della classe SlideShowElement per modificare opportunamente i campi relativi alla forma dell’elemento SVG.
				\end{itemize}	\textbf{\interfacce}:Premi::Model::Command::ConcreteEditCommand invoca il costruttore e la funzione di esecuzione dell’operazione di modifica, EditorShape invocherà quindi i metodi di modifica della forma forniti dalla classe Premi::Model::Presentazione::SVG.\\
                \textbf{\base}: 
                    \begin{itemize}
                    \item Premi::Model::Modifica::Editor.
                    \end{itemize}
                    }
   \subsubsection{Premi::Model::Command}{
		\textbf{\tipo}:All’interno di questo Package viene implementato il Design Pattern command, utile per la gestione di funzioni di annullamento e ripristino.\\
		\textbf{\relaz}:. All’interno del Model, il package è in relazione con Premi::Model::Inserimento, Premi::Model::Eliminazione e Premi::Model::Modifica. Il package comunica, inoltre, con il controller, infatti le sue classi sono generate da Premi::Controller::Presentazione::EditController.\\
	\subsubsubsection{Premi::Model:: Invoker}{
		\textbf{\tipo}: È componente invoker del Design Pattern Command, il suo scopo è tenere traccia delle modifiche atomiche apportate alla presentazione (modifica di elemento, eliminazione di elemento e inserimento di elemento) per poter implementare le funzioni di annulla/ripristina.\\	
		\textbf{\relaz}:
		\begin{itemize}
			\item Premi::Controller::Inserimento::InsertController->crea un oggetto della classe Premi::Model::Command::ConcreteInsertCommand passandolo all’Invoker che lo esegue e lo inserisce nello stack “undo”, richiama il metodo che svuota lo stack “redo”;
            \item Premi::Controller::Eliminazione::RemoveController->crea un oggetto della classe Premi::Model::Command::ConcreteRemoveCommand passandolo all’Invoker che lo esegue e lo inserisce nello stack “undo” , richiama il metodo che svuota lo stack “redo”;
            \item Premi::Controller::Presentazione::EditController->crea un oggetto della classe Premi::Model::Command::ConcreteEditCommand passandolo all’Invoker che lo esegue (execute) e lo inserisce nello stack “undo” , richiama il metodo che svuota lo stack “redo”. Può inoltre invocare il  metodo “unexecute” dell’Invoker che provvede a richiamare il metodo undo del comando sulla cima dello stack “undo” e a spostarlo quindi nello stack “redo”. Alternativamente invoca il  metodo “redo” dell’Invoker che provvede a eseguire il comando sulla cima dello stack “redo” e a spostarlo quindi nello stack “undo”.
		\end{itemize} 
		\textbf{\interfacce}: Viene invocato per effettuare le operazioni di modifica alla presentazione, a sua volta invoca una classe derivata da Premi::Model::Command per eseguire materialmente il comando. Quando un comando viene eseguito, Invoker lo salva in un array \$undo[ ], insieme ai parametri necessari a riportare la presentazione allo stato precedente.\\
	}
	\subsubsubsection{Premi::Model::Command::AbstractCommand}{
				\textbf{\tipo}: È interfaccia astratta del Design Pattern Command, è classe base per i comandi di modifica, inserimento ed eliminazione.\\	
				\textbf{\relaz}: 
				\begin{itemize}
                    \item Premi::Model:: Invoker -> esegue materialmente il comando, richiamandone i metodi di esecuzione; inoltre provvede ad annullare l’ultima operazione 
				\end{itemize}	
                \textbf{\interfacce}:Viene utilizzata per applicare un generico parametro di trasformazione ad un oggetto della presentazione, questo parametro verrà poi specificato dalle classi concrete.\\
                \textbf{\figli}: 
                    \begin{itemize}
                    \item Premi::Model::Command::ConcreteInsertCommand;
                    \item Premi::Model::Command::ConcreteRemoveCommand;
                    \item Premi::Model::Command::ConcreteEditCommand.
                    \end{itemize}
                    }
    \subsubsubsection{Premi::Model::Command::ConcreteInsertCommand}{
				\textbf{\tipo}: È classe concreta del Design Pattern Command, rappresenta un comando per inserire un nuovo elemento nell’oggetto presentazione.\\	
				\textbf{\relaz}: 
				\begin{itemize}
					\item Premi::Controller::Presentazione::EditController -> invoca il costruttore della classe e passa l’oggetto così creato all’Invoker;
Premi::Model::Invoker -> esegue il comando o ne invoca il metodo di annullamento;
                    \item Premi::Model::Inserimento::Inserter <- invoca la classe concreta del  template per l’inserimento di un elemento.
				\end{itemize}	
                \textbf{\interfacce}: Viene utilizzata per gestire i Signal riguardanti l’inserimento di un nuovo elemento ed invocare i corretti metodi del Model;\\
                \textbf{\base}: 
                    \begin{itemize}
                    \item Premi::Model::Command::AbstractCommand.
                    \end{itemize}
                    }
     \subsubsubsection{Premi::Model::Command::ConcreteRemoveCommand}{
				\textbf{\tipo}: È classe concreta del Design Pattern Command, rappresenta un comando per rimuovere un elemento dall’oggetto SlideShow.\\	
				\textbf{\relaz}: 
				\begin{itemize}
					\item Premi::Controller::Presentazione::EditController -> invoca il costruttore della classe e passa l’oggetto così creato all’Invoker;
Premi::Model::Invoker -> esegue il comando o ne invoca il metodo di annullamento;
                    \item Premi::Model::Eliminazione::Remover <- invoca la classe concreta del  template per l’eliminazione di un elemento.
				\end{itemize}	
                \textbf{\interfacce}: Viene utilizzata per gestire i Signal riguardanti l’eliminazione di un elemento ed invocare i corretti metodi del Model.\\
                \textbf{\base}: 
                    \begin{itemize}
                    \item Premi::Model::Command::AbstractCommand.
                    \end{itemize}
                    }
                        \subsubsubsection{Premi::Model::Command::ConcreteEditCommand}{
				\textbf{\tipo}: È classe concreta del Design Pattern Command, rappresenta un comando per modificare un elemento elemento nell’oggetto SlideShow.\\	
				\textbf{\relaz}: 
				\begin{itemize}
					\item Premi::Controller::Presentazione::EditController -> invoca il costruttore della classe e passa l’oggetto così creato all’Invoker;
Premi::Model::Invoker -> esegue il comando o ne invoca il metodo di annullamento;
                    \item Premi::Model::Modifica::Editor <- invoca la classe concreta del design pattern Strategy per la modifica di un elemento.
				\end{itemize}	
                \textbf{\interfacce}: Viene utilizzata per gestire i Signal riguardanti la modifica di un nuovo elemento ed invocare i corretti metodi del Model;\\
                \textbf{\base}: 
                    \begin{itemize}
                    \item Premi::Model::Command::AbstractCommand.
                    \end{itemize}
                    }
}
	