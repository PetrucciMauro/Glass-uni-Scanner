\subsection{Model}{
	\textbf{\tipo}: è la parte Model dell'architettura MVC.\\
	\textbf{\relaz}: ??????????????????????.\\
	\textbf{Package contenuti}: 
	\begin{itemize}
	\item Premi::Model::Inserimento;
    \item Premi::Model::Rimozione;
    \item Premi::Model::Modifica;
    \item Premi::Model::Command;
    \item Premi::Model::Invoker;
    \item Premi::Model::Builder;
    \item Premi::Model::Presentazione;
    \item Premi::Model::MongoHandler.
	\end{itemize}
	\subsubsection{Premi::Model::Inserimento}{
		\textbf{\tipo}: All’interno di questo Package viene implementato il Design Pattern template per l’inserimento di nuovi elementi nella presentazione.\\
		\textbf{\relaz}:. Il package è in relazione con Premi::Model::Command da cui riceve i segnali e i parametri di inserimento dell’elemento. Inoltre comunica con il package Premi::Model::Presentazione, istanziando gli oggetti delle sottoclassi di SlideShowElement e inserendoli in SlideShow.\\
	
	\subsubsubsection{Premi::Model::Inserimento::Inserter}{
		\textbf{\tipo}: Classe astratta definita per l’implementazione del Design Pattern template, per l’inserimento di elementi all’interno di una presentazione.\\	
		\textbf{\relaz}:
		\begin{itemize}
			\item Premi::Model::Command::ConcreteConcreteInsertCommand -> utilizza i metodi messi a disposizione da Inserter e concretizzati dalle sue sottoclassi che a loro volta invocano le funzioni della classe Premi::Model::Presentazione::SlideShow per l’impostazione dei campi relativi. 
		\end{itemize} 
		\textbf{\interfacce}: Definisce le operazioni primitive astratte che le classi concrete sottostanti andranno a sovraccaricare e implementa il metodo template che rappresenta lo scheletro dell'algoritmo per l’inserimento di un elemento nella presentazione.
È il componente receiver del Design Pattern Command.\\
        \textbf{\figli}: 
        \begin{itemize}
            \item Premi::Model::Inserimento::ConcreteTextInserter;
            \item Premi::Model::Inserimento::ConcreteFrameInserter;
            \item Premi::Model::Inserimento::ConcreteSvgInserter;
            \item Premi::Model::Inserimento::ConcreteImageInserter;
            \item Premi::Model::Inserimento::ConcreteVideoInserter;
            \item Premi::Model::Inserimento::ConcreteAudioInserter.
            \item Premi::Model::Eliminazione::ConcreteBackgroundInserter.
        \end{itemize}
	}
	\subsubsubsection{Premi::Controller::Presentazione::Inserimento::ConcreteTextInserter}{
				\textbf{\tipo}: Classe che rappresenta un algoritmo di creazione e inserimento di un elemento testuale all’interno di una presentazione. È uno dei componenti concreti del Design Pattern Template.\\	
				\textbf{\relaz}: 
				\begin{itemize}
					\item Premi::Model::Presentazione::Text <- costruisce un oggetto di classe Text.
					\item Premi::Model::Command::ConcreteInsertCommand -> invoca i metodi per inserire un nuovo elemento di tipo testo nella presentazione.
					
				\end{itemize} 
				\textbf{\interfacce}: Viene invocato per inserire elementi testuali in una presentazione.\\
                \textbf{\base}: 
                    \begin{itemize}
                    \item Premi::Model::Inserimento::Inserter.
                    \end{itemize}
			}
    \subsubsubsection{Premi::Model::Inserimento::ConcreteFrameInserter}{
				\textbf{\tipo}: Classe che rappresenta un algoritmo di inserimento di un elemento frame all’interno di una presentazione.
È uno dei componenti concreti del Design Pattern Template.\\	
				\textbf{\relaz}: 
				\begin{itemize}
				\item Premi::Model::Presentazione::Frame <- costruisce un oggetto di classe Frame.
					\item Premi::Model::Command::ConcreteInsertCommand -> invoca i metodi per inserire un nuovo elemento di tipo Frame nella presentazione.
				\end{itemize} 
				\textbf{\interfacce}: Viene invocato per inserire elementi di tipo frame in una presentazione.\\
                \textbf{\base}: 
                    \begin{itemize}
                    \item Premi::Model::Inserimento::Inserter.
                    \end{itemize}
			}
       \subsubsubsection{Premi::Model::Inserimento::ConcreteSvgInserter}{
				\textbf{\tipo}: Classe che rappresenta un algoritmo di inserimento di un elemento svg all’interno di una presentazione.
È uno dei componenti concreti del Design Pattern Template.\\	
				\textbf{\relaz}: 
				\begin{itemize}
				\item Premi::Model::Presentazione::SVG <- costruisce un oggetto di classe SVG.
					\item Premi::Model::Command::ConcreteInsertCommand -> invoca i metodi per inserire un nuovo elemento di tipo SVG nella presentazione.
				\end{itemize} 
				\textbf{\interfacce}: Viene invocato per inserire elementi svg in una presentazione.\\
                \textbf{\base}: 
                    \begin{itemize}
                    \item Premi::Model::Inserimento::Inserter.
                    \end{itemize}
			}
       \subsubsubsection{Premi::Model::Inserimento::ConcreteImageInserter}{
				\textbf{\tipo}: Classe che rappresenta un algoritmo di inserimento di un elemento immagine all’interno di una presentazione.
È uno dei componenti concreti del Design Pattern Template.\\	
				\textbf{\relaz}: 
				\begin{itemize}
				\item Premi::Model::Presentazione::Image <- costruisce un oggetto di classe Image.
					\item Premi::Model::Command::ConcreteInsertCommand -> invoca i metodi per inserire un nuovo elemento di tipo immagine nella presentazione.
				\end{itemize} 
				\textbf{\interfacce}: Viene invocato per inserire elementi di tipo immagine in una presentazione.\\
                \textbf{\base}: 
                    \begin{itemize}
                    \item Premi::Model::Inserimento::Inserter.
                    \end{itemize}
			}
            \subsubsubsection{Premi::Model::Inserimento:: ConcreteVideoInserter}{

				\textbf{\tipo}: Classe che rappresenta un algoritmo di inserimento di un elemento video all’interno di una presentazione.
È uno dei componenti concreti del Design Pattern Template.\\	
				\textbf{\relaz}: 
				\begin{itemize}
            		\item Premi::Model::Presentazione::Video <- costruisce un oggetto di classe Video.
					\item Premi::Model::Command::ConcreteInsertCommand -> invoca i metodi per inserire un nuovo elemento di tipo video nella presentazione.
				\end{itemize} 
				\textbf{\interfacce}:Viene invocato per inserire elementi di tipo video in una presentazione.\\
                \textbf{\base}: 
                    \begin{itemize}
                    \item Premi::Model::Inserimento::Inserter.
                    \end{itemize}
			}
            \subsubsubsection{Premi::Model::Inserimento:: ConcreteAudioInserter}{
				\textbf{\tipo}: Classe che rappresenta un algoritmo di inserimento di un elemento di tipo audio all’interno di una presentazione.
È uno dei componenti concreti del Design Pattern Template.\\	
				\textbf{\relaz}: 
				\begin{itemize}
				\item Premi::Model::Presentazione::Audio <- costruisce un oggetto di classe Audio.
					\item Premi::Model::Command::ConcreteInsertCommand -> invoca i metodi per inserire un nuovo elemento di tipo audio nella presentazione.
				\end{itemize} 
				\textbf{\interfacce}: Viene invocato per inserire elementi di tipo audio in una presentazione.\\
                \textbf{\base}: 
                    \begin{itemize}
                    \item Premi::Model::Inserimento::Inserter.
                    \end{itemize}
			}
			\subsubsubsection{Premi::Controller::Presentazione::Inserimento::ConcreteBackgroundInserter}{
							\textbf{\tipo}: Classe che rappresenta un algoritmo di creazione e inserimento di un elemento di classe Background in una Presentazione. È uno dei componenti concreti del Design Pattern Template.\\
			                \textbf{\relaz}: 
							\begin{itemize}
								\item Premi::Model::Presentazione::Background <- invoca il metodo getInstance di classe Background.
								\item Premi::Model::Command::ConcreteInsertCommand -> invoca i metodi per inserire un nuovo sfondo nella presentazione.
								
							\end{itemize} 
							\textbf{\interfacce}: Viene invocato per inserire elementi sfondo in una presentazione.\\
			                \textbf{\base}: 
			                    \begin{itemize}
			                    \item Premi::Model::Inserimento::Inserter.
			                    \end{itemize}
						}
	}
    
    	\subsubsection{Premi::Model::Eliminazione}{
		\textbf{\tipo}: All’interno di questo Package viene implementato il Design Pattern template per l’eliminazione di elementi dalla presentazione.\\
		\textbf{\relaz}:Il package è in relazione con Premi::Model::Command da cui riceve i segnali e i parametri di eliminazione dell’elemento. Inoltre comunica con il package Premi::Model::Presentazione, rimuovendo dall’oggetto di classe SlideShow gli oggetti delle sottoclassi di SlideShowElement e distruggendoli.\\
	
	\subsubsubsection{Premi::Model::Eliminazione::Remover}{
		\textbf{\tipo}: Classe astratta definita per l’implementazione del Design Pattern template, per l’eliminazione di elementi all’interno di una presentazione.\\	
		\textbf{\relaz}:
		\begin{itemize}
			\item Premi::Model::Command::ConcreteConcreteRemoveCommand -> utilizza i metodi messi a disposizione da Remover e concretizzati dalle sue sottoclassi che a loro volta invocano le funzioni della classe.
		\end{itemize} 
		\textbf{\interfacce}: Definisce le operazioni primitive astratte che le classi concrete sottostanti andranno a sovraccaricare o definire e implementa il metodo template che rappresenta lo scheletro dell'algoritmo per l’eliminazione di un elemento nella presentazione.\\
È il componente receiver del Design Pattern Command.\\
        \textbf{\figli}: 
        \begin{itemize}
            \item Premi::Model::Eliminazione::ConcreteTextRemover;
            \item Premi::Model::Eliminazione::ConcreteFrameRemover;
            \item Premi::Model::Eliminazione::ConcreteSvgRemover;
            \item Premi::Model::Eliminazione::ConcreteImageRemover;
            \item Premi::Model::Eliminazione::ConcreteVideoRemover;
            \item Premi::Model::Eliminazione::ConcreteAudioRemover;
            \item Premi::Model::Eliminazione::ConcreteBackgroundRemover.
        \end{itemize}
	}
	\subsubsubsection{Premi::Model::Eliminazione:: ConcreteTextRemover}{
				\textbf{\tipo}: Classe che implementa un algoritmo di eliminazione di un elemento testuale all’interno di una presentazione.
È uno dei componenti concreti del Design Pattern Template.\\	
				\textbf{\relaz}: 
				\begin{itemize}
					\item Premi::Model::Command::ConcreteRemoveCommand -> invoca i metodi per eliminare un elemento di tipo testo dalla presentazione.
				\end{itemize} 
				\textbf{\interfacce}: Viene invocato per eliminare elementi testuali in una presentazione.\\
                \textbf{\base}: 
                    \begin{itemize}
                    \item Premi::Model::Eliminazione::Remover.
                    \end{itemize}
			}
    \subsubsubsection{Premi::Model::Eliminazione:: ConcreteFrameRemover}{
				\textbf{\tipo}: Classe che implementa un algoritmo di eliminazione di un elemento di tipo frame all’interno di una presentazione.
È uno dei componenti concreti del Design Pattern Template.\\	
				\textbf{\relaz}: 
				\begin{itemize}
					\item Premi::Model::Command::ConcreteRemoveCommand -> invoca i metodi per eliminare un elemento di tipo frame dalla presentazione.
				\end{itemize} 
				\textbf{\interfacce}: Viene invocato per eliminare elementi di tipo frame da una presentazione.\\
                \textbf{\base}: 
                    \begin{itemize}
                    \item Premi::Model::Eliminazione::Remover.
                    \end{itemize}
			}
       \subsubsubsection{Premi::Model::Eliminazione:: ConcreteSVGtRemover}{
				\textbf{\tipo}: Classe che implementa un algoritmo di eliminazione di un elemento di tipo SVG all’interno di una presentazione.
È uno dei componenti concreti del Design Pattern Template.\\	
				\textbf{\relaz}: 
				\begin{itemize}
					\item Premi::Model::Command::ConcreteRemoveCommand -> invoca i metodi per eliminare un elemento di tipo SVG dalla presentazione.
				\end{itemize} 
				\textbf{\interfacce}: Viene invocato per eliminare elementi SVG da una presentazione.\\
                \textbf{\base}: 
                    \begin{itemize}
                    \item Premi::Model::Eliminazione::Remover.
                    \end{itemize}
			}
       \subsubsubsection{Premi::Model::Eliminazione:: ConcreteImageRemover}{
				\textbf{\tipo}: Classe che implementa un algoritmo di eliminazione di un elemento immagine all’interno di una presentazione.
È uno dei componenti concreti del Design Pattern Template.\\	
				\textbf{\relaz}: 
				\begin{itemize}
					\item Premi::Model::Command::ConcreteRemoveCommand -> invoca i metodi per eliminare un elemento di tipo immagine dalla presentazione.
				\end{itemize} 
				\textbf{\interfacce}: Viene invocato per eliminare elementi immagine in una presentazione.\\
                \textbf{\base}: 
                    \begin{itemize}
                    \item Premi::Model::Eliminazione::Remover.
                    \end{itemize}
			}
            \subsubsubsection{Premi::Model::Eliminazione:: ConcreteVideoRemover}{
				\textbf{\tipo}: Classe che implementa un algoritmo di eliminazione di un elemento di tipo video all’interno di una presentazione.
È uno dei componenti concreti del Design Pattern Template.\\	
				\textbf{\relaz}: 
				\begin{itemize}
					\item Premi::Model::Command::ConcreteRemoveCommand -> invoca i metodi per eliminare un elemento di tipo video dalla presentazione.
				\end{itemize} 
				\textbf{\interfacce}: Viene invocato per eliminare elementi di tipo video da una presentazione.\\
                \textbf{\base}: 
                    \begin{itemize}
                    \item Premi::Model::Eliminazione::Remover.
                    \end{itemize}
			}
            \subsubsubsection{Premi::Model::Eliminazione:: ConcreteAudioRemover}{
				\textbf{\tipo}: Classe che implementa un algoritmo di eliminazione di un elemento di tipo audio all’interno di una presentazione.
È uno dei componenti concreti del Design Pattern Template.\\	
				\textbf{\relaz}: 
				\begin{itemize}
					\item Premi::Model::Command::ConcreteRemoveCommand -> invoca i metodi per eliminare un elemento di tipo audio dalla presentazione.
				\end{itemize} 
				\textbf{\interfacce}: Viene invocato per eliminare elementi di tipo audio da una presentazione.\\
                \textbf{\base}: 
                    \begin{itemize}
                    \item Premi::Model::Eliminazione::Remover.
                    \end{itemize}
			}
			 \subsubsubsection{Premi::Model::Eliminazione:: ConcreteBackgroundRemover}{
							\textbf{\tipo}: Classe che implementa un algoritmo di eliminazione dello sfondo dellaa presentazione.
			È uno dei componenti concreti del Design Pattern Template.\\	
							\textbf{\relaz}: 
							\begin{itemize}
								\item Premi::Model::Command::ConcreteRemoveCommand -> invoca i metodi per eliminare lo sfondo dalla presentazione.
							\end{itemize} 
							\textbf{\interfacce}: Viene invocato per eliminare lo sfondo dlla presentazione.\\
			                \textbf{\base}: 
			                    \begin{itemize}
			                    \item Premi::Model::Eliminazione::Remover.
			                    \end{itemize}
						}
	}
   \subsubsection{Premi::Model::Modifica}{
		\textbf{\tipo}: All’interno di questo Package viene implementato il Design Pattern strategy per la modifica di elementi della presentazione.\\
		\textbf{\relaz}:Il package è in relazione con Premi::Model::Command da cui riceve i segnali e i parametri di modifica dell’elemento. Inoltre comunica con il package Premi::Model::Presentazione, modificando nell’oggetto di classe SlideShow gli oggetti delle sottoclassi di SlideShowElement.\\
	
	\subsubsubsection{Premi::Model::Modifica::Editor}{
		\textbf{\tipo}: Interfaccia per la componente strategy del Design Pattern Strategy per la selezione dell'algoritmo di modifica della presentazione.\\	
		\textbf{\relaz}:
		\begin{itemize}
			\item Premi::Model::Command::ConcreteEditCommand -> Invoca i costruttori delle sottoclassi di Editor; 
            \item Premi::Model::Presentazione::SlideShow <- scorre gli elementi dei membri contenitori all’interno di SlideShow per trovare l’elemento da modificare;
            \item Premi::Model::Presentazione::SlideShowElement <- invoca le funzioni della classe SlideShowElement per modificare opportunamente i campi dell’elemento.
		\end{itemize} 
		\textbf{\interfacce}: Permette di selezionare dinamicamente ed in modo estensibile l'algoritmo di modifica della presentazione.\\
        \textbf{\figli}: 
        \begin{itemize}
            \item Premi::Model::Modifica::Editor::EditorPosition;
            \item Premi::Model::Modifica::Editor::EditorSize;
            \item Premi::Model::Modifica::Editor::EditorContent;
            \item Premi::Model::Modifica::Editor::EditorRotate;
            \item Premi::Model::Modifica::Editor::EditorColor;
            \item Premi::Model::Modifica::Editor::EditorShape.
        \end{itemize}
	}
	\subsubsubsection{Premi::Model::Modifica::EditorPosition}{
				\textbf{\tipo}: Classe concreta del Design Pattern Strategy per la modifica dei campi inerenti alla posizione di un elemento della presentazione.\\	
				\textbf{\relaz}: 
				\begin{itemize}
					\item Premi::Model::Command::ConcreteEditCommand -> Invoca il costruttore di EditorPosition;
                    \item Premi::Model::Presentazione::SlideShow<- scorre gli elementi dei membri contenitori all’interno di SlideShow per trovare l’elemento da modificare; 
                    \item Premi::Model::Presentazione::SlideShowElement <- invoca le funzioni della classe SlideShowElement per modificare opportunamente i campi relativi alla posizione dell’elemento.
				\end{itemize}	\textbf{\interfacce}:Premi::Model::Command::ConcreteEditCommand invoca il costruttore e la funzione di esecuzione dell’operazione di modifica, EditorPosition invocherà quindi i metodi di modifica delle coordinate forniti all’interno della sottoclasse di Premi::Model::Presentazione::SlideShowElement di cui fa parte l’oggetto da modificare.\\
                \textbf{\base}: 
                    \begin{itemize}
                    \item Premi::Model::Modifica::Editor.
                    \end{itemize}
                    }
    \subsubsubsection{Premi::Model::Modifica::EditorSize}{
				\textbf{\tipo}: Classe concreta del Design Pattern Strategy per la modifica dei campi inerenti alla dimensione di un elemento della presentazione.\\	
				\textbf{\relaz}: 
				\begin{itemize}
					\item Premi::Model::Command::ConcreteEditCommand -> Invoca il costruttore di EditorSize;
                    \item Premi::Model::Presentazione::SlideShow<- scorre gli elementi dei membri contenitori all’interno di SlideShow per trovare l’elemento da modificare; 
                    \item Premi::Model::Presentazione::SlideShowElement <- invoca le funzioni della classe SlideShowElement per modificare opportunamente i campi relativi alla dimensione dell’elemento.
				\end{itemize}	\textbf{\interfacce}:Premi::Model::Command::ConcreteEditCommand invoca il costruttore e la funzione di esecuzione dell’operazione di modifica, EditorSize invocherà quindi i metodi di modifica delle dimensioni forniti all’interno della sottoclasse di Premi::Model::Presentazione::SlideShowElement di cui fa parte l’oggetto da modificare.\\
                \textbf{\base}: 
                    \begin{itemize}
                    \item Premi::Model::Modifica::Editor.
                    \end{itemize}
                    }
       \subsubsubsection{Premi::Model::Modifica::EditorRotate}{
				\textbf{\tipo}: Classe concreta del Design Pattern Strategy per la modifica dei campi inerenti all'inclinazione di un elemento della presentazione.\\	
				\textbf{\relaz}: 
				\begin{itemize}
					\item Premi::Model::Command::ConcreteEditCommand -> Invoca il costruttore di EditorRotate;
                    \item Premi::Model::Presentazione::SlideShow<- scorre gli elementi dei membri contenitori all’interno di SlideShow per trovare l’elemento da modificare; 
                    \item Premi::Model::Presentazione::SlideShowElement <- invoca le funzioni della classe SlideShowElement per modificare opportunamente i campi relativi all'inclinazione dell’elemento.
				\end{itemize}	\textbf{\interfacce}:Premi::Model::Command::ConcreteEditCommand invoca il costruttore e la funzione di esecuzione dell’operazione di modifica, EditorPosition invocherà quindi i metodi di modifica dell'inclinazione forniti all’interno della sottoclasse di Premi::Model::Presentazione::SlideShowElement di cui fa parte l’oggetto da modificare.\\
                \textbf{\base}: 
                    \begin{itemize}
                    \item Premi::Model::Modifica::Editor.
                    \end{itemize}
                    }
       \subsubsubsection{Premi::Model::Modifica::EditorContent}{
				\textbf{\tipo}: Classe concreta del Design Pattern Strategy per la modifica dei campi inerenti al contenuto di un elemento di tipo testuale della presentazione.\\	
				\textbf{\relaz}: 
				\begin{itemize}
					\item Premi::Model::Command::ConcreteEditCommand -> Invoca il costruttore di EditorContent;
                    \item Premi::Model::Presentazione::SlideShow<- scorre gli elementi del membro contenitore all’interno di SlideShow per trovare l’elemento testuale da modificare; 
                    \item Premi::Model::Presentazione::SlideShowElement <- invoca le funzioni della classe SlideShowElement per modificare opportunamente i campi relativi alla contenuto dell’elemento testuale.
				\end{itemize}	\textbf{\interfacce}:Premi::Model::Command::ConcreteEditCommand invoca il costruttore e la funzione di esecuzione dell’operazione di modifica, EditorContent invocherà quindi i metodi di modifica del contenuto forniti all’interno della classe Premi::Model::Presentazione::Text.\\
                \textbf{\base}: 
                    \begin{itemize}
                    \item Premi::Model::Modifica::Editor.
                    \end{itemize}
                    }
            \subsubsubsection{Premi::Model::Modifica::EditorShape}{
				\textbf{\tipo}: Classe concreta del Design Pattern Strategy per la modifica dei campi inerenti alla forma di un elemento SVG della presentazione.\\	
				\textbf{\relaz}: 
				\begin{itemize}
					\item Premi::Model::Command::ConcreteEditCommand -> Invoca il costruttore di EditorShape;
                    \item Premi::Model::Presentazione::SlideShow<- scorre gli elementi dei membri contenitori all’interno di SlideShow per trovare l’elemento SVG da modificare; 
                    \item Premi::Model::Presentazione::SlideShowElement <- invoca le funzioni della classe SlideShowElement per modificare opportunamente i campi relativi alla forma dell’elemento SVG.
				\end{itemize}	\textbf{\interfacce}:Premi::Model::Command::ConcreteEditCommand invoca il costruttore e la funzione di esecuzione dell’operazione di modifica, EditorShape invocherà quindi i metodi di modifica della forma forniti all’interno della classe Premi::Model::Presentazione::SVG.\\
                \textbf{\base}: 
                    \begin{itemize}
                    \item Premi::Model::Modifica::Editor.
                    \end{itemize}
                    }
            \subsubsubsection{Premi::Model::Modifica::EditorColor}{
				\textbf{\tipo}: Classe concreta del Design Pattern Strategy per la modifica dei campi inerenti al colore di un elemento SVG della presentazione.\\	
				\textbf{\relaz}: 
				\begin{itemize}
					\item Premi::Model::Command::ConcreteEditCommand -> Invoca il costruttore di EditorColor;
                    \item Premi::Model::Presentazione::SlideShow<- scorre gli elementi del membro contenitore all’interno di SlideShow per trovare l’elemento SVG da modificare; 
                    \item Premi::Model::Presentazione::SlideShowElement <- invoca le funzioni della classe SlideShowElement per modificare opportunamente i campi relativi alla forma dell’elemento SVG.
				\end{itemize}	\textbf{\interfacce}:Premi::Model::Command::ConcreteEditCommand invoca il costruttore e la funzione di esecuzione dell’operazione di modifica, EditorShape invocherà quindi i metodi di modifica della forma forniti dalla classe Premi::Model::Presentazione::SVG.\\
                \textbf{\base}: 
                    \begin{itemize}
                    \item Premi::Model::Modifica::Editor.
                    \end{itemize}
                    }
   \subsubsection{Premi::Model::Command}{
		\textbf{\tipo}:All’interno di questo Package viene implementato il Design Pattern command, utile per la gestione di funzioni di annullamento e ripristino.\\
		\textbf{\relaz}:. All’interno del Model, il package è in relazione con Premi::Model::Inserimento, Premi::Model::Eliminazione e Premi::Model::Modifica. Il package comunica, inoltre, con il controller, infatti le sue classi sono generate da Premi::Controller::Presentazione::EditController.\\
	\subsubsubsection{Premi::Model::Invoker}{
		\textbf{\tipo}: È componente invoker del Design Pattern Command, il suo scopo è tenere traccia delle modifiche atomiche apportate alla presentazione (modifica di elemento, eliminazione di elemento e inserimento di elemento) per poter implementare le funzioni di annulla/ripristina.\\	
		\textbf{\relaz}:
		\begin{itemize}
			\item Premi::Controller::MobileEdit->crea un oggetto di una sottoclasse di Premi::Model::Command::AbstractCommand passandolo all’Invoker che lo esegue e lo inserisce nello stack “undo”, richiama il metodo che svuota lo stack “redo”.\\
			Può inoltre invocare il  metodo “unexecute” dell’Invoker che provvede a richiamare il metodo undo del comando sulla cima dello stack “undo” e a spostarlo quindi nello stack “redo”. Alternativamente invoca il  metodo “redo” dell’Invoker che provvede a eseguire il comando sulla cima dello stack “redo” e a spostarlo quindi nello stack “undo”;
			\item Premi::Controller::DesktopEdit->si comporta in modo analogo a MobileEdit;
			\item Premi::Model::Command::AbstractCommand <- Invoker invoca il metodo execute() dell'oggetto della sottoclasse di AbstractCommand. Alternativamente invoca il metodo undo().
		\end{itemize} 
		\textbf{\interfacce}: Viene invocato per effettuare le operazioni di modifica alla presentazione, a sua volta invoca una classe derivata da Premi::Model::Command per eseguire materialmente il comando. Quando un comando viene eseguito, Invoker lo salva in un array \$undo[ ], insieme ai parametri necessari a riportare la presentazione allo stato precedente.\\
	}
	\subsubsubsection{Premi::Model::Command::AbstractCommand}{
				\textbf{\tipo}: È interfaccia astratta del Design Pattern Command, è classe base per i comandi di modifica, inserimento ed eliminazione.\\	
				\textbf{\relaz}: 
				\begin{itemize}
                    \item Premi::Model:: Invoker -> esegue materialmente il comando, richiamandone i metodi di esecuzione; inoltre provvede ad annullare l’ultima operazione 
				\end{itemize}	
                \textbf{\interfacce}:Viene utilizzata per applicare un generico parametro di trasformazione ad un oggetto della presentazione, questo parametro verrà poi specificato dalle classi concrete.\\
                \textbf{\figli}: 
                    \begin{itemize}
                    \item Premi::Model::Command::ConcreteInsertCommand;
                    \item Premi::Model::Command::ConcreteRemoveCommand;
                    \item Premi::Model::Command::ConcreteEditCommand.
                    \end{itemize}
                    }
    \subsubsubsection{Premi::Model::Command::ConcreteInsertCommand}{
				\textbf{\tipo}: È classe concreta del Design Pattern Command, rappresenta un comando per inserire un nuovo elemento nell’oggetto presentazione.\\	
				\textbf{\relaz}: 
				\begin{itemize}
					\item Premi::Controller::Presentazione::EditController -> invoca il costruttore della classe e passa l’oggetto così creato all’Invoker;
Premi::Model::Invoker -> esegue il comando o ne invoca il metodo di annullamento;
                    \item Premi::Model::Inserimento::Inserter <- invoca la classe concreta del  template per l’inserimento di un elemento.
				\end{itemize}	
                \textbf{\interfacce}: Viene utilizzata per gestire i Signal riguardanti l’inserimento di un nuovo elemento ed invocare i corretti metodi del Model;\\
                \textbf{\base}: 
                    \begin{itemize}
                    \item Premi::Model::Command::AbstractCommand.
                    \end{itemize}
                    }
     \subsubsubsection{Premi::Model::Command::ConcreteRemoveCommand}{
				\textbf{\tipo}: È classe concreta del Design Pattern Command, rappresenta un comando per rimuovere un elemento dall’oggetto SlideShow.\\	
				\textbf{\relaz}: 
				\begin{itemize}
					\item Premi::Controller::Presentazione::EditController -> invoca il costruttore della classe e passa l’oggetto così creato all’Invoker;
Premi::Model::Invoker -> esegue il comando o ne invoca il metodo di annullamento;
                    \item Premi::Model::Eliminazione::Remover <- invoca la classe concreta del  template per l’eliminazione di un elemento.
				\end{itemize}	
                \textbf{\interfacce}: Viene utilizzata per gestire i Signal riguardanti l’eliminazione di un elemento ed invocare i corretti metodi del Model.\\
                \textbf{\base}: 
                    \begin{itemize}
                    \item Premi::Model::Command::AbstractCommand.
                    \end{itemize}
                    }
                        \subsubsubsection{Premi::Model::Command::ConcreteEditCommand}{
				\textbf{\tipo}: È classe concreta del Design Pattern Command, rappresenta un comando per modificare un elemento elemento nell’oggetto SlideShow.\\	
				\textbf{\relaz}: 
				\begin{itemize}
					\item Premi::Controller::Presentazione::EditController -> invoca il costruttore della classe e passa l’oggetto così creato all’Invoker;
Premi::Model::Invoker -> esegue il comando o ne invoca il metodo di annullamento;
                    \item Premi::Model::Modifica::Editor <- invoca la classe concreta del design pattern Strategy per la modifica di un elemento.
				\end{itemize}	
                \textbf{\interfacce}: Viene utilizzata per gestire i Signal riguardanti la modifica di un nuovo elemento ed invocare i corretti metodi del Model;\\
                \textbf{\base}: 
                    \begin{itemize}
                    \item Premi::Model::Command::AbstractCommand.
                    \end{itemize}
                    }
                    
                    
                    
                     \subsubsection{Premi::Model::Presentazione}{
		\textbf{\tipo}:Di questo package fanno parte le classi degli elementi della presentazione e la classe che definisce la presentazione stessa. Sì tratta del package centrale del software.\\
		\textbf{\relaz}:.Premi::Model::Presentazione è in comunicazione con 
        \begin{itemize}
        \item Premi::Model::Inserimento, i cui oggetti durante la modifica della presentazione istanziano oggetti di tipo SlideShowElement;
        \item Premi::Model::Eliminazione, i cui oggetti rimuovono da SlideShow gli oggetti di tipo SlideShowElement e li distruggono;
        \item Premi::Model::Modifica, i cui oggetti invocano metodi degli oggetti SlideShowElement che ne impostano i campi;
        \item Premi::Controller::Presentazione::EditController o Premi::Controller::Presentazione::ExecutionController invocano il costruttore di Loader;
  		\item al momento del caricamento della presentazione gli oggetti di Premi::Model::Presentazione::SlideShowElement sono invece costruiti dalle classi del package Premi::Model::Builder.\\
  		\end{itemize}

	\subsubsubsection{Premi::Model::Presentazione::SlideShow}{
				\textbf{\tipo}: Classe implementata con il Design Pattern Singleton, contiene tutte le impostazioni della presentazione caricata, gli oggetti in essa presenti e i metodi per settarli o inserirne di nuovi. Gli oggetti della presentazione si trovano all’interno di contenitori divisi per classe. Tramite un iteratore è possibile scorrere detti contenitori.\\	
				\textbf{\relaz}: 
				\begin{itemize}
                    \item Premi::Model::Presentazione::Loader -> invoca il metodo SlideShow::getInstance(), che a sua volta invoca il costruttore privato di SlideShow.
				\end{itemize}	
                \textbf{\interfacce}:Viene utilizzata dalla view tramite Premi::Controller::EditController e  Premi::Controller::ExecutionController per creare gli oggetti html sia in fase di esecuzione che in fase di modifica. È la classe principale del software.\\
                    }
    \subsubsubsection{Premi::Model::Presentazione::SlideShowElement}{
				\textbf{\tipo}: Gli oggetti della classe SlideShowElement rappresentano gli elementi della presentazione.\\	
				\textbf{\relaz}: 
				\begin{itemize}
					\item Premi::Model::Inserimento::Inserter-> invoca il costruttore delle sottoclassi di SlideShowElement e li inserisce nei membri contenitori all’interno di Premi::Model::Presentazione::SlideShow;
                    \item Premi::Model::Modifica:Editor -> gli oggetti delle sue sottoclassi richiamano le funzioni delle sottoclassi di SlideShowElement che gestiscono l’impostazione dei campi dati;
                    \item Premi::Model::Eliminazione::Remover -> gli oggetti delle sue sottoclassi rimuovono dai contenitori di SlideShow gli oggetti di classe SlideShowElement e ne richiamano i distruttori.
				\end{itemize}	
                \textbf{\interfacce}: Premi::Model::Inserimento::Inserter instanzia oggetti di sottoclassi di SlideShowElement e li inserisce nei membri contenitori all’interno di Premi::Model::Presentazione::SlideShow\\
                \textbf{\figli}: 
                    \begin{itemize}
                    \item Premi::Model::Presentazione:: Text;
                    \item Premi::Model::Presentazione:: Frame;
                    \item Premi::Model::Presentazione:: Image;
                    \item Premi::Model::Presentazione::SVG;
                    \item Premi::Model::Presentazione::Audio;
                    \item Premi::Model::Presentazione::Video;
                    \item Premi::Model::Presentazione::Background.
                    \end{itemize}
                    }
     \subsubsubsection{Premi::Model::Presentazione::Text}{
				\textbf{\tipo}: Gli oggetti della classe Text rappresentano gli elementi di tipo testuale della presentazione.\\
				\textbf{\relaz}: 
				\begin{itemize}
					\item Premi::Model::Inserimento::ConcreteTextInserter -> invoca il costruttore di Text e inserisce l’oggetto nel membro contenitore all’interno dell’oggetto della classe Premi::Model::Presentazione::SlideShow;
                    \item Premi::Model::Eliminazione::ConcreteTextRemover -> rimuove l’oggetto Text dal membro contenitore all’interno di Premi::Model::Presentazione::SlideShow, ne invoca quindi il distruttore;
                    \item Premi::Model::Modifica::ConcreteTextEditor -> invoca i metodi che impostano i campi dell’oggetto Text.
				\end{itemize}	
                \textbf{\interfacce}: Gli oggetti della classe Text vengono istanziati da Premi::Model::Inserimento::ConcreteTextInserter  o da Premi::Model::Builder::ConcreteTextBuilder e inseriti nei membri contenitori all’interno di Premi::Model::Presentazione::SlideShow.\\
                \textbf{\base}: 
                    \begin{itemize}
                    \item Premi::Model::Presentazione::SlideShowElement.
                    \end{itemize}
                    }
           \subsubsubsection{Premi::Model::Presentazione::Frame}{
				\textbf{\tipo}: Gli oggetti della classe Frame rappresentano gli elementi di tipo frame della presentazione.\\
				\textbf{\relaz}: 
				\begin{itemize}
					\item Premi::Model::Inserimento::ConcreteFrameInserter -> invoca il costruttore di Frame e inserisce l’oggetto nel membro contenitore all’interno dell’oggetto della classe Premi::Model::Presentazione::SlideShow;
                    \item Premi::Model::Eliminazione::ConcreteFrameRemover -> rimuove l’oggetto Frame dal membro contenitore all’interno di Premi::Model::Presentazione::SlideShow, ne invoca quindi il distruttore;
                    \item Premi::Model::Modifica::ConcreteFrameEditor -> invoca i metodi che impostano i campi dell’oggetto Frame.
				\end{itemize}	
                \textbf{\interfacce}: Gli oggetti della classe Frame vengono istanziati da Premi::Model::Inserimento::ConcreteFrameInserter  o da Premi::Model::Builder::ConcreteFrameBuilder e inseriti nei membri contenitori all’interno di Premi::Model::Presentazione::SlideShow.\\
                \textbf{\base}: 
                    \begin{itemize}
                    \item Premi::Model::Presentazione::SlideShowElement.
                    \end{itemize}
                    }
                    \subsubsubsection{Premi::Model::Presentazione::Image}{
				\textbf{\tipo}: Gli oggetti della classe Image rappresentano gli elementi di tipo immagine della presentazione.\\
				\textbf{\relaz}: 
				\begin{itemize}
					\item Premi::Model::Inserimento::ConcreteImageInserter -> invoca il costruttore di Image e inserisce l’oggetto nel membro contenitore all’interno dell’oggetto della classe Premi::Model::Presentazione::SlideShow;
                    \item Premi::Model::Eliminazione::ConcreteImageRemover -> rimuove l’oggetto Image dal membro contenitore all’interno di Premi::Model::Presentazione::SlideShow, ne invoca quindi il distruttore;
                    \item Premi::Model::Modifica::ConcreteFrameEditor -> invoca i metodi che impostano i campi dell’oggetto Image.
				\end{itemize}	
                \textbf{\interfacce}: Gli oggetti della classe Image vengono istanziati da Premi::Model::Inserimento::ConcreteImageInserter  o da Premi::Model::Builder::ConcreteImageBuilder e inseriti nei membri contenitori all’interno di Premi::Model::Presentazione::SlideShow.\\
                \textbf{\base}: 
                    \begin{itemize}
                    \item Premi::Model::Presentazione::SlideShowElement.
                    \end{itemize}
                    }
                    \subsubsubsection{Premi::Model::Presentazione::SVG}{
				\textbf{\tipo}: Gli oggetti della classe SVG rappresentano gli elementi di tipo SVG della presentazione.\\
				\textbf{\relaz}: 
				\begin{itemize}
					\item Premi::Model::Inserimento::ConcreteSVGInserter -> invoca il costruttore di SVG e inserisce l’oggetto nel membro contenitore all’interno dell’oggetto della classe Premi::Model::Presentazione::SlideShow;
                    \item Premi::Model::Eliminazione::ConcreteImageRemover -> rimuove l’oggetto SVG dal membro contenitore all’interno di Premi::Model::Presentazione::SlideShow, ne invoca quindi il distruttore;
                    \item Premi::Model::Modifica::ConcreteFrameEditor -> invoca i metodi che impostano i campi dell’oggetto SVG.
				\end{itemize}	
                \textbf{\interfacce}: Gli oggetti della classe SVG vengono istanziati da Premi::Model::Inserimento::ConcreteSVGInserter  o da Premi::Model::Builder::ConcreteSVGBuilder e inseriti nei membri contenitori all’interno di Premi::Model::Presentazione::SlideShow.\\
                \textbf{\base}: 
                    \begin{itemize}
                    \item Premi::Model::Presentazione::SlideShowElement.
                    \end{itemize}
                    }
                    \subsubsubsection{Premi::Model::Presentazione::Audio}{
				\textbf{\tipo}: Gli oggetti della classe Audio rappresentano gli elementi di tipo audio della presentazione.\\
				\textbf{\relaz}: 
				\begin{itemize}
					\item Premi::Model::Inserimento::ConcreteAudioInserter -> invoca il costruttore di Audio e inserisce l’oggetto nel membro contenitore all’interno dell’oggetto della classe Premi::Model::Presentazione::SlideShow;
                    \item Premi::Model::Eliminazione::ConcreteAudioRemover -> rimuove l’oggetto Audio dal membro contenitore all’interno di Premi::Model::Presentazione::SlideShow, ne invoca quindi il distruttore;
                    \item Premi::Model::Modifica::ConcreteFrameEditor -> invoca i metodi che impostano i campi dell’oggetto Audio.
				\end{itemize}	
                \textbf{\interfacce}: Gli oggetti della classe Audio vengono istanziati da Premi::Model::Inserimento::ConcreteAudioInserter  o da Premi::Model::Builder::ConcreteAudioBuilder e inseriti nei membri contenitori all’interno di Premi::Model::Presentazione::SlideShow.\\
                \textbf{\base}: 
                    \begin{itemize}
                    \item Premi::Model::Presentazione::SlideShowElement.
                    \end{itemize}
                    }
                    \subsubsubsection{Premi::Model::Presentazione::Video}{
				\textbf{\tipo}: Gli oggetti della classe Video rappresentano gli elementi di tipo video della presentazione.\\
				\textbf{\relaz}: 
				\begin{itemize}
					\item Premi::Model::Inserimento::ConcreteVideoInserter -> invoca il costruttore di Video e inserisce l’oggetto nel membro contenitore all’interno dell’oggetto della classe Premi::Model::Presentazione::SlideShow;
                    \item Premi::Model::Eliminazione::ConcreteVideoRemover -> rimuove l’oggetto Video dal membro contenitore all’interno di Premi::Model::Presentazione::SlideShow, ne invoca quindi il distruttore;
                    \item Premi::Model::Modifica::ConcreteFrameEditor -> invoca i metodi che impostano i campi dell’oggetto Video.
				\end{itemize}	
                \textbf{\interfacce}: Gli oggetti della classe Video vengono istanziati da Premi::Model::Inserimento::ConcreteVideoInserter  o da Premi::Model::Builder::ConcreteVideoBuilder e inseriti nei membri contenitori all’interno di Premi::Model::Presentazione::SlideShow.\\
                \textbf{\base}: 
                    \begin{itemize}
                    \item Premi::Model::Presentazione::SlideShowElement.
                    \end{itemize}
                    }     
                 \subsubsubsection{Premi::Model::Presentazione::Background}{
                				\textbf{\tipo}: Gli oggetti della classe Background rappresentano lo sfondo della presentazione.\\
                				\textbf{\relaz}: 
                				\begin{itemize}
                					\item Premi::Model::Inserimento::ConcreteBackgroundInserter -> invoca il costruttore di Background e inserisce l’oggetto nel membro contenitore all’interno dell’oggetto della classe Premi::Model::Presentazione::SlideShow;
                                    \item Premi::Model::Eliminazione::ConcreteBackgroundRemover -> rimuove l’oggetto Video dal membro contenitore all’interno di Premi::Model::Presentazione::SlideShow, ne invoca quindi il distruttore;
                                    \item 
                				\end{itemize}	
                                \textbf{\interfacce}: Gli oggetti della classe Background vengono istanziati da Premi::Model::Inserimento::ConcreteBacgroundInserter  o da Premi::Model::Builder::ConcreteBackgroundBuilder e inseriti nel membro contenitore all’interno di Premi::Model::Presentazione::SlideShow.\\
                                \textbf{\base}: 
                                    \begin{itemize}
                                    \item Premi::Model::Presentazione::SlideShowElement.
                                    \end{itemize}
                                    }              
}

 \subsubsection{Premi::Model::Builder}{
		\textbf{\tipo}:Il package Premi::Model::Builder implementa il Design Pattern builder. Il package ha come scopo principale la creazione degli oggetti delle sottoclassi di Premi::Model::Presentazione::SlideShowElement al momento del caricamento della presentazione nel programma.\\
		\textbf{\relaz}:È in relazione con il package Premi::Model::Presentazione che ne costruisce il Director, e con il package Premi::Model::Presentazione, delle cui classi, sottoclassi di SlideShowElement, costruisce gli oggetti.

	\subsubsubsection{Premi::Model::Builder::Director}{
				\textbf{\tipo}: Implementazione della parte Director del Design Pattern Builder. Fornisce a Premi::Model::Presentazione::Loader gli oggetti della presentazione.\\	
				\textbf{\relaz}: 
				\begin{itemize}
                    \item Premi::Model::Presentazione::Loader -> costruisce Director, passandogli i parametri per la costruizione degli oggetti delle sottoclassi di Premi::Model::Presentazione::SlideShowElement.
                    \item Premi::Model::Builder::AbstractBuilder <- invoca il costruttore degli oggetti delle sottoclassi di AbstractBuilder che genereranno i file delle sottoclassi di Premi::Model::Presentazione::SlideShowElement.
				\end{itemize}	
                \textbf{\interfacce}:Viene invocato da Premi::Model::Presentazione::Loader. È costruito con design pattern singleton. Restituisce gli oggetti di tipo Premi::Model::Presentazione::SlideShowElement a Premi::Model::Presentazione::Loader.\\
                    }
    \subsubsubsection{Premi::Model::Builder::AbstractBuilder}{
				\textbf{\tipo}: Classe astratta del Design Pattern Builder.\\	
                \textbf{\interfacce}: Gli oggetti delle sottoclassi della classe AbstractBuilder vengono istanziati da Premi::Model::Builder::Director e hanno lo scopo di istanziare gli oggetti delle sottoclassi di Premi::Model::Presentazione::SlideShowElement.\\
                \textbf{\figli}: 
                    \begin{itemize}
                    \item Premi::Model::Builder::ConcreteTextBuilder;
                    \item Premi::Model::Builder::ConcreteFrameBuilder;
                    \item Premi::Model::Builder::ConcreteImageBuilder;
                    \item Premi::Model::Builder::ConcreteSVGBuilder;
                    \item Premi::Model::Builder::ConcreteAudioBuilder;
                    \item Premi::Model::Builder::ConcreteVideoBuilder;
                    \item Premi::Model::Builder::ConcreteBackgroundBuilder.
                    \end{itemize}
                    }
     \subsubsubsection{Premi::Model::Builder::ConcreteTextBuilder}{
				\textbf{\tipo}: Classe concreta del Design Pattern Builder. Ha lo scopo di istanziare gli oggetti di classe Premi::Model::Presentazione::Text al momento del caricamento della presentazione e di passarli all’oggetto di classe Director.\\
				\textbf{\relaz}: 
				\begin{itemize}
					\item Premi::Model::Builder::Director -> istanzia gli oggetti passando i parametri ricevuti da Premi::Model::Presentazione::Loader;
                    \item Premi::Model::Presentazione::Text <- istanza gli oggetti della classe Text e li restituisce al director.
				\end{itemize}	
                \textbf{\interfacce}: Gli oggetti della classe ConcreteTextBuilder  vengono istanziati da Premi::Model::Builder::Director e hanno lo scopo di istanziare gli oggetti delle sottoclassi di Premi::Model::Presentazione::Text e di passarli al Director.\\
                \textbf{\base}: 
                    \begin{itemize}
                    \item Premi::Model::Builder::AbstractBuilder.
                    \end{itemize}
                    }
                         \subsubsubsection{Premi::Model::Builder::ConcreteFrameBuilder}{
				\textbf{\tipo}: Classe concreta del Design Pattern Builder. Ha lo scopo di istanziare gli oggetti di classe Premi::Model::Presentazione::Frame al momento del caricamento della presentazione e di passarli all’oggetto di classe Director.\\
				\textbf{\relaz}: 
				\begin{itemize}
					\item Premi::Model::Builder::Director -> istanzia gli oggetti passando i parametri ricevuti da Premi::Model::Presentazione::Loader;
                    \item Premi::Model::Presentazione::Frame <- istanza gli oggetti della classe Frame e li restituisce al director.
				\end{itemize}	
                \textbf{\interfacce}: Gli oggetti della classe ConcreteFrameBuilder  vengono istanziati da Premi::Model::Builder::Director e hanno lo scopo di istanziare gli oggetti delle sottoclassi di Premi::Model::Presentazione::Frame e di passarli al Director.\\
                \textbf{\base}: 
                    \begin{itemize}
                    \item Premi::Model::Builder::AbstractBuilder.
                    \end{itemize}
                    }
                         \subsubsubsection{Premi::Model::Builder::ConcreteImageBuilder}{
				\textbf{\tipo}: Classe concreta del Design Pattern Builder. Ha lo scopo di istanziare gli oggetti di classe Premi::Model::Presentazione::Image al momento del caricamento della presentazione e di passarli all’oggetto di classe Director.\\
				\textbf{\relaz}: 
				\begin{itemize}
					\item Premi::Model::Builder::Director -> istanzia gli oggetti passando i parametri ricevuti da Premi::Model::Presentazione::Loader;
                    \item Premi::Model::Presentazione::Image <- istanza gli oggetti della classe Image e li restituisce al director.
				\end{itemize}	
                \textbf{\interfacce}: Gli oggetti della classe ConcreteImageBuilder  vengono istanziati da Premi::Model::Builder::Director e hanno lo scopo di istanziare gli oggetti delle sottoclassi di Premi::Model::Presentazione::Image e di passarli al Director.\\
                \textbf{\base}: 
                    \begin{itemize}
                    \item Premi::Model::Builder::AbstractBuilder.
                    \end{itemize}
                    }
                         \subsubsubsection{Premi::Model::Builder::ConcreteSVGBuilder}{
				\textbf{\tipo}: Classe concreta del Design Pattern Builder. Ha lo scopo di istanziare gli oggetti di classe Premi::Model::Presentazione::SVG al momento del caricamento della presentazione e di passarli all’oggetto di classe Director.\\
				\textbf{\relaz}: 
				\begin{itemize}
					\item Premi::Model::Builder::Director -> istanzia gli oggetti passando i parametri ricevuti da Premi::Model::Presentazione::Loader;
                    \item Premi::Model::Presentazione::SVG <- istanza gli oggetti della classe SVG e li restituisce al director.
				\end{itemize}	
                \textbf{\interfacce}: Gli oggetti della classe ConcreteSVGBuilder  vengono istanziati da Premi::Model::Builder::Director e hanno lo scopo di istanziare gli oggetti delle sottoclassi di Premi::Model::Presentazione::SVG e di passarli al Director.\\
                \textbf{\base}: 
                    \begin{itemize}
                    \item Premi::Model::Builder::AbstractBuilder.
                    \end{itemize}
                    }
                         \subsubsubsection{Premi::Model::Builder::ConcreteAudioBuilder}{
				\textbf{\tipo}: Classe concreta del Design Pattern Builder. Ha lo scopo di istanziare gli oggetti di classe Premi::Model::Presentazione::Audio al momento del caricamento della presentazione e di passarli all’oggetto di classe Director.\\
				\textbf{\relaz}: 
				\begin{itemize}
					\item Premi::Model::Builder::Director -> istanzia gli oggetti passando i parametri ricevuti da Premi::Model::Presentazione::Loader;
                    \item Premi::Model::Presentazione::Audio <- istanza gli oggetti della classe Audio e li restituisce al director.
				\end{itemize}	
                \textbf{\interfacce}: Gli oggetti della classe ConcreteAudioBuilder  vengono istanziati da Premi::Model::Builder::Director e hanno lo scopo di istanziare gli oggetti delle sottoclassi di Premi::Model::Presentazione::Audio e di passarli al Director.\\
                \textbf{\base}: 
                    \begin{itemize}
                    \item Premi::Model::Builder::AbstractBuilder.
                    \end{itemize}
                    }
                         \subsubsubsection{Premi::Model::Builder::ConcreteVideoBuilder}{
				\textbf{\tipo}: Classe concreta del Design Pattern Builder. Ha lo scopo di istanziare gli oggetti di classe Premi::Model::Presentazione::Video al momento del caricamento della presentazione e di passarli all’oggetto di classe Director.\\
				\textbf{\relaz}: 
				\begin{itemize}
					\item Premi::Model::Builder::Director -> istanzia gli oggetti passando i parametri ricevuti da Premi::Model::Presentazione::Loader;
                    \item Premi::Model::Presentazione::Video <- istanza gli oggetti della classe Video e li restituisce al director.
				\end{itemize}	
                \textbf{\interfacce}: Gli oggetti della classe ConcreteVideoBuilder  vengono istanziati da Premi::Model::Builder::Director e hanno lo scopo di istanziare gli oggetti delle sottoclassi di Premi::Model::Presentazione::Video e di passarli al Director.\\
                \textbf{\base}: 
                    \begin{itemize}
                    \item Premi::Model::Builder::AbstractBuilder.
                    \end{itemize}
                    }
                                         \subsubsubsection{Premi::Model::Builder::ConcreteBackgroundBuilder}{
                    				\textbf{\tipo}: Classe concreta del Design Pattern Builder. Ha lo scopo di istanziare gli oggetti di classe Premi::Model::Presentazione::Background al momento del caricamento della presentazione e di passarli all’oggetto di classe Director.\\
                    				\textbf{\relaz}: 
                    				\begin{itemize}
                    					\item Premi::Model::Builder::Director -> istanzia l'oggetto passando i parametri ricevuti da Premi::Model::Presentazione::Loader;
                                        \item Premi::Model::Presentazione::Background <- istanza l'oggetto della classe Background e lo restituisce al director.
                    				\end{itemize}	
                                    \textbf{\interfacce}: Gli oggetti della classe ConcreteBackgroundBuilder  vengono istanziati da Premi::Model::Builder::Director e hanno lo scopo di istanziare gli oggetti delle sottoclassi di Premi::Model::Presentazione::Background e di passarli al Director.\\
                                    \textbf{\base}: 
                                        \begin{itemize}
                                        \item Premi::Model::Builder::AbstractBuilder.
                                        \end{itemize}
                                        }
}

\subsubsection{Premi::Model::Caricamento}{
		\textbf{\tipo}: All’interno di questo Package viene implementato il Design Pattern strategy per il caricamento di nuovi elementi nella presentazione.\\
		\textbf{\relaz}:. Il package è in relazione con Premi::Controller::MobileEdit, Premi::Controller::DesktopEdit e [[ControllerUtente]] dai quali riceve i segnali e i parametri di caricamento dell’elemento.\\
	
	\subsubsubsection{Premi::Model::Caricamento::Uploader}{
		\textbf{\tipo}: Classe astratta definita per l’implementazione del Design Pattern strategy, per il caricamento di elementi all’interno dello spazio personale di un utente.\\	
		\textbf{\relaz}:
		\begin{itemize}
			\item Premi::Controller::MobileEdit e Premi::Controller::DesktopEdit -> costruiscono un oggetto di una sottoclasse di Uploader e utilizzano i metodi da questi messi a disposizione per caricare il file nel server, inoltre costruiscono un oggetto della classe Premi::Model::Command::InsertCommand e lo danno in pasto Premi::Model::Command::Invoker;
            \item [[ControllerUtente]] -> costruisce un oggetto di una sottoclasse di Uploader e utilizza i metodi da questi messi a disposizione per caricare il file nel server.
		\end{itemize} 
		\textbf{\interfacce}: Definisce le operazioni primitive astratte che le classi concrete sottostanti andranno a sovraccaricare e implementa il metodo strategy che rappresenta lo scheletro dell'algoritmo per il caricamento di un elemento nella presentazione.\\
        \textbf{\figli}: 
        \begin{itemize}
            \item Premi::Model::Caricamento::ConcreteSvgUploader;
            \item Premi::Model::Caricamento::ConcreteImageUploader;
            \item Premi::Model::Caricamento::ConcreteVideoUploader;
            \item Premi::Model::Caricamento::ConcreteAudioUploader.
        \end{itemize}
	}
       \subsubsubsection{Premi::Model::Caricamento::ConcreteSvgUploader}{
				\textbf{\tipo}: Classe che rappresenta un algoritmo di caricamento di un elemento svg all’interno dello spazio personale di un utente.
È uno dei componenti concreti del Design Pattern Strategy.\\	
				\textbf{\relaz}: 
				\begin{itemize}
					\item Premi::Controller::MobileEdit e Premi::Controller::DesktopEdit -> costruiscono un oggetto di classe ConcreteSvgUploader e utilizzano i metodi da questo messi a disposizione per caricare il file nel server, inoltre costruiscono un oggetto della classe Premi::Model::Command::InsertCommand e lo danno in pasto Premi::Model::Command::Invoker;
                    \item [[ControllerUtente]] -> costruisce un oggetto di classe ConcreteSvgUploader e utilizza i metodi da questi messi a disposizione per caricare il file nel server.
				\end{itemize} 
				\textbf{\interfacce}: Viene invocato per caricare elementi SVG nello spazio personale di un utente ed eventualmente inserirlo automaticamente nello spazio personale di un utente ed eventualmente inserirlo automaticamente in una presentazione..\\
                \textbf{\base}: 
                    \begin{itemize}
                    \item Premi::Model::Caricamento::Uploader.
                    \end{itemize}
			}
       \subsubsubsection{Premi::Model::Caricamento::ConcreteImageUploader}{
				\textbf{\tipo}: Classe che rappresenta un algoritmo di caricamento di un elemento immagine all’interno dello spazio personale di un utente.
È uno dei componenti concreti del Design Pattern Strategy.\\	
				\textbf{\relaz}: 
				\begin{itemize}
					\item Premi::Controller::MobileEdit e Premi::Controller::DesktopEdit -> costruiscono un oggetto di classe ConcreteImageUploader e utilizzano i metodi da questo messi a disposizione per caricare il file nel server, inoltre costruiscono un oggetto della classe Premi::Model::Command::InsertCommand e lo danno in pasto Premi::Model::Command::Invoker;
                    \item [[ControllerUtente]] -> costruisce un oggetto di classe ConcreteImageUploader e utilizza i metodi da questi messi a disposizione per caricare il file nel server.
				\end{itemize} 
				\textbf{\interfacce}: Viene invocato per caricare elementi di tipo immagine nello spazio personale di un utente ed eventualmente inserirlo automaticamente in una presentazione..\\
                \textbf{\base}: 
                    \begin{itemize}
                    \item Premi::Model::Caricamento::Uploader.
                    \end{itemize}
			}
            \subsubsubsection{Premi::Model::Caricamento:: ConcreteVideoUploader}{
				\textbf{\tipo}: Classe che rappresenta un algoritmo di caricamento di un elemento video all’interno dello spazio personale di un utente.
È uno dei componenti concreti del Design Pattern Strategy.\\	
				\textbf{\relaz}: 
				\begin{itemize}
					\item Premi::Controller::MobileEdit e Premi::Controller::DesktopEdit -> costruiscono un oggetto di classe ConcreteVideoUploader e utilizzano i metodi da questo messi a disposizione per caricare il file nel server, inoltre costruiscono un oggetto della classe Premi::Model::Command::InsertCommand e lo danno in pasto Premi::Model::Command::Invoker;
                    \item [[ControllerUtente]] -> costruisce un oggetto di classe ConcreteVideoUploader e utilizza i metodi da questi messi a disposizione per caricare il file nel server.
				\end{itemize} 
				\textbf{\interfacce}:Viene invocato per caricare elementi di tipo video nello spazio personale di un utente ed eventualmente inserirlo automaticamente in una presentazione.\\
                \textbf{\base}: 
                    \begin{itemize}
                    \item Premi::Model::Caricamento::Uploader.
                    \end{itemize}
			}
            \subsubsubsection{Premi::Model::Caricamento:: ConcreteAudioUploader}{
				\textbf{\tipo}: Classe che rappresenta un algoritmo di caricamento di un elemento di tipo audio all’interno dello spazio personale di un utente.
È uno dei componenti concreti del Design Pattern Strategy.\\	
				\textbf{\relaz}: 
				\begin{itemize}
					\item Premi::Controller::MobileEdit e Premi::Controller::DesktopEdit -> costruiscono un oggetto di classe ConcreteAudioUploader e utilizzano i metodi da questo messi a disposizione per caricare il file nel server, inoltre costruiscono un oggetto della classe Premi::Model::Command::InsertCommand e lo danno in pasto Premi::Model::Command::Invoker;
                    \item [[ControllerUtente]] -> costruisce un oggetto di classe ConcreteImageUploader e utilizza i metodi da questi messi a disposizione per caricare il file nel server.
				\end{itemize} 
				\textbf{\interfacce}: Viene invocato per caricare elementi di tipo audio nello spazio personale di un utente ed eventualmente inserirlo automaticamente in una presentazione..\\
                \textbf{\base}: 
                    \begin{itemize}
                    \item Premi::Model::Caricamento::Uploader.
                    \end{itemize}
			}
	}
	