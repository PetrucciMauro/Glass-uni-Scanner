\section{Strumenti}
\subsection{HTML}
Si è deciso di utilizzare HTML5 e CSS3 per la presentazione grafica dell'applicazione web.\
Si è scelto di utilizzare HTML5 al posto di xHTML 1.1 perchè ormai è diventato uno standard de facto e permette una maggiore integrazione con i linguaggi di scripting e contenuti multimediali
di cui questa applicazione fa forte uso.

\begin{itemize}
\item \textbf{Vantaggi}: 
\begin{itemize}
\item \textbf{Multi piattaforma}: Poiché l'applicazione deve essere disponibile sia su dispositivi desktop che mobile HTML5 permette la creazione di strutture responsive in grado di adattarsi alle dimensioni dello schermo;
\item \textbf{Integrazione con linguaggi di scripting}: Con HTML5 c'è una maggiore integrazione con i linguaggi di scripting come javacript questo permetterà di rendere l'applicazione dinamica;
\item \textbf{Nessuna installazione}: Il fatto che l'applicazione sia sviluppata con tecnologie web quali HTML permetterà all'utente finale di poter utilizzare il prodotto senza doverlo scaricare e installare.
\end{itemize}
\item \textbf{Svantaggi}:
\begin{itemize}


\item \textbf{Browser}: E' possibile che i browser meno recenti abbiano difficoltà ad interpretare correttamente le informazioni contenute nelle pagine, rendendo difficile , se non impossibile, l'utilizzo dell'applicazione con questo linguaggio.
\end{itemize}
\end{itemize}
\subsection{JavaScript}

