\section{Strumenti}
	\subsection{HTML}{
		Si è deciso di utilizzare HTML5 e CSS3 per la presentazione grafica dell'applicazione web.\
		HTML5 è uno standard da settembre 2014 e permette una più semplice integrazione di contenuti multimediali.
	
		\begin{itemize}
			\item \textbf{Vantaggi}: 
			\begin{itemize}
				\item \textbf{Multi piattaforma}: Poiché l'applicazione deve essere disponibile sia su dispositivi desktop che mobile HTML5 permette la creazione di strutture responsive in grado di adattarsi alle dimensioni dello schermo;
				\item \textbf{Integrazione con linguaggi di scripting}: Con HTML5 c'è una maggiore integrazione con i linguaggi di scripting come javacript questo permetterà di rendere l'applicazione dinamica;
				\item \textbf{Nessuna installazione}: Il fatto che l'applicazione sia sviluppata con tecnologie web quali HTML permetterà all'utente finale di poter utilizzare il prodotto senza doverlo scaricare e installare.
			\end{itemize}
			\item \textbf{Svantaggi}:
			\begin{itemize}
				\item \textbf{Browser}: È possibile che i browser meno recenti abbiano difficoltà ad interpretare correttamente le informazioni contenute nelle pagine, rendendo difficile, se non impossibile, l'utilizzo dell'applicazione con questo linguaggio.
			\end{itemize}
		\end{itemize}
		}
	\subsection{JavaScript}{
		JavaScript è un linguaggio di scripting lato client orientato agli oggetti, comunemente usato nei siti web, ed interpretato dai browser. Ciò permette di alleggerire il server dal peso della computazione, che viene eseguita dal client. Essendo molto popolare e ormai consolidato, JavaScript può essere eseguito dalla maggior parte dei browser, sia desktop che mobile, grazie anche alla sua leggerezza. 		}
	\subsection{jQuery}{
		jQuery è una libreria Javascript cross-platform, disegnata per semplificare lo scripting di HTML lato-client. È la libreria Javascript più popolare al momento; è un software libero ed open-source. \\
		Il nucleo di jQuery è una libreria di manipolazione DOM (Document Object Model). DOM è una struttura ad albero che rappresenta tutti gli elementi di una pagina web e jQuery rende la ricerca, selezione e manipolazione di questi elementi DOM semplice e conveniente.
		I vantaggi nell'uso di jQuery sono l'incoraggiamento alla separazione di Javascript ed HTML, la brevità e la chiarezza, l'eliminazione di incompatibilità cross-browser, l'estendibilità.
	}
	\subsection{MEAN}{
		MEAN è uno stack di software Javascript, libero ed open source per costruire siti web dinamici ed applicazioni web. È una combinazione di MongoDB, Express.js ed Angular.js, eseguita su Node.js.
	}
	\subsubsection{MongoDB}{
		MongoDB è un database NoSQL open source orientato ai documenti, facilmente scalabile e ad alte prestazioni. Si allontana dalla struttura tradizionale basata su tabelle dei database relazionali, in favore di documenti in stile JSON con schema dinamico; questo rende l'integrazione di dati più semplice e facile in alcuni tipi d'applicazioni.
	}
	\subsubsection{Express.js}{
		Express.js è un framework per applicazioni web Node.js, disegnato per costruire applicazioni web single-page, multi-page o ibride.
		È costruito sopra il modulo Connect di Node.js e fa uso della sua architettura middleware; nel nostro sistema è utilizzato in particolar modo per la gestione dei path da cui sono  offerti i servizi per l'interfacciamento con il database Mongo.
	}
	\subsubsection{AngularJS}{
		AngularJS, è un framework per applicazioni web, open-source, manutetenuto da Google e da una comunità di sviluppatori e corporations. Mira a semplificare lo sviluppo ed il test di applicazioni single-page fornendo un framework per l'architettura model-view-whatever lato-client. \\
		Il framework AngularJS come prima cosa legge la pagina HTML, che ha al suo interno degli attributi tag personalizzati; Angular interpreta questi attributi come direttive per legare parti di input o di output della pagina ad un modello che è rappresentato da variabili Javascript standard. Il valore di queste variabili Javascript può essere imporstato manualmente all'interno del codice, oppure ricavato da risorse JSON statiche o dinamiche.
	}
	\subsubsection{Node.js}{
		Node.js è un'ambiente di esecuzione open source e cross-platform per applicazioni lato server; le applicazioni Node.js sono scritte in linguaggio Javascript. Node.js fornisce un'architettura scalabile orientata agli eventi grazie alla sua natura asincrona.
		Node.js usa il motore Javascript V8 di Google per eseguire codice, ed una larga percentuale dei moduli base è scritta in Javascript.
	}
	\subsection{Impress.js}{
		Impress.js è un framework open source che permette di visualizzare i tag div di una pagina html come passi di una presentazione. Si è deciso di affidare la visualizzazione della presentazione a questa libreria in quanto permette di conseguire quasi tutti i requisiti obbligatori relativi all’esecuzione senza dover scrivere ingenti quantità di codice aggiuntivo.
		Si è deciso inoltre di integrare nel framework alcune funzioni in modo da rispondere a tutti i requisiti obbligatori relativi all’esecuzione.
	}