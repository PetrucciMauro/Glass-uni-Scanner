\section{Design Pattern e Pattern Architetturali}{
	\subsection{MVP}{
			\begin{figure}[H]
				\centering
				\includegraphics[scale=0.6]{\imgs {MVP}.png}
				\label{fig:mvp}
				\caption{Model View Presenter}
			\end{figure}
			\begin{itemize}
				\item \textbf{Scopo dell’utilizzo}: è stato scelto il pattern MVP per separare la logica dell'applicazione dalla rappresentazione grafica;
				\item \textbf{Contesto d’utilizzo}: Il pattern MVP viene utilizzato per l'architettura generale dell'applicazione.
				Ogni modifica effettuata dall'utente sulla View viene inviata al Presenter che invoca i metodi delle classi presenti nel Model.
			\end{itemize}
		}	
	\subsection{Command}{
		\begin{figure}[H]
			\centering
			\includegraphics[scale=0.5]{\imgs {Command}.pdf}
			\label{fig:command}
			\caption{Command}
		\end{figure}
		\begin{itemize}
			\item \textbf{Descrizione}: viene utilizzato quando c'è la necessità di disaccoppiare l’invocazione di un comando dai suoi dettagli implementativi, separando colui che invoca il comando da colui che esegue l’operazione.
			
			Tale operazione viene realizzata attraverso questa catena: Client->Invocatore->Ricevitore
			
			Il Client non è tenuto a conoscere i dettagli del comando ma il suo compito è solo quello di chiamare il metodo dell’ Invocatore che si occuperà di intermediare l’operazione.
			L’Invocatore ha l’obiettivo di incapsulare, nascondere i dettagli della chiamata come nome del metodo e parametri.
			Il Ricevitore utilizza i parametri ricevuti per eseguire l’operazione
			\item \textbf{Scopo dell’utilizzo}: si è scelto di utilizzare il pattern Command perché poter accodare o mantenere uno storico delle operazioni e	gestire operazioni cancellabili;
			\item \textbf{Contesto d’utilizzo}: viene utilizzato in fase di modifica delle presentazioni.
		\end{itemize}
		\subsubsection{Premi::Model::SlideShow::SlideShowActions::Command}{
			Il package Premi::Model::SlideShow::SlideShowActions::Command implementa il pattern Command, tuttavia il client è esterno al package ed è individuabile nella classe\\ Premi::Presenter::Presentazione::Edit, che invoca il costruttore delle sottoclassi di\\ Premi::Model::SlideShow::SlideShowActions::Command::AbstractCommand e dà l'oggetto creato in pasto a Premi::Model::SlideShow::SlideShowActions::Command::Invoker, che rappresenta, appunto, la componente invoker del pattern e che mette l'oggetto della sottoclasse di AbstractCommand in un contenitore denominato undo, invoca quindi il metodo Invoker::execute() che a sua volta esegue concretamente il comando.\\
			Premi::Presenter::Presentazione::Edit può invocare il metodo unexecute() di Invoker che a sua volta invoca il metodo AbstractCommand::undoCommand() nell'ultimo oggetto inserito nel membro contenitore undo. Questo metodo esegue le operazioni necessarie per annullare tutte le modifiche apportate dal comando. Quindi Invoker toglie il comando dal contenitore undo e lo inserisce nel contenitore redo. Quando Premi::Presenter::Presentazione::Edit invoca il metodo Invoker::execute(), l'oggetto Invoker esegue il comando e lo sposta nuovamente dal membro contenitore redo e lo mette nel membro undo.    
		}
	}
	\subsection{Observer}{
		\begin{figure}[H]
			\centering
			\includegraphics[scale=0.6]{\imgs {Observer}.jpg}
			\label{fig:observer}
			\caption{Observer}
		\end{figure}
		\begin{itemize}
		\item \textbf{Descrizione}: L'Observer pattern è un design pattern utilizzato per tenere sotto controllo lo stato di diversi oggetti. Sostanzialmente il pattern si basa su uno o più oggetti, chiamati osservatori o listener, che vengono registrati per gestire un evento che potrebbe essere generato dall'oggetto "osservato".
		Oltre all'observer esiste il concrete Observer che si differenzia dal primo perché implementa direttamente le azioni da compiere in risposta ad un messaggio; riepilogando il primo è una classe astratta, il secondo no.
		
		\item \textbf{Scopo dell’utilizzo}: Nel nostro caso il pattern viene utilizzato per garantire update automatici alle presentazioni dell'utente presenti consistentemente nel database MongoDB attraverso chiamate ai servizi offerti dalla web-api nodeApi.
		\item \textbf{Contesto d’utilizzo}: viene utilizzato in fase di modifica delle presentazioni.
		\end{itemize}
		\subsubsection{Premi::Model::ServerRelations::dbConsinstency}{
			    Il package Premi::Model::ServerRelations::dbConsinstency contiene le classi che costituiscono il pattern observer, in particolare le classi ConcreteSubjects definiscono metodi getElement() che ritornano l'elemento presente nel campo presentazione di Model::ServerRelations::Loader::Costruttore che riferiscono. 
		}
		
		