\section{Design Pattern}{
	(da integrare con figure di applicabilità in premi e classi che utilizzano quei design pattern)
	\begin{figure}
	%	\centering
	%	\includegraphics[scale=0.6]{\imgs {fig1}.jpg} %inserire il diagramma UML
	%	\label{fig:fig1}
	%	\caption{Fig1}
	\end{figure}
	\subsection{MVC}{
			\begin{itemize}
				\item \textbf{Scopo dell’utilizzo}: è stato scelto il pattern MVC per separare la logica dell'applicazione dalla rappresentazione grafica;
				\item \textbf{Contesto d’utilizzo}: Il pattern MVC viene utilizzato per l'architettura generale dell'applicazione.//
				Ogni modifica effettuata dall'utente sulla View viene inviata al Controller che invoca i metodi delle classi presenti nel Model.
			\end{itemize}
		}
		
	\subsection{Singleton}{
		\begin{itemize}
			\item \textbf{Descrizione}: è un design pattern di tipo creazionale utilizzato per imporre ad alcune classi di avere solamente un'istanza. Questo design pattern garantisce la consistenza dello stato dell'applicazione che si andrà a sviluppare.\\
			Quando una classe implementa il pattern Singleton, la classe stessa si occupa di controllare il numero di istanze costruite (di solito tramite campi dati statici) e di fornire l'accesso ad esse tramite metodi appositi.\\
			\item \textbf{Scopo dell’utilizzo}: viene usato il pattern Singleton per le classi che devono avere un'unica istanza durante l'esecuzione dell'applicazione;
			\item \textbf{Contesto d’utilizzo}: le classi che devono avere un’unica istanza sono:
			\begin{itemize}
				\item Invoker;
				\item SlideShow.
			\end{itemize}
			\subsubsection{Premi::Model::Command::Invoker}{
				Premi::Controller::Inserimento::InsertController
			}
		\end{itemize}
	}
	\subsection{Utility}{
		\begin{itemize}
			\item ?????????????????????????????????.
		\end{itemize}
	}
	\subsection{Builder}{
		\begin{itemize}
			\item \textbf{Scopo dell’utilizzo}: viene usato il pattern Builder per separare la costruzione di un oggetto dalla sua rappresentazione e poter riusare il processo di costruzione per creare rappresentazioni differenti;
			\item \textbf{Contesto d’utilizzo}: viene utilizzato per il caricamento delle presentazioni.
		\end{itemize}
	}
	\subsection{Command}{
		\begin{itemize}
			\item \textbf{Scopo dell’utilizzo}: il pattern Command viene usato per separare il codice di un’azione dal codice che richiede l’esecuzione dello stesso;
			\item \textbf{Contesto d’utilizzo}: Viene utilizzato per il caricamento delle presentazioni.
		\end{itemize}
	}
	\subsection{Iterator}{
		\begin{itemize}
			\item \textbf{Scopo dell’utilizzo}: il pattern Iterator viene usato per fornire un accesso sequenziale agli elementi che formano un oggetto composto senza esporre all’esterno la struttura dell’oggetto;
			\item \textbf{Contesto d’utilizzo}: viene utilizzato per iterare sugli elementi.
		\end{itemize}
	}
	\subsection{Template Method}{
		\begin{itemize}
			\item \textbf{Scopo dell’utilizzo}: il pattern Template Method viene usato per definire la struttura di un algoritmo e lasciare alle sottoclassi la definizione di alcune parti usate;
			\item \textbf{Contesto d’utilizzo}: viene utilizzato per l’inserimento e la rimozione degli elementi.
		\end{itemize}
	}
	\subsection{Strategy}{
		\begin{itemize}
			\item \textbf{Scopo dell’utilizzo}: il pattern Strategy viene usato per isolare più algoritmi che svolgono la stessa funzione dal codice che esegue la funzione;
			\item \textbf{Contesto d’utilizzo}: viene utilizzato per la modifica degli elementi.
		\end{itemize}
	}
}