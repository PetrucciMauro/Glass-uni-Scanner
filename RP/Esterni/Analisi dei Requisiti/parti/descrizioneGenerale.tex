\section{Descrizione generale}{
Il sistema si pone come obbiettivo quello di permettere la creazione di presentazioni efficaci dal punto di vista dello storytelling anche ad utenti non esperti. \\
L'applicazione \premi\ permette all'utente di creare ed eseguire presentazioni personalizzate. Attraverso la creazione di frame e la loro modellazione, l'utente potrà definire un percorso di presentazione lineare oppure più percorsi che prevedono la possibilità di scegliere con quale continuare il flusso di esecuzione. Questo significa che il percorso\ped{g} di frame\ped{g} che sarà visualizzato sarà scelto dal presentatore in fase di visualizzazione.\\
L'applicazione è strutturata in modo gerarchico, ossia ogni frame ha almeno un padre (tranne la radice che può avere solamente figli). Questo permette di creare presentazioni strutturate a livelli, rendendo molto semplice la possibilità, durante l'esecuzione, di saltare determinati rami della presentazione. Inoltre, l'utente potrà inserire bookmark i quali permettono di saltare ad un frame padre in modo semplice e veloce.\\
Il sistema permetterà di creare e modificare presentazioni se connessi alla rete mentre l'utente potrà eseguire le proprie presentazione anche offline a patto di averle precedentemente scaricate dal server\ped{g}.
Il sistema sarà implementato utilizzando tecnologie WEB\ped{g} che lo renderanno altamente portabile.

\subsection{Funzioni\ped{g} del prodotto}{
	Il prodotto offre un'interfaccia web che permetterà di:
	\begin{itemize}
		\item Registrarsi, accedere al proprio account ed effettuare il logout;
		\item Gestire il proprio account;
		\item Creare una nuova presentazione da dispositivo desktop\ped{g};
		\item Modificare una presentazione da dispositivo desktop\ped{g};
		\item Modificare parzialmente la presentazione da dispositivo mobile\ped{g};
		\item Eseguire una presentazione salvata sul proprio account;
		\item Eseguire una presentazione locale;
		\item Creare infografiche\ped{g} a partire da una presentazione;
		\item Modificare infografiche\ped{g} create;
		\item Gestire il proprio archivio di file media;
		\item Scaricare una presentazione in locale.
	\end{itemize}
}
\subsection{Caratteristiche degli utenti}{
	Il prodotto si rivolge a qualsiasi tipo di utente interessato ad una facile creazione e modellazione di presentazioni ed infografiche. Non emergono quindi restrizioni particolari riguardo le caratteristiche dell'utenza.
}
\subsection{Vincoli generali}{
	Per poter utilizzare il software Premi è necessario disporre di un computer o di un dispositivo mobile con installato almeno uno dei principali browser\ped{g} quali Google Chrome, Mozilla Firefox e Safari scaricabili dai rispettivi siti ufficiali. Inoltre, il dispositivo deve supportare le seguenti tecnologie:
	\begin{itemize}
		\item HTML5;
		\item CSS3;
		\item Javascript.
	\end{itemize}
	Il prodotto non richiede particolari requisiti hardware anche se gli stessi possono influenzarne la velocità di esecuzione.
	}
}