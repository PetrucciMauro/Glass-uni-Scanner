\subsection{UC 0.7 - Gestione archivio utente sul server}{
	\label{uc0.7}
	\begin{figure}[H]
		\centering
		\includegraphics[scale=0.75]{\imgs {UC0.7}.jpg} %inserire il diagramma UML
		\label{fig:uc0.7}
		\caption{Caso d'uso 0.7: Gestione dell'archivio dati utente sul server}
	\end{figure}
	\textbf{Attori}: utente desktop. \\
	\textbf{Descrizione}: l'utente può modificare i contenuti (file media, presentazioni e infografiche) presenti nel proprio account, inserendone di nuovi, rinominando o eliminando quelli presenti. \\
	\textbf{Precondizione}: il sistema è attivo e funzionante; l'utente ha effettuato il login e desidera effettuare modifiche ai propri file nel proprio account.	\\
	\textbf{Postcondizione}: l'utente ha eseguito l'operazione voluta sul contenuto che desiderava: ha inserito un nuovo file media nel proprio spazio oppure ha rinominato una presentazione o una infografica altrimenti eliminato una presentazione, infografica o file media.	\\
	\textbf{Scenario principale}:
	\begin{enumerate}
		\item Inserimento file media \hyperref[uc0.7.1]{(UC 0.7.1)};
		\item Eliminazione file media \hyperref[uc0.7.2]{(UC 0.7.2)};
		\item Rinomina presentazione \hyperref[uc0.7.3]{(UC 0.7.3)};
		\item Rinomina infografica \hyperref[uc0.7.4]{(UC 0.7.4)};
		\item Eliminazione presentazioni \hyperref[uc0.7.5]{(UC 0.7.5)};
		\item Eliminazione infografica \hyperref[uc0.7.6]{(UC 0.7.6)}.
	\end{enumerate}
	}
\subsubsection{UC 0.7.1 - Inserimento file media}{
	\label{uc0.7.1}
	\textbf{Attori}: utente desktop. \\
	\textbf{Descrizione}: l'utente ha la possibilità di caricare un file media (audio, video, immagine) contenuto nel proprio computer o dispositivo. \\
	\textbf{Precondizione}: il sistema ha caricato lo spazio dedicato ai file media dell'utente e l'utente desidera caricare sul server un file media esistente in locale.	\\
	\textbf{Postcondizione}: l'utente ha caricato sul server i file media selezionati.	\\
	\textbf{Scenario principale}:
	\begin{enumerate}
		\item Navigazione nello spazio di lavoro;
		\item Selezione file media;
		\item Conferma e caricamento.
	\end{enumerate}
	\textbf{Scenari alternativi}: 
	\begin{itemize}
		\item L'operazione viene annullata e non si apportano cambiamenti.
	\end{itemize}
	}
	\subsubsection{UC 0.7.2 - Cancellazione file media}{
		\label{uc0.7.2}
		\textbf{Attori}: utente desktop.	\\
		\textbf{Descrizione}: l'utente ha la possibilità di eliminare file media tra quelli presenti all'interno del proprio account. \\
		\textbf{Precondizione}: il sistema ha caricato lo spazio dedicato ai file media dell'utente e l'utente desidera eliminare file media esistenti nel proprio spazio di lavoro sul server.	\\
		\textbf{Postcondizione}: l'utente ha cancellato i file media che desiderava rimuovere.	\\
		\textbf{Scenario principale}:
		\begin{enumerate}
			\item Navigazione nello spazio di lavoro;
			\item Selezione file media;
			\item Conferma selezione ed eliminazione.
		\end{enumerate}
		\textbf{Scenari alternativi}: 
		\begin{itemize}
			\item L'operazione viene annullata e non si apportano cambiamenti.
		\end{itemize}
		}
	\subsubsection{UC 0.7.3 - Rinomina presentazione}{
		\label{uc0.7.3}
		\textbf{Attori}: utente desktop.	\\
		\textbf{Descrizione}: l'utente ha la possibilità di rinominare una presentazione salvata nel proprio account. \\
		\textbf{Precondizione}: il sistema ha caricato lo spazio dedicato alle presentazioni e l'utente desidera rinominarne una.	\\
		\textbf{Postcondizione}: l'utente ha rinominato la presentazione desiderata.	\\
		\textbf{Scenario principale}:
		\begin{enumerate}
			\item Navigazione nello spazio di lavoro;
			\item Selezione presentazione;
			\item Conferma rinominazione della presentazione.
		\end{enumerate}
		}
	\subsubsection{UC 0.7.4 - Rinomina infografica}{
		\label{uc0.7.4}
		\textbf{Attori}: utente desktop.	\\
		\textbf{Descrizione}: l'utente ha la possibilità di rinominare un'infografica salvata nel proprio spazio sul server. \\
		\textbf{Precondizione}: il sistema ha caricato lo spazio dedicato alle infografiche e l'utente desidera rinominare una infografica.	\\
		\textbf{Postcondizione}: l'utente ha rinominato l'infografica desiderata.	\\
		\textbf{Scenario principale}:
		\begin{enumerate}
			\item Navigazione nello spazio di lavoro;
			\item Selezione infografica;
			\item Conferma rinominazione dell'infografica.
		\end{enumerate}
		}
	\subsubsection{UC 0.7.5 - Eliminazione presentazioni}{
		\label{uc0.7.5}
		\textbf{Attori}: utente desktop.	\\
		\textbf{Descrizione}: l'utente ha la possibilità di eliminare presentazioni contenute all'interno del proprio spazio sul server. \\
		\textbf{Precondizione}: il sistema ha caricato lo spazio dedicato ai file media e l'utente desidera eliminare una o più presentazioni esistenti nel proprio spazio sul server.	\\
		\textbf{Postcondizione}: l'utente ha cancellato le presentazioni che desiderava rimuovere.	\\
		\textbf{Scenario principale}:
		\begin{enumerate}
			\item Navigazione nello spazio di lavoro;
			\item Selezione presentazioni da eliminare;
			\item Conferma selezione ed eliminazione.
		\end{enumerate}
		\textbf{Scenari alternativi}: 
		\begin{itemize}
			\item L'operazione viene annullata e non si apportano cambiamenti.
		\end{itemize}
		}
	\subsubsection{UC 0.7.6 - Eliminazione infografiche}{
		\label{uc0.7.6}
		\textbf{Attori}: utente desktop.	\\
		\textbf{Descrizione}: l'utente ha la possibilità di eliminare infografiche contenute nel proprio spazio sul server. \\
		\textbf{Precondizione}: il sistema ha caricato lo spazio dedicato ai file media e l'utente desidera eliminare una o più infografiche.	\\
		\textbf{Postcondizione}: l'utente ha cancellato le infografiche che desiderava rimuovere.	\\
		\textbf{Scenario principale}:
		\begin{enumerate}
			\item Navigazione nello spazio di lavoro;
			\item Selezione infografiche da eliminare;
			\item Conferma selezione ed eliminazione.
		\end{enumerate}
		\textbf{Scenari alternativi}: 
		\begin{itemize}
			\item L'operazione viene annullata e non si apportano cambiamenti.
		\end{itemize}
		}
