\subsection{UC 1.3 - Modifica presentazione da desktop}{
	\label{uc1.3}
	\begin{figure}[H]
		\centering
		\includegraphics[scale=0.75]{\imgs {UC1.3}.jpg} %inserire il diagramma UML
		\label{fig:uc1.3}
		\caption{Caso d'uso 1.3: Modifica desktop della presentazione}
	\end{figure}
	\textbf{Attori}: utente desktop \\
	\textbf{Descrizione}: l'utente ha la possibilità di modificare una presentazione presente nel proprio spazio sul server. \\
	\textbf{Precondizione}: il sistema ha una presentazione caricata correttamente e aperta in modalità modifica.	\\
	\textbf{Postcondizione}: l'utente ha modificato la presentazione selezionata e salvato le modifiche in locale.	\\
	\textbf{Scenario principale}:
	\begin{enumerate}
		\item L'utente può inserire un nuovo frame vuoto \hyperref[uc1.3.1]{(UC 1.3.1)};
		\item L'utente può spostare i frame nel piano della presentazione \hyperref[uc1.3.2]{(UC 1.3.2)};
		\item L'utente può modificare i frame inseriti \hyperref[uc1.3.3]{(UC 1.3.3)};
		\item L'utente può eliminare i frame inseriti \hyperref[uc1.3.4]{(UC 1.3.4)};
		\item L'utente può impostare uno sfondo \hyperref[uc1.3.5]{(UC 1.3.5)};
		\item L'utente può definire i percorsi della presentazione \hyperref[uc1.3.6]{(UC 1.3.6)};
		\item L'utente può inserire e cancellare bookmark \hyperref[uc1.3.7]{(UC 1.3.7)};
		\item L'utente può impostare le opzioni di esecuzione della presentazione \hyperref[uc1.3.8]{(UC 1.3.8)};
		\item L'utente può definire i tempi di permanenza sui frame \hyperref[uc1.3.9]{(UC 1.3.9)};
	\end{enumerate}
	\textbf{Scenari alternativi}: 
	\begin{itemize}
		\item L'operazione viene annullata e non si apportano cambiamenti \hyperref[uc1.5]{(UC 1.5)}.
	\end{itemize}
	}
\subsubsection{UC 1.3.1 - Inserimento nuovo frame}{
	\label{uc1.3.1}
	\begin{figure}[H]
		\centering
		\includegraphics[scale=0.75]{\imgs {UC1.3.1}.jpg} %inserire il diagramma UML
		\label{fig:uc1.3.1}
		\caption{Caso d'uso 1.3.1: Inserimento di un nuovo frame}
	\end{figure}
	\textbf{Attori}: utente desktop \\
	\textbf{Descrizione}: l'utente può inserire un nuovo frame nel piano della presentazione. \\
	\textbf{Precondizione}: il sistema ha una presentazione aperta in modalità modifica.	\\
	\textbf{Postcondizione}: l'utente ha inserito nel piano della presentazione un nuovo frame.	\\
	\textbf{Scenario principale}:
	\begin{enumerate}
		\item L'utente seleziona un frame nel menu a lato del piano della presentazione;
		\item L'utente sposta il frame all'interno del piano della presentazione;
		\item L'utente rilascia il frame.
	\end{enumerate}
}
\subsubsection{UC 1.3.1.1 - Selezione frame}{
	\label{uc1.3.1.1}
	\textbf{Attori}: utente desktop \\
	\textbf{Descrizione}: l'utente può selezionare uno dei frame definiti dal sistema. \\
	\textbf{Precondizione}: il sistema ha una presentazione aperta in modalità modifica.	\\
	\textbf{Postcondizione}: l'utente ha selezionato dal menu il frame desiderato.	\\
	\textbf{Scenario principale}:
	\begin{enumerate}
		\item L'utente seleziona l'azione "nuovo frame";
		\item L'utente seleziona uno dei frame disponibili.
	\end{enumerate}
	}
\subsubsection{UC 1.3.1.2 - Spostamento frame nel piano della presentazione}{
	\label{uc1.3.1.2}
	\textbf{Attori}: utente desktop \\
	\textbf{Descrizione}: l'utente può spostare un nuovo frame nel piano della presentazione. \\
	\textbf{Precondizione}: il sistema ha una presentazione aperta in modalità modifica.	\\
	\textbf{Postcondizione}: l'utente ha inserito nel piano di modifica un nuovo frame.	\\
	\textbf{Scenario principale}:
	\begin{enumerate}
		\item L'utente sposta il frame nel piano della presentazione;
		\item L'utente rilascia la selezione sul frame.
	\end{enumerate}
	}
\subsubsection{UC 1.3.2 - Spostamento frame}{
	\label{uc1.3.2}
	\textbf{Attori}: utente desktop \\
	\textbf{Descrizione}: l'utente può spostare un frame nel piano della presentazione. \\
	\textbf{Precondizione}: il sistema ha una presentazione aperta in modalità modifica.	\\
	\textbf{Postcondizione}: l'utente ha spostato un frame nel piano della presentazione.	\\
	\textbf{Scenario principale}:
	\begin{enumerate}
		\item L'utente seleziona un frame nel piano della presentazione;
		\item L'utente sposta il frame nel piano della presentazione;
		\item L'utente rilascia il frame selezionato.
	\end{enumerate}
	}
\subsubsection{UC 1.3.3 - Modifica desktop di un frame}{
	\label{uc1.3.3}
	\begin{figure}[H]
		\centering
		\includegraphics[scale=0.6]{\imgs {UC1.3.3}.jpg} %inserire il diagramma UML
		\label{fig:uc1.3.3}
		\caption{Caso d'uso 1.3.3: Modifica desktop di un frame}
	\end{figure}
	\textbf{Attori}: utente desktop \\
	\textbf{Descrizione}: l'utente desktop ha scelto l'opzione di modifica di un frame: può scegliere di inserire o modificare un elemento (che può essere del testo, un'immagine, o un video), di spostare un elemento, di eliminare un elemento, di inserire o eliminare una scelta, di modificare la dimensione o la forma del frame, lo spessore o il colore del bordo e lo sfondo. \\
	\textbf{Precondizione}: il sistema ha una presentazione aperta in modalità modifica.	\\
	\textbf{Postcondizione}: nella presentazione è presente un frame modificato.	\\
	\textbf{Scenario principale}:
	\begin{enumerate}
		\item Inserimento di un elemento testo \hyperref[uc1.3.3.1]{(UC 1.3.3.1)};
		\item Inserimento di un elemento immagine \hyperref[uc1.3.3.2]{(UC 1.3.3.2)};
		\item Inserimento di un elemento video \hyperref[uc1.3.3.3]{(UC 1.3.3.3)};
		\item Modifica di un elemento testo \hyperref[uc1.3.3.4]{(UC 1.3.3.4)};
		\item Modifica di un elemento immagine \hyperref[uc1.3.3.5]{(UC 1.3.3.5)};
		\item Modifica di un elemento video \hyperref[uc1.3.3.6]{(UC 1.3.3.6)};
		\item Spostamento di un elemento \hyperref[uc1.3.3.7]{(UC 1.3.3.7)};
		\item Eliminazione di un elemento \hyperref[uc1.3.3.8]{(UC 1.3.3.8)};
		\item Inserimento di un elemento scelta \hyperref[uc1.3.3.9]{(UC 1.3.3.9)};
		\item Modifica di un elemento scelta \hyperref[uc1.3.3.10]{(UC 1.3.3.10)};
		\item Modifica della dimensione del frame \hyperref[uc1.3.3.11]{(UC 1.3.3.11)};
		\item Modifica della forma del frame \hyperref[uc1.3.3.12]{(UC 1.3.3.12)};
		\item Modifica dello spessore del bordo del frame \hyperref[uc1.3.3.13]{(UC 1.3.3.13)};
		\item Modifica del colore del bordo del frame \hyperref[uc1.3.3.14]{(UC 1.3.3.14)};
		\item Modifica dello sfondo del frame \hyperref[uc1.3.3.15]{(UC 1.3.3.15)};
	\end{enumerate}
	}
\subsubsection{UC 1.3.3.1 - Inserimento di un elemento testo}{
	\label{uc1.3.3.1}
	\textbf{Attori}: utente desktop \\
	\textbf{Descrizione}: l'utente desktop inserisce un elemento di tipo testo all'interno del frame. \\
	\textbf{Precondizione}: il sistema ha una presentazione aperta in modalità modifica e l'utente desidera aggiungere un nuovo elemento testuale.	\\
	\textbf{Postcondizione}: nel frame è presente un nuovo elemento testo.	\\
	\textbf{Scenario principale}:
	\begin{enumerate}
		\item L'utente desktop seleziona l'opzione di inserimento testo;
		\item L'utente desktop inserisce il testo desiderato.
	\end{enumerate}
	}
\subsubsection{UC 1.3.3.2 - Inserimento di un elemento immagine}{
	\label{uc1.3.3.2}
	\textbf{Attori}: utente desktop \\
	\textbf{Descrizione}: l'utente desktop inserisce un elemento di tipo immagine all'interno del frame. \\
	\textbf{Precondizione}: il sistema ha una presentazione aperta in modalità modifica e l'utente desidera aggiungere un elemento immagine.	\\
	\textbf{Postcondizione}: nel frame è presente un nuovo elemento immagine.	\\
	\textbf{Scenario principale}:
	\begin{enumerate}
		\item L'utente desktop seleziona l'opzione di inserimento immagine;
		\item L'utente desktop inserisce l'immagine desiderata tra quelle rese disponibili dal sistema oppure tra quelle presenti in locale.
	\end{enumerate}
	}
\subsubsection{UC 1.3.3.3 - Inserimento di un elemento audio/video}{
	\label{uc1.3.3.3}
	\textbf{Attori}: utente desktop \\
	\textbf{Descrizione}: l'utente desktop inserisce un elemento di tipo audio o video all'interno del frame. \\
	\textbf{Precondizione}: il sistema ha una presentazione aperta in modalità modifica e l'utente desidera aggiungere un elemento audio o video.	\\
	\textbf{Postcondizione}: nel frame è presente un nuovo elemento audio o video.	\\
	\textbf{Scenario principale}:
	\begin{enumerate}
		\item L'utente desktop seleziona l'opzione di inserimento audio o video;
		\item L'utente desktop inserisce l'audio o il video desiderato tra quelli presenti in locale oppure come link ad uno presente in rete.
	\end{enumerate}
	}
\subsubsection{UC 1.3.3.4 - Modifica di un elemento testo}{
	\label{uc1.3.3.4}
	\textbf{Attori}: utente desktop \\
	\textbf{Descrizione}: l'utente desktop modifica un elemento testo presente all'interno del frame. \\
	\textbf{Precondizione}: il sistema ha una presentazione aperta in modalità modifica e l'utente desidera modificare il contenuto di un elemento testo.	\\
	\textbf{Postcondizione}: nel frame è presente un elemento testo modificato.	\\
	\textbf{Scenario principale}:
	\begin{enumerate}
		\item L'utente desktop seleziona un elemento testo;
		\item L'utente desktop modifica il testo selezionato.
	\end{enumerate}
	}
\subsubsection{UC 1.3.3.5 - Modifica dimensione di un elemento immagine}{
	\label{uc1.3.3.5}
	\textbf{Attori}: utente desktop \\
	\textbf{Descrizione}: l'utente desktop ridimensiona un elemento immagine presente all'interno del frame. \\
	\textbf{Precondizione}: il sistema ha una presentazione aperta in modalità modifica e l'utente desidera modificare la dimensione di un elemento immagine.	\\
	\textbf{Postcondizione}: nel frame è presente un elemento immagine ridimensionato.	\\
	\textbf{Scenario principale}:
	\begin{enumerate}
		\item L'utente desktop seleziona un elemento immagine;
		\item L'utente desktop ridimensiona l'elemento immagine.
	\end{enumerate}
	}
\subsubsection{UC 1.3.3.6 - Modifica dimensione di un elemento audio/video}{
	\label{uc1.3.3.6}
	\textbf{Attori}: utente desktop \\
	\textbf{Descrizione}: l'utente desktop ridimensiona un elemento audio o video presente all'interno del frame. \\
	\textbf{Precondizione}: il sistema ha una presentazione aperta in modalità modifica e l'utente desidera modificare la dimensione di un elemento audio o video.	\\
	\textbf{Postcondizione}: nel frame è presente un elemento audio o un video ridimensionato.	\\
	\textbf{Scenario principale}:
	\begin{enumerate}
		\item L'utente desktop seleziona un elemento audio o video;
		\item L'utente desktop ridimensiona l'elemento audio o video.
	\end{enumerate}
	}
\subsubsection{UC 1.3.3.7 - Spostamento di un elemento}{
	\label{uc1.3.3.7}
	\textbf{Attori}: utente desktop \\
	\textbf{Descrizione}: l'utente desktop sposta un elemento qualsiasi all'interno del frame. \\
	\textbf{Precondizione}: il sistema ha una presentazione aperta in modalità modifica e l'utente desidera modificare la posizione di un elemento.	\\
	\textbf{Postcondizione}: nel frame è presente un elemento spostato.	\\
	\textbf{Scenario principale}:
	\begin{enumerate}
		\item L'utente destop seleziona un elemento;
		\item L'utente desktop trascina l'elemento selezionato all'interno del frame.
	\end{enumerate}
	}
\subsubsection{UC 1.3.3.8 - Eliminazione di un elemento}{
	\label{uc1.3.3.8}
	\textbf{Attori}: utente desktop \\
	\textbf{Descrizione}: l'utente desktop elimina un elemento qualsiasi all'interno del frame. \\
	\textbf{Precondizione}: il sistema ha una presentazione aperta in modalità modifica e l'utente desidera eliminare un elemento.	\\
	\textbf{Postcondizione}: l'utente desktop ha eliminato un elemento.	\\
	\textbf{Scenario principale}:
	\begin{enumerate}
		\item L'utente desktop seleziona un elemento;
		\item L'utente desktop elimina l'elemento selezionato.
	\end{enumerate}
	}
\subsubsection{UC 1.3.3.9 - Inserimento di un elemento scelta}{
	\label{uc1.3.3.9}
	\textbf{Attori}: utente desktop \\
	\textbf{Descrizione}: l'utente desktop inserisce all'interno di un frame un elemento scelta verso un altro frame non appartenente allo stesso percorso di presentazione del frame. \\
	\textbf{Precondizione}: il sistema ha una presentazione aperta in modalità modifica e l'utente desidera aggiungere un elemento scelta.	\\
	\textbf{Postcondizione}: l'utente desktop ha inserito una nuova scelta.	\\
	\textbf{Scenario principale}:
	\begin{enumerate}
		\item L'utente desktop seleziona l'opzione di inserimento scelta;
		\item L'utente desktop seleziona un altro frame;
		\item L'utente desktop assegna il nome all'elemento scelta.
	\end{enumerate}
	}
\subsubsection{UC 1.3.3.10 - Modifica di un elemento scelta}{
	\label{uc1.3.3.10}
	\textbf{Attori}: utente desktop \\
	\textbf{Descrizione}: l'utente desktop modifica il testo assegnato ad un elemento scelta presente all'interno del frame. \\
	\textbf{Precondizione}: il sistema ha una presentazione aperta in modalità modifica e l'utente desidera modificare il nome assegnato ad un elemento a scelta.	\\
	\textbf{Postcondizione}: l'utente desktop ha modificato il nome di un elemento scelta.	\\
	\textbf{Scenario principale}:
	\begin{enumerate}
		\item L'utente desktop seleziona un elemento scelta;
		\item L'utente desktop modifica il testo del nome assegnato all'elemento scelta.
	\end{enumerate}
	}
\subsubsection{UC 1.3.3.11 - Modifica della dimensione del frame}{
	\label{uc1.3.3.11}
	\textbf{Attori}: utente desktop \\
	\textbf{Descrizione}: l'utente desktop ridimensiona il frame. \\
	\textbf{Precondizione}: il sistema presenta un frame selezionato e l'utente desidera modificarne la dimensione.	\\
	\textbf{Postcondizione}: la presentazione contiene un frame con una nuova dimensione.	\\
	\textbf{Scenario principale}:
	\begin{enumerate}
		\item L'utente desktop seleziona l'opzione per la modifica della dimensione del frame;
		\item L'utente desktop ridimensiona il frame.
	\end{enumerate}
	}
\subsubsection{UC 1.3.3.12 - Modifica della forma di un frame}{
	\label{uc1.3.3.12}
	\textbf{Attori}: utente desktop \\
	\textbf{Descrizione}: l'utente desktop modifica la forma del frame. \\
	\textbf{Precondizione}: il sistema presenta un frame selezionato e l'utente desidera modificarne la forma.	\\
	\textbf{Postcondizione}: la presentazione contiene un frame con una nuova forma.	\\
	\textbf{Scenario principale}:
	\begin{enumerate}
		\item L'utente desktop seleziona l'opzione per la modifica della forma del frame;
		\item L'utente desktop seleziona la nuova forma del frame.
	\end{enumerate}}
\subsubsection{UC 1.3.3.13 - Modifica dello spessore del bordo del frame}{
	\label{uc1.3.3.13}
	\textbf{Attori}: utente desktop \\
	\textbf{Descrizione}: l'utente desktop modifica lo spessore del bordo del frame. \\
	\textbf{Precondizione}: il sistema presenta un frame selezionato e l'utente desidera modificarne lo spessore del bordo.	\\
	\textbf{Postcondizione}: la presentazione contiene un frame con un bordo dal nuovo spessore.	\\
	\textbf{Scenario principale}:
	\begin{enumerate}
		\item L'utente desktop seleziona l'opzione per la modifica dello spessore del bordo del frame;
		\item L'utente desktop ridimensiona lo spessore del bordo del frame.
	\end{enumerate}
	}
\subsubsection{UC 1.3.3.14 - Modifica del colore del bordo del frame}{
	\label{uc1.3.3.14}
	\textbf{Attori}: utente desktop \\
	\textbf{Descrizione}: l'utente desktop modifica il colore del bordo del frame. \\
	\textbf{Precondizione}: il sistema presenta un frame selezionato e l'utente desidera modificarne il colore del bordo.	\\
	\textbf{Postcondizione}: la presentazione contiene un frame con un bordo con un nuovo colore.	\\
	\textbf{Scenario principale}:
	\begin{enumerate}
		\item L'utente desktop seleziona l'opzione per la modifica del colore del bordo del frame;
		\item L'utente desktop seleziona il nuovo colore del bordo del frame.
	\end{enumerate}
	}
\subsubsection{UC 1.3.3.15 - Modifica dello sfondo del frame}{
	\label{uc1.3.3.15}
	\textbf{Attori}: utente desktop \\
	\textbf{Descrizione}: l'utente desktop modifica lo sfondo del frame. \\
	\textbf{Precondizione}: il sistema presenta un frame selezionato e l'utente desidera modificarne lo sfondo.	\\
	\textbf{Postcondizione}: la presentazione contiene un frame con un nuovo sfondo.	\\
	\textbf{Scenario principale}:
	\begin{enumerate}
		\item L'utente desktop seleziona l'opzione per la modifica dello sfondo del frame;
		\item L'utente desktop può selezionare un nuovo colore di sfondo del frame;
		\item L'utente desktop può selezionare una nuova immagine di sfondo del frame.
	\end{enumerate}
	}
\subsubsection{UC 1.3.4 - Eliminazione frame}{
	\label{uc1.3.4}
	\textbf{Attori}: utente desktop \\
	\textbf{Descrizione}: l'utente può eliminare un frame dal piano della presentazione. \\
	\textbf{Precondizione}: il sistema ha una presentazione aperta in modalità modifica.	\\
	\textbf{Postcondizione}: l'utente ha eliminato un frame dal piano della presentazione.	\\
	\textbf{Scenario principale}:
	\begin{enumerate}
		\item L'utente seleziona un frame nel piano della presentazione;
		\item L'utente seleziona l'azione di cancellazione del frame;
		\item L'utente conferma la cancellazione del frame.
	\end{enumerate}
	\textbf{Scenari alternativi}: 
	\begin{itemize}
		\item L'utente non conferma l'azione di cancellazione e si ritorna alla precondizione.
	\end{itemize}
}
\subsubsection{UC 1.3.5 - Impostazione sfondo}{
	\label{uc1.3.5}
	\begin{figure}[H]
		\centering
		\includegraphics[scale=0.75]{\imgs {UC1.3.5}.jpg} %inserire il diagramma UML
		\label{fig:uc1.3.5}
		\caption{Caso d'uso 1.3.5: Impostazione dello sfondo}
	\end{figure}
	\textbf{Attori}: utente desktop \\
	\textbf{Descrizione}: l'utente può impostare un'immagine o un colore di sfondo alla presentazione. \\
	\textbf{Precondizione}: il sistema ha una presentazione aperta in modalità modifica.	\\
	\textbf{Postcondizione}: l'utente ha impostato una sfondo sul piano della presentazione.	\\
	\textbf{Scenario principale}:
	\begin{enumerate}
		\item L'utente inserisce un'immagine o un colore di sfondo in un'area definita \hyperref[uc1.3.5.1]{(UC 1.3.5.1)};
		\item L'utente può spostare l'area di sfondo sul piano della presentazione \hyperref[uc1.3.5.2]{(UC 1.3.5.2)};
		\item L'utente può ridimensionare l'area di sfondo sul piano della presentazione \hyperref[uc1.3.5.3]{(UC 1.3.5.3)};
		\item L'utente conferma l'inserimento dello sfondo.
	\end{enumerate}
	\textbf{Scenari alternativi}: 
	\begin{itemize}
		\item L'utente non conferma l'inserimento dello sfondo e si ritorna alla precondizione.
	\end{itemize}
}
\subsubsection{UC 1.3.5.1 - Inserimento sfondo}{
	\label{uc1.3.5.1}
	\begin{figure}[H]
		\centering
		\includegraphics[scale=0.75]{\imgs {UC1.3.5.1}.jpg} %inserire il diagramma UML
		\label{fig:uc1.3.5.1}
		\caption{Caso d'uso 1.3.5.1: Inserimento di uno sfondo}
	\end{figure}
	\textbf{Attori}: utente desktop \\
	\textbf{Descrizione}: l'utente può inserire una immagine o un colore di sfondo nel piano della presentazione. \\
	\textbf{Precondizione}: il sistema ha una presentazione aperta in modalità modifica.	\\
	\textbf{Postcondizione}: l'utente ha inserito uno sfondo nel piano della presentazione.	\\
	\textbf{Scenario principale}:
	\begin{enumerate}
		\item L'utente può selezionare un'immagine da usare come sfondo \hyperref[uc1.3.5.1.1]{(UC 1.3.5.1)};
		\item L'utente può selezionare un colore da usare come sfondo \hyperref[uc1.3.5.1.2]{(UC 1.3.5.2)};
		\item L'utente definisce un'area nel piano della presentazione per ospitare lo sfondo \hyperref[uc1.3.5.1.3]{(UC 1.3.5.3)};
		\item L'utente conferma l'inserimento dello sfondo \hyperref[uc1.3.5.1.4]{(UC 1.3.5.4)}.
	\end{enumerate}
}
\subsubsection{UC 1.3.5.1.1 - Selezione immagine sfondo}{
	\label{uc1.3.5.1.1}
	\textbf{Attori}: utente desktop \\
	\textbf{Descrizione}: l'utente può selezionare un'immagine per lo sfondo. \\
	\textbf{Precondizione}: il sistema ha una presentazione aperta in modalità modifica e l'utente ha selezionato l'opzione per impostare lo sfondo.	\\
	\textbf{Postcondizione}: l'utente ha selezionato un'immagine per lo sfondo.	\\
	\textbf{Scenario principale}:
	\begin{enumerate}
		\item L'utente scorre i file immagine resi disponibili dal sistema oppure può scegliere di caricare un'immagine presente in locale;
		\item L'utente seleziona un'immagine;
		\item L'utente conferma la selezione.
	\end{enumerate}
}
\subsubsection{UC 1.3.5.1.2 - Selezione colore sfondo}{
	\label{uc1.3.5.1.2}
	\textbf{Attori}: utente desktop \\
	\textbf{Descrizione}: l'utente può selezionare un colore per lo sfondo. \\
	\textbf{Precondizione}: il sistema ha una presentazione aperta in modalità modifica e l'utente ha selezionato l'opzione per impostare il colore di sfondo.	\\
	\textbf{Postcondizione}: l'utente ha selezionato un colore per lo sfondo.	\\
	\textbf{Scenario principale}:
	\begin{enumerate}
		\item L'utente scorre i colori definiti;
		\item L'utente seleziona un colore;
		\item L'utente conferma la selezione.
	\end{enumerate}
}
\subsubsection{UC 1.3.5.1.3 - Definizione area sfondo}{
	\label{uc1.3.5.1.3}
	\textbf{Attori}: utente desktop \\
	\textbf{Descrizione}: l'utente può definire un'area per lo sfondo. \\
	\textbf{Precondizione}: il sistema ha una presentazione aperta in modalità modifica e l'utente ha selezionato un'immagine o un colore per lo sfondo.	\\
	\textbf{Postcondizione}: l'utente ha definito un'area per lo sfondo.	\\
	\textbf{Scenario principale}:
	\begin{enumerate}
		\item L'utente seleziona un'area nel piano di presentazione;
		\item L'utente conferma la definizione dell'area .
	\end{enumerate}
}
\subsubsection{UC 1.3.5.1.4 - Conferma sfondo}{
	\label{uc1.3.5.1.4}
	\textbf{Attori}: utente desktop \\
	\textbf{Descrizione}: l'utente deve confermare l'inserimento di uno sfondo. \\
	\textbf{Precondizione}: l'utente ha selezionato un colore o un'immagine per lo sfondo e ne ha definito l'area.	\\
	\textbf{Postcondizione}: l'utente ha inserito uno sfondo.	\\
	\textbf{Scenario principale}:
	\begin{enumerate}
		\item L'utente conferma l'inserimento dello sfondo.
	\end{enumerate}
}
\subsubsection{UC 1.3.5.2 - Spostamento sfondo}{
	\label{uc1.3.5.2}
	\textbf{Attori}: utente desktop \\
	\textbf{Descrizione}: l'utente può spostare l'area di sfondo nel piano della presentazione. \\
	\textbf{Precondizione}: il sistema ha una presentazione aperta in modalità modifica e vi è presente un'area di sfondo.	\\
	\textbf{Postcondizione}: l'utente ha spostato uno sfondo nel piano della presentazione.	\\
	\textbf{Scenario principale}:
	\begin{enumerate}
		\item L'utente seleziona l'area di sfondo;
		\item L'utente sposta l'area di sfondo sul piano della presentazione.
	\end{enumerate}
}
\subsubsection{UC 1.3.5.3 - Ridimensionamento sfondo}{
	\label{uc1.3.5.3}
	\textbf{Attori}: utente desktop \\
	\textbf{Descrizione}: l'utente può ridimensionare l'area di sfondo nel piano della presentazione. \\
	\textbf{Precondizione}: il sistema ha una presentazione aperta in modalità modifica e vi è presente un'area di sfondo.	\\
	\textbf{Postcondizione}: l'utente ha ridimensionato uno sfondo nel piano della presentazione.	\\
	\textbf{Scenario principale}:
	\begin{enumerate}
		\item L'utente seleziona l'area dello sfondo;
		\item L'utente seleziona l'azione 'ridimensianamento';
		\item L'utente ridefinisce l'area dello sfondo;
		\item L'utente conferma la nuova area.
	\end{enumerate}
}
\subsubsection{UC 1.3.6 - Definizione percorsi di visualizzazione}{
	\label{uc1.3.6}
	\begin{figure}[H]
		\centering
		\includegraphics[scale=0.75]{\imgs {UC1.3.6}.jpg} %inserire il diagramma UML
		\label{fig:uc1.3.6}
		\caption{Caso d'uso 1.3.6: Definizione dei percorsi di visualizzazione}
	\end{figure}
	\textbf{Attori}: utente desktop \\
	\textbf{Descrizione}: l'utente può definire la sequenza dei frame visualizzati di una presentazione. \\
	\textbf{Precondizione}: il sistema è entrato in modalità modifica percorsi e la presentazione caricata presenta almeno un frame.	\\
	\textbf{Postcondizione}: l'utente ha definito la sequenza dei frame visualizzati in fase di esecuzione della presentazione aperta.	\\
	\textbf{Scenario principale}:
	\begin{enumerate}
		\item L'utente passa alla modalità "modifica percorsi di visualizzazione";
		\item L'utente imposta un frame come primo elemento visualizzato di una presentazione;
		\item L'utente può definire nuove transizioni tra frame;
		\item L'utente può escludere un frame parte di un percorso;
		\item L'utente può eliminare una transizione.
	\end{enumerate}
}
\subsubsection{UC 1.3.6.1 - Imposta frame iniziale}{
	\label{uc1.3.6.1}
	\textbf{Attori}: utente desktop \\
	\textbf{Descrizione}: l'utente può impostare il primo frame visualizzato in fase di esecuzione di una presentazione. \\
	\textbf{Precondizione}: il sistema presenta un frame selezionato e l'utente desidera impostarlo come primo frame.	\\
	\textbf{Postcondizione}: l'utente ha impostato il primo frame da visualizzare in fase di esecuzione della presentazione aperta.	\\
	\textbf{Scenario principale}:
	\begin{enumerate}
		\item L'utente seleziona un frame sul piano della presentazione;
		\item L'utente seleziona l'azione "inizio" a lato del frame.
	\end{enumerate}
}
\subsubsection{UC 1.3.6.2 - Nuova transizione}{
	\label{uc1.3.6.2}
	\textbf{Attori}: utente desktop \\
	\textbf{Descrizione}: l'utente può definire una nuova transizione tra due frame. \\
	\textbf{Precondizione}: il sistema presenta un frame selezionato e l'utente desidera impostare il successivo nel percorso di visualizzazione.	\\
	\textbf{Postcondizione}: l'utente ha definito una transizione tra due frame.	\\
	\textbf{Scenario principale}:
	\begin{enumerate}
		\item L'utente seleziona una frame nel piano della presentazione;
		\item L'utente seleziona l'azione "aggiungi transizione";
		\item L'utente seleziona il frame destinazione.
	\end{enumerate}
}
\subsubsection{UC 1.3.6.3 - Escludi frame dal percorso}{
	\label{uc1.3.6.3}
	\textbf{Attori}: utente desktop \\
	\textbf{Descrizione}: l'utente può escludere la visualizzazione di un frame dal percorso. \\
	\textbf{Precondizione}: il sistema presenta un frame selezionato e l'utente desidera escluderlo dal percorso di presentazione.	\\
	\textbf{Postcondizione}: l'utente ha rimosso un frame dal percorso di visualizzazione.	\\
	\textbf{Scenario principale}:
	\begin{enumerate}
		\item L'utente seleziona un frame nel piano della presentazione;
		\item L'utente seleziona l'azione per l'esclusione del frame dal percorso di presentazione;
		\item L'utente conferma l'eliminazione del frame.
	\end{enumerate}
}
\subsubsection{UC 1.3.6.4 - Elimina transizione}{
	\label{uc1.3.6.4}
	\textbf{Attori}: utente desktop \\
	\textbf{Descrizione}: l'utente può eliminare una transizione tra due frame del piano di presentazione. \\
	\textbf{Precondizione}: il sistema presenta una transizione selezionata e l'utente desidera eliminarla.	\\
	\textbf{Postcondizione}: l'utente ha eliminato una transizione tra due frame.	\\
	\textbf{Scenario principale}:
	\begin{enumerate}
		\item L'utente seleziona una transizione nel piano della presentazione;
		\item L'utente seleziona l'azione "cancella transizione".
	\end{enumerate}
}
\subsubsection{UC 1.3.7 - Gestione desktop bookmark}{
	\label{uc1.3.7}
	\begin{figure}[H]
		\centering
		\includegraphics[scale=0.75]{\imgs {UC1.3.7}.jpg} %inserire il diagramma UML
		\label{fig:uc1.3.7}
		\caption{Caso d'uso 1.3.7: Gestione desktop dei bookmark}
	\end{figure}
	\textbf{Attori}: utente desktop \\
	\textbf{Descrizione}: l'utente può inserire e cancellare bookmark da un frame. \\
	\textbf{Precondizione}: il sistema ha una presentazione aperta in modalità modifica.	\\
	\textbf{Postcondizione}: l'utente ha inserito o eliminato bookmark da un frame.	\\
	\textbf{Scenario principale}:
	\begin{enumerate}
		\item L'utente può inserire un nuovo bookmark in un frame;
		\item L'utente può eliminare un bookmark da un frame.
	\end{enumerate}
}
\subsubsection{UC 1.3.7.1 - Inserimento bookmark}{
	\label{uc1.3.7.1}
	\textbf{Attori}: utente desktop \\
	\textbf{Descrizione}: l'utente può inserire un nuovo bookmark su un frame. \\
	\textbf{Precondizione}: il sistema ha una presentazione aperta in modalità modifica e l'utente desidera aggiungere un nuovo bookmark.	\\
	\textbf{Postcondizione}: l'utente ha inserito un bookmark ad un frame.	\\
	\textbf{Scenario principale}:
	\begin{enumerate}
		\item L'utente seleziona un frame associato al frame in cui inserire il bookmark;
		\item L'utente selezione l'azione inserisci bookmark a lato del numero del frame corrispondente.
	\end{enumerate}
}
\subsubsection{UC 1.3.7.2 - Eliminazione bookmark}{
	\label{uc1.3.7.2}
	\textbf{Attori}: utente desktop \\
	\textbf{Descrizione}: l'utente può cancellare bookmark da un frame. \\
	\textbf{Precondizione}: il sistema ha una presentazione aperta in modalità modifica e l'utente desidera rimuovere un bookmark.	\\
	\textbf{Postcondizione}: l'utente ha eliminato un bookmark da un frame.	\\
	\textbf{Scenario principale}:
	\begin{enumerate}
		\item L'utente seleziona il frame associato al frame da cui eliminare il bookmark;
		\item L'utente seleziona l'azione "elimina bookmark" a lato del frame e del numero del frame a cui è associato il bookmark da eliminare.
	\end{enumerate}
}
\subsubsection{UC 1.3.8 - Definizione effetti di transizione}{
	\label{uc1.3.8}
	\begin{figure}[H]
		\centering
		\includegraphics[scale=0.75]{\imgs {UC1.3.8}.jpg} %inserire il diagramma UML
		\label{fig:uc1.3.8}
		\caption{Caso d'uso 1.3.8: Definizione degli effetti di transizione tra frame}
	\end{figure}
	\textbf{Attori}: utente desktop \\
	\textbf{Descrizione}: l'utente può definire gli effetti di transizione da un frame ad un altro; l'effetto verrà attribuito al frame selezionato e verrà applicato nel momento in cui si passerà da quel frame a quello successivo nel percorso di presentazione. \\
	\textbf{Precondizione}: il sistema ha una presentazione aperta in modalità modifica; l'utente ha selezionato l'opzione per la definizione degli effetti di transizione.	\\
	\textbf{Postcondizione}: l'utente ha modificato l'effetto di transizione per le quelle desiderate.	\\
	\textbf{Scenario principale}:
	\begin{enumerate}
		\item L'utente seleziona il frame a cui modificare l'effetto di transizione;
		\item L'utente seleziona l'azione "effetto transizione";
		\item L'utente seleziona un effetto tra quelli definiti;
		\item L'utente può selezionare la velocità di transizione;
		\item L'utente conferma il cambiamento.
	\end{enumerate}
	\textbf{Scenari alternativi}:
	\begin{itemize}
		\item L'utente non conferma la modifica e si ritorna alla precondizione.
	\end{itemize}
	}
\subsubsection{UC 1.3.8.1 - Selezione transizione}{
	\label{uc1.3..8.1}
	\textbf{Attori}: utente desktop \\
	\textbf{Descrizione}: l'utente vuole selezionare una transizione da applicare ad un frame. \\
	\textbf{Precondizione}: il sistema ha una presentazione aperta in modalità modifica e l'utente ha selezionato l'opzione effetti transazioni.	\\
	\textbf{Postcondizione}: l'utente ha selezionato una transizione.	\\
	\textbf{Scenario principale}:
	\begin{enumerate}
		\item L'utente seleziona una transizione per il frame selezionato.
	\end{enumerate}
}
\subsubsection{UC 1.3.8.2 - Selezione effetto}{
	\label{uc1.3.8.2}
	\textbf{Attori}: utente desktop \\
	\textbf{Descrizione}: l'utente può selezionare un effetto grafico di transizione da associare ad una transizione. \\
	\textbf{Precondizione}: il sistema ha una transizione selezionata e l'utente ha selezionato l'opzione effetti transazioni.	\\
	\textbf{Postcondizione}: l'utente ha selezionato un effetto di transizione per una transizione.	\\
	\textbf{Scenario principale}:
	\begin{enumerate}
		\item L'utente seleziona un effetto per una transizione.
	\end{enumerate}
}
\subsubsection{UC 1.3.8.3 - Conferma effetto}{
	\label{uc1.3.8.3}
	\textbf{Attori}: utente desktop \\
	\textbf{Descrizione}: l'utente può confermare la modifica di un effetto grafico su di una transizione. \\
	\textbf{Precondizione}: l'utente una transizione selezionata e l'utente ha selezionato l'opzione effetti transazioni, un effetto grafico e una velocità di visualizzazione per la transizione considerata.	\\
	\textbf{Postcondizione}: l'utente ha confermato la modifica di un effetto grafico su di una transizione.\\
	\textbf{Scenario principale}:
	\begin{enumerate}
		\item L'utente conferma l'effetto selezionato.
	\end{enumerate}
}
\subsubsection{UC 1.3.8.4 - Imposta velocità transizione}{
	\label{uc1.3..8.4}
	\textbf{Attori}: utente desktop \\
	\textbf{Descrizione}: l'impostare una velocita di visualizzazione di un effetto grafico su una transizione. \\
	\textbf{Precondizione}: il sistema ha una transizione selezionata e l'utente ha selezionato l'opzione effetti transazioni e un effetto grafico per la transizione considerata.	\\
	\textbf{Postcondizione}: l'utente ha impostato la velocità di visualizzazione di un effetto grafico su una transizione.\\
	\textbf{Scenario principale}:
	\begin{enumerate}
		\item L'utente imposta la velocità di visualizzazione di un effetto grafico.
	\end{enumerate}
}
\subsubsection{UC 1.3.9 - Definizione tempo di permanenza}{
	\label{uc1.3.9}
	\textbf{Attori}: utente desktop \\
	\textbf{Descrizione}: l'utente può definire il temo di permanenza per ogni frame in modalità esecuzione automatica di una presentazione. \\
	\textbf{Precondizione}: il sistema ha una presentazione aperta in modalità modifica.	\\
	\textbf{Postcondizione}: l'utente ha modificato il tempo di permanenza sui frame in modalità esecuzione automatica della presentazione.	\\
	\textbf{Scenario principale}:
	\begin{enumerate}
		\item L'utente seleziona la voce in menu "tempo di permanenza";
		\item L'utente sceglie il tempo di permanenza in secondi;
		\item L'utente conferma la scelta.
	\end{enumerate}
	\textbf{Scenari alternativi}:
	\begin{itemize}
		\item L'utente non conferma la scelta  e si ritorna alla precondizione.
	\end{itemize}
}