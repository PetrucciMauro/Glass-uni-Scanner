\subsection{UC 1.4 - Modifica presentazione da mobile}{
	\label{uc1.4}
	\begin{figure}[H]
		\centering
		\includegraphics[scale=0.75]{\imgs {UC1.4}.jpg} %inserire il diagramma UML
		\label{fig:uc1.4}
		\caption{Caso d'uso 1.4: Modifica mobile di una presentazione}
	\end{figure}
	\textbf{Attori}: utente mobile \\
	\textbf{Descrizione}: L'utente mobile ha scelto l'opzione di modifica della presentazione. L'utente mobile può scegliere di modificare un frame o di gestire i bookmark. \\
	\textbf{Precondizione}: il sistema ha una presentazione caricata correttamente e aperta in modalità modifica.	\\
	\textbf{Postcondizione}: il sistema contiene una presentazione che l'utente è riuscito a modificare correttamente.	\\
	\textbf{Scenario principale}:
	\begin{enumerate}
		\item Modifica mobile di un frame \hyperref[uc1.4.1]{(UC 1.4.1)};
		\item Gestione mobile dei bookmark \hyperref[uc1.4.2]{(UC 1.4.2)}.
	\end{enumerate}
	\textbf{Scenari alternativi}: 
	\begin{itemize}
		\item L'operazione viene annullata e non si apportano cambiamenti \hyperref[uc1.5]{(UC 1.5)}.
	\end{itemize}
	}
\subsubsection{UC 1.4.1 - Modifica mobile di un frame}{
	\label{uc1.4.1}
	\begin{figure}[H]
		\centering
		\includegraphics[scale=0.75]{\imgs {UC1.4.1}.jpg} %inserire il diagramma UML
		\label{fig:uc1.4.1}
		\caption{Caso d'uso 1.4.1: Modifica mobile di un frame}
	\end{figure}
	\textbf{Attori}: utente mobile \\
	\textbf{Descrizione}: l'utente mobile ha scelto l'opzione di modifica di un frame. L'utente mobile può scegliere di inserire o modificare un elemento testo. \\
	\textbf{Precondizione}: il sistema ha una presentazione caricata correttamente e aperta in modalità modifica.	\\
	\textbf{Postcondizione}: la presentazione contiene un frame che l'utente è riuscito a modficare correttamente.	\\
	\textbf{Scenario principale}:
	\begin{enumerate}
		\item Inserimento di un elemento testo \hyperref[uc1.4.1.1]{(UC 1.4.1.1)};
		\item Modifica di un elemento testo \hyperref[uc1.4.1.2]{(UC 1.4.1.2)}.
	\end{enumerate}
	}
\subsubsection{UC 1.4.1.1 - Inserimento di un elemento testo}{
	\label{uc1.4.1.1}
	\textbf{Attori}: utente mobile \\
	\textbf{Descrizione}: l'utente mobile inserisce un elemento di tipo testo all'interno del frame. \\
	\textbf{Precondizione}: il sistema ha una presentazione caricata e l'utente desidera aggiungere un nuovo elemento testuale.	\\
	\textbf{Postcondizione}: l'utente ha inserito nel frame un nuovo elemento testo.	\\
	\textbf{Scenario principale}:
	\begin{enumerate}
		\item L'utente mobile seleziona l'opzione di inserimento testo;
		\item L'utente mobile inserisce il testo desiderato.
	\end{enumerate}
	}
\subsubsection{UC 1.4.1.2 - Modifica di un elemento testo}{
	\label{uc1.4.1.2}
	\textbf{Attori}: utente mobile \\
	\textbf{Descrizione}: l'utente mobile modifica un elemento testo presente all'interno del frame. \\
	\textbf{Precondizione}: il sistema presenta un elemento testo selezionato e l'utente desidera modificarne il contenuto.	\\
	\textbf{Postcondizione}: nel frame è presente un elemento testo che l'utente ha modificato con successo.	\\
	\textbf{Scenario principale}:
	\begin{enumerate}
		\item L'utente mobile seleziona un elemento testo;
		\item L'utente mobile modifica il testo selezionato.
	\end{enumerate}
	}	
\subsubsection{UC 1.4.2 - Gestione mobile bookmark}{
	\label{uc1.4.2}
	\begin{figure}[H]
		\centering
		\includegraphics[scale=0.75]{\imgs {UC1.4.2}.jpg} %inserire il diagramma UML
		\label{fig:uc1.4.2}
		\caption{Caso d'uso 1.4.2: Gestione mobile dei bookmark}
	\end{figure}
	\textbf{Attori}: utente mobile \\
	\textbf{Descrizione}: l'utente mobile ha scelto l'opzione di gestione dei bookmark. L'utente mobile può scegliere di inserire o rimuovere bookmark. \\
	\textbf{Precondizione}: il sistema ha una presentazione caricata correttamente e aperta in modalità modifica.	\\
	\textbf{Postcondizione}: la presentazione contiene una diversa disposizione dei bookmark.	\\
	\textbf{Scenario principale}:
	\begin{enumerate}
		\item Inserimento di un nuovo bookmark \hyperref[uc1.4.2.1]{(UC 1.4.2.1)};
		\item Rimozione di un bookmark \hyperref[uc1.4.2.2]{(UC 1.4.2.2)}.
	\end{enumerate}
	}
\subsubsection{UC 1.4.2.1 - Inserimento di un nuovo bookmark}{
	\label{uc1.4.2.1}
	\textbf{Attori}: utente mobile \\
	\textbf{Descrizione}: l'utente mobile inserisce un bookmark su un frame che non ne contiene uno. \\
	\textbf{Precondizione}: il sistema ha una presentazione caricata e l'utente desidera aggiungere un nuovo bookmark.	\\
	\textbf{Postcondizione}: il frame selezionato contiene un nuovo bookmark.	\\
	\textbf{Scenario principale}:
	\begin{enumerate}
		\item L'utente mobile seleziona un frame assegnando il bookmark.
	\end{enumerate}
	}
\subsubsection{UC 1.4.2.2 - Rimozione di un bookmark}{
	\label{uc1.4.2.2}
	\textbf{Attori}: utente mobile \\
	\textbf{Descrizione}: l'utente mobile rimuove un bookmark da un frame. \\
	\textbf{Precondizione}: il sistema ha una presentazione caricata e l'utente desidera rimuovere un bookmark.	\\
	\textbf{Postcondizione}: il frame selezionato non contiene un bookmark.	\\
	\textbf{Scenario principale}:
	\begin{enumerate}
		\item L'utente mobile seleziona un frame rimuovendo il bookmark.
	\end{enumerate}
	}