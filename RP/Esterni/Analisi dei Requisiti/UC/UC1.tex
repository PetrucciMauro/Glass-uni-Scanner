\subsection{UC 1 - Caso generale}{
	\label{uc1}
	\begin{figure}[H]
		\centering
		\includegraphics[scale=0.6]{\imgs {UC1}.jpg} %inserire il diagramma UML
		\label{fig:uc1}
		\caption{Caso d'uso 1}
	\end{figure}
	\textbf{Attori}: utente autenticato, utente offline, amministratore di sistema. \\
	\textbf{Descrizione}: il sistema deve mettere l'utente in condizioni di raccogliere idee e materiale multimediale da essere usati in una presentazione, creare una presentazione, modificare una presentazione, creare infografiche a partire da una presentazione, salvare ciò che ha prodotto su di un server ed infine eseguire una presentazione creata con il sistema. \\
	\textbf{Precondizione}: il sistema è avviato e pronto all'uso.	\\
	\textbf{Postcondizione}: è stato eseguito ciò che l'utente desiderava e che il sistema è in grado di eseguire.	\\
	\textbf{Scenario principale}:
	\begin{enumerate}
		\item l'utente si registra creando un account utente \hyperref[uc1.8]{(UC 1.8)};
		\item l'utente può effettuare il login \hyperref[uc1.10]{(UC 1.10)} e il logout \hyperref[uc1.13]{(UC 1.13)};
		\item l'utente può gestire il proprio account \hyperref[uc1.11]{(UC 1.11)};
		\item l'utente può creare una nuova presentazione \hyperref[uc1.2]{(UC 1.2)};
		\item l'utente può modificare una presentazione \hyperref[uc1.3]{(UC 1.3)};
		\item l'utente può eseguire la presentazione \hyperref[uc1.18]{(UC 1.18)} da locale \hyperref[uc1.14]{(UC 1.14)} oppure online \hyperref[uc1.5]{(UC 1.5)};
		\item l'utente può gestire le presentazioni in locale \hyperref[uc1.15]{(UC 1.15)};
		\item l'utente desktop può gestire il proprio archivio sul server tramite il proprio account \hyperref[uc1.7]{(UC 1.7)};
		\item l'utente può ottenere una presentazione dal server tramite il proprio account \hyperref[uc1.16]{(UC 1.16)};		
		\item l'utente può creare un'infografica della presentazione \hyperref[uc1.6]{(UC 1.6)};
		\item l'utente può modificare un'infografica \hyperref[uc1.17]{(UC 1.17)};
		\item l'utente può caricare l'infografica sul proprio spazio a disposizione sul server \hyperref[uc1.12]{(UC 1.12)};
		\item l'utente può gestire i propri template delle infografiche \hyperref[uc1.9]{(UC 1.9)}.
	\end{enumerate}
	}