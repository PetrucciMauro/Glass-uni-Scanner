\section{Obiettivi di qualità} 
\subsection{Qualità di processo}

Al fine di garantire la qualità del prodotto in ogni fase di realizzazione, si deve garantire la qualità dei processi\ped{g} che lo definiscono; per questo motivo si è deciso di utilizzare lo standard ISO/IEC 15504 denominato SPICE\ped{g}, che rende disponibili strumenti adatti a valutarli.


Tutti i processi dovranno quindi essere sottoposti a valutazione in modo da verificarne la qualità ed eventualmente facilitarne il miglioramento. A tale scopo lo SPICE definisce nove attributi di processo per effettuare una migliore valutazione:
\begin{enumerate}
\item \textbf{Process performance}\\
Gli indicatori della performance di processo sono:
\begin{itemize} 
\item I lavori identificati come input al processo (input work products);
\item I lavori identificati come output del processo (output work products);
\item Le azioni compiute per trasformare gli input work products in output work products.
\end{itemize}

\item \textbf{Performance Management}\\
L’attuazione di un processo è pianificata e controllata al fine di generare risultati che rispondono agli obiettivi attesi;
\item \textbf{Work Product Management}\\
L’attuazione di un processo è pianificata e controllata al fine di generare risultati che siano adeguatamente documentati, controllati e verificati;
\item \textbf{Process Definition}\\
L’attuazione di un processo si basa su approcci standardizzati;
\item \textbf{Process Resource}\\
Il processo può contare su adeguate risorse umane, di infrastrutture, ecc. per essere attuato;
\item \textbf{Process Measurement}\\
I risultati conseguiti e le misure rilevate durante l’attuazione di un processo sono utilizzati per assicurarsi che l’attuazione di tale processo supporti efficacemente il raggiungimento di obiettivi specifici;
\item \textbf{Process Control}\\
Un processo è controllato tramite la raccolta, analisi ed utilizzo delle misure di prodotto e di processo rilevate, con l’obbiettivo di correggere, se necessario, le sue modalità di attuazione;
\item \textbf{Process Change}\\
Le modifiche alla definizione, gestione e attuazione di un processo sono controllate;
\item \textbf{Continuous Integration}\\
Le modifiche ad un processo sono identificate ed implementate con lo scopo di assicurare il continuo miglioramento nel raggiungere gli obbiettivi definiti per l’organizzazione.
\end{enumerate}	
Sono inoltre stabiliti quattro differenti livelli di possesso di ciascuno degli attributi:\\
\begin{itemize}
\item \textbf{N - Non posseduto} (0 - 15\% di possesso): non c’è evidenza oppure ce n’è poca del possesso di un attributo;
\item \textbf{P - Parzialmente posseduto} (16 - 50\% di possesso): c’è evidenza di approccio sistematico al raggiungimento del possesso di un attributo e del raggiungimento di tale possesso, ma alcuni aspetti del possesso possono essere non prevedibili;
\item \textbf{L - Largamente posseduto} (51 - 85\% di possesso): vi è evidenza di approccio sistematico al raggiungimento del possesso di un attributo e di un significativo livello di possesso di tale attributo, ma l’attuazione del processo può variare nelle diverse unità operative dell'organizzazione;
\item \textbf{F - (Fully) Pienamente posseduto} (86 - 100\% di possesso): vi è evidenza di un totale e sistematico approccio e di un completo raggiungimento del possesso dell’attributo; non esistono significative differenze nel modo di attuare il processo tra le diverse unità operative.
\end{itemize}

Vi sono poi vari livelli di maturità dei processi che dipendono dal diverso livello di possesso degli attributi:
\begin{itemize}
\item \textbf{Livello 0} - Processo incompleto: il processo non è implementato o non raggiunge gli obiettivi. Non vi è evidenza di approcci sistematici agli attributi definiti;
\item \textbf{Livello 1} - Processo semplicemente attuato: il processo viene messo in atto e raggiunge i suoi obiettivi. Non vi è evidenza di approcci sistematici agli attributi definiti. Il raggiungimento di questo livello è dimostrato attraverso il possesso degli attributi di “Process performance”;
\item \textbf{Livello 2} - Processo gestito: il processo è attuato, ma anche pianificato, tracciato, verificato ed aggiustato se necessario, sulla base di obiettivi ben definiti. Il raggiungimento di questo livello è dimostrato attraverso il possesso degli attributi di “Performance management” e “Work product management”;
\item \textbf{Livello 3} - Processo definito: il processo è attuato, pianificato e controllato sulla base di procedure ben definite, basate sui principi del software engineering. Il raggiungimento di questo livello è dimostrato attraverso il possesso degli attributi di “Process definition” e “Process resource” ;
\item \textbf{Livello 4} - Processo predicibile: il processo è stabilizzato ed è attuato all’interno di definiti limiti riguardo i risultati attesi, le performance, le risorse impiegate ecc. Il raggiungimento di questo livello è dimostrato attraverso il possesso degli attributi di “Process measurement” e “Process control”;
\item \textbf{Livello 5} - Processo ottimizzante: il processo è predicibile ed in grado di adattarsi per raggiungere obiettivi specifici e rilevanti per l’organizzazione.
Il raggiungimento di questo livello è dimostrato attraverso il possesso degli attributi di “Process change” e “Continuous integration”.
\end{itemize}

L’applicazione dello standard ISO/IEC 15504 porta a benefici sia agli sviluppatori del software che ai suoi utilizzatori o acquirenti. Per gli sviluppatori porta vantaggi nell’ottimizzazione dell’uso delle risorse, un contenimento dei costi, una maggiore tempestività di consegna del prodotto ultimato, migliore stima dei rischi e degli impegni e la possibilità di confrontarsi con delle best practice. 
Per gli utenti invece abbiamo una maggior facilità nella selezione dei fornitori, una migliore valutazione dei rischi di progetto, controllo dello stato di avanzamento in corso d’opera, riduzione dei costi di correzione degli errori ed un controllo dei rischi e delle varianti in corso d’opera.

\subsection{Qualità di prodotto}
Per garantire la qualità del prodotto si è deciso di seguire le indicazioni fornite dallo standard ISO/IEC 9126:2001 sostituito dal successivo ISO/IEC 25010:2011. Questo documento fornisce un modello per valutare la qualità esterna (nell’ambiente di utilizzo) ed interna (indipendente dall’ambiente) di un software\ped{g}, individuando sei caratteristiche principali atte a rendere il prodotto qualitativamente accettabile.

\begin{figure}[h]
  \centering
    \includegraphics[width=0.7\textwidth]{./images/ISO-IEC_9126}
  \caption{Rappresentazione del modello ISO/IEC 9126:2001}
  \label{fig:ISO-IEC_9126}
\end{figure}


\subsubsection{Funzionalità}
È un requisito\ped{g} funzionale che indica la capacità del software\ped{g} di soddisfare le esigenze esposte dal capitolato ed individuate durante l’analisi dei requisiti\ped{g}. Per valutare questa caratteristica si considerano l'appropriatezza e l'accuratezza delle funzioni\ped{g} offerte, l'interoperabilità del prodotto rispetto ai diversi sistemi e la sicurezza offerta per la protezione dei dati.\\ 
Si sarà ottenuto un buon risultato in questo settore quando il software\ped{g} avrà superato in maniera positiva tutti i test e assicurerà copertura a tutti i requisiti\ped{g} obbligatori.

\subsubsection{Affidabilità}
È un requisito\ped{g} non funzionale che indica la capacità del software\ped{g} di svolgere correttamente il suo compito, mantenendo delle buone prestazioni anche al variare dell'ambiente nel tempo; vengono considerate la sua tolleranza agli errori, la capacità di evitare fallimenti nell’esecuzione a seguito di malfunzionamenti (detta maturità) e la recuperabilità dei dati e delle prestazioni nell'eventualità di un malfunzionamento inevitabile. Il prodotto può considerarsi affidabile se il numero di esecuzioni andate a buon fine è sufficientemente grande rispetto al numero di esecuzioni totali.

\subsubsection{Efficienza}
È un requisito\ped{g} non funzionale che indica il rapporto tra le prestazioni e le risorse\ped{g} disponibili.
Si valuta se il software\ped{g} utilizza al meglio le risorse\ped{g} a sua disposizione per fornire le funzionalità richieste, considerando il suo comportamento rispetto al tempo, ossia la velocità di risposta e d'elaborazione in determinate condizioni, che rispetto all’uso delle risorse\ped{g}, data dalla capacità d'utilizzarne una quantità adeguata ad eseguire le funzioni\ped{g} richieste. \\
Un modo per valutare l’efficienza di un software\ped{g} è calcolarne i tempi di attesa in seguito all’esecuzione di un comando, tuttavia, nel caso del prodotto Premi l'efficienza è limitata anche dallo stato della rete e dall'utilizzo di componenti grafiche quali video o immagini; per questo motivo il gruppo non può garantire tempi di risposta brevi per ogni azione compiuta dall’utente, ma si impegna a non appesantire ulteriormente tali componenti.

\subsubsection{Usabilità}
È un requisito\ped{g} non funzionale che indica la capacità del software\ped{g} di essere compreso, appreso ed usato con soddisfazione dall'utente. \\
Per far ciò il prodotto deve soddisfare condizioni di comprensibilità, apprendibilità ed operabilità; deve inoltre avere una certa attrattiva nei confronti dell'utente allo scopo di rendergliene piacevole l’utilizzo. Questa caratteristica non è facilmente misurabile in quanto non esistono metriche\ped{g} per quantificarla, perciò si farà affidamento alle linee guida del material design fornite da Google, dato l'alto tasso di adozione rispetto ad altre linee guida.

\subsubsection{Manutenibilità}
È un requisito\ped{g} non funzionale che indica la capacità del software\ped{g} di essere corretto, migliorato o adattato con impegno contenuto; a tale scopo esso deve essere facilmente analizzabile e modificabile, deve garantire stabilità a seguito di modifiche e la testabilità di tali modifiche.  \\
Per misurare questa caratteristica esistono una serie di metriche\ped{g} descritte nella sezione \ref{sec:metriche}.

\subsubsection{Portabilità}
È un requisito\ped{g} non funzionale che indica la capacità del software\ped{g} di adattarsi al cambio di dispositivo e sistema operativo, limitando la necessità di apportare cambiamenti.\\
Per soddisfare questa caratteristica, come espresso dal capitolato, è necessario che il software\ped{g} funzioni\ped{g} sia su computer (indipendentemente dal loro sistema operativo) e su dispositivi mobile\ped{g} Android\ped{g}, iOS e Windows\ped{g} Phone.

\subsection{Procedure di controllo di qualità di processo}
Per applicare il modello SPICE\ped{g} si utilizzerà il ciclo di Deming. Il ciclo di Deming è un sistema iterativo per il miglioramento continuo della qualità dei processi\ped{g} e dei prodotti da essi risultanti, che permette di riconoscere lo stato di avanzamento di un progetto\ped{g} fornendo un metodo di lavoro logico e sistematico.

\begin{figure}[h]
  \centering
    \includegraphics[width=0.5\textwidth]{./images/deming}
  \caption{Schema PDCA}
  \label{fig:deming}
\end{figure}

È chiamato anche ciclo PDCA, in quanto è definito dall'iterazione\ped{g} delle quattro fasi:
 
\begin{itemize}
\item \textbf{Plan}: si stabiliscono obiettivi e processi\ped{g} necessari ad ottenere risultati conformi agli obiettivi attesi;
\item \textbf{Do}: si implementa il piano, si esegue il processo\ped{g} e si realizza il prodotto. Si raccolgono dati da analizzare nei passi successivi;
\item \textbf{Check}: si studiano i risultati ottenuti tramite la raccolta dei dati nella fase "Do" e si paragonano con i risultati attesi (gli obiettivi stabiliti nella fase "Plan"), per verificare la presenza di incongruenze. Si evidenziano le differenze nell'implementazione rispetto al piano;
\item \textbf{Act}: se la fase di Check evidenzia che gli obiettivi fissati nel Plan e implementati nel Do rappresentano un miglioramento rispetto alla baseline precedente, si stabilisce una nuova baseline; in caso contrario la baseline non cambia. In entrambi i casi se la fase di Check ha evidenziato differenze rispetto alle aspettative, sarà necessario svolgere nuovamente il ciclo di PDCA.
\end{itemize}
Una descrizione di come il gruppo applicherà il PDCA è riportata nelle \NormeDiProgetto (\S5.1).