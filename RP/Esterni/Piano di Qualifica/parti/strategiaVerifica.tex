\section{Visione generale delle strategie di verifica}{
\subsection{Organizzazione}{
	Ogniqualvolta avvenga un cambiamento sostanziale nello sviluppo del prodotto, si istanzierà il processo\ped{g} di verifica. \\
	Nello specifico durante ogni fase (Analisi, Progettazione, Realizzazione e Validazione\ped{g}) saranno applicate le tecniche di verifica qui descritte nei seguenti casi:
	\begin{itemize}
		\item Conclusione della prima redazione di un documento;
		\item Conclusione della prima redazione di un file\ped{g} di codice\ped{g};
		\item Conclusione della modifica sostanziale di un documento: quando il versionamento passa da .x.y.z a .x.y+1.0 oppure a .x+1.0.0. Si veda per approfondimento il paragrafo relativo al versionamento nel documento \href{run:../../Esterni/\fNormeDiProgetto}{\fEscapeNormeDiProgetto};
		\item Conclusione della modifica sostanziale di un file\ped{g} di codice\ped{g}, quando cioè il versionamento passa da .x.y.z a .x.y +1.0 oppure a .x+1.0.0. Si veda per approfondimento il paragrafo relativo al versionamento nel documento \href{run:../../Esterni/\fNormeDiProgetto}{\fEscapeNormeDiProgetto}.
	\end{itemize}
	L'obiettivo delle attività di verifica è quello di trovare e rimuovere i problemi presenti. Un problema può verificarsi a vari livelli, e per ogni livello assume un nome diverso:
	\begin{itemize}
		\item Fault (difetto): è l'origine del problema, ciò che fa scaturire il malfunzionamento;
		\item Error (errore): è lo stato per cui il software\ped{g} si trova in un punto sbagliato del flusso di esecuzione o con valori sbagliati rispetto a quanto previsto dalla specifica;
		\item Failure (fallimento, guasto): è un comportamento difforme dalla specifica, cioè la manifestazione dell'errore all'utente del software\ped{g}.
	\end{itemize}
	Esiste una relazione di causa-effetto fra questi tre termini:\\
	\[DIFETTO\longrightarrow ERRORE\longrightarrow FALLIMENTO\]\\
	Non sempre un errore dà origine ad un fallimento: ad esempio potrebbero esserci alcune variabili che si trovano in stato erroneo ma non vengono lette, o non viene percorso\ped{g} il ramo di codice\ped{g} che le contiene.\\
	È necessario prestare particolare attenzione a questo tipo di errori (detti anche quiescenti), avvalendosi anche di strumenti per il rilevamento dei bug.
}
\subsection{Pianificazione strategica e temporale}{
	Al fine di rendere sistematica l'attività di verifica, per poter rispettare le scadenze fissate nel Piano di Progetto\ped{g} ed evitare la propagazione di errori all'interno dei documenti o di file\ped{g} di codice\ped{g} prima della loro verifica, la loro redazione sarà anticipata da una fase di studio preliminare. \\
	Questa fase permetterà di ridurre la necessità di grossi interventi nelle fasi successive, quando la correzione di imprecisioni concettuali e tecniche potrebbe risultare particolarmente gravosa. \\
	Come da Piano di Progetto\ped{g} di seguito si riportano le quattro milestone\ped{g} prima delle quali si effettuerà una verifica del prodotto:
	\begin{itemize}
		\item Revisioni formali:
		\begin{itemize}
			\item Revisione dei Requisiti\ped{g} (28/04/2015)
			\item Revisione di Accettazione (06/07/2015)
		\end{itemize}
		\item Revisioni di progresso:
		\begin{itemize}
			\item Revisione di Progettazione (29/05/2015)
			\item Revisione di Qualifica (18/06/2015)
		\end{itemize}
		Sarà necessario, infine, assicurarsi che ogni requisito\ped{g} sia tracciato consistentemente nel documento di Analisi dei Requisiti\ped{g}.
	\end{itemize}
	}
\subsection{Responsabilità}{
I principali ruoli di responsabilità individuati sono:
\begin{itemize}
	\item Amministratore di Progetto\ped{g}:
	\begin{itemize}
		\item Assicura la funzionalità dell'ambiente di lavoro;
		\item Redige i piani di gestione della qualità e ne verifica l'applicazione.
	\end{itemize}
	\item Responsabile del progetto\ped{g}:
	\begin{itemize}
		\item Assicura lo svolgimento delle attività di verifica;
		\item Assicura il rispetto dei ruoli e delle competenze come descritti nel Piano di Progetto\ped{g};
		\item Approva e sancisce la distribuzione di un documento o di un file\ped{g} di codice\ped{g};
		\item Assicura il rispetto delle scadenze.
	\end{itemize}
\end{itemize}
}
}
\subsection{Risorse}{
Per assicurare che gli obiettivi qualitativi vengano raggiunti è necessario l’utilizzo di risorse sia umane che tecnologiche. Per una dettagliata descrizione dei ruoli e delle loro responsabilità fare riferimento alle Norme di Progetto. Per risorse tecniche e tecnologiche sono da intendersi tutti gli strumenti software e hardware che il gruppo intende utilizzare per attuare le attività di verifica su processi e prodotti. Affinché il lavoro dei Verificatori venga agevolato si sono predisposti numerosi strumenti automatici che eseguono controlli sistematici sui prodotti generati. Tali strumenti sono descritti in modo accurato nelle Norme di Progetto.
}