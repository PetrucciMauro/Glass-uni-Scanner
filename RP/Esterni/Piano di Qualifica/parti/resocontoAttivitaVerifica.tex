\section{Riassunto delle attività di verifica}{
\subsection{Revisione dei Requisiti}{

Durante questa fase sono stati prodotti solamente documenti di testo quindi sono state applicate le tecniche di analisi statica\ped{g} descritte nella sezione \S\ref{sec:tecnicheAnalisi}.\\
Nella verifica dei documenti sono stati riscontrati soprattutto errori grammaticali e di battitura dovuti a disattenzioni durante la stesura.\\
È stato trovato anche qualche errore più grave, come il mancato rispetto delle regole di formattazione riportate nelle \NormeDiProgetto e alcune mancanze all'interno del documento di \AnalisiDeiRequisiti.\\
}
\subsection{Documenti}{
Vengono qui riportati i valori dell’indice Gulpease per ogni documento durante la fase di \textbf{Analisi}. Un documento è considerato valido soltanto se rispetta le metriche\ped{g} descritte su \S\ref{sec:metricadocumenti}.

\begin{table}[H]
	\centering
	\begin{tabular}{p{\dimexpr 0.4\linewidth-2\tabcolsep}p{\dimexpr 0.2\linewidth-2\tabcolsep}
			p{\dimexpr 0.2\linewidth-2\tabcolsep}}
		\toprule Documento & Valore indice & Esito \\
		\midrule
		\PianoDiProgetto & 89 & \textcolor{green}{Superato} \\
		\AnalisiDeiRequisiti & 91 & \textcolor{green}{Superato} \\
		\NormeDiProgetto & 75 & \textcolor{green}{Superato} \\
		\PianoDiQualifica & 82 & \textcolor{green}{Superato} \\
		\StudioDiFattibilita & 82 & \textcolor{green}{Superato} \\
		\SpecificaTecnica & & \textcolor{green}{Superato}\\
		\Glossario & 97 & \textcolor{green}{Superato} \\
		\bottomrule
	\end{tabular}
	\label{tab:costorequisiti}
	\caption{Esiti verifica documenti, Analisi}
\end{table}

Come si può notare dalla tabella, tutti gli indici Gulpease dei documenti rientrano nel range ottimale precedentemente definito e quindi i documenti redatti hanno raggiunto la leggibilità desiderata.
}
\subsection{Progettazione}
Viene qui riportata una tabella riassuntiva che riporta il calcolo dei parametri di accoppiamento afferente ed efferente per i componenti individuati nella progettazione.

\begin{table}[H]
	\centering
	\begin{tabular}{p{\dimexpr 0.7\linewidth-2\tabcolsep}p{\dimexpr 0.1\linewidth-1\tabcolsep}
			p{\dimexpr 0.1\linewidth-1\tabcolsep}}
		\toprule Componente & Afferente& Efferente \\
		\midrule
		Premi::Model & 4 & 5 \\
		Premi::Model::SlideShow & 1 & 3 \\
		Premi::Model::SlideShow::SlideShowActions::InsertEditRemove & 21 & 6 \\
		Premi::Model::SlideShow::SlideShowElements & 3 & 7 \\
		Premi::Model::SlideShow::SlideShowActions::Command & 2 & 21 \\
		Premi::Model::ServerRelations & 12 & 3\\
		Premi::Model::ServerRelations::Loader & 3 & 1 \\
		Premi::Model::ServerRelations::AccessControl & 4 & 1 \\
		Premi::Model::ServerRelations::DBConsistency & 0 & 2\\
		Premi::Model::ServerRelations::Presenter & 5 & 4 \\
		\bottomrule
	\end{tabular}
	\label{tab:accoppiamentoAffEff}
	\caption{Tabella accoppiamento afferente ed efferente delle componenti}
\end{table}

Come si può vedere dalla tabella, l’accoppiamento efferente è generalmente molto basso e quindi positivo, ad eccezione del package Command, per il quale però questo è accettato a causa della natura intrinseca di tale componente. L’accoppiamento afferente mostra invece la stabilità richiesta dalle classi del Model.
}