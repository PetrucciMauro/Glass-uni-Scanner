\section{Processi Organizzativi}
\subsection{Applicazione PDCA}
Per poter garantire un costante miglioramento dei processi il gruppo \gruppo adotterà il PDCA con le seguenti modalità:
\begin{itemize}
\item \textbf{Plan:}
  \begin{itemize}
  \item \emph{Determinare gli obiettivi e i destinatari:} Gli \emph{Analisti} con la supervisione del \emph{Responsabile di progetto} hanno formulato correttamente tutti gli obbiettivi che il gruppo dovrà ottenere. Gli obiettivi devono essere indicati in modo concreto e dettagliato e occorre fornire a tutti i membri del gruppo le informazioni necessarie. Gli obiettivi devono essere quantificati e devono riguardare  problemi che il gruppo può risolvere con la collaborazione di tutte le funzioni.
  Gli obiettivi devono essere presentati al gruppo senza limitazioni di livelli gerarchici. Quanto più il gruppo è orizzontale, e privo di frontiere, tanto più sarà facile coinvolgere i componenti nel raggiungimento degli obiettivi;
  \item \emph{Determinare i metodi per raggiungere gli obiettivi:} Per raggiungere gli obiettivi occorre mettere a punto procedure razionali e facili da seguire. Determinare un metodo significa standardizzarlo e renderlo utile e accessibile.
  Un metodo e una procedura però, non possono essere perfetti e solo l’esperienza e l’abilità dei singoli componenti possono supplire all’inadeguatezza di standard e regole. 
  \end{itemize}
 \item \textbf{Do:}
 \begin{itemize}
  \item \emph{Svolgere il lavoro:} Nessuna procedura basata su standard, ritenuti erroneamente perfetti, può garantire un’esecuzione priva di difetti. Il singolo componente del gruppo applica quanto sa e ha appreso, tenendo presenti gli standard, ma utilizzando la propria  esperienza e abilità. Il singolo componente può però applicare anche solo nel  proprio ambito un ciclo PDCA contribuendo in modo determinante al miglioramento continuo del gruppo;
  \item \emph{Formazione e istruzione:} La formazione dei componenti è indispensabile per la comprensione, applicazione e miglioramento degli standard di lavoro. La distribuzione e la delega di responsabilità, fattore insostituibile per la
  realizzazione di un sistema qualità, risulta possibile solo con componenti formati;
 \end{itemize}
  \item \textbf{Check:} 
  Lo scopo del controllo è scoprire ciò che viene realizzato in modo non accettabile e contrario ai risultati attesi. Il problema, in questo caso , diventa come scoprire le non conformità. A questo scopo vengono incontro tutti gli strumenti per l'analisi statica, l'analisi dinamica e tutti i test;
  \item \textbf{Act:}
  L’essenziale non è trovare le cause delle negatività, quanto prendere le iniziative adeguate per eliminarle. Non è sufficiente apportare modifiche ai fattori casuali individuati, occorre eliminarli. Correggere e prevenire sono due azioni diverse e
  separate. Per eliminare le cause delle criticità è necessario risalire fino alla fonte stessa del problema e prendere le misure adeguate. 
  
\end{itemize}

\subsection{Gestione di progetto}
Le responsabilità di gestione dell’intero progetto, dalla nascita alla conclusione, sono da
attribuire al \emph{Responsabile di Progetto}.
Quest’ultimo dovrà garantire un corretto sviluppo delle attività utilizzando, qualora sia possibile, degli strumenti che gli consentano di:
\begin{itemize}
\item Pianificare, coordinare e controllare le attività;
\item Gestire e controllare le risorse;
\item Analizzare e gestire i rischi;
\item Elaborare i dati.
\end{itemize}
\subsubsection{Pianificazione delle attività}
Per pianificare le attività il \emph{Responsabile di Progetto} deve realizzare un diagramma di
Gantt  per ciascuna fase indicata nella sezione Pianificazione del \href{run:../../Esterni/\fPianoDiProgetto}{\fEscapePianoDiProgetto} , utilizzando Redmine come descritto nella sezione \S\ref{sec:Pianificazione}.
\subsubsection{Coordinazione e controllo delle attività}
Per coordinare e controllare le attività il Responsabile di Progetto deve riportare la struttura creata Redmine sfruttando il suo sistema di ticketing
come descritto nella sezione \S\ref{sec:protocolloSviluppo}.
In questo modo ciascun componente del gruppo sarà avvisato delle attività ad esso assegnate e potrà inserire lo stato delle stesse permettendo al Responsabile di verificare immediatamente l’avanzamento del progetto.

\subsubsection{Gestione e controllo delle risorse}
Per gestire e controllare le risorse il Responsabile di Progetto deve utilizzare Redmine come indicato nella sezione \S\ref{sec:protocolloSviluppo} che gli consente anche di verificare l’avanzamento di ogni processo come riportato nel \href{run:../../Esterni/\fPianoDiProgetto}{\fEscapePianoDiProgetto} .
\subsubsection{Analisi e Gestione dei rischi}
Durante l’avanzamento del progetto il \emph{Responsabile di Progetto} deve monitorare costantemente il verificarsi dei rischi descritti nel \href{run:../../Esterni/\fPianoDiProgetto}{\fEscapePianoDiProgetto} ed eventuali nuovi rischi, attuando le contromisure descritte e riportando gli effettivi riscontri.

\subsubsection{Elaborazione dati}
Il Responsabile di Progetto deve sfruttare i fogli di calcolo elettronico, come descritto nella sezione \S\ref{sec:fogliDiCalcolo}, per elaborare i dati raccolti durante lo sviluppo del progetto e riportarli nel \emph{Piano di Progetto}.

\subsubsection{Delega}
\label{sec:delega}
Il \emph{Responsabile di Progetto}, nel caso in cui abbia redatto una parte di un documento, può delegare l'approvazione di tale documento ad un Verificatore.
\subsubsection{Responsabilità di sotto-progetto}
Ogni macroattività può essere assegnata dal Responsabile ad un responsabile di sottoprogetto, i cui compiti saranno l’assegnazione delle singole attività alle risorse rese disponibili e la gestione dei cambiamenti.
\subsubsubsection{Assegnazione attività}
Per assegnare attività alle risorse disponibili, il responsabile di sotto-progetto dovrà seguire le procedure di ticketing descritte in \S\ref{sec:realizzazioneControllo}.
\subsubsubsection{Gestione dei cambiamenti}
In caso di errori, in seguito alla notifica da parte del \emph{Verificatore} tramite ticket, il responsabile di sotto-progetto dovrà assegnare la correzione mediante la procedura di ticketing descritta in \S\ref{sec:realizzazioneControllo}. Al termine della correzione, sarà compito del responsabile di sotto-progetto accettare o respingere la modifica, e richiederne di conseguenza il rifacimento.
Nel caso la correzione riguardi un’attività di codifica, sarà compito del responsabile di sotto-progetto programmare una nuova esecuzione dei test di unità e di integrazione correlati al modulo modificato.

\subsection{Collaborazione}{
\subsubsection{Comunicazioni}
	\subsubsubsection{Comunicazioni interne}{
		Per le comunicazioni interne \`{e} stato aperto un gruppo privato su Facebook accessibile ai singoli membri del team. \begin{center}
			\url{https://www.facebook.com/groups/1709354699290988}
		\end{center} 
		Inoltre ogni membro del team dovr\`{a} annotare i propri impegni sullo strumento Google Calendar, il quale verr\`{a} utilizzato per segnare qualsiasi tipo di impegno: di gruppo e individuale.\\\\
		In caso di comunicazioni vocali o videoconferenze verrà utilizzato Skype.
		
	 }
	\subsubsubsection{Comunicazioni esterne}{
	Per quanto riguarda le comunicazioni esterne (verso Committente\ped{g} e/o Proponente\ped{g}) \`{e} stata creata una casella di posta elettronica dedicata gestita dal Responsabile di progetto\ped{g}: \begin{center}
		\href{mailto:\mail}{\mail} \end{center} \`{e} compito del Responsabile gestire le informazioni in entrata e in uscita avvisando il proprio gruppo e il committente\ped{g}/proponente\ped{g} di eventuali comunicazioni rispettivamente in entrata e in uscita.
		}
}

\subsubsection{Riunioni}
	\subsubsubsection{Interne}{
		\begin{itemize}
			\item Ogni membro del gruppo pu\`{o} richiedere una riunione interna tramite un post all’interno del gruppo di Facebook (tramite l’uso del tag\ped{g} [Richiesta Riunione Interna $x$] con $x$ numero incrementato di 1 rispetto alla richiesta precedente). Questa richiesta  in base alle risposte degli altri componenti verr\`{a} presa in esame dal Responsabile;
			\item Una volta valutate le motivazioni della richiesta il Responsabile controlla sul calendario del gruppo le disponibilit\`{a} dei vari componenti;
			\item Il Responsabile entro 1 giorno lavorativo pubblica una nuova discussione con tag\ped{g} [Esito Richiesta Interna x], in cui, in caso positivo annuncia orario e luogo della riunione, in caso negativo annulla o rimanda la richiesta al successivo incontro;
			\item Nel caso in cui, per diversi motivi, alla riunione non potessero presenziare pi\`{u} di due membri, si procede a fissare una nuova riunione (vedi punto 2 e seguenti).
		\end{itemize}
		\subsubsubsection{Casi Particolari}{
			Per le richieste di riunioni interne vicine (cinque giorni lavorativi) ad una milestone\ped{g}, se approvate dal Responsabile, verranno indette il giorno stesso o il seguente.
		}
	}
	\subsubsubsection{Esterne}{
		Per le riunioni esterne (quindi gli incontri con il Proponente/Committente\ped{g}) la prassi \`{e} la medesima delle riunioni interne; pu\`{o} essere avanzata da qualsiasi membro del gruppo con il tag\ped{g} [Richiesta Riunione esterna $x$].
		In questo caso il Responsabile avr\`{a} il duplice compito di valutare la richiesta dopo aver consultato il calendario e di contattare  il committente\ped{g}, per accordarsi su tempi e luogo dell’incontro, che verranno poi riferiti sulla piattaforma di comunicazioni interne tramite il tag\ped{g} [Esito Richiesta Riunione Esterna $x$].
		}
	\subsubsubsection{Esito}{
		Ad ogni riunione (sia interna che esterna) il Responsabile ha il dovere di assicurarsi che venga redatto un verbale che riassuma gli argomenti trattati durante l’incontro e tutte le eventuali decisioni prese; i membri del gruppo hanno l’obbligo di applicare le eventuali modifiche o correzioni decise durante la riunione ed \`{e} del responsabile il dovere che i problemi emersi durante il verbale siano stati risolti.
		}
		
\subsubsection{Repository e strumenti per la condivisione di file}

\subsubsubsection{Repository}
Sono stati creati due repository  Git:
\begin{itemize}


\item documents.git  : disponibile all’indirizzo\\
\begin{center}\url{https://github.com/PetrucciMauro/documents}\\\end{center}
conterrà i sorgenti \LaTeX \ e gli script necessari alla stesura dei documenti;
\item source.git : disponibile all’indirizzo\\
\begin{center}
\url{https://github.com/PetrucciMauro/source}\\
\end{center}
conterrà i sorgenti dell’applicazione.\\
\end{itemize}
Una volta terminata la fase di lavorazione di un documento, verrà creato un branch di verifica. In questo modo i Verificatori potranno lavorare parallelamente al resto del gruppo ed effettuare il merge  delle loro modifiche, una volta terminato il lavoro di verifica.
Il meccanismo di verifica e approvazione è descritto in dettaglio nella sezione \S\ref{sec:VerificaDocumenti}.


\subsubsubsection{Condivisione file}
Per la condivisione informale di file e per il lavoro collaborativo su documenti di supporto, si usa la piattaforma di condivisione file online Google Drive.
Trattandosi di strumenti informali, non si definiscono procedure rigorose d’uso e se ne lascia la descrizione alle sezioni \S\ref{sec:condivisioneFile}.