\section{Introduzione}{
	\subsection{Scopo del documento}{
		Il seguente documento viene redatto allo scopo di definire insieme di norme che regolamenteranno lo svolgimento del progetto. Le suddette norme verteranno su tutti gli aspetti del progetto:
		\begin{itemize}
        \item \textbf{Relazioni interpersonali} : comunicazione fra le varie figure professionali \item \textbf{Redazione documenti} : stili di redazione dei vari documenti interni e/o	esterni;
		\item \textbf{Codifica}: stili e convenzioni di scrittura del codice sorgente;
		\item \textbf{Procedure di automazione}: strumenti e procedure per l’automazione di attività tecniche;
		\item \textbf{Definizione dell’ambiente di lavoro}: programmi utilizzati dall’intero gruppo di progetto.
		\end{itemize}
		Tutti i membri del gruppo di progetto sottoscrivono le norme ivi contenute e vi	sottostanno, in modo da migliorare la coerenza fra i vari documenti e incrementare l' efficienza ed efficacia dei vari file prodotti.	Qualora si renda necessario, ogni componente del gruppo potrà proporre all’\emph{Amministratore di Progetto} una modifica alle \emph{Norme di Progetto}; egli, sentiti gli altri componenti del gruppo valuterà se effettuare la modifica o meno.
		
	 }
	\subsection{Glossario}{ 
	Insieme alla documentazione viene allegato il glossario dei termini (file\ped{g} \href{run:../../Esterni/\fGlossario}{\fEscapeGlossario}), che ha il compito di definire tutti i vocaboli tecnici usati, seguendo convenzione, all’interno dei vari documenti.  Ogni occorrenza di vocaboli presenti nel Glossario è marcata da una “g” minuscola in pedice.	
	}
}
   \subsection{Riferimenti}
     \subsubsection{Informativi}
     \begin{itemize}
       \item \textbf{Piano di Progetto}: \href{run:../../Interni/\fPianoDiProgetto}{\fEscapePianoDiProgetto};
       \item \textbf{Piano di Qualifica}:  \href{run:../../Esterni/\fPianoDiQualifica}{\fEscapePianoDiQualifica};
     \end{itemize}
        