\section{Processi di supporto}{

\subsection{Documentazione}{
Questo capitolo descrive tutte le convenzioni scelte ed adottate dai \gruppo\ riguardo alla stesura, verifica e approvazione della documentazione da produrre.
\subsubsection{Template} 
		Tutti i documenti devono essere realizzati utilizzando un template\ped{g} \LaTeX. Onde evitare modifiche manuali che farebbero perdere molto tempo, 
		il nome dei file\ped{g} deve rispondere alla seguente formattazione senza spazi: “[nome documento]-[versione]”. La parte della versione deve riportare la dicitura “v.” seguita dal numero di versione (ad es: NormeDiProgetto\_v.1.0.0.pdf”).\\
 		Tale modello si può trovare nel repository in documents/template.
	\label{sec:docs}
	\subsubsection{Contenuto e struttura dei documenti}{
		\label{sub:content}
		Ogni documento ufficiale deve essere composto dalle seguenti sezioni:
		\begin{itemize}
			\item Prima pagina: deve riportare titolo, logo ed informazioni del documento;
			\item Breve prefazione;
			\item Registro delle modifiche;
			\item Indice del documento;
			\item Indice di figure e tabelle (se presenti);
			\item Introduzione;
			\item Corpo.
		\end{itemize}
		Ogni pagina avrà come intestazione e come piè di pagina:
		\begin{itemize}
			\item \textbf{Intestazione}: logo del gruppo e nome del documento;
			\item \textbf{Piè di pagina}: versione documento, università e anno accademico, numeri di pagina e licenza.
		\end{itemize}
		\subsubsubsection{Verbali}
		Per quanto riguarda i verbali degli incontri, essi devono essere redatti dal Responsabile di Progetto\ped{g} ad ogni riunione. Esso deve rispettare la formattazione regolata alla sezione \S\ref{sub:typo} e successive ma è da considerarsi solo come promemoria per il gruppo.\\
		Il nome di ogni verbale deve rispettare la seguente dicitura: “Verbale\_[tipo incontro]-[data]” dove il tipo incontro può essere di due tipi:
		\begin{itemize}
			\item Interno (INT): incontro effettuato tra i membri del gruppo;
			\item Esterno (EXT): incontro effettuato tra i membri del gruppo e committente\ped{g} e/o proponente\ped{g}.
		\end{itemize}
		La prima pagina di ogni verbale deve obbligatoriamente contenere i seguenti campi, in ordine:
		\begin{itemize}
			\item Data;
			\item Luogo secondo il formato “[città],[provincia],[sede]”;
			\item Ora\ped{g} secondo il formato “dalle ore [hh]:[mm] alle ore [hh]:[mm]” dove hh indica le ore e mm i minuti i quali vanno espressi nel formato 24 ore secondo lo standard ISO\ped{g} 8601:2004;
			\item Partecipanti interni al gruppo elencandoli rispettando il formato “[nome] [cognome][,[...]]”;
			\item Partecipanti esterni al gruppo rispettando il formato “[nome] [cognome][ruolo][,[...]]” in cui il ruolo può essere Committente\ped{g} oppure Proponente\ped{g};
			\item Contenuto dell'incontro;
			\item Firme: devono essere comprese quelle di tutti i partecipanti del gruppo \gruppo a conferma della presa visione del documento.
		\end{itemize}
	 \subsubsubsection{Lettera di presentazione}
	  La lettera di presentazione deve contenere
		\begin{itemize}
			\item Logo del gruppo;
			\item Intestazione nel seguente formato:\\
					Prof. Tullio Vardanega\\
					Università degli Studi di Padova\\
					Dipartimento di Matematica\\
					Via Trieste 63\\
					35121 Padova (PD)
			\item Breve introduzione (facoltativa);
			\item Elenco di tutti i documenti in consegna;
			\item Varie ed eventuali, osservazioni (facoltative);
			\item Firma del responsabile nel seguente formato:\\
					{Nome} {Cognome}\\
					il Responsabile del gruppo \gruppo \\
					{Firma del responsabile}
		\end{itemize}		
	}
		\subsubsection{Norme tipografiche}{
			\label{sub:typo}
			Per rendere la documentazione organizzata, leggibile e standard abbiamo adottato le forme testuali riportate di seguito.
			\begin{itemize}
				\item \textbf{Carattere}: il carattere dovrà avere come dimensione minima 12. Per l'inserimento di linee di codice\ped{g} il carattere da utilizzare dovrà essere di tipo Monospace;
				\item \textbf{Grassetto}: da utilizzare maggiormente per definire i titoli e dare una panoramica generale del testo ed in maniera minore per sottolineare passaggi importanti e parole chiave;
				\item \textbf{Corsivo}: da utilizzare per riportare citazioni da fonti esterne o riferimenti;
				\item \textbf{Sottolineato}: da utilizzare all'interno del testo per dare importanza a determinati concetti;
				\item \textbf{Maiuscolo}: deve essere limitato all’indicazione di acronimi e nei casi specificati nei Formati di Riferimento (\S\ref{sub:rif});
				\item \textbf{Punteggiatura}: adottare la formattazione standard ossia la punteggiatura deve precedere sempre un carattere di spazio e non viceversa;
				\item \textbf{Lettera maiuscola}: deve seguire esclusivamente un punto, un punto esclamativo o un punto interrogativo;
				\item \textbf{Parentesi}: una qualsiasi frase racchiusa fra parentesi non deve iniziare con un carattere di spaziatura e non deve chiudersi con un carattere di punteggiatura e/o di spaziatura;
				\item \textbf{Elenchi puntati o numerati}: ogni elemento dell’elenco deve terminare con un punto e virgola, tranne l’ultimo che deve terminare con un punto. La prima parola deve avere la lettera maiuscola, a meno di casi particolari (es. nome di un file\ped{g});
				\item \textbf{Glossario}: le parole accompagnate da (g) in pedice sono quelle che presentano una corrispondenza nel Glossario;
				\item \textbf{Pagine}: è obbligatorio porre i numeri di pagina in ogni documento nel formato {n} di {totale pagine} e mantenere i margini fissati dal template\ped{g} di cui sopra (\S\ref{sub:content}).
			\end{itemize}
		}
		\subsubsection{Formati di riferimento e altro}{
			\label{sub:rif}
			Per quanto riguarda i riferimenti, è opportuno rispettare le seguenti indicazioni:
			\begin{itemize}
				\item percorsi\ped{g}: per gli indirizzi\ped{g} web\ped{g} completi e indirizzi\ped{g} e-mail deve essere utilizzato il comando appositamente fornito da \LaTeX:\\ \textbackslash url\ped{g} \textbraceleft Percorso\textbraceright;
				\item Ancore: i riferimenti alle sezioni interne del medesimo documento devono essere scritte utilizzando il comando fornito da \LaTeX: \textbackslash ref \textbraceleft label da riferire\textbraceright .
			\end{itemize}
			La \textbf{Data} deve essere espressa, seguendo lo standard ISO\ped{g} 8601:2004, nel formato: AAAA-MM-GG (AAAA rappresenta l'anno in quattro cifre, MM il mese in due cifre e GG il giorno in due cifre).\\
			\\
			Le \textbf{Abbreviazioni} ammesse sono le seguenti e valgono per tutti i documenti:
			\begin{itemize}
				\item \textbf{AR}: Analisi dei Requisiti\ped{g};
				\item \textbf{GL}: Glossario;
				\item \textbf{NP}: Norme di Progetto\ped{g};
				\item \textbf{PQ}: Piano di Qualifica;
				\item \textbf{PP}: Piano di Progetto\ped{g};
				\item \textbf{SF}: Studio di Fattibilità;
				\item \textbf{RR}: Revisione dei Requisiti\ped{g};
				\item \textbf{RP}: Revisione di Progettazione;
				\item \textbf{RQ}: Revisione di Qualifica;
				\item \textbf{RA}: Revisione di Accettazione.
			\end{itemize}
			I \textbf{Nomi ricorrenti} nei vari documenti devono rispettare le seguenti indicazioni:
			\begin{itemize}
				\item Ruoli di progetto\ped{g} e nomi dei documenti: devono essere formattati utilizzando la prima lettera maiuscola di ogni parola che non sia una preposizione (es. Responsabile di Progetto\ped{g});
				\item Nomi dei file\ped{g}: il riferimento deve essere comprensivo dell’estensione\ped{g} del file\ped{g} e formattato in corsivo;
				\item Nomi propri: l’utilizzo dei nomi propri deve seguire il formalismo Cognome Nome;
				\item Nome del gruppo: deve essere sempre espresso nel formato: \gruppo;
				\item Nome del progetto\ped{g}: deve essere sempre espresso nel formato: \premi.
			\end{itemize}
			}
		\subsubsection{Immagini e tabelle}{
			\label{sub:img}
			Tutte le immagini devono essere in formato JPG, PNG o PDF mentre ogni tabella deve rispettare il formato \LaTeX.\\
			Ogni figura o tabella inserita deve avere una breve didascalia composta da un identificativo numerico univoco seguito, ove sia ritenuto necessario, da una breve descrizione. La numerazione di immagini e tabelle sarà attribuita da \LaTeX.\\			
			}
		}
\subsubsection{Glossario}{
 	Il glossario è unico per tutti i documenti e deve essere organizzato come definito nella sezione Documenti \S\ref{sec:docs}. Tutti i membri del gruppo possono modificarlo.\\
 	I termini all'interno del glossario avranno le seguenti caratteristiche:
 	\begin{itemize}
	 	\item Tutti i termini saranno in ordine alfanumerico;
	 	\item Tutti i termini devono essere in grassetto e iniziare con la lettera maiuscola , la definizione del termine sarà preceduta dal carattere '':'' ;
	 	\item Tutti i termini devono fornire chiarimenti su concetti che possono essere confusi quindi non devono essere inseriti termini il cui significato è già noto.
 	\end{itemize}	
   \subsubsubsection{Implementazione}
      L'inserimento dei termini nel glossario viene eseguito tramite un applicazione interna al team, LateGloss, che funziona nel seguente modo:
      \begin{itemize}
	      \item Si inserisce il lemma e la descrizione del lemma negli appositi spazi;
	      \item Si salva il glossario nel formato .tex;
	      \item A tutte le parole presenti nei documenti che hanno una corrispondente definizione nel glossario  verrà aggiunto un pedice (g) per indicare che la parola è presente nel glossario.
      \end{itemize}  
   L'ordine lessicografico non è importante quando si inseriscono nuovi lemmi nel programma\ped{g} dato che vengono ordinati automaticamente.\\
   Il file\ped{g} relativo al Glossario è il seguente: \href{run:../../Esterni/\fGlossario}{\fEscapeGlossario}
}


	\subsection{Verifica}
La verifica di processi, documenti e prodotti è un'attività da eseguire continuamente durante lo sviluppo del Progetto. Di conseguenza, servono modalità operative chiare e dettagliate per i \emph{Verificatori}, in modo da uniformare le attività di verifica svolte ed ottenere il miglior risultato possibile. Si descrivono ora le modalità ordinate e puntuali di verifica di processi, documenti, attività e codice alle quali ci si riferirà in questo documento e alle quali i \emph{Verificatori} dovranno attenersi.

	\subsubsection{Metriche per gli errori riscontrati e gestione dei cambiamenti}
Si definiscono ora delle metriche per gli errori che i \emph{Verificatori} potranno trovare, fornendo criteri per la quantificazione dell’impatto sul prodotto o sul processo e per la definizione delle priorità di intervento. In questo modo si potrà agire prima nella risoluzione di errori a gravità maggiore.


\begin{longtable}[c]{|>{\centering\arraybackslash}m{6cm} | >{\centering\arraybackslash}m{3cm} | >{\centering\arraybackslash}m{3cm} | >{\centering\arraybackslash}m{3cm} |}

 \hline
 \textbf{Errore} & \textbf{Gravità} & \textbf{Priorità risoluzione} & \textbf{Modalità operative}\\
 \hline
 Indici fuori range & Alta & Urgente & Ticket\\
 \hline
 Ritardi superiori a 4-5 giorni nelle attività & Alta & Urgente & Ticket\\
 \hline
 Errato tracciamento di requisiti e casi d'uso & Alta & Urgente & Ticket\\ 
 \hline
 Errore di progettazione & Alta & Urgente & Ticket\\
 \hline
		\caption{Errori nei processi: gravità e procedure di gestione \label{tab:ErroriProcessi}}\\
\end{longtable}


\begin{longtable}[c]{|>{\centering\arraybackslash}m{6cm} | >{\centering\arraybackslash}m{3cm} | >{\centering\arraybackslash}m{3cm} | >{\centering\arraybackslash}m{3cm} |}
 
 \hline
 \textbf{Errore} & \textbf{Gravità} & \textbf{Priorità risoluzione} & \textbf{Modalità operative}\\
 \hline
 Errore ortografico o di formattazione & Bassa & Breve & Correzione immediata\\
 \hline
 Errore sistematico di ortografia o formattazione & Media & Breve & Aggiunta alla checklist\\
 \hline
 Compilazione fallita del documento & Alta & Urgente & Correzione immediata\\ 
 \hline
 Valori Gulpease fuori range & Media & Breve & Aggiunta alla checklist\\
 \hline
 Errore di concetto nel testo & Alta & Urgente & Aggiunta alla checklist\\
 \hline
 Errore di formalismo  UML  2.x & Bassa & Breve & Aggiunta alla checklist\\
 \hline
 Mancata compilazione del codice & Alta & Urgente & Correzione immediata\\ 
 \hline
 Mancato rispetto delle norme di codifica & Medio & Breve & Aggiunta alla checklist\\
 \hline
		\caption{Errori nei documenti e nel codice: gravità e procedure di gestione \label{tab:ErroriDocumentiCodice}}\\
\end{longtable}

La gravità dell’errore può essere:
\begin{itemize}
\item \textbf{Bassa} se l’errore ha impatto su aspetti marginali del prodotto o provoca un basso aumento dei costi o dei tempi del processo;
\item \textbf{Media} se l’errore ha impatto significativo sul prodotto o provoca un aumento percepibile di tempi e costi;
\item \textbf{Alta} se l’errore rende il prodotto inutilizzabile o provoca un forte aumento dei tempi o dei costi.
\end{itemize}

\begin{longtable}[c]{|>{\centering\arraybackslash}m{6cm} | >{\centering\arraybackslash}m{3cm} | >{\centering\arraybackslash}m{3cm} | >{\centering\arraybackslash}m{3cm} |}
 
 \hline
 \textbf{Ambito} & \textbf{Gravità bassa} & \textbf{Gravità media} & \textbf{Gravità alta}\\
 \hline
 Errore nel prodotto & Impatto su aspetti marginali & Impatto su aspetti visibili & Prodotto inutilizzabile\\
 \hline
 Errore nei processi & Aumento costi o tempi < 10\% & Aumento costi o tempi < 25\% & Aumento costi o tempi > 25\%\\
 \hline
		\caption{Gravità dell’errore e impatto su processi e prodotti \label{tab:GravitaErrori}}\\
\end{longtable}

La priorità di risoluzione può essere:
\begin{itemize}
\item \textbf{Breve}: indica che l’errore deve essere risolto entro 4-5 giorni;
\item \textbf{Urgente}: indica che l’errore deve essere risolto appena possibile.
\end{itemize}
Le modalità operative per il \emph{Verificatore} sono le seguenti:
\begin{itemize}
\item \textbf{Correzione immediata}: è richiesto che il \emph{Verificatore} proceda autonomamente alla correzione dell’errore;
\item \textbf{Aggiunta alla checklist}: è richiesto che il Verificatore aggiunga l’errore riscontrato ad una checklist appropriata che poi verrà assegnata a un correttore che apporterà le modifiche riportate.
\end{itemize}
\subsubsection{Verifica dei processi}
Ai Verificatori è richiesto di effettuare quanto segue:
\begin{itemize}
\item \textbf{Controllo delle metriche}: Alla conclusione di ogni fase del progetto, per ogni
macro-attività, definita nel \href{run:../../Esterni/\fPianoDiProgetto}{\fEscapePianoDiProgetto} , si calcolano gli indici definiti
nella sezione Metriche per i processi del \href{run:../../Esterni/\fPianoDiQualifica}{\fEscapePianoDiQualifica} . Al fine di avere
un indice complessivo di fase dovrà essere inoltre calcolato il valore medio di tali indici.

\item \textbf{Grafico PDCA}: Alla conclusione di ogni fase del progetto il \emph{Verificatore} dovrà esportare i dati dal sistema di ticketing utilizzando l’esportazione mediante foglio di calcolo nel formato CSV. I dati esportati devono essere inseriti in un foglio di calcolo ed importarti nel template per la generazione del grafico PDCA.\\
Dopo aver ottenuto il grafico il \emph{Verificatore} con la supervisione del \emph{Responsabile di Progetto} dovrà trarre delle conclusioni generali sulla velocità con cui sono stati portati avanti i processi.
\end{itemize}

	\subsubsection{Verifica dei documenti}
	\label{sec:VerificaDocumenti}
	Il processo di verifica sarà applicato indipendentemente dalla fase di sviluppo del prodotto (Analisi, Progettazione, Validazione) ogni qual volta avvenga un cambiamento sostanziale nello sviluppo del prodotto, ovvero nei seguenti casi:
		\begin{itemize}
			\item Conclusione della prima redazione di un documento;
			\item Conclusione della prima redazione di un file\ped{g} di codice\ped{g};
			\item Conclusione della modifica sostanziale di un documento: quando il versionamento passa da .x.y.z a .x.y+1.0 oppure a .x+1.0.0.
		\end{itemize}
	Per eseguire un'accurata verifica dei documenti redatti è necessario seguire il seguente protocollo:
	\begin{enumerate}
	\item \textbf{Controllo sintattico e del periodo}: Utilizzando TeXstudio e GNU Aspell vengono evidenziati e corretti gli errori di grammatica più evidenti. Gli errori di sintassi, di sostituzione di lettere che provocano la creazione di parole grammaticalmente corrette ma sbagliate nel contesto ed i periodi di difficile comprensione necessitano dell’intervento di un verificatore umano. Per questa ragione ciascun documento dovrà essere sottoposto ad un walkthrough da parte dei verificatori per individuare tali errori;
	\item \textbf{Rispetto delle norme di progetto}: Sono state definite norme tipografiche di carattere generale. Impongono una struttura dei documenti che non può essere verificata in maniera automatica. La verifica delle norme
	per cui non è stato definito uno strumento automatico richiede che i Verificatori eseguano inspection sul rispetto di quelle norme in ciascun documento;
	\item \textbf{Lista di controllo}: Il \emph{Verificatore} dovrà utilizzare la lista di controllo per i documenti, descritta nell’appendice \S\ref{sec:ListaControllo}, e verificare che gli errori tipici non siano	presenti;
	\item \textbf{Verifica del glossario}: Il \emph{Verificatore} si occuperà del controllo dei termini inseriti nel glossario e della corretta pedicizzazione dei termini nei vari documenti,segnalando eventuali errori;
	\item \textbf{Calcolo dell’indice Gulpease}: Su ogni documento redatto il \emph{Verificatore} deve calcolare l’indice di leggibilità. Nel caso in cui l’indice risultasse troppo basso, sarà necessario eseguire un walkthrough del documento alla ricerca delle frasi troppo lunghe o complesse;
	\item \textbf{Miglioramento del processo di verifica}: Per avere un miglioramento del processo di verifica, quando i \emph{Verificatori} eseguono walkthrough di un documento, dovranno riportare gli errori più frequentemente trovati. Grazie a tale pratica sarà possibile eseguire inspection su tali errori nelle verifiche future;
	\item \textbf{Segnalazione degli errori riscontrati}: il \emph{Verificatore} deve generare ticket secondo quando descritto nella
	sezione \S\ref{sec:TicketVerifica}.
	\end{enumerate}
	\subsubsubsection{Verifica diagrammi UML}
	\label{sec:strumentiVerifica}
Al \emph{Verificatore} è richiesto il controllo dei diagrammi UML prodotti:
\begin{itemize}

\item \textbf{Diagrammi di caso d’uso}: Il controllo dei diagrammi di caso d’uso deve avvenire manualmente, controllando il rispetto delle specifiche UML 2.x e il corretto uso delle relazioni di inclusione ed estensione. Il diagramma di caso d’uso deve rappresentare fedelmente quanto descritto dal caso d’uso;
\item \textbf{Diagrammi delle classi}: Al \emph{Verificatore} è chiesto il controllo del formalismo delle specifiche UML 2.x e di controllare la corrispondenza tra progettazione e diagrammi delle classi.

\end{itemize}

	\subsubsection{Verifica del codice}
	Al \emph{Verificatore} è richiesto l’avvio dei test statici e dinamici e l’analisi dei risultati. Di seguito un elenco degli strumenti da usare per l’analisi.
	
	\subsubsubsection{Analisi Statica} 
	\label{sec:analisiStatica}
	  \begin{itemize}
	  \item \textbf{jSHint}: tool che permette di rilevare potenziali errori nel codice\ped{g} javascript\ped{g};
	  \item \textbf{QUnit}: framework\ped{g} per i test d'unità del codice\ped{g} javascript\ped{g};
	  \item \textbf{jsmeter}: strumento per il calcolo di alcune metriche\ped{g} del codice\ped{g} javascript\ped{g}.
	  \end{itemize}
	
	\subsubsubsection{Analisi Dinamica}
	Verranno utilizzati strumenti e plugin interni al browser\ped{g} \emph{Chrome} quali \textbf{SpeedTracer} per verificare la velocità dell'applicazione web\ped{g};
	
	\subsubsubsection{Test}
	\begin{itemize}
	\label{sec:strumentiTest}
    \item \textbf{Jasmine}: framework per behavior-driven per il test sul codice javascript;
	\item \textbf{Mocha}: framework per eseguire test sul codice javascript;
	\item \textbf{Protractor}: framework per eseguire test end to end su angular.js;
	\item \textbf{Karma}: tool per l'automatizzazione dei test javascript;
	\item \textbf{Selenium}: tool per l'automatizzazione dei test sui browser.
   \end{itemize}
	
	\subsubsubsection{Validazione codice}
	
	La validazione\ped{g} del codice\ped{g} HTML e CSS\ped{g} dell’applicazione da noi sviluppata verrà
	fatta tramite il servizio W3C\ped{g} Validator32 del W3C\ped{g}.

		\subsection{Validazione requisiti}
		L'attività di validazione consiste nei seguenti attività:
		\begin{itemize}
		\item Preparare i test dei singoli requisiti e la specifica dei test per l'analisi dei risultati;
		\item Assicurarsi che i requisiti testati riflettano uno specifico uso dell'applicazione;
		\item Esecuzione dei test la quale a sua volta include:
			\begin{itemize}
				\item Stress test e casi limite;
				\item Testare il prodotto software per la sua abilità di isolare e minimizzare l'effetto degli errori; questo per verificare la robustezza e l’affidabilità del software anche negli stress test e casi limite;
				\item Testare che il prodotto software sia in grado di far svolgere all'utente tutti i suoi task.
			\end{itemize}
		\item Assicurarsi che il software Premi soddisfi gli scopi per cui è stato creato.
		\end{itemize}		
		\subsection{Gestione delle modifiche ai requisiti}{
					A tutte le proposte di modifica dei requisiti\ped{g} dovrà essere applicata la seguente procedura:
					\begin{enumerate}
						\item Deduzione, analisi e specifica dei cambiamenti;
						\item Stima dei costi del cambiamento considerando quante modifiche dovranno essere fatte ai requisiti\ped{g} e al progetto\ped{g} del sistema;
						\item Decisione ed eventuale implementazione del cambiamento nei requisiti\ped{g} e nel progetto\ped{g} di sistema.
					\end{enumerate}
					Per gestire i cambiamenti e per facilitare il tracciamento dei requisiti\ped{g} verrà usato un software\ped{g} appositamente creato dal gruppo. L’amministratore avrà il compito di gestire il server\ped{g} e amministrare i diritti di accesso degli utenti alle funzionalità fornite. In particolare gli analisti dovranno usare i modelli definiti all’inizio della fase di analisi. Per evitare problemi dovuti a modifiche concorrenti alla base dati l’amministratore dovrà garantire che ad ogni istante solo un analista possa modificare un certo sotto albero della foresta dei requisiti\ped{g} e dei test.
					}
					
