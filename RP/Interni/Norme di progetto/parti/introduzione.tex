\section{Introduzione}{
	\subsection{Scopo del documento}{
		Il seguente documento viene redatto allo scopo di definire tutto l’insieme di norme che regolamenteranno lo svolgimento del progetto. Le suddette norme verteranno su tutti gli aspetti del progetto:
		\begin{itemize}
        \item \textbf{Relazioni interpersonali} : comunicazione fra le varie figure professionali
		all’interno del gruppo di progetto;
		\item \textbf{Redazione documenti} : stili di redazione dei vari documenti interni e/o
		esterni;
		\item \textbf{Codifica}: stili e convenzioni di scrittura del codice sorgente;
		\item \textbf{Procedure di automazione}: strumenti e procedure per l’automazione
		di attività tecniche;
		\item \textbf{Definizione dell’ambiente di lavoro}: programmi utilizzati dall’intero
		gruppo di progetto.
		Tutti i membri del gruppo di progetto sottoscrivono le norme ivi contenute e vi	sottostanno, in modo da migliorare la coerenza fra i vari documenti e migliorare efficienza ed efficacia dei vari file prodotti.
		Qualora si renda necessario, un qualsiasi membro potrà proporre all’Amministratore di Progetto una modifica alle NdP il quale, sentito il parere di tutti gli altri membri del gruppo, valuterà se effettuare la modifica o meno.
	 }
	\subsection{Glossario}{ 
	Insieme alla documentazione viene allegato il glossario dei termini (file\ped{g} \href{run:../../Esterni/\fGlossario}{\fEscapeGlossario}), il quale ha il compito di definire tutti i vocaboli tecnici usati, seguendo convenzione, all’interno dei vari documenti.  Ogni occorrenza di vocaboli presenti nel Glossario è marcata da una “g” minuscola in pedice.	
	}
}
   \subsection{Riferimenti}
     \subsubsection{Informativi}
     \begin{itemize}
       \item \textbf{Jenkins}: \url{https://wiki.jenkins-ci.org/display/JENKINS/Meet+Jenkins}
       \item \textbf{Piano di progetto}: \href{run:../../Interni/\fPianoDiProgetto}{\fEscapePianoDiProgetto};
       \item \textbf{Piano di qualifica}:  \href{run:../../Interni/\fPianoDiQualifica}{\fEscapePianoDiQualifica};
     \end{itemize}
        