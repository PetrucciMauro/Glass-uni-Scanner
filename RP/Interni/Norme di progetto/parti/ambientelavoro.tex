\section{Ambiente di lavoro} 
\subsection{Risorse}{
\subsubsection{Risorse\ped{g} necessarie:}{
	\subsubsubsection{Risorse\ped{g} umane}{
		I ruoli necessari a garantire la qualità del prodotto sono:
		\begin{itemize}
			\item Responsabile di Progetto\ped{g};
			\item Amministratore;
			\item Verificatore;
			\item Programmatore. 
		\end{itemize}
	}
	\subsubsubsection{Risorse\ped{g} Hardware}{
		Saranno necessari:
		\begin{itemize}
			\item Computer con installato software\ped{g} necessario allo sviluppo del progetto\ped{g} in tutte le sue fasi;
			\item Luoghi in cui svolgere riunioni, preferibilmente dotato di connessione ad Internet.
		\end{itemize}
	}
	\subsubsubsection{Risorse\ped{g} software}{
		Saranno necessari:
		\begin{itemize}
			\item Strumenti per automatizzare i test;
			\item Framework\ped{g} per eseguire test di unità;
			\item Piattaforma di versionamento per la creazione e gestione di ticket\ped{g};
			\item Debugger per i linguaggi di programmazione scelti;
			\item Browser\ped{g} come piattaforma di testing dell'applicazione da sviluppare;
			\item Strumenti per effettuare l'analisi statica\ped{g} del codice\ped{g} per misurare le metriche\ped{g}.
		\end{itemize}
	}
}
\subsubsection{Risorse\ped{g} disponibili}{
	Sono disponibili:
	\begin{itemize}
		\item Computer personali dei membri del gruppo;
		\item Computer presenti nelle aule informatiche del Dipartimento di Matematica;
		\item Aule disponibili per incontri nel Dipartimento di Matematica;
		\item Un dispositivo Raspberry Pi 2 Model B, utilizzato come server\ped{g} per programmi\ped{g} organizzativi e di testing.
	\end{itemize}
	\subsubsubsection{Risorse\ped{g} software}{
	    \begin{itemize}
	     			\item Strumenti per il coordinamento \ref{sec:strumentiCoordinamento};
	     			\item Strumenti per i documenti \ref{sec:strumentiDocumenti};
	     			\item Strumenti per la codifica \ref{sec:strumentiCodifica};
	     			\item Strumenti verifica \ref{sec:strumentiVerifica}.
	    \end{itemize}

	}
}
}

\subsection{Sistemi Operativi}

L’intero sviluppo del progetto\ped{g} viene svolto in ambienti Unix-Like e Windows\ped{g}, nello specifico, Ubuntu\ped{g} , Mac, Windows\ped{g} . Tale scelta è maturata dopo aver appurato che le tecnologie utilizzate per lo sviluppo del progetto\ped{g} sono indipendenti dall’ambiente di sviluppo e di impiego.

\subsection{Coordinamento}
\label{sec:strumentiCoordinamento}
è stato predisposto un server\ped{g} dedicato sul quale sono installate alcune applicazioni web\ped{g}
che facilitano la gestione del progetto\ped{g}. Per connettersi al server\ped{g}, l'indirizzo\ped{g} è il seguente:\\
\begin{center}
\url{http://gioberry.no-ip.org/}
\end{center}
\subsubsection{Software\ped{g} di gestione del progetto} 
\label{subsec:Software di gestione del prodotto}
Come piattaforma di gestione del progetto\ped{g} è stato scelto \textbf{Redmine}. Redmine fornisce:
\begin{itemize}
\item Un sistema flessibile di gestione dei ticket\ped{g};
\item Il grafico Gantt delle attività;
\item Un calendario per organizzare i compiti;
\item La visualizzazione del repository\ped{g} associato al progetto\ped{g};
\item Un sistema di rendicontazione del tempo.
\end{itemize}


\subsubsection{Versionamento}


Come strumento di versionamento si è deciso di utilizzare \textbf{Git}.
Git è uno strumento di versionamento veloce e di facile apprendimento che
rappresenta uno dei migliori strumenti attualmente esistenti.\\ Per lo sviluppo collaborativo abbiamo deciso di appoggiarci al servizio \textbf{Github} che fornisce non solo un repository\ped{g} Git, ma anche strumenti utili alla collaborazione fra più persone, come il servizio di \textbf{Ticket}, \textbf{Wiki} e \textbf{Milestone}.\\
Per quanto riguarda l’uso di Git sui computer di sviluppo, si è deciso l’uso
della versione ufficiale rilasciata dal team di sviluppo di Git(2.3.3).\\
Per interfacciarsi con il repository\ped{g} viene utlizzato \textbf{SmartGit}, un client multi piattaforma che permette di utilizzare Git in maniera rapida.\\
Si descrive ora la procedura di corretto utilizzo del programma\ped{g} SmartGit .
\begin{itemize}

\item 	\textbf{Clonare il repository}: è possibile clonare il repository\ped{g} remoto in locale attraverso la seguente procedura:

\begin{itemize}
\item Premere nel menu in alto il pulsante Repository\ped{g} e successivamente Clone;
\item Nel riquadro comparso, inserire il link\ped{g} del repository\\ \url{https://github.com/PetrucciMauro/documents.git}\\
oppure\\
\url{https://github.com/PetrucciMauro/source.git}\\
successivamente premere il pulsante  next;
\item Tenere la schermata successiva con entrambi i box spuntati e premere next;
\item Selezionare la posizione in cui verrà salvata la versione locale del repository\ped{g}.
\end{itemize}
\item \textbf{Sincronizzare il repository} : Dalla schermata principale premere il pulsante pull; 
\item \textbf{Salvare una modifica in locale}: Dalla schermata principale premendo il pulsante commit\ped{g} e inserendo nell'apposita textbox un "Messaggio di commit",si salvano le modifiche effettuate ai file\ped{g};
\item \textbf{Inviare le modifiche al repository\ped{g} remoto}: Dalla schermata principale premere il pulsante Push e, successivamente alla comparsa del nuovo riquadro, ancora push, ciò comporterà l'invio delle modifiche ai file\ped{g} al repository\ped{g} remoto.

\end{itemize}

\subsubsection{Software\ped{g} di Integrazione Continua}

Si è scelto di adottare \textbf{Travis} per applicare l’integrazione continua allo sviluppo del progetto\ped{g}.\\ 
Tale software\ped{g} permette di pianificare ed eseguire dei compiti da eseguire sui file\ped{g} sorgente.\\
Mette inoltre a disposizione un cruscotto su cui è possibile visualizzare lo stato del codice\ped{g} prodotto. Tale software\ped{g} è infatti in grado di interagire con il software\ped{g} di versionamento,e se disponibile con software\ped{g} per l’esecuzione di test sul codice\ped{g} prodotto.\\ 



\subsubsection{Condivisione dei file}
  \label{sec:condivisioneFile}
  Si è inoltre scelto di utilizzare degli strumenti online che permettono di condividere file\ped{g}
  in modo semplice e veloce e che consentono di organizzare gli appuntamenti personali
  dei singoli componenti del gruppo.
\subsubsubsection{Google Drive}
  In questa piattaforma di condivisione file\ped{g} verranno salvati i documenti che:
  \begin{itemize}
  
  
  \item Non necessitano di controllo di versione ;
  \item Hanno bisogno di grande interattività tra i componenti del gruppo;
  \item Possono essere acceduti tramite l’uso di un semplice browser\ped{g}.
   \end{itemize}
  Questo strumento dovrebbe permettere a 2 o più componenti del gruppo di interagire
  lavorando sugli stessi documenti contemporaneamente. Google Drive viene utilizzato
  come strumento di supporto allo sviluppo della documentazione e del software\ped{g} presente
  su Git .
  


\subsubsection{Google Calendar}
 
Google Calendar viene utilizzato all’interno del gruppo per gestire le risorse umane\ped{g}. In
particolare tale strumento viene utilizzato per notificare in quali giorni un determinato
membro non può essere disponibile e per segnalare date rilevanti per il gruppo, come
ad esempio le date delle riunioni.
\subsection{Pianificazione}
\label{sec:Pianificazione}
Per pianificare le attività legate allo sviluppo del progetto e la gestione delle risorse si è scelto di utilizzare OmniPlan.
OmniPlan è un programma  per il project management. Tale software è stato scelto in quanto possiede le seguenti caratteristiche:
\begin{itemize}

\item Facilità d'uso;
\item Generazione dei diagrammi di gant;
\item Possibilità di esportare il progetto in più estensioni diverse.
\end{itemize}


\subsection{Strumenti per i documenti}
\label{sec:strumentiDocumenti}
\subsubsection{LATEX} 
 
Per la stesura dei documenti è stato scelto di utilizzare il sistema \LaTeX.\\
Il motivo 
principale dietro a questa scelta è la facilità di separazione tra contenuto e formattazione: 
con \LaTeX\ è possibile definire l’aspetto delle pagine in un file\ped{g} template\ped{g} condiviso da tutti i documenti. Altre soluzioni come Microsoft Office, LibreOffice o Google Docs non 
avrebbero consentito questa separazione, duplicando il lavoro di formattazione del testo 
e non garantendo un risultato uniforme.\\
Il grande numero di pacchetti\ped{g} esistenti consente di implementare funzionalità comuni 
in maniera semplice. L’estensibilità\ped{g} di \LaTeX\ può essere sfruttata per creare funzioni\ped{g} e 
variabili globali che rendono la scrittura del contenuto più corretta sotto un punto di 
vista semantico. Un esempio è dato dal comando /role\textbraceleft ruolo\textbraceright\ che identifica ogni ruolo 
all’interno del progetto\ped{g}.\\
Per la scrittura di documenti \LaTeX\  l’editor\ped{g} consigliato è \textbf{TeXstudio}. 


\subsubsection{Controllo ortografico}

Il software\ped{g} per il controllo ortografico è \textbf{Aspell}\ .\ Il programma\ped{g} viene richiamato da linea di comando.


\subsubsection{Grafici UML} 

Per la stesura dei grafici UML viene utilizzato il programma\ped{g} \textbf{Visual Paradigm}. Il programma\ped{g} viene utilizzato in licenza Community Edition la quale ne permette l’uso gratuito per fini non commerciali.

\subsubsection{Fogli di calcolo}
\label{sec:fogliDiCalcolo}
L'utilizzo di fogli di calcolo\ped{g} elettronici quali Calc,Excel e Numbers è a discrezione del singolo componente in base alla propria piattaforma utilizzata.
I fogli di calcolo vengono usati per:
\begin{itemize}
\item Grafici a torta per l'utilizzo delle risorse;
\item Grafici a torta per il costo dedicato a ciascuna risorsa;
\item Istogrammi per le ore assegnate ad ogni componente del gruppo;
\item Tabelle per il confronto tra preventivo e consuntivo;
\item Istogrammi per il confronto tra ore preventivate e ore realmente impiegate da
ciascuna risorsa.
\end{itemize}

\subsection{Strumenti per la codifica}
\label{sec:strumentiCodifica}
\subsubsection{Stesura}
Per al stesura del codice HTML e javascript verrà usata l' IDE \emph{Aptana}, la scelta sull'utilizzo di questa IDE si è basata sulle necessità del gruppo di un ambiente di sviluppo per applicazioni-web completo,gratuito e multipiattaforma.
\emph{Aptana} presenta molti strumenti già integrati, come code assistant e debugger. 
\subsubsection{Verifica}
L’analisi statica e l’analisi dinamica integrate in Travis vengono lanciate automaticamente ad ogni push di codice nel repository. Al verificatore è richiesto quindi soltanto di analizzare l’output del tool, visualizzabile dalla dashboard di Travis.




\subsection{Protocollo per lo sviluppo dell'applicazione}
\label{sec:protocolloSviluppo} 
Per procedere con uno sviluppo controllato dei documenti e del codice\ped{g} si è scelto di adottare il sistema di ticketing\ped{g} \textbf{Redmine}.\\ 
La scelta di tale software\ped{g} è descritta nella sezione \ref{subsec:Software di gestione del prodotto}.

\subsubsection{Creare un nuovo progetto} 

La creazione di un progetto\ped{g} è compito del \emph{Responsabile di Progetto}.\\ 
Un nuovo progetto\ped{g} rappresenta una macro-attività caratterizzata da molte sotto-attività supervisionate da un responsabile.\\\
Per creare un nuovo progetto\ped{g}:
\begin{itemize}
\item Aprire \textbf{Progetti}; 
\item Selezionare \textbf{Nuovo progetto}; 
\item Assegnare un \textbf{Nome} breve ma significativo; 
\item Nel caso in cui si voglia creare un sotto-progetto\ped{g} indicare il nome del progetto\ped{g} padre dall’omonimo campo; 
\item \textbf{Identificativo}: scrivere in minuscolo ed indicare codice\ped{g} della fase a cui si riferisce ;
\item Lasciare inalterati gli altri campi. 
\end{itemize}
 
\subsubsection{Creazione ticket}
 
  I ticket\ped{g} vengono creati da:
 \begin{itemize}
 

    \item \textbf{Responsabile di Progetto}: crea i ticket\ped{g} più importanti che rappresentano le macro fasi evidenziate dalla pianificazione; 
	\item \textbf{Responsabile di Sotto-progetto}: crea i ticket\ped{g} per i processi\ped{g} non pianificati inizialmente, che si evidenziano necessari per l’avanzamento del sotto-progetto\ped{g} assegnato; 
	\item \textbf{Verificatore}: crea i ticket\ped{g} per segnalare errori ed imprecisioni trovate durante il processo\ped{g} di verifica. 
 \end{itemize}


I ticket\ped{g} possono essere di tre tipologie:
\begin{itemize}


\item \textbf{Ticket\ped{g} di pianificazione}: rappresentano le macro-attività di maggiore importanza. Sono organizzate in una gerarchia con vari livelli di priorità.
 Tali attività vengono create da: 
\begin{itemize}
\item \emph{Responsabile di Progetto}: durante la pianificazione identifica le attività più importati e generali; 
\item \emph{Responsabile di Sotto-progetto}: durante lo svolgimento delle attività può scomporre in sotto-problemi l’attività indicata dal Responsabile di Progetto\ped{g}. 
\end{itemize}


\item \textbf{Ticket\ped{g} di realizzazione e controllo}: tutti i documenti redatti, durante la stesura attraversano due stadi: 
\begin{itemize}
\item \textbf{Realizzazione}: un redattore del documento effettua una prima stesura; 
\item \textbf{Controllo}: un redattore, diverso da quello della precedente fase, esegue un primo controllo sui contenuti della parte scritta. 
\end{itemize}


\item \textbf{Ticket\ped{g} di verifica}: rappresentano gli errori identificati dai Verificatori durante 
il controllo che la realizzazione dell'attività sia conforme a quanto richiesto e che 
rispetti tutte le norme.
\end{itemize}



\subsubsubsection{Ticket\ped{g} di pianificazione}

\begin{itemize}


\item Selezionare \textbf{Nuova segnalazione} da menù principale; 
\item \textbf{Tracker}: indicare la natura del ticket\ped{g}: 
	\begin{itemize}
	\item \textbf{Documento}: stesura di un documento. Il tipo di attività svolta dal redattore del documento viene definito durante la rendicontazione; 
	\item \textbf{Codifica}: stesura di codice\ped{g}; 
	\item \textbf{Verifica}: macro-attività di verifica sul prodotto dei sotto-processi\ped{g}. 


	\end{itemize}

\item \textbf{Oggetto}: descrizione breve e significativa; 
\item \textbf{Descrizione}: descrizione comprensibile e con riferimenti esterni mediante link\ped{g} se necessario; 
\item \textbf{Stato}: Plan; 
\item \textbf{Attività principale}: se si vuole creare una \textbf{sotto-attività} indicare l’id del ticket\ped{g} 
padre; 
\item \textbf{Categoria}: PDCA, solo se il ticket\ped{g} viene creato dal \emph{Responsabile di Progetto}; 
\item \textbf{Assegnato a}: inserire il nome del responsabile; 
\item \textbf{Osservatori}: aggiungere eventuali collaboratori.
\end{itemize}  


\subsubsubsection{Ticket\ped{g} di realizzazione e controllo} 
\label{sec:realizzazioneControllo}
		\begin{itemize}
		
		\item Selezionare \textbf{Nuova segnalazione} da menù principale; 
		\item \textbf{Tracker}: indicare la natura del ticket\ped{g}: 
		\begin{itemize}
	\item \textbf{Documento}: stesura di un documento. Il tipo di attività svolta dal redattore del documento viene definito durante la rendicontazione; 
	\item \textbf{Codifica}: stesura di codice\ped{g}; 
	\item \textbf{Verifica}: attività di verifica sui prodotti dei processi\ped{g}. 

	\end{itemize}

\item \textbf{Oggetto}: descrizione breve e significativa secondo il principio: nome ticket\ped{g} padre attività da svolgere (realizzazione o controllo); 
\item \textbf{Descrizione}: descrizione comprensibile e con riferimenti esterni mediante link\ped{g} se 
necessario; 

\item \textbf{Stato}: New; 
\item \textbf{Attività principale}: se si vuole creare una \textbf{sotto-attività} indicare l’id del ticket\ped{g} 
padre; 
\item \textbf{Inizio}: dare una data di inizio presunta; 
\item \textbf{Scadenza}: dare una data di fine presunta; 
\item \textbf{Assegnato a}: inserire il nome del responsabile; 
\item \textbf{Osservatori}: aggiungere eventuali collaboratori. 
\end{itemize} 

\subsubsubsection{Ticket\ped{g} di verifica}
\label{sec:TicketVerifica}

Un \emph{Verificatore} per creare un \emph{ticket\ped{g} di verifica} deve: 
\begin{enumerate}
\item Assicurarsi che esista all’interno del progetto\ped{g} l’attività \emph{Verifica}.
Su tale attività vi devono essere due sotto-attività: “Verifica - realizzazione”, 
“Verifica - approvazione”.\\
Tutti i ticket\ped{g} creati devono essere sotto-attività di: “Verifica - realizzazione”; 
\item Creare quindi il ticket\ped{g} secondo le seguenti direttive: 
		\begin{itemize}
		
		
		\item Selezionare \textbf{Nuova segnalazione} da menù principale; 
		\item \textbf{Tracker}: Bug; 
		\item \textbf{Oggetto}: descrizione breve e significativa dell’errore incontrato; 
		\item \textbf{Descrizione}: descrivere in modo dettagliato e chiaro: la natura e la posizione dell’errore; 
		\item \textbf{Stato}: New; 
		\item \textbf{Attività principale}: tutti i ticket\ped{g} devono essere figli del ticket\ped{g} “Verifica - realizzazione” del progetto\ped{g} su cui si sta eseguendo la verifica; 
		\item \textbf{Assegnato a}: inserire il nome del responsabile del progetto\ped{g} padre (es. 
		responsabile delle \emph{Norme di Progetto}). 
		\end{itemize}
Tutti i campi non segnalati sono da lasciare vuoti. 
Sarà poi compito del responsabile del progetto\ped{g} padre decidere a chi assegnare la correzione dell’errore. Nel caso in cui l’errore segnalato non sia considerato valido dal 
\emph{Responsabile del sotto-progetto} verrà confermato il rifiuto dal \emph{Responsabile di Progetto}. 

\end{enumerate}


\subsubsubsection{Dipendenze temporali}


Dopo la creazione del ticket\ped{g}, per aggiungere \textbf{dipendenze temporali} tra i ticket\ped{g}:
\begin{itemize}
\item Andare su \textbf{segnalazioni}; 
\item Aprire il link\ped{g} alla segnalazione a cui aggiungere la dipendenza; 
\item Nella sezione \textbf{segnalazioni correlate} premere \textbf{aggiungi}; 
\item Scegliere \textbf{segue} e indicare il numero della segnalazione che lo blocca ed eventuali giorni di slack. 

\end{itemize} 

Tutti i campi non segnalati sono da lasciare vuoti. 

\subsubsection{Aggiornamento ticket}

Esistendo due tipologie di ticket\ped{g}, viene qui definito la procedura per effettuare l’aggiornamento di entrambe.

\subsubsubsection{Ticket\ped{g} di pianificazione}

\begin{itemize}
\item Andare sul menù \textbf{Segnalazioni}; 
\item Selezionare il ticket\ped{g} di interesse; 
\item Cliccare il link\ped{g} \textbf{Aggiorna}; 
\item Commentare ciò che si è fatto sulla form \textbf{Note}; 
\item Cambiare lo stato del ticket\ped{g} secondo la seguente logica:
		\begin{itemize}
		\item \textbf{Do}: quando un ticket\ped{g} è in questo stato indica che una o più persone stanno 
		lavorando su tale attività; 
		\item \textbf{Check}: quando un ticket\ped{g} è in questo stato indica che una o più persone 
		stanno lavorando sulla verifica di tale attività; 
		\item \textbf{Act}: l’attività è stata conclusa e verificata, e ne sono state tratte le conclusioni adeguate. 
		
		\end{itemize} 

\item Se viene concluso, aggiornare lo stato del ticket\ped{g} di pianificazione padre. 

\end{itemize}


\subsubsubsection{Ticket\ped{g} di realizzazione e controllo}

\begin{itemize}
\item Andare sul menù \textbf{Segnalazioni}; 
\item Selezionare il ticket\ped{g} di interesse; 
\item Cliccare il link\ped{g} \textbf{Aggiorna}; 
\item Indicare il tempo impiegato in ore; 
\item Indicare il tipo di attività svolta; 
\item Commentare ciò che si è fatto sulla form \textbf{Note}; 
\item Cambiare lo stato del ticket\ped{g} secondo la seguente logica: 
		\begin{itemize}
		\item \textbf{In Progress}: quando un ticket\ped{g} è in questo stato indica che una o più persone 
		stanno lavorando su tale attività. La percentuale di completamento deve 
		essere impostata tra lo 0\% ed il 90\%; 
		\item \textbf{Closed}: l’attività è stata conclusa. La percentuale di completamento dell’attività è al 100\%. 
		 
		\end{itemize} 
\item Aggiornare lo stato del ticket\ped{g} di pianificazione padre secondo tali principi: 
		\begin{itemize}
		\item Ticket\ped{g} figlio passa da New a In Progress: il ticket\ped{g} padre passa da Plan a Do, 
		o da Do a Check; 
		\item Ticket\ped{g} figlio passa a Closed: il ticket\ped{g} padre deve essere in Do o Check; 
		\item Tutti i ticket\ped{g} figli vengono chiusi: il ticket\ped{g} padre passa ad Act.
		\end{itemize}

\end{itemize}



\subsubsubsection{Ticket\ped{g} di verifica}

\begin{itemize}
\item Andare sul menù \textbf{Segnalazioni}; 
\item Selezionare il ticket\ped{g} di interesse; 
\item Cliccare il link\ped{g} \textbf{Aggiorna}; 
\item Indicare il tempo impiegato in ore; 
\item Indicare Verifica come tipo di attività svolta; 
\item Commentare le correzione nella form \textbf{Note}; 
\item Cambiare lo stato del ticket\ped{g} secondo la seguente logica:
		\begin{itemize}
		\item \textbf{In Progress}: quando un ticket\ped{g} è in questo stato indica che una o più persone 
		stanno lavorando su tale attività. La percentuale di completamento deve 
		essere impostata tra lo 0\% ed il 90\%; 
		\item \textbf{Closed}: l’attività è stata conclusa. La percentuale di completamento dell’attività è al 100\%; 
		\item \textbf{Rejected}: l’attività di verifica è stata rifiutata dal \emph{Responsabile del sotto progetto} in accordo con il \emph{Responsabile di Progetto}. 
		
		\end{itemize}

\item Aggiornare lo stato del ticket\ped{g} di pianificazione padre secondo tali principi:
		\begin{itemize}
		\item Ticket\ped{g} figlio passa da New a In Progress: il ticket\ped{g} padre passa da Plan a Do, 
		o da Do a Check; 
		\item Ticket\ped{g} figlio passa a Closed: il ticket\ped{g} padre deve essere in Do o Check; 
		\item Tutti i ticket\ped{g} figli vengono chiusi: il ticket\ped{g} padre passa ad Act. 
		
		\end{itemize} 

\end{itemize} 
\begin{figure}[H]
  \centering
    \includegraphics[width=0.60\textwidth]{\imgs CreazioneTicket}
  \caption{Diagramma attività - Creazione nuovo ticket}
  \label{fig:CreazioneTicket}
\end{figure}

\begin{figure}[H]
  \centering
    \includegraphics[width=0.75\textwidth]{\imgs AggiornamentoTicket}
  \caption{Diagramma attività -Aggiornamento ticket}
  \label{fig:gull}
\end{figure}




\subsubsection{Consigli di utilizzo}
 
	Per avere una immediata visualizzazione dei ticket\ped{g} assegnati, è consigliato personalizzare 
	la pagina personale: 
	\begin{itemize}
		\item Andare alla \textbf{Pagina personale}; 
		\item Cliccare il link\ped{g} \textbf{Personalizza la pagina}; 
		\item Dal menù a tendina \textbf{La mia pagina di blocco}, selezionare \textbf{Le mie segnalazioni} 
		e premere il pulsante verde +; 
		\item Ripetere il punto precedente per aggiungere \textbf{Segnalazioni osservate}. 
	
	\end{itemize}

	Per avere una visualizzazione più chiara delle segnalazioni si consiglia di ordinarle per 
	oggetto. Tale risultato può essere ottenuto premendo \textbf{Oggetto} dalla pagina \textbf{Segnalazioni}.

\subsection{LateTack}
\label{sec:lateTrack}
Per semplificare il tracciamento dei requisiti è stato realizzato internamente il software
LateTrack, un’applicazione web accessibile tramite browser ed ospitata sul server dedicato.
Quest’applicazione consente di gestire il tracciamento dei requisiti

\begin{figure}[H]
\includegraphics[width=1\textwidth]{\imgs LateTrack.png}
  \caption{ScreenShot del programma LateTrack}
  \label{fig:lateTrack}
\end{figure}
\subsubsection{Aggiunta nuovo requisito}
Per aggiungere un nuovo requisito al progetto è necessario eseguire i passi descritti nel
diagramma in figura \ref{fig:CreazioneRequisito}.
\begin{figure}
\includegraphics[width=0.5\textwidth]{\imgs CreazioneRequisito}
  \caption{Creazione nuovo requisito}
  \label{fig:CreazioneRequisito}
\end{figure}