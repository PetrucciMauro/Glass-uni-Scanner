\section{Processo di sviluppo}
	\subsection{Analisi dei Requisiti}{ 
	   \subsubsection{Fattibilità}{
	    		A partire da informazioni preliminari sul capitolato, lo studio di fattibilità dovrà generare un rapporto che indichi la convenienza o meno del gruppo nello sviluppo del sistema. In particolare si dovrà considerare:
	    		\begin{enumerate}
	    			\item Sufficienza di risorse umane\ped{g};
	    			\item Rapporto tra i costi ed i benefici;
	    			\item Rischi individuati.
	    		\end{enumerate}
	    		Nello stimare i benefici dovrà essere data molta importanza alle competenze che i membri del gruppo acquisirebbero nello sviluppo del sistema.
	    	 }
		\subsubsection{Scoperta dei requisiti}{
			\subsubsection{Interviste}
			Al fine di evitare interviste infruttuose verrà preparato un elenco di punti da sottoporre al proponente\ped{g} in modo da dare una direzione precisa all’intervista. Potrebbe essere utile discutere con il proponente\ped{g} dei casi d'uso\ped{g} analizzati internamente al gruppo durante la fase di analisi.
			Le richieste di interviste al proponente\ped{g} avverranno con le modalità descritte in ”comunicazioni esterne”. Durante ogni intervista dovrà essere scritta una minuta che sarà confermata dal proponente\ped{g}, eventualmente con le opportune modifiche. La minuta sarà confermata al termine dell’incontro. Quando non fosse un problema per il proponente\ped{g} l’audio dell’intervista dovrà essere registrato per favorire la futura fase di analisi.
			\subsubsection{Riunioni interne e casi d'uso}
			Individualmente e durante le riunioni interne gli analisti dovranno analizzare le informazioni raccolte dalle interviste con il proponente\ped{g} per individuare problemi e fonti da cui attingere i requisiti\ped{g}.\\
			L’individuazione dei requisiti\ped{g} funzionali sarà guidata dai casi d’uso. I casi d’uso potranno avere rappresentazione a diagrammi ma ogni caso d’uso dovrà avere anche la rappresentazione testuale. In particolare nella rappresentazione testuale si definirà:
			\begin{enumerate}
				\item Identificativo;
				\item Attore primario;
				\item Precondizioni;
				\item Postcondizioni;
				\item Scenario principale;
				\item Estensioni\ped{g}.
			\end{enumerate}
			Per la sintassi si rimanda a ”Dall’idea al codice\ped{g} con UML2.0, Luciano Baresi, Luigi Lavazza, Massimiliano Pianciamore”.
			}
			\subsubsection{Classificazione e priorità}{
				I requisiti\ped{g} dovranno essere classificati in:
				\begin{enumerate}
					\item Requisiti\ped{g} di processo\ped{g};
					\item Requisiti\ped{g} di prodotto.
				\end{enumerate}
				I requisiti\ped{g} di prodotto saranno classificati in base a:
				\begin{enumerate}
					\item Tipologia;
					\item Importanza;
					\item Provenienza.
				\end{enumerate}
				Dove i gradi di importanza saranno:
				\begin{itemize}
						\item \{\textbf{Obbligatorio}\}: requisito da considerarsi \textbf{irrinunciabile} per il cliente. Senza di esso l’applicazione è da considerarsi non soddisfacente per il cliente;
						\item \{\textbf{Desiderabile}\}: requisiti non strettamente necessari, ma che apportano valore aggiunto importante al prodotto;
						\item \{\textbf{Opzionale}\}: requisito relativamente utile/importante o che potrebbe essere soggetto di ulteriore contrattazione.
				\end{itemize}
				La provenienza può essere:
				\begin{itemize}
					\item \{\textbf{Capitolato}\}: da capitolato;
					\item \{\textbf{Interni}\}: da analisi interna;
					\item \{\textbf{Proponente}\}: da incontro con proponente\ped{g}.
				\end{itemize}
				Mentre le tipologie saranno:
				\begin{itemize}
					\item \{\textbf{RF}\}: requisito funzionale, determina le capacità richieste al sistema;
					\item \{\textbf{RQ}\}: requisito\ped{g} di qualità, requisito volto a portare valore aggiunto al sistema;
					\item \{\textbf{RV}\}: requisito\ped{g} di vincolo, requisiti espressamente indicati nel capitolato d’appalto o nei verbali d'incontro con il Proponente o Committente.
				\end{itemize}
			}
			\subsubsection{Specifica}{
				Nella specifica dei requisiti\ped{g} dovrà essere considerato come riferimento lo standard IEEE 830-1998. In particolare saranno da perseguire le seguenti caratteristiche dei requisiti\ped{g}:
				\begin{enumerate}
					\item Non ambigui;
					\item Corretti;
					\item Completi;
					\item Verificabili;
					\item Consistenti;
					\item Modificabili;
					\item Tracciabili;
					\item Ordinati per rilevanza.
				\end{enumerate}
				I requisiti\ped{g} dovranno essere specificati in un documento ”Analisi dei requisiti” secondo la struttura definita nello standard IEEE 830-1998. La specifica dei requisiti\ped{g} dovrà essere documentata in forma tabellare per evitare ambiguità. Per ogni requisito\ped{g} dovranno essere definiti un codice\ped{g}, una descrizione, un riferimento alla fonte e un riferimento alla verifica. Al fine di rendere meno ambigui i requisiti\ped{g} sara redatto un ”Glossario” contenente la definizione di tutti i termini non ovvi usati in fase di analisi.
			}
			\subsubsection{Verifica dei requisiti}{
				Per ogni requisito\ped{g} di processo\ped{g} specificato dovrà essere presente in ”Piano di qualifica” un riferimento alle sezioni di ”Norme di progetto” in cui viene assicurato il soddisfacimento del requisito\ped{g}. Per ogni requisito\ped{g} di prodotto specificato dovrà essere descritto brevemente il metodo che verrà usato per verificarne il soddisfacimento.\\Per favorire la tracciabilità tra requisiti\ped{g} e metodi di verifica dovrà essere presente in ”Piano di qualifica” una tabella in cui si definiscono: codice\ped{g} di requisito\ped{g}, codice\ped{g} di verifica e modalità di verifica. Se il requisito\ped{g} è di processo\ped{g}, la modalità di verifica conterrà i riferimenti alle sezioni corrispondenti in ”Norme di progetto”.
			}
		}
		\subsection{Validazione\ped{g} dei requisiti}{
			\subsubsection{Interna}{
				Saranno verificate la correttezza e la completezza dei requisiti\ped{g} rispetto ai bisogni. Ciò verrà fatto tramite tracciamento tra specifica dei requisiti\ped{g} e bisogni individuati.\\Saranno verificate la correttezza e la completezza dei metodi di verifica dei requisiti\ped{g}
				rispetto ai requisiti\ped{g}. Ciò verrà fatto tramite tracciamento tra specifica dei requisiti\ped{g} e metodi di verifica.
			}
				\subsubsection{Esterna}{
					Terminata la validazione\ped{g} interna verranno presentati al proponente\ped{g} i documenti ”Analisi dei requisiti” e ”Piano di qualifica”, se accettati costituiranno una baseline per la fase successiva del progetto\ped{g} altrimenti verranno gestite le richieste di modifica secondo i metodi descritti in ”Gestione delle modifiche ai requisiti”.
				}
		}
		\subsection{Gestione delle modifiche ai requisiti}{
			A tutte le proposte di modifica dei requisiti\ped{g} dovrà essere applicata la seguente procedura:
			\begin{enumerate}
				\item Deduzione, analisi e specifica dei cambiamenti;
				\item Stima dei costi del cambiamento considerando quante modifiche dovranno essere fatte ai requisiti\ped{g} e al progetto\ped{g} del sistema;
				\item Decisione ed eventuale implementazione del cambiamento nei requisiti\ped{g} e nel progetto\ped{g} di sistema.
			\end{enumerate}
			Per gestire i cambiamenti e per facilitare il tracciamento dei requisiti\ped{g} verrà usato un software\ped{g} appositamente creato dal gruppo. L’amministratore avrà il compito di gestire il server\ped{g} e amministrare i diritti di accesso degli utenti alle funzionalità fornite. In particolare gli analisti dovranno usare i modelli definiti all’inizio della fase di analisi. Per evitare problemi dovuti a modifiche concorrenti alla base dati l’amministratore dovrà garantire che ad ogni istante solo un analista possa modificare un certo sotto albero della foresta dei requisiti\ped{g} e dei test.
			}
			
   \subsection{Progettazione}
	\subsubsection{Progettazione architetturale}

\subsubsubsection{Attività}
Lo scopo di questa attività è quello di realizzare una visione globale di ciò che dovrà essere il sistema a fronte dei requisiti ricavati dall’attività di analisi.
Terminata l’attività di progettazione architetturale si deve produrre un documento completo ed esplicativo: la \emph{Specifica Tecnica}.
Le attività necessarie alla redazione del documento sono:
\begin{itemize}
\item Definizione dell’architettura di prodotto a partire dall’Analisi dei Requisiti;
\item Individuazione e studio dei design pattern applicabili;
\item Individuazione della struttura dei package;
\item Individuazione delle classi che compongono il sistema e delle relazioni tra esse;
\item Analisi delle tecnologie da adottare;
\item Studio di fattibilità;
\item Tracciamento componenti-requisiti.
\end{itemize}
Si deve inoltre definire, in un documento specifico  (\href{run:../../Esterni/\fPianoDiQualifica}{\fEscapePianoDiQualifica}), vari
test da eseguire sulle parti del sistema per verificarne la corretta interazione:
\begin{itemize}
\item \textbf{Input}: Analisi dei Requisiti;
\item \textbf{Output}: Specifica Tecnica, test di integrazione;
\item \textbf{Risorse}: Progettisti, documentazione, strumentazione;
\item \textbf{Misurazioni}: avanzamento dell’elaborazione del documento \emph{Specifica Tecnica} rispetto alla totalità dei requisiti definiti nel documento di Analisi dei Requisiti;
\item \textbf{Norme}: descritte in seguito.
\end{itemize}

		\subsubsubsection{Diagrammi UML}
		Data la visione a livello medio-alto di dettaglio richiesta per questo documento si dovranno utilizzare schemi UML 2.x in grado di descrivere formalmente i vari componenti	del sistema.
		In particolare si andrà ad utilizzare i seguenti tipi di diagrammi:
		\begin{itemize}
		
		
		\item \textbf{Diagrammi di package}: saranno utilizzati per raggruppare più elementi UML aventi funzionalità simili. Ogni package dovrà essere identificato da un	nome che risulti completamente qualificato e univoco all’interno dello spazio
		dei nomi. Schemi di questo tipo sono utili per individuare le dipendenze tra classi e per stimare la complessità strutturale del sistema.
		\item \textbf{Diagrammi di classe}: utilizzati per descrivere i tipi di oggetti che fanno parte di un sistema e le relazioni che vi sono tra di essi. Nella prima fase di progettazione non è richiesto l'elenco di tutti gli attributi e i metodi. Durante
		la progettazione di dettaglio si consiglia di riportare nei diagrammi da inserire nei documenti anche tutti gli attributi e i metodi. Per garantire una buona	leggibilità dello schema si consiglia di valutare l'inserimento degli elementi di una classe in base al loro numero. E' possibile omettere gli elementi di una classe anche nel caso in cui lo schema debba riportare un numero elevato di classi. Anche il livello di dettaglio della segnatura dei metodi è a discrezione	del progettista con l’indicazione di considerare la seguente lista di priorità:
		\begin{itemize}		
		\item Nome del metodo;
		\item Livello di accessibilità;
		\item Tipo di ritorno, tipo dei parametri in ingresso ed eccezioni lanciabili;
		\item Nome dei parametri in ingresso.
		\end{itemize}
		\item \textbf{Diagrammi di sequenza}: utilizzati per descrivere la collaborazione tra più oggetti che hanno lo scopo di implementare collettivamente un comportamento. Non sono adatti per la modellazione della logica di controllo e vanno preferiti i diagrammi di attività se si intende modellare dei cicli o delle condizioni;
		\item \textbf{Diagrammi di attività}: descrivono la procedura logica con la quale vengono eseguite delle operazioni. Vanno utilizzati quando si vuole descrivere	l’esecuzione di flussi paralleli.
        \end{itemize}
	\subsubsubsection{Design pattern}
	I \textit{Progettisti} devono descrivere i design pattern utilizzati per realizzare l'architettura:
	di essi si deve includere una breve descrizione e un diagramma che ne esemplifichi il funzionamento e la struttura.
	\subsubsubsection{Tracciamento componenti}
	Ogni requisito deve essere tracciato al componente che lo soddisfa. Il software LateTrack genera automaticamente le tabelle di tracciamento come descritto nella sezione \ref{sec:lateTrack}. In questo modo sarà possibile misurare il progresso nell'attività di progettazione e garantire che ogni requisito venga soddisfatto.
	\subsubsubsection{Definizione di Prodotto}
	I Progettisti devono produrre la Definizione di Prodotto dove viene descritta la progettazione di dettaglio del sistema ampliando quanto scritto nella Specifica Tecnica.
	

	\subsubsubsection{Test}
	\begin{itemize}
	
	
\item \textbf{Test di unità}: test che si effettuano per ogni unità del software con il massimo grado di parallelismo;
\item \textbf{Test di integrazione}: verifica dei componenti formati dall’integrazione delle varie unità che hanno passato il test di unità;
\item \textbf{Test di sistema e di collaudo}: verifica che il sistema in cui andrà installato il software rispetti i requisiti richiesti, o che il software riesca ad adattarsi correttamente al contesto dell’azienda proponente. Il collaudo sarà sul software installato, finito il quale avverrà il rilascio del prodotto;
\item \textbf{Test di regressione}: nel caso di una modifica ad un singolo componente,
andranno effettuati nuovamente tutti i test di unità e, se necessario, di
integrazione riferiti a quel componente.
\end{itemize}

	
	\subsection{Codifica}
	Le convenzioni di codifica che tutti i membri del gruppo devono seguire sono quelle
	specificate alla seguente pagina :\\
	\begin{center} \url{http://www.w3schools.com/js/js_conventions.asp} \end{center}
	\subsubsection{Nomi}
	\begin{itemize}
	\item I nomi di variabili, metodi e funzioni dovranno essere espressi in dromedaryCase;
	\item I nomi delle classi dovranno essere espressi in CamelCase;
	\item nomi di variabili globali e costanti dovranno essere in UPPERCASE.
	\end{itemize}
	\subsubsection{Documentazione}
	I file contenenti codice dovranno essere provvisti di un'intestazione contenente:
	\begin{lstlisting}
    /*!
  	* \file Nome del file
  	* \author Autore (indirizzo email dell'autore)
  	* \date Data di creazione
  	* \brief Breve descrizione del file
  	*
  	* Descrizione dettagliata del file
  	*/
  	
  	Prima di ogni metodo dovra essere presente un commento contenente:
  	/*!
  	* \brief Breve descrizione della funzione
  	* \param Nome del primo parametro
  	* \param Nome del secondo parametro
  	* \return Valore ritornato dalla funzione
  	*/
    \end{lstlisting}