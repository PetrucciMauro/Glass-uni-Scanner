\section{Collaborazione}{
\subsection{Comunicazioni}
	\subsection{Comunicazioni interne}{
		Per le comunicazioni interne \`{e} stato aperto un gruppo privato su Facebook accessibile ai singoli membri del team. \begin{center}
			\url{https://www.facebook.com/groups/1709354699290988}
		\end{center} 
		Inoltre ogni membro del team dovr\`{a} annotare i propri impegni sullo strumento Google Calendar, il quale verr\`{a} utilizzato per segnare qualsiasi tipo di impegno: di gruppo e individuale.\\\\
		In caso di comunicazioni vocali o videoconferenze verrà utilizzato Skype.
		
	 }
	\subsection{Comunicazioni esterne}{
	Per quanto riguarda le comunicazioni esterne (verso Committente\ped{g} e/o Proponente\ped{g}) \`{e} stata creata una casella di posta elettronica dedicata gestita dal Responsabile di progetto\ped{g}: \begin{center}
		\href{mailto:\mail}{\mail} \end{center} \`{e} compito del Responsabile gestire le informazioni in entrata e in uscita avvisando il proprio gruppo e il committente\ped{g}/proponente\ped{g} di eventuali comunicazioni rispettivamente in entrata e in uscita.
		}
}

\subsection{Riunioni}
	\subsubsection{Interne}{
		\begin{itemize}
			\item Ogni membro del gruppo pu\`{o} richiedere una riunione interna tramite un post all’interno del gruppo di Facebook (tramite l’uso del tag\ped{g} [Richiesta Riunione Interna $x$] con $x$ numero incrementato di 1 rispetto alla richiesta precedente). Questa richiesta  in base alle risposte degli altri componenti verr\`{a} presa in esame dal Responsabile;
			\item Una volta valutate le motivazioni della richiesta il Responsabile controlla sul calendario del gruppo le disponibilit\`{a} dei vari componenti;
			\item Il Responsabile entro 1 giorno lavorativo pubblica una nuova discussione con tag\ped{g} [Esito Richiesta Interna x], in cui, in caso positivo annuncia orario e luogo della riunione, in caso negativo annulla o rimanda la richiesta al successivo incontro;
			\item Nel caso in cui, per diversi motivi, alla riunione non potessero presenziare pi\`{u} di due membri, si procede a fissare una nuova riunione (vedi punto 2 e seguenti).
		\end{itemize}
		\subsubsubsection{Casi Particolari}{
			Per le richieste di riunioni interne vicine (cinque giorni lavorativi) ad una milestone\ped{g}, se approvate dal Responsabile, verranno indette il giorno stesso o il seguente.
		}
	}
	\subsubsection{Esterne}{
		Per le riunioni esterne (quindi gli incontri con il Proponente/Committente\ped{g}) la prassi \`{e} la medesima delle riunioni interne; pu\`{o} essere avanzata da qualsiasi membro del gruppo con il tag\ped{g} [Richiesta Riunione esterna $x$].
		In questo caso il Responsabile avr\`{a} il duplice compito di valutare la richiesta dopo aver consultato il calendario e di contattare  il committente\ped{g}, per accordarsi su tempi e luogo dell’incontro, che verranno poi riferiti sulla piattaforma di comunicazioni interne tramite il tag\ped{g} [Esito Richiesta Riunione Esterna $x$].
		}
	\subsubsection{Esito}{
		Ad ogni riunione (sia interna che esterna) il Responsabile ha il dovere di assicurarsi che venga redatto un verbale che riassuma gli argomenti trattati durante l’incontro e tutte le eventuali decisioni prese; i membri del gruppo hanno l’obbligo di applicare le eventuali modifiche o correzioni decise durante la riunione ed \`{e} del responsabile il dovere che i problemi emersi durante il verbale siano stati risolti.
		}
		
\subsection{Repository e strumenti per la condivisione di file}

\subsubsection{Repository}
Sono stati creati due repository  Git:
\begin{itemize}


\item documents.git  : disponibile all’indirizzo\\
\begin{center}\url{https://github.com/PetrucciMauro/documents}\\\end{center}
conterrà i sorgenti \LaTeX \ e gli script necessari alla stesura dei documenti;
\item source.git : disponibile all’indirizzo\\
\begin{center}
\url{https://github.com/PetrucciMauro/source}\\
\end{center}
conterrà i sorgenti dell’applicazione.\\
\end{itemize}
Una volta terminata la fase di lavorazione di un documento, verrà creato un branch di verifica. In questo modo i Verificatori potranno lavorare parallelamente al resto del gruppo ed effettuare il merge  delle loro modifiche, una volta terminato il lavoro di verifica.
Il meccanismo di verifica e approvazione è descritto in dettaglio nella sezione----da aggiungere----


\subsubsection{Condivisione file}
Per la condivisione informale di file e per il lavoro collaborativo su documenti di supporto, si usa la piattaforma di condivisione file online Google Drive.
Trattandosi di strumenti informali, non si definiscono procedure rigorose d’uso e se ne lascia la descrizione alle sezioni ---da completare---.