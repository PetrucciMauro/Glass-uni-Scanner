\section{Ruoli di progetto}
Durante lo sviluppo del progetto vi saranno diversi ruoli che i membri del gruppo andranno a ricoprire. Tali ruoli rappresentano figure aziendali specializzate, indispensabili per il buon esito del progetto. Ciascun componente del gruppo dovrà ricoprire almeno una volta ogni ruolo(vincolo organigramma). Si deve inoltre certificare che non vi siano conflitti di interesse nello svolgimento delle attività di verifica e di approvazione.\\
Per garantire che la rotazione dei ruoli non provochi conflitti è necessario che le attività di stesura e verifica vengano pianificate dettagliatamente e che i soggetti interessati rispettino i compiti a loro assegnati. Sarà poi compito del \textit{Verificatore} controllare attentamente il diario delle modifiche di ogni documento per individuare eventuali incongruenze.\\
Si descrivono ora i diversi ruoli di progetto, con le relative responsabilità e le modalità operative affinché essi possano svolgere i compiti assegnati con l'ausilio dei software scelti per il progetto.
\subsection{Responsabile di Progetto}
Il \textit{Responsabile di Progetto} rappresenta il progetto, in quanto accentra su di sé le responsabilità di scelta ed approvazione, ed il gruppo, in quanto presenta al committente i risultati del lavoro svolto.
Detiene il potere decisionale, quindi la responsabilità su:
\begin{itemize}
\item Pianificazione, coordinamento e controllo delle attività;
\item Gestione e controllo delle risorse;
\item Analisi e gestione dei rischi;
\item Approvazione dei documenti;
\item Approvazione dell'offerta economica.
\end{itemize}
Di conseguenza, ha il compito di assicurarsi che le attività di verifica vengano svolte sistematicamente seguendo le \textit{Norme di Progetto}, vengano rispettati i ruoli e le competenze assegnate nel \textit{Piano di Progetto}, non vi siano conflitti di interesse tra redattori e verificatori. Egli è l'unico a poter decidere l'approvazione di un documento e a sancirne
la distribuzione. Solo in casi particolari il \textit{Responsabile} può delegare ad un verificatore l'approvazione di un documento come descritto nella sezione ---5.1.6 \\
Ha inoltre l'incarico di gestire la creazione e l'assegnazione dei ticket delle macro-fasi e di assegnare ad un membro del gruppo il ruolo di responsabile di quest’ultima.
Redige il \textit{Piano di Progetto} e collabora alla stesura del \textit{Piano di Qualifica}, in particolare nella sezione relativa alla pianificazione.

\subsection{Amministratore}
L’\textit{Amministratore} è responsabile del controllo, dell'efficienza e dell'operatività dell'ambiente di lavoro. Le mansioni di primaria importanza che gli competono sono:
\begin{itemize}

\item Ricerca di strumenti che possano automatizzare qualsiasi compito che possa essere tolto all'umano;
\item Risoluzione dei problemi legati alle difficoltà di gestione e controllo dei processi e delle risorse. La risoluzione di tali problemi richiede l'adozione di strumenti adatti;
\item Controllo delle versioni e delle configurazioni del prodotto;
\item Gestione dell'archiviazione e del versionamento della documentazione di progetto;
\item Fornire procedure e strumenti per il monitoraggio e la segnalazione per il controllo qualità.
Redige le \textit{Norme di Progetto}, dove spiega e norma l'utilizzo degli strumenti, redige la sezione del \textit{Piano di Qualifica} dove vengono descritti strumenti e metodi di verifica;
\end{itemize}

\subsection{Analista}
L’\textit{Analista} è responsabile delle attività di analisi. Le responsabilità di spicco per tale ruolo sono:
\begin{itemize}
\item Produrre una specifica di progetto comprensibile, sia per il Proponente, sia per il
Committente che per il \textit{Progettista}, e motivata in ogni suo punto;
\item Comprendere appieno la natura e la complessità del problema.
\end{itemize}
Redige lo \textit{Studio di Fattibilità}, l’\textit{Analisi dei Requisiti} e parte del \textit{Piano di Qualifica}.
Partecipa alla redazione del \textit{Piano di Qualifica} in quanto conosce l’ambito del progetto ed ha chiari i livelli di qualità richiesta e le procedure da applicare per ottenerla.

\subsection{Progettista}
Il \textit{Progettista} è responsabile delle attività di progettazione. Le responsabilità di tale ruolo sono:
\begin{itemize}
\item Produrre una soluzione attuabile, comprensibile e motivata;
\item Effettuare scelte su aspetti progettuali che applichino al prodotto soluzioni note
ed ottimizzate;
\item Effettuare scelte su aspetti progettuali e tecnologici che rendano il prodotto facilmente manutenibile.
\end{itemize}
Redige la \textit{Specifica Tecnica}, la \textit{Definizione di Prodotto} e le sezioni inerenti le metriche di verifica della programmazione del \textit{Piano di Qualifica}.

\subsection{Verificatore}
Il \textit{Verificatore} è responsabile delle attività di verifica. Ha il compito di effettuare la verifica dei documenti utilizzando gli strumenti e i metodi proposti dal \textit{Piano di Qualifica}
e attenendosi a quanto descritto nelle \textit{Norme di Progetto}. Le responsabilità di tale ruolo sono:
\begin{itemize}

\item Assicurare che l'attuazione delle attività sia conforme alle norme stabilite;
\item Controllare la conformità di ogni stadio del ciclo di vita del prodotto.
\end{itemize}
Redige la sezione del \textit{Piano di Qualifica} che illustra l'esito e la completezza delle verifiche e delle prove effettuate.

\subsection{Programmatore}
Il \textit{Programmatore} è responsabile delle attività di codifica e delle componenti di ausilio necessarie per l'esecuzione delle prove di verifica e validazione. Le responsabilità di tale
ruolo sono:
\begin{itemize}
\item Implementare rigorosamente le soluzioni descritte dal \textit{Progettista}, da cui seguirà quindi la realizzazione del prodotto;
\item Scrivere codice documentato, versionato, manutenibile e che rispetti gli standard stabiliti per la scrittura del codice;
\item Implementare i test sul codice scritto, necessari per prove di verifica e validazione.
Redige il \textit{Manuale Utente} e produce una abbondante documentazione del codice.
\end{itemize}