\section{Procedure al supporto dei processi}{
Dopo aver descritto i ruoli di progetto e le relative funzioni, si procede ora ad elencare le procedure che essi devono seguire in modo rigoroso per convergere agli obiettivi posti nel \textit{Piano di Qualifica}.
	\subsection{Fattibilità}{
		A partire da informazioni preliminari sul capitolato, lo studio di fattibilità dovrà generare un rapporto che indichi la convenienza o meno del gruppo nello sviluppo del sistema. In particolare si dovrà considerare:
		\begin{enumerate}
			\item Sufficienza di risorse umane\ped{g};
			\item Rapporto tra i costi ed i benefici;
			\item Rischi individuati.
		\end{enumerate}
		Nello stimare i benefici dovrà essere data molta importanza alle competenze che i membri del gruppo acquisirebbero nello sviluppo del sistema.
	 }
	\subsection{Deduzione e analisi dei requisiti}{ 
	\phantomsection
		\subsubsection{Scoperta dei requisiti}{
			\textbf{Interviste}\\\\
			al fine di evitare interviste infruttuose verrà preparato un elenco di punti da sottoporre al proponente\ped{g} in modo da dare una direzione precisa all’intervista. Potrebbe essere utile discutere con il proponente\ped{g} dei casi d'uso\ped{g} analizzati internamente al gruppo durante la fase di analisi.
			Le richieste di interviste al proponente\ped{g} avverranno con le modalità descritte in ”comunicazioni esterne”. Durante ogni intervista dovrà essere scritta una minuta che sarà confermata dal proponente\ped{g}, eventualmente con le opportune modifiche. La minuta sarà confermata al termine dell’incontro. Quando non fosse un problema per il proponente\ped{g} l’audio dell’intervista dovrà essere registrato per favorire la futura fase di analisi.\\\\
			\textbf{Riunioni interne e casi d'uso}\\\\
			Individualmente e durante le riunioni interne gli analisti dovranno analizzare le informazioni raccolte dalle interviste con il proponente\ped{g} per individuare problemi e fonti da cui attingere i requisiti\ped{g}.\\
			L’individuazione dei requisiti\ped{g} funzionali sarà guidata dai casi d’uso. I casi d’uso potranno avere rappresentazione a diagrammi ma ogni caso d’uso dovrà avere anche la rappresentazione testuale. In particolare nella rappresentazione testuale si definirà:
			\begin{enumerate}
				\item Identificativo;
				\item Attore primario;
				\item Precondizioni;
				\item Postcondizioni;
				\item Scenario principale;
				\item estensioni\ped{g}.
			\end{enumerate}
			Per la sintassi si rimanda a ”Dall’idea al codice\ped{g} con UML2.0, Luciano Baresi, Luigi Lavazza, Massimiliano Pianciamore”.
			}
			\subsubsection{Classificazione e priorità}{
				I requisiti\ped{g} dovranno essere classificati in:
				\begin{enumerate}
					\item Requisiti\ped{g} di processo\ped{g};
					\item Requisiti\ped{g} di prodotto.
				\end{enumerate}
				I requisiti\ped{g} di prodotto saranno classificati in base a:
				\begin{enumerate}
					\item Tipologia;
					\item Importanza;
					\item Provenienza.
				\end{enumerate}
				Dove i gradi di importanza saranno:
				\begin{itemize}
						\item \{\textbf{Obbligatorio}\}: requisito\ped{g} obbligatorio;
						\item \{\textbf{Desiderabile}\}: requisito\ped{g} desiderabile;
						\item \{\textbf{Opzionale}\}: requisito\ped{g} opzionale.
				\end{itemize}
				La provenienza può essere:
				\begin{itemize}
					\item \{\textbf{Capitolato}\}: da capitolato;
					\item \{\textbf{Interni}\}: da analisi interna;
					\item \{\textbf{Proponente}\}: da incontro con proponente\ped{g}.
				\end{itemize}
				Mentre le tipologie saranno:
				\begin{itemize}
					\item \{\textbf{RF}\}: requisito\ped{g} funzionale;
					\item \{\textbf{RQ}\}: requisito\ped{g} di qualità;
					\item \{\textbf{RV}\}: requisito\ped{g} di vincolo.
				\end{itemize}
			}
			\subsubsection{Specifica}{
				Nella specifica dei requisiti\ped{g} dovrà essere considerato come riferimento lo standard IEEE 830-1998. In particolare saranno da perseguire le seguenti caratteristiche dei requisiti\ped{g}:
				\begin{enumerate}
					\item Non ambigui;
					\item Corretti;
					\item Completi;
					\item Verificabili;
					\item Consistenti;
					\item Modificabili;
					\item Tracciabili;
					\item Ordinati per rilevanza.
				\end{enumerate}
				I requisiti\ped{g} dovranno essere specificati in un documento ”Analisi dei requisiti” secondo la struttura definita nello standard IEEE 830-1998. La specifica dei requisiti\ped{g} dovrà essere documentata in forma tabellare per evitare ambiguità. Per ogni requisito\ped{g} dovranno essere definiti un codice\ped{g}, una descrizione, un riferimento alla fonte e un riferimento alla verifica. Al fine di rendere meno ambigui i requisiti\ped{g} sara redatto un ”Glossario” contenente la definizione di tutti i termini non ovvi usati in fase di analisi.
			}
			\subsubsection{Verifica dei requisiti}{
				Per ogni requisito\ped{g} di processo\ped{g} specificato dovrà essere presente in ”Piano di qualifica” un riferimento alle sezioni di ”Norme di progetto” in cui viene assicurato il soddisfacimento del requisito\ped{g}. Per ogni requisito\ped{g} di prodotto specificato dovrà essere descritto brevemente il metodo che verrà usato per verificarne il soddisfacimento.\\Per favorire la tracciabilità tra requisiti\ped{g} e metodi di verifica dovrà essere presente in ”Piano di qualifica” una tabella in cui si definiscono: codice\ped{g} di requisito\ped{g}, codice\ped{g} di verifica e modalità di verifica. Se il requisito\ped{g} è di processo\ped{g}, la modalità di verifica conterrà i riferimenti alle sezioni corrispondenti in ”Norme di progetto”.
			}
		}
		\subsection{Validazione\ped{g} dei requisiti}{
			\subsubsection{Interna}{
				Saranno verificate la correttezza e la completezza dei requisiti\ped{g} rispetto ai bisogni. Ciò verrà fatto tramite tracciamento tra specifica dei requisiti\ped{g} e bisogni individuati.\\Saranno verificate la correttezza e la completezza dei metodi di verifica dei requisiti\ped{g}
				rispetto ai requisiti\ped{g}. Ciò verrà fatto tramite tracciamento tra specifica dei requisiti\ped{g} e metodi di verifica.
			}
				\subsubsection{Esterna}{
					Terminata la validazione\ped{g} interna verranno presentati al proponente\ped{g} i documenti ”Analisi dei requisiti” e ”Piano di qualifica”, se accettati costituiranno una baseline per la fase successiva del progetto\ped{g} altrimenti verranno gestite le richieste di modifica secondo i metodi descritti in ”Gestione dei cambiamenti”.
				}
		}
		\subsection{Gestione delle modifiche ai requisiti}{
			A tutte le proposte di modifica dei requisiti\ped{g} dovrà essere applicata la seguente procedura:
			\begin{enumerate}
				\item Deduzione, analisi e specifica dei cambiamenti;
				\item Stima dei costi del cambiamento considerando quante modifiche dovranno essere fatte ai requisiti\ped{g} e al progetto\ped{g} del sistema;
				\item Decisione ed eventuale implementazione del cambiamento nei requisiti\ped{g} e nel progetto\ped{g} di sistema.
			\end{enumerate}
			Per gestire i cambiamenti e per facilitare il tracciamento dei requisiti\ped{g} verrà usato un software\ped{g} appositamente creato dal gruppo. L’amministratore avrà il compito di gestire il server\ped{g} e amministrare i diritti di accesso degli utenti alle funzionalità fornite. In particolare gli analisti dovranno usare i modelli definiti all’inizio della fase di analisi. Per evitare problemi dovuti a modifiche concorrenti alla base dati l’amministratore dovrà garantire che ad ogni istante solo un analista possa modificare un certo sotto albero della foresta dei requisiti\ped{g} e dei test.
			}