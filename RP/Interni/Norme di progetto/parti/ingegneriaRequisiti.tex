\section{Procedure al supporto dei processi}{
Dopo aver descritto i ruoli di progetto e le relative funzioni, si procede ora ad elencare le procedure che essi devono seguire in modo rigoroso per convergere agli obiettivi posti nel \textit{Piano di Qualifica}.

	\subsection{Analisi dei Requisiti}{ 
	   \subsubsection{Fattibilità}{
	    		A partire da informazioni preliminari sul capitolato, lo studio di fattibilità dovrà generare un rapporto che indichi la convenienza o meno del gruppo nello sviluppo del sistema. In particolare si dovrà considerare:
	    		\begin{enumerate}
	    			\item Sufficienza di risorse umane\ped{g};
	    			\item Rapporto tra i costi ed i benefici;
	    			\item Rischi individuati.
	    		\end{enumerate}
	    		Nello stimare i benefici dovrà essere data molta importanza alle competenze che i membri del gruppo acquisirebbero nello sviluppo del sistema.
	    	 }
		\subsubsection{Scoperta dei requisiti}{
			\subsubsection{Interviste}
			al fine udi evitare interviste infruttuose verrà preparato un elenco di punti da sottoporre al proponente\ped{g} in modo da dare una direzione precisa all’intervista. Potrebbe essere utile discutere con il proponente\ped{g} dei casi d'uso\ped{g} analizzati internamente al gruppo durante la fase di analisi.
			Le richieste di interviste al proponente\ped{g} avverranno con le modalità descritte in ”comunicazioni esterne”. Durante ogni intervista dovrà essere scritta una minuta che sarà confermata dal proponente\ped{g}, eventualmente con le opportune modifiche. La minuta sarà confermata al termine dell’incontro. Quando non fosse un problema per il proponente\ped{g} l’audio dell’intervista dovrà essere registrato per favorire la futura fase di analisi.
			\subsubsection{Riunioni interne e casi d'uso}
			Individualente e durante le riunioni interne gli analisti dovranno analizzare le informazioni raccolte dalle interviste con il proponente\ped{g} per individuare problemi e fonti da cui attingere i requisiti\ped{g}.\\
			L’individuazione dei requisiti\ped{g} funzionali sarà guidata dai casi d’uso. I casi d’uso potranno avere rappresentazione a diagrammi ma ogni caso d’uso dovrà avere anche la rappresentazione testuale. In particolare nella rappresentazione testuale si definirà:
			\begin{enumerate}
				\item Identificativo;
				\item Attore primario;
				\item Precondizioni;
				\item Postcondizioni;
				\item Scenario principale;
				\item estensioni\ped{g}.
			\end{enumerate}
			Per la sintassi si rimanda a ”Dall’idea al codice\ped{g} con UML2.0, Luciano Baresi, Luigi Lavazza, Massimiliano Pianciamore”.
			}
			\subsubsection{Classificazione e priorità}{
				I requisiti\ped{g} dovranno essere classificati in:
				\begin{enumerate}
					\item Requisiti\ped{g} di processo\ped{g};
					\item Requisiti\ped{g} di prodotto.
				\end{enumerate}
				I requisiti\ped{g} di prodotto saranno classificati in base a:
				\begin{enumerate}
					\item Tipologia;
					\item Importanza;
					\item Provenienza.
				\end{enumerate}
				Dove i gradi di importanza saranno:
				\begin{itemize}
						\item \{\textbf{Obbligatorio}\}: requisito\ped{g} obbligatorio;
						\item \{\textbf{Desiderabile}\}: requisito\ped{g} desiderabile;
						\item \{\textbf{Opzionale}\}: requisito\ped{g} opzionale.
				\end{itemize}
				La provenienza può essere:
				\begin{itemize}
					\item \{\textbf{Capitolato}\}: da capitolato;
					\item \{\textbf{Interni}\}: da analisi interna;
					\item \{\textbf{Proponente}\}: da incontro con proponente\ped{g}.
				\end{itemize}
				Mentre le tipologie saranno:
				\begin{itemize}
					\item \{\textbf{RF}\}: requisito\ped{g} funzionale;
					\item \{\textbf{RQ}\}: requisito\ped{g} di qualità;
					\item \{\textbf{RV}\}: requisito\ped{g} di vincolo.
				\end{itemize}
			}
			\subsubsection{Specifica}{
				Nella specifica dei requisiti\ped{g} dovrà essere considerato come riferimento lo standard IEEE 830-1998. In particolare saranno da perseguire le seguenti caratteristiche dei requisiti\ped{g}:
				\begin{enumerate}
					\item Non ambigui;
					\item Corretti;
					\item Completi;
					\item Verificabili;
					\item Consistenti;
					\item Modificabili;
					\item Tracciabili;
					\item Ordinati per rilevanza.
				\end{enumerate}
				I requisiti\ped{g} dovranno essere specificati in un documento ”Analisi dei requisiti” secondo la struttura definita nello standard IEEE 830-1998. La specifica dei requisiti\ped{g} dovrà essere documentata in forma tabellare per evitare ambiguità. Per ogni requisito\ped{g} dovranno essere definiti un codice\ped{g}, una descrizione, un riferimento alla fonte e un riferimento alla verifica. Al fine di rendere meno ambigui i requisiti\ped{g} sara redatto un ”Glossario” contenente la definizione di tutti i termini non ovvi usati in fase di analisi.
			}
			\subsubsection{Verifica dei requisiti}{
				Per ogni requisito\ped{g} di processo\ped{g} specificato dovrà essere presente in ”Piano di qualifica” un riferimento alle sezioni di ”Norme di progetto” in cui viene assicurato il soddisfacimento del requisito\ped{g}. Per ogni requisito\ped{g} di prodotto specificato dovrà essere descritto brevemente il metodo che verrà usato per verificarne il soddisfacimento.\\Per favorire la tracciabilità tra requisiti\ped{g} e metodi di verifica dovrà essere presente in ”Piano di qualifica” una tabella in cui si definiscono: codice\ped{g} di requisito\ped{g}, codice\ped{g} di verifica e modalità di verifica. Se il requisito\ped{g} è di processo\ped{g}, la modalità di verifica conterrà i riferimenti alle sezioni corrispondenti in ”Norme di progetto”.
			}
		}
		\subsection{Validazione\ped{g} dei requisiti}{
			\subsubsection{Interna}{
				Saranno verificate la correttezza e la completezza dei requisiti\ped{g} rispetto ai bisogni. Ciò verrà fatto tramite tracciamento tra specifica dei requisiti\ped{g} e bisogni individuati.\\Saranno verificate la correttezza e la completezza dei metodi di verifica dei requisiti\ped{g}
				rispetto ai requisiti\ped{g}. Ciò verrà fatto tramite tracciamento tra specifica dei requisiti\ped{g} e metodi di verifica.
			}
				\subsubsection{Esterna}{
					Terminata la validazione\ped{g} interna verranno presentati al proponente\ped{g} i documenti ”Analisi dei requisiti” e ”Piano di qualifica”, se accettati costituiranno una baseline per la fase successiva del progetto\ped{g} altrimenti verranno gestite le richieste di modifica secondo i metodi descritti in ”Gestione dei cambiamenti”.
				}
		}
		\subsection{Gestione delle modifiche ai requisiti}{
			A tutte le proposte di modifica dei requisiti\ped{g} dovrà essere applicata la seguente procedura:
			\begin{enumerate}
				\item Deduzione, analisi e specifica dei cambiamenti;
				\item Stima dei costi del cambiamento considerando quante modifiche dovranno essere fatte ai requisiti\ped{g} e al progetto\ped{g} del sistema;
				\item Decisione ed eventuale implementazione del cambiamento nei requisiti\ped{g} e nel progetto\ped{g} di sistema.
			\end{enumerate}
			Per gestire i cambiamenti e per facilitare il tracciamento dei requisiti\ped{g} verrà usato un software\ped{g} appositamente creato dal gruppo. L’amministratore avrà il compito di gestire il server\ped{g} e amministrare i diritti di accesso degli utenti alle funzionalità fornite. In particolare gli analisti dovranno usare i modelli definiti all’inizio della fase di analisi. Per evitare problemi dovuti a modifiche concorrenti alla base dati l’amministratore dovrà garantire che ad ogni istante solo un analista possa modificare un certo sotto albero della foresta dei requisiti\ped{g} e dei test.
			}
			
   \subsection{Progettazione}
	\subsubsection{Specifica Tecnica}
	I Progettisti devono descrivere la progettazione ad alto livello dell'architettura dell'applicazione e dei singoli componenti nella \emph{Specifica Tecnica} e provvedere alla progettazione	di opportuni test di integrazione.
	\subsubsubsection{Diagrammi UML}
	Devono essere realizzati i seguenti diagrammi:
	\begin{itemize}
	\item Diagrammi dei package;
	\item Diagrammi delle classi;
	\item Diagrammi di sequenza;
	\item Diagrammi di attività.
	\end{itemize}
	\subsubsubsection{Design pattern}
	I \textit{Progettisti} devono descrivere i design pattern utilizzati per realizzare l'architettura:
	di essi si deve includere una breve descrizione e un diagramma che ne esemplifichi il funzionamento e la struttura.
	\subsubsubsection{Tracciamento componenti}
	Ogni requisito deve essere tracciato al componente che lo soddisfa. Il software LateTrack genera automaticamente le tabelle di tracciamento come descritto nella sezione xxx. In questo modo sarà possibile misurare il progresso nell'attività di progettazione e garantire che ogni requisito venga soddisfatto.
	\subsubsubsection{Test di integrazione}
	I Progettisti devono definire delle classi di verifica necessarie per verificare che i componenti del sistema funzionino nella maniera prevista.
	\subsubsubsection{Definizione di Prodotto}
	I Progettisti devono produrre la Definizione di Prodotto dove viene descritta la progettazione di dettaglio del sistema ampliando quanto scritto nella Specifica Tecnica.
	
	\subsubsubsection{Diagrammi UML}
	Devono essere aggiornati i seguenti diagrammi:
	\begin{itemize}
    \item Diagrammi delle classi;
	\item Diagrammi di sequenza;
	\item Diagrammi di attività.
	\end{itemize}

	\subsubsubsection{Test di unità}
	I Progettisti dovranno definire i test d'unità necessari per verificare che i componenti
	del sistema funzionino nel modo previsto.
	\subsection{Verifica}
	L'obiettivo delle attività di verifica è quello di trovare e rimuovere i problemi presenti. Un problema può verificarsi a vari livelli, e per ogni livello assume un nome diverso:
		\begin{itemize}
			\item Fault (difetto): è l'origine del problema, ciò che fa scaturire il malfunzionamento;
			\item Error (errore): è lo stato per cui il software\ped{g} si trova in un punto sbagliato del flusso di esecuzione o con valori sbagliati rispetto a quanto previsto dalla specifica;
			\item Failure (fallimento, guasto): è un comportamento difforme dalla specifica, cioè la manifestazione dell'errore all'utente del software\ped{g}.
		\end{itemize}
		Esiste una relazione di causa-effetto fra questi tre termini:\\
		\[DIFETTO\longrightarrow ERRORE\longrightarrow FALLIMENTO\]\\
		Non sempre un errore dà origine ad un fallimento: ad esempio potrebbero esserci alcune variabili che si trovano in stato erroneo ma non vengono lette, o non viene percorso\ped{g} il ramo di codice\ped{g} che le contiene.\\
		E' necessario prestare particolare attenzione a questo tipo di errori (detti anche quiescenti), avvalendosi anche di strumenti per il rilevamento dei bug.
	}
	
	\subsubsection{Verifica dei documenti}
	\label{sec:VerificaDocumenti}
	Ogni qualvolta avvenga un cambiamento sostanziale nello sviluppo del prodotto, si istanzierà il processo\ped{g} di verifica. \\
		Nello specifico durante ogni fase (Analisi, Progettazione, Realizzazione e Validazione\ped{g}) saranno applicate le tecniche di verifica qui descritte nei seguenti casi:
		\begin{itemize}
			\item Conclusione della prima redazione di un documento;
			\item Conclusione della prima redazione di un file\ped{g} di codice\ped{g};
			\item Conclusione della modifica sostanziale di un documento: quando il versionamento passa da .x.y.z a .x.y+1.0 oppure a .x+1.0.0. Si veda per approfondimento il paragrafo relativo al versionamento nel documento \href{run:../../Esterni/\fNormeDiProgetto}{\fEscapeNormeDiProgetto};

		\end{itemize}
	Per eseguire un'accurata verifica dei documenti redatti è necessario seguire il seguente protocollo:
	\begin{enumerate}
	\item \textbf{Controllo sintattico e del periodo}: Utilizzando TeXstudio e GNU Aspell vengono evidenziati e corretti gli errori di grammatica più evidenti. Gli errori di sintassi, di sostituzione di lettere che provocano la creazione di parole grammaticalmente corrette ma sbagliate nel contesto ed i periodi di difficile comprensione necessitano dell’intervento di un verificatore umano. Per questa ragione ciascun documento dovrà essere sottoposto ad un walkthrough da parte dei verificatori per individuare tali errori;
	\item \textbf{Rispetto delle norme di progetto}: Sono state definite norme tipografiche di carattere generale. Impongono una struttura dei documenti che non può essere verificata in maniera automatica. La verifica delle norme
	per cui non è stato definito uno strumento automatico richiede che i Verificatori eseguano inspection sul rispetto di quelle norme in ciascun documento;
	\item \textbf{Lista di controllo}: Il \emph{Verificatore} dovrà utilizzare la lista di controllo per i
	documenti, descritta nell’appendice \ref{sec:ListaControllo}, e verificare che gli errori tipici non siano	presenti;
	\item \textbf{Verifica del glossario}: Il \emph{Verificatore} si occuperà del controllo dei termini inseriti nel glossario e della corretta pedicizzazione dei termini nei vari documenti,segnalando eventuali errori;
	\item \textbf{Calcolo dell’indice Gulpease}: Su ogni documento redatto il \emph{Verificatore} deve calcolare l’indice di leggibilità. Nel caso in cui l’indice risultasse troppo basso, sarà necessario eseguire un walkthrough del documento alla ricerca delle frasi troppo lunghe o complesse;
	\item \textbf{Miglioramento del processo di verifica}: Per avere un miglioramento del processo di verifica, quando i \emph{Verificatori} eseguono walkthrough di un documento, dovranno riportare gli errori più frequentemente trovati. Grazie a tale pratica sarà possibile eseguire inspection su tali errori nelle verifiche future;
	\item \textbf{Segnalazione degli errori riscontrati}: il \emph{Verificatore} deve generare ticket secondo quando descritto nella
	sezione \ref{sec:TicketVerifica}.
	\end{enumerate}
	\subsubsection{Verifica diagrammi UML}
	al \emph{Verificatore} è richiesto di fare un controllo manuale sui diagrammi UML generati per verificare che sia stato rispettato il formalismo UML 2.x
	\subsubsection{Verifica del codice}
	al \emph{Verificatore} è richiesto l’avvio dei test statici e dinamici e l’analisi dei risultati. Di seguito un elenco degli strumenti da usare per l’analisi.
	
	\subsubsubsection{Analisi Statica} 
	
	  \begin{itemize}
	  \item \textbf{Mocha}: framework per eseguire test sul codice node.js;
	  \item \textbf{Jasmine}: framework per eseguire test su codice javascript;
	  \item \textbf{jSHint}: tool che permette di rilevare potenziali errori nel codice\ped{g} javascript\ped{g};
	  \item \textbf{QUnit}: framework\ped{g} per i test d'unità del codice\ped{g} javascript\ped{g};
	  \item \textbf{jsmeter}: strumento per il calcolo di alcune metriche\ped{g} del codice\ped{g} javascript\ped{g}.
	  \end{itemize}
	
	\subsubsubsection{Analisi Dinamica}
	Verranno utilizzati strumenti e plugin interni al browser\ped{g} \emph{Chrome} quali \textbf{SpeedTracer} per verificare la velocità dell'applicazione web\ped{g};
	
	
	\subsubsubsection{Validazione}
	
	La validazione\ped{g} del codice\ped{g} HTML e CSS\ped{g} dell’applicazione da noi sviluppata verrà
	fatta tramite il servizio W3C\ped{g} Validator32 del W3C\ped{g}.
	
	\subsection{Codifica}
	Le convenzioni di codifica che tutti i membri del gruppo devono seguire sono quelle
	specificate alla seguente pagina :\\
	\begin{center} \url{http://www.w3schools.com/js/js_conventions.asp} \end{center}
	\subsubsection{Nomi}
	I nomi di variabili, classi, funzioni, metodi e commenti dovranno essere in camelCase. I nome di variabili, metodi e funzioni dovranno avere la prima lettera	minuscola.
	I nomi di variabili globali e costanti dovranno essere in UPPERCASE.
	\subsubsection{Documentazione}
	I file contenenti codice dovranno essere provvisti di un'intestazione contenente:
	\begin{lstlisting}
    /*!
  	* \file Nome del file
  	* \author Autore (indirizzo email dell'autore)
  	* \date Data di creazione
  	* \brief Breve descrizione del file
  	*
  	* Descrizione dettagliata del file
  	*/
  	
  	Prima di ogni metodo dovra essere presente un commento contenente:
  	/*!
  	* \brief Breve descrizione della funzione
  	* \param Nome del primo parametro
  	* \param Nome del secondo parametro
  	* \return Valore ritornato dalla funzione
  	*/
    \end{lstlisting}