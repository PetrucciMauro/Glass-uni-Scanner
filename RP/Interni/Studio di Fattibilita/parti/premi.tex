\section{C4 - Premi}{
 \subsection{Descrizione}
   Il capitolato scelto prevede la realizzazione di un Software\ped{g} di presentazione di "slide" non basato sul modello di PowerPoint, che funzioni\ped{g} sia su Desktop\ped{g} che su dispositivo mobile\ped{g}. Devono essere realizzati effetti grafici a supporto dello  “storytelling” che siano di livello comparabile con Prezi.
   Il Software\ped{g} dovrà coprire i due momenti fondamentali per questo tipo di attività:
   \begin{enumerate}
   		\item La creazione da parte dell' autore e la presentazione al pubblico, sia in presenza diretta che via WEB\ped{g};
   		\item Il Progetto\ped{g} vuole essere fortemente sperimentale, indagando su nuove possibilità  offerte dai sistemi moderni con tecnologie WEB\ped{g} sia nel campo degli effetti durante le presentazioni che sullo svolgimento non lineare delle stesse.
   \end{enumerate}
   Le presentazioni create con Prezi facilmente sconfinano nel terreno delle Infografiche\ped{g}.

\subsection{Dominio tecnologico}
Non ci sono vincoli per l'utilizzo di specifiche tecnologie ,tuttavia, dopo un attenta analisi si è potuto osservare come sia meglio creare una applicazione WEB\ped{g} basata su linguaggi e tecnologie moderne e dinamiche e quindi si andranno ad utilizzare le seguenti tecnologie:
\begin{itemize}
\item \textbf{Node.js}  per la realizzazione della componente Server\ped{g};
\item \textbf{Express}  per la realizzazione dell'infrastruttura della WEB\ped{g} application generata;
\item \textbf{MongoDB}  per il recupero dei dati;
\item Conoscenza di Framework\ped{g} per la componente \emph{front-end}, come Angular.js;

\end{itemize}
\subsection{Valutazione}
Vengono di seguito elencati gli aspetti positivi che hanno determinato la scelta del capitolato:

\begin{itemize}

\item Apprendimento di tecnologie innovative che portano un bagaglio di conoscenze ritenuto importante dato il grande uso di quest’ultime nella panoramica delle tecnologie presenti attualmente nel
mercato;
\item Interesse del gruppo nel vedere la propria applicazione dare vita alla fantasia degli utenti nella creazione di presentazioni sempre più coinvolgenti;
\item Libertà nella definizione dei Requisiti\ped{g}.
\end{itemize}
Similmente, il gruppo ha trovato aspetti negativi:
\begin{itemize}
\item Le tecnologie utilizzate nello sviluppo del Progetto\ped{g} non sono conosciute da nessun membro del gruppo \gruppo\ e quindi richiederanno un tempo di formazione per il loro apprendimento
considerevole;
\item La mole di lavoro per lo sviluppo del Progetto\ped{g} al gruppo sembra notevole.
\end{itemize}