
\section{Ambiente di lavoro} 

\subsection{Sistemi Operativi}

L’intero sviluppo del progetto viene svolto in ambienti Unix-Like e Windows, nello specifico, Ubuntu , Mac , Windows . Tale scelta è maturata dopo aver appurato che le tecnologie utilizzate per lo sviluppo del progetto sono indipendenti dall’ambiente di sviluppo e di impiego.

\subsection{Coordinamento}

È stato predisposto un server dedicato sul quale sono installate alcune applicazioni web
che facilitano la gestione del progetto. Per connettersi al server, l'indirizzo è il seguente:\\
\begin{center}
\url{http://gioberry.no-ip.org/}
\end{center}
\subsubsection{Software di gestione del progetto} 
\label{subsec:Software di gestione del prodotto}
Come piattaforma di gestione del progetto è stato scelto \textbf{Redmine}. Redmine fornisce:
\begin{itemize}
\item Un sistema flessibile di gestione dei ticket;
\item Il grafico Gantt delle attività;
\item Un calendario per organizzare i compiti;
\item La visualizzazione del repository associato al progetto;
\item Un sistema di rendicontazione del tempo.
\end{itemize}


\subsubsection{Versionamento}


Come strumento di versionamento si è deciso di utilizzare \textbf{Git}.
Git è uno strumento di versionamento veloce e di facile apprendimento che
rappresenta uno dei migliori strumenti attualmente esistenti.\\ Per lo sviluppo collaborativo abbiamo deciso di appoggiarci al servizio \textbf{Github} che fornisce non solo un repository Git, ma anche strumenti utili alla collaborazione fra più persone, come il servizio di \textbf{Ticket}, \textbf{Wiki} e \textbf{Milestone}.\\
Per quanto riguarda l’uso di Git sui computer di sviluppo, si è deciso l’uso
della versione ufficiale rilasciata dal team di sviluppo di Git(2.3.3).\\
Per interfacciarsi con il repository viene utlizzato \textbf{SmartGit} , un client multipiattaforma che permette di utilizzare Git in maniera rapida.\\
Si descrive ora la procedura di corretto utilizzo del programma smartgit .
\begin{itemize}

\item 	\textbf{Clonare il repository}: è possibile clonare il repository remoto in locale attraverso la seguente procedura:

\begin{itemize}
\item Premere nel menu in alto il pulsante Repository e successivamente Clone;
\item Nella riquadro comparso, inserire il link del repository\\ \url{https://github.com/PetrucciMauro/Premi.git}\\successivamente premere il pulsante  next;
\item Tenere la schermata successiva con entrambi i box spuntati e premere next;
\item Selezionare dove posizionare la cartella dove verrà salvato in locale il repository.
\end{itemize}
\item \textbf{Sincronizzare il repository} : Dalla schermata principale premere il pulsante pull; 
\item \textbf{Salvare una modifica in locale}: Dalla schermata principale premendo il pulsante commit e inserendo nell'apposita textbox un "Messaggio di commit",si salvano le modifiche effettuate ai file;
\item \textbf{Inviare le modifiche al repository remoto}: Dalla schermata principale premere il pulsante Push e successivamente ,alla comparta del nuovo riquadro , ancora push, ciò comporterà l'invio delle modifiche ai file al repository remoto;

\end{itemize}

\subsubsection{Software di Integrazione Continua}

Si è scelto di adottare \textbf{Jenkins} per applicare l’integrazione continua allo sviluppo del progetto.\\ 
Tale software permette di pianificare ed impostare la compilazione del codice.\\
Mette inoltre a disposizione un cruscotto su cui è possibile visualizzare lo stato del codice prodotto. Tale software è infatti in grado di interagire con il software di versionamento,e se disponibile con software per l’esecuzione di test sul codice prodotto.\\ 
Attualmente Jenkins viene utilizzato per la compilazione dei documenti \LaTeX. 


\subsubsection{Condivisione dei file}

  Si è inoltre scelto di utilizzare degli strumenti online che permettono di condividere file
  in modo semplice e veloce e che consentono di organizzare gli appuntamenti personali
  dei singoli componenti del gruppo.
\subsubsubsection{Google Drive}
  In questa piattaforma di condivisione file verranno salvati i documenti che:
  \begin{itemize}
  
  
  \item Non necessitano di controllo di versione ;
  \item Hanno bisogno di grande interattività tra i componenti del gruppo;
  \item Possono essere acceduti tramite l’uso di un semplice browser.
   \end{itemize}
  Questo strumento dovrebbe permettere a 2 o più componenti del gruppo di interagire
  lavorando sugli stessi documenti contemporaneamente. Google Drive viene utilizzato
  come strumento di supporto allo sviluppo della documentazione e del software presente
  su Git .
  


\subsubsection{Google Calendar}
 
Google Calendar viene utilizzato all’interno del gruppo per gestire le risorse umane. In
particolare tale strumento viene utilizzato per notificare in quali giorni un determinato
membro non può essere disponibile e per segnalare date rilevanti per il gruppo, come
ad esempio le date delle riunioni.



\subsection{Strumenti per i documenti}
\subsubsection{LATEX} 
 
Per la stesura dei documenti è stato scelto di utilizzare il sistema \LaTeX.\\
Il motivo 
principale dietro a questa scelta è la facilità di separazione tra contenuto e formattazione: 
con \LaTeX\ è possibile definire l’aspetto delle pagine in un file template condiviso da tutti i documenti. Altre soluzioni come Microsoft Office, LibreOffice o Google Docs non 
avrebbero consentito questa separazione, duplicando il lavoro di formattazione del testo 
e non garantendo un risultato uniforme.\\
Il grande numero di pacchetti esistenti consente di implementare funzionalità comuni 
in maniera semplice. L’estensibilità di \LaTeX\ può essere sfruttata per creare funzioni e 
variabili globali che rendono la scrittura del contenuto più corretta sotto un punto di 
vista semantico. Un esempio è dato dal comando /role{ruolo} che identifica ogni ruolo 
all’interno del progetto.\\
Per la scrittura di documenti \LaTeX\  l’editor consigliato è \textbf{TeXstudio}. 

\subsubsection{Controllo ortografico}

Il software per il controllo ortografico è \textbf{Aspell}\ .\ Il programma è accessibile tramite il \emph{Makefile} oppure tramite interfaccia grafica nell’editor consigliato TeXstudio.


\subsubsection{Grafici UML} 

Per la stesura dei grafici UML viene utilizzato il programma \textbf{Visual Paradigm}. Il programma viene utilizzato in licenza Community Edition la quale ne permette l’uso per fini non commerciali.



\subsubsection{Ambiente di verifica}

Vengono qui elencati e sommariamente descritti gli strumenti automatizzati per effettuare la verifica dei documenti redatti e del codice prodotto.\\ 
Le metriche ed i metodi per effettuare verifica sono ampiamente e dettagliatamente descritti nel \emph{Piano di Qualifica} . A tale documento si fa inoltre riferimento per le 
caratteristiche di fondamentale importanza per la verifica degli strumenti qui riportati. 

\begin{itemize}
\item \textbf{TeXstudio}: Per la scrittura di documenti è consigliato utilizzare l’ambiente grafico TeXstudio. Tale strumento integra i dizionari di OpenOffice.org e segnala i potenziali errori ortografici presenti nel 
testo durante la stesura del testo stesso; 
\item \textbf{Aspell}: strumento per la correzione tipografica dei documenti redatti in \LaTeX. 
\item \textbf{Axiom}: permette di automatizzare il tracciamento dei requisiti; 
\end{itemize} 


\subsubsection{Fogli di calcolo}
Per l’elaborazione dei dati si utilizza il software Calc del pacchetto LibreOffice in quanto
tale prodotto è open source.


\subsubsection{Codice} 

La verifica del codice è suddivisa in statica e dinamica.\\ 
Per entrambe vengono riportati gli strumenti da utilizzare. 

\subsubsubsection{Analisi Statica} 

  \begin{itemize}
  \item \textbf{jSHint}: tool che permette di rilevare potenziali errori nel codice javascript;
  \item \textbf{QUnit}: Framework per i test d'unità del codice javascript.
  \end{itemize}

\subsubsubsection{Analisi Dinamica}



\subsubsubsection{Validazione}

La validazione del codice HTML e CSS dell’applicazione da noi sviluppata verrà
fatta tramite il servizio W3C Validator32 del W3C.


\section{Protocollo per lo sviluppo dell'applicazione} 
Per procedere con uno sviluppo controllato dei documenti e del codice si è scelto di adottare il sistema di ticketing \textbf{Redmine}.\\ 
La scelta di tale software è descritta nella sezione \ref{subsec:Software di gestione del prodotto}.

\subsection{Creare un nuovo progetto} 

La creazione di un progetto è compito del \emph{Responsabile di Progetto}.\\ 
Un nuovo progetto rappresenta una macro-attività caratterizzata da molte sotto-attività 
supervisionate da un responsabile.\\\\ 
Per creare un nuovo progetto:
\begin{itemize}
\item Aprire \textbf{Progetti}; 
\item Selezionare \textbf{Nuovo progetto}; 
\item Assegnare un \textbf{Nome} breve ma significativo; 
\item Nel caso in cui si voglia creare un sotto-progetto indicare il nome del progetto padre dall’omonimo campo; 
\item \textbf{Identificativo}: scrivere in minuscolo ed indicare codice della fase a cui si riferisce ;
\item Lasciare inalterati gli altri campi. 
\end{itemize}
 
\subsection{Creazione ticket}
 
  I ticket vengono creati da:
 \begin{itemize}
 

    \item \textbf{Responsabile di Progetto}: crea i ticket più importanti che rappresentano le macro fasi evidenziate dalla pianificazione; 
	\item \textbf{Responsabile di Sotto-progetto}: crea i ticket per i processi non pianificati inizialmente, che si evidenziano necessari per l’avanzamento del sotto-progetto assegnato; 
	\item \textbf{Verificatore}: crea i ticket per segnalare errori ed imprecisioni trovate durante il processo di verifica. 
 \end{itemize}


I ticket possono essere di tre tipologie:
\begin{itemize}


\item \textbf{Ticket di pianificazione}: rappresentano le macro-attività di maggiore importanza. Sono organizzate in una gerarchia con vari livelli di importanza.
 Tali attività vengono create da: 
\begin{itemize}
\item \emph{Responsabile di Progetto} che durante la pianificazione identifica le attività più importati e generali; 
\item \emph{Responsabile di Sotto-progetto} che durante lo svolgimento delle attività può scomporre in sotto-problemi l’attività indicata dal Responsabile di Progetto. 
\end{itemize}


\item \textbf{Ticket di realizzazione e controllo}: tutti i documenti redatti, durante la stesura attraversano due stadi: 
\begin{itemize}
\item \textbf{Realizzazione}: un redattore del documento effettua una prima stesura; 
\item \textbf{Controllo}: un redattore, diverso da quello della precedente fase, esegue un primo controllo sui contenuti della parte scritta. 
\end{itemize}


\item \textbf{Ticket di verifica}: rappresentano gli errori identificati dai Verificatori durante 
il controllo che la realizzazione dell’attività sia conforme a quanto richiesto e che 
rispetti tutte le norme.
\end{itemize}
}

\phantomsection
\subsubsection{Ticket di pianificazione}

\begin{itemize}


\item Selezionare \textbf{Nuova segnalazione} da menù principale; 
\item \textbf{Tracker}: indicare la natura del ticket: 
	\begin{itemize}
	\item \textbf{Documento}: stesura di un documento. Il tipo di attività svolta dal redattore del documento viene definito durante la rendicontazione; 
	\item \textbf{Codifica}: stesura di codice; 
	\item \textbf{Verifica}: macro-attività di verifica sul prodotto dei sotto-processi. 


	\end{itemize}

\item \textbf{Oggetto}: descrizione breve e significativa; 
\item \textbf{Descrizione}: descrizione comprensibile e con riferimenti esterni mediante link se necessario; 
\item \textbf{Stato}: Plan; 
\item \textbf{Attività principale}: se si vuole creare una \textbf{sotto-attività} indicare l’id del ticket 
padre; 
\item \textbf{Categoria}: PDCA, solo se il ticket viene creato dal \emph{Responsabile di Progetto}; 
\item \textbf{Assegnato a}: inserire il nome del responsabile; 
\item \textbf{Osservatori}: aggiungere eventuali collaboratori.
\end{itemize}  


\subsubsection{Ticket di realizzazione e controllo} 

		\begin{itemize}
		
		\item Selezionare \textbf{Nuova segnalazione} da menù principale; 
		\item \textbf{Tracker}: indicare la natura del ticket: 
		\begin{itemize}
	\item \textbf{Documento}: stesura di un documento. Il tipo di attività svolta dal redattore del documento viene definito durante la rendicontazione; 
	\item \textbf{Codifica}: stesura di codice; 
	\item \textbf{Verifica}: attività di verifica sui prodotti dei processi. 

	\end{itemize}

\item \textbf{Oggetto}: descrizione breve e significativa secondo il principio: nome ticket padre attività da svolgere (realizzazione o controllo); 
\item \textbf{Descrizione}: descrizione comprensibile e con riferimenti esterni mediante link se 
necessario; 

\item \textbf{Stato}: New; 
\item \textbf{Attività principale}: se si vuole creare una \textbf{sotto-attività} indicare l’id del ticket 
padre; 
\item \textbf{Inizio}: dare una data di inizio presunta; 
\item \textbf{Scadenza}: dare una data di fine presunta; 
\item \textbf{Assegnato a}: inserire il nome del responsabile; 
\item \textbf{Osservatori}: aggiungere eventuali collaboratori. 
\end{itemize} 

\subsubsection{Ticket di verifica}


Un \emph{Verificatore} per creare un \emph{ticket di verifica} deve: 
\begin{enumerate}
\item assicurarsi che esista all’interno del progetto l’attività \emph{Verifica}.
Su tale attività vi devono essere due sotto-attività: “Verifica - realizzazione”, 
“Verifica - approvazione”. 
Tutti i ticket creati devono essere sotto-attività di: “Verifica - realizzazione”; 
\item Creare quindi il ticket secondo le seguenti direttive: 
		\begin{itemize}
		
		
		\item Selezionare \textbf{Nuova segnalazione} da menù principale; 
		\item \textbf{Tracker}: Bug; 
		\item \textbf{Oggetto}: descrizione breve e significativa dell’errore incontrato; 
		\item \textbf{Descrizione}: descrivere in modo dettagliato e chiaro: la natura e la posizione dell’errore; 
		\item \textbf{Stato}: New; 
		\item \textbf{Attività principale}: tutti i ticket devono essere figli del ticket “Verifica - realizzazione” del progetto su cui si sta eseguendo la verifica; 
		\item \textbf{Assegnato a}: inserire il nome del responsabile del progetto padre (es. 
		responsabile delle \emph{Norme di Progetto}). 
		\end{itemize}
Tutti i campi non segnalati sono da lasciare vuoti. 
Sarà poi compito del responsabile del progetto padre decidere a chi assegnare la correzione dell’errore. Nel caso in cui l’errore segnalato non sia considerato valido dal 
\emph{Responsabile del sotto-progetto} verrà confermato il rifiuto dal \emph{Responsabile di Progetto}. 

\end{enumerate}


\subsubsection{Dipendenze temporali}


Dopo la creazione del ticket, per aggiungere \textbf{dipendenze temporali} tra i ticket:
\begin{itemize}
\item Andare su \textbf{segnalazioni}; 
\item Aprire il link alla segnalazione a cui aggiungere la dipendenza; 
\item Nella sezione \textbf{segnalazioni correlate} premere \textbf{aggiungi}; 
\item Scegliere \textbf{segue} e indicare il numero della segnalazione che lo blocca ed eventuali giorni di slack. 

\end{itemize} 

Tutti i campi non segnalati sono da lasciare vuoti. 

\subsection{Aggiornamento ticket}

Esistendo due tipologie di ticket, viene qui definito la procedura per effettuare l’aggiornamento di entrambe.
\phantomsection 
\subsubsection{Ticket di pianificazione}

\begin{itemize}
\item Andare sul menù \textbf{Segnalazioni}; 
\item Selezionare il ticket di interesse; 
\item Cliccare il link \textbf{Aggiorna}; 
\item Commentare ciò che si è fatto sulla form \textbf{Note}; 
\item Cambiare lo stato del ticket secondo la seguente logica:
		\begin{itemize}
		\item \textbf{Do}: quando un ticket è in questo stato indica che una o più persone stanno 
		lavorando su tale attività; 
		\item \textbf{Check}: quando un ticket è in questo stato indica che una o più persone 
		stanno lavorando sulla verifica di tale attività; 
		\item \textbf{Act}: l’attività è stata conclusa e verificata, e ne sono state tratte le conclusioni adeguate. 
		
		\end{itemize} 

\item Se viene concluso, aggiornare lo stato del ticket di pianificazione padre. 

\end{itemize}


\subsubsection{Ticket di realizzazione e controllo}

\begin{itemize}
\item Andare sul menù \textbf{Segnalazioni}; 
\item Selezionare il ticket di interesse; 
\item Cliccare il link \textbf{Aggiorna}; 
\item Indicare il tempo impiegato in ore; 
\item Indicare il tipo di attività svolta; 
\item Commentare ciò che si è fatto sulla form \textbf{Note}; 
\item Cambiare lo stato del ticket secondo la seguente logica: 
		\begin{itemize}
		\item \textbf{In Progress}: quando un ticket è in questo stato indica che una o più persone 
		stanno lavorando su tale attività. La percentuale di completamento deve 
		essere impostata tra lo 0\% ed il 90\%; 
		\item \textbf{Closed}: l’attività è stata conclusa. La percentuale di completamento dell’attività è al 100\%. 
		 
		\end{itemize} 
\item Aggiornare lo stato del ticket di pianificazione padre secondo tali principi: 
		\begin{itemize}
		\item ticket figlio passa da New a In Progress: il ticket padre passa da Plan a Do, 
		o da Do a Check; 
		\item ticket figlio passa a Closed: il ticket padre deve essere in Do o Check; 
		\item tutti i ticket figli vengono chiusi: il ticket padre passa ad Act.
		\end{itemize}

\end{itemize}



\subsubsection{Ticket di verifica}

\begin{itemize}
\item Andare sul menù \textbf{Segnalazioni}; 
\item Selezionare il ticket di interesse; 
\item Cliccare il link \textbf{Aggiorna}; 
\item Indicare il tempo impiegato in ore; 
\item Indicare Verifica come tipo di attività svolta; 
\item Commentare le correzione nella form \textbf{Note}; 
\item Cambiare lo stato del ticket secondo la seguente logica:
		\begin{itemize}
		\item \textbf{In Progress}: quando un ticket è in questo stato indica che una o più persone 
		stanno lavorando su tale attività. La percentuale di completamento deve 
		essere impostata tra lo 0\% ed il 90\%; 
		\item \textbf{Closed}: l’attività è stata conclusa. La percentuale di completamento dell’attività è al 100\%; 
		\item \textbf{Rejected}: l’attività di verifica è stata rifiutata dal \emph{Responsabile del sottoprogetto} in accordo con il \emph{Responsabile di Progetto}. 
		
		\end{itemize}

\item Aggiornare lo stato del ticket di pianificazione padre secondo tali principi:
		\begin{itemize}
		\item ticket figlio passa da New a In Progress: il ticket padre passa da Plan a Do, 
		o da Do a Check; 
		\item ticket figlio passa a Closed: il ticket padre deve essere in Do o Check; 
		\item tutti i ticket figli vengono chiusi: il ticket padre passa ad Act. 
		
		\end{itemize} 

\end{itemize} 




\subsection{Consigli di utilizzo}
\phantomsection
\subsubsection{Pagina personale}
 
	Per avere una immediata visualizzazione dei ticket assegnati, è consigliato personalizzare 
	la pagina personale: 
	\begin{itemize}
		\item Andare alla \textbf{Pagina personale}; 
		\item Cliccare il link \textbf{Personalizza la pagina}; 
		\item Dal menù a tendina \textbf{La mia pagina di blocco}, selezionare \textbf{Le mie segnalazioni} 
		e premere il pulsante verde +; 
		\item Ripetere il punto precedente per aggiungere \textbf{Segnalazioni osservate}. 
	
	\end{itemize}

	
\subsubsection{Visualizzare segnalazioni}

	Per avere una visualizzazione più chiara delle segnalazioni si consiglia di ordinarle per 
	oggetto. Tale risultato può essere ottenuto premendo \textbf{Oggetto} dalla pagina \textbf{Segnalazioni}.


