\section{???}{
	\subsection{Gestione degli incarichi}{
		Per il servizio di gestione degli incarichi verrà usato lo strumento Redmine, seguendo una specifica normativa.\\Di seguito è elencata la prassi relativa all’assegnazione degli incarichi ai membri del team:
		\begin{enumerate}
			\item Il Responsabile crea una milestone e la renderà pubblica al gruppo. E’ suo incarico controllare lo stato di avanzamento della milestone e dei relativi ticket.
			\item I ticket saranno creati dal Responsabile; ognuno di essi rappresenterà un lavoro da svolgere. Ogni ticket verrà assegnato al componente del gruppo più idoneo a soddisfare il relativo compito. Ogni ticket verrà compilato secondo le seguenti direttive:
			\begin{itemize}
				\item \textbf{Title}: specifica, in maniera sintetica, l’oggetto del compito
				\item \textbf{Status}: può essere \emph{open, accepted o close} e rappresenta lo stato attuale del ticket
				\item \textbf{Owner}: rappresenta il membro del gruppo che dovrà adempiere all'incarico
				\item \textbf{Labels}: indica il campo del ticket, può essere \textbf{D} (per i documenti) oppure \textbf{C} (per il codice)
				\item \textbf{Private}: indica il compito come privato e la scelta è decisa dal Responsabile
				\item \textbf{Summary}: contiene la descrizione di ciò che il proprietario dell’incarico dovrà svolgere
			\end{itemize}
			\item Il Responsabile avrà  il compito di  creare i \textbf{ticket di verifica}: ad ogni ticket segnalato closed ne verrà associato uno nuovo con i seguenti campi:
			\begin{itemize}
				\item \textbf{Title}: è di tipo: \begin{center}
				VERIFICA \{\emph{Oggetto del compito da verificare}\}
				\end{center}
				\item \textbf{Status}: come in precedenza è da impostare come \emph{open}
				\item \textbf{Owner}: questa volta sarà il riferimento del Verificatore proprietario di questo ticket
				\item \textbf{Labels}: si aggiunge al label del ticket originale il suffisso \textbf{V}
				\item \textbf{Private}: indica il compito come privato e la scelta è lasciata al Responsabile 
				\item \textbf{Summary}: descrizione fatta dal Responsabile
			\end{itemize}
			\item Il Verificatore può rifiutare il ticket impostando il campo \textbf{Status} a \emph{won't fix} specificandone il motivo nel campo \textbf{Summary}. Nel caso accettasse il ticket imposterà lo \textbf{Status} a \emph{accepted}  e procederà con la fase di verifica.
			Se la verifica ha esito positivo lo stato passerà a \emph{closed} mentre nel caso opposto il Verificatore dovrà creare un nuovo ticket con i seguenti campi:
			
			\begin{itemize}
				\item \textbf{Title}: specifica, in maniera sintetica, l’oggetto del compito
				\item \textbf{Status}: impostato a \emph{open}
				\item \textbf{Owner}: in questo caso il riferimento sarà il Responsabile 
				\item \textbf{Labels}: label del ticket originale
				\item \textbf{Private}: indica il compito come privato e la scelta è decisa dal Verificatore
				\item \textbf{Summary}: descrive l’errore
			\end{itemize}
			Il Responsabile dovrà decidere se accettare o rifiutare il ticket di verifica. In caso positivo dovrà segnare nel campo \textbf{Owner} il riferimento all’addetto del soddisfacimento di quel compito. Una volta che quel ticket verrà segnalato come closed il Responsabile dovrà seguire una nuova procedura di verifica.
			\item Infine una volta che tutti i ticket appartenenti alla milestone saranno segnalati come \emph{closed} il Responsabile potrà rendere la milestone \emph{closed} ed aprirne una nuova ripartendo dal primo punto.
		\end{enumerate}

		
	 }
}