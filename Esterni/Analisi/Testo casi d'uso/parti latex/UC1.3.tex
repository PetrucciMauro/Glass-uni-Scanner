\subsection{UC 1.3 - Modifica presentazione}{
	\label{uc1.3}
	\begin{figure}[H]
		\centering
		\includegraphics[scale=0.75]{\imgs {UC1.3}.jpg} %inserire il diagramma UML
	\end{figure}
	\textbf{Attori}: utente desktop \\
	\textbf{Descrizione}: l'utente ha la possibilità di modificare una presentazione. \\
	\textbf{Precondizione}: è stata selezionata una presentazione in locale e aperta in modalità modifica.	\\
	\textbf{Postcondizione}: l’utente ha modificato la presentazione selezionata e salvato le modifiche in locale.	\\
	\textbf{Procedura principale}:
	\begin{enumerate}
		\item l’utente può inserire un nuovo frame vuoto (UC1.3.1);
		\item l’utente può spostare i frame nel piano della presentazione (UC1.3.2);
		\item l’utente può modificare i frame inseriti (UC1.3.3);
		\item l’utente può eliminare i frame inseriti (UC1.3.4);
		\item l’utente può impostare uno sfondo (UC1.3.5);
		\item l’utente può definire i percorsi della presentazione (UC1.3.6);
		\item l’utente può inserire e cancellare bookmark (UC1.3.7);
		\item l’utente può impostare le opzioni di esecuzione della presentazione (UC1.3.8);
		\item l’utente può salvare le modifiche in locale.
	\end{enumerate}
	}
\subsection{UC 1.3.1 - Inserimento nuovo frame}{
	\label{uc1.3.1}
	\begin{figure}[H]
		\centering
		\includegraphics[scale=0.75]{\imgs {UC1.3.1}.jpg} %inserire il diagramma UML
	\end{figure}
	\textbf{Attori}: utente desktop \\
	\textbf{Descrizione}: l'utente può inserire un nuovo frame nel piano della presentazione. \\
	\textbf{Precondizione}: è stata selezionata una presentazione in locale e aperta in modalità modifica.	\\
	\textbf{Postcondizione}: l’utente ha inserito nel piano di modifica un nuovo frame.	\\
	\textbf{Procedura principale}:
	\begin{enumerate}
		\item l’utente seleziona un frame nel menu a lato del piano della presentazione;
		\item l’utente sposta il frame all’interno del piano della presentazione;
		\item l’utente rilascia il frame.
	\end{enumerate}
}
\subsection{UC 1.3.1.1 - Selezione frame}{
	\label{uc1.3.1.1}
	\textbf{Attori}: utente desktop \\
	\textbf{Descrizione}: l’utente può selezionare uno dei frame definiti nel menu a lato del piano della presentazione. \\
	\textbf{Precondizione}: è stata selezionata una presentazione in locale e aperta in modalità modifica.	\\
	\textbf{Postcondizione}: l’utente ha selezionato dal menu il frame desiderato.	\\
	\textbf{Procedura principale}:
	\begin{enumerate}
		\item l’utente seleziona l’icona “nuovo frame” nel menu a sinistra del piano della presentazione;
		\item l’utente seleziona uno dei frame ora selezionabili nel menu a sinistra del piano della presentazione.
	\end{enumerate}
	}
\subsection{UC 1.3.1.2 - Spostamento frame nel piano della presentazione}{
	\label{uc1.3.1.2}
	\textbf{Attori}: utente desktop \\
	\textbf{Descrizione}: l'utente può spostare un nuovo frame nel piano della presentazione. \\
	\textbf{Precondizione}: l’utente ha aperto una presentazione in modalità modifica ed ha selezionato un frame da inserire nel menu a sinistra del piano della presentazione.	\\
	\textbf{Postcondizione}: l’utente ha inserito nel piano di modifica un nuovo frame.	\\
	\textbf{Procedura principale}:
	\begin{enumerate}
		\item l’utente sposta il frame nel piano della presentazione;
		\item l’utente rilascia la selezione sul frame.
	\end{enumerate}
	}
\subsection{UC 1.3.2 - Spostamento frame}{
	\label{uc1.3.2}
	\textbf{Attori}: utente desktop \\
	\textbf{Descrizione}: l'utente può spostare un frame nel piano della presentazione. \\
	\textbf{Precondizione}: l’utente ha aperto una presentazione in modalità modifica.	\\
	\textbf{Postcondizione}: l’utente ha spostato un frame nel piano della presentazione.	\\
	\textbf{Procedura principale}:
	\begin{enumerate}
		\item l’utente seleziona un frame nel piano della presentazione;
		\item l’utente sposta il frame nel piano della presentazione;
		\item l’utente rilascia il frame selezionato.
	\end{enumerate}
	}
\subsection{UC 1.3.3 - Modifica desktop di un frame}{
	\label{uc1.3.3}
	\begin{figure}[H]
		\centering
		\includegraphics[scale=0.75]{\imgs {UC1.3.3}.jpg} %inserire il diagramma UML
	\end{figure}
	\textbf{Attori}: utente desktop \\
	\textbf{Descrizione}: l’utente desktop ha scelto l’opzione di modifica di un frame. L’utente desktop può scegliere di inserire o modificare un elemento (che può essere del testo, un’immagine, o un video), di spostare un elemento, di eliminare un elemento, di inserire o eliminare una scelta. Inoltre può modificare la dimensione e la forma del frame, lo spessore e il colore del bordo, e modificare lo sfondo. \\
	\textbf{Precondizione}: l’utente desktop intende modificare un frame.	\\
	\textbf{Postcondizione}: nella presentazione è presente un frame modificato.	\\
	\textbf{Procedura principale}:
	\begin{enumerate}
		\item inserimento di un elemento testo (UC1.3.3.1);
		\item inserimento di un elemento immagine (UC1.3.3.2);
		\item inserimento di un elemento video (UC1.3.3.3);
		\item modifica di un elemento testo (UC1.3.3.4);
		\item modifica di un elemento immagine (UC1.3.3.5);
		\item modifica di un elemento video (UC1.3.3.6);
		\item spostamento di un elemento (UC1.3.3.7);
		\item eliminazione di un elemento (UC1.3.3.8);
		\item inserimento di un elemento scelta (UC1.3.3.9);
		\item modifica di un elemento scelta (UC1.3.3.10);
		\item modifica della dimensione del frame (UC1.3.3.11);
		\item modifica della forma del frame (UC1.3.3.12);
		\item modifica dello spessore del bordo del frame (UC1.3.3.13);
		\item modifica del colore del bordo del frame (UC1.3.3.14);
		\item modifica dello sfondo del frame (UC1.3.3.15);
	\end{enumerate}
	}
\subsection{UC 1.3.3.1 - Inserimento di un elemento testo}{
	\label{uc1.3.3.1}
	\textbf{Attori}: utente desktop \\
	\textbf{Descrizione}: l’utente desktop inserisce un elemento di tipo testo all’interno del frame. \\
	\textbf{Precondizione}: l’utente desktop desidera inserire un nuovo elemento testo.	\\
	\textbf{Postcondizione}: nel frame è presente un nuovo elemento testo..	\\
	\textbf{Procedura principale}:
	\begin{enumerate}
		\item l’utente desktop seleziona l’opzione di inserimento testo;
		\item l’utente desktop inserisce il testo desiderato.
	\end{enumerate}
	}
\subsection{UC 1.3.3.2 - Inserimento di un elemento immagine}{
	\label{uc1.3.3.2}
	\textbf{Attori}: utente desktop \\
	\textbf{Descrizione}: l’utente desktop inserisce un elemento di tipo immagine all’interno del frame. \\
	\textbf{Precondizione}: l’utente desktop desidera inserire un elemento immagine.	\\
	\textbf{Postcondizione}: nel frame è presente un nuovo elemento immagine.	\\
	\textbf{Procedura principale}:
	\begin{enumerate}
		\item l’utente desktop seleziona l’opzione di inserimento immagine;
		\item l’utente desktop inserisce l’immagine desiderata.
	\end{enumerate}
	}
\subsection{UC 1.3.3.3 - Inserimento di un elemento video}{
	\label{uc1.3.3.3}
	\textbf{Attori}: utente desktop \\
	\textbf{Descrizione}: l’utente desktop inserisce un elemento di tipo video all’interno del frame. \\
	\textbf{Precondizione}: l’utente desktop desidera inserire un elemento video.	\\
	\textbf{Postcondizione}: nel frame è presente un nuovo elemento video.	\\
	\textbf{Procedura principale}:
	\begin{enumerate}
		\item l’utente desktop seleziona l’opzione di inserimento video;
		\item l’utente desktop inserisce il video desiderato.
	\end{enumerate}
	}
\subsection{UC 1.3.3.4 - Modifica di un elemento testo}{
	\label{uc1.3.3.4}
	\textbf{Attori}: utente desktop \\
	\textbf{Descrizione}: l’utente desktop modifica un elemento testo presente all’interno del frame. \\
	\textbf{Precondizione}: l’utente desktop desidera modificare un elemento testo.	\\
	\textbf{Postcondizione}: nel frame è presente un elemento testo modificato.	\\
	\textbf{Procedura principale}:
	\begin{enumerate}
		\item l’utente desktop seleziona un elemento testo;
		\item l’utente desktop modifica il testo selezionato.
	\end{enumerate}
	}
\subsection{UC 1.3.3.5 - Modifica di un elemento immagine}{
	\label{uc1.3.3.5}
	\textbf{Attori}: utente desktop \\
	\textbf{Descrizione}: l’utente desktop modifica la dimensione di un elemento immagine presente all’interno del frame. \\
	\textbf{Precondizione}: l’utente desktop desidera modificare la dimensione di un elemento immagine.	\\
	\textbf{Postcondizione}: nel frame è presente un elemento immagine con una dimensione modificata.	\\
	\textbf{Procedura principale}:
	\begin{enumerate}
		\item l’utente desktop seleziona un elemento immagine;
		\item l’utente desktop modifica la dimensione dell’elemento immagine.
	\end{enumerate}
	}
\subsection{UC 1.3.3.6 - Modifica di un elemento video}{
	\label{uc1.3.3.6}
	\textbf{Attori}: utente desktop \\
	\textbf{Descrizione}: l’utente desktop modifica la dimensione di un elemento video presente all’interno del frame. \\
	\textbf{Precondizione}: l’utente desktop desidera modificare la dimensione di un elemento video.	\\
	\textbf{Postcondizione}: nel frame è presente un elemento video con una dimensione modificata.	\\
	\textbf{Procedura principale}:
	\begin{enumerate}
		\item l’utente desktop seleziona un elemento video;
		\item l’utente desktop modifica la dimensione dell’elemento video.
	\end{enumerate}
	}
\subsection{UC 1.3.3.7 - Spostamento di un elemento}{
	\label{uc1.3.3.7}
	\textbf{Attori}: utente desktop \\
	\textbf{Descrizione}: l’utente desktop sposta un elemento qualsiasi all’interno del frame. \\
	\textbf{Precondizione}: l’utente desktop desidera spostare un elemento.	\\
	\textbf{Postcondizione}: nel frame è presente un elemento spostato.	\\
	\textbf{Procedura principale}:
	\begin{enumerate}
		\item l’utente desktop seleziona un elemento e lo trascina all’interno del frame.
	\end{enumerate}
	}
\subsection{UC 1.3.3.8 - Eliminazione di un elemento}{
	\label{uc1.3.3.8}
	\textbf{Attori}: utente desktop \\
	\textbf{Descrizione}: l’utente desktop elimina un elemento qualsiasi all’interno del frame. \\
	\textbf{Precondizione}: l’utente desktop desidera eliminare un elemento.	\\
	\textbf{Postcondizione}: l’utente desktop ha eliminato un elemento.	\\
	\textbf{Procedura principale}:
	\begin{enumerate}
		\item l’utente desktop seleziona un elemento;
		\item l’utente desktop elimina l’elemento selezionato.
	\end{enumerate}
	}
\subsection{UC 1.3.3.9 - Inserimento di un elemento scelta}{
	\label{uc1.3.3.9}
	\textbf{Attori}: utente desktop \\
	\textbf{Descrizione}: l’utente desktop inserisce all’interno di un frame un elemento scelta verso un altro frame non appartenente allo stesso percorso del frame. \\
	\textbf{Precondizione}: l’utente desktop desidera inserire una scelta.	\\
	\textbf{Postcondizione}: l’utente desktop ha inserito una nuova scelta.	\\
	\textbf{Procedura principale}:
	\begin{enumerate}
		\item l’utente desktop seleziona l’opzione di inserimento scelta;
		\item l’utente desktop seleziona un altro frame;
		\item l’utente desktop assegna il nome all’elemento scelta.
	\end{enumerate}
	}
\subsection{UC 1.3.3.10 - Modifica di un elemento scelta}{
	\label{uc1.3.3.10}
	\textbf{Attori}: utente desktop \\
	\textbf{Descrizione}: l’utente desktop modifica il testo assegnato ad un elemento scelta presente all’interno del frame. \\
	\textbf{Precondizione}: l’utente desktop desidera modificare il nome assegnato ad una scelta all’interno del frame.	\\
	\textbf{Postcondizione}: l’utente desktop ha modificato il nome di un elemento scelta.	\\
	\textbf{Procedura principale}:
	\begin{enumerate}
		\item l’utente desktop seleziona un elemento scelta;
		\item l’utente desktop modifica il testo del nome assegnato all’elemento scelta.
	\end{enumerate}
	}
\subsection{UC 1.3. - }{
	\label{uc1.3.}
	\textbf{Attori}: utente desktop \\
	\textbf{Descrizione}: . \\
	\textbf{Precondizione}: .	\\
	\textbf{Postcondizione}: .	\\
	\textbf{Procedura principale}:
	\begin{enumerate}
		\item ;
		\item ;
		\item .
	\end{enumerate}
	}
\subsection{UC 1.3.3.11 - Modifica della dimensione del frame}{
	\label{uc1.3.3.11}
	\textbf{Attori}: utente desktop \\
	\textbf{Descrizione}: l’utente desktop modifica la dimensione del frame. \\
	\textbf{Precondizione}: l’utente desktop desidera modificare la dimensione del frame.	\\
	\textbf{Postcondizione}: la presentazione contiene un frame con una nuova dimensione.	\\
	\textbf{Procedura principale}:
	\begin{enumerate}
		\item l’utente desktop seleziona l’opzione per la modifica della dimensione del frame;
		\item l’utente desktop ridimensiona il frame.
	\end{enumerate}
	}
\subsection{UC 1.3.3.12 - Modifica della forma di un frame}{
	\label{uc1.3.3.12}
	\textbf{Attori}: utente desktop \\
	\textbf{Descrizione}: l’utente desktop modifica la forma del frame. \\
	\textbf{Precondizione}: l’utente desktop desidera modificare la forma del frame.	\\
	\textbf{Postcondizione}: la presentazione contiene un frame con una nuova forma.	\\
	\textbf{Procedura principale}:
	\begin{enumerate}
		\item l’utente desktop seleziona l’opzione per la modifica della forma del frame;
		\item l’utente desktop seleziona la nuova forma del frame.
	\end{enumerate}
	}
\subsection{UC 1.3.3.13 - Modifica dello spessore del bordo del frame}{
	\label{uc1.3.3.13}
	\textbf{Attori}: utente desktop \\
	\textbf{Descrizione}: l’utente desktop modifica lo spessore del bordo del frame. \\
	\textbf{Precondizione}: l’utente desktop desidera modificare lo spessore del bordo del frame.	\\
	\textbf{Postcondizione}: la presentazione contiene un frame con un bordo con un nuovo spessore.	\\
	\textbf{Procedura principale}:
	\begin{enumerate}
		\item l’utente desktop seleziona l’opzione per la modifica dello spessore del bordo del frame;
		\item l’utente desktop ridimensiona lo spessore del bordo del frame.
	\end{enumerate}
	}
\subsection{UC 1.3.3.14 - Modifica del colore del bordo del frame}{
	\label{uc1.3.3.14}
	\textbf{Attori}: utente desktop \\
	\textbf{Descrizione}: l’utente desktop modifica il colore del bordo del frame. \\
	\textbf{Precondizione}: l’utente desktop desidera modificare il colore del bordo del frame.	\\
	\textbf{Postcondizione}: la presentazione contiene un frame con un bordo con un nuovo colore.	\\
	\textbf{Procedura principale}:
	\begin{enumerate}
		\item l’utente desktop seleziona l’opzione per la modifica del colore del bordo del frame;
		\item l’utente desktop seleziona il nuovo colore del bordo del frame.
	\end{enumerate}
	}
\subsection{UC 1.3.3.15 - Modifica dello sfondo del frame}{
	\label{uc1.3.3.15}
	\textbf{Attori}: utente desktop \\
	\textbf{Descrizione}: l’utente desktop modifica lo sfondo del frame. \\
	\textbf{Precondizione}: l’utente desktop desidera modificare lo sfondo del frame.	\\
	\textbf{Postcondizione}: la presentazione contiene un frame con un nuovo sfondo.	\\
	\textbf{Procedura principale}:
	\begin{enumerate}
		\item l’utente desktop seleziona l’opzione per la modifica dello sfondo del frame;
		\item l’utente desktop può selezionare un nuovo colore di sfondo del frame;
		\item l’utente desktop può selezionare una nuova immagine di sfondo del frame.
	\end{enumerate}
	}