\subsection{UC 1.4 - Modifica mobile di una presentazione}{
	\label{uc1.4}
	\begin{figure}[H]
		\centering
		\includegraphics[scale=0.75]{\imgs {UC1.4}.jpg} %inserire il diagramma UML
	\end{figure}
	\textbf{Attori}: utente mobile \\
	\textbf{Descrizione}: L'utente mobile ha scelto l'opzione di modifica della presentazione. L'utente mobile può scegliere di modificare un frame o di gestire i bookmark. \\
	\textbf{Precondizione}: l'utente mobile intende modificare una presetazione.	\\
	\textbf{Postcondizione}: il sistema contiene una presentazione modificata.	\\
	\textbf{Procedura principale}:
	\begin{enumerate}
		\item modifica mobile di un frame \hyperref[uc1.4.1]{(UC 1.4.1)};
		\item gestione mobile dei bookmark \hyperref[uc1.4.2]{(UC 1.4.2)}.
	\end{enumerate}
	\textbf{Scenari alternativi}: 
	\begin{itemize}
		\item solo nel caso in cui venga effettuata una modifica l'utente mobile può scegliere di annullare l'ultima modifica eseguita.
	\end{itemize}
	}
\subsection{UC 1.4.1 - Modifica mobile di un frame}{
	\label{uc1.4.1}
	\begin{figure}[H]
		\centering
		\includegraphics[scale=0.75]{\imgs {UC1.4.1}.jpg} %inserire il diagramma UML
	\end{figure}
	\textbf{Attori}: utente mobile \\
	\textbf{Descrizione}: l'utente mobile ha scelto l'opzione di modifica di un frame. L'utente mobile può scegliere di inserire o modificare un elemento testo. \\
	\textbf{Precondizione}: l'utente mobile intende modificare un frame.	\\
	\textbf{Postcondizione}: la presentazione contiene un frame modificato.	\\
	\textbf{Procedura principale}:
	\begin{enumerate}
		\item inserimento di un elemento testo \hyperref[uc1.4.1.1]{(UC 1.4.1.1)};
		\item modifica di un elemento testo \hyperref[uc1.4.1.2]{(UC 1.4.1.2)}.
	\end{enumerate}
	}
\subsection{UC 1.4.1.1 - Inserimento di un elemento testo}{
	\label{uc1.4.1.1}
	\textbf{Attori}: utente mobile \\
	\textbf{Descrizione}: l'utente mobile inserisce un elemento di tipo testo all'interno del frame. \\
	\textbf{Precondizione}: l'utente mobile desidera inserire un nuovo elemento testo.	\\
	\textbf{Postcondizione}: nel frame è presente un nuovo elemento testo.	\\
	\textbf{Procedura principale}:
	\begin{enumerate}
		\item l'utente mobile seleziona l'opzione di inserimento testo;
		\item l'utente mobile inserisce il testo desiderato.
	\end{enumerate}
	}
\subsection{UC 1.4.1.2 - Modifica di un elemento testo}{
	\label{uc1.4.1.2}
	\textbf{Attori}: utente mobile \\
	\textbf{Descrizione}: l'utente mobile modifica un elemento testo presente all'interno del frame. \\
	\textbf{Precondizione}: l'utente mobile desidera modificare un elemento testo.	\\
	\textbf{Postcondizione}: nel frame è presente un elemento testo modificato.	\\
	\textbf{Procedura principale}:
	\begin{enumerate}
		\item l'utente mobile seleziona un elemento testo;
		\item l'utente mobile modifica il testo selezionato.
	\end{enumerate}
	}	
\subsection{UC 1.4.2 - Gestione mobile dei bookmark}{
	\label{uc1.4.2}
	\begin{figure}[H]
		\centering
		\includegraphics[scale=0.75]{\imgs {UC1.4.2}.jpg} %inserire il diagramma UML
	\end{figure}
	\textbf{Attori}: utente mobile \\
	\textbf{Descrizione}: l'utente mobile ha scelto l'opzione di gestione dei bookmark. L'utente mobile può scegliere di inserire o rimuovere bookmark. \\
	\textbf{Precondizione}: l'utente mobile intende gestire i bookmark.	\\
	\textbf{Postcondizione}: la presentazione contiene una diversa gestione dei bookmark.	\\
	\textbf{Procedura principale}:
	\begin{enumerate}
		\item inserimento di un nuovo bookmark \hyperref[uc1.4.2.1]{(UC 1.4.2.1)};
		\item rimozione di un bookmark \hyperref[uc1.4.2.2]{(UC 1.4.2.2)}.
	\end{enumerate}
	}
\subsection{UC 1.4.2.1 - Inserimento di un nuovo bookmark}{
	\label{uc1.4.2.1}
	\textbf{Attori}: utente mobile \\
	\textbf{Descrizione}: l'utente mobile inserisce un bookmark su un frame che non ne contiene uno. \\
	\textbf{Precondizione}: l'utente mobile intende inserire un bookmark su un frame che non contiene bookmark.	\\
	\textbf{Postcondizione}: il frame selezionato contiene un bookmark.	\\
	\textbf{Procedura principale}:
	\begin{enumerate}
		\item l'utente mobile seleziona un frame assegnando il bookmark.
	\end{enumerate}
	}
\subsection{UC 1.4.2.2 - Rimozione di un bookmark}{
	\label{uc1.4.2.2}
	\textbf{Attori}: utente mobile \\
	\textbf{Descrizione}: l'utente mobile rimuove un bookmark su un frame che ne contiene uno. \\
	\textbf{Precondizione}: l'utente mobile intende rimuovere un bookmark da un frame che ne contiene uno.	\\
	\textbf{Postcondizione}: il frame selezionato non contiene un bookmark.	\\
	\textbf{Procedura principale}:
	\begin{enumerate}
		\item l'utente mobile seleziona un frame rimuovendo il bookmark.
	\end{enumerate}
	}