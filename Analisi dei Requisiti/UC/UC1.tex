\subsection{UC 1 - Caso generale}{
	\label{uc1}
	\begin{figure}[H]
		\centering
		\includegraphics[scale=0.75]{\imgs {UC1}.jpg} %inserire il diagramma UML
	\end{figure}
	\textbf{Attori}: utente desktop, utente mobile, amministratore di sistema \\
	\textbf{Descrizione}: il sistema deve mettere l'utente in condizioni di raccogliere idee e materiale multimediale da essere usati in una presentazione, creare una presentazione, modificare una presentazione, creare infografiche a partire da una presentazione, salvare ciò che ha prodotto su di un server ed infine eseguire una presentazione creata con il sistema. \\
	\textbf{Precondizione}: il sistema è avviato e pronto all'uso.	\\
	\textbf{Postcondizione}: è stato eseguito ciò che l'utente desiderava e che il sistema è in grado di eseguire.	\\
	\textbf{Procedura principale}:
	\begin{enumerate}
		\item l'utente si registra creando un account utente \hyperref[uc1.8]{(UC 1.8)};
		\item l'utente può gestire il suo account \hyperref[uc1.11]{(UC 1.11)};
		\item l'utente raccoglie idee e materiale multimediale per la presentazione \hyperref[uc1.1]{(UC 1.1)};
		\item l'utente può sincronizzare i contenuti di preparazione sul proprio spazio a disposizione sul server dopo aver effettuato il login con il suo account \hyperref[uc1.12]{(UC 1.12)};
		\item l'utente desktop può creare una nuova presentazione \hyperref[uc1.2]{(UC 1.2)};
		\item l'utente desktop può caricare o ottenere una presentazione sul proprio spazio a disposizione sul server dopo aver effettuato il login con il suo account \hyperref[uc1.7]{(UC 1.7)};
		\item l'utente può creare una infografica della presentazione \hyperref[uc1.6]{(UC 1.6)};
		\item l'utente può caricare l'infografica sul proprio spazio a disposizione sul server \hyperref[uc1.13]{(UC 1.13)};
		\item l'utente può ottenere una presentazione dal server tramite il suo account \hyperref[uc1.7]{(UC 1.7)};
		\item l'utente desktop può modificare la presentazione \hyperref[uc1.3]{(UC 1.3)};
		\item l'utente mobile può modificare­ parzialmente la presentazione \hyperref[uc1.4]{(UC 1.4)};
		\item l'utente può esegue la presentazione \hyperref[uc1.5]{(UC 1.5)}.
	\end{enumerate}
	}