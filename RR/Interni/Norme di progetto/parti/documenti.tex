\section{Documenti}{
	\label{sec:docs}
	\subsection{Contenuto e struttura dei documenti}{
		\label{sub:content}
		Tutti i documenti devono essere realizzati utilizzando un template\ped{g} \LaTeX. Onde evitare modifiche manuali che farebbero perdere molto tempo, il template\ped{g} permette di effettuare modifiche di stile le quali si riflettono in maniera automatica in tutti i documenti. In particolare, il template\ped{g} regola:
		\begin{itemize}
			\item Pacchetti\ped{g} \LaTeX\ usati;
			\item Variabili globali valide per ogni documento;
			\item Formattazione stili delle pagine, degli header e dei footer.
		\end{itemize}
		Qualora si volessero effettuare delle modifiche al template\ped{g} sarà il responsabile di progetto\ped{g} ad applicarle.\\
		\\
		Il nome dei file\ped{g} deve rispondere alla seguente formattazione senza spazi: “[nome documento]-[versione]”. La parte della versione deve riportare la dicitura “v.” seguita dal numero di versione (ad es: NormeDiProgetto-v.1.0.0.pdf”).\\
		Ogni documento ufficiale deve essere composto dalle seguenti sezioni:
		\begin{itemize}
			\item Prima pagina: deve riportare titolo, logo ed informazioni del documento
			\item Breve prefazione;
			\item Registro delle modifiche;
			\item Indice del documento;
			\item Indice di figure e tabelle (se presenti);
			\item Introduzione;
			\item Corpo.
		\end{itemize}
		Ogni pagina deve avere un'intestazione e un piè di pagina:
		\begin{itemize}
			\item \textbf{Intestazione}: logo del gruppo e nome del documento;
			\item \textbf{Piè di pagina}: versione documento, università e anno accademico, numeri di pagina e licenza.
		\end{itemize}
		Per quanto riguarda i \textbf{Verbali degli incontri}, essi devono essere redatti dal Responsabile di Progetto\ped{g} ad ogni riunione. Esso deve rispettare la formattazione regolata alla sezione \ref{sub:typo} e successive ma è da considerarsi solo come promemoria per il gruppo.\\
		Il nome di ogni Verbale deve rispettare la seguente dicitura: “Verbale\_[tipo incontro]-[data]” dove il tipo incontro può essere di due tipi:
		\begin{itemize}
			\item Interno (INT): incontro effettuato tra i membri del gruppo;
			\item Esterno (EXT): incontro effettuato tra i membri del gruppo e committente\ped{g} e/o proponente\ped{g}.
		\end{itemize}
		La prima pagina di ogni verbale deve obbligatoriamente contenere i seguenti campi, in ordine:
		\begin{itemize}
			\item Data;
			\item Luogo secondo il formato “[città],[provincia],[sede]”;
			\item Ora secondo il formato “dalle ore [hh]:[mm] alle ore [hh]:[mm]” dove hh indica le ore e mm i minuti i quali vanno espressi nel formato 24 ore secondo lo standard ISO\ped{g} 8601:2004;
			\item Partecipanti interni al gruppo elencandoli rispettando il formato “[nome] [cognome][,[...]]”;
			\item Partecipanti esterni al gruppo rispettando il formato “[nome] [cognome][ruolo][,[...]]” in cui il ruolo può essere Committente\ped{g} oppure Proponente\ped{g};
			\item Contenuto dell'incontro;
			\item Firme: devono essere comprese quelle di tutti i partecipanti del gruppo \gruppo\ a conferma della presa visione del documento.
		\end{itemize}
		Infine, la \textbf{Lettera di presentazione} dei documenti deve contenere:
		\begin{itemize}
			\item Logo del gruppo;
			\item Intestazione nel seguente formato:\\
					Prof. Tullio Vardanega\\
					Università degli Studi di Padova\\
					Via Trieste 63\\
					35121 Padova (PD)
			\item Breve introduzione (facoltativa);
			\item Elenco di tutti i documenti in consegna;
			\item Varie ed eventuali, osservazioni (facoltative);
			\item Firma del responsabile nel seguente formato:\\
					{Nome} {Cognome}\\
					il Responsabile del gruppo \gruppo \\
					{Firma del responsabile}
		\end{itemize}		
	}
		\subsection{Norme tipografiche}{
			\label{sub:typo}
			Per rendere la documentazione organizzata, leggibile e standard abbiamo adottato le forme testuali riportate di seguito.
			\begin{itemize}
				\item \textbf{Carattere}: il carattere dovrà avere come dimensione minima 12. Per l'inserimento di linee di codice\ped{g} il carattere da utilizzare dovrà essere di tipo Monospace;
				\item \textbf{Grassetto}: da utilizzare maggiormente per definire i titoli e dare una panoramica generale del testo ed in maniera minore per sottolineare passaggi importanti e parole chiave;
				\item \textbf{Corsivo}: da utilizzare per riportare citazioni da fonti esterne o riferimenti;
				\item \textbf{Sottolineato}: da utilizzare all'interno del testo per dare importanza a determinati concetti;
				\item \textbf{Maiuscolo}: deve essere limitato all’indicazione di acronimi e nei casi specificati nei Formati di Riferimento (\ref{sub:rif});
				\item \textbf{Punteggiatura}: adottare la formattazione standard ossia la punteggiatura deve precedere sempre un carattere di spazio e non viceversa;
				\item \textbf{Lettera maiuscola}: deve seguire esclusivamente un punto, un punto esclamativo o un punto interrogativo;
				\item \textbf{Parentesi}: una qualsiasi frase racchiusa fra parentesi non deve iniziare con un carattere di spaziatura e non deve chiudersi con un carattere di punteggiatura e/o di spaziatura;
				\item \textbf{Elenchi puntati o numerati}: ogni elemento dell’elenco deve terminare con un punto e virgola, tranne l’ultimo che deve terminare con un punto. La prima parola deve avere la lettera maiuscola, a meno di casi particolari (es. nome di un file\ped{g});
				\item \textbf{Glossario}: le parole accompagnate da (g) in pedice sono quelle che presentano una corrispondenza nel Glossario;
				\item \textbf{Pagine}: è obbligatorio porre i numeri di pagina in ogni documento nel formato {n} di {totale pagine} e mantenere i margini fissati dal template\ped{g} di cui sopra (\ref{sub:content}).
			\end{itemize}
		}
		\subsection{Formati di riferimento e altro}{
			\label{sub:rif}
			Per quanto riguarda i riferimenti, è opportuno rispettare le seguenti indicazioni:
			\begin{itemize}
				\item Percorsi\ped{g}: per gli indirizzi\ped{g} web\ped{g} completi e indirizzi\ped{g} e-mail deve essere utilizzato il comando appositamente fornito da \LaTeX:\\ \textbackslash url\ped{g} \textbraceleft Percorso\ped{g} nel formato tipografico monospace \textbraceright;
				\item Link\ped{g} a file\ped{g} PDF: devono essere contrassegnati in corsivo mediante l’uso del comando personalizzato: \textbackslash nomePdf;
				\item Ancore: i riferimenti alle sezioni interne del medesimo documento devono essere scritte utilizzando il comando fornito da \LaTeX: \textbackslash ref \textbraceleft label da riferire \textbraceright .
			\end{itemize}
			La \textbf{Data} deve essere espressa, seguendo lo standard ISO\ped{g} 8601:2004, nel formato: AAAA-MM-GG (AAAA rappresenta l'anno in quattro cifre, MM il mese in due cifre e GG il giorno in due cifre).\\
			\\
			Le \textbf{Abbreviazioni} ammesse sono le seguenti e valgono per tutti i documenti:
			\begin{itemize}
				\item \textbf{AR}: Analisi dei Requisiti\ped{g};
				\item \textbf{GL}: Glossario;
				\item \textbf{NP}: Norme di Progetto\ped{g};
				\item \textbf{PQ}: Piano di Qualifica;
				\item \textbf{PP}: Piano di Progetto\ped{g};
				\item \textbf{SF}: Studio di Fattibilità;
				\item \textbf{RR}: Revisione dei Requisiti\ped{g};
				\item \textbf{RP}: Revisione di Progettazione;
				\item \textbf{RQ}: Revisione di Qualifica;
				\item \textbf{RA}: Revisione di Accettazione.
			\end{itemize}
			I \textbf{Nomi ricorrenti} nei vari documenti devono rispettare le seguenti indicazioni:
			\begin{itemize}
				\item Ruoli di progetto\ped{g} e nomi dei documenti: devono essere formattati utilizzando la prima lettera maiuscola di ogni parola che non sia una preposizione (es. Responsabile di Progetto\ped{g});
				\item Nomi dei file\ped{g}: il riferimento deve essere comprensivo dell’estensione del file\ped{g} e formattato in corsivo;
				\item Nomi propri: l’utilizzo dei nomi propri deve seguire il formalismo Cognome Nome;
				\item Nome del gruppo: deve essere sempre espresso nel formato: \gruppo;
				\item Nome del progetto\ped{g}: deve essere sempre espresso nel formato: \premi.
			\end{itemize}
			}
		\subsection{Immagini e tabelle}{
			\label{sub:img}
			Tutte le immagini devono essere in formato JPG, PNG o PDF mentre ogni tabella deve rispettare il formato \LaTeX.\\
			Ogni figura o tabella inserita deve avere una breve didascalia composta da un identificativo numerico univoco seguito, ove sia ritenuto necessario, da una breve descrizione. La numerazione di immagini e tabelle sarà attribuita da \LaTeX e dovrà essere inserito un indice che riporti tutte le figure presenti.\\			
			}
		}