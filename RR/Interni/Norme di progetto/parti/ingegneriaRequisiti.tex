\section{Processi di ingegneria dei requisiti}{
	\subsection{Fattibilit\`{a}}{
		A partire da informazioni preliminari sul capitolato, lo studio di fattibilit\`{a} dovr\`{a} generare un rapporto che indichi la convenienza o meno del gruppo nello sviluppo del sistema. In particolare si dovr\`{a} considerare:
		\begin{enumerate}
			\item Sufficienza di risorse\ped{g} umane;
			\item Rapporto tra i costi ed i benefici;
			\item Rischi individuati.
		\end{enumerate}
		Nello stimare i benefici dovr\`{a} essere data molta importanza alle competenze che i membri del gruppo acquisirebbero nello sviluppo del sistema.
	 }
	\subsection{Deduzione e analisi dei requisiti}{ 
	\phantomsection
		\subsubsection{Scoperta dei requisiti}{
			\textbf{Problemi e fonti}\\\\
			L’analisi dei requisiti\ped{g} dovr\`{a} iniziare dall’identificazione dei bisogni. Questi dovranno essere ottenuti da:
			\begin{enumerate}
				\item \{\textbf{CAP}\}: capitolato d'appalto proposto;
				\item \{\textbf{PRO}\}: minute degli incontri con il proponente\ped{g};
				\item \{\textbf{INT}\}: standard di qualit\`{a} del software\ped{g} (ISO/IEC 9126:2001).
			\end{enumerate}
			I bisogni dovranno essere enumerati così da essere tracciabili con i requisiti\ped{g} specificati.
			La enumerazione dovr\`{a} considerare la provenienza usando i codici\ped{g} sopra descritti. La numerazione dei bisogni non sar\`{a} sequenziale per permettere maggiore flessibilit\`{a} nella loro gestione.\\\\
			\textbf{Interviste}\\\\
			al fine di evitare interviste infruttuose verr\`{a} preparato un elenco di punti da sottoporre al proponente\ped{g} in modo da dare una direzione precisa all’intervista. Potrebbe essere utile discutere con il proponente\ped{g} dei casi d'uso analizzati internamente al gruppo durante la fase di analisi.
			Le richieste di interviste al proponente\ped{g} avverranno con le modalit\`{a} descritte in ”comunicazioni esterne”. Durante ogni intervista dovr\`{a} essere scritta una minuta che sar\`{a} confermata dal proponente\ped{g}, eventualmente con le opportune modifiche. La minuta sar\`{a} confermata al termine dell’incontro. Quando non fosse un problema per il proponente\ped{g} l’audio dell’intervista dovr\`{a} essere registrato per favorire la futura fase di analisi.\\\\
			\textbf{Riunioni interne e casi d'uso}\\\\
			Individualmente e durante le riunioni interne gli analisti dovranno analizzare le informazioni raccolte dalle interviste con il proponente\ped{g} per individuare problemi e fonti da cui attingere i requisiti\ped{g}.\\
			L’individuazione dei requisiti\ped{g} funzionali sar\`{a} guidata dai casi d’uso. I casi d’uso potranno avere rappresentazione a diagrammi ma ogni caso d’uso dovr\`{a} avere anche la rappresentazione testuale. In particolare nella rappresentazione testuale si definir\`{a}:
			\begin{enumerate}
				\item Identificativo;
				\item Attore primario;
				\item Precondizioni;
				\item Postcondizioni;
				\item Scenario principale;
				\item Estensioni\ped{g}.
			\end{enumerate}
			Per la sintassi si rimanda a ”Dall’idea al codice\ped{g} con UML2.0, Luciano Baresi, Luigi Lavazza, Massimiliano Pianciamore”.
			}
			\subsubsection{Classificazone e priorit\`{a}}{
				I requisiti\ped{g} dovranno essere classificati in:
				\begin{enumerate}
					\item Requisiti\ped{g} di processo\ped{g};
					\item Requisiti\ped{g} di prodotto.
				\end{enumerate}
				I requisiti\ped{g} di prodotto saranno classificati in base a:
				\begin{enumerate}
					\item Importanza;
					\item Provenienza;
					\item Tipologia.
				\end{enumerate}
				Dove i gradi di importanza saranno:
				\begin{itemize}
						\item \{\textbf{OB}\}: requisito\ped{g} obbligatorio;
						\item \{\textbf{DE}\}: requisito\ped{g} desiderabile;
						\item \{\textbf{OP}\}: requisito\ped{g} opzionale.
				\end{itemize}
				La provenienza pu\`{o} essere:
				\begin{itemize}
					\item \{\textbf{CAP}\}: da capitolato;
					\item \{\textbf{INT}\}: da analisi interna;
					\item \{\textbf{PRO}\}: da incontro con proponente\ped{g}.
				\end{itemize}
				Mentre le tipologie saranno:
				\begin{itemize}
					\item \{\textbf{F}\}: requisito\ped{g} funzionale;
					\item \{\textbf{P}\}: requisito\ped{g} prestazionale;
					\item \{\textbf{Q}\}: requisito\ped{g} di qualit\`{a};
					\item \{\textbf{V}\}: requisito\ped{g} di vincolo.
				\end{itemize}
			}
			\subsubsection{Specifica}{
				Nella specifica dei requisiti\ped{g} dovr\`{a} essere considerato come riferimento lo standard IEEE 830-1998. In particolare saranno da perseguire le seguenti caratteristiche dei requisiti\ped{g}:
				\begin{enumerate}
					\item Non ambigui;
					\item Corretti;
					\item Completi;
					\item Verificabili;
					\item Consistenti;
					\item Modificabili;
					\item Tracciabili;
					\item Ordinati per rilevanza.
				\end{enumerate}
				I requisiti\ped{g} dovranno essere specificati in un documento ”Analisi dei requisiti” secondo la struttura definita nello standard IEEE 830-1998. La specifica dei requisiti\ped{g} dovr\`{a} essere documentata in forma tabellare per evitare ambiguit\`{a}. Per ogni requisito\ped{g} dovranno essere definiti un codice\ped{g}, una descrizione, un riferimento alla fonte e un riferimento alla verifica. Al fine di rendere meno ambigui i requisiti\ped{g} sara redatto un ”Glossario” contenente la definizione di tutti i termini non ovvi usati in fase di analisi.
			}
			\subsubsection{Verifica dei requisiti}{
				Per ogni requisito\ped{g} di processo\ped{g} specificato dovr\`{a} essere presente in ”Piano di qualifica” un riferimento alle sezioni di ”Norme di progetto” in cui viene assicurato il soddisfacimento del requisito\ped{g}. Per ogni requisito\ped{g} di prodotto specificato dovr\`{a} essere descritto brevemente il metodo che verr\`{a} usato per verificarne il soddisfacimento.\\Per favorire la tracciabilit\`{a} tra requisiti\ped{g} e metodi di verifica dovr\`{a} essere presente in ”Piano di qualifica” una tabella in cui si definiscono: codice\ped{g} di requisito\ped{g}, codice\ped{g} di verifica e modalit\`{a} di verifica. Se il requisito\ped{g} \`{e} di processo\ped{g}, la modalit\`{a} di verifica conterr\`{a} i riferimenti alle sezioni corrispondenti in ”Norme di progetto”.
			}
		}
		\subsection{Validazione dei requisiti}{
			\subsubsection{Interna}{
				Saranno verificate la correttezza e la completezza dei requisiti\ped{g} rispetto ai bisogni. Ci\`{o} verr\`{a} fatto tramite tracciamento tra specifica dei requisiti\ped{g} e bisogni individuati.\\Saranno verificate la correttezza e la completezza dei metodi di verifica dei requisiti\ped{g}
				rispetto ai requisiti\ped{g}. Ci\`{o} verr\`{a} fatto tramite tracciamento tra specifica dei requisiti\ped{g} e metodi di verifica.
			}
				\subsubsection{Esterna}{
					Terminata la validazione\ped{g} interna verranno presentati al proponente\ped{g} i documenti ”Analisi dei requisiti” e ”Piano di qualifica”, se accettati costituiranno una baseline per la fase successiva del progetto\ped{g} altrimenti verranno gestite le richieste di modifica secondo i metodi descritti in ”Gestione dei cambiamenti”.
				}
		}
		\subsection{Gestione delle modifiche ai requisiti}{
			A tutte le proposte di modifica dei requisiti\ped{g} dovr\`{a} essere applicata la seguente procedura:
			\begin{enumerate}
				\item Deduzione, analisi e specifica dei cambiamenti;
				\item Stima dei costi del cambiamento considerando quante modifiche dovranno essere fatte ai requisiti\ped{g} e al progetto\ped{g} del sistema;
				\item Decisione ed eventuale implementazione del cambiamento nei requisiti\ped{g} e nel progetto\ped{g} di sistema.
			\end{enumerate}
			Per gestire i cambiamenti e per facilitare il tracciamento dei requisiti\ped{g} verrà usato un software\ped{g} appositamente creato dal gruppo. L’amministratore avrà il compito di gestire il server e amministrare i diritti di accesso degli utenti alle funzionalit\`{a} fornite. In particolare gli analisti dovranno usare i modelli definiti all’inizio della fase di analisi. Per evitare problemi dovuti a modifiche concorrenti alla base dati l’amministratore dovr\`{a} garantire che ad ogni istante solo un analista possa modificare un certo sottoalbero della foresta dei requisiti\ped{g} e dei test.
			}