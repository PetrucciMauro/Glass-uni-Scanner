\section{???}{
	\subsection{Gestione degli incarichi}{
		Per il servizio di gestione degli incarichi verr\`{a} usato lo strumento Redmine, seguendo una specifica normativa.\\Di seguito \`{e} elencata la prassi relativa all’assegnazione degli incarichi ai membri del team:
		\begin{enumerate}
			\item Il Responsabile crea una milestone\ped{g} e la render\`{a} pubblica al gruppo. E’ suo incarico controllare lo stato di avanzamento della milestone\ped{g} e dei relativi ticket\ped{g}.
			\item I ticket\ped{g} saranno creati dal Responsabile; ognuno di essi rappresenter\`{a} un lavoro da svolgere. Ogni ticket\ped{g} verr\`{a} assegnato al componente del gruppo pi\`{u} idoneo a soddisfare il relativo compito. Ogni ticket\ped{g} verr\`{a} compilato secondo le seguenti direttive:
			\begin{itemize}
				\item \textbf{Title}: specifica, in maniera sintetica, l’oggetto del compito
				\item \textbf{Status}: pu\`{o} essere \emph{open, accepted o close} e rappresenta lo stato attuale del ticket\ped{g}
				\item \textbf{Owner}: rappresenta il membro del gruppo che dovr\`{a} adempiere all'incarico
				\item \textbf{Labels}: indica il campo del ticket\ped{g}, pu\`{o} essere \textbf{D} (per i documenti) oppure \textbf{C} (per il codice\ped{g})
				\item \textbf{Private}: indica il compito come privato e la scelta \`{e} decisa dal Responsabile
				\item \textbf{Summary}: contiene la descrizione di ci\`{o} che il proprietario dell’incarico dovr\`{a} svolgere
			\end{itemize}
			\item Il Responsabile avr\`{a}  il compito di  creare i \textbf{ticket di verifica}: ad ogni ticket\ped{g} segnalato closed ne verr\`{a} associato uno nuovo con i seguenti campi:
			\begin{itemize}
				\item \textbf{Title}: \`{e} di tipo: \begin{center}
				VERIFICA \{\emph{Oggetto del compito da verificare}\}
				\end{center}
				\item \textbf{Status}: come in precedenza \`{e} da impostare come \emph{open}
				\item \textbf{Owner}: questa volta sar\`{a} il riferimento del Verificatore proprietario di questo ticket\ped{g}
				\item \textbf{Labels}: si aggiunge al label del ticket\ped{g} originale il suffisso \textbf{V}
				\item \textbf{Private}: indica il compito come privato e la scelta \`{e} lasciata al Responsabile 
				\item \textbf{Summary}: descrizione fatta dal Responsabile
			\end{itemize}
			\item Il Verificatore pu\`{o} rifiutare il ticket\ped{g} impostando il campo \textbf{Status} a \emph{won't fix} specificandone il motivo nel campo \textbf{Summary}. Nel caso accettasse il ticket\ped{g} imposter\`{a} lo \textbf{Status} a \emph{accepted}  e proceder\`{a} con la fase di verifica.
			Se la verifica ha esito positivo lo stato passer\`{a} a \emph{closed} mentre nel caso opposto il Verificatore dovr\`{a} creare un nuovo ticket\ped{g} con i seguenti campi:
			
			\begin{itemize}
				\item \textbf{Title}: specifica, in maniera sintetica, l’oggetto del compito
				\item \textbf{Status}: impostato a \emph{open}
				\item \textbf{Owner}: in questo caso il riferimento sar\`{a} il Responsabile 
				\item \textbf{Labels}: label del ticket\ped{g} originale
				\item \textbf{Private}: indica il compito come privato e la scelta \`{e} decisa dal Verificatore
				\item \textbf{Summary}: descrive l’errore
			\end{itemize}
			Il Responsabile dovr\`{a} decidere se accettare o rifiutare il ticket\ped{g} di verifica. In caso positivo dovr\`{a} segnare nel campo \textbf{Owner} il riferimento all’addetto del soddisfacimento di quel compito. Una volta che quel ticket\ped{g} verr\`{a} segnalato come closed il Responsabile dovr\`{a} seguire una nuova procedura di verifica.
			\item Infine una volta che tutti i ticket\ped{g} appartenenti alla milestone\ped{g} saranno segnalati come \emph{closed} il Responsabile potr\`{a} rendere la milestone\ped{g} \emph{closed} ed aprirne una nuova ripartendo dal primo punto.
		\end{enumerate}

		
	 }
}