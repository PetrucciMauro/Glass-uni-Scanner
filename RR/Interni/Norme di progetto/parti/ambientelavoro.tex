\section{Ambiente di lavoro} 

\subsection{Sistemi Operativi}

L’intero sviluppo del progetto viene svolto in ambienti Unix-Like e Windows, nello specifico, Ubuntu , Mac , Windows . Tale scelta \`{e} maturata dopo aver appurato che le tecnologie utilizzate per lo sviluppo del progetto sono indipendenti dall’ambiente di sviluppo e di impiego.

\subsection{Coordinamento}

\`{e} stato predisposto un server dedicato sul quale sono installate alcune applicazioni web
che facilitano la gestione del progetto. Per connettersi al server, l'indirizzo \`{e} il seguente:\\
\begin{center}
\url{http://gioberry.no-ip.org/}
\end{center}
\subsubsection{Software di gestione del progetto} 
\label{subsec:Software di gestione del prodotto}
Come piattaforma di gestione del progetto \`{e} stato scelto \textbf{Redmine}. Redmine fornisce:
\begin{itemize}
\item Un sistema flessibile di gestione dei ticket;
\item Il grafico Gantt delle attivit\`{a};
\item Un calendario per organizzare i compiti;
\item La visualizzazione del repository associato al progetto;
\item Un sistema di rendicontazione del tempo.
\end{itemize}


\subsubsection{Versionamento}


Come strumento di versionamento si \`{e} deciso di utilizzare \textbf{Git}.
Git \`{e} uno strumento di versionamento veloce e di facile apprendimento che
rappresenta uno dei migliori strumenti attualmente esistenti.\\ Per lo sviluppo collaborativo abbiamo deciso di appoggiarci al servizio \textbf{Github} che fornisce non solo un repository Git, ma anche strumenti utili alla collaborazione fra pi\`{u} persone, come il servizio di \textbf{Ticket}, \textbf{Wiki} e \textbf{Milestone}.\\
Per quanto riguarda l’uso di Git sui computer di sviluppo, si \`{e} deciso l’uso
della versione ufficiale rilasciata dal team di sviluppo di Git(2.3.3).\\
Per interfacciarsi con il repository viene utlizzato \textbf{SmartGit}, un client multipiattaforma che permette di utilizzare Git in maniera rapida.\\
Si descrive ora la procedura di corretto utilizzo del programma smartgit .
\begin{itemize}

\item 	\textbf{Clonare il repository}: \`{e} possibile clonare il repository remoto in locale attraverso la seguente procedura:

\begin{itemize}
\item Premere nel menu in alto il pulsante Repository e successivamente Clone;
\item Nel riquadro comparso, inserire il link del repository\\ \url{https://github.com/PetrucciMauro/Premi.git}\\successivamente premere il pulsante  next;
\item Tenere la schermata successiva con entrambi i box spuntati e premere next;
\item Selezionare la posizione in cui verrà salvata la versione locale del repository.
\end{itemize}
\item \textbf{Sincronizzare il repository} : Dalla schermata principale premere il pulsante pull; 
\item \textbf{Salvare una modifica in locale}: Dalla schermata principale premendo il pulsante commit e inserendo nell'apposita textbox un "Messaggio di commit",si salvano le modifiche effettuate ai file;
\item \textbf{Inviare le modifiche al repository remoto}: Dalla schermata principale premere il pulsante Push e, successivamente alla comparsa del nuovo riquadro, ancora push, ci\`{o} comporter\`{a} l'invio delle modifiche ai file al repository remoto.

\end{itemize}

\subsubsection{Software di Integrazione Continua}

Si \`{e} scelto di adottare \textbf{Jenkins} per applicare l’integrazione continua allo sviluppo del progetto.\\ 
Tale software permette di pianificare ed eseguire dei compiti da eseguire sui file sorgente.\\
Mette inoltre a disposizione un cruscotto su cui \`{e} possibile visualizzare lo stato del codice prodotto. Tale software \`{e} infatti in grado di interagire con il software di versionamento,e se disponibile con software per l’esecuzione di test sul codice prodotto.\\ 
Attualmente Jenkins non viene utilizzato, si è solo cercato di impararne il funzionamento. 


\subsubsection{Condivisione dei file}

  Si \`{e} inoltre scelto di utilizzare degli strumenti online che permettono di condividere file
  in modo semplice e veloce e che consentono di organizzare gli appuntamenti personali
  dei singoli componenti del gruppo.
\subsubsubsection{Google Drive}
  In questa piattaforma di condivisione file verranno salvati i documenti che:
  \begin{itemize}
  
  
  \item Non necessitano di controllo di versione ;
  \item Hanno bisogno di grande interattivit\`{a} tra i componenti del gruppo;
  \item Possono essere acceduti tramite l’uso di un semplice browser.
   \end{itemize}
  Questo strumento dovrebbe permettere a 2 o pi\`{u} componenti del gruppo di interagire
  lavorando sugli stessi documenti contemporaneamente. Google Drive viene utilizzato
  come strumento di supporto allo sviluppo della documentazione e del software presente
  su Git .
  


\subsubsection{Google Calendar}
 
Google Calendar viene utilizzato all’interno del gruppo per gestire le risorse umane. In
particolare tale strumento viene utilizzato per notificare in quali giorni un determinato
membro non pu\`{o} essere disponibile e per segnalare date rilevanti per il gruppo, come
ad esempio le date delle riunioni.



\subsection{Strumenti per i documenti}
\subsubsection{LATEX} 
 
Per la stesura dei documenti \`{e} stato scelto di utilizzare il sistema \LaTeX.\\
Il motivo 
principale dietro a questa scelta \`{e} la facilit\`{a} di separazione tra contenuto e formattazione: 
con \LaTeX\ \`{e} possibile definire l’aspetto delle pagine in un file template condiviso da tutti i documenti. Altre soluzioni come Microsoft Office, LibreOffice o Google Docs non 
avrebbero consentito questa separazione, duplicando il lavoro di formattazione del testo 
e non garantendo un risultato uniforme.\\
Il grande numero di pacchetti esistenti consente di implementare funzionalit\`{a} comuni 
in maniera semplice. L’estensibilit\`{a} di \LaTeX\ pu\`{o} essere sfruttata per creare funzioni e 
variabili globali che rendono la scrittura del contenuto pi\`{u} corretta sotto un punto di 
vista semantico. Un esempio \`{e} dato dal comando /role \textbraceleft ruolo\textbraceright che identifica ogni ruolo 
all’interno del progetto.\\
Per la scrittura di documenti \LaTeX\  l’editor consigliato \`{e} \textbf{TeXstudio}. 

\subsubsection{Controllo ortografico}

Il software per il controllo ortografico \`{e} \textbf{Aspell}\ .\ Il programma \`{e} accessibile tramite il \emph{Makefile}.


\subsubsection{Grafici UML} 

Per la stesura dei grafici UML viene utilizzato il programma \textbf{Visual Paradigm}. Il programma viene utilizzato in licenza Community Edition la quale ne permette l’uso gratuito per fini non commerciali.



\subsubsection{Ambiente di verifica}

Si rimanda al \href{run:../../Esterni/Piano di qualifica/\PianoQual}{\PianoQualEscape} che specifica in maniera dettagliata
le tecniche e le modalità con cui verranno condotte le attività di verifica e validazione durante l’intero sviluppo del progetto.

\subsubsection{Fogli di calcolo}
L'utilizzo di fogli di calcolo elettronici quali Calc,Excel e Numbers è a discrezione del singolo componente in base alla propria piattaforma utilizzata.


\subsubsection{Codice} 

La verifica del codice \`{e} suddivisa in statica e dinamica.\\ 
Per entrambe vengono riportati gli strumenti da utilizzare. 

\subsubsubsection{Analisi Statica} 

  \begin{itemize}
  \item \textbf{jSHint}: tool che permette di rilevare potenziali errori nel codice javascript;
  \item \textbf{QUnit}: Framework per i test d'unit\`{a} del codice javascript;
  \item \textbf{jsmeter}: Strumento per il calcolo di alcune metriche del codice javascript.
  \end{itemize}

\subsubsubsection{Analisi Dinamica}
Verranno utilizzati strumenti e plugin interni al browser \emph{Chrome} quali \textbf{SpeedTracer} per verificare la velocità dell'applicazione web;


\subsubsubsection{Validazione}

La validazione del codice HTML e CSS dell’applicazione da noi sviluppata verr\`{a}
fatta tramite il servizio W3C Validator32 del W3C.


\section{Protocollo per lo sviluppo dell'applicazione} 
Per procedere con uno sviluppo controllato dei documenti e del codice si \`{e} scelto di adottare il sistema di ticketing \textbf{Redmine}.\\ 
La scelta di tale software \`{e} descritta nella sezione \ref{subsec:Software di gestione del prodotto}.

\subsection{Creare un nuovo progetto} 

La creazione di un progetto \`{e} compito del \emph{Responsabile di Progetto}.\\ 
Un nuovo progetto rappresenta una macro-attivit\`{a} caratterizzata da molte sotto-attivit\`{a} 
supervisionate da un responsabile.\\\\ 
Per creare un nuovo progetto:
\begin{itemize}
\item Aprire \textbf{Progetti}; 
\item Selezionare \textbf{Nuovo progetto}; 
\item Assegnare un \textbf{Nome} breve ma significativo; 
\item Nel caso in cui si voglia creare un sotto-progetto indicare il nome del progetto padre dall’omonimo campo; 
\item \textbf{Identificativo}: scrivere in minuscolo ed indicare codice della fase a cui si riferisce ;
\item Lasciare inalterati gli altri campi. 
\end{itemize}
 
\subsection{Creazione ticket}
 
  I ticket vengono creati da:
 \begin{itemize}
 

    \item \textbf{Responsabile di Progetto}: crea i ticket pi\`{u} importanti che rappresentano le macro fasi evidenziate dalla pianificazione; 
	\item \textbf{Responsabile di Sotto-progetto}: crea i ticket per i processi non pianificati inizialmente, che si evidenziano necessari per l’avanzamento del sotto-progetto assegnato; 
	\item \textbf{Verificatore}: crea i ticket per segnalare errori ed imprecisioni trovate durante il processo di verifica. 
 \end{itemize}


I ticket possono essere di tre tipologie:
\begin{itemize}


\item \textbf{Ticket di pianificazione}: rappresentano le macro-attivit\`{a} di maggiore importanza. Sono organizzate in una gerarchia con vari livelli di priorità.
 Tali attivit\`{a} vengono create da: 
\begin{itemize}
\item \emph{Responsabile di Progetto} : durante la pianificazione identifica le attivit\`{a} pi\`{u} importati e generali; 
\item \emph{Responsabile di Sotto-progetto}: durante lo svolgimento delle attivit\`{a} pu\`{o} scomporre in sotto-problemi l’attivit\`{a} indicata dal Responsabile di Progetto. 
\end{itemize}


\item \textbf{Ticket di realizzazione e controllo}: tutti i documenti redatti, durante la stesura attraversano due stadi: 
\begin{itemize}
\item \textbf{Realizzazione}: un redattore del documento effettua una prima stesura; 
\item \textbf{Controllo}: un redattore, diverso da quello della precedente fase, esegue un primo controllo sui contenuti della parte scritta. 
\end{itemize}


\item \textbf{Ticket di verifica}: rappresentano gli errori identificati dai Verificatori durante 
il controllo che la realizzazione dell’attivit\`{a} sia conforme a quanto richiesto e che 
rispetti tutte le norme.
\end{itemize}
}


\subsubsection{Ticket di pianificazione}

\begin{itemize}


\item Selezionare \textbf{Nuova segnalazione} da men\`{u} principale; 
\item \textbf{Tracker}: indicare la natura del ticket: 
	\begin{itemize}
	\item \textbf{Documento}: stesura di un documento. Il tipo di attivit\`{a} svolta dal redattore del documento viene definito durante la rendicontazione; 
	\item \textbf{Codifica}: stesura di codice; 
	\item \textbf{Verifica}: macro-attivit\`{a} di verifica sul prodotto dei sotto-processi. 


	\end{itemize}

\item \textbf{Oggetto}: descrizione breve e significativa; 
\item \textbf{Descrizione}: descrizione comprensibile e con riferimenti esterni mediante link se necessario; 
\item \textbf{Stato}: Plan; 
\item \textbf{Attivit\`{a} principale}: se si vuole creare una \textbf{sotto-attivit\`{a}} indicare l’id del ticket 
padre; 
\item \textbf{Categoria}: PDCA, solo se il ticket viene creato dal \emph{Responsabile di Progetto}; 
\item \textbf{Assegnato a}: inserire il nome del responsabile; 
\item \textbf{Osservatori}: aggiungere eventuali collaboratori.
\end{itemize}  


\subsubsection{Ticket di realizzazione e controllo} 

		\begin{itemize}
		
		\item Selezionare \textbf{Nuova segnalazione} da men\`{u} principale; 
		\item \textbf{Tracker}: indicare la natura del ticket: 
		\begin{itemize}
	\item \textbf{Documento}: stesura di un documento. Il tipo di attivit\`{a} svolta dal redattore del documento viene definito durante la rendicontazione; 
	\item \textbf{Codifica}: stesura di codice; 
	\item \textbf{Verifica}: attivit\`{a} di verifica sui prodotti dei processi. 

	\end{itemize}

\item \textbf{Oggetto}: descrizione breve e significativa secondo il principio: nome ticket padre attivit\`{a} da svolgere (realizzazione o controllo); 
\item \textbf{Descrizione}: descrizione comprensibile e con riferimenti esterni mediante link se 
necessario; 

\item \textbf{Stato}: New; 
\item \textbf{Attivit\`{a} principale}: se si vuole creare una \textbf{sotto-attivit\`{a}} indicare l’id del ticket 
padre; 
\item \textbf{Inizio}: dare una data di inizio presunta; 
\item \textbf{Scadenza}: dare una data di fine presunta; 
\item \textbf{Assegnato a}: inserire il nome del responsabile; 
\item \textbf{Osservatori}: aggiungere eventuali collaboratori. 
\end{itemize} 

\subsubsection{Ticket di verifica}


Un \emph{Verificatore} per creare un \emph{ticket di verifica} deve: 
\begin{enumerate}
\item assicurarsi che esista all’interno del progetto l’attivit\`{a} \emph{Verifica}.
Su tale attivit\`{a} vi devono essere due sotto-attivit\`{a}: “Verifica - realizzazione”, 
“Verifica - approvazione”. 
Tutti i ticket creati devono essere sotto-attivit\`{a} di: “Verifica - realizzazione”; 
\item Creare quindi il ticket secondo le seguenti direttive: 
		\begin{itemize}
		
		
		\item Selezionare \textbf{Nuova segnalazione} da men\`{u} principale; 
		\item \textbf{Tracker}: Bug; 
		\item \textbf{Oggetto}: descrizione breve e significativa dell’errore incontrato; 
		\item \textbf{Descrizione}: descrivere in modo dettagliato e chiaro: la natura e la posizione dell’errore; 
		\item \textbf{Stato}: New; 
		\item \textbf{Attivit\`{a} principale}: tutti i ticket devono essere figli del ticket “Verifica - realizzazione” del progetto su cui si sta eseguendo la verifica; 
		\item \textbf{Assegnato a}: inserire il nome del responsabile del progetto padre (es. 
		responsabile delle \emph{Norme di Progetto}). 
		\end{itemize}
Tutti i campi non segnalati sono da lasciare vuoti. 
Sar\`{a} poi compito del responsabile del progetto padre decidere a chi assegnare la correzione dell’errore. Nel caso in cui l’errore segnalato non sia considerato valido dal 
\emph{Responsabile del sotto-progetto} verr\`{a} confermato il rifiuto dal \emph{Responsabile di Progetto}. 

\end{enumerate}


\subsubsection{Dipendenze temporali}


Dopo la creazione del ticket, per aggiungere \textbf{dipendenze temporali} tra i ticket:
\begin{itemize}
\item Andare su \textbf{segnalazioni}; 
\item Aprire il link alla segnalazione a cui aggiungere la dipendenza; 
\item Nella sezione \textbf{segnalazioni correlate} premere \textbf{aggiungi}; 
\item Scegliere \textbf{segue} e indicare il numero della segnalazione che lo blocca ed eventuali giorni di slack. 

\end{itemize} 

Tutti i campi non segnalati sono da lasciare vuoti. 

\subsection{Aggiornamento ticket}

Esistendo due tipologie di ticket, viene qui definito la procedura per effettuare l’aggiornamento di entrambe.

\subsubsection{Ticket di pianificazione}

\begin{itemize}
\item Andare sul men\`{u} \textbf{Segnalazioni}; 
\item Selezionare il ticket di interesse; 
\item Cliccare il link \textbf{Aggiorna}; 
\item Commentare ci\`{o} che si \`{e} fatto sulla form \textbf{Note}; 
\item Cambiare lo stato del ticket secondo la seguente logica:
		\begin{itemize}
		\item \textbf{Do}: quando un ticket \`{e} in questo stato indica che una o pi\`{u} persone stanno 
		lavorando su tale attivit\`{a}; 
		\item \textbf{Check}: quando un ticket \`{e} in questo stato indica che una o pi\`{u} persone 
		stanno lavorando sulla verifica di tale attivit\`{a}; 
		\item \textbf{Act}: l’attivit\`{a} \`{e} stata conclusa e verificata, e ne sono state tratte le conclusioni adeguate. 
		
		\end{itemize} 

\item Se viene concluso, aggiornare lo stato del ticket di pianificazione padre. 

\end{itemize}


\subsubsection{Ticket di realizzazione e controllo}

\begin{itemize}
\item Andare sul men\`{u} \textbf{Segnalazioni}; 
\item Selezionare il ticket di interesse; 
\item Cliccare il link \textbf{Aggiorna}; 
\item Indicare il tempo impiegato in ore; 
\item Indicare il tipo di attivit\`{a} svolta; 
\item Commentare ci\`{o} che si \`{e} fatto sulla form \textbf{Note}; 
\item Cambiare lo stato del ticket secondo la seguente logica: 
		\begin{itemize}
		\item \textbf{In Progress}: quando un ticket \`{e} in questo stato indica che una o pi\`{u} persone 
		stanno lavorando su tale attivit\`{a}. La percentuale di completamento deve 
		essere impostata tra lo 0\% ed il 90\%; 
		\item \textbf{Closed}: l’attivit\`{a} \`{e} stata conclusa. La percentuale di completamento dell’attivit\`{a} \`{e} al 100\%. 
		 
		\end{itemize} 
\item Aggiornare lo stato del ticket di pianificazione padre secondo tali principi: 
		\begin{itemize}
		\item ticket figlio passa da New a In Progress: il ticket padre passa da Plan a Do, 
		o da Do a Check; 
		\item ticket figlio passa a Closed: il ticket padre deve essere in Do o Check; 
		\item tutti i ticket figli vengono chiusi: il ticket padre passa ad Act.
		\end{itemize}

\end{itemize}



\subsubsection{Ticket di verifica}

\begin{itemize}
\item Andare sul men\`{u} \textbf{Segnalazioni}; 
\item Selezionare il ticket di interesse; 
\item Cliccare il link \textbf{Aggiorna}; 
\item Indicare il tempo impiegato in ore; 
\item Indicare Verifica come tipo di attivit\`{a} svolta; 
\item Commentare le correzione nella form \textbf{Note}; 
\item Cambiare lo stato del ticket secondo la seguente logica:
		\begin{itemize}
		\item \textbf{In Progress}: quando un ticket \`{e} in questo stato indica che una o pi\`{u} persone 
		stanno lavorando su tale attivit\`{a}. La percentuale di completamento deve 
		essere impostata tra lo 0\% ed il 90\%; 
		\item \textbf{Closed}: l’attivit\`{a} \`{e} stata conclusa. La percentuale di completamento dell’attivit\`{a} \`{e} al 100\%; 
		\item \textbf{Rejected}: l’attivit\`{a} di verifica \`{e} stata rifiutata dal \emph{Responsabile del sottoprogetto} in accordo con il \emph{Responsabile di Progetto}. 
		
		\end{itemize}

\item Aggiornare lo stato del ticket di pianificazione padre secondo tali principi:
		\begin{itemize}
		\item ticket figlio passa da New a In Progress: il ticket padre passa da Plan a Do, 
		o da Do a Check; 
		\item ticket figlio passa a Closed: il ticket padre deve essere in Do o Check; 
		\item tutti i ticket figli vengono chiusi: il ticket padre passa ad Act. 
		
		\end{itemize} 

\end{itemize} 




\subsection{Consigli di utilizzo}
\phantomsection
\subsubsection{Pagina personale}
 
	Per avere una immediata visualizzazione dei ticket assegnati, \`{e} consigliato personalizzare 
	la pagina personale: 
	\begin{itemize}
		\item Andare alla \textbf{Pagina personale}; 
		\item Cliccare il link \textbf{Personalizza la pagina}; 
		\item Dal men\`{u} a tendina \textbf{La mia pagina di blocco}, selezionare \textbf{Le mie segnalazioni} 
		e premere il pulsante verde +; 
		\item Ripetere il punto precedente per aggiungere \textbf{Segnalazioni osservate}. 
	
	\end{itemize}

	
\subsubsection{Visualizzare segnalazioni}

	Per avere una visualizzazione pi\`{u} chiara delle segnalazioni si consiglia di ordinarle per 
	oggetto. Tale risultato pu\`{o} essere ottenuto premendo \textbf{Oggetto} dalla pagina \textbf{Segnalazioni}.


