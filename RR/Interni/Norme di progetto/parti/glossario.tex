\section{Glossario}{
 	Il glossario \`{e} unico per tutti i documenti e deve essere organizzato come definito nella sezione Documenti \ref{sec:docs}. Tutti i membri del gruppo possono modificarlo.\\
 	I termini all'interno del glossario saranno organizzati nel seguente modo:
 	\begin{itemize}
 	\item Tutti termini saranno in ordine alfanumerico;
 	\item Tutti i termini devono essere in grassetto e iniziare con la lettera maiuscola , la definizione del termine sarà preceduta dal carattere '':'' ;
 	\item Tutti i termini devono fornire chiarimenti su concetti che possono essere confusi quindi non devono essere inseriti termini il cui significato è già noto.
 	\end{itemize}	
   \subsection{Implementazione}
      L'inserimento dei termini nel glossario viene eseguito tramite un applicazione interna al team che funziona nel seguente modo:\\
      \begin{itemize}
      \item Si inserisce il lemma e la descrizione del lemma negli appositi spazi;
      \item Si salva il glossario nel formato .tex;
      \item A tutte le parole presenti nei documenti che hanno una corrispondente definizione nel glossario  verrà aggiunto un pedice |g|, per indicare che la parola è presente nel glossario.
      \end{itemize}       
   L'ordine lessicografico non è importante quando si inserisce un nuovo lemma nel programma dato che i lemmi vengono ordinati automaticamente.\\
   Il file relativo al Glossario \`{e} il seguente: \href{run:../../Esterni/Glossario/\Glossario}{\GlossarioEscape}
