\section{Riunioni}
	\subsection{Interne}{
		\begin{itemize}
			\item Ogni membro del gruppo pu\`{o} richiedere una riunione interna tramite un post all’interno del gruppo di Facebook (tramite l’uso del tag\ped{g} [Richiesta Riunione Interna $x$] con $x$ numero incrementato di 1 rispetto alla richiesta precedente). Questa richiesta  in base alle risposte degli altri componenti verr\`{a} presa in esame dal Responsabile;
			\item Una volta valutate le motivazioni della richiesta il Responsabile controlla sul calendario del gruppo le disponibilit\`{a} dei vari componenti;
			\item Il Responsabile entro 1 giorno lavorativo pubblica una nuova discussione con tag\ped{g} [Esito Richiesta Interna x], in cui, in caso positivo annuncia orario e luogo della riunione, in caso negativo annulla o rimanda la richiesta al successivo incontro;
			\item Nel caso in cui, per diversi motivi, alla riunione non potessero presenziare pi\`{u} di due membri, si procede a fissare una nuova riunione (vedi punto 2 e seguenti).
		\end{itemize}
		\subsubsection{Casi Particolari}{
			Per le richieste di riunioni interne vicine (cinque giorni lavorativi) ad una milestone\ped{g}, se approvate dal Responsabile, verranno indette il giorno stesso o il seguente.
		}
	}
	\subsection{Esterne}{
		Per le riunioni esterne (quindi gli incontri con il Proponente/Committente\ped{g}) la prassi \`{e} la medesima delle riunioni interne; pu\`{o} essere avanzata da qualsiasi membro del gruppo con il tag\ped{g} [Richiesta Riunione esterna $x$].
		In questo caso il Responsabile avr\`{a} il duplice compito di valutare la richiesta dopo aver consultato il calendario e di contattare  il committente\ped{g}, per accordarsi su tempi e luogo dell’incontro, che verranno poi riferiti sulla piattaforma di comunicazioni interne tramite il tag\ped{g} [Esito Richiesta Riunione Esterna $x$].
		}
	\subsection{Esito}{
		Ad ogni riunione (sia interna che esterna) il Responsabile ha il dovere di assicurarsi che venga redatto un verbale che riassuma gli argomenti trattati durante l’incontro e tutte le eventuali decisioni prese; i membri del gruppo hanno l’obbligo di applicare le eventuali modifiche o correzioni decise durante la riunione ed \`{e} del responsabile il dovere che i problemi emersi durante il verbale siano stati risolti.
		}
