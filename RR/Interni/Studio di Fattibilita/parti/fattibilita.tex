\section{C1 - Applicazione Cloud per il monitoraggio dei BigData nei Social Network}{
	\subsection{Elementi di valutazione}{
		Elementi a favore:
		\begin{itemize}
			\item Attenzione alla scalabilità;
			\item Tecnologie sconosciute dal gruppo ma ritenute professionalmente interessanti;
			\item Disponibilità del proponente ad introdurre dette tecnologie;
			\item Studio ed applicazione di tecniche di analisi dei dati e data mining ritenute utili professionalmente non conosciute dal gruppo;
			\item Studio e utilizzo delle API dei social network;
			\item Creazione di una API per accedere alle funzionalità rese disponibili dal sistema;
			\item Possibilità di distribuire il prodotto con licenza open source.
		\end{itemize}
		
		Elementi a sfavore:
		\begin{itemize}
			\item Le tecnologie nuove per il gruppo;
			\item Tecniche di data mining non conosciute dal gruppo.
		\end{itemize}
	}
	\subsection{Criticità}{
		Il maggior rischio individuato è la mancanza di conoscenza del gruppo in tecniche di data mining e di tecnologie cloud.
	}
}
\section{C2 - Gus Controllo qualità del vetro}{
	\subsection{Elementi di valutazione}{
		Elementi a favore:
		\begin{itemize}
			\item Studio e applicazione di algoritmi complessi in un ambito pratico e nuovo per il gruppo;
			\item Uso del linguaggio C++ con particolare attenzione alle prestazioni;
			\item Analisi dei requisiti verso una azienda non fornitrice di software;
			\item Possibilità di distribuire il prodotto con licenza open source.
		\end{itemize}
		
		Elementi a sfavore:
		\begin{itemize}
			\item Linguaggi e tecnologie da usare già conosciute dai membri del gruppo;
			\item Algoritmi e tecniche di analisi di immagini.
		\end{itemize}
	}
	\subsection{Criticità}{
		Il maggior rischio individuato per questo capitolato crediamo risieda inevitablimente nella fase di studio di algoritmi e tecniche per l'analisi di immagini, necessari per buona riuscita di questo Capitolato.
		Il ritiro da parte del Proponente di questo Capitolato ci costringe ad accantonare questa proposta.
	}
}
\section{C3 - Norris}{
	\subsection{Elementi di valutazione}{
		Elementi a favore:
		\begin{itemize}
			\item Progettazione di un framework;
			\item Tecnologie per il processamento real time dell'informazione;
			\item Possibilità di distribuire il prodotto con licenza open source.
		\end{itemize}
		
		Elementi a sfavore:
		\begin{itemize}
			\item Nuove tecnologie da imparare.
		\end{itemize}
	}
	\subsection{Criticità}{
		I maggiori rischi in questo capitolato risiedono nella progettazione di un framework e nel dover integrare al suo interno tecnologie che non abbiamo ancora utilizzato.
	}
}
\section{C4 - Premi}{
	\subsection{Elementi di valutazione}{
		Elementi a favore:
		\begin{itemize}
			\item Sviluppo di una web application;
			\item Fase di analisi in cui sono richieste idee creative;
			\item Studio e utilizzo di tecnologie web ritenute interessanti professionalmente;
			\item Libertà nella scelta delle tecnologie per lo sviluppo.
		\end{itemize}
		
		Elementi a sfavore:
		\begin{itemize}
			\item Difficoltà di verifica di un'applicazione web.
		\end{itemize}
	}
	\subsection{Criticità}{
		Pur attirati dal Capitolato, riconosciamo che il prodotto che s'andrà a sviluppare mira più ad esplorare i limiti delle tecnologie web.
	}
}
\section{C5 - sHike}{
	\subsection{Elementi di valutazione}{
		Elementi a favore:
		\begin{itemize}
			\item Sviluppo di un'applicazione per dispositivi indossabili (trend importante);
			\item Studio ed uso di tecnologie cloud (spendibile professionalmente);
			\item Particolare attenzione all'efficienza (fattore di spinta al miglioramento nelle fasi di progettazione e codifica).
		\end{itemize}
		
		Elementi a sfavore:
		\begin{itemize}
			\item Impossibilità di rendere il prodotto visibile in repository pubblici;
			\item Nuove tecnologie da utilizzate.
		\end{itemize}
	}
	\subsection{Criticità}{
		I maggiori rischi individuati per questo capitolato risiedono nello nelle tecnologie necessarie allo sviluppo del sistema e i vincoli che un dispositivo indossabile mette nello sviluppo di applicazioni. 
	}
}