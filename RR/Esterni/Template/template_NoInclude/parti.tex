%	Questo file è di esempio per le varie parti. Sono presenti alcuni metodi per scrivere il testo nel modo corretto con latex, dagli elenchi a tutto il resto.
%	Chiunque utilizzi comandi nuovi è pregato di riportarlo qui in modo che tutti ne conoscano l'effetto

%	\section è fondamentale da utilizzare per la creazione dell'indice. Section è la macro-parte, subsection tutte le altre. La numerazione di tutte le varie parti la fa Latex direttamente (parte 1, sottoparte 1.1,1.2 etc)
\section{Titolo del capitolo}{
	\subsection{sotto capitolo 1}{
		bla bla bla	 }
	\subsection{sotto capitolo 2}{ 
		bla bla bla	}
}

%	Formattazione testo e altro
\textbf{} % grassetto
\uppercase{} % maiuscolo
\emph{} % corsivo
\\ %per andare a capo

%grandezza del carattere dal più piccolo al più grande
\tiny{} 
\small{}
\normalsize{}
\large{}
\Large{}
\huge{}
\Huge{}

%Per richiamare il Glossario usare il pedice:
blablabla \ped{(g)}

%elenchi
%puntati
\begin{itemize}
	\item primo
	\item secondo
\end{itemize}
%numerati
\begin{enumerate}
	\item primo
	\item secondo
\end{enumerate}
%con descrizione: Tipo Primo	primo item
%Secondo 	secondo item
\begin{description}
	\item[Primo] \hfill  primo item %\hfill riempie di spazi la riga e primo item lo fa scrivere leggermente più sotto rispetto a Primo
	\item[Secondo] secondo item
\end{description}
%si possono innestare più elenchi uno dentro l'altro, per i puntati il tipo di punto viene cambiato da latex automaticamente

%RIFERIMENTI
%nei documenti può risultar essere necessario riferirsi ad altre parti. In questi casi bisogna creare un riferimento ipertestuale che ti rimanda a quella parte. Come fare? Facile:
\label{nome label} %bisogna mettere una label sulla parte che deve essere riferita
%per richiamare quella parte
\ref{nome label} %nel pdf verrà visualizzato il numero del capitolo
\pageref{nome label} %dà il numero di pagina dove si trova il riferimento

%riporto ciò che wiki dice in merito ai riferimenti:
%Since you can use exactly the same commands to reference almost anything, you might get a bit confused after you have introduced a lot of references. It is common practice among LaTeX users to add a few letters to the label to describe what you are referencing. Some packages, such as fancyref, rely on this meta information. Here is an example:
%ch:			chapter
%sec:		section
%subsec:	subsection
%fig:		figure
%tab:		table
%eq:			equation
%lst: 		code listing
%itm:		enumerated list item
%alg:		algorithm
%app:		appendix subsection

%Se il riferimento è a qualcosa che non è nello stesso documento ma in un altro, in questo caso allora bisogna scrivere:
\externaldocument{nomefile} %nomefile deve essere un .tex (non serve scriverlo dentro le graffe)
%poi si procede come prima, usando \ref{nome label}
%questa cosa ancora non l'ho provata

% TABELLE
%farle con latex può essere molto intorcolato... puoi farci tutti i tipi di tabelle che vuoi, anche quelle che ti fanno il caffè, ma vi rimando qui per approfondire: http://en.wikibooks.org/wiki/LaTeX/Tables

%allora: begin{table} è opzionale e si mette se la tabella deve avere una didascalia. Latex numera le tabelle in modo automatico.
%begin{tabular} crea fisicamente la tabella. toprule è la prima riga, midrule è per aggiungere righe che contengono i dati veri e propri e bottomrule è per chiudere la tabella.
\begin{table}[h]
	\begin{tabular}{p{0.2\textwidth} p{0.7\textwidth}}
		\toprule \textbf{Versione 1.0}	&	\textbf{Nominativo}\\
		\midrule Redazione	& \\
		\midrule Verifica &	\\
		\midrule Approvazione	&	\\
		\bottomrule
	\end{tabular}
	\caption{Storico ruoli pre-RR}
\end{table}

%un altro modo è questo:
\begin{tabular}{ l | c || r }
  \hline                       
  1 & 2 & 3 \\
  4 & 5 & 6 \\
  7 & 8 & 9 \\
  \hline  
\end{tabular}
%\\ va a capo e crea un'altra riga
%\hline crea una riga orizzontale
%{ l | c || r }: l sta per allineamento a sx, c in centro e r a dx.  Il numero di letterine che ci sono indica il numero di colonne quindi in qst caso 3. le prime due colonne sono divise da una riga verticale, la seconda e terza colonna invece da 2 colonne (||)

%Il tabular NON può essere usato per tabelle grandi che occupano più pagine. In quel caso allora si utilizza \begin{longtabu}. sul file registroMod.tex c'è un esempio