\subsection{caso d'uso (es: UC 2 - blabla)}{
	\label{uc1.2} %label se serve richiamarla: deve essere scritta uc e numero del caso d'uso
	\begin{figure}[H] %H: serve per mettere il testo sotto l'immagine
			\centering
			\includegraphics[scale=0.75]{\imgs {nomeimmagine}.jpg} %inserire il diagramma UML
	\end{figure}
	\textbf{Attori}: %scrivere gli attori
	\\ %questo è per andare a capo
	\textbf{Descrizione}: %breve descrizione
	\\
	\textbf{Precondizione}: %mettere la pre
	\\
	\textbf{Postcondizione}: %mettere la post
	\\
	%il seguente, se il caso d'uso è senza uml è inutile metterlo (sarebbe una ripetizione di quello che viene detto nella descrizione)
	\textbf{Procedura principale}: %descrivere il procedimento principale. E' opportuno usare un elenco numerato
		\begin{enumerate}
			\item ...
		\end{enumerate}
	
	%Se richiesto allora anche il seguito:
	\\
	\textbf{Scenario alternativo}: %se presente solo uno scenario alternativo
	%altrimenti
	\textbf{Scenari alternativi}: %se presenti più scenari
	\begin{itemize}
		\item %descrizione scenario
		\item %descrizione scenario etc
	\end{itemize}
	}
	