\section{Visione generale delle strategie di verifica}{
\subsection{Organizzazione}{
	Ogni qualvolta avvenga un cambiamento sostanziale nello sviluppo del prodotto, si istanzierà il processo di verifica. \\
	Nello specifico durante ogni fase (Analisi, Progettazione, Realizzazione e Validazione) saranno applicate le tecniche di verifica qui descritte nei seguenti casi:
	\begin{itemize}
		\item Conclusione della prima redazione di un documento;
		\item Conclusione della prima redazione di un file di codice;
		\item Conclusione della modifica sostanziale di un documento: quando il versionamento passa da .x.y.z a .x.y+1.0 oppure a .x+1.0.0. Si veda per approfondimento il paragrafo relativo al versionamento nel documento xxxxxx;
		\item Conclusione della modifica sostanziale di un file di codice, quando cioè il versionamento passa da .x.y.z a .x.y +1.0 oppure a .x+1.0.0. Si veda per approfondimento il paragrafo relativo al versionamento nel documento xxxxxx.
	\end{itemize}
	L'obiettivo delle attività di verifica è quello di trovare e rimuovere i problemi presenti. Un problema può verificarsi a vari livelli, e per ogni livello assume un nome diverso:
	\begin{itemize}
		\item Fault (difetto): è l'origine del problema, ciò che fa scaturire il malfunzionamento;
		\item Error (errore): è lo stato per cui il software si trova in un punto sbagliato del flusso di esecuzione o con valori sbagliati rispetto a quanto previsto dalla specifica;
		\item Failure (fallimento, guasto): è un comportamento difforme dalla specifica, cioè la manifestazione dell'errore all'utente del software.
	\end{itemize}
	Esiste una relazione di causa-effetto fra questi tre termini:\\
	\[DIFETTO\longrightarrow ERRORE\longrightarrow FALLIMENTO\]\\
	Non sempre un errore dà origine ad un fallimento: ad esempio potrebbero esserci alcune variabili che si trovano in stato erroneo ma non vengono lette, o non viene percorso il ramo di codice che le contiene.\\
	E' necessario prestare particolare attenzione a questo tipo di errori (detti anche quiescenti), avvalendosi anche di strumenti per il rilevamento dei bug.
}
\subsection{Pianificazione strategica e temporale}{
	Ai fini di rendere sistematica l'attività di verifica, per poter rispettare le scadenze fissate nel Piano di Progetto ed evitare la propagazione di errori all'interno dei documenti o di file di codice prima della loro verifica, la loro redazione sarà anticipata da una fase di studio preliminare. \\
	Questa fase permetterà di ridurre la necessità di intervenire con grossi interventi a posteriori, quando la correzione di imprecisioni concettuali e tecniche potrebbe risultare particolarmente gravosa. \\
	Come da Piano di Progetto di seguito si riportano le quattro milestone prefissate prima delle quali si effettuerà una verifica del prodotto:
	\begin{itemize}
		\item Revisioni formali:
		\begin{itemize}
			\item Revisione dei Requisiti (27/04/2015)
			\item Revisione di Accettazione (06/07/2015)
		\end{itemize}
		\item Revisioni di progresso:
		\begin{itemize}
			\item Revisione di Progettazione (29/05/2015)
			\item Revisione di Qualifica (18/07/2015)
		\end{itemize}
		Sarà necessario, infine, assicurarsi che ogni requisito sia tracciato consistentemente nel documento di Analisi dei Requisiti.
	\end{itemize}
	}
\subsection{Responsabilità}{
I principali ruoli di responsabilità individuati sono:
\begin{itemize}
	\item Amministratore di Progetto:
	\begin{itemize}
		\item Assicura la funzionalità dell'ambiente di lavoro;
		\item Redige i piani di gestione della qualità e ne verifica l'applicazione.
	\end{itemize}
	\item Responsabile del progetto:
	\begin{itemize}
		\item Assicura lo svolgimento delle attività di verifica;
		\item Assicura il rispetto dei ruoli e delle competenze come descritti nel Piano di Progetto;
		\item Approva e sancisce la distribuzione di un documento o di un file di codice;
		\item Assicura il rispetto delle scadenze.
	\end{itemize}
\end{itemize}
}
}
\section{Risorse}{
\subsection{Risorse necessarie:}{
	\subsubsection{Risorse umane}{
		I ruoli necessari a garantire la qualità del prodotto sono:
		\begin{itemize}
			\item Responsabile di Progetto;
			\item Amministratore;
			\item Verificatore;
			\item Programmatore. 
		\end{itemize}
	}
	\subsubsection{Risorse Hardware}{
		Saranno necessari:
		\begin{itemize}
			\item Computer con installato software necessario allo sviluppo del progetto in tutte le sue fasi;
			\item Luoghi in cui svolgere riunioni, preferibilmente dotato di connessione ad Internet.
		\end{itemize}
	}
	\subsubsection{Risorse software}{
		Saranno necessari:
		\begin{itemize}
			\item Strumenti per automatizzare i test;
			\item Framework per eseguire test di unità;
			\item Piattaforma di versionamento per la creazione e gestione di ticket;
			\item Debugger per i linguaggi di programmazione scelti;
			\item Browser come piattaforma di testing dell'applicazione da sviluppare;
			\item Strumenti per effettuare l'analisi statica del codice per misurare le metriche.
		\end{itemize}
	}
}
\subsection{Risorse disponibili}{
	Sono disponibili:
	\begin{itemize}
		\item Computer personali dei membri del gruppo;
		\item Computer presenti nelle aule informatiche del Dipartimento di Matematica;
		\item Aule disponibili per incontri nel Dipartimento di Matematica;
		\item Un dispositivo Raspberry Pi 2 Model B, utilizzato come server per programmi organizzativi e di testing.
	\end{itemize}
	\subsubsection{Risorse software}{
		Si rimanda alla sezione Strumenti \ref{sec:strumenti}.
	}
}
}