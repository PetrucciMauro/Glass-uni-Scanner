\section{Verifica dei Requisiti}{
\subsection{Verifica dei Requisiti Funzionali}{
		\renewcommand*{\arraystretch}{1.4}
		\begin{longtable} [c]{| p{3cm} | p{6cm} |p{3cm}|}
			\caption{Verifica dei Requisiti Funzionali \label{tab:verReqFunz}}\\
			 \hline
			 \textbf{Test} & \textbf{Descrizione} & \textbf{Requisito} \\
			 \hline
			 \endfirsthead
			 \hline
			 \textbf{Test} & \textbf{Descrizione} & \textbf{Requisito} \\
			 \hline
			\endhead
			 \hline
			 \endfoot
			 \hline
			 \endlastfoot
			TS 1 & Viene verificato che ci si possa registrare al sistema inserendo username e password & RF 1\\
			\hline
			TS 3 & Viene verificato che ci si possa autenticare con username e password & RF 3\\
			\hline
			TS 4 & Viene verificato che si possa creare una nuova presentazione vuota & RF 4\\
			\hline
			TS 7 & Viene verificato che si possa passare in modalità modifica di una presentazione da desktop & RF 7\\
			\hline
			TS 7.1.1 & Viene verificato che si possa inserire un nuovo frame nel piano della presentazione & RF 7.1.1\\
			\hline
			TS 7.1.1.1 & Viene verificato che si possa scegliere il tipo di frame da inserire & RF 7.1.1.1\\
			\hline
			TS 7.1.4 & Viene verificato che si possa spostare un frame in modalità modifica & RF 7.1.4\\
			\hline
			TS 7.1.7 & Viene verificato che si possa passare in modalità modifica di un frame & RF 7.1.7\\
			\hline
			TS 7.1.7.1 & Viene verificato che su possa inserire del testo all'interno di un frame & RF 7.1.7.1\\
			\hline
			TS 7.1.7.4 & Viene verificato che si possa modificare del testo già presente all'interno di un frame & RF 7.1.7.4\\
			\hline
			TS 7.1.7.7 & Viene verificato che si possa inserire un immagine all'interno del frame & RF 7.1.7.7\\
			\hline
			TS 7.1.7.10 & Viene verificato che si possa modificare la dimensione di un immagine & RF 7.1.7.10\\
			\hline
			TS 7.1.7.13 & Viene verificato che si possa inserire un video all'interno di un frame & RF 7.1.7.13\\
			\hline
			TS 7.1.7.16 & Viene verificato che si possa modificare la dimensione di un video all'interno di un frame & RF 7.1.7.16\\
			\hline
			TS 7.1.7.19 & Viene verificato che si possa spostare un elemento all'interno del frame & RF 7.1.7.19\\
			\hline
			TS 7.1.7.22 & Viene verificato che si possa eliminare un elemento presente all'interno del frame & RF 7.1.7.22 \\
			\hline
			TS 7.1.7.25 & Verificare che si possa modificare il testo di un elemento scelta & RF 7.1.7.25\\
			\hline
			TS 7.1.7.28 & Viene verificato che si possa modificare la dimensione di un frame & RF 7.1.7.28\\
			\hline
			TS 7.1.7.31 & Viene verificato che si possa modificare la forma di un frame & RF 7.1.7.31\\
			\hline
			TS 7.1.7.34 & Viene verificato che si possa modificare lo spessore del bordo di un frame & RF 7.1.7.34\\
			\hline
			TS 7.1.7.37 & Viene verificato che si possa modificare il colore del bordo di un frame & RF 7.1.7.37\\
			\hline
			TS 7.1.7.40 & Viene verificato che si possa modificare lo sfondo di un frame & RF 7.1.7.40, 7.1.7.43\\
			\hline
			TS 7.1.10 & Viene verificato che si possa eliminare un frame dal piano di una presentazione & RF 7.1.10\\
			\hline
			TS 7.1.13 & Viene verificato che si possa inserire un'immagine di sfondo in un'area della presentazione & RF 7.1.13 \\
			\hline
			TS 7.1.16 & Viene verificato che si possa inserire un colore di sfondo in un area della presentazione & RF 7.1.16\\
			\hline
			TS 7.1.19 & Viene verificato che si possa definire un percorso di visualizzazione & RF 7.1.19\\
			\hline				
			TS7.1.19.1 & Viene verificato che si possa impostare un frame iniziale per il percorso di presentazione & RF7.1.19.1\\
			TS7.1.19.4 & Viene verificato che si possa definire una transizione tra due frame & RF7.1.19.4\\
			\hline
			TS7.1.19.7 & Viene veificato che si possa definire una transizione scelta tra due frame & RF7.1.19.7\\
			\hline
			TS7.1.19.10 & Viene verificato che si possa eliminare una transizione tra due frame & RF7.1.19.10\\
			\hline
			TS7.1.19.13 & Viene verificato che si possa togliere un frame dal percorso di presentazione & RF7.1.19.13\\
			\hline
			TS7.1.22 & Viene verificato che si possa assegnare un bookmark ad un frame & RF7.1.22\\
			\hline
			TS7.1.25 & Viene verificato che si possa rimuovere un bookmark da un frame & RF7.1.25\\
			\hline
			TS7.1.28 & Viene verificato che si possa modificare la velocità di transizione tra due frame consecutivi & RF7.1.28\\
			\hline
			TS7.1.31 & Viene verificato che si possa impostare un effetto di transizione tra due frame consecutivi & RF7.1.31\\
			\hline
			TS7.1.34 & Viene verificato che si possa impostare il tempo di attesa tra due frame consecutivi durante la "riproduzione automatica" & RF7.1.34\\
			\hline
			TS7.1.37 & Viene verificato che si possa annullare e ripristinare una modifica appena effettuata & RF7.1.37, 7.1.40\\
			\hline			 
			TS 10 & Viene verificato passare in modalità modifica di una presentazione da mobile & RF 10\\
			\hline
			TS 10.1 & Viene verificato che si possa editare testo da mobile all'interno di un frame & RF 10.1\\
			\hline
			TS 10.4 & Viene verificato che si possa modificare da mobile il testo presente all'interno di un frame & RF 10.4\\
			\hline
			TS 10.3 & Viene verificato che si possa annullare da mobile una modifica appena effettuata & RF 10.3\\
			\hline
			TS 10.5 & Viene verificato che si possa asseggnare un bookmark ad un frame da mobile & RF 10.5\\
			\hline
			TS 10.8 & Viene verificato che si possa rimuover un bookmark ad un frame da mobile & RF 10.8\\
			\hline
			TS 13 & Viene verificato che si possa caricare un immagine dal proprio File System alla propria parte dedicata alle immagini & RF 13\\
			\hline
			TS 16 & Viene verificato che si possano eliminare dal server le immagini caricate & RF 16\\
			\hline
			TS 19 & Viene verificato che si possa creare nuove cartelle e spostare i file all'interno delle cartelle & RF 19\\
			\hline
			TS 25 & Viene verificato che si possano organizzare le proprie presentazione con una struttura a cartelle & RF 25\\
			\hline
			TS 31 & Viene verificato che si possano spostare le proprie infografiche all'interno della cartella dedicata sul server & RF 31\\
			\hline
			TS 34 & Viene verificato che si possa eliminare dal server una presentazione creata & RF 34\\
			\hline
			TS 37 & Viene verificato che si possa eliminare dal server un'infografica creata & RF 37\\
			\hline
			TS 43 & Viene verificato che si possa modificare la propria password di accesso al sistema & RF 43\\
			\hline
			TS 46 & Viene verificato che si possa scaricare in locale un'infografica creata sul server & RF 46\\
			\hline
			TS 49 & Viene verificato che si possa salvare in locale una presentazione creata sul server & RF 49\\
			\hline
			TS 52 & Viene verificato che si possa rimuover una presentazione salvata in locale & RF 52\\
			\hline
			TS 55 & Viene verificato che si possa eseguire una presentazione salvata sul server & RF 55\\
			\hline
			TS 58 & L'utente deve essere in grado di eseguire una presentazione salvata in locale & RF 58\\
			\hline
			TS 61.1 & Viene verificato che si possa eseguire una presentazione in modalità manuale & RF 61.1\\
			\hline
			TS 61.1.1 & Viene verificato che durante la presentazione si possa passare al frame successivo o al precedente & RF 61.1.1\\
			\hline
			TS 61.1.4 & Viene verificato che si possa selezionare un elemento scelta se presente nel frame & RF 61.1.4\\
			\hline
			TS 61.1.7 & Viene verificato che si possa passare al frame con bookmark successivo o precedente & RF 61.1.7\\
			\hline
			TS 61.1.10 & Viene verificato che si possa passare da un frame visualizzato al suo frame contenitore & RF 61.1.10\\
			\hline
			TS 61.1.13 & Viene verificato che si possa eseguire dello zoom in una parte qualsiasi del frame & RF 61.1.13\\
			\hline
			TS 61.1.16.1 & Viene verificato che si possa far partire l'esecuzione di un video all'interno di un frame & RF 61.1.16.1\\
			\hline
			TS 61.1.16.4 & Viene verificato che si possa sospendere e poi riprendere l'esecuzione di un video all'interno di un frame & RF 61.1.16.4\\
			\hline
			TS 61.1.16.7 & Viene verificato che si possa eseguire un video da un punto qualsiasi dello stesso & RF 61.1.16.7\\
			\hline
			TS 61.1.16.10 & Viene verificato che si possa interrompere l'esecuzione di un video & RF 61.1.16.10\\
			\hline
			TS 61.4 & Viene verificato che si possa eseguire una presentazione in modalità automatica & RF 61.4\\
			\hline
			TS 61.4.1 & Viene verificato che si possa chiudere una presentazione in esecuzione automatica & RF 61.4.1\\
			\hline
			TS 61.4.4 & Viene verificato che si possa sospendere e riavviare una presentazione in esecuzione automatica & RF 61.4.4\\
			\hline
			TS 61.4.7 & Viene verificato che si possa impostare la velocità di riproduzione della presentazione & RF 61.4.7\\
			\hline
			TS 61.4.10 & Viene verificato che si possa saltare la riproduzione di un video nel frame visualizzato & RF 61.4.10\\
			\hline
			TS 61.7 & Viene verificato che si possa passare da presentazione automatica a presentazione manuale e viceversa & RF 61.7, 61.10\\
			\hline
			TS 64 & Viene verificato che si possa effettuare il logout dal server & RF 64\\
			\hline
			TS 67.1 & Viene verificato che l'amministratore possa inserire dei template di presentazioni & RF 67.1\\
			\hline
			TS 67.4 & Viene verificato che l'amministratore possa inserire template di infografiche & RF 67.4\\
			\hline
			TS 67.7 & Viene verificato che l'amministratore possa inserire elementi grafici & RF 67.7\\
			\hline
			TS 67.10 & Viene verificato che l'amministratore possa eliminare un template & RF 67.10\\
			\hline
			TS 67.13 & Viene verificato che l'amministratore possa annullare l'ultima eliminazione di un template & RF 67.13\\
			\hline
			TS 70.1 & Viene verificato che si possa selezionare una presentazione da cui produrre l'infografica  & RF 70.1\\
			\hline
			TS 70.4 & Viene verificato che si possa selezionare template di infografica & RF 70.4\\
			\hline
			TS 70.7 & Viene verificato che si possano selezionare gli elementi del  & RF 70.7\\
			\hline
			TS 70.10 & Viene verificato che si possa passare in modalità modifica di un'infografica & RF 70.10\\
			\hline
			TS 70.10.1 & Viene verificato che si possa modificare un elemento di un infografica & RF 70.10.1\\
			\hline
			TS 70.10.1.1 & Viene verificato che si possano modificare le dimensioni di un immagine & RF 70.10.1.1\\
			\hline
			TS 70.10.1.4 & Viene verificato che si possa modificare un elemento testo  & RF 70.10.1.4\\
			\hline
			TS 70.10.1.4.1 & Viene verificato che si possa modificare il font del testo & RF 70.10.1.4.1\\
			\hline
			TS 70.10.1.4.4 & Viene verificato che si possa modificare la dimensione del carattere & RF 70.10.1.4.4\\
			\hline
			TS 70.10.1.4.7 & Viene verificato che si possa modificare lo stile del testo & RF 70.10.1.4.7\\
			\hline
			TS 70.10.1.4.10 & Viene verificato che si possa modificare il colore della scritta del testo & RF 70.10.1.4.10\\
			\hline
			TS 70.10.1.4.13 & Viene verificato che si possa modicare il colore dell'evidenziazione del testo & RF 70.10.1.4.13\\
			\hline
			TS 70.10.1.7 & Viene verificato che si possa cambiare la posizione dell'elemento & RF 70.10.1.7\\
			\hline
			TS 70.10.4 & Viene verificato che si possa rimuover lo sfondo dell'infografica & RF 70.4.4\\
			\hline
			TS 70.10.7 & Viene verificato che si possa inserire uno sfondo nell'infografica & RF 70.10.7\\
			\hline
			TS 70.10.10 & Viene verificato che si possa inserire un elemento immagine nell'infografica & RF 70.10.10\\
			\hline
			TS 70.10.13 & Viene verificato che si possa inserire del testo nell'infografica & RF 70.10.13\\
			\hline
			TS 70.10.16 & Viene verificato che si possa inserire un frame nella sua interezza presente nella presentazione & RF 70.10.16\\
			\hline
			TS 70.10.19 & Viene verificato che si possano eliminare elementi immagini o testuali & RF 70.10.19\\
			\hline
			TS 70.13 & Viene verificato che si possa salvare l'infografica nel suo spazio & RF 70.13\\
			\hline
			TS 70.16 & Viene verificato che si possa annullare e ripristinare una modifica appena effettuata & RF 70.16\\
			\hline
			TS 70.19 & Viene verificato che si possa esportare un'infografica in formato stampabile & RF 70.19\\
			\hline
			TS 73 & Viene verificato che si possa creare un'infografica & RF 73\\
			\hline
\end{longtable}
}
\newpage
\subsection{Verifica dei Requisiti di Qualità e Vincoli}{
		\renewcommand*{\arraystretch}{1.4}
		\begin{longtable} [c]{| p{7cm} |p{4cm}|}
			\caption{Verifica dei Requisiti di Qualità e Vincoli \label{tab:verReqQualVinc}}\\
			 \hline
			 \textbf{Descrizione} & \textbf{Requisito} \\
			 \hline
			 \endfirsthead
			 \hline
			 \textbf{Descrizione} & \textbf{Requisito} \\
			 \hline
			\endhead
			 \hline
			 \endfoot
			 \hline
			 \endlastfoot
			Viene verificato che ogni funzionalità dell'applicazione sia documentato & RQ 1, 7\\
			\hline
			Viene verificato che sia disponibile un tutorial interattivo per la creazione delle presentazioni & RQ4\\
			\hline
			Viene verificato che sia disponibile della documentazione sui test eseguiti & RQ10\\
			\hline
			Viene verificato che il sistema dovrà offrire la possibilità di eseguire offline le presentazioni & RQ13\\
			\hline
			Viene verificato che il sistema sia funzionante su dispositivi Desktop e Mobile (Android, Ios, Windows Phone) & RQ16\\
			\hline
			Viene verificato che il sistema segua le linea guida del material design fornite dalla Google & RV 1\\
\end{longtable}
}
}
