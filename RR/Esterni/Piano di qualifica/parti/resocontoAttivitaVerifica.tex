\section{Resoconto delle attività di verifica}
\subsection{Riassunto delle attività di verifica}{
\subsubsection{Revisione dei Requisiti}{

Durante questa fase sono stati prodotti solamente documenti di testo quindi sono state applicate le tecniche di analisi statica effettuando \textit{walkthrough} e rispettando i metodi definiti nella sezione \ref{sec:Metodi}.\\
Nella verifica dei documenti sono stati riscontrati soprattutto errori grammaticali e di battitura dovuti a disattenzioni durante la stesura.\\
È stato trovato anche qualche errore più grave, come il mancato rispetto delle regole di formattazione riportate nelle \NormeDiProgetto e alcune mancanze all'interno del documento di \AnalisiDeiRequisiti.\\
}
\subsubsection{Documenti}{
Vengono qui riportati i valori dell’indice Gulpease per ogni documento durante la fase di \textbf{Analisi}. Un documento è considerato valido soltanto se rispetta le metriche descritte su \ref{sec:metricadocumenti}.

\begin{table}[H]
	\centering
	\begin{tabular}{p{\dimexpr 0.4\linewidth-2\tabcolsep}p{\dimexpr 0.2\linewidth-2\tabcolsep}
			p{\dimexpr 0.2\linewidth-2\tabcolsep}}
		\toprule Documento & Valore indice & Esito \\
		\midrule
		\PianoDiProgetto & 89 & \textcolor{green} Superato \\
		\AnalisiDeiRequisiti & 91 & \textcolor{green} Superato \\
		\NormeDiProgetto & 75 & \textcolor{green} Superato \\
		\PianoDiQualifica & 82 & \textcolor{green} Superato \\
		\StudioDiFattibilita & 82 & \textcolor{green} Superato \\
		\Glossario & 97 & \textcolor{green} Superato \\
		\bottomrule
	\end{tabular}
	\label{tab:costorequisiti}
	\caption{Esiti verifica documenti, Analisi}
\end{table}

Come si può notare dalla tabella, tutti gli indici Gulpease dei documenti rientrano nel range ottimale precedentemente definito e quindi i documenti redatti hanno raggiunto la leggibilità desiderata.
}
}