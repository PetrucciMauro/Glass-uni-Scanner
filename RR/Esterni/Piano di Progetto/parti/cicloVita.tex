\section{Ciclo di vita}{
	Si è scelto di applicare ai processi il modello incrementale per i seguenti motivi:
	\begin{itemize}
		\item Ottenere il prima possibile un sistema funzionante sulle parti critiche del sistema;
		\item Testare maggiormente le parti critiche del sistema grazie anche all'integrazione successiva delle parti desiderabili o opzionali;
		\item A causa dell'inesperienza del gruppo nella previsione dei tempi di sviluppo, in questo modo si limita il rischio di aver sottostimato i tempi riguardanti la progettazione in dettaglio e codifica dei requisiti obbligatori poiché saranno trattati nella prima iterazione.
	\end{itemize}
	L'adozione di questo modello permette di rilasciare al committente una base di prodotto con l'insieme delle funzionalità fondamentali il prima possibile, così da permettere al committente di valutare in corso d'opera il lavoro svolto. 
	Questo modello permette in caso di sottostima dei tempi di realizzazione di avere comunque un prodotto con le funzionalità di base richieste poiché queste saranno trattate nella prima iterazione.
	Si avrà quindi il vantaggio di spendere inizialmente le risorse nella realizzazione di una base di prodotto funzionante che presenti gli aspetti del sistema di maggiore importanza. 
	Si potrà in seguito migliorare tale base ed utilizzarla per ampliare il prodotto con le funzionalità opzionali e desiderabili.
}


