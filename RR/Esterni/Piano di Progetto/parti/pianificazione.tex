
\section{Pianificazione}{
	
	Date le scadenze in (\ref{sec:scadenze}) si \`{è} diviso il progetto in 4 stati di sviluppo:
	\begin{itemize}
		\item \textbf{Analisi dei Requisiti} (AN)
		\item \textbf{Progettazione Architetturale}(PA)
		\item \textbf{Progettazione in dettaglio e Codifica}(PDC)
		\item \textbf{Verifica e Validazione}(VV)
	\end{itemize}
	
	Per ogni stato del progetto sono state individuate attivit\`{a} e sotto-attivit\`{a} a cui sono state associate le risorse del gruppo.
	Per ogni stato del processo \`{e} stato riportato il Gantt(g) con evidenziato in colore rosso il cammino critico.
	Le attivit\`{a} parte di questo cammino saranno monitorate con maggiore attenzione in quanto un ritardo su queste attivit\`{a} 
	sarebbe dannoso per l'efficienza del gruppo e porterebbe a ritardi nell'avanzamento dello stato del progetto.
	Si \`{e} deciso di non riportare i grafici PERT(g) poich\'{e} le attivit\`{a} critiche sono gi\`{a} state evidenziate nel GANT(g) come detto in precedenza.
	Per ogni stato del progetto \`{e} stato riportato il diagramma WBS(g) cos\`{i} da rendere esplicita la composizione delle attivit\`{a} e rendere immediata la costruzione del prospetto economico.
	
	\subsection{Analisi dei Requisiti}{
	\textbf{Periodo}: da X a 3-04-2015 \\
	
	I documenti redatti in questo periodo sono:
	\begin{itemize}
		\item \textbf{Norme di Progetto}: il Responsabile e l'Amministratore scrivono il documento "Norme di Progetto" che norma le attivit\`{a} da svolgersi nel corso del progetto.
		\item \textbf{Studio di Fattibilit\`{a}}: vengono valutati ...(?)... secondo quanto previsto nelle "Norme di Progetto" sezione "X".
		\item \textbf{Analisi dei Requisiti}: a partire dallo studio di fattibilit\`{a} vengono analizzati in profondit\`{a} i requisiti per il capitolato scelto con i metodi esplicitati in "Norme di Progetto" sezione "Y".
		\item \textbf{Piano di progetto}: il responsabile di progetto durante questa fase si prende carico di definire un piano di progetto in cui vengono definite le macro-attivit\`{a} da svolgersi durante questo e i successivi stati di sviluppo del sistema. Successivamente alle attivit\`{a} saranno associate risorse cos\`{i} da poter redigere un prospetto economico per il proponente.
		\item \textbf{Piano di Qualifica}: Analista, Verificatore e Responsabile redigono il piano di qualifica.
		\item \textbf{Glossario}: scritto dai redattori dei documenti in modo incrementale.
		\item \textbf{Lettera di Presentazione}: Lettera di Presentazione del gruppo da consegnare al committente per partecipare alla gara d'appalto.
	\end{itemize}
	
	\begin{figure}[H]
		\includegraphics[scale=0.3]{\imgs pianoRequisiti.jpg}
		\label{fig:pianorequisiti}
		\caption{Piano dei Requisiti}
	\end{figure}
	
	%TABELLA RUOLI 
	\begin{longtable} [c]{| l | l | l | l |}
		 \hline
		 \textbf{Macro-Attivit\`{a}} & \textbf{Attivit\`{a}} & \textbf{Ruolo} & \textbf{Ore}\\
		 \hline
		 \endfirsthead
		 \hline
		 \textbf{Macro-Attivit\`{a}} & \textbf{Attivit\`{a}} & \textbf{Ruolo} & \textbf{Ore}\\
		 \hline
		\endhead
		 \hline
		 \endfoot
		 \hline
		 \caption{Attivit\`{a} e ruoli Piano dei Requisiti}
		 \endlastfoot
		Norme di Progetto & Comunicazioni & Responsabile 1 & 1\\
		Norme di Progetto & Requisiti & Responsabile 2 & 4\\
		Norme di Progetto & Progettazione & Responsabile 3 & 3\\
		Norme di Progetto & Codifica & Responsabile 4 & 3\\
		Norme di Progetto & Documentazione & Responsabile 3 & 5\\
		Norme di Progetto & Strumenti	&	Responsabile 5	&	4\\
		&	&	Amministratore1	&	12\\
		&	&	Amministratore 2	&	2\\
		Norme di Progetto & Norme ticketing &	Responsabile 1	&	1\\
		&	&	Amministratore 3 & 1\\
		Norme di Progetto & Repository &	Responsabile 4	&	1\\
		&	&	Amministratore 1	&	1\\
		Norme di Progetto & Verifica & Verificatore 3 & 2\\
		&	&	Verificatore 3 & 2\\
		&	&	Responsabile 3 & 1\\
		Analisi dei Requisiti & Descrizione & Analista1 & 6\\
		Analisi dei Requisiti & Casi d'uso & Analista 1 & 9\\
		&	&	Analista 2 & 8\\
		&	&	Analista 3 & 9\\
		&	&	Analista 4 & 8\\
		&	&	Analista 5 & 8\\
		&	&	Analista 6 & 8\\
		Analisi dei Requisiti & Specifica & Analista 1 & 7\\
		&	&	Analista 3 & 7\\
		&	&	Analista 4 & 6\\
		Analisi dei Requisiti & Verifica & Verificatore 1 & 2\\
		&	&	Verificatore 2 & 3\\
		&	&	Responsabile 3 & 1\\
		Piano di Progetto & Calendario & Responsabile 2 & 2\\
		Piano di Progetto & Organigramma & Responsabile 6 & 1\\
		Piano di Progetto & Ciclo di vita & Responsabile 2 & 2\\
		Piano di Progetto & Analisi dei rischi & Responsabile 2 & 2\\
		Piano di Progetto & Individuazione attivit\`{a} & Responsabile 2 & 6\\
		Piano di Progetto & Assegnamento risorse & Responsabile 5 & 4\\
		&	&	Responsabile 2 & 2\\
		Piano di Progetto & Prospetto economico & Responsabile5 & 2\\
		Piano di Progetto & Verifica &  Verificatore 5 & 3\\
		&	&	Responsabile 1 & 1\\
		Piano di Qualifica & Visione generale & Verificatore 3 & 5\\
		&	&	Responsabile 4 & 3\\
		&	&	Responsabile 6 & 3\\
		Piano di Qualifica & Gestione amministrativa della revisione & & 2\\
		Piano di Qualifica & Resoconto verifiche & Verificatore 2 & 2\\
		Piano di Qualifica & Pianificazione validazione requisiti & Verificatore 5 & 10\\
		Piano di Qualifica & Verifica & Verificatore 4 & 2\\
		&	&	Responsabile 6 & 1\\
		Glossario & Stesura & / & 10\\
		Glossario & Verifica & Verificatore & 1\\
	\end{longtable}
}
\subsection{Progettazione}{
	\textbf{Periodo}: dal 7-04-2015 al 30-04-2015 \\
	
	Le attivit\`{a} da svolgere in questo periodo saranno;
	\begin{itemize}
		\item incremento e verifica dei documenti portati in Revisione dei Requisiti;
		\item stesura del documento \textbf{Specifica Tecnica} in cui il Progettista esporr\`{a} le scelte progettuali di alto livello del sistema e i design pattern che saranno utilizzati nel sistema. Si verificher\`{a} il tracciamento dal sistema ai requisiti e dai requisiti al sistema.
	\end{itemize}

	\begin{figure}[H]
		\includegraphics[scale=0.75]{\imgs pianoProgettazione.jpg}
		\label{fig:pianoprogettazione}
		\caption{Piano di Progettazione}
	\end{figure}
	
	GRAFICO WBS
	
	\begin{table}[H] %TABELLA RUOLI
	\begin{tabular}{llll}
	Macro-Attivit\`{i} & Attivit\`{i} & Ruolo & Ore \\
	Norme di Progetto & Incremento & \begin{tabular}[c]{@{}l@{}}Responsabile2\\ Amministratore\end{tabular} & \begin{tabular}[c]{@{}l@{}}6\\ 10\end{tabular} \\
	Norme di Progetto & Verifica & Verificatore2 & 2 \\
	Analisi dei Requisiti & Incremento & Analista & 6 \\
	Analisi dei Requisiti & Verifica & Verificatore1 & 1 \\
	Specifica Tecnica & Strumenti e tecnologie & Progettista5 & 6 \\
	Specifica Tecnica & Notazione & Progettista2 & 2 \\
	Specifica Tecnica & Architettura & \begin{tabular}[c]{@{}l@{}}Progettista2\\ Progettista3\\ Progettista4\end{tabular} & \begin{tabular}[c]{@{}l@{}}6\\ 6\\ 8\end{tabular} \\
	Specifica Tecnica & Design Patterns & \begin{tabular}[c]{@{}l@{}}Progettista1\\ Progettista2\\ Progettista3\end{tabular} & \begin{tabular}[c]{@{}l@{}}8\\ 8\\ 8\end{tabular} \\
	Specifica Tecnica & Componenti & \begin{tabular}[c]{@{}l@{}}Progettista2\\ Progettista3\\ Progettista4\end{tabular} & \begin{tabular}[c]{@{}l@{}}8\\ 8\\ 8\end{tabular} \\
	Specifica Tecnica & Tracciamento & \begin{tabular}[c]{@{}l@{}}Verificatore1\\ Verificatore4\end{tabular} & \begin{tabular}[c]{@{}l@{}}5\\ 5\end{tabular} \\
	Specifica Tecnica & Verifica & Verificatore3Responsabile2 & 51 \\
	Piano di Qualifica & Incremento & Responsabile1 & 4 \\
	Piano di Qualifica & Pianificazione Test & \begin{tabular}[c]{@{}l@{}}Progettista5\\ Verificatore1\\ Verificatore2\\ Responsabile1\\ Amministratore\end{tabular} & \begin{tabular}[c]{@{}l@{}}6\\ 5\\ 6\\ 4\\ 6\end{tabular} \\
	Piano di Qualifica & Esito verifiche & Verificatore3 & 4 \\
	Piano di Qualifica & Verifica & \begin{tabular}[c]{@{}l@{}}Verificatore3\\ Responsabile1\end{tabular} & \begin{tabular}[c]{@{}l@{}}5\\ 1\end{tabular} \\
	Glossario & Incremento & / & 6 \\
	Glossario & Verifica & Verificatore4 & 2 \\
	Piano di Progetto & Consuntivo & Responsabile2 & 2 
	\end{tabular}
	\end{table}
}
	
\subsection{Progettazione in dettaglio e Codifica}{
	\textbf{Periodo}: dal 4-05-2015 a 13-06-2015 \\
	 
	 Le attivit\`{a} della parte \textbf{Progettazione in dettaglio e Codifica} sono:
	 \begin{itemize}
		 \item \textbf{Definizione di Prodotto}: che contiene la descrizione approfondita delle componenti del prodotto.
		 \item \textbf{Codifica}: sviluppo del codice del prodotto da parte dei programmatori che devono seguire quanto riportato nel documento Definizione di Prodotto.
		 \item \textbf{Esecuzione test automatici}: esecuzione automatica dei test di unit\`{a} e integrazione e rapporto sul risultato.
		 \item \textbf{Incremento Specifica Tecnica}: incremento del documento di specifica tecnica con la progettazione riguardante i requisiti non obbligatori.
		 \item \textbf{Manualistica}: verranno reddatti i documenti sul prodotto per l'utente finale.
	 \end{itemize}
	
	% IMMAGINE "perLatex/pianoCodifica.png" (da inserire)
	
	GRAFICO WBS
	
	\begin{table}[H] %TABELLA RUOLI
	\begin{tabular}{llll}
	Macro-Attivit\`{i} & Attivit\`{i} & Ruoli & Ore \\
	Norme di Progetto & Incremento & \begin{tabular}[c]{@{}l@{}}Responsabile\\ Amministratore\end{tabular} & \begin{tabular}[c]{@{}l@{}}8\\ 2\end{tabular} \\
	Norme di Progetto & Verifica & Verificatore4 & 1 \\
	Piano di Progetto & Rivalutazione & Responsabile & 2 \\
	Piano di Progetto & Verifica & Verificatore1 & 1 \\
	Specifica Tecnica & Incremento Design Patterns & \begin{tabular}[c]{@{}l@{}}Progettista1\\ Progettista2\end{tabular} & \begin{tabular}[c]{@{}l@{}}4\\ 4\end{tabular} \\
	Specifica Tecnica & Incremento Componenti & \begin{tabular}[c]{@{}l@{}}Progettista1\\ Progettista2\end{tabular} & \begin{tabular}[c]{@{}l@{}}8\\ 8\end{tabular} \\
	Specifica Tecnica & Verifica & Verificatore4 & 4 \\
	Definizione Prodotto & Specifica Componenti Base & \begin{tabular}[c]{@{}l@{}}Progettista3\\ Progettista4\\ Progettista5\\ Progettista6\end{tabular} & \begin{tabular}[c]{@{}l@{}}10\\ 10\\ 10\\ 10\end{tabular} \\
	Definizione Prodotto & Specifica - Verifica Base & \begin{tabular}[c]{@{}l@{}}Verificatore1\\ Verificatore2\end{tabular} & \begin{tabular}[c]{@{}l@{}}6\\ 2\end{tabular} \\
	Definizione Prodotto & Specifica Componenti Incremento & \begin{tabular}[c]{@{}l@{}}Progettista1\\ Progettista2\end{tabular} & \begin{tabular}[c]{@{}l@{}}10\\ 10\end{tabular} \\
	Definizione Prodotto & Specifica - Verifica Incremento & Verificatore2 & 4 \\
	Definizione Prodotto & Specifica - Tracciamento Base & \begin{tabular}[c]{@{}l@{}}Progettista3\\ Verificatore1\end{tabular} & \begin{tabular}[c]{@{}l@{}}4\\ 6\end{tabular} \\
	Definizione Prodotto & Specifica - Tracciamento Incremento & \begin{tabular}[c]{@{}l@{}}Progettista1\\ Verificatore3\end{tabular} & \begin{tabular}[c]{@{}l@{}}2\\ 4\end{tabular} \\
	Definizione Prodotto & Verifica & Verificatore2 & 4 \\
	Codifica & Base & \begin{tabular}[c]{@{}l@{}}Programmatore1\\ Programmatore2\\ Programmatore3\\ Programmatore4\\ Programmatore5\\ Programmatore6\end{tabular} & \begin{tabular}[c]{@{}l@{}}20\\ 19\\ 17\\ 21\\ 24\\ 20\end{tabular} \\
	Codifica & Incremento & \begin{tabular}[c]{@{}l@{}}Programmatore1\\ Programmatore2\\ Programmatore3\\ Programmatore4\end{tabular} & \begin{tabular}[c]{@{}l@{}}6\\ 7\\ 10\\ 8\end{tabular} \\
	Codifica & Preparazione test unit\`{i} base & \begin{tabular}[c]{@{}l@{}}Verificatore1\\ Verificatore4\end{tabular} & \begin{tabular}[c]{@{}l@{}}15\\ 15\end{tabular} \\
	Codifica & Verifica Base & \begin{tabular}[c]{@{}l@{}}Verificatore2\\ Verificatore3\end{tabular} & \begin{tabular}[c]{@{}l@{}}13\\ 11\end{tabular} \\
	Codifica & Preparazione test unit\`{i} incremento & \begin{tabular}[c]{@{}l@{}}Verificatore5\\ Verificatore6\end{tabular} & \begin{tabular}[c]{@{}l@{}}10\\ 5\end{tabular} \\
	Codifica & Verifica Incremento & \begin{tabular}[c]{@{}l@{}}Verificatore5\\ Verificatore6\end{tabular} & \begin{tabular}[c]{@{}l@{}}5\\ 5\end{tabular} \\
	Piano di Progetto & Consuntivo & Responsabile & 2 \\
	Piano di Qualifica & Resoconto attivit\`{i} verifica & \begin{tabular}[c]{@{}l@{}}Verificatore4\\ Verificatore3\end{tabular} & \begin{tabular}[c]{@{}l@{}}4\\ 4\end{tabular} \\
	Piano di Qualifica & Verifica & Verificatore4 & 2 
	\end{tabular}
	\end{table}
}
	\subsection{Verifica e Validazione}
	\textbf{Periodo}: dal 19-06-2015 al 01-07-2015
	
	Le attivit\`{i} svolte in questo periodo saranno:
	\begin{itemize}
		\item \textbf{Esecuzione test}: non eseguiti durante il periodo di Progettazione e Codifica.
		\item \textbf{Validazione}: del sistema rispetto ai metodi previsti in Piano di Qualifica per ogni requisito del sistema.
		\item \textbf{Incremento manualistica}: destinata all'utente finale e documentazione per il rilascio.
	\end{itemize}
	
	\begin{figure}[H]
		\includegraphics[scale=0.5]{\imgs pianoAccettazione.jpg}
		\label{fig:pianoaccettazione}
		\caption{Piano di Accettazione}
	\end{figure}
	\begin{longtable} [c]{| l | l | l | l |}
	 \hline
	 \textbf{Macro-Attivit\`{a}} & \textbf{Attivit\`{a}} & \textbf{Ruolo} & \textbf{Ore}\\
	 \hline
	 \endfirsthead
	 \hline
	 \textbf{Macro-Attivit\`{a}} & \textbf{Attivit\`{a}} & \textbf{Ruolo} & \textbf{Ore}\\
	 \hline
		\endhead
	 \hline
	 \endfoot
	 \hline
	 \caption{Attivit\`{a} e ruoli Piano di Accettazione}
	 \endlastfoot
		Norme & Incremento & Responsabile 1 & 2 \\
		Norme & Verifica & Verificatore 1 & 1 \\
		Glossario & Incremento & / & 4 \\
		Glossario & Verifica & Verificatore 2 & 1 \\
		Piano di Qualifica & Incremento Verifiche & Verificatore 2 & 4 \\
		Piano di Qualifica & Validazione Interna & Programmatore 1 & 9 \\
		&	&	Verificatore 1	&	4\\
		&	&	Progettista 1	&	8\\
		&	&	Responsabile 2	&	6\\
		&	&	Amministratore	&	4\\
		Piano di Qualifica & Verifica & Verificatore 3 & 2 \\
		Definizione Prodotto & Incremento & Progettista 2 & 12 \\
		Definizione Prodotto & Verifica & Verificatore 3 & 2\\
		&	&	Verificatore 6 & 1\\
		Specifica Tecnica & Incremento & Progettista3 & 8 \\
		Specifica Tecnica & Verifica & Verificatore 4 & 5\\
		&	&	Verificatore 5 & 1\\
		Manualistica & Incremento & Progettista 4 & 6\\
		&	&	Progettista 5 & 4\\
		&	&	Amministratore & 3\\
		&	&	Responsabile 1 & 2\\
		Manualistica & Verifica & Verificatore 5, Verificatore 4 & 22 \\
		Piano di Progetto & Consuntivo & Responsabile 1 & 2 \\
	\end{longtable}
}