\section{Introduzione}
\subsection{Scopo del documento}
Lo scopo del documento è presentare al lettore un'analisi approfondita della pianificazione effettuata dal gruppo \gruppo, descrivendo e argomentando le scelte e le attività da effettuare durante il processo di sviluppo del prodotto.
\subsection{Glossario}
Al fine di evitare ogni ambiguità di linguaggio e massimizzare la comprensione dei documenti, i termini tecnici, di dominio, gli acronimi e le parole che necessitano di essere chiarite, sono riportate nel documento (((Glossario v?.?.?))). Ogni occorrenza di vocaboli presenti nel Glossario è marcata da una “G” maiuscola in pedice.
\subsection{Riferimenti}

\subsubsection{Normativi}
\begin{itemize}

\item Regole del progetto didattico, reperibili all'indirizzo:\\ \url{http://www.math.unipd.it/~tullio/IS-1/2014/Progetto/PD01.pdf}
\item Vincoli di organigramma, consultabili all’indirizzo:\\ \url{http://www.math.unipd.it/~tullio/IS-1/2014/Progetto/PD01b.html}
\item Norme di progetto: (((Norme di Progetto v?.?.?)));
\item Capitolato d’appalto C4: Premi: Software di presentazione “better than Prezi” \\
\url{http://www.math.unipd.it/~tullio/IS-1/2014/Progetto/C4.pdf}.
\end{itemize}

\subsubsection{Informativi}
\begin{itemize}
\item Slide dell'insegnamento Ingegneria del Software modulo A:\\
\begin{itemize}
\item Il ciclo di vita del software;
\item Gestione di progetto.
\end{itemize}
\url{http://www.math.unipd.it/~tullio/IS-1/2014/} ;

\item Ingegneria del software - Ian Sommerville - 9a Edizione (2010).
\end{itemize}

