\section{Introduzione}
\subsection{Scopo del documento}
Lo scopo del documento è presentare al lettore un'analisi approfondita della pianificazione effettuata dal gruppo \gruppo, descrivendo e argomentando le scelte e le attività da effettuare durante il processo\ped{g} di sviluppo del prodotto.
\subsection{Glossario}
Al fine di evitare ogni ambiguità di linguaggio e massimizzare la comprensione dei documenti, i termini tecnici, di dominio, gli acronimi e le parole che necessitano di essere chiarite, sono riportate nel documento \href{run:../../Esterni/\fGlossario}{\fEscapeGlossario}. Ogni occorrenza di vocaboli presenti nel Glossario è marcata da una “g” minuscola in pedice.
\subsection{Riferimenti}

\subsubsection{Normativi}
\begin{itemize}

\item Regole del progetto\ped{g} didattico, reperibili all'indirizzo\ped{g}:\\ \url{http://www.math.unipd.it/~tullio/IS-1/2014/Progetto/PD01.pdf}
\item Vincoli di organigramma, consultabili all’indirizzo\ped{g}:\\ \url{http://www.math.unipd.it/~tullio/IS-1/2014/Progetto/PD01b.html}
\item Norme di progetto\ped{g}: \href{run:../../Interni/\fNormeDiProgetto}{\fEscapeNormeDiProgetto};
\item Capitolato d’appalto C4: Premi: Software\ped{g} di presentazione “better than Prezi” \\
\url{http://www.math.unipd.it/~tullio/IS-1/2014/Progetto/C4.pdf}.
\end{itemize}

\subsubsection{Informativi}
\begin{itemize}
\item Slide dell'insegnamento Ingegneria del Software\ped{g} modulo A:\\
\begin{itemize}
\item Il ciclo di vita\ped{g} del software\ped{g};
\item Gestione di progetto\ped{g}.
\end{itemize}
\url{http://www.math.unipd.it/~tullio/IS-1/2014/} ;

\item Ingegneria del software\ped{g} - Ian Sommerville - 9a Edizione (2010).
\end{itemize}

