\section{Analisi dei rischi}{
Sono di seguito elencati i rischi evidenziati nella parte di analisi del progetto. 
I rischi saranno caratterizzati dalla pericolosit che potrà essere: bassa, media o alta.
Un'altra caratterizzazione è data dalla probabilità che l'evento associato al rischio si verifichi, la probabilità potrà essere: bassa, media o alta.\\
Per ogni rischio sarà definito un metodo con cui il rischio stesso sarà valutato nell'evolversi del progetto.\\
Per ogni rischio saranno definiti i metodi da usare come contromisure per diminuire la probabilità che il rischio si verifichi oppure per limitare il danno che il rischio creerebbe nel momento in cui si verificasse l'evento associato.\\
All'inizio di ogni nuova fase di progetto il responsabile dovrà aggiornare l'analisi dei rischi con nuovi rischi evidenziati dall'avanzare dello stato del progetto e aggiornando pericolosità e probabilità di avverarsi dei rischi precedentemente inseriti.

	\renewcommand*{\arraystretch}{1.6}
	\begin{longtable} [c]{|>{\centering\arraybackslash}m{3cm} | >{\centering\arraybackslash}m{6cm} | >{\centering\arraybackslash}m{6cm} |}
			\caption{Analisi dei Rischi \label{tab:analisirischi}}\\
			 \hline
			 \textbf{Rischio} & \textbf{Pericolosità} & \textbf{Probabilità} \\
			 \hline \endfirsthead
			 \hline
			 \textbf{Rischio} & \textbf{Pericolosità} & \textbf{Probabilità} \\
			 \hline \endhead
			 \hline \endfoot
			 \hline \endlastfoot
			  \multirow{3}{3cm}{Conoscenza delle tecnologie adottate} & Media & Media\\
			  \cline{2-3}
			  & \multicolumn{2}{p{12cm}|}{\textbf{Controllo}: il responsabile dovrà verificare la conoscenza da parte dei membri del gruppo dei linguaggi di programmazione e delle tecnologie che saranno adottate per lo sviluppo del sistema prima che il progetto entri in fase di Codifica.} \\
			  \cline{2-3}
			  & \multicolumn{2}{p{12cm}|}{\textbf{Contromisure}: il responsabile fornirà o consiglierà i documenti contenenti la base teorica e pratica per un utilizzo efficace dei linguaggi e delle tecnologie adottate per lo sviluppo del sistema.} \\
			  \hline
			  \multirow{3}{3cm}{Conoscenza degli strumenti di progetto} & Media & Media\\
			  \cline{2-3}
			  & \multicolumn{2}{p{12cm}|}{\textbf{Controllo}: l'amministratore prima che il progetto entri nella fase di progettazione dovrà verificare che tutti i componenti abbiano le conoscenze necessarie per utilizzare efficacemente gli strumenti per lo sviluppo e l'amministrazione del progetto.} \\
			  \cline{2-3}
			  & \multicolumn{2}{p{12cm}|}{\textbf{Contromisure}: l'amministratore fornirà i documenti contenenti la base teorica e pratica per un utilizzo efficace degli strumenti scelti per lo sviluppo e l'amministrazione del progetto.} \\
			  \hline
			  \multirow{3}{3cm}{Inesperienza di pianificazione} & Alta & Media\\
			  \cline{2-3}
			  & \multicolumn{2}{p{12cm}|}{\textbf{Controllo}: il responsabile di progetto dovrà monitorare il completamento delle attività assegnate ai componenti e confrontare lo stato del progetto con lo stato atteso dalla pianificazione.} \\
			%  \cline{2-3}
			  & \multicolumn{2}{p{12cm}|}{\textbf{Contromisure}: per ridurre la pericolosità del rischio è stato deciso di adottare un ciclo di vita incrementale ovvero in una prima iterazione si provvederà alla progettazione in dettaglio e codifica di una base di prodotto che comprenderà i requisiti obbligatori mentre ad una seconda iterazione verrà effettuata progettazione in dettaglio e codifica dei requisiti considerati desiderabili o opzionali. In questo modo anche se fosse stato sottostimato lo sforzo per lo sviluppo dei requisiti obbligatori del sistema si verrebbe comunque ad avere una base di prodotto con le funzionalità fondamentali, nel caso in cui il proponente considerasse di grande valore i requisiti desiderabili o opzionali che non potrebbero essere garantiti dall'attuale piano di progetto si provvederà ad aggiornare la pianificazione con inevitabili conseguenze sul prospetto economico.} \\
			  \hline
			  \multirow{3}{3cm}{Problemi hardware del server} & Alta & Bassa\\
			  \cline{2-3}
			  & \multicolumn{2}{p{12cm}|}{\textbf{Controllo}: ogni componente del gruppo usando i servizi offerti dal server controlleranno che esso funzioni correttamente, in caso contrario contatteranno l'amministratore.} \\
			  \cline{2-3}
			  & \multicolumn{2}{p{12cm}|}{\textbf{Contromisure}: ogni due giorni dovrà esser fatto in automatico il backup dei dati presenti sul server in un'apposita cartella su Google Drive.} \\
			  \hline
			  \multirow{3}{3cm}{Problemi personali dei componenti} & Media & Alta\\
			  \cline{2-3}
			  & \multicolumn{2}{p{12cm}|}{\textbf{Controllo}: quando un componente non potrà essere in grado di ricoprire i ruoli a lui assegnati dovrà segnalare il fatto attraverso il calendario condiviso di gruppo e comunicare il problema al responsabile di progetto.} \\
			  \cline{2-3}
			  & \multicolumn{2}{p{12cm}|}{\textbf{Contromisure}: il responsabile di progetto ottenuta la segnalazione di impossibilità da parte di un componente di rivestire il ruolo a lui assegnato provvederà a modificare la pianificazione in base a quanto riportato nel calendario di gruppo.} \\			  
			  \hline
			  \multirow{3}{3cm}{Problemi di relazione tra i componenti} & Alta & Media\\
			  \cline{2-3}
			  & \multicolumn{2}{p{12cm}|}{\textbf{Controllo}: il responsabile dovrà monitorare la nascita di problemi relazionali tra i componenti del gruppo.} \\
			  \cline{2-3}
			  & \multicolumn{2}{p{12cm}|}{\textbf{Contromisure}: una volta osservato un problema relazionale tra due componenti del gruppo il responsabile dovr à intervenire favorendo la convergenza di opinione oppure provvedere a modificare la pianificazione per ridurre la necessit à di relazione tra i due componenti.} \\
			  \hline
			  \multirow{3}{3cm}{Ambiguità dei requisiti} & Alta & Media\\
			  \cline{2-3}
			  & \multicolumn{2}{p{12cm}|}{\textbf{Controllo}: al fine di ridurre l'ambiguità dei requisiti si dovrà verificare che ogni termine non ovvio presente in analisi sia compreso nel glossario di progetto.} \\
			  \cline{2-3}
			  & \multicolumn{2}{p{12cm}|}{\textbf{Contromisure}: stesura di un glossario di progetto, inoltre gli incontri con il proponente dovranno produrre un verbale secondo quanto descritto nelle Norme di Progetto. La specifica dei requisiti dovrà essere accettata dal proponente prima di passare alla fase di Progettazione.}\\
	\end{longtable}
}


				   