\section{Analisi dei rischi}{
Sono di seguito elencati i rischi evidenziati nella parte di analisi del progetto. 
I rischi saranno caratterizzati dalla pericolosit che potr \`{a} essere: bassa, media o alta.
Un'altra caratterizzazione  \`{e} data dalla probabilit \`{a} che l'evento associato al rischio si verifichi, la probabilit \`{a} potr \`{a} essere: bassa, media o alta.\\
Per ogni rischio sar \`{a} definito un metodo con cui il rischio stesso sar \`{a} valutato nell'evolversi del progetto.\\
Per ogni rischio saranno definiti i metodi da usare come contromisure per diminuire la probabilit \`{a} che il rischio si verifichi oppure per limitare il danno che il rischio creerebbe nel momento in cui si verificasse l'evento associato.\\
All'inizio di ogni nuova fase di progetto il responsabile dovr \`{a} aggiornare l'analisi dei rischi con nuovi rischi evidenziati dall'avanzare dello stato del progetto e aggiornando pericolosit \`{a} e probabilit \`{a} di avverarsi dei rischi precedentemente inseriti.

	\begin{table}[H]
		\centering
		\begin{tabularx}{\textwidth}{p{0.2\textwidth}|p{0.4\textwidth}p{0.4\textwidth}}
				\hline
			   \textbf{Rischio} & \textbf{Pericolosit\`{a}} & \textbf{Probabilit\`{a}} \\
			   \multirow{5}{3cm}{Conoscenza dei linguaggi di programmazione e delle tecnologie adottate} & Media & Media\\
			   & \multicolumn{2}{p{0.7\textwidth}}{\textbf{Controllo}} \\
			   & \multicolumn{2}{p{0.7\textwidth}}{Il responsabile dovr\`{a} verificare la conoscenza da parte dei membri del gruppo dei linguaggi di programmazione e delle tecnologie che saranno adottate per lo sviluppo del sistema prima che il progetto entri in fase di Codifica} \\
			   & \multicolumn{2}{p{0.7\textwidth}}{\textbf{Contromisure}}\\
			   & \multicolumn{2}{p{0.7\textwidth}}{Il responsabile fornir\`{a} o consiglier\`{a} i documenti contenenti la base teorica e pratica per un utilizzo efficace dei linguaggi e delle tecnologie adottate per lo sviluppo del sistema} \\
			   \hline
		 \end{tabularx}
		 	\label{tab:analisirischi}
		 	\caption{Analisi dei Rischi}
		\end{table}
}

Conoscenza degli strumenti di progetto & Media & Media & L'amministratore prima che il progetto entri nella fase di progettazione dovr\`{a} verificare che tutti i componenti abbiano le conoscenze necessarie per utilizzare efficacemente gli strumenti per lo sviluppo e l'amministrazione del progetto & L'amministratore fornir\`{a} i documenti contenenti la base teorica e pratica per un utilizzo efficace degli strumenti scelti per lo sviluppo e l'amministrazione del progetto \\
				   Inesperienza di pianificazione & Alta & Media & Il responsabile di progetto dovr\`{a} monitorare il completamento delle attivit\`{a} assegnate ai componenti e confrontare lo stato del progetto con lo stato atteso dalla pianificazione & Per ridurre la pericolosit\`{a} del rischio \`{e} stato deciso di adottare un ciclo di vita incrementale ovvero in una prima iterazione si provveder\`{a} alla progettazione in dettaglio e codifica di una base di prodotto che comprender\`{a} i requisiti obbligatori mentre ad una seconda iterazione verr\`{a} effettuata progettazione in dettaglio e codifica dei requisiti considerati desiderabili o opzionali. In questo modo anche se fosse stato sottostimato lo sforzo per lo sviluppo dei requisiti obbligatori del sistema si verrebbe comunque ad avere una base di prodotto con le funzionalit\`{a} fondamentali, nel caso in cui il proponente considerasse di grande valore i requisiti desiderabili o opzionali che non potrebbero essere garantiti dall'attuale piano di progetto si provveder\`{a} ad aggiornare la pianificazione con inevitabili conseguenze sul prospetto economico. \\
				   Problemi hardware del server & Alta & Bassa & Ogni componente del gruppo usando i servizi offerti dal server controlleranno che esso funzioni correttamente, in caso contrario contatteranno l'amministratore & Ogni due giorni dovr\`{a} esser fatto in automatico il backup dei dati presenti sul server in un'apposita cartella su Google Drive \\
				   Problemi personali dei componenti & Media & Alta & Quando un componente non potr\`{a} essere in grado di ricoprire i ruoli a lui assegnati dovr\`{a} segnalare il fatto attraverso il calendario condiviso di gruppo e comunicare il problema al responsabile di progetto & Il responsabile di progetto ottenuta la segnalazione di impossibilit\`{a} da parte di un componente di rivestire il ruolo a lui assegnato provveder\`{a} a modificare la pianificazione in base a quanto riportato nel calendario di gruppo \\
				   Problemi di relazione tra i componenti & Alta & Media & Il responsabile dovr\`{a} monitorare la nascita di problemi relazionali tra i componenti del gruppo & Una volta osservato un problema relazionale tra due componenti del gruppo il responsabile dovr \`{a} intervenire favorendo la convergenza di opinione oppure provvedere a modificare la pianificazione per ridurre la necessit \`{a} di relazione tra i due componenti \\
				   Ambiguit\`{a} dei requisiti & Alta & Media & Al fine di ridurre l'ambiguit\`{a} dei requisiti si dovr\`{a} verificare che ogni termine non ovvio presente in analisi sia compreso nel glossario di progetto & Stesura di un glossario di progetto, inoltre gli incontri con il proponente dovranno produrre un verbale secondo quanto descritto in Norme di Progetto sezione X. La specifica dei requisiti dovr\`{a} essere accettata dal proponente prima di passare alla fase di Progettazione. \\
				   