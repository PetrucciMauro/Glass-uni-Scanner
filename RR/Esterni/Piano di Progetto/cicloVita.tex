\section{Ciclo di vita}{

Si � scelto di applicare ai processi il modello incrementale per i seguenti motivi:

\begin{itemize}

\item ottenere il prima possibile un sistema funzionante sulle parti critiche del sistema

\item testare maggiormente le parti critiche del sistema grazie anche all'integrazione successiva delle parti desiderabili o opzionali

\item a causa dell'inesperienza del gruppo nella previsione dei tempi di sviluppo, in questo modo si limita il rischio di aver sottostimato i tempi riguardanti la progettazione in dettaglio e codifica dei requisiti obbligatori poich� saranno trattati nella prima iterazione

L'adozione di questo modello permette di rilasciare al committente una base di prodotto con  l'insieme delle funzionalit� fondamentali il prima possibile, cos� da permettere al committente di valutare in corso d'opera il lavoro svolto. 

Questo modello permette in caso di sottostima dei tempi di realizzazione di avere comunque un prodotto con le funzionalit� di base richieste poich� queste saranno trattate nella prima iterazione.

Si avr� quindi il vantaggio di spendere inizialmente le risorse nella realizzazione di una base di prodotto funzionante che presenti gli aspetti del sistema di maggiore importanza. 

Si potr� in seguito migliorare tale base ed utilizzarla per ampliare il prodotto con le funzionalit� opzionali e desiderabili.
}


