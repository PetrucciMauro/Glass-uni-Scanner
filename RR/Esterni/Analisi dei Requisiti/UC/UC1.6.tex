\subsection{UC 1.6 - Creazione di un'infografica}{
	\label{uc1.6}
	%\begin{figure}[H]
	%	\centering
	%	\includegraphics[scale=0.75]{\imgs {UC1.6}.jpg} %inserire il diagramma UML
	%	\label{fig:uc1.6}
	%	\caption{Caso d'uso 1.6}
	%\end{figure}
	\textbf{Attori}: amministratore di sistema. \\
	\textbf{Descrizione}: l'utente è in grado di prendere gli elementi di una presentazione slide e di inserirli in un unico documento stampabile e quindi lineare. \\
	\textbf{Precondizione}: il programma è acceso e funzionante.	\\
	\textbf{Postcondizione}: l'utente ha creato un'infografica.	\\
	\textbf{Procedura principale}:
	\begin{enumerate}
		\item una volta creata l'infografica si entra sempre in modalità modifica \hyperref[uc1.17]{(UC 1.17)}.
	\end{enumerate}
	}
