\subsection{UC 1.6 - Creazione di un'infografica}{
	\label{uc1.6}
	\begin{figure}[H]
		\centering
		\includegraphics[scale=0.75]{\imgs {UC1.6}.jpg} %inserire il diagramma UML
		\label{fig:uc1.6}
		\caption{Caso d'uso 1.6}
	\end{figure}
	\textbf{Attori}: utente desktop, amministratore di sistema. \\
	\textbf{Descrizione}: l'utente è in grado di prendere gli elementi di una presentazione slide e di inserirli in un unico documento stampabile e quindi lineare. \\
	\textbf{Precondizione}: il programma è acceso e funzionante.	\\
	\textbf{Postcondizione}: l'utente ha creato un'infografica.	\\
	\textbf{Procedura principale}:
	\begin{enumerate}
		\item selezione della modalità "crea infografica" \hyperref[uc1.6.1]{(UC 1.6.1)};
		\item una volta creata l'infografica si entra sempre in modalità modifica \hyperref[uc1.17]{(UC 1.17)}.
	\end{enumerate}
	}
\subsubsection{UC 1.6.1 - Selezione della modalità infografica}{
	\label{uc1.6.1}
	\textbf{Attori}: utente desktop \\
	\textbf{Descrizione}: permettere all'utente di entrare nella modalità adatta alla creazione di un'infografica. \\
	\textbf{Precondizione}: il programma è attivo e funzionante.	\\
	\textbf{Postcondizione}: l'utente ha avuto accesso alla sezione del programma dedicata alla creazione e modifica dell'infografica.	\\
	\textbf{Procedura principale}:
	\begin{enumerate}
		\item  selezione di un riquadro apposito tramite mouse o tastiera.
	\end{enumerate}
	}