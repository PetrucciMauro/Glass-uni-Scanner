\subsection{UC 1.7 - Gestione archivio utente sul server}{
	\label{uc1.7}
	\begin{figure}[H]
		\centering
		\includegraphics[scale=0.75]{\imgs {UC1.7}.jpg} %inserire il diagramma UML
		\label{fig:uc1.7}
		\caption{Caso d'uso 1.7}
	\end{figure}
	\textbf{Attori}: utente autenticato. \\
	\textbf{Descrizione}: l'utente può modificare i contenuti (immagini e presentazioni) presenti nel proprio spazio all'interno del server, inserendone di nuovi o eliminando quelli presenti. \\
	\textbf{Precondizione}: il programma è attivo e funzionante; l'utente ha effettuato il login e desidera effettuare modifiche ai propri file nello spazio all'interno del server.	\\
	\textbf{Postcondizione}: l'utente elimina o carica un'immagine o presentazione selezionata nel proprio spazio all'interno del server.	\\
	\textbf{Procedura principale}:
	\begin{enumerate}
		\item inserimento file media \hyperref[uc1.7.1]{(UC 1.7.1)};
		\item cancellazione file media \hyperref[uc1.7.2]{(UC 1.7.2)};
		\item spostamento file media \hyperref[uc1.7.3]{(UC 1.7.3)};
		\item spostamento presentazioni \hyperref[uc1.7.4]{(UC 1.7.4)};
		\item eliminazione presentazioni \hyperref[uc1.7.5]{(UC 1.7.5)}.
	\end{enumerate}
	}
\subsubsection{UC 1.7.1 - Inserimento file media}{
	\label{uc1.7.1}
	\textbf{Attori}: utente autenticato. \\
	\textbf{Descrizione}: l'utente ha la possibilità di caricare uno o più file media (audio, video, immagini) contenuti nel proprio computer o dispositivo. \\
	\textbf{Precondizione}: l'utente desidera caricare sul server uno o più file media esistenti in locale.	\\
	\textbf{Postcondizione}: l'utente ha caricato sul server i file media selezionate.	\\
	\textbf{Procedura principale}:
	\begin{enumerate}
		\item selezione file media;
		\item navigazione nello spazio di lavoro;
		\item conferma e caricamento.
	\end{enumerate}
	\textbf{Scenari alternativi}: 
	\begin{itemize}
		\item l'operazione viene annullata e non si apportano cambiamenti.
	\end{itemize}
	}
	\subsubsection{UC 1.7.2 - Cancellazione file media}{
		\label{uc1.7.2}
		\textbf{Attori}: utente autenticato.	\\
		\textbf{Descrizione}: l'utente ha la possibilità di eliminare file media presenti all'interno del proprio spazio di lavoro sul server. \\
		\textbf{Precondizione}: l'utente desidera eliminare file media esistenti nel proprio spazio di lavoro sul server.	\\
		\textbf{Postcondizione}: l'utente ha cancellato i file media che desiderava rimuovere.	\\
		\textbf{Procedura principale}:
		\begin{enumerate}
			\item navigazione nello spazio di lavoro;
			\item selezione file media;
			\item conferma selezione ed eliminazione.
		\end{enumerate}
		\textbf{Scenari alternativi}: 
		\begin{itemize}
			\item l'operazione viene annullata e non si apportano cambiamenti.
		\end{itemize}
		}
	\subsubsection{UC 1.7.3 - Spostamento file media}{
		\label{uc1.7.3}
		\textbf{Attori}: utente autenticato.	\\
		\textbf{Descrizione}: l'utente ha la possibilità di selezionare file media presenti in una cartella all'interno del proprio spazio di lavoro nel server e spostarli. \\
		\textbf{Precondizione}: l'utente desidera spostare file media esistenti nel proprio spazio di lavoro sul server.	\\
		\textbf{Postcondizione}: l'utente ha spostato i file media che desiderava spostare dalla posizione in cui queste si trovavano, in una posizione valida.	\\
		\textbf{Procedura principale}:
		\begin{enumerate}
			\item navigazione nello spazio di lavoro;
			\item selezione file media;
			\item selezione destinazione;
			\item conferma selezione e spostamento.
		\end{enumerate}
		\textbf{Scenari alternativi}: 
		\begin{itemize}
			\item l'operazione viene annullata e non si apportano cambiamenti.
		\end{itemize}
		}
	\subsubsection{UC 1.7.4 - Spostamento presentazioni}{
		\label{uc1.7.4}
		\textbf{Attori}: utente autenticato.	\\
		\textbf{Descrizione}: l'utente ha la possibilità di selezionare delle presentazioni inserite in una cartella all'interno del proprio spazio di lavoro nel server e spostarle. \\
		\textbf{Precondizione}: l'utente desidera spostare una o più presentazioni esistenti nel proprio spazio sul server.	\\
		\textbf{Postcondizione}: l'utente ha spostato le presentazioni che desiderava spostare dalla posizione in cui queste si trovavano, in una posizione valida.	\\
		\textbf{Procedura principale}:
		\begin{enumerate}
			\item navigazione nello spazio di lavoro;
			\item selezione presentazioni da spostare;
			\item selezione destinazione;
			\item conferma selezione e spostamento.
		\end{enumerate}
		\textbf{Scenari alternativi}: 
		\begin{itemize}
			\item l'operazione viene annullata e non si apportano cambiamenti.
		\end{itemize}
		}
	\subsubsection{UC 1.7.5 - Eliminazione presentazioni}{
		\label{uc1.7.5}
		\textbf{Attori}: utente autenticato.	\\
		\textbf{Descrizione}: l'utente ha la possibilità di eliminare presentazioni contenute all'interno del proprio spazio di lavoro sul server. \\
		\textbf{Precondizione}: l'utente desidera eliminare una o più presentazioni esistenti nel proprio spazio sul server.	\\
		\textbf{Postcondizione}: l'utente ha cancellato le presentazioni che desiderava rimuovere.	\\
		\textbf{Procedura principale}:
		\begin{enumerate}
			\item navigazione nello spazio di lavoro;
			\item selezione presentazioni da eliminare;
			\item conferma selezione ed eliminazione.
		\end{enumerate}
		\textbf{Scenari alternativi}: 
		\begin{itemize}
			\item l'operazione viene annullata e non si apportano cambiamenti.
		\end{itemize}
		}
	\subsubsection{UC 1.7.6 - Spostamento infografiche}{
		\label{uc1.7.6}
		\textbf{Attori}: utente autenticato.	\\
		\textbf{Descrizione}: l'utente ha la possibilità di selezionare delle infografiche presenti all'interno del proprio spazio di lavoro nel server e spostarle. \\
		\textbf{Precondizione}: l'utente desidera spostare una o più infografica esistente nel proprio spazio di lavoro sul server.	\\
		\textbf{Postcondizione}: l'utente ha spostato le infografiche che desiderava spostare dalla posizione in cui queste si trovavano, in una posizione valida.	\\
		\textbf{Procedura principale}:
		\begin{enumerate}
			\item navigazione nello spazio di lavoro;
			\item selezione infografiche da spostare;
			\item selezione di destinazione;
			\item conferma selezione e spostamento.
		\end{enumerate}
		\textbf{Scenari alternativi}: 
		\begin{itemize}
			\item l'operazione viene annullata e non si apportano cambiamenti.
		\end{itemize}
		}
	\subsubsection{UC 1.7.7 - Eliminazione infografiche}{
		\label{uc1.7.7}
		\textbf{Attori}: utente autenticato.	\\
		\textbf{Descrizione}: l'utente ha la possibilità di eliminare infografiche contenute all'interno del proprio spazio li lavoro sul server. \\
		\textbf{Precondizione}: l'utente desidera eliminare una o più infografica esistente nel proprio spazio sul server.	\\
		\textbf{Postcondizione}: l'utente ha cancellato le infografiche che desiderava rimuovere.	\\
		\textbf{Procedura principale}:
		\begin{enumerate}
			\item navigazione nello spazio di lavoro;
			\item selezione infografiche da eliminare;
			\item conferma selezione ed eliminazione.
		\end{enumerate}
		\textbf{Scenari alternativi}: 
		\begin{itemize}
			\item l'operazione viene annullata e non si apportano cambiamenti.
		\end{itemize}
		}
