\subsection{UC 1.7 - Gestione archivio utente sul server}{
	\label{uc1.7}
	\begin{figure}[H]
		\centering
		\includegraphics[scale=0.75]{\imgs {UC1.7}.jpg} %inserire il diagramma UML
		\label{fig:uc1.7}
		\caption{Caso d'uso 1.7}
	\end{figure}
	\textbf{Attori}: utente autenticato. \\
	\textbf{Descrizione}: l'utente può modificare i contenuti (immagini e presentazioni) presenti nel proprio spazio all'interno del server, inserendone di nuovi o eliminando quelli presenti. \\
	\textbf{Precondizione}: il programma è attivo e funzionante; l'utente ha effettuato il login e desidera effettuare modifiche ai propri file nello spazio all'interno del server.	\\
	\textbf{Postcondizione}: l'utente elimina o carica un'immagine o presentazione selezionata nel proprio spazio all'interno del server.	\\
	\textbf{Procedura principale}:
	\begin{enumerate}
		\item inserimento immagini \hyperref[uc1.7.1]{(UC 1.7.1)};
		\item cancellazione immagini \hyperref[uc1.7.2]{(UC 1.7.2)};
		\item spostamento immagini nel FileSystem \hyperref[uc1.7.3]{(UC 1.7.3)};
		\item spostamento presentazioni \hyperref[uc1.7.4]{(UC 1.7.4)};
		\item eliminazione presentazioni \hyperref[uc1.7.5]{(UC 1.7.5)}.
	\end{enumerate}
	}
\subsubsection{UC 1.7.1 - Inserimento immagini}{
	\label{uc1.7.1}
	\begin{figure}[H]
		\centering
		\includegraphics[scale=0.75]{\imgs {UC1.7.1}.jpg} %inserire il diagramma UML
		\label{fig:uc1.7.1}
		\caption{Caso d'uso 1.7.1}
	\end{figure}
	\textbf{Attori}: utente autenticato. \\
	\textbf{Descrizione}: l'utente ha la possibilità di caricare una o più immagini contenute nel proprio computer o dispositivo. \\
	\textbf{Precondizione}: l'utente desidera caricare sul server una o più immagini esistenti in locale.	\\
	\textbf{Postcondizione}: l'utente ha caricato sul server le immagini selezionate.	\\
	\textbf{Procedura principale}:
	\begin{enumerate}
		\item selezione immagini \hyperref[uc1.7.1.1]{(UC 1.7.1.1)};
		\item navigazione nel FileSystem delle immagini \hyperref[uc1.7.1.2]{(UC 1.7.1.2)};
		\item conferma e caricamento \hyperref[uc1.7.1.3]{(UC 1.7.1.3)}.
	\end{enumerate}
	\textbf{Scenari alternativi}: 
	\begin{itemize}
		\item l'operazione viene annullata e non si apportano cambiamenti.
	\end{itemize}
	}
	\subsubsection{UC 1.7.1.1 - Selezione immagini}{
		\label{uc1.7.1.1}
		\begin{figure}[H]
			\centering
			\includegraphics[scale=0.75]{\imgs {UC1.7.1.1}.jpg} %inserire il diagramma UML
			\label{fig:uc1.7.1.1}
			\caption{Caso d'uso 1.7.1.1}
		\end{figure}
		\textbf{Attori}: utente autenticato.\\
		\textbf{Descrizione}: l'utente deve selezionare una o più immagini contenute nel proprio dispositivo. \\
		\textbf{Precondizione}: il sistema è in attesa che l'utente specifichi un percorso da cui selezionare le immagini.	\\
		\textbf{Postcondizione}: l'utente ha inizializzato l'attività di upload di immagini e ha a disposizione le immagini che desidera caricare.	\\
		\textbf{Procedura principale}:
		\begin{enumerate}
			\item navigazione FileSystem locale \hyperref[uc1.7.1.1.1]{(UC 1.7.1.1.1)};
			\item selezione immagini locali \hyperref[uc1.7.1.1.2]{(UC 1.7.1.1.2)};
			\item conferma selezione \hyperref[uc1.7.1.1.3]{(UC 1.7.1.1.3)}.
		\end{enumerate}
		}
		\subsubsection{UC 1.7.1.1.1 - Navigazione FileSystem locale}{
			\label{uc1.7.1.1.1}
			\textbf{Attori}: utente autenticato.\\
			\textbf{Descrizione}: l'utente deve selezionare una posizione esistente in cui sono contenute le immagini da caricare. \\
			\textbf{Precondizione}: il sistema mostra all'utente le cartelle e le immagini presenti nella directory corrente.	\\
			\textbf{Postcondizione}: l'utente si trova nella cartella che contiene le immagini da caricare.	\\
			}
		\subsubsection{UC 1.7.1.1.2 - Selezione immagini locali}{
			\label{uc1.7.1.1.2}
			\textbf{Attori}: utente \\
			\textbf{Descrizione}: l'utente seleziona una o più immagini da caricare. \\
			\textbf{Precondizione}: l'utente ha scelto la cartella da cui caricare le immagini; le immagini sono presenti nella cartella. \\
			\textbf{Postcondizione}: l'utente ha scelto le immagini dalla cartella corrente.	\\
			}
		\subsubsection{UC 1.7.1.1.3 - Conferma selezione}{
			\label{uc1.7.1.1.3}
			\textbf{Attori}: utente autenticato.\\
			\textbf{Descrizione}: l'utente conferma la propria scelta ed è pronto a proseguire. \\
			\textbf{Precondizione}: l'utente ha selezionato una o più immagini presenti nella cartella corrente.	\\
			\textbf{Postcondizione}: l'utente conferma le operazioni precedenti.	\\
			}
	\subsubsection{UC 1.7.1.2 - Navigazione nel FileSystem delle immagini}{
		\label{uc1.7.1.2}
		\begin{figure}[H]
			\centering
			\includegraphics[scale=0.75]{\imgs {UC1.7.1.2}.jpg} %inserire il diagramma UML
			\label{fig:uc1.7.1.2}
			\caption{Caso d'uso 1.7.1.2}
		\end{figure}
		\textbf{Attori}: utente autenticato.\\
		\textbf{Descrizione}: l'utente seleziona una directory nel server in cui inserire le immagini scelte in precedenza. \\
		\textbf{Precondizione}: l'utente ha selezionato una o più immagini da una directory locale; il sistema mostra all'utente le proprie directory nel FileSystem delle immagini all'interno del server.	\\
		\textbf{Postcondizione}: l'utente ha selezionato una directory in cui eseguire il salvataggio.	\\
		\textbf{Procedura principale}:
		\begin{enumerate}
			\item navigazione cartelle FileSystem immagini \hyperref[uc1.7.1.2.1]{(UC 1.7.1.2.1)};
			\item selezione cartella di destinazione \hyperref[uc1.7.1.2.2]{(UC 1.7.1.2.2)};
			\item conferma scelta cartella \hyperref[uc1.7.1.2.3]{(UC 1.7.1.2.3)}.
		\end{enumerate}
		}
		\subsubsection{UC 1.7.1.2.1 - Navigazione cartelle FileSystem immagini}{
			\label{uc1.7.1.2.1}
			\textbf{Attori}: utente autenticato.	\\
			\textbf{Descrizione}: l'utente deve selezionare la posizione in cui inserire i file selezionati; può creare una nuova cartella in cui inserirli. \\
			\textbf{Precondizione}: il sistema mostrae le cartelle dell'utente inserite nello spazio riservato alle immagini nel server.	\\
			\textbf{Postcondizione}: l'utente si trova nella cartella in cui desidera siano inserite le immagini che aveva selezionato in precedenza.	\\
			}
		\subsubsection{UC 1.7.1.2.2 - Selezione cartella di destinazione}{
			\label{uc1.7.1.2.2}
			\textbf{Attori}: utente autenticato.	\\
			\textbf{Descrizione}: l'utente conferma la directory corrente come quella in cui eseguire il caricamento. \\
			\textbf{Precondizione}: l'utente si trova nella directory in cui eseguire il caricamento.	\\
			\textbf{Postcondizione}: l'utente ha confermato la scelta fatta.	\\
			}
		\subsubsection{UC 1.7.1.2.3 - Conferma scelta cartella}{
			\label{uc1.7.1.2.3}
			\textbf{Attori}: utente autenticato.	\\
			\textbf{Descrizione}: il sistema memorizza la directory selezionata dall'utente per il salvataggio. \\
			\textbf{Precondizione}: l'utente ha selezionato la cartella in cui salvare le immagini.	\\
			\textbf{Postcondizione}: il sistema ha in memoria la directory selezionata dall'utente.	\\
			}
	\subsubsection{UC 1.7.1.3 - Conferma e caricamento}{
		\label{uc1.7.1.3}
		\textbf{Attori}: utente autenticato.\\
		\textbf{Descrizione}: il sistema scrive le immagini selezionate dall'utente all'interno della cartella indicata. \\
		\textbf{Precondizione}: l'utente ha selezionato le immagini da caricare e la posizione in cui salvarle.	\\
		\textbf{Postcondizione}: le immagini scelte vengono scritte nella posizione indicata.	\\
		}
	\subsubsection{UC 1.7.2 - Cancellazione immagini}{
		\label{uc1.7.2}
		\begin{figure}[H]
			\centering
			\includegraphics[scale=0.75]{\imgs {UC1.7.2}.jpg} %inserire il diagramma UML
			\label{fig:uc1.7.2}
			\caption{Caso d'uso 1.7.2}
		\end{figure}
		\textbf{Attori}: utente autenticato.	\\
		\textbf{Descrizione}: l'utente ha la possibilità di eliminare immagini presenti in una cartella all'interno del proprio spazio riservato alle immagini sul server. \\
		\textbf{Precondizione}: l'utente desidera eliminare immagini esistenti nel proprio spazio sul server.	\\
		\textbf{Postcondizione}: l'utente ha cancellato le immagini che desiderava rimuovere.	\\
		\textbf{Procedura principale}:
		\begin{enumerate}
			\item navigazione FileSystem immagini \hyperref[uc1.7.2.1]{(UC 1.7.2.1)};
			\item selezione cartella \hyperref[uc1.7.2.2]{(UC 1.7.2.2)};
			\item selezione immagini \hyperref[uc1.7.2.3]{(UC 1.7.2.3)};
			\item conferma selezione ed eliminazione \hyperref[uc1.7.2.4]{(UC 1.7.2.4)}.
		\end{enumerate}
		\textbf{Scenari alternativi}: 
		\begin{itemize}
			\item l'operazione viene annullata e non si apportano cambiamenti.
		\end{itemize}
		}
		\subsubsection{UC 1.7.2.1 - Navigazione FileSystem immagini}{
			\label{uc1.7.2.1}
			\textbf{Attori}: utente autenticato. \\
			\textbf{Descrizione}: l'utente deve selezionare un percorso valido all'interno del server in cui sono inserite le immagini che intende eliminare. \\
			\textbf{Precondizione}: l'utente ha iniziato l'operazione d'eliminazione di immagini; il sistema è in attesa che l'utente specifichi il percorso; il server mostra i file immagine e le cartelle dell'utente all'interno dello spazio delle immagini.	\\
			\textbf{Postcondizione}: l'utente si trova nella cartella con i file immagine da eliminare.	\\
			}
		\subsubsection{UC 1.7.2.2 - Selezione cartella}{
			\label{uc1.7.2.2}
			\textbf{Attori}: utente autenticato. \\
			\textbf{Descrizione}: l'utente conferma la posizione. \\
			\textbf{Precondizione}: l'utente ha selezionato la cartella in cui sono presenti le immagini da eliminare.	\\
			\textbf{Postcondizione}: l'utente si trova all'interno della cartella in cui sono presenti le immagini da eliminare. 	\\
			}
		\subsubsection{UC 1.7.2.3 - Selezione immagini}{
			\label{uc1.7.2.3}
			\textbf{Attori}: utente autenticato. \\
			\textbf{Descrizione}: l'utente seleziona una o più immagini esistenti da rimuovere. \\
			\textbf{Precondizione}: l'utente ha selezionato la cartella in cui sono presenti le immagini che intende eliminare e si trova all'interno di tale cartella; il sistema mostra all'utente le immagini presenti.	\\
			\textbf{Postcondizione}: l'utente ha selezionato una o più immagini esistenti da eliminare.	\\
			}
		\subsubsection{UC 1.7.2.4 - Conferma selezione ed eliminazione}{
			\label{uc1.7.2.4}
			\textbf{Attori}: utente autenticato. \\
			\textbf{Descrizione}: l'utente conferma la selezione ed il sistema procede all'eliminazione. \\
			\textbf{Precondizione}: l'utente ha selezionato i file immagine da eliminare; tali file esistono.	\\
			\textbf{Postcondizione}: il sistema ha eliminato i file immagini selezionati dalla cartella selezionata.	\\
			}
	\subsubsection{UC 1.7.3 - Spostamento immagini nel FileSystem delle immagini}{
		\label{uc1.7.3}
		\begin{figure}[H]
			\centering
			\includegraphics[scale=0.75]{\imgs {UC1.7.3}.jpg} %inserire il diagramma UML
			\label{fig:uc1.7.3}
			\caption{Caso d'uso 1.7.3}
		\end{figure}
		\textbf{Attori}: utente autenticato.	\\
		\textbf{Descrizione}: l'utente ha la possibilità di selezionare immagini presenti in una cartella all'interno del proprio spazio riservato alle immagini sul server e spostarle in un'altra directory. \\
		\textbf{Precondizione}: l'utente desidera spostare immagini esistenti nel proprio spazio sul server.	\\
		\textbf{Postcondizione}: l'utente ha spostato le immagini che desiderava spostare dalla posizione in cui queste si trovavano, in una posizione valida.	\\
		\textbf{Procedura principale}:
		\begin{enumerate}
			\item navigazione nel FileSystem immagini \hyperref[uc1.7.3.1]{(UC 1.7.3.1)};
			\item selezione cartella di lavoro \hyperref[uc1.7.3.2]{(UC 1.7.3.2)};
			\item selezione immagini \hyperref[uc1.7.3.3]{(UC 1.7.3.3)};
			\item selezione cartella di destinazione \hyperref[uc1.7.3.4]{(UC 1.7.3.4)};
			\item conferma selezione e spostamento \hyperref[uc1.7.3.5]{(UC 1.7.3.5)}.
		\end{enumerate}
		\textbf{Scenari alternativi}: 
		\begin{itemize}
			\item l'operazione viene annullata e non si apportano cambiamenti.
		\end{itemize}
		}
		\subsubsection{UC 1.7.3.1 - Navigazione nel FileSystem immagini}{
			\label{uc1.7.3.1}
			\textbf{Attori}: utente autenticato. \\
			\textbf{Descrizione}: l'utente deve selezionare un percorso valido all'interno del server in cui sono inserite le immagini che intende spostare. \\
			\textbf{Precondizione}: l'utente ha iniziato l'operazione di spostamento di immagini; il sistema è in attesa che l'utente specifichi il percorso di origine; il server mostra i file immagine e le cartelle dell'utente all'interno dello spazio delle immagini.	\\
			\textbf{Postcondizione}: l'utente si trova nella cartella con i file immagine da spostare.	\\
			}
		\subsubsection{UC 1.7.3.2 - Selezione cartella di lavoro}{
			\label{uc1.7.3.2}
			\textbf{Attori}: utente autenticato. \\
			\textbf{Descrizione}: l'utente conferma la posizione. \\
			\textbf{Precondizione}: l'utente ha selezionato la cartella in cui sono presenti le immagini da spostare.	\\
			\textbf{Postcondizione}: l'utente si trova all'interno della cartella in cui sono presenti le immagini da spostare. 	\\
			}
		\subsubsection{UC 1.7.3.3 - Selezione immagini}{
			\label{uc1.7.3.3}
			\textbf{Attori}: utente autenticato. \\
			\textbf{Descrizione}: l'utente seleziona una o più immagini esistenti da spostare. \\
			\textbf{Precondizione}: l'utente ha selezionato la cartella in cui sono presenti le immagini che intende spostare e si trova all'interno di tale cartella; il sistema mostra all'utente le immagini presenti.	\\
			\textbf{Postcondizione}: l'utente ha selezionato una o più immagini esistenti da spostare.	\\
			}
		\subsubsection{UC 1.7.3.4 - Selezione cartella di destinazione}{
			\label{uc1.7.3.4}
			\textbf{Attori}: utente autenticato. \\
			\textbf{Descrizione}: l'utente seleziona dal FileSystem immagini una cartella tra quelle presenti o ne crea una di nuova. \\
			\textbf{Precondizione}: l'utente ha selezionato delle immagini all'interno di un percorso valido nel FileSystem delle immagini; il sistema è in attesa che l'utente specifichi il percorso di destinazione; il server mostra i file immagine e le cartelle dell'utente all'interno dello spazio delle immagini.	\\
			\textbf{Postcondizione}: l'utente ha selezionato una cartella di destinazione valida all'interno del FileSystem delle immagini.	\\
			}
		\subsubsection{UC 1.7.3.5 - Conferma selezione e spostamento}{
			\label{uc1.7.3.5}
			\textbf{Attori}: utente autenticato. \\
			\textbf{Descrizione}: l'utente conferma la selezione ed il sistema procede allo spostamento. \\
			\textbf{Precondizione}: l'utente ha selezionato i file immagine da spostare e la destinazione in cui trasferirli; tali file esistono.	\\
			\textbf{Postcondizione}: il sistema ha spostato i file immagini selezionati nella cartella selezionata dall'utente.	\\
			}
	\subsubsection{UC 1.7.4 - Spostamento presentazioni}{
		\label{uc1.7.4}
		\begin{figure}[H]
			\centering
			\includegraphics[scale=0.75]{\imgs {UC1.7.4}.jpg} %inserire il diagramma UML
			\label{fig:uc1.4}
			\caption{Caso d'uso 1.4}
		\end{figure}
		\textbf{Attori}: utente autenticato.	\\
		\textbf{Descrizione}: l'utente ha la possibilità di selezionare delle presentazioni inserite in una cartella all'interno del proprio spazio riservato alle presentazioni nel server e spostarle in un'altra directory, oppure in una nuova. \\
		\textbf{Precondizione}: l'utente desidera spostare una o più presentazioni esistenti nel proprio spazio sul server.	\\
		\textbf{Postcondizione}: l'utente ha spostato le presentazioni che desiderava spostare dalla posizione in cui queste si trovavano, in una posizione valida.	\\
		\textbf{Procedura principale}:
		\begin{enumerate}
			\item navigazione nel FileSystem delle presentazioni \hyperref[uc1.7.4.1]{(UC 1.7.4.1)};
			\item selezione cartella di lavoro \hyperref[uc1.7.4.2]{(UC 1.7.4.2)};
			\item selezione presentazioni da spostare \hyperref[uc1.7.4.3]{(UC 1.7.4.3)};
			\item selezione cartella di destinazione \hyperref[uc1.7.4.4]{(UC 1.7.4.4)};
			\item conferma selezione e spostamento \hyperref[uc1.7.4.5]{(UC 1.7.4.5)}.
		\end{enumerate}
		\textbf{Scenari alternativi}: 
		\begin{itemize}
			\item l'operazione viene annullata e non si apportano cambiamenti.
		\end{itemize}
		}
		\subsubsection{UC 1.7.4.1 - Navigazione nel FileSystem delle presentazioni}{
			\label{uc1.7.4.1}
			\textbf{Attori}: utente autenticato. \\
			\textbf{Descrizione}: l'utente deve selezionare la cartella in cui sono inserite le presentazioni che intende spostare. Tale cartella si trova nel server, nello spazio riservato alle presentazioni. \\
			\textbf{Precondizione}: l'utente ha iniziato l'operazione di spostamento di presentazioni; il sistema è in attesa che l'utente specifichi il percorso di origine; il server mostra presentazioni e cartelle dell'utente all'interno dello spazio delle presentazioni.	\\
			\textbf{Postcondizione}: l'utente si trova nella cartella con le presentazioni da spostare.	\\
			}
		\subsubsection{UC 1.7.4.2 - Selezione cartella di lavoro}{
			\label{uc1.7.4.2}
			\textbf{Attori}: utente autenticato. \\
			\textbf{Descrizione}: l'utente deve selezionare la cartella in cui sono inserite le presentazioni che intende spostare. Tale cartella si trova nel server, nello spazio riservato alle presentazioni. \\
			\textbf{Precondizione}: l'utente ha iniziato l'operazione di spostamento di presentazioni; il sistema è in attesa che l'utente specifichi il percorso di origine; il server mostra presentazioni e cartelle dell'utente all'interno dello spazio delle presentazioni.	\\
			\textbf{Postcondizione}: l'utente si trova nella cartella che contiene le presentazioni da spostare.	\\
			}
		\subsubsection{UC 1.7.4.3 - Selezione presentazioni da spostare}{
			\label{uc1.7.4.3}
			\textbf{Attori}: utente autenticato. \\
			\textbf{Descrizione}: l'utente seleziona una o più presentazioni esistenti da spostare. \\
			\textbf{Precondizione}: l'utente ha selezionato la cartella in cui sono presenti le presentazioni che intende spostare e si trova all'interno di tale cartella; il sistema mostra all'utente le presentazioni contenute in tale cartella.	\\
			\textbf{Postcondizione}: l'utente ha selezionato una o più presentazioni esistenti da spostare.	\\
			}				
		\subsubsection{UC 1.7.4.4 - Selezione cartella di destinazione}{
			\label{uc1.7.4.4}
			\textbf{Attori}: utente autenticato. \\
			\textbf{Descrizione}: l'utente seleziona dal FileSystem delle presentazioni una cartella tra quelle presenti o ne crea una di nuova. \\
			\textbf{Precondizione}: l'utente ha selezionato delle presentazioni all'interno di un percorso valido nel FileSystem delle presentazioni; il sistema è in attesa che l'utente specifichi il percorso di destinazione; il server mostra le presentazioni e le cartelle dell'utente all'interno dello spazio delle presentazioni.	\\
			\textbf{Postcondizione}: l'utente ha selezionato una cartella di destinazione valida all'interno del FileSystem delle presentazioni.	\\
			}
		\subsubsection{UC 1.7.4.5 - Conferma selezione e spostamento}{
			\label{uc1.7.4.5}
			\textbf{Attori}: utente autenticato. \\
			\textbf{Descrizione}: l'utente conferma la selezione ed il sistema procede allo spostamento. \\
			\textbf{Precondizione}: l'utente ha selezionato le presentazioni da spostare e la destinazione in cui trasferirle; tali presentazioni esistono.	\\
			\textbf{Postcondizione}: il sistema ha spostato le presentazioni selezionate, nella cartella selezionata dall'utente.	\\
			}
	\subsubsection{UC 1.7.5 - Eliminazione presentazioni}{
		\label{uc1.7.5}
		\begin{figure}[H]
			\centering
			\includegraphics[scale=0.75]{\imgs {UC1.7.5}.jpg} %inserire il diagramma UML
			\label{fig:uc1.7.5}
			\caption{Caso d'uso 1.7.5}
		\end{figure}
		\textbf{Attori}: utente autenticato.	\\
		\textbf{Descrizione}: l'utente ha la possibilità di eliminare presentazioni contenute in una cartella, all'interno del proprio spazio riservato alle presentazioni sul server. \\
		\textbf{Precondizione}: l'utente desidera eliminare una o più presentazioni esistenti nel proprio spazio sul server.	\\
		\textbf{Postcondizione}: l'utente ha cancellato le presentazioni che desiderava rimuovere.	\\
		\textbf{Procedura principale}:
		\begin{enumerate}
			\item navigazione nel FileSystem delle presentazioni \hyperref[uc1.7.5.1]{(UC 1.7.5.1)};
			\item selezione cartella di lavoro \hyperref[uc1.7.5.2]{(UC 1.7.5.2)};
			\item selezione presentazioni da eliminare \hyperref[uc1.7.5.3]{(UC 1.7.5.3)};
			\item conferma selezione ed eliminazione \hyperref[uc1.7.5.4]{(UC 1.7.5.4)}.
		\end{enumerate}
		\textbf{Scenari alternativi}: 
		\begin{itemize}
			\item l'operazione viene annullata e non si apportano cambiamenti.
		\end{itemize}
		}
		\subsubsection{UC 1.7.5.1 - Navigazione nel FileSystem delle presentazioni}{
			\label{uc1.7.5.1}
			\textbf{Attori}: utente autenticato. \\
			\textbf{Descrizione}: l'utente deve selezionare la cartella in cui sono inserite le presentazioni che intende rimuovere. Tale cartella si trova nel server, nello spazio riservato alle presentazioni. \\
			\textbf{Precondizione}: l'utente ha iniziato l'operazione di cancellazione di presentazioni; il sistema è in attesa che l'utente specifichi il percorso di origine; il server mostra presentazioni e cartelle dell'utente all'interno dello spazio delle presentazioni.	\\
			\textbf{Postcondizione}: l'utente si trova nella cartella con le presentazioni da eliminare.	\\
			}
		\subsubsection{UC 1.7.5.2 - Selezione cartella di lavoro}{
			\label{uc1.7.5.2}
			\textbf{Attori}: utente autenticato. \\
			\textbf{Descrizione}: l'utente conferma la posizione. \\
			\textbf{Precondizione}: l'utente ha selezionato la cartella in cui sono presenti le presentazioni da eliminare.	\\
			\textbf{Postcondizione}: l'utente si trova all'interno della cartella in cui sono inserite le presentazioni da eliminare. 	\\
			}
		\subsubsection{UC 1.7.5.3 - Selezione presentazioni da eliminare}{
			\label{uc1.7.5.3}
			\textbf{Attori}: utente autenticato. \\
			\textbf{Descrizione}: l'utente seleziona una o più presentazioni esistenti da rimuovere. \\
			\textbf{Precondizione}: l'utente ha selezionato la cartella in cui sono  inserite le presentazioni che intende eliminare e si trova all'interno di tale cartella; il sistema mostra all'utente le presentazioni che lì si trovano.	\\
			\textbf{Postcondizione}: l'utente ha selezionato una o più presentazioni esistenti da eliminare.	\\
			}
		\subsubsection{UC 1.7.5.4 - Conferma selezione ed eliminazione}{
			\label{uc1.7.5.4}
			\textbf{Attori}: utente autenticato. \\
			\textbf{Descrizione}: l'utente conferma la selezione ed il sistema procede all'eliminazione. \\
			\textbf{Precondizione}: l'utente ha selezionato le presentazioni da eliminare; queste esistono.	\\
			\textbf{Postcondizione}: il sistema ha eliminato le presentazioni selezionate dalla cartella scelta.	\\
			}
	\subsubsection{UC 1.7.6 - Spostamento infografiche}{
		\label{uc1.7.6}
		\begin{figure}[H]
			\centering
			\includegraphics[scale=0.75]{\imgs {UC1.7.6}.jpg} %inserire il diagramma UML
			\label{fig:uc1.7.6}
			\caption{Caso d'uso 1.7.6}
		\end{figure}
		\textbf{Attori}: utente autenticato.	\\
		\textbf{Descrizione}: l'utente ha la possibilità di selezionare delle infografiche inserite in una cartella all'interno del proprio spazio riservato alle infografiche nel server e spostarle in un'altra directory, oppure in una nuova. \\
		\textbf{Precondizione}: l'utente desidera spostare una o più infografica esistente nel proprio spazio sul server.	\\
		\textbf{Postcondizione}: l'utente ha spostato le infografiche che desiderava spostare dalla posizione in cui queste si trovavano, in una posizione valida.	\\
		\textbf{Procedura principale}:
		\begin{enumerate}
			\item navigazione nel FileSystem delle infografiche \hyperref[uc1.7.6.1]{(UC 1.7.6.1)};
			\item selezione cartella di lavoro \hyperref[uc1.7.6.2]{(UC 1.7.6.2)};
			\item selezione infografiche da spostare \hyperref[uc1.7.6.3]{(UC 1.7.6.3)};
			\item selezione cartella di destinazione \hyperref[uc1.7.6.4]{(UC 1.7.6.4)};
			\item conferma selezione e spostamento \hyperref[uc1.7.6.5]{(UC 1.7.6.5)}.
		\end{enumerate}
		\textbf{Scenari alternativi}: 
		\begin{itemize}
			\item l'operazione viene annullata e non si apportano cambiamenti.
		\end{itemize}
		}
		\subsubsection{UC 1.7.6.1 - Navigazione nel FileSystem delle infografiche}{
			\label{uc1.7.6.1}
			\textbf{Attori}: utente autenticato. \\
			\textbf{Descrizione}: l'utente deve selezionare la cartella in cui sono inserite le infografiche che intende spostare. Tale cartella si trova nel server, nello spazio riservato alle infografiche. \\
			\textbf{Precondizione}: l'utente ha iniziato l'operazione di spostamento di infografiche; il sistema è in attesa che l'utente specifichi il percorso di origine; il server mostra infografiche e cartelle dell'utente all'interno dello spazio delle infografiche.	\\
			\textbf{Postcondizione}: l'utente si trova nella cartella con le infografiche da spostare.	\\
			}
		\subsubsection{UC 1.7.6.2 - Selezione cartella di lavoro}{
			\label{uc1.7.6.2}
			\textbf{Attori}: utente autenticato. \\
			\textbf{Descrizione}: l'utente deve selezionare la cartella in cui sono inserite le infografiche che intende spostare. Tale cartella si trova nel server, nello spazio riservato alle infografiche. \\
			\textbf{Precondizione}: l'utente ha iniziato l'operazione di spostamento di infografiche; il sistema è in attesa che l'utente specifichi il percorso di origine; il server mostra infografiche e cartelle dell'utente all'interno dello spazio delle infografiche.	\\
			\textbf{Postcondizione}: l'utente si trova nella cartella che contiene le infografiche da spostare.	\\
			}
		\subsubsection{UC 1.7.6.3 - Selezione infografiche da spostare}{
			\label{uc1.7.6.3}
			\textbf{Attori}: utente autenticato. \\
			\textbf{Descrizione}: l'utente seleziona una o più infografiche esistenti da spostare. \\
			\textbf{Precondizione}: l'utente ha selezionato la cartella in cui sono presenti le infografiche che intende spostare e si trova all'interno di tale cartella; il sistema mostra all'utente le infografiche contenute in tale cartella.	\\
			\textbf{Postcondizione}: l'utente ha selezionato una o più infografiche esistenti da spostare.	\\
			}				
		\subsubsection{UC 1.7.6.4 - Selezione cartella di destinazione}{
			\label{uc1.7.6.4}
			\textbf{Attori}: utente autenticato. \\
			\textbf{Descrizione}: l'utente seleziona dal FileSystem delle infografiche una cartella tra quelle presenti o ne crea una di nuova. \\
			\textbf{Precondizione}: l'utente ha selezionato delle infografiche all'interno di un percorso valido nel FileSystem delle infografiche; il sistema è in attesa che l'utente specifichi il percorso di destinazione; il server mostra le infografiche e le cartelle dell'utente all'interno dello spazio delle infografiche.	\\
			\textbf{Postcondizione}: l'utente ha selezionato una cartella di destinazione valida all'interno del FileSystem delle infografiche.	\\
			}
		\subsubsection{UC 1.7.6.5 - Conferma selezione e spostamento}{
			\label{uc1.7.6.5}
			\textbf{Attori}: utente autenticato. \\
			\textbf{Descrizione}: l'utente conferma la selezione ed il sistema procede allo spostamento. \\
			\textbf{Precondizione}: l'utente ha selezionato le infografiche da spostare e la destinazione in cui trasferirle; tali infografiche esistono.	\\
			\textbf{Postcondizione}: il sistema ha spostato le infografiche selezionate, nella cartella selezionata dall'utente.	\\
			}
	\subsubsection{UC 1.7.7 - Eliminazione infografiche}{
		\label{uc1.7.7}
		\begin{figure}[H]
			\centering
			\includegraphics[scale=0.75]{\imgs {UC1.7.7}.jpg} %inserire il diagramma UML
			\label{fig:uc1.7.7}
			\caption{Caso d'uso 1.7.7}
		\end{figure}
		\textbf{Attori}: utente autenticato.	\\
		\textbf{Descrizione}: l'utente ha la possibilità di eliminare infografiche contenute in una cartella, all'interno del proprio spazio riservato alle infografiche sul server. \\
		\textbf{Precondizione}: l'utente desidera eliminare una o più infografica esistente nel proprio spazio sul server.	\\
		\textbf{Postcondizione}: l'utente ha cancellato le infografiche che desiderava rimuovere.	\\
		\textbf{Procedura principale}:
		\begin{enumerate}
			\item navigazione nel FileSystem delle infografiche \hyperref[uc1.7.7.1]{(UC 1.7.7.1)};
			\item selezione cartella di lavoro \hyperref[uc1.7.7.2]{(UC 1.7.7.2)};
			\item selezione infografiche da eliminare \hyperref[uc1.7.7.3]{(UC 1.7.7.3)};
			\item conferma selezione ed eliminazione \hyperref[uc1.7.7.4]{(UC 1.7.7.4)}.
		\end{enumerate}
		\textbf{Scenari alternativi}: 
		\begin{itemize}
			\item l'operazione viene annullata e non si apportano cambiamenti.
		\end{itemize}
		}
		\subsubsection{UC 1.7.7.1 - Navigazione nel FileSystem delle infografiche}{
			\label{uc1.7.7.1}
			\textbf{Attori}: utente autenticato. \\
			\textbf{Descrizione}: l'utente deve selezionare la cartella in cui sono inserite le infografiche che intende rimuovere. Tale cartella si trova nel server, nello spazio riservato alle presentazioni. \\
			\textbf{Precondizione}: l'utente ha iniziato l'operazione di cancellazione di infografiche; il sistema è in attesa che l'utente specifichi il percorso di origine; il server mostra infografiche e cartelle dell'utente all'interno dello spazio delle infografiche.	\\
			\textbf{Postcondizione}: l'utente si trova nella cartella con le infografiche da eliminare.	\\
			}
		\subsubsection{UC 1.7.7.2 - Selezione cartella di lavoro}{
			\label{uc1.7.7.2}
			\textbf{Attori}: utente autenticato. \\
			\textbf{Descrizione}: l'utente conferma la posizione. \\
			\textbf{Precondizione}: l'utente ha selezionato la cartella in cui sono presenti le infografiche da eliminare.	\\
			\textbf{Postcondizione}: l'utente si trova all'interno della cartella in cui sono inserite le infografiche da eliminare. 	\\
			}
		\subsubsection{UC 1.7.7.3 - Selezione infografiche da eliminare}{
			\label{uc1.7.7.3}
			\textbf{Attori}: utente autenticato. \\
			\textbf{Descrizione}: l'utente seleziona una o più infografiche esistenti da rimuovere. \\
			\textbf{Precondizione}: l'utente ha selezionato la cartella in cui sono  inserite le infografiche che intende eliminare e si trova all'interno di tale cartella; il sistema mostra all'utente le presentazioni che lì si trovano.	\\
			\textbf{Postcondizione}: l'utente ha selezionato una o più infografica esistente da eliminare.	\\
			}
		\subsubsection{UC 1.7.7.4 - Conferma selezione ed eliminazione}{
			\label{uc1.7.7.4}
			\textbf{Attori}: utente autenticato. \\
			\textbf{Descrizione}: l'utente conferma la selezione ed il sistema procede all'eliminazione. \\
			\textbf{Precondizione}: l'utente ha selezionato le infografiche da eliminare; queste esistono.	\\
			\textbf{Postcondizione}: il sistema ha eliminato le infografiche selezionate dalla cartella scelta.	\\
			}
