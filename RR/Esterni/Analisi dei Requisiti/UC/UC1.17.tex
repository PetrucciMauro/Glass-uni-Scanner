\subsection{UC 1.17 - Gestione infografiche}{
	\label{uc1.17}
	\begin{figure}[H]
		\centering
		\includegraphics[scale=0.6]{\imgs {UC1.17}.jpg} %inserire il diagramma UML
		\label{fig:uc1.17}
		\caption{Caso d'uso 1.17}
	\end{figure}
	\textbf{Attori}: utente desktop. \\
	\textbf{Descrizione}: l'utente è in grado di prendere gli elementi di una presentazione slide e di inserirli in un unico documento stampabile e quindi lineare. \\
	\textbf{Precondizione}: il programma è acceso e funzionante.	\\
	\textbf{Postcondizione}: l'utente ha modificato un'infografica che può aver salvato in locale e online.	\\
	\textbf{Procedura principale}:
	\begin{enumerate}
		\item Selezione della presentazione di cui produrre l’infografica \hyperref[uc1.17.1]{(UC 1.17.1)};
		\item Selezione di un template di infografica \hyperref[uc1.17.2]{(UC 1.17.2)};
		\item Selezione di un elemento dell’infografica \hyperref[uc1.17.3]{(UC 1.17.3)};
		\item Modifica e rimozione di un elemento grafico o di un elemento di testo \hyperref[uc1.17.4]{(UC 1.17.4)};
		\item Salvataggio dell'infografica \hyperref[uc1.17.5]{(UC 1.17.5)}.
		\item Annulla/ripristina ultima operazione \hyperref[uc1.17.6]{(UC 1.17.6)}.
	\end{enumerate}
	}
\subsubsection{UC 1.17.1 - Selezione della presentazione}{
	\label{uc1.17.1}
	\textbf{Attori}: utente desktop. \\
	\textbf{Descrizione}: l'utente è in grado di selezionare la presentazione di cui produrre l'infografica per potervi attingere elementi direttamente. \\
	\textbf{Precondizione}: il programma è acceso e funzionante.	\\
	\textbf{Postcondizione}: nell'editor viene caricata una presentazione sui quali oggetti si può andare a interagire.
	}
\subsubsection{UC 1.17.2 - Selezione di un template}{
	\label{uc1.17.2}
	\textbf{Attori}: utente desktop. \\
	\textbf{Descrizione}: permettere all'utente di selezionare il template che preferisce. \\
	\textbf{Precondizione}: il programma è attivo e funzionante.\\
	\textbf{Postcondizione}: l'utente ha caricato nell'editor un template che potrà modificare per creare la sua infografica.
	}
\subsubsection{UC 1.17.3 - Selezione di un oggetto dell'infografica}{
	\label{uc1.17.3}
	\textbf{Attori}: utente desktop. \\
	\textbf{Descrizione}: l'utente è in grado di selezionare elementi specifici del template dell'infografica o di quelli da lui inseriti. \\
	\textbf{Precondizione}: il programma è acceso e funzionante, un template di infografica è stato caricato nell'editor.	\\
	\textbf{Postcondizione}: l'oggetto è stato selezionato.
	}
\subsubsection{UC 1.17.4 - Modifica infografica}{
	\label{uc1.17.4}
	\begin{figure}[H]
		\centering
		\includegraphics[scale=0.75]{\imgs {UC1.17.4}.jpg} %inserire il diagramma UML
		\label{fig:uc1.17.4}
		\caption{Caso d'uso 1.17.4}
	\end{figure}
	\textbf{Attori}: utente desktop. \\
	\textbf{Descrizione}: l'utente è in grado di selezionare elementi specifici del template dell'infografica o di quelli da lui inseriti. \\
	\textbf{Precondizione}: il programma è acceso e funzionante, è attiva un'infografica.	\\
	\textbf{Postcondizione}: l'infografica attiva è stata modificata.	\\
	\textbf{Procedura principale}:
	\begin{enumerate}
		\item Modifica di un oggetto dell'infografica \hyperref[uc1.17.4.1]{(UC 1.17.4.1)};
		\item Rimozione dello sfondo \hyperref[uc1.17.4.2]{(UC 1.17.4.2)};
		\item Inserimento di uno sfondo \hyperref[uc1.17.4.3]{(UC 1.17.4.3)};
		\item Inserimento di un elemento grafico \hyperref[uc1.17.4.4]{(UC 1.17.4.4)};
		\item Inserimento di un elemento di testo \hyperref[uc1.17.4.5]{(UC 1.17.4.5)};
		\item Inserimento slide \hyperref[uc1.17.4.6]{(UC 1.17.4.6)};
		\item Eliminazione di un elemento \hyperref[uc1.17.4.7]{(UC 1.17.4.7)};
	\end{enumerate}
	\textbf{Scenari alternativi}: 
	\begin{itemize}
		\item Solo nel caso in cui venga effettuata una modifica l'utente desktop può scegliere di annullare l'ultima modifica eseguita \hyperref[uc1.17.6]{(UC 1.17.6)}.
	\end{itemize}
}
\subsubsection{UC 1.17.4.1 - Modifica di un oggetto dell'infografica}{
	\label{uc1.17.4.1}
	\begin{figure}[H]
		\centering
		\includegraphics[scale=0.75]{\imgs {UC1.17.4.1}.jpg} %inserire il diagramma UML
		\label{fig:uc1.17.4.1}
		\caption{Caso d'uso 1.17.4.1}
	\end{figure}
	\textbf{Attori}: utente desktop. \\
	\textbf{Descrizione}: l'utente è in grado di selezionare elementi specifici del template dell'infografica o di quelli da lui inseriti. \\
	\textbf{Precondizione}: il programma è acceso e funzionante, un oggetto della presentazione è stato selezionato.	\\
	\textbf{Postcondizione}: l'oggetto selezionato è stato modificato.	\\
	\textbf{Procedura principale}:
	\begin{enumerate}
		\item Se l'oggetto è un elemento grafico è possibile ingrandirne o ridurne le dimensioni \hyperref[uc1.17.4.1.1]{(UC 1.17.4.1.1)};
		\item Se l'oggetto è un elemento testuale è possibile effettuare diverse modifiche \hyperref[uc1.17.4.1.2]{(UC 1.17.4.1.2)};
		\item In entrambi i casi è possibile cambiare la posizione dell'oggetto \hyperref[uc1.17.4.1.3]{(UC 1.17.4.1.3)}.
	\end{enumerate}
	}
\subsubsection{UC 1.17.4.1.1 - Modifica dimensioni di un oggetto grafico}{
	\label{uc1.17.4.1.1}
	\textbf{Attori}: utente desktop. \\
	\textbf{Descrizione}: l'utente è in grado di aumentare e ridurre le dimensioni di un oggetto grafico. \\
	\textbf{Precondizione}: l'oggetto grafico è stato selezionato.\\
	\textbf{Postcondizione}: l'oggetto ha cambiato le sue dimensioni.
	}
\subsubsection{UC 1.17.4.1.2 - Modifica di elemento testuale}{
	\label{uc1.17.4.1.2}
	\begin{figure}[H]
		\centering
		\includegraphics[scale=0.75]{\imgs {UC1.17.4.1.2}.jpg} %inserire il diagramma UML
		\label{fig:uc1.17.4.1.2}
		\caption{Caso d'uso 1.17.4.1.2}
	\end{figure}
	\textbf{Attori}: utente desktop. \\
	\textbf{Descrizione}: l'utente è in grado di modificare un elemento testuale. \\
	\textbf{Precondizione}: un oggetto testuale della presentazione è stato selezionato.	\\
	\textbf{Postcondizione}: l'elemento testuale è stato modificato.	\\
	\textbf{Procedura principale}:
	\begin{enumerate}
		\item Cambiare il carattere \hyperref[uc1.17.4.1.2.1]{(UC 1.17.4.1.2.1)};
		\item Modificare le dimensioni del carattere \hyperref[uc1.17.4.1.2.2]{(UC 1.17.4.1.2.2)};
		\item Una formattazione (corsivo, grassetto, sottolineato) \hyperref[uc1.17.4.1.2.3]{(UC 1.17.4.1.2.3)};
		\item Cambiare colore del carattere \hyperref[uc1.17.4.1.2.4]{(UC 1.17.4.1.2.4)};
		\item Cambiare il colore di sfondo \hyperref[uc1.17.4.1.2.5]{(UC 1.17.4.1.2.5)}.
	\end{enumerate}
	\textbf{Scenari alternativi}:
	\begin{itemize}
		\item E' sempre possibile annullare l'ultima modifica effettuata ed eventualmente ripristinarla \hyperref[uc1.17.6]{(UC 1.17.6)}.
	\end{itemize}
	}
\subsubsection{UC 1.17.4.1.2.1 - Modifica del carattere}{
	\label{uc1.17.4.1.2.1}
	\textbf{Attori}: utente desktop. \\
	\textbf{Descrizione}: l'utente è in grado di cambiare il carattere delle lettere dell’elemento testuale. \\
	\textbf{Precondizione}: l'oggetto testuale è stato selezionato e si trova ora in modalità modifica.\\
	\textbf{Postcondizione}: le lettere del testo hanno cambiato carattere.	
	}
\subsubsection{UC 1.17.4.1.2.2 - Modifica delle dimensioni del carattere}{
	\label{uc1.17.4.1.2.2}
	\textbf{Attori}: utente desktop. \\
	\textbf{Descrizione}: l'utente è in grado di cambiare le dimensioni delle lettere dell’elemento testuale. \\
	\textbf{Precondizione}: l'oggetto testuale è stato selezionato e si trova ora in modalità modifica.\\
	\textbf{Postcondizione}: le lettere del testo ha cambiato dimensione.
	}
\subsubsection{UC 1.17.4.1.2.3 - Modifica della formattazione}{
	\label{uc1.17.4.1.2.3}
	\textbf{Attori}: utente desktop. \\
	\textbf{Descrizione}: l'utente è in grado di cambiare la formattazione dell’elemento testuale. \\
	\textbf{Precondizione}: l'oggetto testuale è stato selezionato e si trova ora in modalità modifica.\\
	\textbf{Postcondizione}: le lettere del testo hanno cambiato formato (corsivo, grassetto, sottolineato).
	}
\subsubsection{UC 1.17.4.1.2.4 - Modifica del colore}{
	\label{uc1.17.4.1.2.4}
	\textbf{Attori}: utente desktop. \\
	\textbf{Descrizione}: l'utente è in grado di cambiare il colore delle lettere dell’elemento testuale. \\
	\textbf{Precondizione}: l'oggetto testuale è stato selezionato e si trova ora in modalità modifica.\\
	\textbf{Postcondizione}: le lettere del testo hanno cambiato colore.
	}
\subsubsection{UC 1.17.4.1.2.5 - Modifica sfondo}{
	\label{uc1.17.4.1.2.5}
	\textbf{Attori}: utente desktop. \\
	\textbf{Descrizione}: l'utente è in grado di cambiare il colore dello sfondo dell’elemento testuale (evidenziatura). \\
	\textbf{Precondizione}: l'oggetto testuale è stato selezionato e si trova ora in modalità modifica.\\
	\textbf{Postcondizione}: le lettere del testo sono state evidenziate.
	}
\subsubsection{UC 1.17.4.1.3 - Modifica della posizione dell'oggetto}{
	\label{uc1.17.4.1.3}
	\textbf{Attori}: utente desktop. \\
	\textbf{Descrizione}: l'utente è in grado di cambiare la posizione di un oggetto. \\
	\textbf{Precondizione}: l'oggetto grafico è stato selezionato.\\
	\textbf{Postcondizione}: l'oggetto è stato spostato all’interno dell’infografica.
	}
\subsubsection{UC 1.17.4.2 - Rimozione dello sfondo}{
	\label{uc1.17.4.2}
	\textbf{Attori}: utente desktop. \\
	\textbf{Descrizione}: l'utente è in grado di rimuovere lo sfondo dell’infografica. \\
	\textbf{Precondizione}: il programma è acceso e funzionante, è stato caricato un template o un'infografica personale dell'utente.	\\
	\textbf{Postcondizione}: lo sfondo dell'infografica è stato eliminato.
	}
\subsubsection{UC 1.17.4.3 - Inserimento di uno sfondo}{
	\label{uc1.17.4.3}
	\textbf{Attori}: utente desktop. \\
	\textbf{Descrizione}: l'utente è in grado di scegliere uno sfondo per l'infografica. \\
	\textbf{Precondizione}: il programma è acceso e funzionante, è stato caricato un template o un'infografica personale dell'utente.	\\
	\textbf{Postcondizione}: lo sfondo dell'infografica è stato cambiato.
	}
\subsubsection{UC 1.17.4.4 - Inserimento di un elemento grafico}{
	\label{uc1.17.4.4}
	\textbf{Attori}: utente desktop. \\
	\textbf{Descrizione}: l'utente è in grado di inserire un elemento grafico. \\
	\textbf{Precondizione}: il programma è acceso e funzionante, è stato caricato un template o un'infografica personale dell'utente.	\\
	\textbf{Postcondizione}: è stato inserito un oggetto precedentemente assente nell'infografica.\\
	\textbf{Procedura principale}:
	\begin{enumerate}
		\item Quando un elemento grafico viene inserito si entra automaticamente in modalità modifica \hyperref[uc1.17.4.1]{(UC 1.17.4.1)}.
	\end{enumerate}
	}
\subsubsection{UC 1.17.4.5 - Inserimento di un elemento testuale}{
	\label{uc1.17.4.5}
	\textbf{Attori}: utente desktop. \\
	\textbf{Descrizione}: l'utente è in grado di inserire un elemento testuale nell'infografica. \\
	\textbf{Precondizione}: il programma è acceso e funzionante, è stato caricato un template o un'infografica personale dell'utente.	\\
	\textbf{Postcondizione}: è stato inserito un oggetto testuale precedentemente assente nell'infografica.	\\
	\textbf{Procedura principale}:
	\begin{enumerate}
		\item Quando viene inserito un elemento testuale questo entra automaticamente in modalità modifica \hyperref[uc1.17.4.1.2]{(UC 1.17.4.1.2)}.
	\end{enumerate}
	}
\subsubsection{UC 1.17.4.6 - Inserimento di una slide}{
	\label{uc1.17.4.6}
	\textbf{Attori}: utente desktop. \\
	\textbf{Descrizione}: l'utente è in grado di inserire un elemento testuale nell'infografica. \\
	\textbf{Precondizione}: il programma è acceso e funzionante, è stato caricato un template o un'infografica personale dell'utente, è stata caricata una presentazione di cui l'utente vuole fare l'infografica.	\\
	\textbf{Postcondizione}: nell'infografica è stata inserita come immagine vettoriale una slide precedentemente assente.
	}
\subsubsection{UC 1.17.4.7 - Eliminazione di un elemento}{
	\label{uc1.17.4.7}
	\textbf{Attori}: utente desktop. \\
	\textbf{Descrizione}: l'utente è in grado di eliminare un oggetto dell'infografica. \\
	\textbf{Precondizione}: il programma è acceso e funzionante, è stato caricato un template o un'infografica personale dell'utente, è stato selezionato un oggetto dell'infografica.	\\
	\textbf{Postcondizione}: nell'infografica è stato rimosso un oggetto.
	}
\subsubsection{UC 1.17.5 - Salvataggio infografica}{
	\label{uc1.17.5}
	\textbf{Attori}: utente desktop. \\
	\textbf{Descrizione}: permettere all’utente di salvare un’infografica. \\
	\textbf{Precondizione}: l'utente ha caricato un template.	\\
	\textbf{Postcondizione}: l'utente ha salvato un’infografica nel suo spazio personale.
	}
\subsubsection{UC 1.17.6 - Annulla/Ripristina}{
	\label{uc1.17.6}
	\textbf{Attori}: utente desktop. \\
	\textbf{Descrizione}: permettere all’utente di annullare o ripristinare l'ultima modifica. \\
	\textbf{Precondizione}: è stata effettuata una modifica all'infografica.	\\
	\textbf{Postcondizione}: l'infografica è stata riportata alla situazione antecedente all'ultima operazione.
	}
\subsubsection{UC 1.17.7 - Esportazione infografica}{
	\label{uc1.17.7}
	\textbf{Attori}: utente desktop. \\
	\textbf{Descrizione}: permettere all’utente di esportare un’infografica. \\
	\textbf{Precondizione}: l'utente ha caricato un template.	\\
	\textbf{Postcondizione}: l'utente ha esportato l'infografica sottoforma di file immagine nel proprio dispositivo.
	}