\subsection{UC 1.9 - Gestione template infografiche}{
	\label{uc1.9}
	\begin{figure}[H]
		\centering
		\includegraphics[scale=0.50]{\imgs {UC1.9}.jpg} %inserire il diagramma UML
		\label{fig:uc1.9}
		\caption{Caso d'uso 1.9}
	\end{figure}
	\textbf{Attori}: amministratore di sistema. \\
	\textbf{Descrizione}: l'amministratore ha effettuato il login e il programma è correttamente funzionante. L'amministratore ha la possibilità di inserire nuovi template di infografica, nuovi oggetti grafici per comporre i template ed eliminare template presenti nel server.\\
	\textbf{Precondizione}: il programma è in funzione e l'amministratore ha effettuato il login.\\
	\textbf{Postcondizione}: le operazioni svolte dall'amministratore sono state eseguite con successo.\\
	\textbf{Procedura principale}:
	\begin{enumerate}
		\item caricamento di un template \hyperref[uc1.9.1]{(UC1.9.1)};
		\item caricamento di un elemento \hyperref[uc1.9.3]{(UC1.9.3)};
		\item eliminazione di un template \hyperref[uc1.9.5]{(UC1.9.5)};
	\end{enumerate}
	\textbf{Scenari alternativi}:
	\begin{itemize}
		\item l'amministratore può ripristinare un template eliminato \hyperref[uc1.9.5]{(UC1.9.5)};
		\item l'amministratore può ripristinare le modifiche eseguite;
		\item visualizzazione di un errore nel caso vengano inseriti un oggetto o un template già presenti nel database.
	\end{itemize}
	}
\subsection{UC1.9.1 - Caricamento di un template}{
	\label{uc1.9.1}
	\begin{figure}[H]
		\centering
		\includegraphics[scale=0.75]{\imgs {UC1.9.1}.jpg} %inserire il diagramma UML
		\label{fig:uc1.9.1}
		\caption{Caso d'uso 1.9.1}
	\end{figure}
	\textbf{Attori}: amministratore di sistema. \\
	\textbf{Descrizione}: l'amministratore esegue il caricamento di un nuovo template nel database. \\
	\textbf{Precondizione}: il sistema  è funzionante e l'amministratore ha effettuato il login.	\\
	\textbf{Postcondizione}: il template è stato caricato correttamente nel database.	\\
	\textbf{Procedura principale}:
	\begin{enumerate}
		\item l'amministratore naviga nel file system per trovare il template da inserire \hyperref[uc1.9.1.1]{(UC 1.9.1.1)};
		\item l'amministratore inserisce il template nel database \hyperref[uc1.9.1.2]{(UC 1.9.1.2)}.
	\end{enumerate}
	\textbf{Scenari alternativi}:
	\begin{itemize}
		\item il template è già presente nel database.
	\end{itemize}
	}
\subsection{UC 1.9.1.1 - Navigazione nel file system}{
	\label{uc1.9.1.1}
	\textbf{Attori}: amministratore di sistema. \\
	\textbf{Descrizione}: l'amministratore può navigare nel filesystem per selezionare la posizione da cui caricare il file. \\
	\textbf{Precondizione}: l'amministratore ha intenzione di caricare un nuovo template.	\\
	\textbf{Postcondizione}: lo stato dell'attività di caricamento è stato completato.	\\
	}
\subsection{UC 1.9.1.2 - Selezione file}{
	\label{uc1.9.1.2}
	\textbf{Attori}: amministratore di sistema. \\
	\textbf{Descrizione}: l'amministratore seleziona il file da caricare nel database. \\
	\textbf{Precondizione}: l'amministratore ha inizializzato l'attività di caricamento di un file.	\\
	\textbf{Postcondizione}: il file è stato caricato.	\\
	}
\subsection{UC 1.9.3 - Caricamento di un elemento}{
	\label{uc1.9.3}
	\begin{figure}[H]
		\centering
		\includegraphics[scale=0.75]{\imgs {UC1.9.3}.jpg} %inserire il diagramma UML
		\label{fig:uc1.9.3}
		\caption{Caso d'uso 1.9.3}
	\end{figure}
	\textbf{Attori}: amministratore di sistema. \\
	\textbf{Descrizione}: l'amministratore esegue il caricamento di un nuovo elemento nel database. \\
	\textbf{Precondizione}: il sistema  è funzionante e l'amministratore ha effettuato il login.	\\
	\textbf{Postcondizione}: l'elemento grafico è stato caricato correttamente nel database.	\\
	\textbf{Procedura principale}:
	\begin{enumerate}
		\item l'amministratore naviga nel file system per trovare l'elemento grafico da inserire \hyperref[uc1.9.3.1]{(UC 1.9.3.1)};
		\item l'amministratore inserisce l'elemento grafico nel database \hyperref[uc1.9.3.2]{(UC 1.9.3.2)}.
	\end{enumerate}
	\textbf{Scenari alternativi}:
	\begin{itemize}
		\item il file grafico è già presente nel database.
	\end{itemize}
	}
\subsection{UC 1.9.3.1 - Navigazione nel file system}{
	\label{uc1.9.3.1}
	\textbf{Attori}: amministratore di sistema. \\
	\textbf{Descrizione}: l'amministratore può navigare nel filesystem per selezionare la posizione da cui caricare il file. \\
	\textbf{Precondizione}: l'amministratore ha inizializzato l'attività di caricamento di un file.	\\
	\textbf{Postcondizione}: lo stato dell'attività di caricamento è stato completato.	\\
	}
\subsection{UC 1.9.3.2 - Selezione file}{
	\label{uc1.9.3.2}
	\textbf{Attori}: amministratore di sistema. \\
	\textbf{Descrizione}: l'amministratore seleziona il file da caricare nel database. \\
	\textbf{Precondizione}: l'amministratore ha inizializzato l'attività di caricamento di un file.	\\
	\textbf{Postcondizione}: il file è stato caricato.	\\
	}
\subsection{UC 1.9.4- Eliminazione di un template}{
	\label{uc1.9.4}
	\textbf{Attori}: amministratore di sistema. \\
	\textbf{Descrizione}: l'amministratore elimina uno dei template. \\
	\textbf{Precondizione}: il sistema è funzionante e l'amministratore decide di eliminare un template.	\\
	\textbf{Postcondizione}: il template è stato eliminato dal sistema.	\\
	\textbf{Procedura principale}:
	\begin{enumerate}
		\item l’amministratore elimina il template selezionato.
	\end{enumerate}
	\textbf{Scenari alternativi}:
	\begin{itemize}
		\item l'amministratore decide di ripristinare il template eliminato \hyperref[uc1.9.5.3]{(UC 1.9.5.3)}.
	\end{itemize}
	}
\subsection{UC 1.9.5 - Annulla/Ripristina}{
	\label{uc1.9.5}
	\begin{figure}[H]
		\centering
		\includegraphics[scale=0.75]{\imgs {UC1.9.5}.jpg} %inserire il diagramma UML
		\label{fig:uc1.9.5}
		\caption{Caso d'uso 1.9.5}
	\end{figure}
	\textbf{Attori}: amministratore di sistema. \\
	\textbf{Descrizione}: l'amministratore ha deciso di annullare una o più modifiche applicate o di ripristinare una modifica annullata. \\
	\textbf{Precondizione}: lo storico delle modifiche non è vuoto.	\\
	\textbf{Postcondizione}: la scena ha il livello di modifiche desiderato.	\\
	\textbf{Procedura principale}:
	\begin{enumerate}
		\item annullamento comando \hyperref[uc1.9.5.1]{(UC 1.9.5.1)};
		\item ripristino del comando annullato \hyperref[uc1.9.5.2]{(UC 1.9.5.2)};
		\item ripristino del template eliminato \hyperref[uc1.9.5.3]{(UC 1.9.5.3)}
	\end{enumerate}
	}
\subsection{UC 1.9.5.1 - Annullamento comando}{
	\label{uc1.9.5.1}
	\textbf{Attori}: amministratore di sistema. \\
	\textbf{Descrizione}: l'amministratore può annullare l'ultimo comando di modifica del template presente nello storico. \\
	\textbf{Precondizione}: il sistema ha registrato almeno un comando di modifica del template eseguito dall'amministratore.	\\
	\textbf{Postcondizione}: il sistema ha ripristinato lo stato precedente all'ultimo comando di modifica del template dall'amministratore, e ha memorizzato nello storico degli	annullamenti il comando annullato.	\\
	}
\subsection{UC 1.9.5.2 - Ripristino del comando annullato}{
	\label{uc1.9.5.2}
	\textbf{Attori}: amministratore di sistema. \\
	\textbf{Descrizione}: l'amministratore può ripristinare l'ultimo comando presente nello	storico dei comandi annullati. \\
	\textbf{Precondizione}: il sistema ha annullato l'ultimo comando di modifica del template
	eseguito dall'amministratore.	\\
	\textbf{Postcondizione}: il sistema ha eseguito nuovamente l'ultimo comando di modifica	del template presente nello storico dei comandi annullati, e ha rimosso il comando eseguito nuovamente dallo storico.	\\
	}
\subsection{UC 1.9.5.3 - Ripristino del template eliminato}{
	\label{uc1.9.5.3}
	\textbf{Attori}: amministratore di sistema. \\
	\textbf{Descrizione}: l'amministratore può ripristinare il template appena eliminato. \\
	\textbf{Precondizione}: il sistema ha annullato il comando di cancellazione del template eseguito dall'amministratore.	\\
	\textbf{Postcondizione}: il sistema ha ripristinato lo stato precedente alla cancellazione del template da parte dell'amministratore.	\\
	}