\section{Descrizione generale}{
Il sistema si pone come obbiettivo quello di permettere la creazione di presentazioni efficaci dal punto di vista dello story-telling anche ad utenti non esperti nel farlo.
L'utente potrà definire più percorsi di presentazione con la possibilità di avere più di una visualizzazione possibile per una certa presentazione, il percorso di frame che sarà visualizzato sarà scelto dal presentatore in fase di visualizzazione.\\
Il sistema permetterà di creare e modificare presentazioni se connessi alla rete mentre l'utente potrà eseguire una sua presentazione anche offline a patto di averla precedentemente scaricata dal server.
Il sistema sarà implementato utilizzando tecnologie web che lo renderanno altamente portabile.

\subsection{Funzioni del prodotto}{
	Il sistema permetterà di :
	\begin{itemize}
		\item Creare una nuova presentazione da dispositivo desktop;
		\item Modificare una presentazione da dispositivio desktop;
		\item Modificare parzialmente la presentazione da dispositivo mobile;
		\item Eseguire una presentazione salvata su un server;
		\item Eseguire una presentazione locale;
		\item Creare infografiche a partire da una presentazione;
		\item Modificare infografiche create.
	\end{itemize}
}
\subsection{Vincoli generali}{
	Il sistema dovrà essere sviluppato usando le tecnologie web, in particolare i linguaggi:
	\begin{itemize}
		\item HTML5;
		\item CSS3;
		\item Javascript.
	\end{itemize}
	
	\noindent
	Il sistema dovrà funzionare correttamente sui principali broswer scaricabili dai rispettivi siti ufficiali.
	}
}