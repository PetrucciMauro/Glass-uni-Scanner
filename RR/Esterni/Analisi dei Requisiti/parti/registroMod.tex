\Large{\textbf{Registro delle modifiche}}\\
\normalsize

%	Ordine di inserimento: dall'ultima versione alla prima
\renewcommand*{\arraystretch}{1.4}
\begin{longtable} [c]{|>{\centering\arraybackslash}m{2cm} | >{\centering\arraybackslash}m{4cm} | >{\centering\arraybackslash}m{3cm} | >{\centering\arraybackslash}m{6cm} |}
		\caption{Versionamento del documento \label{tab:versionamento}}\\
		 \hline
		 \textbf{Versione} & \textbf{Autore} & \textbf{Data} & \textbf{Descrizione}\\
		 \hline
		 \endfirsthead
		 \hline
		 \textbf{Versione} & \textbf{Autore} & \textbf{Data} & \textbf{Descrizione}\\
		 \hline
		\endhead
		 \hline
		 \endfoot
		 \hline
		 \endlastfoot
		 0.5 & \BM & 10 aprile 2015 & Ordinamento del file e formattazione del testo\\
		 0.1 & \TP, \GP, \FM, \VG, \PM, \BM & 2 marzo 2015 & Prima stesura del documento. Analisi da parte di ogni componente del gruppo di una parte dei casi d'uso trovati\\
\end{longtable}

\newpage
\Large{\textbf{Storico }}\\
\normalsize \\

%	Per mettere più tabelle di storico basta copiare e incollare la seguente porzione di codice e modificarla in base ai dati nuovi
\textbf{pre-RR}
\label{tabVers1}
\begin{table}[h]
	\begin{tabular}{p{0.2\textwidth} p{0.7\textwidth}}
		\toprule \textbf{Versione 1.0}	&	\textbf{Nominativo}\\
		\midrule Redazione	& Tutti i componenti del gruppo\\
		\midrule Verifica & \PM, \BM\\
		\midrule Approvazione	&	\\
		\bottomrule
	\end{tabular}
	\caption{Storico ruoli pre-RR}
\end{table}