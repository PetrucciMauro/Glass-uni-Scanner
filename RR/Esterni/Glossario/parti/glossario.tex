\section{Glossario}{ 
\hfill\Huge{\textbf{A}} \\ 
\normalsize 
\begin{longtabu} to \textwidth{p{0.1\textwidth} p{0.8\textwidth}} 
\toprule \\ 
\textbf{Account} : & L'insieme di funzionalit?, strumenti e contenuti attribuiti ad un nome utente all'interno del programma. \\ 
 \\ 
\textbf{Analisi Dinamica} : & Analisi del software svolta eseguendo il programma. \\ 
 \\ 
\textbf{Analisi Statica} : & Analisi del software svolta senza eseguire il programma. \\ 
 \\ 
\end{longtabu} 
\newpage 
\hfill\Huge{\textbf{B}} \\ 
\normalsize 
\begin{longtabu} to \textwidth{p{0.1\textwidth} p{0.8\textwidth}} 
\toprule \\ 
\textbf{Browser} : & Un browser ? un programma che consente di usufruire dei servizi di connettivit? in Rete e di navigare sul World Wide Web. \\ 
 \\ 
\end{longtabu} 
\newpage 
\hfill\Huge{\textbf{C}} \\ 
\normalsize 
\begin{longtabu} to \textwidth{p{0.1\textwidth} p{0.8\textwidth}} 
\toprule \\ 
\textbf{CSS} : & (Cascading Style Sheets, in italiano fogli di stile), linguaggio usato per descrivere l?aspetto e la formattazione di un documento scritto con un linguaggio di marcatura. \\ 
 \\ 
\textbf{Ciclo di vita} : & Scomposizione dell'attivit? da parte della metodologia di sviluppo di realizzazione di prodotti software in sottoattivit? fra loro coordinate, il cui risultato finale ? il prodotto stesso e tutta la documentazione ad esso associata. \\ 
 \\ 
\textbf{Codice} : & Testo di un algoritmo di un programma scritto in un linguaggio di programmazione da parte di un programmatore in fase di programmazione. \\ 
 \\ 
\textbf{Codifica} : & Stesura di un programma in un certo linguaggio di programmazione, effettuata sulla base di un flusso logico e di algoritmi definiti. \\ 
 \\ 
\textbf{Commit} : & Nel contesto di scienza di computer e gestione di dati, commit si riferisce all?idea di rendere permanenti i cambiamenti effettuati nel repository locale. \\ 
 \\ 
\textbf{Commitente} : & Colui che commissiona, per conto di un proponente, un capitolato d?appalto ad un particolare gruppo. \\ 
 \\ 
\end{longtabu} 
\newpage 
\hfill\Huge{\textbf{D}} \\ 
\normalsize 
\begin{longtabu} to \textwidth{p{0.1\textwidth} p{0.8\textwidth}} 
\toprule \\ 
\textbf{Desktop} : & Si intende tipologia di computer contraddistinto dall?essere general purpose, monoutente, destinato ad un utilizzo non in mobilit? e principalmente produttivo e di dimensioni tali per cui l?installazione in una scrivania risulta la pi? appropriata per un comodo utilizzo. \\ 
 \\ 
\textbf{Dispositivo Mobile} : & Dispositivo portatile che permette di ricevere, elaborare ed esportare dati senza usare una connessione Internet cablata. Di norma si usa per indicare sia smartphone che tablet. \\ 
 \\ 
\end{longtabu} 
\newpage 
\hfill\Huge{\textbf{E}} \\ 
\normalsize 
\begin{longtabu} to \textwidth{p{0.1\textwidth} p{0.8\textwidth}} 
\toprule \\ 
\textbf{Estensibilit?} : & va definita seriamente \\ 
 \\ 
\textbf{Estensione} : & Breve sequenza di caratteri alfanumerici (tipicamente tre) , posta alla fine del nome di un file e separata dalla parte precedente con un punto. Attraverso l?estensione del file il sistema operativo riesce a distinguere il tipo di contenuto (testo, musica, immagine...) e il formato utilizzato e aprirlo, di conseguenza, con la corrispondente applicazione. \\ 
 \\ 
\end{longtabu} 
\newpage 
\hfill\Huge{\textbf{F}} \\ 
\normalsize 
\begin{longtabu} to \textwidth{p{0.1\textwidth} p{0.8\textwidth}} 
\toprule \\ 
\textbf{File} : & Contenitore di informazioni in forma digitale che si basa su un sistema di archiviazione durevole, ? cio? a disposizione di altri programmi anche quando il programma che l?ha creato termina la sua esecuzione. \\ 
 \\ 
\textbf{Foglio di calcolo} : & Programma che permette di effettuare calcoli, elaborare dati e tracciare efficaci rappresentazioni grafiche; solitamente agisce tramite l?uso di tabelle. \\ 
 \\ 
\textbf{Framework} : & Nella produzione del software, il framework ? una struttura di supporto su cui un software pu? essere organizzato e progettato. \\ 
 \\ 
\textbf{Funzione} : & Particolare costrutto sintattico, utilizzato all?interno del codice di un programma, che permette di raggruppare una sequenza di istruzioni in un unico blocco espletando cos? una determinata e in generale pi? complessa operazione, azione o elaborazione sui dati del programma stesso. \\ 
 \\ 
\end{longtabu} 
\newpage 
\hfill\Huge{\textbf{G}} \\ 
\normalsize 
\begin{longtabu} to \textwidth{p{0.1\textwidth} p{0.8\textwidth}} 
\toprule \\ 
\textbf{Gantt} : & Strumento di supporto alla gestione de progetti tramite diagramma. \\ 
 \\ 
\textbf{GitHub} : & Servizio web di hosting per lo sviluppo di progetti software, che usa il sistema di controllo di versione Git . \\ 
 \\ 
\end{longtabu} 
\newpage 
\hfill\Huge{\textbf{I}} \\ 
\normalsize 
\begin{longtabu} to \textwidth{p{0.1\textwidth} p{0.8\textwidth}} 
\toprule \\ 
\textbf{ISO} : & ISO (International Organization for Standardization)? la pi? importante organizzazione a livello mondiale per la definizione di norme tecniche. \\ 
 \\ 
\textbf{Indirizzo} : & v. percorso. Indirizzo web: Nome che contiene, in forma esplicita, informazioni sulla posizione di una pagina internet o di un file all'interno del web. \\ 
 \\ 
\textbf{Iterare} : & Ripetere. \\ 
 \\ 
\textbf{Iterazione} : & Ripetizioni. \\ 
 \\ 
\end{longtabu} 
\newpage 
\hfill\Huge{\textbf{J}} \\ 
\normalsize 
\begin{longtabu} to \textwidth{p{0.1\textwidth} p{0.8\textwidth}} 
\toprule \\ 
\textbf{JavaScript} : & In informatica JavaScript ? un linguaggio di scripting orientato agli oggetti e agli eventi. \\ 
 \\ 
\end{longtabu} 
\newpage 
\hfill\Huge{\textbf{L}} \\ 
\normalsize 
\begin{longtabu} to \textwidth{p{0.1\textwidth} p{0.8\textwidth}} 
\toprule \\ 
\textbf{Link} : & Rinvio da un'unit? informativa su supporto digitale ad un'altra. ? ci? che caratterizza la non linearit? dell'informazione propria di un ipertesto. \\ 
 \\ 
\end{longtabu} 
\newpage 
\hfill\Huge{\textbf{M}} \\ 
\normalsize 
\begin{longtabu} to \textwidth{p{0.1\textwidth} p{0.8\textwidth}} 
\toprule \\ 
\textbf{Metrica} : & Standard per la misura di alcune propriet? del software o delle sue specifiche. \\ 
 \\ 
\textbf{Milestone} : & Importante traguardo intermedio nello svolgimento del progetto, viene evidenziata in maniera diversa dalle altre attivit? nell'ambito dei documenti di progetto. \\ 
 \\ 
\end{longtabu} 
\newpage 
\hfill\Huge{\textbf{P}} \\ 
\normalsize 
\begin{longtabu} to \textwidth{p{0.1\textwidth} p{0.8\textwidth}} 
\toprule \\ 
\textbf{PC} : & Personal Computer \\ 
 \\ 
\textbf{Pacchetto} : & Insieme di programmi distribuiti congiuntamente. In senso pi? specifico, un pacchetto indica un software per computer compresso in un formato archivio per essere installato da un sistema di gestione dei pacchetti o da unprogramma d'installazione autonomo. \\ 
 \\ 
\textbf{Percorso} : & Nome che contiene in forma esplicita informazioni sulla posizione di un file all'interno del sistema. \\ 
 \\ 
\textbf{Pert} : & (programme evaluation and review technique) tecnica di valutazione e revisione dei programmi. \\ 
 \\ 
\textbf{Processo} : & Serie di operazioni tecniche attraverso le quali viene svolta un'attivit? produttiva. \\ 
 \\ 
\textbf{Progetto} : & Un progetto consiste, in senso generale, nell'organizzazione di azioni nel tempo per il perseguimento di uno scopo predefinito. \\ 
 \\ 
\textbf{Proponente} : & Colui che ha proposto al committente il proprio capitolato d?appalto. \\ 
 \\ 
\end{longtabu} 
\newpage 
\hfill\Huge{\textbf{R}} \\ 
\normalsize 
\begin{longtabu} to \textwidth{p{0.1\textwidth} p{0.8\textwidth}} 
\toprule \\ 
\textbf{RA} : & Revisione di Accettazione. \\ 
 \\ 
\textbf{RP} : & Revisione di Progettazione. \\ 
 \\ 
\textbf{RQ} : & Revisione di Qualifica. \\ 
 \\ 
\textbf{RR} : & Revisione dei Requisiti. \\ 
 \\ 
\textbf{Repository} : & Luogo di stoccaggio dati centralizzato in cui risiedono i componenti software, ed e in grado, inoltre, di fornire funzionalit? di versionamento per tener traccia della storia dei prodotti. \\ 
 \\ 
\textbf{Requisito} : & Ciascuna delle qualit? necessarie e richieste per uno scopo determinato. \\ 
 \\ 
\textbf{Risorsa} : & Ogni componente, fisico o virtuale, che offra una certa funzionalit? con disponibilit? limitata/finita all'interno di un sistema informatico. \\ 
 \\ 
\textbf{Risorsa Umana} : & Personale che lavora al progetto. \\ 
 \\ 
\end{longtabu} 
\newpage 
\hfill\Huge{\textbf{S}} \\ 
\normalsize 
\begin{longtabu} to \textwidth{p{0.1\textwidth} p{0.8\textwidth}} 
\toprule \\ 
\textbf{SPICE} : & Acronimo di Simulation Program with Integrated Circuit Emphasis , ? un insieme di documenti tecnici per il processo di sviluppo software e le relative funzioni di gestione delle risorse. \\ 
 \\ 
\textbf{Smartphone} : & Uno smartphone ? un dispositivo mobile che abbina funzionalit? di telefono cellulare a quelle di gestione di dati personali. \\ 
 \\ 
\textbf{Software} : & set di istruzioni interpretabili da una macchina che guidano le componenti di un computer a svolgere specifiche operazioni. \\ 
 \\ 
\end{longtabu} 
\newpage 
\hfill\Huge{\textbf{T}} \\ 
\normalsize 
\begin{longtabu} to \textwidth{p{0.1\textwidth} p{0.8\textwidth}} 
\toprule \\ 
\textbf{Tablet} : & Dispositivo mobile con dimensioni dello schermo maggiori ai 7? ,con potenza di calcolo e numero di porte di comunicazione simili a quelle di uno smartphone. \\ 
 \\ 
\textbf{Template} : & Il termine in inglese template( letteralmente, sagoma) indica un documento o programma che fornisce uno scheletro standard per altri documenti o programmi. \\ 
 \\ 
\textbf{Ticket} : & Insieme di informazioni che possono essere usate per definire un?attivita, per permettere una migliore organizzazione del lavoro. \\ 
 \\ 
\textbf{Ticketing} : & Servizio basato sull?utilizzo dei ticket. \\ 
 \\ 
\end{longtabu} 
\newpage 
\hfill\Huge{\textbf{U}} \\ 
\normalsize 
\begin{longtabu} to \textwidth{p{0.1\textwidth} p{0.8\textwidth}} 
\toprule \\ 
\textbf{URL} : & Uniform Resource Locator: sequenza di caratteri che identifica univocamente l'indirizzo di una risorsa in Internet, tipicamente presente su un host server, come ad esempio un documento, un'immagine, un video. \\ 
 \\ 
\end{longtabu} 
\newpage 
\hfill\Huge{\textbf{V}} \\ 
\normalsize 
\begin{longtabu} to \textwidth{p{0.1\textwidth} p{0.8\textwidth}} 
\toprule \\ 
\textbf{Validazione} : & Conferma attraverso l?esame e l'apporto di evidenza oggettiva che i requisiti particolari per l'utilizzo previsto di file, documenti e programmi siano soddisfatti. \\ 
 \\ 
\end{longtabu} 
\newpage 
\hfill\Huge{\textbf{W}} \\ 
\normalsize 
\begin{longtabu} to \textwidth{p{0.1\textwidth} p{0.8\textwidth}} 
\toprule \\ 
\textbf{W3C} : & Il W3C ( World Wide Web Consortium) ? un associazione nata nel 1994 con lo scopo di migliorare gli esistenti protocolli e linguaggi per il World Wide Web e di aiutare il web a sviluppare tutte le sue potenzialit?. Sono membri di questa associazione importanti aziende informatiche, compagnie telefoniche, societa interessate alla crescita delWeb,organizzazioni no-prot, universita e istituzioni per la ricerca. \\ 
 \\ 
\textbf{WBS} : & (work breakdown structure) detta anche struttura di scomposizione del lavoro (traduzione letterale) o struttura analitica di progetto, si intende l'elenco di tutte le attivit? di un progetto. \\ 
 \\ 
\textbf{WEB} : & Uno dei principali servizi di Internet che permette di navigare e usufruire di un insieme vastissimo di contenuti (multimediali e non) collegati tra loro attraverso legami, e di ulteriori servizi accessibili a tutti o ad una parte selezionata degli utenti di Internet. \\ 
 \\ 
\textbf{Windows} : & Sistema operativo di Microsoft per computer desktop e laptop. Windows Phone: Sistema operativo per dispositivi mobile di Microsoft. \\ 
 \\ 
\end{longtabu} 
 }