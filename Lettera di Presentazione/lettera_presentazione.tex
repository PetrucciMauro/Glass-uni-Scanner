\documentclass[a4paper,10pt]{letter}
\usepackage{../Template/format}
\PassOptionsToPackage{a4paper,top=3cm,bottom=2cm,left=2cm,right=2cm}{geometry}

%	HEADER
\rhead{Lettera di Presentazione}

%	CORPO
\address{
	\Vardanega\\
	Università degli Studi di Padova\\
	DiP. di Matematica Pura e Applicata\\
	Via Trieste 63\\
	35121 Padova}

\signature{
	Pietro Tollot \\	
	il Responsabile del gruppo\\
	\gruppo}
 	
\begin{document}	
	\begin{letter}{Oggetto: \textbf{Consegna Revisione di Accettazione.}}
	\opening {Egregio \Vardanega ,}
	Con la presente, il gruppo \gruppo intende inviarLe i documenti necessari alla presentazione della Revisione di Accettazione prevista in data 2015-08-24.
	Alla presente si allegano i seguenti documenti:
			\begin{itemize}
				\item Interni
					\begin{itemize}
						\item Norme di Progetto - v1.0;
						\item Verbale riunione interna;						
					\end{itemize}
				\item Esterni
					\begin{itemize}
						\item Analisi dei Requisiti - v1.0;
						\item Definizione di Prodotto - v1.0;
						\item Glossario - v3.0;
						\item Manuale Amministratore - v1.0;
						\item Manuale Installazione - v1.0;
						\item Manuale Utente - v1.0;
						\item Piano di Qualifica - v1.0;
						\item Piano di Progetto - v1.0;
						\item Specifica Tecnica - v1.0.
					\end{itemize}
			\end{itemize} 
		Il consuntivo finale del prodotto e l’esito dei test sono trattati in modo approfondito nei documenti allegati.\\
		Oltre alla documentazione per la Revisione di Accettazione, verranno allegati anche i Javadoc, il codice sorgente e il codice dei test del prodotto.\\
		Tutto il materiale è presente nel CD fornito in duplice copia.\\
		Il gruppo sarà  contattabile per eventuali chiarimenti tramite l'e-mail in calce alla lettera. La ringrazio per la Sua attenzione.
		
		\thispagestyle{fancy}
		\closing{Distinti saluti.}

   \end{letter}
\end{document}