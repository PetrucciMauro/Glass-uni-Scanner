\section{Gestione amministrativa della revisione}


\subsection{Comunicazione e risoluzione di anomalie}

Un'\textit{anomalia} corrisponde a:
\begin{itemize}

\item Violazione delle norme tipografiche in un documento;
\item Uscita dal range d'accettazione degli indici di misurazione;
\item Incongruenza del prodotto con funzionalità presenti nell' analisi dei requisiti;
\item Incongruenza del codice con il design del prodotto.

\end{itemize}

In caso un verificatore riscontri un'anomalia,aprirà un ticket nel sistema di ticketing con le modalità specificate nelle Norme di Progetto.
Le modalità di risoluzione di quest'ultimo e la sua struttura vengono descritte in modo dettagliato all'interno del documento NormeDiprogetto-v1.0.pdf.
Quando viene rilasciata una nuova versione di un documento od un modulo,il Verificatore controlla il registro delle modifiche ed in base adesso effettua una verifica alla ricerca di anomalie da correggere.Se ne trova, apre un ticket e lo comunica all'Amministratore; s'occuperà della correzione la persona che ha apportato la modifica al documento o modulo le nuove modifiche dovranno essere approvate dall'Amministratore.