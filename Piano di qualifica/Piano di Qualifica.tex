%% LyX 2.0.6 created this file.  For more info, see http://www.lyx.org/.
%% Do not edit unless you really know what you are doing.
\documentclass[english]{article}
\usepackage[T1]{fontenc}
\usepackage[latin9]{inputenc}
\usepackage{url}
\usepackage{subscript}
\usepackage{babel}
\begin{document}

\paragraph{\textsubscript{Descrizione}}

Documento riguardante le norme e le procedure di validazione adottate
da \textit{LateButSafe} per il progetto \textbf{Premi}, proposto da
\textit{Zucchetti S.p.A.}


\section{Introduzione}


\subsection{Scopo del documento}

Il \textit{Piano di Qualifica} ha l'obiettivo di descrivere le strategie
adottate nei processi di verifica e validazione per assicurare la
qualit� del progetto \textbf{Premi} e dei processi volti alla sua
produzione.


\subsection{Scopo del prodotto}

Lo scopo del progetto � lo sviluppo di una piattaforma per la creazione
e la visualizzazione di presentazioni su dispositivi desktop che in
ambito mobile, permettendo la visione non lineare delle slide e favorendo
lo storytelling.


\subsection{Glossario}

Al fine d'evitare abiguit� rispetto ai termini tecnici utilizzati
e del linguaggio usato, viene allegato il file \textit{Glossario\_v1.0.pdf},
nel quale vengono descritti i termini contrassegnati dal simbolo \textbf{\textit{\textsubscript{G}}}
alla fine della parola. 


\subsection{Riferimenti}


\subsubsection{Normativi}
\begin{itemize}
\item Norme di Proetto
\item Capitolato d'appalto C4: \textbf{Premi}, software di presentazione
"better than Prezi" \url{http://www.math.unipd.it/~tullio/IS-1/2014/Progetto/C4.pdf}.
\end{itemize}

\subsubsection{Informativi:}
\begin{itemize}
\item Piano di Progetto: \textsl{Piano di Progetto v1.0};
\item Slide dell'insegnamento Ingegneria del Software mod. A: \url{http://www.math.unipd.it/~tullio/IS-1/2014/};
\item SWEBOK - Version 3 (2004): capitolo 11 - Software Quality \url{http://www.computer.org/web/swebok/v3};
\item Ingegneria del software - Ian Sommerville - 8a Ed. (2007):

\begin{itemize}
\item Cap. 27 - Gestione della qualit�;
\item Cap. 26 - Miglioramento dei Processi.
\end{itemize}
\item ISO/IEC\textbf{\textit{\textsubscript{G}}}9126:2001 (inglobato da
ISO/IEC\textbf{\textit{\textsubscript{G}}}25010:2011): Systems and
software engineering \textendash{} Systems and software Quality Requirements
and Evaluation(SQuaRE) \textendash{} System and software quality models
\url{http://en.wikipedia.org/wiki/ISO/IEC_9126} \url{http://www2.cnipa.gov.it/site/_contentfiles/01379900/1379951_ISO%209126.pdf}.
\item ISO/IEC\textbf{\textit{\textsubscript{G}}}12207:2008: Software Life
Cycle Processes \url{http://en.wikipedia. org/wiki/ISO/IEC_12207}
\item ISO/IEC\textbf{\textit{\textsubscript{G}}}14598:2001 (inglobato da
ISO/IEC |g| 25040:2011): Systems and software engineering \textendash{}
Systems and software Quality Requirements and Evaluation (SQuaRE)
\textendash{} Evaluation process \url{http://www2.cnipa.gov.it/site/_contentfiles/ 01379900/1379952_ISO%2014598.pdf}.
\item ISO/IEC\textbf{\textit{\textsubscript{G}}}15504:1998: Information
Tecnology \textendash{} Process Assessment, conosci- uto come SPICE
(Software Process Improvement and Capability dEtermination) \url{http://www2.cnipa.gov.it/site/_contentfiles/00310300/310320_15504.pdf}\url{http://en.wikipedia.org/wiki/ISO/IEC_15504}.
\end{itemize}

\section{Visione generale delle strategie di verifica}


\subsection{Definizione obiettivi}


\subsubsection{Qualit� di processo}

Al fine di garantire la qualit� del prodotto in ogni fase di realizzazione,
si deve garantire la qualit� dei processi che lo definiscono; per
questo motivo si � deciso di utilizzare lo standard ISO/IEC\textbf{\textit{\textsubscript{G}}}15504
denominato SPICE, che rende disponibili strumenti adatti a valutarli.\\
Per applicare il modello SPICE s'utilizza il ciclo di Deming (ciclo
PDCA) che descrive una metodologia di controllo dei processi durante
il loro ciclo di vita, permettendo di migliorarne in modo conrinuativo
la qualit�.


\subsubsection{Qualit� di prodotto}


\subsection{Procedure di controllo di qualit� di processo}


\subsection{Procedure di controllo di qualit� di prodotto}


\subsection{Organizzazione}


\subsection{Pianificazione strategica e temporale}


\subsection{Responsabilit�}


\subsection{Risorse}


\subsection{Tecniche di analisi}


\subsubsection{Analisi statica}


\paragraph{{\small{2.8.1.1 Walkthorough}}}


\paragraph{{\small{2.8.1.2 Inspection}}}


\subsubsection{Analisi dinamica}


\subsection{Misure e metriche}


\subsubsection{Metriche per i processi}


\subsubsection{Metriche per i documenti}


\subsubsection{Metriche per il software}


\section{Risorse}


\subsection{Risorse Necessarie:}


\subsubsection{Risorse umane}

I ruoli necessari a garantire la qualit� del prodotto sono:
\begin{itemize}
\item Responsabile di Progetto: 
\item Amministratore 
\item Verificatore: 
\item Programmatore: 
\end{itemize}

\subsubsection{Risorse Hardware}

Saranno necessari:
\begin{itemize}
\item computer con installato software necessario allo sviluppo del progeto
in tutte le sue fasi;
\item luogo dove incontrarsi, preferibilmente dotato di connessione ad Internet.
\end{itemize}

\subsubsection{Risorse software}

Saranno necessari:
\begin{itemize}
\item Strumenti per automatizzare i test
\item Framework\textbf{\textit{\textsubscript{G}}} per eseguire test di
unit�;
\item Piattaforma di versionamento per la creazione e gestione di ticket\textbf{\textit{\textsubscript{G}}};
\item Debugger\textbf{\textit{\textsubscript{G}}} per i linguaggi di programmazione
scelti;
\item Browser\textbf{\textit{\textsubscript{G}}} come piattaforma di testing
dell'applicazione da sviluppare;
\item Strumenti per effettuare l'analisi statica del codice per misurare
le metriche.
\end{itemize}

\subsection{Risorse disponibili}


\subsubsection{Risorse hardware}

Saranno necessari:
\begin{itemize}
\item Computer personali dei membri del gruppo;
\item Computer presenti nelle aule infomatiche del Dipartimento di Matematica;
\item Aule disponibili per incontri nel Dipartimento di Matematica.
\end{itemize}

\subsubsection{Risorse software}

Vengono elencate le risorse software\textbf{\textit{\textsubscript{G}}}
disponibili:
\begin{itemize}
\item Eclipse con plugin\textbf{\textit{\textsubscript{G}}} aggiuntivi,
Aptana e Google Chrome Developer Tools per il debugging\textbf{\textit{\textsubscript{G}}};
\item Aspell per il controllo ortografico;
\item JUnit per i test di unit�;
\item Git e SourceForge per il versioning\textbf{\textit{\textsubscript{G}}}
e la gestione dei ticket\textbf{\textit{\textsubscript{G}}};
\item Validatori W3C\textbf{\textit{\textsubscript{G}}}.
\end{itemize}

\section{Gestione amministrativa della revisione}


\subsection{Comunicazione e risoluzione di anomalie}

Un'\textit{anomalia} corrisponde a:
\begin{itemize}
\item Violazione delle norme tipografiche in un documento;
\item Uscita dal range d'accettazione degli indici di misurazione {[}descritti
in 2.{]};
\item Incongruenza del prodotto con funzionalit� presenti nell'analisi dei
requisiti;
\item Incongruenza del codice con il design del prodotto.
\end{itemize}
In caso un \textit{Verificatore} riscontri un'anomalia, aprir� un
ticket\textbf{\textit{\textsubscript{G}}} nel sistema di ticketing
con le modalit� specificate nelle \textit{Norme di Progetto}.

Le modalit� di risoluzione di quest'ultimo e la sua struttura vengono
descritte in modo dettagliato all'interno del documento \textit{NormeDiProgetto\_v1.0.pdf}.

Quando viene rilasciata una nuova versione di un documento od un modulo\textbf{\textit{\textsubscript{G}}},
il Verificatore controlla il registro delle modifiche ed in base ad
esso effettua una verifica alla ricerca di anomalie da correggere.
Se ne trova, apre un ticket e lo comunica all'Amministratore; s'occuper�
della correzione la persona che ha apportato la modifica al documento
o modulo\textbf{\textit{\textsubscript{G}}}; le nuove modifiche dovranno
essere approvate dall'Amministratore.


\section*{A }
\end{document}
